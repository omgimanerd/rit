\documentclass[letterpaper, 12pt]{math}

\title{University Physics 1A}
\author{Alvin Lin}
\date{October 2nd, 2017}

\begin{document}

\maketitle

\section*{Work}

\subsubsection*{Practice Problem}
A force given by \( F = (-3.0x+2.0)N \) acts on an object that is initially at
\( x = 3.0m \). What is the work done by this force as the object is moved to
\( x = 5.0m \)?
\begin{align*}
  W &= \int{F_x\diff{x}} \\
  &= \int_{3}^{5}(-3x+2)\diff{x} \\
  &= \frac{3x^2}{2}+2x\bigg]_3^5 \\
  &= \frac{-3(5^2)}{2}+2(5)-\left(-\frac{3(3^2)}{2}+2(3)\right) \\
  &= -20J
\end{align*}

\subsubsection*{Practice Problem}
A force \( \vec{F} = (-Ax,-By^2)N \) acts on an object that is initially at rest
at the origin. It travels only along the y-axis to \( y = 4.0m \).
\begin{enumerate}
  \item What are the units of \( A \) and \( B \)? \\
  \( A \) is in \( \frac{N}{m} \) and \( B \) is in \( \frac{N}{m^2} \).
  \item What is the work done by this force as the object is moved to
  \( y = 4.0m \)?
  \begin{align*}
    W &= \int{F_y\diff{y}} \\
    &= \int_{0}^{4}-By^2\diff{y} \\
    &= \frac{-By^3}{3}\bigg]_{0}^{4} \\
    &= \frac{-64B}{3}m^2
  \end{align*}
\end{enumerate}

\subsubsection*{Practice Problem}
A force \( \vec{F} = (-10,-5.0,-4.0)N \) acts on an object that is initially at
the origin and moves to a position \( \vec{r} = (2.0,3.0,0.0)m \). What is the
work done on the object by this force during the motion?
\begin{align*}
  W &= \vec{F}\cdot\vec{d} \\
  &= (-20)+(-15)+0 \\
  &= -35J
\end{align*}

\subsection*{Work-Kinetic Energy Theorem}
\[ W = \frac{1}{2}mv^2 \]
\[ W = KE_f-KE_i \]

\begin{center}
  You can find all my notes at \url{http://omgimanerd.tech/notes}. If you have
  any questions, comments, or concerns, please contact me at
  alvin@omgimanerd.tech
\end{center}

\end{document}
