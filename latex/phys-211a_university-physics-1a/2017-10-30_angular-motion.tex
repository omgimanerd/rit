\documentclass{math}

\title{University Physics 1A}
\author{Alvin Lin}
\date{October 30th, 2017}

\usepackage{tikz}

\begin{document}

\maketitle

\section*{Angular Motion}
\begin{center}
  \begin{tikzpicture}
    \draw[<->] (-5,0) -- (5,0) node[right] {\( x \)};
    \draw[<->] (0,-5) -- (0,5) node[above] {\( y \)};
    \draw (0,0) circle (3cm);
    \draw[thick,->] (0,0) -- (2,2.2) node[above right] {\( (x,y) \)}
      node[pos=0.5,above] {\( r \)};
    \draw (1,0) arc (0:48:1cm) node[pos=0.5,left] {\( \theta \)};
    \draw[line width=2] (3,0) arc (0:48:3) node[pos=0.5, right] {\( s \)};
  \end{tikzpicture}
\end{center}
\[ \theta = \frac{s}{r} \]

\subsection*{Kinematics vs Rotational Motion}
\begin{center}
  \renewcommand{\arraystretch}{1.5}
  \begin{tabular}{|c|c|c|c|}
    \hline
    position & \( x \) & average position & \( \theta \) \\
    \hline
    average velocity & \( \frac{\Delta x}{\Delta t} \) &
      average angular velocity &
      \( \omega_{avg} = \frac{\Delta\theta}{\Delta t} \) \\
    \hline
    instantaneous velocity & \( \ddiff{x}{t} = v \) &
      instantaneous angular velocity & \( \omega = \ddiff{\theta}{t} \) \\
    \hline
    acceleration & \( a = \ddiff{v}{t} \) & average acceleration &
      \( \alpha = \ddiff{\omega}{t} \) \\
    \hline
  \end{tabular}
\end{center}
\begin{align*}
  s &= r\theta \\
  \ddiff{s}{t} &= r\ddiff{\theta}{t} = r\omega = v \\
  a_{centripetal} &= \frac{v^2}{r} = \frac{r^2\omega^2}{r} = r\omega^2 \\
  a_{tangential} &= \alpha r
\end{align*}
\( \theta \) and \( \Delta\theta \) are not vectors. \( \omega \) and
\( \alpha \) are vectors.
\begin{align*}
  \text{Linear} &\longleftrightarrow \text{Rotational} \\
  v = \ddiff{x}{t} &\longleftrightarrow \omega \\
  a = \ddiff{v}{t} &\longleftrightarrow \alpha
\end{align*}

\subsubsection*{Constant Acceleration}
\begin{align*}
  x &= x_0+v_0t+\frac{1}{2}at^2 \\
  v &= v_0+at \\
  v^2 &= v_0^2+2a\Delta x
\end{align*}

\subsubsection*{Constant \( \alpha \)}
\begin{align*}
  \theta &= \theta_0+\omega_0t+\frac{1}{2}\alpha t^2 \\
  \omega &= \omega_0+\alpha t \\
  \omega^2 &= \omega_0^2+2\alpha\Delta\theta
\end{align*}

\begin{center}
  You can find all my notes at \url{http://omgimanerd.tech/notes}. If you have
  any questions, comments, or concerns, please contact me at
  alvin@omgimanerd.tech
\end{center}

\end{document}
