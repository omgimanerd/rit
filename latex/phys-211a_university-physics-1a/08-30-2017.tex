\documentclass[letterpaper, 12pt]{math}

\usepackage{amsmath}
\usepackage{amssymb}

\title{University Physics 1A}
\author{Alvin Lin}
\date{August 2017 - December 2017}

\begin{document}

\maketitle

\section*{Factor Label Method}
For converting units of measure, you can multiply by one, using equal unit
measures as equivalent fractions:
\[ 1 = \frac{2.54in}{1cm} = \frac{5280ft}{1mi} = \frac{1mi}{5280ft} \]

\subsection*{Example}
A wheel rotates at 1.234 rad/s. Convert this to rev/min.
\[ \frac{1.234rad}{1s}\times\frac{1rev}{2\pi rad}\times\frac{60s}{1min} =
  \frac{74.04rev}{2\pi min} = 116.3\frac{rev}{min} \]

\section*{Uncertainty}
Uncertainties should have the same level of precision as the measurement.
\[ 1.23\pm0.06 \]
\[ 4.56\pm0.09 \]
When adding numbers with uncertainty, the uncertainties are added:
\[ 1.23+4.56\pm(0.06+0.06) \]
When subtracting numbers with uncertainty, the uncertainties are still
added because it is still compounded by the operation:
\[ 4.56-1.23\pm(0.06+0.09) \]
When multiplying numbers with uncertainty, we sum together the relative
uncertainty:
\[ (1.23\pm0.06)(4.56\pm0.09) =
  (1.23)(4.56)(1\pm(\frac{0.06}{1.23}+\frac{0.09}{4.56})) \]
When dividing numbers with uncertainty, it follows the same rule as
multiplication:
\begin{align*}
  \frac{1.23\pm0.06}{4.56}{0.09} &=
    \frac{1.23}{4.56}(1\pm(\frac{0.06}{1.23}+\frac{0.09}{4.56})) \\
  &= 0.27(1\pm(0.05+0.02)) \\
  &= 0.27\pm(0.27)(0.07) \\
  &= 0.27\pm0.02
\end{align*}
Uncertainty arises from the tools of measurement and the precision associated
with the instrument.

\begin{center}
  If you have any questions, comments, or concerns, please contact me at
  alvin@omgimanerd.tech
\end{center}

\end{document}
