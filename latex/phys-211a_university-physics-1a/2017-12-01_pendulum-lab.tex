\documentclass{math}

\title{University Physics 1A}
\author{Alvin Lin}
\date{December 1st, 2017}

\begin{document}

\maketitle

\section*{Pendulum Lab}
For a large physical pendulum:
\[ -Lmg\sin\theta = I\ddiff{^2\theta}{t^2} \]
For small angles \( \theta \) in radians, we can approximate this to:
\[ \sin\theta \approx \theta \]
\[ -Lmg\theta = I\ddiff{^2\theta}{t^2} \]
We know that:
\begin{align*}
  \theta &= A\cos(\omega t+\phi) \\
  \ddiff{\theta}{t} &= -A\omega\sin(\omega t+\phi) \\
  \ddiff{^2\theta}{t^2} &= -A\omega^2\cos(\omega t+\phi) \\
  &= -\omega^2\theta \\
  -Lmg\theta &= I(-\omega^2\theta) \\
  \omega &= \sqrt{\frac{Lmg}{I}}
\end{align*}

\end{document}
