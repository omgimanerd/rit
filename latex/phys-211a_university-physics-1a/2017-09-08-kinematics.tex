\documentclass[letterpaper, 12pt]{math}

\title{University Physics 1A}
\author{Alvin Lin}
\date{September 8th, 2017}

\begin{document}

\maketitle

\section*{Kinematic Equations}
The following equations can be derived though calculus:
\[ x = x_0+v_0t+\frac{1}{2}at^2 \]
\[ v = v_0+at \]
\[ v^2 = (v_0)^2+2a(x-x_0) \]
One additional equation can be derived from the combinations of the above
equations:
\[ x-x_0 = \frac{1}{2}(v+v_0)t \]
These equations are only valid when the acceleration is constant and assume that
motion begins at \( t = 0 \). If we assume that motion begins at \( t = t_i \):
\[ x = x_i+v_i(t-t_i)+\frac{1}{2}a(t-t_i)^2 \]
\[ v = v_0+a(t-t_i) \]
These equations are useful in situations without friction, air resistance, etc,
such as freefall (where acceleration \( a = -9.81m/s^2 \) assuming y is
positive upwards).
\[ g \equiv 9.81m/s^2 \]
\[ a_{gravity} = -g \]

\subsubsection*{Example}
Driver A is driving along a road at constant velocity \( v_A = 50m/s \). As
driver A passes police car P, P starts accelerating at \( 2m/s^2 \). How long
will it take for P to catch A? This question implies that the two cars must be
at the same point at the same time.
\begin{align*}
  x_A = x_P &= x_{A_0}+v_{A_0}t+\frac{1}{2}a_At^2 =
    x_{P_0}t+v_{P_0}+\frac{1}{2}a_Pt^2 \\
  0+50t+\frac{1}{2}(0)t^2 &= 0+0t+\frac{1}{2}(2)t^2 \\
  50t &= t^2 \\
  t^2-50t &= 0 \\
  t(t-50) &= 0
\end{align*}
The two solutions for this are \( t = 0s \) and \( t = 50s \). We can rule out
\( t = 0 \) because that's when driver A first passes police car P.

\section*{Reminders and Homework}
Complete the homework on TheExpertTA and WebAssign. \\
\textbf{Remember to bring the Activities Manual}

\begin{center}
  You can find all my notes at \url{http://omgimanerd.tech/notes}. If you have
  any questions, comments, or concerns, please contact me at
  alvin@omgimanerd.tech
\end{center}

\end{document}
