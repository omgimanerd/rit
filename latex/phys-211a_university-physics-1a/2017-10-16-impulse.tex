\documentclass[letterpaper, 12pt]{math}

\usepackage{tikz}

\title{University Physics 1A}
\author{Alvin Lin}
\date{October 16th, 2017}

\begin{document}

\maketitle

\section*{Impulse}
\[ W_{net} = \int\overrightarrow{F_{net}}\cdot\diff{\vec{x}} =
  \frac{1}{2}mv_f^2 = \frac{1}{2}mv_i^2 \]
\begin{align*}
  \int\overrightarrow{F_{net}}\diff{t} &= \int m\vec{a}\diff{t} \\
  &= \int m\ddiff{\vec{v}}{t}\diff{t} \\
  &= \int m\diff{\vec{v}} \\
  &= m\int\diff{v} \\
  &= m\vec{v_f}-m\vec{v_i} \\
  \text{impulse} &= \text{change in momentum}
\end{align*}
This is known as the impulse-momentum theorem.
\[ \int F\diff{t} = F_{avg}\Delta{t} \]
This applies to systems as well as individual objects. Suppose we have the
system:
\begin{center}
  \begin{tikzpicture}
    \draw[->] (-2,0) -- (-1,0) node[pos=0.5,above] {\( F \)};
    \draw[->] (-2,0) -- (-2,-1) node[pos=0.5,left] {\( m_1g \)};
    \draw[->] (2,0) -- (1,0) node[pos=0.5,above] {\( F \)};
    \draw[->] (2,0) -- (2,-1) node[pos=0.5,right] {\( m_2g \)};
  \end{tikzpicture}
\end{center}
The internal forces cancel out and the net impulse is only due to the external
forces.
\begin{align*}
  0 &= m_1v_{1f}+m_2v_{2f}-(m_1v_{1i}+m_2v_{2i}) \\
  m_1v_{1i}+m_2v_{2i} &= m_1v_{1f}+m_2v_{2f} \\
\end{align*}
This is known as the conservation of momentum, and is true when the net
external force is zero.

\subsubsection*{Example}
A superball of mass \( m = 0.0300 \) kg heads towards the right with velocity
\( v = 20.0 \) m/s west (?). It hits a wall and then bounces back with the same
speed in the opposite direction. The ball is in contact with the wall for
\( t = 0.0300 \) s. What is the average \textit{vector} force \textbf{on the
ball} during the collision?
\begin{align*}
  F_{avg}\Delta{t} &= m\vec{v_f}-m\vec{v_i} \\
  F_{avg} &= \frac{m\vec{v_f}-m\vec{v_i}}{\Delta{t}} \\
  &= \frac{(0.03)(20)-(0.03)(-20)}{0.03} \\
  &= 40N
\end{align*}

\subsubsection*{Example}
A tennis ball of mass \( m = 0.0300 \) kg is thrown towards the right with
velocity \( v = 20.0 \) m/s west. It hits a wall with a speed of \( v = 20.0 \)
m/s, and the angle between the vertical wall and the initial velocity is
\( 70.0^{\circ} \). It bounces off the wall, still headed upwards with a speed
of 10.0 m/s, again making an angle of \( 70.0^{\circ} \) with the vertical wall.
The ball is in contact with the wall for \( t = 0.0300 \) s. What is the average
\textit{vector} force \textbf{on the ball} during the collision written in
component form?
\begin{align*}
  v_{xi} &= 20\sin(70) \\
  v_{xf} &= -10\sin(70) \\
  v_{yi} &= 20\cos(70) \\
  v_{yf} &= 10\cos(70) \\
  F_{net~x} &= \frac{mv_{xf}-mv_{xi}}{\Delta{t}} \\
  &= \frac{(0.03)(-10\sin(70))-(0.03)(20\sin(70))}{0.03} \\
  &= -28.2N \\
  F_{net~y} &= \frac{mv_{yf}-mv_{yi}}{\Delta{t}} \\
  &= \frac{(0.03)(10\cos(70))-(0.03)(20\cos(70))}{0.03} \\
  &= -3.42N \\
  F_{net} &= \langle-28.2,-3.42\rangle N
\end{align*}

\begin{center}
  You can find all my notes at \url{http://omgimanerd.tech/notes}. If you have
  any questions, comments, or concerns, please contact me at
  alvin@omgimanerd.tech
\end{center}

\end{document}
