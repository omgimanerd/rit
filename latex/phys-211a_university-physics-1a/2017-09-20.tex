\documentclass[letterpaper, 12pt]{math}

\usepackage{tikz}

\title{University Physics 1A}
\author{Alvin Lin}
\date{September 20th, 2017}

\begin{document}

\maketitle

\section*{Newton's Laws}

\begin{enumerate}
  \item The natural state of motion of an object is to move with
    constant velocity (which could be zero), and a force is needed to
    change that.
  \item \( F_{net} = ma \)
  \item \( F_{initial} = F_{final} \)
\end{enumerate}
We draw free body diagrams from a single dot. Example:
\begin{center}
  \begin{tikzpicture}
    \draw[fill] (0,0) circle (0.1cm);
    \draw[->] (0,0) -- (0,-2) node[left] {\( F_g \)};
    \draw[->] (0,0) -- (0,1.5) node[above] {\( F_{air\ resistance} \)};
    \draw[->] (0,0) -- (3,0) node[right] {\( F_{wind} \)};
  \end{tikzpicture}
\end{center}


\section*{Reminders and Homework}
Complete the homework on TheExpertTA and WebAssign. \\
\textbf{Remember to bring the Activities Manual}

\begin{center}
  You can find all my notes at \url{http://omgimanerd.tech/notes}. If you have
  any questions, comments, or concerns, please contact me at
  alvin@omgimanerd.tech
\end{center}

\end{document}
