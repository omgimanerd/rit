\documentclass[letterpaper, 12pt]{math}

\title{University Physics 1A}
\author{Alvin Lin}
\date{October 5th, 2017}

\begin{document}

\maketitle

\section*{Potential Energy}
Any path from A to B has the same work by the force of gravity. The work around
a closed path is 0. This is called a ``conservative'' force. For a conservative
force \( \vec{F} \):
\[ \int\vec{F}\cdot\vec{\diff{s}} = 0 \]
on any closed path. We can define a potential function \( u \) which is a
function of position.
\begin{align*}
  \triangle{u} &= u_B-u_A \\
  &= -\int_{A}^{B}\vec{F}\cdot\vec{\diff{s}} \\
  &= -W_{A~to~B} \\
  &= -\int_{A}^{B}mg~\diff{s}\cos(180) \\
  &= mg\int_{A}^{B}\diff{s} \\
  &= mgh \\
  &= \text{gravitational potential energy of the mass } m
\end{align*}
For a frictionless spring where equilibrium is position 0, compression distance
is \( x_B \), and the force by the spring is \( -kx \):
\begin{align*}
  \triangle{u} &= u_B-u_A \\
  &= -\int_{0}^{x_B}\overrightarrow{F_{spring}}\cdot\vec{\diff{x}} \\
  &= -\int_{0}^{x_B}|-kx|~|\diff{x}|\cos(180^{\circ}) \\
  &= \int_{0}^{x_B}x\diff{x} \\
  &= \frac{1}{2}kx^2\bigg]_{0}^{x_B} \\
  &= \frac{1}{2}kx_B^2 \\
  &= \text{potential energy of the spring at }x = x_B
\end{align*}

\begin{center}
  You can find all my notes at \url{http://omgimanerd.tech/notes}. If you have
  any questions, comments, or concerns, please contact me at
  alvin@omgimanerd.tech
\end{center}

\end{document}
