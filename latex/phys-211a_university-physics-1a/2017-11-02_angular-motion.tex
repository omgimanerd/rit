\documentclass{math}

\title{University Physics 1A}
\author{Alvin Lin}
\date{November 2nd, 2017}

\usepackage{enumerate}
\usepackage{tikz}

\begin{document}

\maketitle

\section*{Angular Motion}
A pulley consists of a hub of radius 0.160m attached to a wheel of radius
0.600m. Initially the pulley is rotating clockwise at 6.50 rad/s, and at some
time later it is rotating counterclockwise at 2.50 rad/s. The angular position
at the second time is 24.0 rad clockwise relative to the first time. A string
wrapped around the hub and is attached to a block.
\begin{enumerate}[(a)]
  \item Find the angular acceleration of the pulley. Be sure to give a vector
  direction, not using ``clockwise'' or ``counterclockwise''.
  \begin{align*}
    \omega^2 &= \omega_0^2+2\alpha\theta \\
    (-6.5)^2 &= 2.5^2+2\alpha24.0 \\
    \alpha &= 0.75\frac{rad}{s} \quad (+\k \text{ direction})
  \end{align*}
  \item Find the time between the two points mentioned.
  \begin{align*}
    \omega &= \omega_0+\alpha t \\
    2.5 &= (-6.5)+0.75t \\
    t &= 12s
  \end{align*}
  \item Find the acceleration of the block, including its direction.
  \begin{align*}
    a_{tangential} &= \alpha r \\
    &= 0.75(0.160) \\
    &= 0.12\frac{m}{s^2}
  \end{align*}
  \item The string from the block touches the pulley at the hub. At the initial
  time find the centripetal acceleration of that point on the hub.
  \begin{align*}
    a_{centripetal} &= \frac{v^2}{r} \\
    &= r\omega^2 \\
    &= 0.160(6.5^2) \\
    &= 6.76\frac{m}{s^2}
  \end{align*}
\end{enumerate}

\section*{Moment of Inertia}
\begin{align*}
  KE &= \frac{1}{2}mv^2 \\
  &= \frac{1}{2}m(r\omega^2) \\
  &= \frac{1}{2}(mr^2)\omega^2 \\
  I_{ring} &= \text{moment of inertia} = mr^2 \\
  I_{disk} &= \sum_{i}m_ir_i^2 \\
  &= \int r^2\diff{m}
\end{align*}

\begin{center}
  You can find all my notes at \url{http://omgimanerd.tech/notes}. If you have
  any questions, comments, or concerns, please contact me at
  alvin@omgimanerd.tech
\end{center}

\end{document}
