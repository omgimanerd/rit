\documentclass[letterpaper, 12pt]{math}

\title{University Physics 1A}
\author{Alvin Lin}
\date{October 23rd, 2017}

\begin{document}

\maketitle

\section*{Practice Test Review}

\subsubsection*{Example}
Car A of mass 800kg and car B of mass 1200kg are headed towards each other as
show (top view) Immediately before impact, car A has a speed of 30.0 m/s and
makes an angle of \( 40^{\circ} \) with the x-axis as shown. Car B has a speed
of 20.0 m/s and makes an angle of \( 75^{\circ} \) with the y-axis as shown.
After the collision, the cars stick together and move as one. What is the
velocity of the cars immediately after the collision, expressed in unit vector
notation.
\begin{align*}
  m_A\vec{v_A}+m_B\vec{v_B} &= (m_A+m_B)\vec{v_f} \\
  (800)(30)(\cos(40)\i+\sin(40)\j)+(1200)(-20)(\sin(75)\i+\cos(75)\j) &=
    (800+1200)\vec{v_f} \\
  \vec{v_f} &= -2.4\i+10.8\j
\end{align*}
Is the collision elastic or inelastic? \\
Totally inelastic.
\begin{align*}
  KE_i &= \frac{1}{2}m_A\vec{v_A}^2+\frac{1}{2}m_B\vec{v_B}^2 \\
  &= \frac{1}{2}(800)(30^2)+\frac{1}{2}(1200)(20^2) \\
  &= 600000J \\
  KE_f &= \frac{1}{2}(1200+800)(2.4^2+10.8^2) \\
  &= 122400J \\
  &\ne KE_i
\end{align*}

\subsubsection*{Example}
A block of mass \( m \) starts from rest at the top of a frictionless hill of
height \( H \). It slides down the hill and then over another hill of height
\( h \). The top of the second smaller hill can be modeled as part of a circle
with radius \( h \). If \( H = \frac{5h}{4} \), what is the normal force that
the track exerts on the block when it is at the top of the second hill? Express
your answer in terms of \( m \) and \( g \).
\begin{align*}
  KE+PE &= 0+mgh \\
  &= \frac{1}{2}mv^2+mgh \\
  v^2 &= 2gH-2gh \\
  &= 2g(H-h) \\
  F_{net} &= F_N-mg = -m\frac{v^2}{r} \\
  F_N-mg &= -m\frac{v^2}{r} \\
  &= -\frac{m}{h}(2g(H-h)) \\
  F_N &= -\frac{m}{h}2g(H-h)+mg \\
  &= mg(3-2\frac{H}{h}) \\
  &= mg(3-2\frac{\frac{5}{4}h}{h}) \\
  &= mg(3-\frac{10}{4}) \\
  &= \frac{1}{2}mg
\end{align*}

\subsubsection*{Example}
A block of mass \( m \) travels over a frictionless circular hill of radius and
height \( R \). At the top of the hill, the normal force on the moving block is
found to be \( \frac{mg}{4} \).
\begin{align*}
  F_N-mg &= (\frac{1}{4}-1)mg \\
  &= -\frac{3}{4}mg \\
  &= -m\frac{v_0^2}{R} \\
  v_0^2 &= \frac{3}{4}gR \\
  PE_i+KE_i+W_{friction} &= PE_f+KE_f \\
  mgR+\frac{1}{2}mv_0^2-\mu_k(4mg)d &= 0 \\
  mgR(1+\frac{3}{8}) &= \mu_k(mgd) \\
  R(\frac{11}{8}) &= \mu_k4d \\
  d &= \frac{11R}{32\mu_k}
\end{align*}

\subsubsection*{Collisions}
Inelastic Collision: \( KE_f < KE_i \) \\
Perfectly Elastic Collision: \( KE_f = KE_i \) \\
Superelastic Collision: \( KE_f > KE_i \)

\subsubsection*{Example}
A raft is floating in the water. A scout stands on the end of a raft nearest
the shore. He is 150.0m from shore, and the raft is 8.00m long. The mass of the
scout is 90.0 kg and the mass of the raft is 500.0 kg. The scout walks to the
other end of the raft. How far is the scout from shore after he has walked to
the far end?
\begin{align*}
  x_{CoM~initial} &= \frac{(90kg)(150m)+(500kg)(154m)}{500kg+90kg} = 153.4m \\
  x_{CoM~final} &= x_{CoM~initial} = 153.4m \\
  &= \frac{(90)x_{scout}+500x_{raft}}{590} \\
  &= \frac{90x_{scout}+500(x_{scout}-4)}{590} \\
  x_{scout} &= 156.8m
\end{align*}

\begin{center}
  You can find all my notes at \url{http://omgimanerd.tech/notes}. If you have
  any questions, comments, or concerns, please contact me at
  alvin@omgimanerd.tech
\end{center}

\end{document}
