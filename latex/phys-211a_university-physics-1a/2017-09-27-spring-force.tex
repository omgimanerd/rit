\documentclass[letterpaper, 12pt]{math}

\usepackage{tikz}

\title{University Physics 1A}
\author{Alvin Lin}
\date{September 27th, 2017}

\begin{document}

\maketitle

\section*{Uniform Circular Motion}
\begin{center}
  \begin{tikzpicture}[scale=0.8]
    \draw[thick,<->] (-5,0) -- (5,0) node[above] {x axis};
    \draw[thick,<->] (0,-5) -- (0,5) node[above] {y axis};
    \draw (0,0) circle (3cm);
    \draw[very thick,<-] (0,0) -- (2,2.2) node[above right] {\( (x,y) \)}
      node[pos=0.5, above] {\( \vec{a} \)};
    \draw[very thick,->] (0,0) -- (-1,-2.8)
      node[pos=0.5, above] {\( \vec{r} \)};
    \draw[very thick,->] (2,2.2) -- (1,3.2) node[above] {\( \vec{v} \)};
    \node (B) at (0.7,0.4) {\( \theta \)};
  \end{tikzpicture}
\end{center}
\[ a = \frac{v^2}{r} \]
\[ \overrightarrow{F_{net}} = m\vec{a} \]
For uniform circular motion:
\[ F_{centripetal} = F_{net~radial} = m\frac{v^2}{r} \]
where \( F_{net~radial} \) is the net force on the object towards the center of
the circle.

\section*{Spring Force}
\textbf{Hooke's Law}:
\[ F_{by~spring} = -kx \]
where \( k \) is the spring constant and \( x \) is the distance that it has
been stretched or compressed from rest. Any spring that follows this law is
called a Hooke's Law spring.

\section*{Reminders and Homework}
Complete the homework on TheExpertTA and WebAssign. \\
\textbf{Remember to bring the Activities Manual.} \\

\begin{center}
  You can find all my notes at \url{http://omgimanerd.tech/notes}. If you have
  any questions, comments, or concerns, please contact me at
  alvin@omgimanerd.tech
\end{center}

\end{document}
