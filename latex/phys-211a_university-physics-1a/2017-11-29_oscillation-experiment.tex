\documentclass{math}

\title{University Physics 1A}
\author{Alvin Lin}
\date{November 29th, 2017}

\begin{document}

\section*{Oscillation Experiment}
\begin{align*}
  F_{by~spring} &= -kx \\
  -kx &= F = ma = m\ddiff{^2x}{t^2} \\
  \ddiff{^2x}{t^2} &= \frac{m}{m}\ddiff{^2x}{t^2} \\
  &= -\frac{kx}{m} \\
\end{align*}
Assume \( x(t) = A\cos(\omega t+\phi) \):
\begin{align*}
  x(t) &= A\cos(\omega t+\phi) \\
  \ddiff{x}{t} &= -\omega A\sin(\omega t+\phi) \\
  \ddiff{^2x}{t^2} &= -\omega^2A\cos(\omega t+\phi) \\
  &\stackrel{?}{=} -\frac{k}{m}A\cos(\omega t+\phi) \\
  \omega^2 &= \frac{k}{m} \\
  \omega &= \sqrt{\frac{k}{m}}
\end{align*}
\( A \) is the maximum value of \( x \). \( \omega \) is the angular frequency.
\begin{align*}
  2\pi &= \omega T \\
  T &= \frac{2\pi}{\omega}
\end{align*}

\maketitle

\end{document}
