\documentclass{math}

\usepackage{tikz}

\title{University Physics 1A}
\author{Alvin Lin}
\date{December 6th, 2017}

\begin{document}

\maketitle

\section*{Waves}
\[ v = \frac{\omega}{k} = f\lambda \]
\[ \text{speed of wave} = v =
  \sqrt{\frac{elastic~property}{inertial~property}} \]
\[ \text{for a string(spring) v} = \sqrt{\frac{tension}{mass~density}} =
  \sqrt{\frac{F_{tension}}{\lambda}} \]
\[ \text{for air(sound) v} =
  \sqrt{\frac{bulk~modulus~of~air}{volume~mass~density}} =
  \sqrt{\frac{B}{\rho}} \]
Principle of superposition: two waves at the same point just add their values.
While this does not hold for other concepts, it holds for most of the mediums
that we will work with.

\subsubsection*{Interference}
Two waves going in the same direction with the same amplitude:
\begin{align*}
  y &= y_{max}\sin(kx-\omega t)+y_{max}\sin(kx-\omega t+\phi) \\
  &= 2y_{max}\cos(\frac{\phi}{2})\sin(kx-\omega t+\frac{\phi}{2}) \\
\end{align*}
If \( \phi = \pi \) and the two waves are half a wavelength apart, then the
two waves are destructive and cancel each other out. If \( \phi = 0 \), then
the two are constructive and amplify each other. Any other values of \( \phi \)
are partially constructive. This is an example of wave interference. \par
Two waves travelling in the opposite direction:
\begin{align*}
  y &= y_{max}\sin(kx-\omega t)-y_{max}\sin(kx-\omega t) \\
  &= 2y_{max}\cos(\omega t)\sin(kx)
\end{align*}
This forms a standing wave.

\begin{center}
  You can find all my notes at \url{http://omgimanerd.tech/notes}. If you have
  any questions, comments, or concerns, please contact me at
  alvin@omgimanerd.tech
\end{center}

\end{document}
