\documentclass{math}

\title{University Physics 1A}
\author{Alvin Lin}
\date{October 19th, 2017}

\begin{document}

\maketitle

\section*{Conservation of Momentum}
If there is zero net external force on a system:
\[ m_1\overrightarrow{v_{1i}}+m_2\overrightarrow{v_{2i}}+\dots =
  m_1\overrightarrow{v_{2f}}+m_2\overrightarrow{v_{2f}}+\dots \]
\[ \text{total initial momentum} = \text{total final momentum} \]
\[ m_1v_{1ix}+m_2v_{2ix}+\dots = m_1v_{1fx}+m_2v_{2fx}+\dots \]
\[ m_1v_{1iy}+m_2v_{2iy}+\dots = m_1v_{1fy}+m_2v_{2fy}+\dots \]

\subsubsection*{Example}
Two pucks move on a horizontal air-hockey table. One puck has mass 3.00 kg and
initially moves east at 4.00 m/s. The other puck has a mass of 6.00 kg and
moves at 5.00 m/s at \( 60.0^{\circ} \) north of east. The two pucks collide and
stick together. Find the velocity after the collision of the combined pucks.
\begin{align*}
  m_1v_{1ix}+m_2v_{2ix} &= m_{12}v_{12fx} \\
  3(4)+6(5\cos(60)) &= (3+6)v_{12fx} \\
  v_{12fx} &= \frac{12+30\cos(60)}{9} \\
  &= 3.00m/s \\
  m_1v_{1iy}+m_2v_{2iy} &= m_{12}v_{12fy} \\
  (3)(0)+6(5\sin(60)) &= (3+6)v_{12fy} \\
  v_{12fy} &= \frac{30\sin(50)}{9} \\
  &= 2.89m/s \\
  v_{12f} &= 4.167m/s \\
  \theta &= 43.8^{\circ}
\end{align*}

\begin{center}
  You can find all my notes at \url{http://omgimanerd.tech/notes}. If you have
  any questions, comments, or concerns, please contact me at
  alvin@omgimanerd.tech
\end{center}

\end{document}
