\documentclass{math}

\usepackage{tikz}

\title{University Physics 1A}
\author{Alvin Lin}
\date{November 20th, 2017}

\begin{document}

\maketitle

\section*{Angular Momentum}
Angular Momentum:
\[ L = I\omega = \vec{r}\times m\vec{v} \]

\subsubsection*{Example}
A thin rod of mass \( M_R \) and length \( d \) is pivoted at one end and can
rotate without friction along a horizontal surface. The diagram is a top view.
Initially it is rotating clockwise at \( \omega_{R,i} \). A piece of putty of
mass \( m_p \) slides without friction toward the end of the rod as shown with
a speed of \( v_p \). The putty collides and sticks to the rod at the very end.
What is the final angular velocity of the rod and putty?
\begin{center}
  \begin{tikzpicture}
    \draw[fill=black] (0.05,0) circle (0.1cm);
    \draw (0,0) -- (0.1,0) -- (0.1,-4) -- (0,-4) -- cycle node[pos=0.5,left]
      {\( d \)};
    \draw[->] (0.5,-4.2) -- (-0.5,-4.2) node[pos=0.5,below]
      {\( \omega_{R,i} \)};
    \draw[fill=black] (-2,-4) circle (0.1cm) node[above] {\( m_p \)};
    \draw[->] (-2,-4) -- (-1,-4) node[pos=0.5,below] {\( v_p \)};
  \end{tikzpicture}
\end{center}
\begin{align*}
  I_{rod}\omega_{R,i}-(dm_pv_p) &= I_{rod~and~putty}\omega_f \\
  \frac{1}{3}M_Rd^2-dm_pv_p &= (\frac{1}{3}M_Rd^2+m_pd^2)\omega_f \\
  \omega_f &= \frac{\frac{1}{3}M_Rd^2-dm_pv_p}{\frac{1}{3}M_Rd^2+m_pd^2}
\end{align*}

\end{document}
