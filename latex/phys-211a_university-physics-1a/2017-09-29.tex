\documentclass[letterpaper, 12pt]{math}

\usepackage{tikz}

\title{University Physics 1A}
\author{Alvin Lin}
\date{September 29th, 2017}

\begin{document}

\maketitle

\section*{Work}
\[ \text{work} = \overrightarrow{force}\times\overrightarrow{displacement} \]
\[ W = \vec{F}\cdot\overrightarrow{\Delta x} \]
There are two ways to define the dot product.
\begin{align*}
  \vec{A}\cdot\vec{B} &= |A||B|\cos(\theta) \\
  \vec{A} &= A_x\i+A_y\j \\
  \vec{B} &= B_x\i+B-y\j \\
  \vec{A}\cdot\vec{B} &= (A_x\i+A_y\j)\cdot(B_x\i+B_y\j) \\
  &= A_xB_x\i\cdot\i+A_yB_x\j\cdot\i+A_xB_y\i\cdot\j+A_yB_y\j\cdot\j \\
  \vec{A}\cdot\vec{B} &= A_xB_x+0+0+A_yB_y
\end{align*}
The calculation for work assumes a constant force and a straight line
displacement. A curved path can be broken up into infinitely many straight
segments. The total work is therefore the sum of the work in each segment.
\[ W = \sum{\vec{F}\cdot\overrightarrow{\Delta x}} \]
This can be generalized to:
\[ W = \int\vec{F}\cdot\vec{\diff{x}} \]
where \( \vec{\diff{x}} \) is an infinitesimal displacement tangent to the
path. This is the work done by force \( \vec{F} \) on an object moving along
the path. Work is expressed in the units of newton meters, or joules.

\subsection*{Power}
Power is the rate at which work is done, or the rate of change of work.
\begin{align*}
  P &= \frac{\text{work done}}{\text{time}} \\
  &= \ddiff{W}{t} \\
  \frac{J}{s} &= Watt = W \\
  1~horsepower &= 746W
\end{align*}

\section*{Reminders and Homework}
Complete the homework on TheExpertTA and WebAssign. \\
\textbf{Remember to bring the Activities Manual.} \\

\begin{center}
  You can find all my notes at \url{http://omgimanerd.tech/notes}. If you have
  any questions, comments, or concerns, please contact me at
  alvin@omgimanerd.tech
\end{center}

\end{document}
