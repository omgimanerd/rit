\documentclass[letterpaper, 12pt]{math}

\title{Linear Algebra}
\author{Alvin Lin}
\date{August 2017 - December 2017}

\begin{document}

\maketitle

\section*{Systems of Linear Equations}
A \textbf{linear equation} in variables \( x_1,x_2,\dots,x_n \) can be written
in this form:
\[ a_1x_1+a_2x_2+\dots+a_nx_n = b \]
where \( a_1,a_2,\dots,a_n,b \) are constants. The following examples are all
linear equations:
\begin{enumerate}
  \item \( 4x-5y = -2 \)
  \item \( r-\frac{1}{3}s+\frac{1}{5}t = \pi^2 \)
  \item \( 2x_1+4x_2 = 5-x_3+5x_4 \)
  \item \( \sqrt{3}x+\frac{\pi}{4}y-\sin(\frac{\pi}{5})z = 1 \)
\end{enumerate}
The following examples are NOT linear equations:
\begin{enumerate}
  \item \( 2xy+3z = 10 \)
  \item \( (x_1)^2-(x_2)^2 = 47 \)
  \item \( \frac{x}{y}+4z = 936 \)
  \item \( \sqrt{2x}+\frac{\pi}{4}y-\sin(\frac{\pi}{12}z) = 2 \)
  \item \( \sin(x_1)+4x_2+4^{x_3} = 22 \)
\end{enumerate}

\subsubsection*{Example}
Consider \( 3x-4y = -1 \). Characterize all solutions to this linear system.
Let \( x = t \).
\begin{align*}
  3t-4y &= 1 \\
  -4y &= -1-3t \\
  y &= \frac{1}{4}+\frac{3}{4}t
\end{align*}
\[ \vec{x} = \begin{bmatrix}x \\ y\end{bmatrix} =
  \begin{bmatrix}t \\ \frac{1}{4}+\frac{3}{4}t\end{bmatrix} =
  \begin{bmatrix}0 \\ \frac{1}{4}\end{bmatrix}+
  t\begin{bmatrix}1 \\ \frac{3}{4}\end{bmatrix} \]

\subsection*{Solving Systems of Linear Equations}
A system of \textbf{linear equations} is a finite collection of linear
equations. Solving the system involves finding the set of all ordered n-tuples
satisfying the given system. Suppose we want to find all \( \begin{bmatrix}x \\
y\end{bmatrix} \) satisfying the equations:
\begin{align*}
  2x+y &= 8 \\
  x-3y &= -3
\end{align*}
We can do this algebraically to get:
\[ \vec{v} = \begin{bmatrix}x \\ y\end{bmatrix} = \begin{bmatrix}
  3 \\ 2
\end{bmatrix} \]

\subsubsection*{Consistency}
Let \( S \) be a linear system. We say \( S \) is \textbf{consistent} if \( S \)
has at least 1 solution. Otherwise, we say \( S \) is \textbf{inconsistent}.
Any linear system \( S \) has three possibilities.
\begin{enumerate}
  \item 0 solutions
  \item 1 solutions
  \item \( \infty \) - many solutions
\end{enumerate}
For example:
\begin{align*}
  x-y &= 1 \\
  x-y &= 4
\end{align*}
In this example, \( x-y \) cannot be 2 different things. The solution set here
is \( \emptyset \), thus this system is inconsistent.

\subsubsection*{Equivalence}
Let \( S \) and \( S' \) be 2 linear systems. If \( S \) and \( S' \) have the
same solutions, they are \textbf{equivalent}. As a corollary, every inconsistent
system is equivalent.

\subsection*{Row Echelon Form}
For the system:
\begin{align*}
  a_{11}x_1+a_{12}x_2+\dots+a_{1n}x_n &= b_1 \\
  a_{21}x_1+a_{22}x_2+\dots+a_{2n}x_n &= b_2 \\
  \vdots \\
  a_{m1}x_1+a_{m2}x_2+\dots+a_{mn}x_n &= b_m
\end{align*}
\( x_1,x_2,\dots,x_n \) are variables and \( a_{ij} \) are constants. The
variables can be represented as:
\[ \begin{bmatrix} x_1 \\ x_2 \\ \vdots \\ x_n \end{bmatrix} \]
The coefficients can be represented using the coefficient matrix \( A \):
\[ A = \begin{bmatrix}
  a_{11} & a_{12} & \dots & a_{1n} \\
  a_{21} & a_{22} & \dots & a_{2n} \\
  \vdots & \vdots & \vdots & \vdots \\
  a_{m1} & a_{m2} & \dots & a_{mn}
\end{bmatrix} \]
The augmented matrix \( A' = [A|\vec{b}] \) where:
\[ \vec{b} = \begin{bmatrix} b_1 \\ b_2 \\ \vdots \\ b_m \end{bmatrix} \]
A matrix \( A \) is in row echelon form is:
\begin{enumerate}
  \item any row of all 0's occurs at the bottom of the matrix.
  \item in each nonzero row, the 1st nonzero entry (leading entry) is in a
    column to the left of any leading entries below it.
\end{enumerate}
We can use row echelon form to solve a linear equation.

\subsubsection*{Examples}
Solve the linear system represented by:
\[ A = \left[\begin{array}{cc|c}
  2 & 4 & 1 \\
  0 & -1 & 2 \\
  0 & 0 & 0
\end{array}\right] \]
\begin{align*}
  2x+4y &= 1 \\
  -y &= 2 \\
  2x+4(-2) &= 1 \\
  2x &= 9
  \begin{bmatrix}x_1 \\ x_2\end{bmatrix} = \begin{bmatrix}
    \frac{9}{2} \\ -2
  \end{bmatrix}
\end{align*}
Solve the linear system represented by:
\[ A = \left[\begin{array}{cc|c}
  1 & 0 & 1 \\
  0 & 1 & 5 \\
  0 & 0 & 4
\end{array}\right] \]
\[ x = 1 \quad y = 5 \quad 0 = 4 \]
This system is inconsistent and has no solution.

\subsubsection*{Example}
Consider:
\begin{align*}
  2x+y-z &= 3 \\
  x+5z &= 1\\
  -x+3y-2z &= 0
\end{align*}
\[ A = \begin{bmatrix}
  2 & 1 & -1 \\
  1 & 0 & 5 \\
  -1 & 3 & -2
\end{bmatrix} \]
\[ A' = \left[\begin{array}{ccc|c}
  2 & 1 & -1 & 3 \\
  1 & 0 & 5 & 1 \\
  -1 & 3 & -2 & 0
\end{array}\right] \]

\subsection*{Putting a Matrix into Reduced Echelon Form}
We will use \textbf{elementary row operations} to do this:
\begin{enumerate}
  \item We can multiply a row by a nonzero scalar.
  \item We can interchange two rows.
  \item We can add a multiple of 1 row to another row.
\end{enumerate}

\subsubsection*{Example}
\[ A = \begin{bmatrix}
  1 & 2 & -4 & -4 & 5 \\
  2 & 4 & 0 & 0 & 2 \\
  2 & 3 & 2 & 1 & 5 \\
  -1 & 1 & 3 & 6 & 5
\end{bmatrix} \]
\begin{align*}
  -2R_1+R_2 &\to R_2 \\
  -2R_1+R_3 &\to R_3 \\
  R_1+R_4 &\to R_4
\end{align*}
\[ A = \begin{bmatrix}
  1 & 2 & -4 & -4 & 5 \\
  0 & 0 & 8 & 8 & -8 \\
  0 & -1 & 10 & 9 & -5 \\
  0 & 3 & -1 & 2 & 10
\end{bmatrix} \]
\[ R_2 \leftrightarrow R_3 \]
\[ A = \begin{bmatrix}
  1 & 2 & -4 & -4 & 5 \\
  0 & -1 & 10 & 9 & -5 \\
  0 & 0 & 8 & 8 & -8 \\
  0 & 3 & -1 & 2 & 10
\end{bmatrix} \]
\[ 3R_2+R_4 \to R_4 \]
\[ A = \begin{bmatrix}
  1 & 2 & -4 & -4 & 5 \\
  0 & -1 & 10 & 9 & -5 \\
  0 & 0 & 8 & 8 & -8 \\
  0 & 0 & 29 & 29 & -5
\end{bmatrix} \]
\[ \frac{1}{8}R_3 \to R_3 \]
\[ A = \begin{bmatrix}
  1 & 2 & -4 & -4 & 5 \\
  0 & -1 & 10 & 9 & -5 \\
  0 & 0 & 1 & 1 & -1 \\
  0 & 0 & 29 & 29 & -5
\end{bmatrix} \]
\[ -29R_3+R_4 \to R_4 \]
\[ A = \begin{bmatrix}
  1 & 2 & -4 & -4 & 5 \\
  0 & -1 & 10 & 9 & -5 \\
  0 & 0 & 1 & 1 & -1 \\
  0 & 0 & 0 & 0 & 24
\end{bmatrix} \]
This process is called Gaussian elimination. The \textbf{rank} of the resulting
matrix \( A \) (number of nonzero rows) is 4.

\subsubsection*{Example}
Solve the system:
\begin{align*}
  2x_2+3x_3 &= 8 \\
  2x_1+3x_2+x_3 &= 5 \\
  x1-x_2-2x_3 &= -5
\end{align*}
\[ A' = \left[\begin{array}{ccc|c}
  0 & 2 & 3 & 8 \\
  2 & 3 & 1 & 5 \\
  1 & -1 & -2 & -5
\end{array}\right] \]
\[ R_1 \leftrightarrow R_3 \]
\[ A' = \left[\begin{array}{ccc|c}
  1 & -1 & -2 & -5 \\
  2 & 3 & 1 & 5 \\
  0 & 2 & 3 & 8 \\
\end{array}\right] \]
\[ -2R_1+R_2 \to R_2 \]
\[ A' = \left[\begin{array}{ccc|c}
  1 & -1 & -2 & -5 \\
  0 & 5 & 5 & 15 \\
  0 & 2 & 3 & 8
\end{array}\right] \]
\[ \frac{1}{5}R_2 \to R_2 \]
\[ A' = \left[\begin{array}{ccc|c}
  1 & -1 & -2 & -5 \\
  0 & 1 & 1 & 3 \\
  0 & 2 & 3 & 8
\end{array}\right] \]
\[ -2R_2+R_3 \to R_3 \]
\[ A' = \left[\begin{array}{ccc|c}
  1 & -1 & -2 & -5 \\
  0 & 1 & 1 & 3 \\
  0 & 0 & 1 & 2
\end{array}\right] \]
\begin{align*}
  x-y-3z &= -5 \\
  y+z &= 3 \\
  z &= 2
\end{align*}
This resolve to the solution:
\[ \vec{x} = \begin{bmatrix}x \\ y \\ z\end{bmatrix} = \begin{bmatrix}
  0 \\ 1 \\ 2
\end{bmatrix} \]

\subsection*{Reduced Row Echelon Form}
A matrix is in \textbf{reduced row echelon form} if:
\begin{enumerate}
  \item it is in reduced echelon form.
  \item the leading entries of each row equal 1.
  \item all other entries in a column with a leading 1 are zero.
\end{enumerate}
Row echelon forms for a linear equation are not unique, while reduced row
echelon forms are unique.

\subsubsection*{Example}
Extending off the previous example:
\[ A' = \left[\begin{array}{ccc|c}
  1 & -1 & -2 & -5 \\
  0 & 1 & 1 & 3 \\
  0 & 0 & 1 & 2
\end{array}\right] \]
\begin{align*}
  -R_3+R_2 &\to R_2 \\
  2R_3+R_1 &\to R_1 \\
\end{align*}
\[ A' = \left[\begin{array}{ccc|c}
  1 & -1 & 0 & -1 \\
  0 & 1 & 0 & 1 \\
  0 & 0 & 1 & 2
\end{array}\right] \]
\[ R_2+R_1 \to R_1 \]
\[ A' = \left[\begin{array}{ccc|c}
  1 & 0 & 0 & 0 \\
  0 & 1 & 0 & 1 \\
  0 & 0 & 1 & 2
\end{array}\right] \]

\subsubsection*{Example}
Solve the system:
\begin{align*}
  w-x-y+2z &= 1 \\
  2w-2x-y+3z &= 3 \\
  -w+x-y &= -3
\end{align*}
\[ A' = \left[\begin{array}{cccc|c}
  1 & -1 & -1 & 2 & 1 \\
  2 & -2 & -1 & 3 & 3 \\
  -1 & 1 & -1 & 0 & -3
\end{array}\right] \]
\begin{align*}
  -2R_1+R_2 &\to R_2 \\
  R_1+R_3 &\to R_3
\end{align*}
\[ A' = \left[\begin{array}{cccc|c}
  1 & -1 & -1 & 2 & 1 \\
  0 & 0 & 1 & -1 & 1 \\
  0 & 0 & -2 & 2 & -2
\end{array}\right] \]
\[ 2R_2+R_3 \to R_3 \]
\[ A' = \left[\begin{array}{cccc|c}
  1 & -1 & -1 & 2 & 1 \\
  0 & 0 & 1 & -1 & 1 \\
  0 & 0 & 0 & 0 & 0
\end{array}\right] \]
\[ R_2+R_1 \to R_1 \]
\[ A' = \left[\begin{array}{cccc|c}
  1 & -1 & 0 & 1 & 2 \\
  0 & 0 & 1 & -1 & 1 \\
  0 & 0 & 0 & 0 & 0
\end{array}\right] \]
\begin{align*}
  w-x+z &= 2 \\
  y-z &= 1
\end{align*}
Let \( x = s \) and \( y = t \):
\[ \vec{x} = \begin{bmatrix}
  w \\ x \\ y \\ z
\end{bmatrix} = \begin{bmatrix}
  s-t+2 \\
  s \\
  1+t \\
  t
\end{bmatrix} = \begin{bmatrix}
  2 \\ 0 \\ 1 \\ 0
\end{bmatrix}+s\begin{bmatrix}
  1 \\ 1 \\ 0 \\ 0
\end{bmatrix}+t\begin{bmatrix}
  -1 \\ 0 \\ 1 \\ 1
\end{bmatrix} (s,t\in\R) \]

\subsection*{Rank Theorem}
Suppose we have a consistent linear system. Suppose we have \( n \) variables.
If \( A \) is the coefficient matrix:
\[ \text{the number of free variables} = n-rank(A) \]

\subsubsection*{Example}
Solve the system:
\begin{align*}
  x_1-x_2+2x_3 &= 3 \\
  x_1+2x_2-x_3 &= -3 \\
  2x_2-2x_3 &= 1
\end{align*}
\[ \left[\begin{array}{ccc|c}
  1 & -1 & 2 & 3 \\
  1 & 2 & -1 & -3 \\
  0 & 2 & -2 & 1
\end{array}\right] \]
\[ -R_1+R_2 \to R_2 \]
\[ \left[\begin{array}{ccc|c}
  1 & -1 & 2 & 3 \\
  0 & 3 & -3 & -6 \\
  0 & 2 & -2 & 1
\end{array}\right] \]
\[ \frac{1}{2}R_2 \to R_2 \]
\[ \left[\begin{array}{ccc|c}
  1 & -1 & 2 & 3 \\
  0 & 1 & -1 & -2 \\
  0 & 2 & -2 & 1
\end{array}\right] \]
\[ -2R_2+R_3 \to R_3 \]
\[ \left[\begin{array}{ccc|c}
  1 & -1 & 2 & 3 \\
  0 & 1 & -1 & -2 \\
  0 & 0 & 0 & 5
\end{array}\right] \]
This system is inconsistent and there is no solution.

\subsection*{Homogeneous Linear Systems}
In a homogeneous linear system, all constant terms on the right side are 0.
\begin{align*}
  a_{11}x_1+a_{12}x_2+\dots+a_{1n}x_n &= 0 \\
  a_{21}x_1+a_{22}x_2+\dots+a_{2n}x_n &= 0 \\
  \vdots \\
  a_{m1}x_1+a_{m2}x_2+\dots+a_{mn}x_n &= 0
\end{align*}
\[ \vec{x} = \begin{bmatrix}0 \\ 0 \\ \vdots \\ 0\end{bmatrix} = \vec{0} \]
For example:
\begin{align*}
  2x+3y-z &= 0 \\
  -x+5y+2z &= 0
\end{align*}
\textbf{Theorem}: In a homogeneous system with \( n \) variables, if we have
\( m \) equations with \( m < n \), then the system has infinitely many
solutions.

\begin{center}
  If you have any questions, comments, or concerns, please contact me at
  alvin@omgimanerd.tech
\end{center}

\end{document}
