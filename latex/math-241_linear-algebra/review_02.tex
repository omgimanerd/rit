\documentclass{math}

\title{Linear Algebra}
\author{Alvin Lin}
\date{August 2017 - December 2017}

\begin{document}

\maketitle

\section*{Review 2}
Exam 2 covers section 2.3, 3.1, 3.2, 3.3, 3.5, 3.6, and homeworks 5, 6, and 7.

\subsubsection*{Example}
Prove that, for square matrices \( A \) and \( B \), \( AB = BA \) if and only
if \( (A-B)(A+B) = A^2-B^2 \).
\begin{align*}
  \text{Assume } AB &= BA \\
  (A-B)(A+B) &= A^2+AB-BA-B^2 \\
  &= A^2-B^2 \\
\end{align*}
Converse:
\begin{align*}
  \text{Assume } (A-B)(A+B) &= A^2-B^2 \\
  A^2-AB-BA-B^2 &= A^2-B^2 \\
  AB-BA &= 0 \\
  AB &= BA
\end{align*}

\subsubsection*{Example}
Prove that the product of 2 upper triangular matrices is upper triangular. \\
Let \( A,B \) be upper triangular matrices. We want to show that \( P = AB \)
is upper triangular:
\begin{align*}
  [P]_{ij} &= 0 \text{ iff } i > j \\
  [P]_{ij} &= row_i(A)\cdot col_j(B) \\
  &= \sum_{k=1}^{n}a_{ij}b_{kj} \\
  &= \sum_{k=1}^{i-1}a_{ij}b_{kj}+\sum_{k=i}^{n}a_{ij}b_{kj} \\
  &= \sum_{k=1}^{i-1}0b_{kj}+\sum_{k=i}^{n}a_{ij}0 \\
  &= 0
\end{align*}

\subsubsection*{Example}
Prove that the main diagonal of a skew-symmetric matrix must consist entirely
of 0's. \\
\( A \) is \textbf{skew-symmetric} if \( A^T = -A \).
\[ [A]_{ii} = [A^T]_{ii} = [-A]_{ii} \]
For all \( i \), \( a_{ii} = -a_{ii} \), so \( a_{ii} = 0 \) for all \( i \).

\subsubsection*{Example}
Prove that is \( A \) and \( B \) are skew-symmetric, then so is \( A+B \).
\begin{align*}
  (A+B)^T &= A^T+B^T \\
  &= -A+(-B) \\
  &= (-1)(A+B)
\end{align*}
Therefore, \( A+B \) is skew-symmetric.

\subsubsection*{Example}
Prove that is \( A \) is an \( n\times n \) matrix, then \( A-A^T \) is skew
symmetric.
\begin{align*}
  (A-A^T)^T &= A^T-(A^T)^T \\
  &= A^T-A \\
  &= (-1)(A-A^T)
\end{align*}
Therefore \( A-A^T \) is skew-symmetric.

\subsubsection*{Example}
Definition: If \( A \) is a square matrix:
\[ tr(A) = \sum_{i=1}^{n}a_{ii} \]
If \( A,B \) are \( n\times n \) matrices, show that:
\begin{itemize}
  \item \( tr(A+B) = tr(A)+tr(B) \)
  \begin{align*}
    tr(A+B) &= \sum_{i=1}^{n}(a_{ii}+b_{ii}) \\
    &= \sum_{i=1}^{n}a_{ii}+\sum_{i=1}^{n}b_{ii} \\
    &= tr(A)+tr(B)
  \end{align*}
  \item \( tr(kA) = k~tr(A) \)
  \begin{align*}
    tr(kA) &= \sum_{i=1}^{n}(ka_{ii}) \\
    &= k\left(\sum_{i=1}^{n}a_{ii}\right) \\
    &= k~tr(A)
  \end{align*}
\end{itemize}

\subsubsection*{Example}
If \( A \) is any matrix, what does \( tr(AA^T) \) equal.
\begin{align*}
  tr(AA^T) &= \sum_{i=1}^{n}row_i(A)\cdot col_i(A^T) \\
  &= \sum_{i=1}^{n}row_i(A)\cdot row_i(A) \\
  &= \sum_{i=1}^{n}\|row_i(A)\|^2
\end{align*}

\subsubsection*{Example}
Show that if \( A \) is a square matrix satisfying \( A^2-2A+I = 0 \), then
\( A^{-1} = 2I - A \).
\begin{align*}
  AA^{-1} &= I \\
  A(2I-A) &= I \\
  2A-A^2 &= I \\
  A^2-2A+I &= 0 \quad \text{given to be true} \\
  A^{-1} = 2I-A
\end{align*}

\subsubsection*{Example}
Prove if a symmetric matrix is invertible, then its inverse is symmetric too.
Let \( A \) be a symmetric and invertible matrix.
\begin{align*}
  (A^{-1})^T &= (A^T)^{-1} \\
  &= (A^{-1})^T
\end{align*}
Therefore, \( A^{-1} \) is symmetric.

\subsubsection*{Example}
In \( \R^2: S = \{\begin{bmatrix}x \\ y\end{bmatrix}\mid x\ge0 \wedge y\ge0\}
\). Is \( S \) a subspace of \( \R^2 \)? \\
Note that \( \begin{bmatrix}1 \\ 1\end{bmatrix}\in S \).
\[ -1\begin{bmatrix}1 \\ 1\end{bmatrix} = \begin{bmatrix}-1 \\ -1\end{bmatrix}
  \notin S \]
So \( S \) is not a subspace.

\subsubsection*{Example}
Is \( S = \{\begin{bmatrix}x \\ y\end{bmatrix}\in\R^2\mid xy\ge0\} \) a subspace
of \( \R^2 \).
\begin{align*}
  \begin{bmatrix}-50 \\ 0\end{bmatrix} &\in S \\
  \begin{bmatrix}1 \\ 100\end{bmatrix} &\in S \\
  \begin{bmatrix}-50 \\ 0\end{bmatrix}+\begin{bmatrix}1 \\ 100\end{bmatrix}
    &= \begin{bmatrix}-49 \\ 100\end{bmatrix} \notin S
\end{align*}
This subspace is not closed under addition and thus is not a subspace.

\subsubsection*{Example}
Is \( S = \{\begin{bmatrix}x \\ y \\ z\end{bmatrix}\mid x = 2x \wedge y = 0 \}
\) a subspace of \( \R^3 \)? Why?
\[ \begin{bmatrix}x \\ y \\ z\end{bmatrix} =
  \begin{bmatrix}x \\ 0 \\ 2x\end{bmatrix} =
  x\begin{bmatrix}1 \\ 0 \\ 2\end{bmatrix} \]
\[ S = span(\begin{bmatrix}1 \\ 0 \\ 2\end{bmatrix}) \]
So \( S \) is a subspace.

\subsubsection*{Example}
In \( \R^2 \), \( S \) consists of the union of the x-axis and y-axis. Is
\( S \) a subspace of \( \R^2 \)?
\[ \vec{e_1}\in S \wedge \vec{e_2}\in S \]
\[ \vec{e_1}+\vec{e_2} = \begin{bmatrix}1 \\ 1\end{bmatrix}\notin S \]
Therefore, \( S \) is not a subspace.

\subsubsection*{Example}
\( A \) is a \( 3\times 5 \) matrix. Why are the column vectors of \( A \)
linearly dependent. \\
Fact: If \( A \) is an \( m\times n \) matrix:
\begin{align*}
  rank(A) &\le min(m,n) \\
  A\vec{x} &= \vec{0} \\
  5 &= rank(A)+nullity(A) \\
  rank(A) &\le 3 \\
\end{align*}
The nullity of \( A \) must be at least 2, therefore \( A\vec{x} = \vec{0} \)
has a non-trivial solution. Thus, the columns of \( A \) are linearly
dependent.

\begin{center}
  You can find all my notes at \url{http://omgimanerd.tech/notes}. If you have
  any questions, comments, or concerns, please contact me at
  alvin@omgimanerd.tech
\end{center}

\end{document}
