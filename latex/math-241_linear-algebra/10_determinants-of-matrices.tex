\documentclass{math}

\usepackage{enumerate}

\title{Linear Algebra}
\author{Alvin Lin}
\date{August 2017 - December 2017}

\begin{document}

\maketitle

\section*{Determinants of Matrices}
If \( A = \begin{bmatrix}a & b \\ c & d\end{bmatrix} \), then the determinant
of \( A \) is \( ad-bc \).
\[ \det(A) = |A| = ad-c \]
This only works for \( 2\times2 \) matrices, however. For larger matrices we
have a different recursive definition. Let \( A \) be an \( n\times n \)
matrix. Let \( A_{ij} \) be the matrix \( A \) with row \( i \), col \( j \)
deleted. The \( (i,j) \)-cofactor of \( A \) is defined by:
\[ C_{ij} = (-1)^{i+j}det(A_{ij}) \]
Then:
\[ \det(A) = \sum_{j=1}^{n}a_{ij}C_{ij} \]
\textbf{Theorem:}
\begin{align*}
  \det(A) &= \sum_{j=1}^{n}a_{ij}C_{ij} \\
  &= \sum_{i=1}^{n}a_{ij}C_{ij}
\end{align*}

\subsubsection*{Example}
Find \( \det(A) \).
\[ A = \begin{bmatrix}5 & -3 & 2 \\ 1 & 0 & 2 \\ 2 & -1 & 3\end{bmatrix} \]
\begin{align*}
  \det(A) &= (-1)(1)A_{21}+0A_{22}+(-1)(2)A_{23} \\
  \det(A) &= (-1)(1)\begin{vmatrix}-3 & 2 \\ -1 & 3\end{vmatrix}+
    0\begin{vmatrix}5 & 2 \\ 2 & 3\end{vmatrix}+
    (-1)(2)\begin{vmatrix}5 & -3 \\ 2 & -1\end{vmatrix} \\
  &= -1(9-(-2))+0+(-2)(-5-(-6)) \\
  &= (-1)(-7)+(-2)(1) \\
  &= 7-2 = 5
\end{align*}

\subsection*{Some Helpful Facts}
Let \( A \) be a triangular matrix (upper or lower). Then \( \det(A) \) is
the product of the entries on the main diagonal.
\[ \det(A) = |A| = \prod_{i=1}^{n}a_{ii} \]
\[ \det\left(\begin{bmatrix}2 & 3 & 4 \\ 0 & 5 & 6 \\ 0 & 0 & 7
  \end{bmatrix}\right) = 2(5)(7) \]
\( A \) is invertible if and only if \( \det(A) \ne 0 \).

\subsection*{How Determinants Are Useful}
\begin{enumerate}[(i)]
  \item The determinant of \( A \) tells you whether or not \( A \) is
  invertible. \( A \) is invertible if and only if \( \det(A) \ne 0 \).
  \item \textbf{Kramer's Rule:} Suppose we have
  \begin{align*}
    a_{11}x_1+\dots+a_{1n}x_n &= b_1 \\
    a_{21}x_1+\dots+a_{2n}x_n &= b_2 \\
    \vdots & \vdots \\
    a_{n1}x_1+\dots+a_{nn}x_n &= b_n
  \end{align*}
  and we want to solve
  \[ \vec{b} = \begin{bmatrix}b_1 \\ b_2 \\ \vdots \\ b_n\end{bmatrix} \]
  Let \( A_i(\vec{b}) = A \) when we replace column \( i \) of \( A \) with
  \( b \). If \( \vec{x} = \begin{bmatrix}x_1 \\ x_2 \\ \vdots \\ x_n
  \end{bmatrix} \):
  \[ x_i = \frac{|A_i(\vec{b})|}{|A|} \]
  Why is Cramer's Rule valid? Consider:
  \begin{align*}
    AI_i(\vec{x}) &= A[\vec{e_1}\dots\vec{x}\vec{e_n}] \\
    &= [A\vec{e_1}\dots A\vec{x}\dots A\vec{e_n}] \\
    &= [\vec{a_1}\dots\vec{x}\dots\vec{a_n}] \\
    &= A_i(\vec{x}) \\
    |A|x_i &= |A||I_i(\vec{x})| \\
    &= |AI_i(\vec{x})| \\
    &= |A_i(\vec{b})| \\
    x_i &= \frac{|A_i(\vec{b})|}{|A|}
  \end{align*}
\end{enumerate}

\subsubsection*{Example}
Use Cramer's Rule to solve:
\begin{align*}
  x_1+2x_2 &= 2 \\
  -x_1+4x_2 &= 1
\end{align*}
\begin{align*}
  A &= \begin{bmatrix}
    1 & 2 \\
    -1 & 4
  \end{bmatrix} \quad \vec{b} = \begin{bmatrix}2 \\ 1\end{bmatrix} \\
  |A_1(\vec{b})| &= \begin{vmatrix}
    2 & 2 \\
    1 & 4
  \end{vmatrix} = 2(4)-1(2) = 6 \\
  |A_2(\vec{b})| &= \begin{vmatrix}
    1 & 2 \\
    -1 & 1
  \end{vmatrix} = 1(1)-(-1)(2) = 3 \\
  |A| &= 1(4)-(-1)(2) = 6 \\
  x_1 &= \frac{|A_1(\vec{b})|}{|A|} = 1 \\
  x_2 &= \frac{|A_2(\vec{b})|}{|A|} = \frac{1}{2} \\
  \vec{x} &= \begin{bmatrix}
    x_1 \\ x_2
  \end{bmatrix} = \begin{bmatrix}
    1 \\ \frac{1}{2}
  \end{bmatrix}
\end{align*}

\subsubsection*{Example}
Use Cramer's Rule to solve:
\begin{align*}
  2x-y &= 5 \\
  x+3y &= -1
\end{align*}
\begin{align*}
  A &= \begin{bmatrix}
    2 & -1 \\
    1 & 3
  \end{bmatrix} \quad \vec{b} = \begin{bmatrix}5 \\ -1\end{bmatrix} \\
  |A_1(\vec{b})| &= \begin{vmatrix}
    5 & -1 \\
    -1 & 3
  \end{vmatrix} = 5(3)-(-1)(-1) = 14 \\
  |A_2(\vec{b})| &= \begin{vmatrix}
    2 & 5 \\
    1 & -1
  \end{vmatrix} = 2(-1)-1(5) = -7 \\
  |A| &= 2(3)-1(-1) = 7 \\
  x_1 &= \frac{|A_1(\vec{b})|}{|A|} = \frac{14}{7} = 2 \\
  x_2 &= \frac{|A_2(\vec{b})|}{|A|} = \frac{-7}{7} = -1 \\
  \vec{x} &= \begin{bmatrix}x_1 \\ x_2\end{bmatrix} = \begin{bmatrix}
    2 \\ -1
  \end{bmatrix}
\end{align*}

\subsection*{Adjoint Formula for \( A^{-1} \)}
\[ A^{-1} = \frac{1}{\det(A)}adjoint(A) \]
\[ C_{ij} = (-1)^{i+j}(A_{ij}) \]
Let \( [C_{ij}] \) be the cofactor matrix of \( A \):
\[ C = [C_{ij}] \quad adjoint(A) = C^T \]

\subsubsection*{Example}
Show that the eigenvalues of \( A = \begin{bmatrix}a & b \\ c & d\end{bmatrix}
\) are the solutions of:
\[ \lambda^2-tr(A)\lambda+\det(A) = 0 \]
\begin{align*}
  |A-\lambda I| &= \begin{vmatrix}
    a-\lambda & b \\
    c & d-\lambda
  \end{vmatrix}  = 0 \\
  &= (a-\lambda)(d-\lambda)-bc \\
  &= ad-a\lambda-d\lambda+\lambda^2-bc \\
  &= \lambda^2-(a+d)\lambda+(ad-bc) \\
  &= \lambda^2-tr(A)\lambda+\det(A) = 0
\end{align*}

\subsubsection*{Example}
Show that the eigenvalues in the above example are:
\[ \lambda = (\frac{1}{2})((a+2)\pm\sqrt{(a-d)^2+4bc}) \]
This comes from the quadratic formula:
\begin{align*}
  A\lambda^2+B\lambda+C &= 0 \\
  \lambda &= \frac{-B\pm\sqrt{B^2-4AC}}{2A} \\
  A &= 1 \\
  B &= -(a+d) \\
  C &= ad-bc \\
  \lambda &= (\frac{1}{2})((a+2)\pm\sqrt{(a-d)^2+4bc})
\end{align*}

\subsubsection*{Example}
Show that the trace and determinant of \( A \) are given by:
\begin{align*}
  tr(A) &= \lambda_1+\lambda_2 \\
  \det(A) &= \lambda_1\lambda_2 \\
  (\lambda-\lambda_1)(\lambda-\lambda_2) &=
    \lambda^2+\lambda(-\lambda_1-\lambda_2)+\lambda_1\lambda_2 \\
  &= \lambda^2-\lambda(\lambda_1+\lambda_2)+\lambda_1\lambda_2
\end{align*}

\subsubsection*{Example}
Find all values of \( k \) such that \( A \) is invertible.
\[ A = \begin{bmatrix}k & -k & 3 \\ 0 & k+1 & 1 \\ k & -8 & k-1\end{bmatrix} \]
Find all values of \( k \) such that \( |A|\ne0 \).
\begin{align*}
  A &= k\begin{vmatrix}
    1 & -k & 3 \\
    0 & k+1 & 1 \\
    1 & -8 & k-1
  \end{vmatrix} \quad \text{(factor from column)}\\
  &= k\begin{vmatrix}
    1 & -k & 3 \\
    0 & k+1 & 1 \\
    0 & -8+k & k-4
  \end{vmatrix} \\
  &= k\begin{vmatrix}
    k+1 & 1 \\
    -8+k & k-4
  \end{vmatrix} \\
  &= k((k+1)(k-4)-(-8+k)) \\
  &= k(k^2-3k-4+8-k) \\
  &= k(k^2-4k+4) \\
  &= k(k-2)^2 \\
  0 &\ne k(k-2)^2 \\
  k &\ne 0,2
\end{align*}

\subsubsection*{Example}
Find all values of \( k \) such that \( A \) is invertible:
\[ A = \begin{bmatrix}k & k & 0 \\ k^2 & 2 & k \\ 0 & k & k\end{bmatrix} \]
\begin{align*}
  |A| &= |A^T| \\
  &= \begin{vmatrix}
    k & k^2 & 0 \\
    k & 2 & k \\
    0 & k & k
  \end{vmatrix} \\
  &= k\begin{vmatrix}
    1 & k^2 & 0 \\
    1 & 2 & k \\
    0 & k & k
  \end{vmatrix} \\
  &= k\begin{vmatrix}
    1 & k^2 & 0 \\
    0 & 2-k^2 & k \\
    0 & k & k
  \end{vmatrix} \\
  &= k\begin{vmatrix}
    2-k^2 & k \\
    k & k
  \end{vmatrix} \\
  &= k^2\begin{vmatrix}
    2-k^2 & k \\
    1 & 1
  \end{vmatrix} \\
  &= k^2((2-k^2)-k) \\
  &= k^2(-k^2-k+1) \\
  &= k^2(k^2+k-2) \\
  &= k^2(k+2)(k-1) \\
  0 &\ne k^2(k+2)(k-1) \\
  k &\ne 0,-2,1
\end{align*}

\subsubsection*{Example}
Suppose \( A \) and \( B \) are \( n\times n \) matrices:
\[ |A| = 3 \quad |B| = -2\]
Evaluate:
\begin{enumerate}[(a)]
  \item \( |AB| \):
  \[ |AB| = |A||B| = 3(-2) = 6 \]
  \item \( |A^2| \):
  \[ |A^2| = |AA| = |A||A| = (3)(3) = 9 \]
  \item \( |B^{-1}A| \):
  \[ |B^{-1}A| = |B^{-1}A| = |B^{-1}||A| = \frac{1}{|B|}|A| = \frac{1}{-2}3 \]
  \item \( |2A| \):
  \[ |2A| = 2^n|A| = (2^n)(3) \]
  \item \( |3B^T| \):
  \[ |3B^T| = 3^n|B^T| = (3^n)|B| = 3^n(-2) \]
\end{enumerate}

\subsubsection*{Example}
If \( A \) is an \( n\times n \) invertible matrix, show that \( adjoint(A) \)
is invertible.
\begin{align*}
  adjoint(A)^{-1} &= \frac{1}{|A|}A \\
  &= adjoint(A^{-1}) \\
  C &= [C_{ij}] \\
  C_{ij} &= (-1)^{i+1}|A_{ij}| \\
  adjoint(A) &= C^T \\
  A^{-1} &= \frac{1}{|A|}adjoint(A) \\
  \therefore |A|A^{-1} &= adjoint(A)
\end{align*}

\subsubsection*{Example}
If \( A \) is an \( n\times n \) matrix, then show:
\[ |adjoint(A)| = |A|^{n-1} \]
\[ \bigg||A|A^{-1}\bigg| = |A|^n|A^{-1}| = |A|^n(\frac{1}{|A|}) = |A|^{n-1} \]

\subsubsection*{Example}
Show that, for any square matrix \( A \), \( A \) and \( A^T \) have the same
characteristic polynomial.
\[ (A-\lambda I)^T = A^T-(\lambda I)^T = A^T-\lambda I \]
The characteristic polynomial of \( A = |A-\lambda I| = |(A-\lambda I)^T| =
|A^T-\lambda I| \).

\subsubsection*{Example}
Let \( A \) be a nilpotent matrix. Show that \( \lambda = 0 \) is the only
eigenvalue of \( A \). Show that \( \lambda = 0 \) is a legitimiate eigenvalue.
\[ |A^m| = |0| = 0 \]
Thus \( \lambda = 0 \) is an eigenvalue. Show that there are no other
eigenvalues. \\
Suppose \( \lambda \) is another eigenvalue \( (\lambda\ne0) \).
Then \( \lambda^m \) is an eigenvalue of \( A^m \). This forces \( \lambda^m =
0 \therefore \lambda = 0 \).

\subsubsection*{Example}
Let \( A \) be an idempotent matrix. Show that \( \lambda = 0,\lambda = 1 \) are
the only eigenvalues. \\
Since \( A \) is idempotent, it means that \( A^2 = A \). Let \( \lambda \) be
an eigenvalue of \( A \), then \( \lambda^2 \) is an eigenvalue of
\( A^2 = A \).
\begin{align*}
  \lambda^2 &= \lambda \\
  \lambda^2-\lambda &= 0 \\
  \lambda(\lambda-1) &= 0 \\
  \lambda &= 0,1
\end{align*}

\subsubsection*{Example}
If \( \vec{v} \) is an eigenvector of \( A \) with eigenvalue \( \lambda \) and
\( c \) is a scalar, show that \( \vec{v} \) is an eigenvector of \( A-cI \)
with corresponding eigenvalue \( \lambda-c \).
\[ (A-cI)(\vec{v}) = A\vec{v}-(cI)(\vec{v}) = \lambda\vec{v}-c\vec{v} =
  (\lambda-c)\vec{v} \]

\subsection*{Companion Matrix}
Let \( p(x) = x^n+a_{n-1}x^{n-1}+\dots+a_1x+a_0 \). The companion matrix of
\( p \) is defined by:
\[ C(p) = \begin{bmatrix}
  -a_{n-1} & -a_{n-2} & \dots & -a_1 & -a_0 \\
  1 & 0 & \dots & 0 & 0 \\
  0 & 1 & \dots & 0 & 0 \\
  \vdots & \vdots & \vdots & \vdots & \vdots \\
  0 & 0 & \dots & 1 & 0
\end{bmatrix} \]

\subsubsection*{Example}
Find the companion matrix of \( p(x) = x^2-7x+12 \).
\[ C(p) = \begin{bmatrix}7 & -12 \\ 1 & 0\end{bmatrix} \]
Now find the characteristic polynomial for \( C(p) \):
\begin{align*}
  |C(p)-\lambda I| &= \begin{vmatrix}
    7-\lambda & -12 \\
    1 & -\lambda
  \end{vmatrix} \\
  &= (-\lambda)(7-\lambda)-1(-12) \\
  &= \lambda^2-7\lambda+12 \\
  &= (-1)^2p(\lambda)
\end{align*}

\subsubsection*{Example}
Find the companion matrix of \( p(x) = x^3+3x^2-4x+12 \).
\[ C(p) = \begin{bmatrix}
  -3 & 4 & 12 \\
  1 & 0 & 0 \\
  0 & 1 & 0
\end{bmatrix} \]
What is the characteristic polynomial for \( C(p) \)? We find the characteristic
polynomial for \( C(p) \) by taking:
\begin{align*}
  |C(p)-\lambda I| &= -p(\lambda) \\
  &= (-1)^3p(\lambda)
\end{align*}
In general, if \( p(x) = x^n+a_{n-1}x^{n-1}+\dots+a_1x+a_0 \), the
characteristic polynomial of \( C(p) = (-1)^np(\lambda) \).

\begin{center}
  You can find all my notes at \url{http://omgimanerd.tech/notes}. If you have
  any questions, comments, or concerns, please contact me at
  alvin@omgimanerd.tech
\end{center}

\end{document}
