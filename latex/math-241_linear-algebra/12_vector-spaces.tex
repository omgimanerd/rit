\documentclass{math}

\title{Linear Algebra}
\author{Alvin Lin}
\date{August 2017 - December 2017}

\begin{document}

\maketitle

\section*{Vector Spaces}
\textbf{Definition:} Let \( V \) be a set with 2 operations called addition or
multiplication. If \( \vec{u},\vec{v}\in V \), denote their sum by \( \vec{u}+
\vec{v} \). If \( c \) is a scalar, the \textbf{scalar multiple} of
\( \vec{u} \) by \( c \) is denoted \( c\vec{u} \). If the following axioms
hold for all \( \vec{u},\vec{v},\vec{w}\in V \) and for all scalars \( c,d \),
then \( V \) is called a \textbf{vector space} and its elements are vectors.
\begin{enumerate}
  \item \( \vec{u}+\vec{v}\in V \) (Closure under addition)
  \item \( \vec{u}+\vec{v} = \vec{v}+\vec{u} \)
  \item \( (\vec{u}+\vec{v})+\vec{w} = \vec{u}+(\vec{v}+\vec{w}) \)
  \item These exists and element \( \vec{0} \), called the zero vector such that
  \( \vec{u}+\vec{0} = \vec{u} \).
  \item For each vector \( \vec{u}\in V \), there exists \( -\vec{u}\in V \)
  such that \( \vec{u}+(-\vec{u}) = \vec{0} \).
  \item \( c\vec{u}\in V \) (Closure under scalar multiplication)
  \item \( c(\vec{u}+\vec{v}) = c\vec{u}+c\vec{v} \)
  \item \( (c+d)\vec{u} = c\vec{u}+d\vec{u} \)
  \item \( c(d\vec{u}) = (cd)\vec{u} \)
  \item \( 1\vec{u} = \vec{u} \)
\end{enumerate}
The scalars \( c,d \) will be taken from a field \( F \). Usually \( F = \R~or~
\mathbb{C} \). If \( F = \R \), we say \( V \) is a real vector space. If
\( F = \mathbb{C} \), we say \( V \) is a complex vector space.

\subsubsection*{Examples of Vector Spaces}
\begin{enumerate}
  \item \( \R^n \)
  \item \( M_{mn} \), the set of all \( m\times n \) matrices.
  \item Consider the set \( P_2 = \{a_0+a_1x+a_2x^2\mid a_i\in\R\} \) is a
  vector space where the zero polynomial is the zero vector.
  \item Consider \( P_n = \{a_0+a_1x+\dots+a_nx^n\mid a_i\in\R\} \) is a
  vector space.
  \item \( P \) the set of all polynomials.
  \item \( F = \{f:\R\to\R\} \)
  \begin{itemize}
    \item \( (f+g)(x) = f(x)+g(x) \)
    \item \( (cf)(x) = c(f(x)) \)
  \end{itemize}
  \item \( F[a,b] = \{f:[a,b]\to\R\} \)
\end{enumerate}
The set of integers \( \Z \) is not a vector space since it is not closed under
multiplication. The set of complex numbers \( \mathbb{C} \) is a complex vector
space. \( \mathbb{C}^n \) is also a complex vector space.

\subsection*{Theorem}
Let \( V \) be a vector space with \( \vec{u}\in V \) and \( c \) a scalar.
\begin{enumerate}
  \item \( 0\vec{u} = \vec{0} \)
  \item \( c\vec{0} = \vec{0} \)
  \item \( (-1)\vec{u} = -\vec{u} \)
  \item \( c\vec{u} = \vec{0}\to c = 0~or~\vec{u}=\vec{0} \)
\end{enumerate}

\subsection*{Subspaces of \( V \)}
Let \( V \) be a vector space. A subset \( W\subseteq V \) is a subspace if:
\begin{enumerate}
  \item \( \vec{0}\in W \)
  \item \( \vec{u},\vec{u}\in W\to \vec{u}+\vec{v}\in W \)
  \item If \( \vec{u}\in W \), then \( c\vec{u}\in W \)
\end{enumerate}

\subsubsection*{Example}
Let \( W \) be the subset of \( M_{nn} \) that are symmetric. Verify that
\( W \) is a subspace.
\begin{enumerate}
  \item Is \( \vec{0}\in W \)?
  \[ 0^T = 0 \]
  \item Let \( A,B\in W \).
  \[ (A+B)^T = A^T+B^T = A+B\in W \]
  \item Let \( A\in W \)
  \[ (cA)^T = ca^T = cA\in W \]
\end{enumerate}

\subsubsection*{Example}
Let \( C \) be the set of continuous functions and let \( D \) be the set of
differentiable functions. Verify that \( C \) is a subspace.
\begin{enumerate}
  \item \( \vec{0}\in C \)
  \item Let \( f,g\in C \). Then \( (f+g)\in C \) by the limit laws.
  \item Let \( f\in C \).
  \[ (cf)\in C \]
\end{enumerate}
Verify that \( D \) is a subspace.
\begin{enumerate}
  \item \( \vec{0}\in D \)
  \item Let \( f,g\in D \)
  \[ (f+g)' = f'+g'\in D \]
  \item Let \( f\in D \).
  \[ (cf)' = cf'\in D \]
\end{enumerate}

\begin{center}
  You can find all my notes at \url{http://omgimanerd.tech/notes}. If you have
  any questions, comments, or concerns, please contact me at
  alvin@omgimanerd.tech
\end{center}

\end{document}
