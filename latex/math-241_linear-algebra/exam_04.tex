\documentclass{math}

\usepackage{enumerate}

\title{Linear Algebra}
\author{Alvin Lin, Michael Hennessey, Uno Myo}
\date{August 2017 - December 2017}

\begin{document}

\maketitle

\section*{Exam 4}

\subsection*{Problem 1a}
Let \( x_1, x_2, x_3 \) be numbers. Let
\[ V_3 = \begin{bmatrix}
  1 & x_1 & x_1^2 \\
  1 & x_2 & x_2^2 \\
  1 & x_3 & x_3^2
\end{bmatrix} \]
Show that \( |V_3| = (x_2-x_1)(x_3-x_1)(x_3-x_2) \).
\begin{align*}
  |V_3| &= \begin{vmatrix}
    1 & x_1 & x_1^2 \\
    1 & x_2 & x_2^2 \\
    1 & x_3 & x_3^2
  \end{vmatrix} \\
  &= 1(x_2x_3^2-x_3x_2^2)-x_1(x_3^2-x_2^2)+x_1^2(x_3-x_2) \\
  &= x_2x_3^2-x_3x_2^2-x_1x_3^2+x_1x_2^2+x_3x_1^2-x_2x_1^2 \\
  &= x_2x_3^2-x_3x_2^2-x_1x_3^2+x_1x_2^2+x_3x_1^2-x_2x_1^2+
    (x_1x_2x_3-x_1x_2x_3) \\
  &= (x_2x_3^3-x_1x_2x_3-x_1x_3^2+x_1^2x_3)-
    (x_2^2x_3+x_2^2x_1+x_1x_2x_3-x_1^2x_2) \\
  &= (x_2x_3-x_2x_1-x_1x_3+x_1^2)(x_3-x_2) \\
  &= (x_2-x_1)(x_3-x_1)(x_3-x_2)
\end{align*}

\subsection*{Problem 1b}
Let \( x_1,x_2,\dots,x_n \) be numbers. Write:
\[ V_n = \begin{bmatrix}
  1 & x_1 & x_1^2 & \dots & x_1^{n-1} \\
  1 & x_2 & x_2^2 & \dots & x_2^{n-1} \\
  \vdots & \vdots & \vdots & \vdots & \vdots \\
  1 & x_n & x_n^2 & \dots & x_n^{n-1}
\end{bmatrix} \]
Use mathematical induction to show that:
\[ |V_n| = \prod_{i<j}(x_j-x_i) \]
\begin{align*}
  V_n &= \begin{vmatrix}
    1 & x_1 & x_1^2 & \dots & x_1^{n-1} \\
    1 & x_2 & x_2^2 & \dots & x_2^{n-1} \\
    \vdots & \vdots & \vdots & \vdots & \vdots \\
    1 & x_n & x_n^2 & \dots & x_n^{n-1}
  \end{vmatrix} \\
  & \text{subtract row 1 from all other rows} \\
  &= \begin{vmatrix}
    1 & x_1 & x_1^2 & \dots & x_1^{n-2} & x_1^{n-1} \\
    0 & x_2-x_1 & x_2^2-x_1^2 & \dots & x_2^{n-2}-x_1^{n-2} &
      x_2^{n-1}-x_1^{n-1} \\
    0 & \vdots & \vdots & \vdots & \vdots & \vdots \\
    0 & x_n-x_1 & x_n^2-x_1^2 & \dots & x_n^{n-2}-x_1^{n-2} &
      x_n^{n-1}-x_1^{n-1}
  \end{vmatrix} \\
  & \text{subtract }(x_1\times\text{column }n-1) \text{ from column } n \\
  & \text{subtract }(x_1\times\text{column }n-2) \text{ from column } n-1 \\
  & \dots \\
  & \text{subtract }(x_1\times\text{column 1}) \text{ from column } 2 \\
  &= \begin{vmatrix}
    1 & 0 & 0 & \dots & 0 \\
    0 & x_2-x_1 & x_2^2-x_1^2-(x_1x_2-x_1^2) & \dots &
      x_2^{n-1}-x_1^{n-1}-(x_1x_2^{n-2}-x_1^{n-1}) \\
    0 & \vdots & \vdots & \vdots & \vdots \\
    0 & x_n-x_1 & x_n^2-x_1^2-(x_1x_n-x_1^2) & \dots &
      x_n^{n-1}-x_1^{n-1}-(x_1x_n^{n-2}-x_1^{n-1})
  \end{vmatrix} \\ \\
  &= \begin{vmatrix}
    1 & 0 & 0 & \dots & 0 \\
    0 & x_2-x_1 & x_2^2-x_1x_2 & \dots & x_2^{n-1}-x_1x_2^{n-2} \\
    0 & \vdots & \vdots & \vdots & \vdots \\
    0 & x_n-x_1 & x_n^2-x_1x_n & \dots & x_n^{n-1}-x_1x_n^{n-2}
  \end{vmatrix}
\end{align*}
\begin{align*}
  &= \begin{vmatrix}
    1 & 0 & 0 & \dots & 0 \\
    0 & x_2-x_1 & x_2(x_2-x_1) & \dots & x_2^{n-2}(x_2-x_1) \\
    0 & \vdots & \vdots & \vdots & \vdots \\
    0 & x_n-x_1 & x_n(x_n-x_1) & \dots & x_n^{n-2}(x_n-x_1)
  \end{vmatrix} \\
  &= (x_2-x_1)\begin{vmatrix}
    1 & 0 & 0 & \dots & 0 \\
    0 & 1 & x_2 & \dots & x_2^{n-2} \\
    0 & \vdots & \vdots & \vdots & \vdots \\
    0 & x_n-x_1 & x_n(x_n-x_1) & \dots & x_n^{n-2}(x_n-x_1)
  \end{vmatrix} \\
  &= (x_2-x_1)\dots(x_n-x_1)\begin{vmatrix}
    1 & 0 & 0 & \dots & 0 \\
    0 & 1 & x_2 & \dots & x_2^{n-2} \\
    0 & \vdots & \vdots & \vdots & \vdots \\
    0 & 1 & x_n & \dots & x_n^{n-2}
  \end{vmatrix} \\
  &= \prod_{i=2}^n(x_i-x_1)\begin{vmatrix}
    1 & x_2 & \dots & x_2^{n-2} \\
    \vdots & \vdots & \vdots & \vdots \\
    1 & x_n & \dots & x_n^{n-2}
  \end{vmatrix} \\
  &= \prod_{i=2}^n(x_i-x_1)|V_{n-1}| \\
  &= \prod_{i=2}^n(x_i-x_1)\prod_{i=3}^n(x_i-x_2)|V_{n-2}| \\
  &= \prod_{i<j}^n(x_j-x_i)
\end{align*}

\subsection*{Problem 2}
Let \( P \) be the vector space of all polynomials with real-valued
coefficients. Let \( D:P\to P \) be the differentiation operator. In general,
\( D^n:P\to P \) will be the linear transformation ``take the n'th derivative''.
\begin{enumerate}[(a)]
  \item Compute the kernel of \( D \).
  \begin{align*}
    D(a+bx) &= b \\
    ker(D) &= \{a+bx\mid D(a+bx) = 0\} \\
    &= \{a+bx\mid b = 0\} \\
    &= \{a\mid a\in\R\}
  \end{align*}
  \item Compute the kernel of \( D^2 \).
  \begin{align*}
    D^2(a+bx+cx^2) &= 2c \\
    ker(D^2) &= \{a+bx+cx^2\mid D^2(a+bx+cx^2) = 0\} \\
    &= \{a+bx+cx^2\mid 2c = 0\} \quad (c = 0) \\
    &= \{a+bx\mid a,b\in\R\}
  \end{align*}
  \item Compute the kernel of \( D^n \).
  \[ D^n(a_0+a_1x+a_2x^2+\dots+a_nx^n) = na_n \]
  \begin{align*}
    ker(D^n) &= \{a_0+a_1x+a_2x^2+\dots+a_nx^n\mid
      D^n(a_0+a_1x+a_2x^2+\dots+a_nx^n) = 0\} \\
    &= \{a_0+a_1x+a_2x^2+\dots+a_nx^n\mid na_n = 0\} \quad (a_n = 0) \\
    &= \{a_0+a_1x+a_2x^2+\dots+a_{n-1}x^{n-1}\mid
      a_0,a_1,a_2,\dots,a_{n-1}\in\R\} \\
  \end{align*}
\end{enumerate}

\subsection*{Problem 3}
Let \( V \) be the vector space of all infinitely differentiable functions, and
let \( D:V\to V \) be the derivative.
\begin{enumerate}
  \item Let \( L = D-I \), where \( I \) is the identity mapping. Compute the
  kernel of \( L \).
  \item Let \( L = D-aI \), where \( a \) is a number.
\end{enumerate}

\subsection*{Problem 4}
\begin{enumerate}[(a)]
  \item Let \( L:V\to V \) be a linear mapping such that \( L^2 = 0 \). Show
  that \( I-L \) is invertible (\( I \) is the identity mapping on \( V \)).
  \begin{align*}
    L^2 &= 0 \\
    I-L^2 &= (I-L)(I+L) \\
    I+0 &= (I-L)(I+L) \\
    I &= (I-L)(I+L) \\
    (I-L)^{-1}I &= (I-L)^{-1}(I-L)(I+L) \\
    (I-L)^{-1} &= (I+L)
  \end{align*}
  This yields the identity mapping when multiplied by \( (I-L) \).
  \item Let \( L:V\to V \) be a linear mapping such that \( L^2+2L+I = 0 \).
  Show that \( L \) is invertible.
  \begin{align*}
    L^2+2L+I &= 0 \\
    L(L+2I+L^{-1}I) &= 0 \\
    L+2I+L^{-1}I &= 0 \\
    L^{-1} &= -L-2I \\
  \end{align*}
  This yields the identity mapping when multiplied by \( L \).
  \item Let \( L:V\to V \) be a linear mapping such that \( L^n = 0 \), where
  \( n\ge3 \) is an integer. Show that \( I-L \) is invertible.
  \begin{align*}
    L^n &= 0 \quad (n\ge3) \\
    I-L^n &= (I-L)\left(\sum_{i=0}^{n-1}L^i\right) \\
    (I-L)^{-1}(I-0) &= (I-L)^{-1}(I-L)\left(\sum_{i=0}^{n-1}L^i\right) \\
    (I-L)^{-1} &= \left(\sum_{i=0}^{n-1}L^i\right)
  \end{align*}
  This yields the identity mapping when multiplied by \( I-L \).
\end{enumerate}

\subsection*{Problem 5}
Let \( V \) be a vector space. Let \( P:V\to V \) be a linear map such that
\( P^2 = P \).
\begin{enumerate}
  \item Show that \( V = ker(P)+range(P) \), and \( ker(P)\cap range(P) =
  \{\vec{0}\} \).
  \begin{align*}
    P^2 &= P \\
    P^2\vec{v} &= P\vec{v} \quad (\vec{v}\in V) \\
    range(P) &= P(\vec{v}) \\
    P^2\vec{v}-P\vec{v} &= \vec{0} \\
    ker(P) &= \vec{v}-P(\vec{v}) \\
    ker(P)+range(P) &= \vec{v}-P(\vec{v})+P(\vec{v}) \\
    &= \vec{v}\in V \\
    &= V
  \end{align*}
  \( P(P\vec{v}) \) can only be zero if only \( P\vec{v} \) is zero and no
  other element in the range can be in the kernel, therefore \( ker(P)\cap
  range(P) = \vec{0} \) since the zero vector is the only intersection.
  \item Use the result of part (a) to conclude that \( V \) is the
  \textit{direct sum} of \( ker(P) \) and \( range(P) \). \par
  Because the kernel of \( P \) and the range of \( P \) only share the zero
  vector (whose sum is \( \vec{0} \)), \( V \) is the \textit{direct sum} of
  the kernel and range of \( P \).
\end{enumerate}

\begin{center}
  You can find all my notes at \url{http://omgimanerd.tech/notes}. If you have
  any questions, comments, or concerns, please contact me at
  alvin@omgimanerd.tech
\end{center}

\end{document}
