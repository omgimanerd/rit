\documentclass{math}

\usepackage{multicol}

\geometry{letterpaper, margin=0.5in}

\title{Linear Algebra: Homework 4}
\author{Alvin Lin}
\date{August 2016 - December 2016}

\begin{document}

\maketitle

\section*{Section 2.2}

\subsubsection*{Exercise 1}
Determine whether the given matrix is in row echelon form. If it is, state
whether it is also in reduced row echelon form.
\[ \begin{bmatrix}
  1 & 0 & 1 \\
  0 & 0 & 3 \\
  0 & 1 & 1
\end{bmatrix} \]
No.

\subsubsection*{Exercise 3}
Determine whether the given matrix is in row echelon form. If it is, state
whether it is also in reduced row echelon form.
\[ \begin{bmatrix}
  0 & 1 & 3 & 0 \\
  0 & 0 & 0 & 1
\end{bmatrix} \]
Yes, and it is in reduced row echelon form.

\subsubsection*{Exercise 5}
Determine whether the given matrix is in row echelon form. If it is, state
whether it is also in reduced row echelon form.
\[ \begin{bmatrix}
  1 & 0 & 3 & -4 & 0 \\
  0 & 0 & 0 & 0 & 0 \\
  0 & 1 & 5 & 0 & 1
\end{bmatrix} \]
No.

\subsubsection*{Exercise 7}
Determine whether the given matrix is in row echelon form. If it is, state
whether it is also in reduced row echelon form.
\[ \begin{bmatrix}
  1 & 2 & 3 \\
  1 & 0 & 0 \\
  0 & 1 & 1 \\
  0 & 0 & 1
\end{bmatrix} \]
No.

\subsubsection*{Exercise 9}
Determine whether the given matrix is in row echelon form. If it is, state
whether it is also in reduced row echelon form.
\[ \begin{bmatrix}
  0 & 0 & 1 \\
  0 & 1 & 1 \\
  1 & 1 & 1
\end{bmatrix} \]
No.

\subsubsection*{Exercise 11}
Use elementary row operations to reduce the given matrix to row echelon form
and reduced row echelon form.
\begin{align*}
  A &= \begin{bmatrix}
    0 & 0 & 1 \\
    0 & 1 & 1 \\
    1 & 1 & 1
  \end{bmatrix} \\
  R_1 &\leftrightarrow R_3 \\
  REF &= \begin{bmatrix}
    1 & 1 & 1 \\
    0 & 1 & 1 \\
    0 & 0 & 1
  \end{bmatrix} \\
  R_1-R_2 &\to R_1 \\
  RREF &= \begin{bmatrix}
    1 & 0 & 0 \\
    0 & 1 & 0 \\
    0 & 0 & 1
  \end{bmatrix}
\end{align*}

\subsubsection*{Exercise 13}
Use elementary row operations to reduce the given matrix to row echelon form
and reduced row echelon form.
\begin{multicols}{2}
  \begin{align*}
    A &= \begin{bmatrix}
      3 & -2 & -1 \\
      2 & -1 & -1 \\
      4 & -3 & -1
    \end{bmatrix} \\
    -2R_2+R_3 &\to R_3 \\
    &= \begin{bmatrix}
      3 & -2 & -1 \\
      2 & -1 & -1 \\
      0 & -1 & 1
    \end{bmatrix} \\
    2R_1 &\to R_1 \\
    3R_2 &\to R_2 \\
    &= \begin{bmatrix}
      6 & -4 & -2 \\
      6 & -3 & -3 \\
      0 & -1 & 1
    \end{bmatrix} \\
    R_2-R_1 &\to R_2 \\
    &= \begin{bmatrix}
      6 & -4 & -2 \\
      0 & 1 & -1 \\
      0 & -1 & 1
    \end{bmatrix}
  \end{align*}\break
  \begin{align*}
    R_3-R_2 &\to R_3 \\
    REF &= \begin{bmatrix}
      6 & -4 & -2 \\
      0 & 1 & -1 \\
      0 & 0 & 0
    \end{bmatrix} \\
    R_1-R_2 &\to R_1 \\
    &= \begin{bmatrix}
      6 & 0 & -6 \\
      0 & 1 & -1 \\
      0 & 0 & 0
    \end{bmatrix} \\
    \frac{1}{6}R_1 &\to R_1 \\
    RREF &= \begin{bmatrix}
      1 & 0 & -1 \\
      0 & 1 & -1 \\
      0 & 0 & 0
    \end{bmatrix}
  \end{align*}
\end{multicols}

\subsubsection*{Exercise 17}
Show that the given matrices are row equivalent and find a sequence of
elementary row operations that will convert \( A \) into \( B \).
\[ A = \begin{bmatrix}
  1 & 2 \\
  3 & 4
\end{bmatrix}\quad B = \begin{bmatrix}
  3 & -1 \\
  1 & 0
\end{bmatrix} \]
\begin{align*}
  A &= \begin{bmatrix}
    1 & 2 \\
    3 & 4
  \end{bmatrix} \\
  -2R_1+R_2 &\to R_2 \\
  &= \begin{bmatrix}
    1 & 2 \\
    1 & 0
  \end{bmatrix} \\
  -\frac{1}{2}R_1 &\to R_1 \\
  &= \begin{bmatrix}
    -\frac{1}{2} & -1 \\
    1 & 0
  \end{bmatrix} \\
  -6R_2+R_1 &\to R_1 \\
  &= \begin{bmatrix}
    3 & -1 \\
    1 & 0
  \end{bmatrix} = B
\end{align*}

\subsubsection*{Exercise 23}
What is the rank of each of the matrices in Exercises 1-8?
\begin{enumerate}
  \item Rank: 3
  \item Rank: 2
  \item Rank: 2
  \item Rank: 0
  \item Rank: 2
  \item Rank: 3
  \item Rank: 3
  \item Rank: 3
\end{enumerate}

\subsubsection*{Exercise 25}
Solve the given system of equations:
\begin{align*}
  x_1+2x_2-3x_3 &= 9 \\
  2x_1-x_2+x_3 &= 0 \\
  4x_1-x_2+x_3 &= 4 \\
\end{align*}
\newpage
\begin{multicols}{2}
  \begin{align*}
    A &= \begin{bmatrix}
      1 & 2 & -3 & 9 \\
      2 & -1 & 1 & 0 \\
      4 & -1 & 1 & 4
    \end{bmatrix} \\
    -2R_2+R_3 &\to R_3 \\
    &= \begin{bmatrix}
      1 & 2 & -3 & 9 \\
      2 & -1 & 1 & 0 \\
      0 & 1 & -1 & 4
    \end{bmatrix} \\
    -2R_1+R_2 &\to R_2 \\
    &= \begin{bmatrix}
      1 & 2 & -3 & 9 \\
      0 & -5 & 7 & -18 \\
      0 & 1 & -1 & 4
    \end{bmatrix} \\
    5R_3+R_2 &\to R_2 \\
    R_2 &\leftrightarrow R_3 \\
    &= \begin{bmatrix}
      1 & 2 & -3 & 9 \\
      0 & 1 & -1 & 4 \\
      0 & 0 & 2 & 2
    \end{bmatrix} \\
  \end{align*}
  \begin{align*}
    \frac{1}{2}R_3 &\to R_3 \\
    &= \begin{bmatrix}
      1 & 2 & -3 & 9 \\
      0 & 1 & -1 & 4 \\
      0 & 0 & 1 & 1
    \end{bmatrix} \\
    -2R_2+R_1 &\to R_1 \\
    R_3+R_2 &\to R_2 \\
    &= \begin{bmatrix}
      1 & 0 & -1 & 1 \\
      0 & 1 & 0 & 3 \\
      0 & 0 & 1 & 1
    \end{bmatrix} \\
    R_1+R_3 &\to R_1 \\
    &= \begin{bmatrix}
      1 & 0 & 0 & 2 \\
      0 & 1 & 0 & 3 \\
      0 & 0 & 1 & 1
    \end{bmatrix} \\
    x_1 &= 2 \\
    x_2 &= 3 \\
    x_3 &= 1
  \end{align*}
\end{multicols}

\subsubsection*{Exercise 27}
Solve the given system of equations:
\begin{align*}
  x_1-3x_2-2x_3 &= 0 \\
  -x_1+2x_2+x_3 &= 0 \\
  2x_1+4x_2+6x_3 &= 0
\end{align*}
\begin{multicols}{2}
  \begin{align*}
    A &= \begin{bmatrix}
      1 & -3 & -2 & 0 \\
      -1 & 2 & 1 & 0 \\
      2 & 4 & 6 & 0
    \end{bmatrix} \\
    2R_2+R_3 &\to R_3 \\
    &= \begin{bmatrix}
      1 & -3 & -2 & 0 \\
      -1 & 2 & 1 & 0 \\
      0 & 8 & 8 & 0
    \end{bmatrix} \\
    \frac{1}{8}R_3 &\to R_3 \\
    &= \begin{bmatrix}
      1 & -3 & -2 & 0 \\
      -1 & 2 & 1 & 0 \\
      0 & 1 & 1 & 0
    \end{bmatrix} \\
  \end{align*}
  \begin{align*}
    -2R_3+R_2 &\to R_2 \\
    &= \begin{bmatrix}
      1 & -3 & -2 & 0 \\
      -1 & 0 & -1 & 0 \\
      0 & 1 & 1 & 0
    \end{bmatrix} \\
    R_2+R_1 &\to R_1 \\
    &= \begin{bmatrix}
      0 & -3 & -3 & 0 \\
      -1 & 0 & -1 & 0 \\
      0 & 1 & 1 & 0
    \end{bmatrix} \\
    &= \begin{bmatrix}
      1 & 0 & 1 & 0 \\
      0 & 1 & 1 & 0 \\
      0 & 0 & 0 & 0
    \end{bmatrix} \\
    x &= y = z
  \end{align*}
\end{multicols}

\subsubsection*{Exercise 29}
Solve the given system of equations:
\begin{align*}
  2r+s &= 3 \\
  4r+s &= 7 \\
  2r+5s &= -1
\end{align*}
\begin{multicols}{2}
  \begin{align*}
    A &= \begin{bmatrix}
      2 & 1 & 3 \\
      4 & 1 & 7 \\
      2 & 5 & -1
    \end{bmatrix} \\
    R_3-R_1 &\to R_3 \\
    &= \begin{bmatrix}
      2 & 1 & 3 \\
      4 & 1 & 7 \\
      0 & 4 & -4
    \end{bmatrix} \\
    \frac{1}{4}R_3 &\to R_3 \\
    -2R_1+R_2 &\to R_2 \\
    &= \begin{bmatrix}
      2 & 1 & 3 \\
      0 & -1 & 1 \\
      0 & 1 & -1
    \end{bmatrix} \\
  \end{align*}
  \begin{align*}
    R_1-R_3 &\to R_1 \\
    R_3+R_2 &\to R_2 \\
    R_3 &\leftrightarrow R_2 \\
    &= \begin{bmatrix}
      2 & 0 & 4 \\
      0 & 1 & -1 \\
      0 & 0 & 0
    \end{bmatrix} \\
    \frac{1}{2}R_1 &\to R_1 \\
    &= \begin{bmatrix}
      1 & 0 & 2 \\
      0 & 1 & -1 \\
      0 & 0 & 0
    \end{bmatrix} \\
    r &= 2 \\
    s &= -1
  \end{align*}
\end{multicols}

\subsubsection*{Exercise 31}
Solve the given system of equations:
\[ \begin{alignedat}{4}
  \frac{1}{2}x_1 &+x_2 &-x_3 &-6x_4 & &= 2 \\
  \frac{1}{6}x_1 &+\frac{1}{2}x_2 & &-3x_4 &+x_5 &= -1 \\
  \frac{1}{3}x_1 & &-2x_3 & &-4x_5 &= 8
\end{alignedat} \]
\newpage
\begingroup
\renewcommand*{\arraystretch}{1.5}
\begin{multicols}{2}
  \begin{align*}
    A &= \begin{bmatrix}
      \frac{1}{2} & 1 & -1 & -6 & 0 & 2 \\
      \frac{1}{6} & \frac{1}{2} & 0 & -3 & 1 & -1 \\
      \frac{1}{3} & 0 & -2 & 0 & -4 & 8
    \end{bmatrix} \\
    2R_1 &\to R_1 \\
    6R_2 &\to R_2 \\
    3R_3 &\to R_3 \\
    &= \begin{bmatrix}
      1 & 2 & -2 & -12 & 0 & 4 \\
      1 & 3 & 0 & -18 & 6 & -6 \\
      3 & 0 & -6 & 0 & -12 & 24
    \end{bmatrix} \\
    R_1-R_2 &\to R_2 \\
    3R_1-R_3 &\to R_3 \\
    &= \begin{bmatrix}
      1 & 2 & -2 & -12 & 0 & 4 \\
      0 & -1 & -2 & 6 & -6 & 10 \\
      0 & 6 & 0 & -36 & 12 & -12
    \end{bmatrix} \\
    6R_2+R_3 &\to R_3 \\
  \end{align*}
  \begin{align*}
    &= \begin{bmatrix}
      1 & 2 & -2 & -12 & 0 & 4 \\
      0 & -1 & -2 & 6 & -6 & 10 \\
      0 & 0 & -12 & 0 & -24 & 48
    \end{bmatrix} \\
    \frac{-1}{12}R_3 &\to R_3 \\
    2R_2+R_1 &\to R_1 \\
    &= \begin{bmatrix}
      1 & 0 & -6 & 0 & -12 & 24 \\
      0 & -1 & -2 & 6 & -6 & 10 \\
      0 & 0 & 1 & 0 & 2 & -4
    \end{bmatrix} \\
    6R_3+R_1 &\to R_1 \\
    2R_3+R_2 &\to R_2 \\
    &= \begin{bmatrix}
      1 & 0 & 0 & 0 & 0 & 0 \\
      0 & -1 & 0 & 6 & -2 & 2 \\
      0 & 0 & 1 & 0 & 2 & -4
    \end{bmatrix} \\
    x_1 &= 0 \\
    -1x_2+6x_4-2x_5 &= 2 \\
    x_3+2x_5 &= -4
  \end{align*}
\end{multicols}
\endgroup

\subsubsection*{Exercise 33}
Solve the given system of equations:
\[ \begin{alignedat}{5}
  &w &+x &+2y &+z &= 1 \\
  &w &-x &-y  &+z &= 0 \\
  &  &x  &+y  &   &= -1 \\
  &w &+x &    &+z &= 2
\end{alignedat} \]
\begin{align*}
  A &= \begin{bmatrix}
    1 & 1 & 2 & 1 & 1 \\
    1 & -1 & -1 & 1 & 0 \\
    0 & 1 & 1 & 0 & -1 \\
    1 & 1 & 0 & 1 & 2
  \end{bmatrix} \\
  R_4-R_1 &\to R_4 \\
  R_2-R_1 &\to R_1 \\
  A &= \begin{bmatrix}
    1 & 1 & 2 & 1 & 1 \\
    0 & -2 & -3 & 0 & -1 \\
    0 & 1 & 1 & 0 & -1 \\
    0 & 0 & -2 & 0 & 1
  \end{bmatrix} \\
  R_1-R_3 &\to R_1 \\
  R_2+2R_3 &\to R_2 \\
  A &= \begin{bmatrix}
    1 & 0 & 1 & 1 & 2 \\
    0 & 0 & -1 & 0 & 1 \\
    0 & 1 & 1 & 0 & -1 \\
    0 & 0 & -2 & 0 & 1
  \end{bmatrix} \\
  -2R_2+R_4 &\to R_4 \\
  R_1+R_2 &\to R_2 \\
  A &= \begin{bmatrix}
    1 & 0 & 1 & 1 & 2 \\
    0 & 0 & -1 & 0 & 1 \\
    0 & 1 & 1 & 0 & -1 \\
    0 & 0 & 0 & 0 & -1
  \end{bmatrix} \\
  0 &= -1
\end{align*}
System is inconsistent.

\subsubsection*{Exercise 39}
Show that if \( ad-bc \ne 0 \), then the system
\begin{align*}
  ax+by &= r \\
  cx+dy &= s
\end{align*}
has a unique solution.
\begin{align*}
  A &= \begin{bmatrix}a & b \\ c & d\end{bmatrix} \\
  \begin{bmatrix}a & b \\ c & d\end{bmatrix}
    \begin{bmatrix}x \\ y\end{bmatrix} &= \begin{bmatrix}r \\ s\end{bmatrix} \\
  \begin{bmatrix}x \\ y\end{bmatrix} &= \frac{1}{\det(A)}A
    \begin{bmatrix}r \\ s\end{bmatrix} \\
  &= \frac{1}{ad-bc}A\begin{bmatrix}r \\ s\end{bmatrix}
\end{align*}
If \( ad-bc \ne 0 \), then the system has a unique solution.

\subsubsection*{Exercise 40}
For what values of \( k \), if any, will the systems have no solution, a unique
solution, and infinitely many solutions?
\begin{align*}
  kx+2y &= 3 \\
  2x-4y &= -6 \\
  A &= \begin{bmatrix}
    k & 2 & 3 \\
    2 & -4 & -6
  \end{bmatrix} \\
  &= \begin{bmatrix}
    k & 2 & 3 \\
    1 & -2 & -3
  \end{bmatrix} \\
  &= \begin{bmatrix}
    k & 2 & 3 \\
    k+1 & 0 & 0
  \end{bmatrix}
\end{align*}
\( k = -1 \) will have infinitely many solutions. The system will have a unique
solution for all other values.

\subsubsection*{Exercise 41}
For what values of \( k \), if any, will the systems have no solution, a unique
solution, and infinitely many solutions?
\begin{align*}
  x+ky &= 1 \\
  kx+y &= 1 \\
  A &= \begin{bmatrix}
    1 & k & 1 \\
    k & 1 & 1
  \end{bmatrix}
\end{align*}
\( k = 1 \) will have infinitely many solutions. The system will have a unique
solution for all other values.

\subsubsection*{Exercise 42}
For what values of \( k \), if any, will the systems have no solution, a unique
solution, and infinitely many solutions?
\begin{align*}
  x-2y+3z &= 2 \\
  x+y+z &= k \\
  2x-y+4z &= k^2 \\
  A &= \begin{bmatrix}
    1 & -2 & 3 & 2 \\
    1 & 1 & 1 & k \\
    2 & -1 & 4 & k^2
  \end{bmatrix} \\
  &= \begin{bmatrix}
    1 & -2 & 3 & 2 \\
    0 & -3 & 2 & 2-k \\
    0 & -3 & 2 & 4-k^2
  \end{bmatrix} \\
  &= \begin{bmatrix}
    1 & -2 & 3 & 2 \\
    0 & -3 & 2 & 2-k \\
    0 & 0 & 0 & 2-k+(4-k^2)
  \end{bmatrix} \\
  2-k+4-k^2 &= 0 \\
  k^2+k-6 &= 0 \\
  (k+3)(k-2) &= 0 \\
  k &= -3 \quad k = 2
\end{align*}
The system has infinitely many solutions when \( k = 2 \), no solutions
otherwise.

\subsubsection*{Exercise 43}
For what values of \( k \), if any, will the systems have no solution, a unique
solution, and infinitely many solutions?
\begin{align*}
  x+y+kz &= 1 \\
  x+ky+z &= 1 \\
  kz+y+z &= -2 \\
  A &= \begin{bmatrix}
    1 & 1 & k & 1 \\
    1 & k & 1 & 1 \\
    k & 1 & 1 & -2
  \end{bmatrix}
\end{align*}
The system has no solution when \( k = 1 \), and a unique solution otherwise.

\subsubsection*{Exercise 44}
Give examples of homogeneous systems of \( m \) linear equations in \( n \)
variables with \( m = n \) and with \( m > n \) that have infinitely many
solutions and a unique solution. \\
\( m = n \): infinite solutions
\begin{align*}
  x+y &= 0 \\
  2x+2y &= 0
\end{align*}
\( m = n \): unique solution
\begin{align*}
  x+2y &= 0 \\
  4x+y &= 0
\end{align*}
\( m > n \): infinite solutions
\begin{align*}
  x+y &= 0 \\
  2x+2y &= 0 \\
  10x+10y &= 0 \\
\end{align*}
\( m > n \): unique solution
\begin{align*}
  x+y &= 0 \\
  3x+1y &= 0 \\
  10x+3y &= 0 \\
\end{align*}

\subsubsection*{Exercise 45}
Find the line of intersection of the given planes.
\begin{align*}
  3x+2y+z &= -1 \\
  2x-y+4z &= 5
\end{align*}
Let \( z = t \):
\begin{align*}
  3x+2y &= -1-t \\
  2x-y &= 5-4t \\
  A &= \begin{bmatrix}
    3 & 2 & -1-t \\
    2 & -1 & 5-4t
  \end{bmatrix} \\
  &= \begin{bmatrix}
    6 & 4 & -2-2t \\
    3 & -1 & 5-4t
  \end{bmatrix} \\
  &= \begin{bmatrix}
    0 & 6 & 6t-12 \\
    3 & -1 & 5-4t
  \end{bmatrix} \\
  &= \begin{bmatrix}
    0 & 1 & t-2 \\
    3 & 0 & 3-3t
  \end{bmatrix} \\
  &= \begin{bmatrix}
    0 & 1 & t-2 \\
    1 & 0 & 1-t
  \end{bmatrix} \\
  l(t) &= \begin{cases}
    x &= t-2 \\
    y &= 1-t \\
    z &= t
  \end{cases}
\end{align*}

\begin{center}
  If you have any questions, comments, or concerns, please contact me at
  alvin@omgimanerd.tech
\end{center}

\end{document}
