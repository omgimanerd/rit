\documentclass[letterpaper, 12pt]{math}

\usepackage{amsmath}
\usepackage{amssymb}

\title{Linear Algebra}
\author{Alvin Lin}
\date{August 2017 - December 2017}

\begin{document}

\maketitle

\section*{Lines in \( \R^2 \) and \( \R^3 \)}
In \( \R^2 \), a line has the following general equation:
\[ ax+by = c \]
If \( b \ne 0 \):
\[ by = -ax+c \]
\[ y = \frac{-a}{b}x+\frac{c}{b} \]
Consider the line \( l \) given by:
\[ 2x+y = 0 \]
\[ y = -2x \]
This is a line going through the origin with a slope of -2. We can represent
the line as:
\[ 2x+y = \begin{bmatrix}2 \\ 1\end{bmatrix}\cdot
  \begin{bmatrix}x \\ y\end{bmatrix} = \vec{n}\cdot\vec{x} = 0 \]
For any line \( ax+by = 0 \), we can write it as:
\[ \vec{n}\cdot\vec{x} = 0 \]

\subsection*{Vector Form of a Line}
Let \( x = t \):
\[ \vec{x} = \begin{bmatrix}x \\ y\end{bmatrix} =
  \begin{bmatrix}t \\ -2t\end{bmatrix} = t\begin{bmatrix}1 \\-2\end{bmatrix} \]
In this example, \( \begin{bmatrix}1 \\ -2\end{bmatrix} \) is referred to as
\( \vec{d} \), the direction vector for \( l \).

\subsection*{Parametric Form of a Line}
\begin{align*}
  x &= t \\
  y &= -2t
\end{align*}
given \( t\in\R \).

\subsection*{Normal Form of a Line}
Suppose line \( l \) is described by \( 2x+y = 5 \) (general form of a line).
Let \( \vec{x} \) be a general point on \( l \) and let \( \vec{p} \) be a
fixed point on \( l \).
\begin{align*}
  \vec{n}\cdot(\vec{x}-\vec{p}) &= 0 \\
  \vec{n}\cdot\vec{x}-\vec{n}\cdot\vec{p} &= 0 \\
  \vec{n}\cdot\vec{x} = \vec{n}\cdot\vec{p}
\end{align*}
In vector form, this is represented as:
\[ \vec{x} = \begin{bmatrix}x \\ y\end{bmatrix} =
  \begin{bmatrix}t \\ 5-2t\end{bmatrix} =
  \begin{bmatrix}0 \\ 5\end{bmatrix}+
  t\begin{bmatrix}1 \\ -2\end{bmatrix} \]
In parametric form, this is represented as:
\begin{align*}
  x &= t \\
  y &= 5-2t
\end{align*}
given \( t\in\R \).

\subsubsection*{Example}
Find the vector and parametric equations for \( l \) in \( \R^3 \) given \( l \)
goes through \( P = \begin{bmatrix}1 \\ 2 \\ -1\end{bmatrix} \) and has
direction vector \( \vec{d} = \begin{bmatrix}5 \\ -1 \\ 3\end{bmatrix} \).
\[ \vec{x} = \begin{bmatrix}x \\ y \\ z\end{bmatrix} =
  \begin{bmatrix}1 \\ 2 \\ -1\end{bmatrix}+
  t\begin{bmatrix}5 \\ -1 \\ 3\end{bmatrix} \]
\[ l = \begin{cases}
  x &= 1+5t \\
  y &= 2-t \\
  z &= -1+3t
\end{cases} \]

\subsubsection*{Example}
It is possible to determine a line by knowing 2 points on it. Find the equation
of the line in \( \R^3 \) given points \( P = (-1,5,0) \) and \( Q = (2,1,1) \).
\[ \overrightarrow{PQ} = \langle3,-4,1\rangle \]
\[ \vec{x} = \begin{bmatrix}-1 \\ 5 \\ 0\end{bmatrix}+
  t\begin{bmatrix}3 \\ -4 \\ 1\end{bmatrix} \]
\[ l = \begin{cases}
  x &= -1+3t \\
  y &= 5-4t \\
  z &= t
\end{cases} \]

\section*{Planes in \( \R^3 \)}
The general form of a plane can be written as:
\[ ax+by+cz = d \]
We can rewrite this as a dot product:
\[ \begin{bmatrix}a \\ b \\ c\end{bmatrix}\cdot
  \begin{bmatrix}x \\ y \\ z\end{bmatrix} = d \]
\[ \vec{n}\cdot(\vec{x}-\vec{p}) = 0 \]
\[ \therefore \vec{n}\cdot\vec{x} = \vec{n}\cdot\vec{p} \]

\subsection*{Vector form of a Plane}
\[ \vec{x} = \begin{bmatrix}x \\ y \\ z\end{bmatrix} =
  \vec{p}+s\vec{u}+t\vec{v} \]
where \( s,t \) are parameters and \( \vec{p} \) is a fixed point on the plane.

\subsubsection*{Example}
Find the normal and general forms of plane \( \mathbb{P} \) where point
\( P = (6,0,1) \) is on \( \mathbb{P} \) with normal vector:
\[ \vec{n} = \begin{bmatrix}1 \\ 2 \\ 3\end{bmatrix} \]
Normal form:
\begin{align*}
  \vec{n}\cdot\vec{x} &= \vec{n}\cdot\vec{p} \\
  \begin{bmatrix}1 \\ 2 \\ 3\end{bmatrix}\cdot
    \begin{bmatrix}x \\ y \\ z\end{bmatrix} &=
    \begin{bmatrix}1 \\ 2 \\ 3\end{bmatrix}\cdot
    \begin{bmatrix}6 \\ 0 \\ 1\end{bmatrix} \\
\end{align*}
General form:
\begin{align*}
  x+2y+3z &= 6+0+3 \\
  x+3y+3z &= 9
\end{align*}
And for good measure, the vector form. Let \( y = s \) and \( z = t \):
\[ \vec{x} = \begin{bmatrix}x \\ y \\ z\end{bmatrix} =
  \begin{bmatrix}-2s-3t+9 \\ s \\ t\end{bmatrix} =
  \begin{bmatrix}9 \\ 0 \\ 0\end{bmatrix}+
  s\begin{bmatrix}-2 \\ 1 \\ 0\end{bmatrix}+
  t\begin{bmatrix}-3 \\ 0 \\ 1\end{bmatrix} \]

\section*{Systems of Linear Equations}
A \textbf{linear equation} in variables \( x_1,x_2,\dots,x_n \) can be written
in this form:
\[ a_1x_1+a_2x_2+\dots+a_nx_n = b \]
where \( a_1,a_2,\dots,a_n,b \) are constants. The following examples are all
linear equations:
\begin{enumerate}
  \item \( 4x-5y = -2 \)
  \item \( r-\frac{1}{3}s+\frac{1}{5}t = \pi^2 \)
  \item \( 2x_1+4x_2 = 5-x_3+5x_4 \)
  \item \( \sqrt{3}x+\frac{\pi}{4}y-\sin(\frac{\pi}{5})z = 1 \)
\end{enumerate}
The following examples are NOT linear equations:
\begin{enumerate}
  \item \( 2xy+3z = 10 \)
  \item \( (x_1)^2-(x_2)^2 = 47 \)
  \item \( \frac{x}{y}+4z = 936 \)
  \item \( \sqrt{2x}+\frac{\pi}{4}y-\sin(\frac{\pi}{12}z) = 2 \)
  \item \( \sin(x_1)+4x_2+4^{x_3} = 22 \)
\end{enumerate}

\subsubsection*{Example}
Consider \( 3x-4y = -1 \). Characterize all solutions to this linear system.
Let \( x = t \).
\begin{align*}
  3t-4y &= 1 \\
  -4y &= -1-3t \\
  y &= \frac{1}{4}+\frac{3}{4}t
\end{align*}
\[ \vec{x} = \begin{bmatrix}x \\ y\end{bmatrix} =
  \begin{bmatrix}t \\ \frac{1}{4}+\frac{3}{4}t\end{bmatrix} =
  \begin{bmatrix}0 \\ \frac{1}{4}\end{bmatrix}+
  t\begin{bmatrix}1 \\ \frac{3}{4}\end{bmatrix} \]

\subsection*{Solving Systems of Linear Equations}
A system of \textbf{linear equations} is a finite collection of linear
equations. Solving the system involves finding the set of all ordered n-tuples
satisfying the given system. Find all \( \begin{bmatrix}x \\ y\end{bmatrix} \)
satisfying the equations.
\begin{align*}
  2x+y &= 8 \\
  x-3y &= -3
\end{align*}
We can do this algebraically to get:
\[ \vec{v} = \begin{bmatrix}x \\ y\end{bmatrix} = \begin{bmatrix} 3 \\ 2
  \end{bmatrix} \]
Or we can use \textbf{reduced row echelon form} to solve it.
\[ \left[\begin{array}{cc|c}
  2 & 1 & 8 \\
  1 & -3 & -3 \\
\end{array}\right] \]

\subsection*{Consistency}
Let \( S \) be a linear system. We say \( S \) is \textbf{consistent} if \( S \)
has at least 1 solution. Otherwise, we say \( S \) is \textbf{inconsistent}.
Any linear system \( S \) has three possibilities.
\begin{enumerate}
  \item 0 solutions
  \item 1 solutions
  \item \( \infty \) - many solutions
\end{enumerate}

\subsubsection*{Example}
\begin{align*}
  x-y &= 1 \\
  x-y &= 4
\end{align*}
In this example, \( x-y \) cannot be 2 different things. The solution set here
is \( \emptyset \), thus this system is inconsistent.

\subsection*{Equivalence}
Let \( S \) and \( S' \) be 2 linear systems. If \( S \) and \( S' \) have the
same solutions, they are \textbf{equivalent}. As a corollary, every inconsistent
system is equivalent.

\subsubsection*{Example}
The following system is in a form called upper triangular form:
\begin{align*}
  x-y-z &= 2 \\
  y+3z &= 5 \\
  5z &= 10 \\
  z &= 2 \\
  y+6 &= 5 \\
  y &= -1 \\
  x+1-2 &= 2 \\
  x &= 3
\end{align*}

\begin{center}
  If you have any questions, comments, or concerns, please contact me at
  alvin@omgimanerd.tech
\end{center}

\end{document}
