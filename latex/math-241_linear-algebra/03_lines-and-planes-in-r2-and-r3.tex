\documentclass[letterpaper, 12pt]{math}

\usepackage{amsmath}
\usepackage{amssymb}

\title{Linear Algebra}
\author{Alvin Lin}
\date{August 2017 - December 2017}

\begin{document}

\maketitle

\section*{Lines in \( \R^2 \) and \( \R^3 \)}
In \( \R^2 \), a line has the following general equation:
\[ ax+by = c \]
If \( b \ne 0 \):
\[ by = -ax+c \]
\[ y = \frac{-a}{b}x+\frac{c}{b} \]
Consider the line \( l \) given by:
\[ 2x+y = 0 \]
\[ y = -2x \]
This is a line going through the origin with a slope of -2. We can represent
the line as:
\[ 2x+y = \begin{bmatrix}2 \\ 1\end{bmatrix}\cdot
  \begin{bmatrix}x \\ y\end{bmatrix} = \vec{n}\cdot\vec{x} = 0 \]
For any line \( ax+by = 0 \), we can write it as:
\[ \vec{n}\cdot\vec{x} = 0 \]

\subsection*{Vector Form of a Line}
Let \( x = t \):
\[ \vec{x} = \begin{bmatrix}x \\ y\end{bmatrix} =
  \begin{bmatrix}t \\ -2t\end{bmatrix} = t\begin{bmatrix}1 \\-2\end{bmatrix} \]
In this example, \( \begin{bmatrix}1 \\ -2\end{bmatrix} \) is referred to as
\( \vec{d} \), the direction vector for \( l \).

\subsection*{Parametric Form of a Line}
\begin{align*}
  x &= t \\
  y &= -2t
\end{align*}
given \( t\in\R \).

\subsection*{Normal Form of a Line}
Suppose line \( l \) is described by \( 2x+y = 5 \) (general form of a line).
Let \( \vec{x} \) be a general point on \( l \) and let \( \vec{p} \) be a
fixed point on \( l \).
\begin{align*}
  \vec{n}\cdot(\vec{x}-\vec{p}) &= 0 \\
  \vec{n}\cdot\vec{x}-\vec{n}\cdot\vec{p} &= 0 \\
  \vec{n}\cdot\vec{x} = \vec{n}\cdot\vec{p}
\end{align*}
In vector form, this is represented as:
\[ \vec{x} = \begin{bmatrix}x \\ y\end{bmatrix} =
  \begin{bmatrix}t \\ 5-2t\end{bmatrix} =
  \begin{bmatrix}0 \\ 5\end{bmatrix}+
  t\begin{bmatrix}1 \\ -2\end{bmatrix} \]
In parametric form, this is represented as:
\begin{align*}
  x &= t \\
  y &= 5-2t
\end{align*}
given \( t\in\R \).

\subsubsection*{Example}
Find the vector and parametric equations for \( l \) in \( \R^3 \) given \( l \)
goes through \( P = \begin{bmatrix}1 \\ 2 \\ -1\end{bmatrix} \) and has
direction vector \( \vec{d} = \begin{bmatrix}5 \\ -1 \\ 3\end{bmatrix} \).
\[ \vec{x} = \begin{bmatrix}x \\ y \\ z\end{bmatrix} =
  \begin{bmatrix}1 \\ 2 \\ -1\end{bmatrix}+
  t\begin{bmatrix}5 \\ -1 \\ 3\end{bmatrix} \]
\[ l = \begin{cases}
  x &= 1+5t \\
  y &= 2-t \\
  z &= -1+3t
\end{cases} \]

\subsubsection*{Example}
It is possible to determine a line by knowing 2 points on it. Find the equation
of the line in \( \R^3 \) given points \( P = (-1,5,0) \) and \( Q = (2,1,1) \).
\[ \overrightarrow{PQ} = \langle3,-4,1\rangle \]
\[ \vec{x} = \begin{bmatrix}-1 \\ 5 \\ 0\end{bmatrix}+
  t\begin{bmatrix}3 \\ -4 \\ 1\end{bmatrix} \]
\[ l = \begin{cases}
  x &= -1+3t \\
  y &= 5-4t \\
  z &= t
\end{cases} \]

\section*{Planes in \( \R^3 \)}
The general form of a plane can be written as:
\[ ax+by+cz = d \]
We can rewrite this as a dot product:
\[ \begin{bmatrix}a \\ b \\ c\end{bmatrix}\cdot
  \begin{bmatrix}x \\ y \\ z\end{bmatrix} = d \]
\[ \vec{n}\cdot(\vec{x}-\vec{p}) = 0 \]
\[ \therefore \vec{n}\cdot\vec{x} = \vec{n}\cdot\vec{p} \]

\subsection*{Vector form of a Plane}
\[ \vec{x} = \begin{bmatrix}x \\ y \\ z\end{bmatrix} =
  \vec{p}+s\vec{u}+t\vec{v} \]
where \( s,t \) are parameters and \( \vec{p} \) is a fixed point on the plane.

\subsubsection*{Example}
Find the normal and general forms of plane \( \mathbb{P} \) where point
\( P = (6,0,1) \) is on \( \mathbb{P} \) with normal vector:
\[ \vec{n} = \begin{bmatrix}1 \\ 2 \\ 3\end{bmatrix} \]
Normal form:
\begin{align*}
  \vec{n}\cdot\vec{x} &= \vec{n}\cdot\vec{p} \\
  \begin{bmatrix}1 \\ 2 \\ 3\end{bmatrix}\cdot
    \begin{bmatrix}x \\ y \\ z\end{bmatrix} &=
    \begin{bmatrix}1 \\ 2 \\ 3\end{bmatrix}\cdot
    \begin{bmatrix}6 \\ 0 \\ 1\end{bmatrix} \\
\end{align*}
General form:
\begin{align*}
  x+2y+3z &= 6+0+3 \\
  x+3y+3z &= 9
\end{align*}
And for good measure, the vector form. Let \( y = s \) and \( z = t \):
\[ \vec{x} = \begin{bmatrix}x \\ y \\ z\end{bmatrix} =
  \begin{bmatrix}-2s-3t+9 \\ s \\ t\end{bmatrix} =
  \begin{bmatrix}9 \\ 0 \\ 0\end{bmatrix}+
  s\begin{bmatrix}-2 \\ 1 \\ 0\end{bmatrix}+
  t\begin{bmatrix}-3 \\ 0 \\ 1\end{bmatrix} \]

\begin{center}
  If you have any questions, comments, or concerns, please contact me at
  alvin@omgimanerd.tech
\end{center}

\end{document}
