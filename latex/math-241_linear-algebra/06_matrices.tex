\documentclass[letterpaper, 12pt]{math}

\title{Linear Algebra}
\author{Alvin Lin}
\date{August 2017 - December 2017}

\begin{document}

\maketitle

\section*{Matrices}
A \textbf{matrix} is a rectangular array of numbers. Example:
\[ A_1 = \begin{bmatrix}1 & 2 \\ 3 & 4\end{bmatrix} \]
\[ \begin{bmatrix}\sqrt{7} & 1 & 0 \\ 2 & 3\pi & \frac{3}{4}\end{bmatrix} \]
\[ \begin{bmatrix}2 \\ 3 \\ 17\end{bmatrix} \text{(column matrix)} \]
\[ \begin{bmatrix}1 & 1 & 0 & 1\end{bmatrix} \text{(row matrix)} \]
An \( m\times n \) matrix has \( m \) rows and \( n \) columns.
\[ A = \begin{bmatrix}
  a_{11} & a_{12} & \dots & a_{1n} \\
  a_{21} & a_{22} & \dots & a_{2n} \\
  \vdots & \vdots & \vdots & \vdots \\
  a_{m1} & a_{m2} & \dots a_{mn}
\end{bmatrix} \]
Columns of \( A \) are denoted as: \( \vec{a_1},\vec{a_2},\dots,\vec{a_n} \) \\
Rows of \( A \) are denoted as: \( \vec{A_1},\vec{A_2},\dots,\vec{A_m} \).
\[ A = \begin{bmatrix}\vec{a_1} & \vec{a_2} & \dots & \vec{a_n}\end{bmatrix} =
  \begin{bmatrix}\vec{A_1} \\ \vec{A_2} \\ \vdots \\ \vec{A_m}\end{bmatrix} \]
For any matrix \( A \), \( a_{ij} \) is the entry in row \( i \) and column
\( j \).

\subsection*{Special Matrices}
\begin{enumerate}
  \item \textbf{Square Matrix}: A matrix with the same number of row and
    columns, an \( n \times n \) matrix.
  \item \textbf{Diagonal Matrix}: A square matrix where every entry off the
    diagonal is 0.
    \[ \begin{bmatrix}1 & 0 & 0 \\ 0 & 2 & 0 \\ 0 & 0 & 3\end{bmatrix} \]
  \item \textbf{Scalar Matrix}: A diagonal matrix where all entries on the
    diagonal are equal.
    \[ \begin{bmatrix}4 & 0 & 0 \\ 0 & 4 & 0 \\ 0 & 0 & 4\end{bmatrix} \]
  \item \textbf{Identity Matrix}: A square matrix where \( a_{ii} = 1 \) for
    \( 1\le i\le n \) is 1.
    \[ I_2 = \begin{bmatrix}1 & 0 \\ 0 & 1\end{bmatrix} \]
    \[ I_3 = \begin{bmatrix}1 & 0 & 0 \\ 0 & 1 & 0 \\ 0 & 0 & 1\end{bmatrix} \]
  \item \textbf{Zero Matrix}: The matrix where all entries are 0.
\end{enumerate}

\subsection*{Matrix Equality}
In order for matricies \( A \) and \( B \) to be equal, the sizes of \( A \) and
\( B \) must agree, and all of the corresponding entries are equal.
\begin{align*}
  A &= \begin{bmatrix}
    5 & 0 & 2 \\
    4 & 1 & -2
  \end{bmatrix} \\
  B &= \begin{bmatrix}
    5 & 0 & 2 \\
    x & y & z
  \end{bmatrix} \\
  x &= 4 \quad y = 1 \quad z = -2 \\
  B^T &= \begin{bmatrix}
    5 & x \\
    0 & y \\
    2 & z
  \end{bmatrix} \\
  B^T &\ne B \\
  B^T &\ne A \\
  \begin{bmatrix}1 & 2 & 3\end{bmatrix} &\ne
    \begin{bmatrix}1 \\ 2 \\ 3\end{bmatrix}
\end{align*}

\subsection*{Addition of Matrices}
\[ A = [a_{ij}] \quad B = [b_{ij}] \]
If \( A \) and \( B \) have the same sizes, let \( S = A+B \) where:
\[ s_{ij} = a_{ij}+b_{ij} \]
\textbf{Additive Identity}: Let \( 0_{m\times n} \) (or 0) denote the zero
matrix:
\begin{enumerate}
  \item \( A+0 = A = 0+A \)
  \item \( A+(-B) = A-B \)
\end{enumerate}

\subsubsection*{Example}
\[ A = \begin{bmatrix}1 & 2 \\ 3 & 4\end{bmatrix} \quad
  B = \begin{bmatrix}2 & 3 \\ 4 & 5 \\ 6 & 7\end{bmatrix} \]
\( A+B \) is not defined in this case.
\[ C = \begin{bmatrix}-1 & 4 \\ 2 & 3\end{bmatrix} \]
\[ A+C = \begin{bmatrix}0 & 6 \\ 5 & 7\end{bmatrix} \]

\subsection*{Scalar Multiplication}
Let \( c \) be a scalar and \( A \) be a matrix:
\[ cA = [ca_{ij}] \]

\subsubsection*{Example}
\[ 3\begin{bmatrix}2 & 3 \\ 4 & 5\end{bmatrix} =
  \begin{bmatrix}6 & 9 \\ 12 & 15\end{bmatrix} \]

\subsection*{Matrix Multiplication}
Let \( A \) be an \( m\times n \) matrix and \( B \) be an \( n\times r \)
matrix. \( A\times B = P \) is an \( m\times r \) matrix derived from
the following:
\[ P_{ij} = \vec{A_i}\cdot\vec{b_{j}} = \sum_{k=1}^{n}a_{ij}b_{kj} \]
\[ A\times B = A\begin{bmatrix}\vec{b_1} & \vec{b_2} & \dots &
  \vec{b_r}\end{bmatrix} = \begin{bmatrix}A\vec{b_1} & A\vec{b_2} & \dots
  A\vec{b_r}\end{bmatrix} \]
\[ A\times B = \begin{bmatrix}\vec{A_1} \\ \vec{A_2} \\ \vdots \\ \vec{A_m}
  \end{bmatrix}B = \begin{bmatrix}\vec{A_1}B \\ \vec{A_2}B \\ \vdots \\
  \vec{A_m}B\end{bmatrix} \]

\subsubsection*{Example}
\begin{align*}
  A = \begin{bmatrix}1 & -2 \\ 3 & 2\end{bmatrix} \\
  B = \begin{bmatrix}4 & 1 \\ -1 & 3\end{bmatrix} \\
  A\times B &= \begin{bmatrix}
    1(4)+(-2)(-1) & 1(1)+(-2)(3) \\
    3(4)+(2)(-1) & (-2)(-1)+(2)(3) \\
  \end{bmatrix} = \begin{bmatrix}6 & -5 \\ 10 & 9\end{bmatrix}
\end{align*}

\subsubsection*{Example}
Consider the linear system:
\begin{align*}
  2x-3y+5z &= 7 \\
  -3x+4y-6z &= 2 \\
  4x+y-z &= 0
\end{align*}
As a matrix equation:
\[ \begin{bmatrix}
  2 & -3 & 5 \\
  -3 & 4 & -6 \\
  4 & 1 & -1
\end{bmatrix}\begin{bmatrix}
  x \\ y \\ z
\end{bmatrix} = \begin{bmatrix}
  7 \\ 2 \\ 0
\end{bmatrix} \]
\[ A\vec{x} = \vec{b} \]

Recall the ``standard basis vectors'' \( \vec{e_i} \) where there is a 1 in
the \( i \)th place and 0's elsewhere. Let \( \vec{e_i} \) be the \( i \)th
standard column vector and \( \vec{e_j} \) be the standard row vector.
\begin{align*}
  A\vec{e_i} &= \vec{a_i} \\
  \vec{e_j} &= \vec{A_j}
\end{align*}

\subsection*{Power of Matrices}
Let \( A \) be a square matrix, look at:
\[ A^k = A\times A\times\dots A \]
By convention:
\begin{align*}
  A^0 &= I \\
  A^1 &= A \\
  A^k &= A^{k-1}A
\end{align*}

\subsubsection*{Example}
Let:
\[ A = \begin{bmatrix}1 & 1 \\ 1 & 1\end{bmatrix} \]
Then:
\begin{align*}
  A^2 &= \begin{bmatrix}2 & 2 \\ 2 & 2\end{bmatrix} \\
  A^3 &= A^2A = \begin{bmatrix}2 & 2 \\ 2 & 2\end{bmatrix}
    \begin{bmatrix}1 & 1 \\ 1 & 1\end{bmatrix} =
    \begin{bmatrix}4 & 4 \\ 4 & 4\end{bmatrix} \\
  A^4 &= A^3A = \begin{bmatrix}4 & 4 \\ 4 & 4\end{bmatrix}
    \begin{bmatrix}1 & 1 \\ 1 & 1\end{bmatrix} =
    \begin{bmatrix}8 & 8 \\ 8 & 8\end{bmatrix} \\
  A^n &= \begin{bmatrix} 2^{n-1} & 2^{n-1} \\ 2^{n-1} & 2^{n-1}\end{bmatrix}
\end{align*}
Proof by Induction:
\begin{itemize}
  \item Base Case:
  \[ A^1 = \begin{bmatrix}1 & 1 \\ 1 & 1\end{bmatrix} =
    \begin{bmatrix}2^0 & 2^0 \\ 2^0 & 2^0\end{bmatrix} \]
  \item Induction Step: Assume it is true for \( n \), prove:
  \begin{align*}
    A^{n+1} &= \begin{bmatrix}2^n & 2^n \\ 2^n & 2^n\end{bmatrix} \\
    A^{n+1} &= A^nA \\
    &= \begin{bmatrix}2^{n-1} & 2^{n-1} \\ 2^{n-1} & 2^{n-1}\end{bmatrix}
      \begin{bmatrix}1 & 1 \\ 1 & 1\end{bmatrix} \\
    &= \begin{bmatrix}2^{n-1}2^{n-1} & 2^{n-1}2^{n-1} \\
      2^{n-1}2^{n-1} & 2^{n-1}2^{n-1}\end{bmatrix} \\
    &= \begin{bmatrix}2^n & 2^n \\ 2^n & 2^n\end{bmatrix}
  \end{align*}
\end{itemize}

\subsection*{Transpose of a Matrix}
\( A^T \) will denote the \textbf{transpose} of \( A \). The row \( i \),
column \( j \) entry of \( A^T = a_{ji} \):
\begin{align*}
  A &= \begin{bmatrix}1 & 2 & 3 \\ 4 & 5 & 6\end{bmatrix} \\
  A^T &= \begin{bmatrix}1 & 4 \\ 2 & 5 \\ 3 & 6\end{bmatrix}
\end{align*}
If \( A \) is not square: \( A^T \ne A \) \\
Recall that the dot product \( \vec{u}\cdot\vec{v} \) is equal to:
\begin{align*}
  \vec{u}\cdot\vec{v} &= \sum_{i=1}^{n}u_iv_i \\
  &= \begin{bmatrix}u_1 & u_2 & \dots & u_n\end{bmatrix}\cdot
    \begin{bmatrix}v_1 \\ v_2 \\ \vdots \\ v_n\end{bmatrix} \\
  &= \vec{u^T}\vec{v}
\end{align*}

\subsection*{Symmetric Matrices}
A square matrix \( A \) is \textbf{symmetric} if \( A = A^T \). Example:
\[ I_2 = \begin{bmatrix}1 & 0 \\ 0 & 1\end{bmatrix}\quad
  I_2^T = \begin{bmatrix}1 & 0 \\ 0 & 1\end{bmatrix} \]
\( I_2^T = I_2 \), therefore \( I_2 \) is symmetric. \( I_n \) is also
symmetric.

\subsection*{Properties of Matrices}
Let \( A,B,C \) be matrices of the same size. Let \( c,d \) be scalars.
\begin{enumerate}
  \item \( A+B = B+A \)
  \item \( (A+B)+C = A+(B+C) \)
  \item \( A+0 = A \)
  \item \( A+(-A) = 0 \)
  \item \( c(A+B) = cA+cB \)
  \item \( (c+d)A = cA+dA \)
  \item \( c(dA) = (cd)A \)
  \item \( 1A = A \)
\end{enumerate}
Properties of matrix multiplication (\( k \) is a scalar):
\begin{enumerate}
  \item \( A(BC) = (AB)C \)
  \item \( A(B+C) = AB+AC \)
  \item \( (A+B)C = AC+BC \)
  \item \( k(AB) = (kA)B = A(kB) \)
  \item \( IA = AI = A \)
\end{enumerate}
Properties of transpose:
\begin{enumerate}
  \item \( (A^T)^T = A \)
  \item \( (A+B)^T = A^T+B^T \)
  \item \( (kA)^T = k(A^T) \)
  \item \( (AB)^T = B^TA^T \)
  \item \( (A^n)^T = (A^T)^n \)
\end{enumerate}
\textbf{Theorems}:
\begin{enumerate}
  \item If \( A \) is a square matrix, \( A+A^T \) is symmetric.
  \item For any matrix \( A \), \( AA^T \) and \( A^TA \) are symmetric.
\end{enumerate}
Extension:
\begin{enumerate}
  \item \( (A_1+A_2+\dots+A_n)^T = \sum_{i=1}^{n}(A_i)^T \)
  \item \( (A_1\cdot A_2\cdot\dots\cdot A_n)^T = (A_n)^T(A_{n-1})^T(A_1)^T \)
\end{enumerate}

\subsubsection*{Example}
Let \( A_1 = \begin{bmatrix}0 & 1 \\ -1 & 0\end{bmatrix} \quad
  A_2 = \begin{bmatrix}1 & 0 \\ 0 & 1\end{bmatrix} \quad
  A_3 = \begin{bmatrix}1 & 1 \\ 1 & 1\end{bmatrix} \). Suppose
\( \begin{bmatrix}w & x \\ y & z\end{bmatrix}\in span(A_1,A_2,A_3) \).
\begin{align*}
  c_1A_1+c_2A_2+c_3A_3 = \begin{bmatrix}
    w & x \\
    y & z
  \end{bmatrix} \\
  \begin{bmatrix}
    c_2+c_3 & c_1+c_3 \\
    -c_1+c_3 & c_2+c_3
  \end{bmatrix} &= \begin{bmatrix}
    w & x \\
    y & z
  \end{bmatrix}
  c_2+c_3 &= w \\
  c_1+c_3 &= x \\
  -c_1+c_3 = y \\
  c_2+c_3 = z \\
  \left[\begin{array}{ccc|c}
    0 & 1 & 1 & w \\
    1 & 0 & 1 & x \\
    -1 & 0 & 1 & y \\
    0 & 1 & 1 & z
  \end{array}\right] &\to \left[\begin{array}{ccc|c}
    1 & 0 & 0 & x-\frac{x+y}{2} \\
    0 & 1 & 0 & w-\frac{x+y}{2} \\
    0 & 0 & 1 & \frac{x+y}{2} \\
    0 & 0 & 0 & z-w
  \end{array}\right]
\end{align*}
This describes the general form of all matrixes in the span.

\subsection*{Inverse of a Matrix}
Assume \( A \) is a square matrix. If we try to solve \( A\vec{x} = \vec{b} \),
we would like to cancel \( A \) out. We want a matrix \( A' \) such that
\( A'A = AA' = I \).
\begin{align*}
  A\vec{x} &= \vec{b} \\
  A'(A\vec{x}) &= A'\vec{b} \\
  (A'A)\vec{x} &= A'\vec{b} \\
  I\vec{x} &= A'\vec{b} \\
  \vec{x} &= A'\vec{v}
\end{align*}
When this \( A' \) does exist, we say \( A \) is invertible. \\
\textbf{Theorem}: If \( A \) is an invertible \( n\times n \) matrix, then the
system of linear equations given by \( A\vec{x} = \vec{b} \) has a unique
solution. \\
Let \( A \) be a \( 2\times2 \) matrix.
\[ A = \begin{bmatrix}a & b \\ c & d\end{bmatrix} \]
Suppose \( ad-bc \ne 0 \), then \( A \) is invertible:
\begin{align*}
  A' &= \frac{1}{ad-bc}\begin{bmatrix}d & -b \\ -c & a\end{bmatrix} \\
  &= \frac{1}{det(A)}\begin{bmatrix}d & -b \\ -c & a\end{bmatrix}
\end{align*}
The inverse of a matrix is also denoted as \( A^{-1} \).

\subsection*{General Procedure for Finding \( A^{-1} \)}
\[ [A|I_n] \rightarrow [I|A^{-1}] \]
By reducing a matrix to reduced row echelon form, it is possible to find the
inverse of \( A \). This is possible when \( rank(A) < n \). \\
\textbf{Fact}:
\[ (AB)^{-1} = B^{-1}A^{-1} \]

\subsection*{Properties of Invertible Matrices}
Suppose \( A \) is invertible.
\begin{enumerate}
  \item \( (A^{-1})^{-1} = A \)
  \item \( (cA)^{-1}) = \frac{1}{c}A^{-1} \)
  \item \( (AB)^{-1} = B^{-1}A^{-1} \)
  \item \( (A^T)^{-1} = (A^{-1})^T \)
  \item \( (A^n)^{-1} = (A^{-1})^T \)
\end{enumerate}

\begin{center}
  You can find all my notes at \url{http://omgimanerd.tech/notes}. If you have
  any questions, comments, or concerns, please contact me at
  alvin@omgimanerd.tech
\end{center}

\end{document}
