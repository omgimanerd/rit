\documentclass[letterpaper, 12pt]{math}

\title{Linear Algebra}
\author{Alvin Lin}
\date{August 2017 - December 2017}

\begin{document}

\maketitle

\section*{Mathematical Induction}
We will try to prove that some statement \( P(n) \) for all integers \( n \ge
n_{\circ} \).
\begin{enumerate}
  \item (Basis) Show \( P(n_{\circ}) \) is true.
  \item (Induction) Induction Hypothesis: We assume \( P(n) \) is true.
    We must show \( P(n+1) \) is also true.
  \item We then conclude \( P(n) \) is true for all integers
    \( n \ge n_{\circ} \).
\end{enumerate}

\subsubsection*{Example}
Show that:
\[ 1+2+3+\dots+n = \frac{n(n+1)}{2} = P(n) \]
\begin{enumerate}
  \item (Basis): \( n = 1 \)
    \[ 1 = \frac{1(1+1)}{2} \]
  \item (Induction) Assume \( P(n) \) is true. Show \( P(n+1) \) is also true.
    \begin{align*}
      1+2+3+\dots+n+(n+1) &= \frac{(n+1)(n+1+1)}{2} \\
      &= \frac{(n+1)(n+2)}{2} \\
    \end{align*}
    Show that the left and right hand sides are equal.
    \begin{align*}
      (1+2+3+\dots+n)+(n+1) &= \frac{n(n+1)}{2}+\frac{2(n+1)}{2} \\
      &= \frac{n^2+n+2n+2}{2} \\
      &= \frac{n^2+3n+2}{2} \\
      &= \frac{(n+1)(n+2)}{2}
    \end{align*}
\end{enumerate}

\subsubsection*{Example}
Define:
\[ f(n) = \begin{cases}
  1 &\quad n = 0 \\
  nf(n-1) &\quad n > 0
\end{cases} \]
\( f(n) \) is the factorial function \( f(n) = n! \). Prove it for all \( n \ge
0 \).
\begin{enumerate}
  \item (Basis): \( n = 0 \)
    \[ f(0) \stackrel{?}{=} 0! \]
    \[ f(0) = 1 = 0! \quad \text{by definition} \]
  \item (Induction): Assume it is true for \( n \), assume \( f(n) = n! \).
    Show that \( f(n+1) = (n+1)! \).
    \begin{align*}
      f(n+1) &= (n+1)f(n) \\
      &= (n+1)(n!) \\
      &= (n+1)!
    \end{align*}
\end{enumerate}
\( f(n) \) is true for all integers \( n \ge 0 \).

\subsubsection*{Problem 63}
Show that:
\[ \vec{u}\cdot\vec{v} = \frac{1}{4}\|\vec{u}+\vec{v}\|^2-
  \frac{1}{4}\|\vec{u}-\vec{v}\|^2 \]
for all \( \vec{u},\vec{v}\in\R^2 \).
\begin{align*}
  \frac{1}{4}\|\vec{u}+\vec{v}\|^2-\frac{1}{4}\|\vec{u}-\vec{v}\|^2
  &= \frac{1}{4}((\vec{u}+\vec{v})\cdot(\vec{u}+\vec{v}))-
    \frac{1}{4}((\vec{u}-\vec{v})\cdot(\vec{u}-\vec{v})) \\
  &= \frac{1}{4}(\vec{u}\cdot\vec{u}+2\vec{u}\cdot\vec{v}+\vec{v}\cdot\vec{v})-
    \frac{1}{4}(\vec{u}\cdot\vec{u}-2\vec{u}\cdot\vec{v}+\vec{v}\cdot\vec{v}) \\
  &= \frac{2}{4}\vec{u}\cdot\vec{v}+\frac{2}{4}\vec{u}\cdot\vec{v} \\
  &= \vec{u}\cdot\vec{v}
\end{align*}

\subsubsection*{Problem 74}
Show that:
\[ \left\|\sum_{i=1}^{n}\vec{v_i}\right\| \le \sum_{i=1}^{n}\|\vec{v_i}\| \]
\begin{enumerate}
  \item (Basis): \( n = 2 \):
    \[ \|\vec{v_1}+\vec{v_2}\| \le \|\vec{v_1}\|+\|\vec{v_2}\| \]
  \item (Induction): Assume \( P(n) \) is true. Show that \( P(n+1) \) is also
    true.
    \begin{align*}
      \left\|\sum_{i=1}^{n}\vec{v_i}\right\| &\le
        \sum_{i=1}^{n+1}\|\vec{v_i}\| \\
      \left\|(\sum_{i=1}^{n}\vec{v_i})+\vec{v_{n+1}}\right\|
      &\le \left\|\sum_{i=1}^{n}\vec{v_i}\right\|+\|\vec{v_{n+1}}\| \\
      &\le \sum_{i=1}^{n}\|\vec{v_{i}}\|+\vec{v_{n+1}} \\
      &= \sum_{i}^{n+1}\|\vec{v_i}\|
    \end{align*}
\end{enumerate}

\begin{center}
  If you have any questions, comments, or concerns, please contact me at
  alvin@omgimanerd.tech
\end{center}

\end{document}
