\documentclass{math}

\usepackage{enumerate}

\title{Linear Algebra}
\author{Alvin Lin}
\date{August 2017 - December 2017}

\begin{document}

\maketitle

\section*{Eigenvectors and Eigenvalues}
Let \( A \) be an \( n\times n \) matrix. A scalar \( \lambda \) is called an
\textbf{eigenvalue} of \( A \) if there exists a non-zero vector \( \vec{x}\in
\R^n \) such that \( A\vec{x} = \lambda\vec{x} \). Such a vector \( \vec{x} \)
is called an \textbf{eigenvector} of \( A \) corresponding to \( \lambda \). \\
In \( \R^2 \), \( A\vec{x} = \lambda x \) means that the action of \( A \) on
\( \vec{x} \) just yields a vector parallel to \( \vec{x} \).

\subsection*{Properties}
\begin{itemize}
  \item \( A\vec{x} = \lambda\vec{x} \) for some \( \vec{x}\ne\vec{0} \).
  \item \( (A-\lambda I)\vec{x} = \vec{0} \) for some \( \vec{x}\ne\vec{0} \).
  \item \( null(A-\lambda I) \) is nontrivial.
  \item \( \det(A-\lambda I) = 0 \) (allows us to solve for \( \lambda \) to
  find eigenvalues).
  \item \( |A-\lambda I| = 0 \) is called the characteristic polynomial.
\end{itemize}
We define \( E_{\lambda} = null(A-\lambda I) \) to be the eigenspace
corresponding to \( \lambda \). Eigenvalues are the roots of \( |A-\lambda I| \)
and eigenspaces \( E_{\lambda_i} = null(A-\lambda_iI) \).

\subsubsection*{Example}
Show that \( \vec{x} = \begin{bmatrix}1 \\ 1\end{bmatrix} \) is an
eigenvector for \( A = \begin{bmatrix}3 & 1 \\ 1 & 3\end{bmatrix} \).
\begin{align*}
  A\vec{x} &= \begin{bmatrix}
    3 & 1 \\
    1 & 3
  \end{bmatrix}\begin{bmatrix}1 \\ 1\end{bmatrix}
    = \begin{bmatrix}4 \\ 4\end{bmatrix}
    = 4\begin{bmatrix}1 \\ 1\end{bmatrix} \\
  &= 4\lambda
\end{align*}
So \( \lambda = 4 \) is an eigenvalue for \( A \) and \( \vec{x} \) is a
corresponding eigenvector.

\subsubsection*{Example}
Show there exists a non-zero vector \( \vec{x} \) satisfying \( A\vec{x} =
5\vec{x} \).
\[ A = \begin{bmatrix}1 & 2 \\ 4 & 3\end{bmatrix} \]
Conclude: \( \lambda = 5 \) is an eigenvalue of \( A \).
\begin{align*}
  A\vec{x} &= 5\vec{x} \\
  (A-5I)\vec{x} &= \vec{0} \\
  A-5I &= \begin{bmatrix}-4 & 2 \\ 4 & -2\end{bmatrix}
    \sim \begin{bmatrix}-4 & 2 \\ 0 & 0\end{bmatrix}
    \sim \begin{bmatrix}1 & \frac{-1}{2} \\ 0 & 0\end{bmatrix} \\
  x_1-\frac{1}{2}x_2 &= 0 \\
  x_1 &= \frac{1}{2}x_2 \\
  \vec{x} &= \begin{bmatrix}x_1 \\ x_2\end{bmatrix}
    = x_1\begin{bmatrix}1 \\ 2\end{bmatrix} \\
  E_5 &= span\left(\begin{bmatrix}1 \\ 2\end{bmatrix}\right)
\end{align*}

\subsubsection*{Example}
Show that \( \lambda = 6 \) is an eigenvalue of:
\[ A = \begin{bmatrix}7 & 1 & -2 \\ -3 & 3 & 6 \\ 2 & 2 & 2\end{bmatrix} \]
We need to show \( (A-6I)\vec{x} = 6 \) has a non-trivial solution.
\begin{align*}
  A-6I &= \begin{bmatrix}
    1 & 1 & -2 \\
    -3 & -3 & 6 \\
    2 & 2 & -4
  \end{bmatrix} \sim \begin{bmatrix}
    1 & 1 & -2 \\
    0 & 0 & 0 \\
    0 & 0 & 0
  \end{bmatrix}
\end{align*}
\begin{align*}
  x_1+x_2-2x_3 &= 0 \\
  x_1 &= -x_2+2x_3 \\
  \vec{x} &= \begin{bmatrix}x_1 \\ x_2 \\ x_3\end{bmatrix}
    = \begin{bmatrix}-x_2+2x_3 \\ x_2 \\ x_3\end{bmatrix}
    = x_2\begin{bmatrix}-1 \\ 1 \\ 0\end{bmatrix}+
    x_2\begin{bmatrix}2 \\ 0 \\ 0\end{bmatrix} \\
  E_6 &= span\left(\begin{bmatrix}-1 \\ 1 \\ 0\end{bmatrix},
    \begin{bmatrix}2 \\ 0 \\ 0\end{bmatrix}\right)
\end{align*}

\subsubsection*{Example}
Find the eigenvectors and eigenvalues over \( \R \) and \( \mathbb{C} =
\{a+bi\mid a,b\in\R\}, i = \sqrt{-1} \) of the following:
\begin{enumerate}[(i)]
  \item \( A = \begin{bmatrix}1 & 0 \\ 0 & -1\end{bmatrix} \)
  \begin{align*}
    A-\lambda I &= \begin{bmatrix}
      1-\lambda & 0 \\
      0 & -1-\lambda
    \end{bmatrix} \\
    (1-\lambda)(-1-\lambda) &= 0 \\
    1-\lambda &= 0 \\
    -1-\lambda &= 0 \\
    \lambda &= \pm1 \\
  \end{align*}
  For \( \lambda = 1 \):
  \begin{align*}
    A-I &= \begin{bmatrix}0 & 0 \\ 0 & -1\end{bmatrix}
      \sim \begin{bmatrix}0 & 0 \\ 0 & 1\end{bmatrix} \\
    \vec{x} &= \begin{bmatrix}x_1 \\ x_2\end{bmatrix}
      = \begin{bmatrix}x_1 \\ 0\end{bmatrix}
      = x_1\begin{bmatrix}1 \\ 0\end{bmatrix} \\
    E_1 &= span\left(\begin{bmatrix}1 \\ 0\end{bmatrix}\right)
  \end{align*}
  For \( \lambda = -1 \):
  \begin{align*}
    A+I &= \begin{bmatrix}2 & 0 \\ 0 & 0\end{bmatrix}
      \sim \begin{bmatrix}1 & 0 \\ 0 & 0\end{bmatrix} \\
    \vec{x} &= \begin{bmatrix}x_1 \\ x_2\end{bmatrix}
      = \begin{bmatrix}0 \\ x_2\end{bmatrix}
      = x_2\begin{bmatrix}0 \\ 1\end{bmatrix} \\
    E_1 &= span\left(\begin{bmatrix}0 \\ 1\end{bmatrix}\right)
  \end{align*}
  \item \( B = \begin{bmatrix}3 & 1 \\ 1 & 3\end{bmatrix} \)
  \begin{align*}
    B-\lambda I &= \begin{bmatrix}
      3-\lambda & 1 \\
      1 & 3-\lambda
    \end{bmatrix} \\
    (3-\lambda)(3-\lambda)-1 &= 0 \\
    9-6\lambda+\lambda^2-1 &= 0 \\
    (\lambda-4)(\lambda-2) &= 0 \\
    \lambda = 4 &\quad \lambda = 2
  \end{align*}
  For \( \lambda = 2 \):
  \begin{align*}
    B-2I &= \begin{bmatrix}1 & 1 \\ 1 & 1\end{bmatrix}
      \sim \begin{bmatrix}1 & 1 \\ 0 & 0\end{bmatrix} \\
    x_1+x_2 &= 0 \\
    x_1 &= -x_2 \\
    \vec{x} &= \begin{bmatrix}x_1 \\ x_2\end{bmatrix}
      = \begin{bmatrix}-x_2 \\ x_2\end{bmatrix}
      = x_2\begin{bmatrix}-1 \\ 1\end{bmatrix} \\
    E_2 &= span\left(\begin{bmatrix}-1 \\ 1\end{bmatrix}\right)
  \end{align*}
  For \( \lambda = 4 \):
  \begin{align*}
    B-4I &= \begin{bmatrix}-1 & 1 \\ 1 & -1\end{bmatrix}
      \sim \begin{bmatrix}-1 & 1 \\ 0 & 0\end{bmatrix}
      \sim \begin{bmatrix}1 & -1 \\ 0 & 0\end{bmatrix} \\
    x_1-x_2 &= 0 \\
    x_1 &= x_2 \\
    \vec{x} &= \begin{bmatrix}x_1 \\ x_2\end{bmatrix}
      = \begin{bmatrix}x_1 \\ x_2\end{bmatrix}
      = x_1\begin{bmatrix}1 \\ 1\end{bmatrix} \\
    E_4 &= span\left(\begin{bmatrix}1 \\ 1\end{bmatrix}\right)
  \end{align*}
  \item \( C = \begin{bmatrix}0 & -1 \\ 1 & 0\end{bmatrix} \)
  \begin{align*}
    C-\lambda I &= \begin{bmatrix}
      -\lambda & -1 \\
      1 & -\lambda
    \end{bmatrix} \\
    (-\lambda)^2-(1)(-1) &= 0 \\
    \lambda^2+1 &= 0 \\
    \lambda^2 &= -1 \\
    \lambda &= \pm i
  \end{align*}
  Depending on our domain of interest, we have no real-valued eigenvalues and
  complex eigenvalues \( \pm i \).
\end{enumerate}

\begin{center}
  You can find all my notes at \url{http://omgimanerd.tech/notes}. If you have
  any questions, comments, or concerns, please contact me at
  alvin@omgimanerd.tech
\end{center}

\end{document}
