\documentclass{math}

\title{Linear Algebra}
\author{Alvin Lin}
\date{August 2017 - December 2017}

\begin{document}

\maketitle

\section*{Review 1}
Exam 1 Point Values:
\begin{enumerate}
  \item 8 points
  \item 12 points
  \item 8 points
  \item 8 points
  \item 10 points
  \item 10 points
  \item 12 points
  \item 12 points
  \item 4 points
  \item 16 points
\end{enumerate}
Total: 100 points \\ \\

\noindent Exam 1 Topics:
\begin{enumerate}
  \item A question about orthogonal vectors (refer to Section 1.2). If two
    vectors are orthogonal, their dot product is 0.
  \item Refer to Exercises 66 and 67 from Section 1.2.
  \item Prove a formula about vector length and dot products in \( \R^n \).
    Look at 61-65 from Section 1.2.
  \item Again, refer to exercises 61-65 in Section 1.2
  \item True/False questions. Be able to recognize linear equations.
  \item Given a certain system of linear equations, solve it. Example
    \begin{align*}
      x-y+z &= 0 \\
      -x+3y+z &= 5 \\
      3x+y+7z &= 2
    \end{align*}
    \begin{align*}
      A' &= \left[\begin{array}{ccc|c}
        1 & -1 & 1 & 0 \\
        -1 & 3 & 1 & 5 \\
        3 & 1 & 7 & 2
      \end{array}\right] \\
      R_1+R_2 &\to R_2 \\
      -3R_1+R_3 &\to R_3 \\
      A' &= \left[\begin{array}{ccc|c}
        1 & -1 & 1 & 0 \\
        0 & 2 & 2 & 5 \\
        0 & 4 & 4 & 2
      \end{array}\right] \\
      -2R_2+R_3 &\to R_3 \\
      A' &= \left[\begin{array}{ccc|c}
        1 & -1 & 1 & 0 \\
        0 & 2 & 2 & 5 \\
        0 & 0 & 0 & -8
      \end{array}\right] \\
      0 &\ne -8
    \end{align*}
    In this example, this system is inconsistent and has no solution.
  \item Let \( A \) be a 6x9 matrix and consider the homogeneous system
    \( A\vec{x} = \vec{0} \) where \( \vec{x} \) is in \( \R^n \).
    \begin{itemize}
      \item Is the system consistent? \\
        Yes this system is consistent because \( \vec{x} = \vec{0} \) solves it.
      \item What is the smallest number of free variables the system could have?
        \[ \text{free variables} = n-rank(A) = 9-rank(A) \ge 9-6 = 3 \]
        by the Rank Theorem.
      \item Could the solution set for this system be a point, a line, or a
        plane? \\
        No, because the dimension of these are not high enough.
    \end{itemize}
  \item Examine exercises 40-43 from Section 2.2.
  \item Given 2 vectors \( \vec{u} \) and \( \vec{v} \), compute the projection
    vector:
    \[ \text{proj}_{\vec{u}}\vec{v} =
      (\frac{\vec{u}\cdot\vec{v}}{\vec{u}\cdot\vec{u}}\vec{u}) \]
  \item Refer to question 39 from Section 2.2.
\end{enumerate}

\begin{center}
  You can find all my notes at \url{http://omgimanerd.tech/notes}. If you have
  any questions, comments, or concerns, please contact me at
  alvin@omgimanerd.tech
\end{center}

\end{document}
