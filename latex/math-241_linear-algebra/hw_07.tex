\documentclass[letterpaper, 12pt]{math}

\usepackage{enumerate}

\geometry{letterpaper, margin=0.5in}

\title{Linear Algebra: Homework 7}
\author{Alvin Lin}
\date{August 2016 - December 2016}

\begin{document}

\maketitle

\section*{Section 3.5}

\subsubsection*{Exercise 1}
Let \( S \) be the collection of vectors \( \begin{bmatrix}x \\ y\end{bmatrix}
\) in \( \R^2 \) that satisfy the given property. In each case, either prove
that \( S \) forms a subspace of \( \R^2 \) or give a counterexample to show
that it does not.
\[ x = 0 \]
\[ S = \begin{bmatrix}x \\ y\end{bmatrix} =
  \begin{bmatrix}0 \\ y\end{bmatrix} =
  y\begin{bmatrix}0 \\ 1\end{bmatrix} =
  span\left(\begin{bmatrix}0 \\ 1\end{bmatrix}\right) \]
Since \( \begin{bmatrix}0 \\ 1\end{bmatrix}\in\R^2 \), \( S \) is a subspace
of \( \R^2 \).

\subsubsection*{Exercise 3}
Let \( S \) be the collection of vectors \( \begin{bmatrix}x \\ y\end{bmatrix}
\) in \( \R^2 \) that satisfy the given property. In each case, either prove
that \( S \) forms a subspace of \( \R^2 \) or give a counterexample to show
that it does not.
\[ y = 2x \]
\[ S = \begin{bmatrix}x \\ y\end{bmatrix} =
  \begin{bmatrix}x \\ 2x\end{bmatrix} =
  x\begin{bmatrix}1 \\ 2\end{bmatrix} =
  span\left(\begin{bmatrix}1 \\ 2\end{bmatrix}\right) \]
Since \( \begin{bmatrix}1 \\ 2\end{bmatrix}\in\R^2 \), \( S \) is a subspace of
\( \R^2 \).

\subsubsection*{Exercise 5}
Let \( S \) be the collection of vectors \( \begin{bmatrix}x \\ y \\ z
\end{bmatrix} \) in \( \R^3 \) that satisfy the given property. In each case,
either prove that \( S \) forms a subspace of \( \R^2 \) or give a
counterexample to show that it does not.
\[ x = y = z \]
\[ S = \begin{bmatrix}x \\ y \\ z\end{bmatrix} =
  \begin{bmatrix}x \\ x \\ x\end{bmatrix} =
  x\begin{bmatrix}1 \\ 1 \\ 1\end{bmatrix} =
  span\left(\begin{bmatrix}1 \\ 1 \\ 1\end{bmatrix}\right) \]
Since \( \begin{bmatrix}1 \\ 1 \\ 1\end{bmatrix}\in\R^3 \), \( S \) is a
subspace of \( \R^3 \).

\subsubsection*{Exercise 7}
Let \( S \) be the collection of vectors \( \begin{bmatrix}x \\ y \\ z
\end{bmatrix} \) in \( \R^3 \) that satisfy the given property. In each case,
either prove that \( S \) forms a subspace of \( \R^2 \) or give a
counterexample to show that it does not.
\[ x-y+z = 1 \]
\[ 0-0+0 \ne 1 \]
Since the zero vector is not in \( S \), \( S \) is not a subspace of
\( \R^3 \).

\subsubsection*{Exercise 9}
Prove that every line through the origin in \( \R^3 \) is a subspace of
\( \R^3 \).
\[ \begin{bmatrix}x \\ y \\ z\end{bmatrix} =
  t\begin{bmatrix}a_1 \\ a_2 \\ a_3\end{bmatrix} =
  span\left(\begin{bmatrix}a_1 \\ a_2 \\ a_3\end{bmatrix}\right) \]
Every line through the origin can be described as the span of a vector
\( \begin{bmatrix}a_1 \\ a_2 \\ a_3\end{bmatrix}\in\R^3 \). Therefore, every
line through the origin is a subspace of \( \R^3 \).

\subsubsection*{Exercise 10}
Suppose \( S \) consists of all points in \( \R^2 \) that are on the x-axis or
the y-axis (or both). Is \( S \) a subspace of \( \R^2 \)? Why or why not?
\begin{align*}
  S &= \left\{\begin{bmatrix}x \\ 0\end{bmatrix}\mid x\in\R\right\}\cup
  \left\{\begin{bmatrix}0 \\ y\end{bmatrix}\mid y\in\R\right\} \\
  \begin{bmatrix}1 \\ 0\end{bmatrix} &\in S \\
  \begin{bmatrix}0 \\ 1\end{bmatrix} &\in S \\
  \begin{bmatrix}1 \\ 0\end{bmatrix}+\begin{bmatrix}0 \\ 1\end{bmatrix} &=
    \begin{bmatrix}1 \\ 1\end{bmatrix}\notin S
\end{align*}
The subspace \( S \) is not closed under addition.

\subsubsection*{Exercise 15}
If \( A \) is the matrix in Exercise 11, is \( \vec{v} = \begin{bmatrix}
-1 \\ 3 \\ -1\end{bmatrix} \) in \( null(A) \)?
\begin{align*}
  A &= \begin{bmatrix}
    1 & 0 & -1 \\
    1 & 1 & 1
  \end{bmatrix} \\
  A\vec{v} &= \begin{bmatrix}
    1 & 0 & -1 \\
    1 & 1 & 1
  \end{bmatrix}\begin{bmatrix}-1 & 3 & -1\end{bmatrix} \\
  &= \begin{bmatrix}0 \\ 1\end{bmatrix} \\
  \therefore \vec{v} &\notin null(A)
\end{align*}

\subsubsection*{Exercise 16}
If \( A \) is the matrix in Exercise 12, is \( \vec{v} = \begin{bmatrix}
7 \\ -1 \\ 2\end{bmatrix} \) in \( null(A) \)?
\begin{align*}
  A &= \begin{bmatrix}
    1 & 1 & -3 \\
    0 & 2 & 1 \\
    1 & -1 & -4
  \end{bmatrix} \\
  A\vec{v} &= \begin{bmatrix}
    1 & 1 & -3 \\
    0 & 2 & 1 \\
    1 & -1 & -4
  \end{bmatrix}\begin{bmatrix}7 \\ -1 \\ 2\end{bmatrix} \\
  &= \begin{bmatrix}0 \\ 0 \\ 0\end{bmatrix} \\
  \therefore \vec{v} &\in null(A)
\end{align*}

\subsubsection*{Exercise 17}
Give bases for \( row(A), col(A), \text{ and } null(A) \).
\begin{align*}
  A &= \begin{bmatrix}1 & 0 & -1 \\ 1 & 1 & 1\end{bmatrix} \\
  &= \begin{bmatrix}1 & 0 & -1 \\ 0 & 1 & 2\end{bmatrix} \\
  \text{basis for } row(A) &= \left\{\begin{bmatrix}
    1 & 0 & -1
  \end{bmatrix},\begin{bmatrix}
    0 & 1 & 2
  \end{bmatrix}\right\} \\
  \text{basis for } col(A) &= \left\{\begin{bmatrix}
    1 \\ 1
  \end{bmatrix},\begin{bmatrix}
    0 \\ 1
  \end{bmatrix}\right\} \\
  A\begin{bmatrix}x_1 \\ x_2 \\ x_3\end{bmatrix} &= \vec{0} \\
  1x_1-1x_3 &= 0 \\
  1x_2+2x_3 &= 0 \\
  null(A) &= \begin{bmatrix}x_1 \\ -2x_1 \\ x_1\end{bmatrix} \\
  &= x_1\begin{bmatrix}1 \\ -2 \\ 1\end{bmatrix} \\
  \text{basis for } null(A) &= \left\{\begin{bmatrix}
    1 \\ -2 \\ 1
  \end{bmatrix}\right\}
\end{align*}

\subsubsection*{Exercise 19}
Give bases for \( row(A), col(A), \text{ and } null(A) \).
\begin{align*}
  A &= \begin{bmatrix}
    1 & 1 & 0 & 1 \\
    0 & 1 & -1 & 1 \\
    0 & 1 & -1 & -1
  \end{bmatrix} \\
  &= \begin{bmatrix}
    1 & 1 & 0 & 1 \\
    0 & 1 & -1 & 1 \\
    0 & 0 & 0 & -2
  \end{bmatrix} \\
  &= \begin{bmatrix}
    1 & 1 & 0 & 0 \\
    0 & 1 & -1 & 0 \\
    0 & 0 & 0 & 1
  \end{bmatrix} \\
  &= \begin{bmatrix}
    1 & 0 & 1 & 0 \\
    0 & 1 & -1 & 0 \\
    0 & 0 & 0 & 1
  \end{bmatrix} \\
  \text{basis for } row(A) &= \left\{\begin{bmatrix}
    1 & 0 & 1 & 0
  \end{bmatrix},\begin{bmatrix}
    0 & 1 & -1 & 0
  \end{bmatrix},\begin{bmatrix}
    0 & 0 & 0 & 1
  \end{bmatrix}\right\} \\
  \text{basis for } col(A) &= \left\{\begin{bmatrix}
    1 \\ 0 \\ 0
  \end{bmatrix},\begin{bmatrix}
    1 \\ 1 \\ 1
  \end{bmatrix},\begin{bmatrix}
    1 \\ 1 \\ -1
  \end{bmatrix}\right\} \\
  A\vec{x} &= 0 \\
  x_1+x_3 &= 0 \\
  x_2-x_3 &= 0 \\
  x_4 &= 0 \\
  null(A) &= \begin{bmatrix}
    x_1 \\ -x_1 \\ -x_1 \\ 0
  \end{bmatrix} \\
  \text{basis for } null(A) &= \left\{
    \begin{bmatrix}1 \\ -1 \\ -1 \\ 0\end{bmatrix}\right\}
\end{align*}

\subsubsection*{Exercise 27}
Find a basis for the span of the given vectors.
\[ \begin{bmatrix}1 \\ -1 \\ 0\end{bmatrix}\quad
  \begin{bmatrix}-1 \\ 0 \\ 1\end{bmatrix}\quad
  \begin{bmatrix}0 \\ 1 \\ -1\end{bmatrix} \]
\begin{align*}
  \begin{bmatrix}
    1 & -1 & 0 \\
    -1 & 0 & 1 \\
    0 & 1 & -1
  \end{bmatrix} &= \begin{bmatrix}
    1 & -1 & 0 \\
    0 & -1 & 1 \\
    0 & 1 & -1
  \end{bmatrix} \\
  &= \begin{bmatrix}
    1 & -1 & 0 \\
    0 & 1 & -1 \\
    0 & 0 & 0
  \end{bmatrix} \\
  \text{basis} &= \left\{\begin{bmatrix}
    1 \\ -1 \\ 0
  \end{bmatrix},\begin{bmatrix}
    -1 \\ 0 \\ 1
  \end{bmatrix}\right\}
\end{align*}

\subsubsection*{Exercise 29}
Find a basis for the span of the given vectors.
\[ \begin{bmatrix}2 & -3 & -1\end{bmatrix}\quad
  \begin{bmatrix}1 & -1 & 0\end{bmatrix}\quad
  \begin{bmatrix}4 & -4 & 1\end{bmatrix} \]
\begin{align*}
  \begin{bmatrix}
    2 & -3 & -1 \\
    1 & -1 & 0 \\
    4 & -4 & 1
  \end{bmatrix} &= \begin{bmatrix}
    0 & -1 & -1 \\
    1 & -1 & 0 \\
    0 & 0 & 1
  \end{bmatrix} \\
  &= \begin{bmatrix}
    1 & 0 & 0 \\
    0 & 1 & 0 \\
    0 & 0 & 1
  \end{bmatrix} \\
  \text{basis} &= \{e_1,e_2,e_3\}
\end{align*}

\subsubsection*{Exercise 35}
Give the rank and nullity of the matrix in Exercise 17.
\begin{align*}
  A &= \begin{bmatrix}1 & 0 & -1 \\ 1 & 1 & 1\end{bmatrix} \\
  &= \begin{bmatrix}1 & 0 & -1 \\ 0 & 1 & 2\end{bmatrix} \\
  rank(A) &= 2 \\
  nullity(A) &= 3-rank(A) = 1
\end{align*}

\subsubsection*{Exercise 37}
Give the rank and nullity of the matrix in Exercise 19.
\begin{align*}
  A &= \begin{bmatrix}
    1 & 1 & 0 & 1 \\
    0 & 1 & -1 & 1 \\
    0 & 1 & -1 & -1
  \end{bmatrix} \\
  &= \begin{bmatrix}
    1 & 0 & 1 & 0 \\
    0 & 1 & -1 & 0 \\
    0 & 0 & 0 & 1
  \end{bmatrix} \\
  rank(A) &= 3 \\
  nullity(A) &= 4-rank(A) = 1
\end{align*}

\subsubsection*{Exercise 39}
If \( A \) is a \( 3\times5 \) matrix, explain why the columns of \( A \) must
be linearly dependent.
\begin{align*}
  rank(A) &\le min(m,n) \\
  A\vec{x} &= \vec{0} \\
  5 &= rank(A)+nullity(A) \\
  rank(A) &\le 3 \\
\end{align*}
The nullity of \( A \) must be at least 2, therefore \( A\vec{x} = \vec{0} \)
has a non-trivial solution. Thus, the columns of \( A \) are linearly
dependent. Since there are 5 columns and only 3 can be linearly independent,
they must be linearly dependent.

\subsubsection*{Exercise 40}
If \( A \) is a \( 4\times2 \) matrix, explain why the rows of \( A \) must
be linearly dependent. \\
There are four rows in total, but only two rows can be linearly independent in
a \( 4\times2 \) matrix, therefore the rows must be linearly dependent.

\subsubsection*{Exercise 41}
If \( A \) is a \( 3\times5 \) matrix, what are the possible values of
\( nullity(A) \)?
\begin{align*}
  rank(A) &\le 3 \\
  n &= rank(A)+nullity(A) \\
  5 &= rank(A)+nullity(A) \\
  5-nullity(A) &\le 3 \\
  -nullity(A) &\le -2 \\
  nullity(A) &\ge 2 \\
  2 &\le nullity(A) \le 5
\end{align*}

\subsubsection*{Exercise 42}
If \( A \) is a \( 4\times2 \) matrix, what are the possible values of
\( nullity(A) \)?
\begin{align*}
  rank(A) &\le 2 \\
  n &= rank(A)+nullity(A) \\
  2 &= rank(A)+nullity(A) \\
  2-nullity(A) &\le 2 \\
  -nullity(A) &\le 0 \\
  nullity(A) &\ge 0 \\
  0 &\le nullity(A) \le 2
\end{align*}

\subsubsection*{Exercise 43}
Find all possible values of \( rank(A) \) as \( a \) varies.
\begin{align*}
  A &= \begin{bmatrix}
    1 & 2 & a \\
    -2 & 4a & 2 \\
    a & -2 & 1
  \end{bmatrix} \\
\end{align*}


\subsubsection*{Exercise 44}
\subsubsection*{Exercise 45}
\subsubsection*{Exercise 47}
\subsubsection*{Exercise 51}
\subsubsection*{Exercise 52}
\subsubsection*{Exercise 57}
\subsubsection*{Exercise 58}
\subsubsection*{Exercise 59a}
\subsubsection*{Exercise 60a}
\subsubsection*{Exercise 61}
\subsubsection*{Exercise 62}
\subsubsection*{Exercise 63}
\subsubsection*{Exercise 64}
\subsubsection*{Exercise 65}
\subsubsection*{Exercise 66}

\section*{Section 3.6}
\subsubsection*{Exercise 1}
\subsubsection*{Exercise 3}
\subsubsection*{Exercise 5}
\subsubsection*{Exercise 7}
\subsubsection*{Exercise 9}
\subsubsection*{Exercise 11}
\subsubsection*{Exercise 13}
\subsubsection*{Exercise 40}
\subsubsection*{Exercise 42}
\subsubsection*{Exercise 44}
\subsubsection*{Exercise 52}
\subsubsection*{Exercise 53}
\subsubsection*{Exercise 54}

\begin{center}
  If you have any questions, comments, or concerns, please contact me at
  alvin@omgimanerd.tech
\end{center}

\end{document}
