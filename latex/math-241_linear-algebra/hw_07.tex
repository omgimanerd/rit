\documentclass{math}

\usepackage{enumerate}

\geometry{letterpaper, margin=0.5in}

\title{Linear Algebra: Homework 7}
\author{Alvin Lin}
\date{August 2016 - December 2016}

\begin{document}

\maketitle

\section*{Section 3.5}

\subsubsection*{Exercise 1}
Let \( S \) be the collection of vectors \( \begin{bmatrix}x \\ y\end{bmatrix}
\) in \( \R^2 \) that satisfy the given property. In each case, either prove
that \( S \) forms a subspace of \( \R^2 \) or give a counterexample to show
that it does not.
\[ x = 0 \]
\[ S = \begin{bmatrix}x \\ y\end{bmatrix} =
  \begin{bmatrix}0 \\ y\end{bmatrix} =
  y\begin{bmatrix}0 \\ 1\end{bmatrix} =
  span\left(\begin{bmatrix}0 \\ 1\end{bmatrix}\right) \]
Since \( \begin{bmatrix}0 \\ 1\end{bmatrix}\in\R^2 \), \( S \) is a subspace
of \( \R^2 \).

\subsubsection*{Exercise 3}
Let \( S \) be the collection of vectors \( \begin{bmatrix}x \\ y\end{bmatrix}
\) in \( \R^2 \) that satisfy the given property. In each case, either prove
that \( S \) forms a subspace of \( \R^2 \) or give a counterexample to show
that it does not.
\[ y = 2x \]
\[ S = \begin{bmatrix}x \\ y\end{bmatrix} =
  \begin{bmatrix}x \\ 2x\end{bmatrix} =
  x\begin{bmatrix}1 \\ 2\end{bmatrix} =
  span\left(\begin{bmatrix}1 \\ 2\end{bmatrix}\right) \]
Since \( \begin{bmatrix}1 \\ 2\end{bmatrix}\in\R^2 \), \( S \) is a subspace of
\( \R^2 \).

\subsubsection*{Exercise 5}
Let \( S \) be the collection of vectors \( \begin{bmatrix}x \\ y \\ z
\end{bmatrix} \) in \( \R^3 \) that satisfy the given property. In each case,
either prove that \( S \) forms a subspace of \( \R^2 \) or give a
counterexample to show that it does not.
\[ x = y = z \]
\[ S = \begin{bmatrix}x \\ y \\ z\end{bmatrix} =
  \begin{bmatrix}x \\ x \\ x\end{bmatrix} =
  x\begin{bmatrix}1 \\ 1 \\ 1\end{bmatrix} =
  span\left(\begin{bmatrix}1 \\ 1 \\ 1\end{bmatrix}\right) \]
Since \( \begin{bmatrix}1 \\ 1 \\ 1\end{bmatrix}\in\R^3 \), \( S \) is a
subspace of \( \R^3 \).

\subsubsection*{Exercise 7}
Let \( S \) be the collection of vectors \( \begin{bmatrix}x \\ y \\ z
\end{bmatrix} \) in \( \R^3 \) that satisfy the given property. In each case,
either prove that \( S \) forms a subspace of \( \R^2 \) or give a
counterexample to show that it does not.
\[ x-y+z = 1 \]
\[ 0-0+0 \ne 1 \]
Since the zero vector is not in \( S \), \( S \) is not a subspace of
\( \R^3 \).

\subsubsection*{Exercise 9}
Prove that every line through the origin in \( \R^3 \) is a subspace of
\( \R^3 \).
\[ \begin{bmatrix}x \\ y \\ z\end{bmatrix} =
  t\begin{bmatrix}a_1 \\ a_2 \\ a_3\end{bmatrix} =
  span\left(\begin{bmatrix}a_1 \\ a_2 \\ a_3\end{bmatrix}\right) \]
Every line through the origin can be described as the span of a vector
\( \begin{bmatrix}a_1 \\ a_2 \\ a_3\end{bmatrix}\in\R^3 \). Therefore, every
line through the origin is a subspace of \( \R^3 \).

\subsubsection*{Exercise 10}
Suppose \( S \) consists of all points in \( \R^2 \) that are on the x-axis or
the y-axis (or both). Is \( S \) a subspace of \( \R^2 \)? Why or why not?
\begin{align*}
  S &= \left\{\begin{bmatrix}x \\ 0\end{bmatrix}\mid x\in\R\right\}\cup
  \left\{\begin{bmatrix}0 \\ y\end{bmatrix}\mid y\in\R\right\} \\
  \begin{bmatrix}1 \\ 0\end{bmatrix} &\in S \\
  \begin{bmatrix}0 \\ 1\end{bmatrix} &\in S \\
  \begin{bmatrix}1 \\ 0\end{bmatrix}+\begin{bmatrix}0 \\ 1\end{bmatrix} &=
    \begin{bmatrix}1 \\ 1\end{bmatrix}\notin S
\end{align*}
The subspace \( S \) is not closed under addition.

\subsubsection*{Exercise 15}
If \( A \) is the matrix in Exercise 11, is \( \vec{v} = \begin{bmatrix}
-1 \\ 3 \\ -1\end{bmatrix} \) in \( null(A) \)?
\begin{align*}
  A &= \begin{bmatrix}
    1 & 0 & -1 \\
    1 & 1 & 1
  \end{bmatrix} \\
  A\vec{v} &= \begin{bmatrix}
    1 & 0 & -1 \\
    1 & 1 & 1
  \end{bmatrix}\begin{bmatrix}-1 & 3 & -1\end{bmatrix} \\
  &= \begin{bmatrix}0 \\ 1\end{bmatrix} \\
  \therefore \vec{v} &\notin null(A)
\end{align*}

\subsubsection*{Exercise 16}
If \( A \) is the matrix in Exercise 12, is \( \vec{v} = \begin{bmatrix}
7 \\ -1 \\ 2\end{bmatrix} \) in \( null(A) \)?
\begin{align*}
  A &= \begin{bmatrix}
    1 & 1 & -3 \\
    0 & 2 & 1 \\
    1 & -1 & -4
  \end{bmatrix} \\
  A\vec{v} &= \begin{bmatrix}
    1 & 1 & -3 \\
    0 & 2 & 1 \\
    1 & -1 & -4
  \end{bmatrix}\begin{bmatrix}7 \\ -1 \\ 2\end{bmatrix} \\
  &= \begin{bmatrix}0 \\ 0 \\ 0\end{bmatrix} \\
  \therefore \vec{v} &\in null(A)
\end{align*}

\subsubsection*{Exercise 17}
Give bases for \( row(A), col(A), \text{ and } null(A) \).
\begin{align*}
  A &= \begin{bmatrix}1 & 0 & -1 \\ 1 & 1 & 1\end{bmatrix} \\
  &= \begin{bmatrix}1 & 0 & -1 \\ 0 & 1 & 2\end{bmatrix} \\
  \text{basis for } row(A) &= \left\{\begin{bmatrix}
    1 & 0 & -1
  \end{bmatrix},\begin{bmatrix}
    0 & 1 & 2
  \end{bmatrix}\right\} \\
  \text{basis for } col(A) &= \left\{\begin{bmatrix}
    1 \\ 1
  \end{bmatrix},\begin{bmatrix}
    0 \\ 1
  \end{bmatrix}\right\} \\
  A\begin{bmatrix}x_1 \\ x_2 \\ x_3\end{bmatrix} &= \vec{0} \\
  1x_1-1x_3 &= 0 \\
  1x_2+2x_3 &= 0 \\
  null(A) &= \begin{bmatrix}x_1 \\ -2x_1 \\ x_1\end{bmatrix} \\
  &= x_1\begin{bmatrix}1 \\ -2 \\ 1\end{bmatrix} \\
  \text{basis for } null(A) &= \left\{\begin{bmatrix}
    1 \\ -2 \\ 1
  \end{bmatrix}\right\}
\end{align*}

\subsubsection*{Exercise 19}
Give bases for \( row(A), col(A), \text{ and } null(A) \).
\begin{align*}
  A &= \begin{bmatrix}
    1 & 1 & 0 & 1 \\
    0 & 1 & -1 & 1 \\
    0 & 1 & -1 & -1
  \end{bmatrix} \\
  &= \begin{bmatrix}
    1 & 1 & 0 & 1 \\
    0 & 1 & -1 & 1 \\
    0 & 0 & 0 & -2
  \end{bmatrix} \\
  &= \begin{bmatrix}
    1 & 1 & 0 & 0 \\
    0 & 1 & -1 & 0 \\
    0 & 0 & 0 & 1
  \end{bmatrix} \\
  &= \begin{bmatrix}
    1 & 0 & 1 & 0 \\
    0 & 1 & -1 & 0 \\
    0 & 0 & 0 & 1
  \end{bmatrix} \\
  \text{basis for } row(A) &= \left\{\begin{bmatrix}
    1 & 0 & 1 & 0
  \end{bmatrix},\begin{bmatrix}
    0 & 1 & -1 & 0
  \end{bmatrix},\begin{bmatrix}
    0 & 0 & 0 & 1
  \end{bmatrix}\right\} \\
  \text{basis for } col(A) &= \left\{\begin{bmatrix}
    1 \\ 0 \\ 0
  \end{bmatrix},\begin{bmatrix}
    1 \\ 1 \\ 1
  \end{bmatrix},\begin{bmatrix}
    1 \\ 1 \\ -1
  \end{bmatrix}\right\} \\
  A\vec{x} &= 0 \\
  x_1+x_3 &= 0 \\
  x_2-x_3 &= 0 \\
  x_4 &= 0 \\
  null(A) &= \begin{bmatrix}
    x_1 \\ -x_1 \\ -x_1 \\ 0
  \end{bmatrix} \\
  \text{basis for } null(A) &= \left\{
    \begin{bmatrix}1 \\ -1 \\ -1 \\ 0\end{bmatrix}\right\}
\end{align*}

\subsubsection*{Exercise 27}
Find a basis for the span of the given vectors.
\[ \begin{bmatrix}1 \\ -1 \\ 0\end{bmatrix}\quad
  \begin{bmatrix}-1 \\ 0 \\ 1\end{bmatrix}\quad
  \begin{bmatrix}0 \\ 1 \\ -1\end{bmatrix} \]
\begin{align*}
  \begin{bmatrix}
    1 & -1 & 0 \\
    -1 & 0 & 1 \\
    0 & 1 & -1
  \end{bmatrix} &= \begin{bmatrix}
    1 & -1 & 0 \\
    0 & -1 & 1 \\
    0 & 1 & -1
  \end{bmatrix} \\
  &= \begin{bmatrix}
    1 & -1 & 0 \\
    0 & 1 & -1 \\
    0 & 0 & 0
  \end{bmatrix} \\
  \text{basis} &= \left\{\begin{bmatrix}
    1 \\ -1 \\ 0
  \end{bmatrix},\begin{bmatrix}
    -1 \\ 0 \\ 1
  \end{bmatrix}\right\}
\end{align*}

\subsubsection*{Exercise 29}
Find a basis for the span of the given vectors.
\[ \begin{bmatrix}2 & -3 & -1\end{bmatrix}\quad
  \begin{bmatrix}1 & -1 & 0\end{bmatrix}\quad
  \begin{bmatrix}4 & -4 & 1\end{bmatrix} \]
\begin{align*}
  \begin{bmatrix}
    2 & -3 & -1 \\
    1 & -1 & 0 \\
    4 & -4 & 1
  \end{bmatrix} &= \begin{bmatrix}
    0 & -1 & -1 \\
    1 & -1 & 0 \\
    0 & 0 & 1
  \end{bmatrix} \\
  &= \begin{bmatrix}
    1 & 0 & 0 \\
    0 & 1 & 0 \\
    0 & 0 & 1
  \end{bmatrix} \\
  \text{basis} &= \{e_1,e_2,e_3\}
\end{align*}

\subsubsection*{Exercise 35}
Give the rank and nullity of the matrix in Exercise 17.
\begin{align*}
  A &= \begin{bmatrix}1 & 0 & -1 \\ 1 & 1 & 1\end{bmatrix} \\
  &= \begin{bmatrix}1 & 0 & -1 \\ 0 & 1 & 2\end{bmatrix} \\
  rank(A) &= 2 \\
  nullity(A) &= 3-rank(A) = 1
\end{align*}

\subsubsection*{Exercise 37}
Give the rank and nullity of the matrix in Exercise 19.
\begin{align*}
  A &= \begin{bmatrix}
    1 & 1 & 0 & 1 \\
    0 & 1 & -1 & 1 \\
    0 & 1 & -1 & -1
  \end{bmatrix} \\
  &= \begin{bmatrix}
    1 & 0 & 1 & 0 \\
    0 & 1 & -1 & 0 \\
    0 & 0 & 0 & 1
  \end{bmatrix} \\
  rank(A) &= 3 \\
  nullity(A) &= 4-rank(A) = 1
\end{align*}

\subsubsection*{Exercise 39}
If \( A \) is a \( 3\times5 \) matrix, explain why the columns of \( A \) must
be linearly dependent.
\begin{align*}
  rank(A) &\le min(m,n) \\
  A\vec{x} &= \vec{0} \\
  5 &= rank(A)+nullity(A) \\
  rank(A) &\le 3 \\
\end{align*}
The nullity of \( A \) must be at least 2, therefore \( A\vec{x} = \vec{0} \)
has a non-trivial solution. Thus, the columns of \( A \) are linearly
dependent. Since there are 5 columns and only 3 can be linearly independent,
they must be linearly dependent.

\subsubsection*{Exercise 40}
If \( A \) is a \( 4\times2 \) matrix, explain why the rows of \( A \) must
be linearly dependent. \\
There are four rows in total, but only two rows can be linearly independent in
a \( 4\times2 \) matrix, therefore the rows must be linearly dependent.

\subsubsection*{Exercise 41}
If \( A \) is a \( 3\times5 \) matrix, what are the possible values of
\( nullity(A) \)?
\begin{align*}
  rank(A) &\le 3 \\
  n &= rank(A)+nullity(A) \\
  5 &= rank(A)+nullity(A) \\
  5-nullity(A) &\le 3 \\
  -nullity(A) &\le -2 \\
  nullity(A) &\ge 2 \\
  2 &\le nullity(A) \le 5
\end{align*}

\subsubsection*{Exercise 42}
If \( A \) is a \( 4\times2 \) matrix, what are the possible values of
\( nullity(A) \)?
\begin{align*}
  rank(A) &\le 2 \\
  n &= rank(A)+nullity(A) \\
  2 &= rank(A)+nullity(A) \\
  2-nullity(A) &\le 2 \\
  -nullity(A) &\le 0 \\
  nullity(A) &\ge 0 \\
  0 &\le nullity(A) \le 2
\end{align*}

\subsubsection*{Exercise 43}
Find all possible values of \( rank(A) \) as \( a \) varies.
\begin{align*}
  A &= \begin{bmatrix}
    1 & 2 & a \\
    -2 & 4a & 2 \\
    a & -2 & 1
  \end{bmatrix} \\
  &= \begin{bmatrix}
    1 & 2 & a \\
    0 & 4a+4 & 2+2a \\
    1 & -\frac{2}{a} & \frac{1}{a}
  \end{bmatrix} \\
  &= \begin{bmatrix}
    1 & 2 & a \\
    0 & 4a+4 & 2+2a \\
    0 & -\frac{2}{a}-2 & \frac{1}{a}-a
  \end{bmatrix} \\
  &= \begin{bmatrix}
    1 & 2 & a \\
    0 & 4a+4 & 2+2a \\
    0 & -2-2a & 1-a^2
  \end{bmatrix} \\
  &= \begin{bmatrix}
    1 & 2 & a \\
    0 & 2a+2 & 1+a \\
    0 & 2a+2 & a^2-1
  \end{bmatrix} \\
  &= \begin{bmatrix}
    1 & 2 & a \\
    0 & 2a+2 & 1+a \\
    0 & 0 & a^2-a-2
  \end{bmatrix} \\
  2a+2 &= 0 \\
  a &= -1 \\
  &\to \begin{bmatrix}
    1 & 2 & -1 \\
    0 & 0 & 0 \\
    0 & 0 & 2
  \end{bmatrix} \\
  a^2-a-2 &= 0 \\
  (a-2)(a+1) &= 0 \\
  a &= 1, -2 \\
  &\to \begin{bmatrix}
    1 & 2 & 1 \\
    0 & 4 & 2 \\
    0 & 0 & 0
  \end{bmatrix} \\
  &\to \begin{bmatrix}
    1 & 2 & a \\
    0 & -2 & -1 \\
    0 & 0 & 0
  \end{bmatrix} \\
  rank(A) &= \begin{cases}
    1 & \text{if } a = 1 \\
    2 & \text{if } a = -1 \vee a = 2 \\
    3 & \text{otherwise}
  \end{cases}
\end{align*}

\subsubsection*{Exercise 44}
Find all possible values of \( rank(A) \) as \( a \) varies.
\begin{align*}
  A &= \begin{bmatrix}
    a & 2 & -1 \\
    3 & 3 & -2 \\
    -2 & -1 & a
  \end{bmatrix} \\
  &= \begin{bmatrix}
    1 & \frac{2}{a} & -\frac{1}{a} \\
    1 & 2 & -2+a \\
    -2 & -1 & a
  \end{bmatrix} \\
  &= \begin{bmatrix}
    1 & \frac{2}{a} & -\frac{1}{a} \\
    1 & 2 & -2+a \\
    0 & 3 & a+2(-2+a)
  \end{bmatrix} \\
  &= \begin{bmatrix}
    1 & \frac{2}{a} & -\frac{1}{a} \\
    0 & 2-\frac{2}{a} & -2+a+\frac{1}{a} \\
    0 & 3 & 3a-4
  \end{bmatrix} \\
  &= \begin{bmatrix}
    a & 2 & -1 \\
    0 & 2a-2 & -2a+a^2+1 \\
    0 & 3 & 3a-4
  \end{bmatrix} \\
  a &= 1 \\
  &\to \begin{bmatrix}
    1 & 2 & -1 \\
    0 & 0 & 0 \\
    0 & 3 & -1
  \end{bmatrix} \\
  rank(A) &= \begin{cases}
    2 & \text{if } a = 1 \\
    3 & \text{otherwise}
  \end{cases}
\end{align*}

\subsubsection*{Exercise 45}
Do \( \begin{bmatrix}1 \\ 1 \\ 0\end{bmatrix},\begin{bmatrix}1 \\ 0 \\ 1
\end{bmatrix},\begin{bmatrix}0 \\ 1 \\ 1\end{bmatrix} \) form a basis for
\( \R^3 \)?
\begin{align*}
  \begin{bmatrix}
    1 & 1 & 0 \\
    1 & 0 & 1 \\
    0 & 1 & 1
  \end{bmatrix} &= \begin{bmatrix}
    1 & 0 & -1 \\
    1 & 0 & 1 \\
    0 & 1 & 1
  \end{bmatrix} \\
  &= \begin{bmatrix}
    1 & 0 & -1 \\
    0 & 0 & 2 \\
    0 & 1 & 1
  \end{bmatrix} \\
  &= \begin{bmatrix}
    1 & 0 & 0 \\
    0 & 0 & 1 \\
    0 & 1 & 0
  \end{bmatrix} \\
  &= \begin{bmatrix}
    1 & 0 & 0 \\
    0 & 1 & 0 \\
    0 & 0 & 1
  \end{bmatrix}
\end{align*}
Since the vectors are linearly independent and form the standard basis vectors,
they form a basis for \( \R^3 \).

\subsubsection*{Exercise 47}
Do \( \begin{bmatrix}1 \\ 1 \\ 1 \\ 0\end{bmatrix},\begin{bmatrix}1 \\ 1 \\ 0 \\
1\end{bmatrix},\begin{bmatrix}1 \\ 0 \\ 1 \\ 1\end{bmatrix},\begin{bmatrix}0 \\
1 \\ 1 \\ 1\end{bmatrix} \) for a basis for \( \R^4 \)?
\begin{align*}
  \begin{bmatrix}
    1 & 1 & 1 & 0 \\
    1 & 1 & 0 & 1 \\
    1 & 0 & 1 & 1 \\
    0 & 1 & 1 & 1
  \end{bmatrix} &= \begin{bmatrix}
    1 & 1 & 1 & 0 \\
    0 & 0 & -1 & 1 \\
    0 & -1 & 0 & 1 \\
    0 & 1 & 1 & 1
  \end{bmatrix} \\
  &= \begin{bmatrix}
    1 & 1 & 1 & 0 \\
    0 & 0 & -1 & 1 \\
    0 & 0 & 1 & 2 \\
    0 & 1 & 1 & 1
  \end{bmatrix} \\
  &= \begin{bmatrix}
    1 & 1 & 1 & 0 \\
    0 & 0 & 0 & 3 \\
    0 & 0 & 1 & 2 \\
    0 & 1 & 1 & 1
  \end{bmatrix} \\
  &= \begin{bmatrix}
    1 & 0 & 0 & 0 \\
    0 & 0 & 0 & 1 \\
    0 & 0 & 1 & 0 \\
    0 & 1 & 0 & 0
  \end{bmatrix} \\
  &= \begin{bmatrix}
    1 & 0 & 0 & 0 \\
    0 & 1 & 0 & 0 \\
    0 & 0 & 1 & 0 \\
    0 & 0 & 0 & 1
  \end{bmatrix}
\end{align*}
Since the vectors form the standard basis vectors, they form a basis for
\( \R^4 \).

\subsubsection*{Exercise 51}
Show that \( w \) is in \( span(B) \) and find the coordinate vector
\( [w]_B \).
\[ B = \left\{\begin{bmatrix}1 \\ 2 \\ 0\end{bmatrix},
  \begin{bmatrix}1 \\ 0 \\ -1\end{bmatrix}\right\} \quad
  \vec{w} = \begin{bmatrix}1 \\ 6 \\ 2\end{bmatrix} \]
\begin{align*}
  c_1\begin{bmatrix}1 \\ 2 \\ 0\end{bmatrix}+
    c_2\begin{bmatrix}1 \\ 0 \\ -1\end{bmatrix} &=
    \begin{bmatrix}1 \\ 6 \\ 2\end{bmatrix} \\
  \begin{bmatrix}
    1 & 1 & 1 \\
    2 & 0 & 6 \\
    0 & -1 & 2
  \end{bmatrix} &= \begin{bmatrix}
    1 & 1 & 1 \\
    0 & -2 & 4 \\
    0 & -1 & 2
  \end{bmatrix} \\
  &= \begin{bmatrix}
    1 & 0 & 3 \\
    0 & 1 & -2 \\
    0 & 0 & 0
  \end{bmatrix} \\
  [w]_B = \begin{bmatrix}3 \\ -2\end{bmatrix}
\end{align*}

\subsubsection*{Exercise 52}
Show that \( w \) is in \( span(B) \) and find the coordinate vector
\( [w]_B \).
\[ B = \left\{\begin{bmatrix}3 \\ 1 \\ 4\end{bmatrix},
  \begin{bmatrix}5 \\ 1 \\ 6\end{bmatrix}\right\} \quad
  \vec{w} = \begin{bmatrix}1 \\ 3 \\ 4\end{bmatrix} \]
\begin{align*}
  c_1\begin{bmatrix}3 \\ 1 \\ 4\end{bmatrix}+
    c_2\begin{bmatrix}5 \\ 1 \\ 6\end{bmatrix} &=
    \begin{bmatrix}1 \\ 3 \\ 4\end{bmatrix} \\
  \begin{bmatrix}
    3 & 5 & 1 \\
    1 & 1 & 3 \\
    4 & 6 & 4
  \end{bmatrix} &= \begin{bmatrix}
    0 & 2 & -8 \\
    1 & 1 & 3 \\
    2 & 3 & 2
  \end{bmatrix} \\
  &= \begin{bmatrix}
    0 & 1 & -4 \\
    1 & 1 & 3 \\
    0 & 1 & -4
  \end{bmatrix} \\
  &= \begin{bmatrix}
    1 & 0 & 7 \\
    0 & 1 & -4 \\
    0 & 0 & 0
  \end{bmatrix} \\
  [w]_B &= \begin{bmatrix}7 \\ -4\end{bmatrix}
\end{align*}

\subsubsection*{Exercise 57}
If \( A \) is \( m\times n \), prove that every vector in \( null(A) \) is
orthogonal to every vector in \( row(A) \).
\begin{align*}
  A\vec{x} &= \vec{0} \quad \forall{\vec{x}}\in\R^m \\
  \vec{y} &= \sum_{i=1}^{m}c_ia_i \quad \forall{\vec{u}}\in row(A) \\
  &= \begin{bmatrix}col_1(A) & \dots & col_m(A)\end{bmatrix}
    \begin{bmatrix}c_1 \\ \vdots \\ c_m\end{bmatrix} \\
  &= A^T\begin{bmatrix}c_1 \\ \vdots \\ c_m\end{bmatrix} \\
  x^Ty &= x^TA^Tc \\
  x^Ty &= (Ax)^Tc \\
  x\cdot y
   &= 0c = 0
\end{align*}

\subsubsection*{Exercise 58}
If \( A \) and \( B \) are \( n\times n \) of rank \( n \), prove that \( AB \)
has rank \( n \).
\begin{align*}
  nullity(A) &= 0 \\
  nullity(B) &= 0 \\
  B\vec{x} &= 0 \\
  A(B\vec{x}) &= 0 \\
  (AB)\vec{x} &= 0 \\
  nullity(AB) &= 0 \\
  rank(AB) &= n-nullity(AB) = n
\end{align*}

\subsubsection*{Exercise 59a}
Prove that \( rank(AB)\le rank(B) \).
\begin{align*}
  n &= rank(AB)+nullity(AB) \\
  &= rank(B)+nullity(B) \\
  nullity(B) &= dim(null(B)) \\
  &\le dim(null(AB)) \\
  &\le nullity(AB) \\
  rank(AB)+nullity(AB) &= rank(B)+nullity(B) \\
  \therefore rank(AB) &\le rank(B)
\end{align*}

\subsubsection*{Exercise 60a}
Prove that \( rank(AB)\le rank(A) \).
\begin{align*}
  AB &= A\begin{bmatrix}col_1(B) & \dots & col_n(B)\end{bmatrix} \\
  col(AB) &\subseteq col(B) \\
  dim(col(AB)) &\le dim(col(B)) \\
  rank(AB) &\le rank(B)
\end{align*}

\subsubsection*{Exercise 61}
Prove that if \( U \) is invertible, then \( rank(UA) = rank(A) \).
\begin{align*}
  A &= IA \\
  A &= U^{-1}UA \\
  rank(U) &= n \\
  \therefore rank(A) &= n = rank(U)
\end{align*}
Prove that if \( V \) is invertible, then \( rank(AV) = rank(A) \).
\begin{align*}
  A &= AI \\
  A &= AV^{-1}V \\
  rank(U) &= n \\
  \therefore rank(A) &= n = rank(V)
\end{align*}

\subsubsection*{Exercise 62}
Prove that an \( m\times n \) matrix \( A \) has rank 1 if and only if \( A \)
can be written as the outer product \( uv^T \) of a vector \( u\in\R^m \) and
\( v\in\R^n \). \\
Suppose:
\begin{align*}
  A &= \begin{bmatrix}a_1 \\ \vdots \\ a_n\end{bmatrix}
    = \begin{bmatrix}c_1 \\ \vdots \\ c_n\end{bmatrix}w \\
  u &= \begin{bmatrix}c_1 \\ \vdots \\ c_n\end{bmatrix} \\
  w &= v^T \\
  A &= uv^T \\
  rank(A) &= rank(uv^T) \\
  &\le rank(u) \\
  &\le 1 \\
  rank(A) &= 1
\end{align*}

\subsubsection*{Exercise 63}
Prove that an \( m\times n \) matrix \( A \) has rank \( r \), prove that
\( A \) can be written as the sum of \( r \) matrices, each of which has rank 1.
\begin{align*}
  rank(A) &= r \\
  &= dim(row(A)) \\
  A &= \begin{bmatrix}a_1 \\ \vdots \\ a_m\end{bmatrix} \\
  &= \begin{bmatrix}
    c_{11}v_1+c_{12}v_2+\dots+c_{1r}v_r \\
    c_{21}v_1+c_{22}v_2+\dots+c_{2r}v_r \\
    \vdots \\
    c_{m1}v_1+v_{m2}v_2+\dots+c_{mr}v_r
  \end{bmatrix} \\
  &= \begin{bmatrix}c_{11} \\ c_{21} \\ \vdots \\ c_{m1}\end{bmatrix}v_1+\dots+
    \begin{bmatrix}c_{1r} \\ c_{2r} \\ \vdots \\ c_{mr}\end{bmatrix}v_r \\
\end{align*}
Because \( \begin{bmatrix}c_{11} \\ c_{21} \\ \vdots \\ c_{m1}\end{bmatrix} \),
has rank 1, each of the \( r \) matrices has rank 1.

\subsubsection*{Exercise 64}
Prove that, for \( m\times n \) matrices \( A \) and \( B \), \( rank(A+B)\le
rank(A)+rank(B) \).
\[ row_i(A+B) = row_i(A)+row_i(B) \]
The rows of \( A+B \) can be expressed as linear combinations of the respective
rows of \( A \) and \( B \).

\subsubsection*{Exercise 65}
Let \( A \) be an \( n\times n \) matrix such that \( A^2 = 0 \). Prove that
\( rank(A) \le \frac{n}{2} \).
\begin{align*}
  A^2 &= A\begin{bmatrix}col_1(A) & \dots & col_n(A)\end{bmatrix} \\
  &= 0 \\
  A\vec{x} &= 0 \\
  col(A) &\subseteq null(A) \\
  rank(A)+nullity(A) &= n \\
  rank(A)+rank(A) &\le n \\
  2rank(A) &\le n \\
  rank(A) &\le \frac{n}{2}
\end{align*}

\subsubsection*{Exercise 66}
Let \( A \) be a skew-symmetric \( n\times n \) matrix.
\begin{itemize}
  \item Prove that \( x^TAx = 0 \) for all \( x\in\R^n \).
  \begin{align*}
    x^TAx &= (x^TAx)^T) \\
    &= (Ax)^T(x^T)^T \\
    &= x^TA^Tx \\
    &= x^T(-A)x \\
    x^TAx &= -x^TAx \\
    \therefore x^TAx &= 0
  \end{align*}
  \item Prove that \( I+A \) is invertible. \\
  If this is true \( (I+A)x = 0 \) has only the trivial solution.
  \begin{align*}
    (I+A)x &= 0 \\
    x+Ax &= 0 \\
    x^Tx+x^TAx &= 0(x^T) \\
    x^Tx+0 &= 0 \\
    \therefore x &= 0
  \end{align*}
\end{itemize}

\section*{Section 3.6}

\subsubsection*{Exercise 1}
Let \( T_A:\R^2\to\R^2 \) be the matrix transformation corresponding to
\( A = \begin{bmatrix}2 & -1 \\ 3 & 4\end{bmatrix} \). Find \( T_A(\vec{u}) \)
and \( T_A(\vec{v}) \), where \( \vec{u} = \begin{bmatrix}1 \\ 2\end{bmatrix} \)
and \( \vec{v} = \begin{bmatrix}3 \\ -2\end{bmatrix} \).
\begin{align*}
  T_A(\vec{u}) &= \begin{bmatrix}0 \\ 11\end{bmatrix} \\
  T_A(\vec{v}) &= \begin{bmatrix}8 \\ 1\end{bmatrix}
\end{align*}

\subsubsection*{Exercise 3}
Prove that the given transformation is a linear transformation, using the
definition.
\begin{align*}
  T\begin{bmatrix}x \\ y\end{bmatrix} &=
    \begin{bmatrix}x+y \\ x-y\end{bmatrix} \\
  T(u+v) &= T\left(\begin{bmatrix}
    u_x+v_x \\
    u_y+v_y
  \end{bmatrix}\right) \\
  &= \begin{bmatrix}
    u_x+v_x+u_y+v_y \\
    u_x+v_x-u_y-v_y
  \end{bmatrix} \\
  &= \begin{bmatrix}
    (u_x+u_y)+(v_x+v_y) \\
    (u_x-u_y)+(v_x-v_y)
  \end{bmatrix} \\
  &= T(u)+T(v) \\
  T(cu) &= \begin{bmatrix}
    cu_x+cu_y \\
    cu_x-cu_y
  \end{bmatrix} \\
  &= c\begin{bmatrix}
    u_x+u_y \\
    u_x-u_y
  \end{bmatrix} \\
  &= cT(u)
\end{align*}

\subsubsection*{Exercise 5}
Prove that the given transformation is a linear transformation, using the
definition.
\begin{align*}
  T\begin{bmatrix}x \\ y \\ z\end{bmatrix} &= \begin{bmatrix}
    x-y+z \\
    2x+y-3z
  \end{bmatrix} \\
  T(u+v) &= T\left(\begin{bmatrix}
    u_x+v_x \\
    u_y+v_y
  \end{bmatrix}\right) \\
  &= \begin{bmatrix}
    (u_x+v_x)-(u_y+v_y)+(u_z+v_z) \\
    2(u_x+v_x)+(u_y+v_y)-3(u_z+v_z)
  \end{bmatrix} \\
  &= \begin{bmatrix}
    (u_x-u_y+u_z)+(v_x-v_y+v_z) \\
    (2u_x+u_y-3u_z)+(2v_x+v_y-3v_z)
  \end{bmatrix} \\
  &= T(u)+T(v) \\
  T(cu) &= \begin{bmatrix}
    cu_x-cu_y+cu_z \\
    c2u_x+cu_y-c3u_z
  \end{bmatrix} \\
  &= c\begin{bmatrix}
    u_x-u_y+u_z \\
    2u_x+u_y-3u_z
  \end{bmatrix} \\
  &= cT(u)
\end{align*}

\subsubsection*{Exercise 7}
Give a counterexample to show that the given transformation is not a linear
transformation.
\[ T\begin{bmatrix}x \\ y\end{bmatrix} = \begin{bmatrix}y \\ x^2\end{bmatrix} \]
\begin{align*}
  T\left(2\begin{bmatrix}3 \\ 3\end{bmatrix}\right) &=
    \begin{bmatrix}6 \\ 36\end{bmatrix} \\
  &\ne 2T\left(\begin{bmatrix}3 \\ 3\end{bmatrix}\right) \\
  &\ne \begin{bmatrix}6 \\ 18\end{bmatrix}
\end{align*}

\subsubsection*{Exercise 9}
Give a counterexample to show that the given transformation is not a linear
transformation.
\[ T\begin{bmatrix}x \\ y\end{bmatrix} =
  \begin{bmatrix}xy \\ x+y\end{bmatrix} \]
\begin{align*}
  T\left(2\begin{bmatrix}3 \\ 3\end{bmatrix}\right) &=
    \begin{bmatrix}36 \\ 12\end{bmatrix} \\
  &\ne 2T\left(\begin{bmatrix}3 \\ 3\end{bmatrix}\right) \\
  &\ne \begin{bmatrix}18 \\ 12\end{bmatrix}
\end{align*}

\subsubsection*{Exercise 11}
Find the standard matrix of the linear transformation.
\begin{align*}
  T\begin{bmatrix}x \\ y\end{bmatrix} &=
    \begin{bmatrix}x+y \\ x-y\end{bmatrix} \\
  &= \begin{bmatrix}
    1 & 1 \\
    1 & -1
  \end{bmatrix}
\end{align*}

\subsubsection*{Exercise 13}
Find the standard matrix of the linear transformation.
\begin{align*}
  T\begin{bmatrix}x \\ y \\ z\end{bmatrix} &= \begin{bmatrix}
    x-y+z \\
    2x+y-3z
  \end{bmatrix} \\
  &= \begin{bmatrix}
    1 & -1 & 1 \\
    2 & 1 & -3
  \end{bmatrix}
\end{align*}

\subsubsection*{Exercise 40}
Use matrices to prove the given statements about the transformations from
\( \R^2 \) to \( \R^2 \). If \( R_{\theta} \) denotes a rotation (about the
origin) through the angle \( \theta \), then \( R_{\alpha}\circ R_{\beta} =
R_{\alpha}+R_{\beta} \).
\begin{align*}
  R_{\alpha} &= \begin{bmatrix}
    \cos\alpha & -\sin\alpha \\
    \sin\alpha & \cos\alpha
  \end{bmatrix} \\
  R_{\beta} &= \begin{bmatrix}
    \cos\beta & -\sin\beta \\
    \sin\beta & \cos\beta
  \end{bmatrix} \\
  R_{\alpha}\circ R_{\beta} &= \begin{bmatrix}
    \cos\alpha & -\sin\alpha \\
    \sin\alpha & \cos\alpha
  \end{bmatrix}\begin{bmatrix}
    \cos\beta & -\sin\beta \\
    \sin\beta & \cos\beta
  \end{bmatrix} \\
  &= \begin{bmatrix}
    \cos\alpha\cos\beta-\sin\alpha\sin\beta &
      -\cos\alpha\sin\beta-\sin\alpha\cos\beta \\
    \sin\alpha\cos\beta+\cos\alpha\sin\beta &
      -\sin\alpha\sin\beta+\cos\alpha\cos\beta
  \end{bmatrix} \\
  &= \begin{bmatrix}
    \cos\alpha\cos\beta-\sin\alpha\sin\beta &
      -(\cos\alpha\sin\beta+\sin\alpha\cos\beta) \\
    \sin\alpha\cos\beta+\cos\alpha\sin\beta &
      \cos\alpha\cos\beta-\sin\alpha\sin\beta
  \end{bmatrix} \\
  &= \begin{bmatrix}
    \cos(\alpha+\beta) & -\sin(\alpha+\beta) \\
    \sin(\alpha+\beta) & \cos(\alpha+\beta)
  \end{bmatrix} \\
  &= R_{\alpha+\beta}
\end{align*}

\subsubsection*{Exercise 42}
\begin{enumerate}[(a)]
  \item If \( P \) is a projection, then \( P\circ P = P \).
  \begin{align*}
    P &= \begin{bmatrix}
      \frac{d_x^2}{d_x^2+d_y^2} & \frac{d_xd_y}{d_x^2+d_y^2} \\[0.25em]
      \frac{d_xd_y}{d_x^2+d_y^2} & \frac{d_y^2}{d_x^2+d_y^2}
    \end{bmatrix} \\
    P\circ P &= \begin{bmatrix}
      \frac{d_x^4+d_x^2d_y^2}{(d_x^2+d_y^2)^2} &
        \frac{d_x^3d_y+d_xd_y^3}{(d_x^2+d_y^2)^2} \\[0.25em]
      \frac{d_x^3d_y+d_xd_y^3}{(d_x^2+d_y^2)^2} &
        \frac{d_y^4+d_x^2d_y^2}{(d_1^2+d_2^2)^2}
    \end{bmatrix} \\
    &= P
  \end{align*}
\end{enumerate}

\subsubsection*{Exercise 44}
Let \( T \) be a linear transformation from \( \R^2 \) to \( \R^2 \). Prove
that \( T \) maps a straight line to a straight line or a point.
\begin{align*}
  l &= \vec{x}+t\vec{d} \\
  T(\vec{x}+t\vec{d}) &= T(\vec{x})+T(t\vec{d}) \\
  &= T(\vec{x})+tT(\vec{d})
\end{align*}
When \( t = 0 \), the result is a point, otherwise, the resulting mapping is
a line.

\subsubsection*{Exercise 52}
Prove that \( P_l(c\vec{v}) = cP_l(\vec{v}) \) for an scalar c.
\begin{align*}
  P_l(c\vec{v}) &=
    \left(\frac{\vec{d}\cdot(c\vec{v})}{\vec{d}\cdot\vec{d}}\vec{d}\right) \\
  &= \left(\frac{c(\vec{d}\cdot\vec{v})}{\vec{d}\cdot\vec{d}}\vec{d}\right) \\
  &= c\left(\frac{\vec{d}\cdot\vec{v}}{\vec{d}\cdot\vec{d}}~\vec{d}\right) \\
  &= cP_l(\vec{v})
\end{align*}

\subsubsection*{Exercise 53}
Prove that \( T:\R^n\to\R^m \) is a linear transformation if and only if:
\[ T(c_1\vec{v_1}+c_2\vec{v_2}) = c_1T(\vec{v_1})+c_2T(\vec{v_2}) \]
\begin{align*}
  T(c_1\vec{v_1}+c_2\vec{v_2}) &= T(c_1\vec{v_1})+T(c_2\vec{v_2}) \\
  &= c_1T(\vec{v_1})+c_2T(\vec{v_2})
\end{align*}

\subsubsection*{Exercise 54}
Prove that (as noted at the beginning of this section) the range of a linear
transformation \( T:\R^n\to\R^m \) is the column space of its matrix \( [T] \).
\begin{align*}
  [T] &= \begin{bmatrix}
    a_{11} & \dots & a_{1n} \\
    \vdots & \vdots & \vdots \\
    a_{m1} & \dots & a_{mn}
  \end{bmatrix} \\
  \vec{u} &= \begin{bmatrix}u_1 \\ \vdots \\ u_n\end{bmatrix} \\
  T(\vec{u}) &= u_1\begin{bmatrix}a_{11} \\ \vdots \\ a_{m1}\end{bmatrix}+\dots+
    u_n\begin{bmatrix}a_{1n} \\ \vdots \\ a_{mn}\end{bmatrix}
\end{align*}
The range of \( T \) is a linear combination of the columns, so it is a subset
of the column space. The converse is also true, the column space is a subset of
the range. Therefore, the two sets are equal.

\begin{center}
  If you have any questions, comments, or concerns, please contact me at
  alvin@omgimanerd.tech
\end{center}

\end{document}
