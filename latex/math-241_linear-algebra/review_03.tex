\documentclass{math}

\usepackage{enumerate}

\title{Linear Algebra}
\author{Alvin Lin}
\date{August 2017 - December 2017}

\begin{document}

\maketitle

\section*{Review 3}
Test Topics:
\begin{enumerate}
  \item Use Cramer's Rule to solve a \( 2\times2 \) or \( 3\times3 \) linear
  system.
  \[ x_i = \frac{|A_i(\vec{b})|}{|A|} \]
  \item Similar to \#45,46 from Section 4.2
  \item Similar to Section 4.2: \#47-52
  \item Similar to one or more exercises from Section 4.2: \#53-56, 65, 66
  \item Similar to the following from Section 4.4: \#36, 37, 42-51, 52a
\end{enumerate}

\subsubsection*{Example}
Solve the system:
\begin{align*}
  2x-y &= 5 \\
  x+3y &= -1
\end{align*}
\begin{align*}
  A &= \begin{bmatrix}2 & -1 \\ 1 & 3\end{bmatrix} \quad
  \vec{b} = \begin{bmatrix}5 \\ -1\end{bmatrix} \\
  |A_1(\vec{b})| &= \begin{vmatrix}5 & -1 \\ -1 & 3\end{vmatrix} = 14 \\
  |A_2(\vec{b})| &= \begin{vmatrix}2 & 5 \\ 1 & -1\end{vmatrix} = -7 \\
  |A| &= 6 \\
  x_1 &= \frac{14}{7} = 2 \\
  x_2 &= \frac{-7}{7} = -1 \\
  \vec{x} &= \begin{bmatrix}2 \\ -1\end{bmatrix}
\end{align*}

\subsubsection*{Example}
If \( A \) and \( B \) are invertible matrices, show \( AB \) and \( BA \) are
similar. Find an invertible matrix \( P \) such that:
\[ P^{-1}ABP = BA \]
\[ ABP = PAB \]
\( P = A \) satisfies this.

\subsubsection*{Example}
If \( A \) and \( B \) are similar matrices, show \( tr(A) = tr(B) \). \par
There exists an invertible matrix \( P \) such that \( P^{-1}AP = B \).
\begin{align*}
  tr(B) &= tr(P^{-1}AP) \\
  &= tr((P^{-1}A)P) \\
  &= tr(P(P^{-1}A)) \\
  &= tr(PP^{-1}A) \\
  &= tr(IA) \\
  &= tr(A)
\end{align*}

\subsubsection*{Example}
Show that if \( A\sim B \), then \( A^T\sim B^T \).
\begin{align*}
  A\sim B &\to P^{-1}AP = B \\
  (P^{-1}AP)^T = B^T &\to P^TA^T(P^{-1})^T = B^T \\
  Let~Q &= (P^{-1})^T \\
  Q^{-1}A^TQ &= B^T \\
  A^T &\sim B^T
\end{align*}

\subsubsection*{Example}
Prove that if \( A \) is diagonalizable, so is \( A^T \). If \( A \) is
diagonalizable, then \( A\sim D \) where \( D \) is the diagonal matrix.
\par
If \( A \) is diagonalizable, there exists an invertible matrix \( P \) such
that \( P^{-1}AP = D \).
\begin{align*}
  (P^{-1}AP)^T = D^T &\to P^TA^T(P^{-1})^T = D^T = D \\
  Q &= (P^{-1})^T \\
  Q^{-1}A^TQ &= D \\
  A^T &\sim D
\end{align*}
Thus, \( A^T \) is diagonalizable.

\subsubsection*{Example}
Let \( A \) be an invertible matrix. If \( A \) is diagonalizable, so is
\( A^{-1} \).
\begin{align*}
  P^{-1}AP &= D \\
  (P^{-1}AP)^{-1} &= D^{-1} \\
  P^{-1}A^{-1}(P^{-1})^{-1} &= D^{-1} \\
  P^{-1}A^{-1}P &= D^{-1} \\
  A^{-1} &\sim D^{-1}
\end{align*}
Therefore, \( A \) is diagonalizable.

\subsubsection*{Example}
Prove that if \( A \) is a diagonalizable matrix with only 1 eigenvalue
\( \lambda \), then \( A = \lambda I \).
\begin{align*}
  P^{-1}AP &= D \\
  &= \begin{bmatrix}
    \lambda & 0 & \dots & 0 \\
    0 & \lambda & \dots & 0 \\
    0 & \vdots & \vdots & 0 \\
    0 & 0 & \dots & \lambda
  \end{bmatrix} \\
  &= \lambda I \\
  A &= P(\lambda I)P^{-1} \\
  &= \lambda(PIP^{-1}) \\
  &= \lambda(IPP^{-1}) \\
  &= \lambda(II) \\
  &= \lambda I
\end{align*}

\subsubsection*{Example}
Let \( A,B \) be similar matrices. Prove that the algebraic multiplicity of the
eigenvalues of \( A,B \) are the same. \par
Since \( A \) and \( B \) are similar, they have the same characteristic
polynomial.

\subsubsection*{Example}
Prove that if \( A \) is a diagonalizable matrix such that every eigenvalue of
\( A \) is 0 or 1, then \( A \) is idempotent (\( A^2 = A \)).
\begin{align*}
  P^{-1}AP &= D \\
  A &= PDP^{-1} \\
  A^2 &= (PDP^{-1})^2 \\
  &= (PDP^{-1})(PDP^{-1}) \\
  &= PD^2P^{-1} \\
  &= PDP^{-1} \\
  &= A
\end{align*}

\subsubsection*{Example}
Let \( A \) be a nilpotent matrix (\( A^m = 0 \) for some \( m>1 \)). Prove that
if \( A \) is diagonalizable, then \( A = 0 \). \par
If \( A \) is diagonalizable, there exists an invertible matrix \( P \) such
that:
\begin{align*}
  P^{-1}AP &= D \\
  (P^{-1}AP)^m &= D^m \\
  P^{-1}A^mP &= D^m \\
  P^{-1}0P &= D^m \\
  0 &= D^m \\
  &= \begin{bmatrix}
    \lambda & 0 & \dots & 0 \\
    0 & \lambda & \dots & 0 \\
    0 & \vdots & \vdots & 0 \\
    0 & 0 & \dots & \lambda
  \end{bmatrix} \\
  \lambda_1^m &= \dots = \lambda_n^m = 0 \\
  D &= 0 \\
  P^{-1}AP &= 0 \\
  A &= PDP^{-1} \\
  &= P0P^{-1} \\
  &= 0
\end{align*}

\subsubsection*{Example}
Suppose \( A \) is a \( 6\times6 \) matrix with characteristic polynomial
\( C_A(\lambda) = (1+\lambda)(1-\lambda)^2(2-\lambda)^3 \). Prove that we can
not find 3 linearly independent vectors \( \vec{v_1},\vec{v_2},\vec{v_3} \)
such that \( A\vec{v_1} = 1\vec{v_1}, A\vec{v_2} = 1\vec{v_2}, A\vec{v_3} =
1\vec{v_3} \). \par
The algebraic multiplicity of \( \lambda = 1 \) is 2. The geometric multiplicity
of \( \lambda = 1 \) is at least 3. This cannot happen since the algebraic
multiplicity must upper bound the geometric multiplicity. \\ \\
If \( A \) is diagonalizable, compute the geometric multiplicities of
\( \lambda = -1,1,2 \).
\begin{align*}
  dim(E_{-1}) &= 1 \\
  dim(E_1) &= 2 \\
  dim(E_2) &= 3
\end{align*}

\subsubsection*{Example}
Assume we are working over just the real numbers.
\[ A = \begin{bmatrix}a & b \\ c & d\end{bmatrix} \]
Explain why \( A \) is diagonalizable if \( (a-d)^2+4bc>0 \).
\begin{align*}
  C_A(\lambda) &= |A-\lambda I| \\
  &= \begin{vmatrix}
    a-\lambda & b \\
    c & d-\lambda
  \end{vmatrix} \\
  &= ad+\lambda(-a-d)+\lambda^2-bc \\
  &= \lambda^2+\lambda(-a-d)+(ad-bc) \\
  \textbf{Quadratic Discriminant } D &= B^2-4AC \\
  &= (-a-d)^2-4(1)(ad-bc) \ge 0 \\
  &= (a+d)^2-4ad+4bc \ge 0 \\
  &= a^2+2ad+d^2-4ad+4bc \ge 0 \\
  &= a^2-2ad+d^2+4bc \ge 0 \\
  &= (a-d)^2+4bc \ge 0
\end{align*}

\subsubsection*{Example}
Let \( A,B \) be similar matrices. Prove that the geometric multiplicities of
\( A \) and \( B \) are the same. \par
Show that if \( B = P^{-1}AP \), then every eigenvector of \( B \) is of the
form \( P^{-1}\vec{v} \) for some eigenvector \( \vec{v} \) of \( A \). Let
\( \vec{w} \) be an eigenvector of \( B \).
\begin{align*}
  B\vec{w} = \lambda\vec{w} &\to (P^{-1}AP)(\vec{w}) = \lambda\vec{w} \\
  &\to (AP)\vec{w} = P(\lambda\vec{w}) \\
  A(P\vec{w}) &= \lambda(P\vec{w}) \\
  Let~\vec{v} &= P\vec{w} \\
  A\vec{v} &= \lambda\vec{v} \\
  \vec{v} = P\vec{w} &\to \vec{w} = P^{-1}\vec{v}
\end{align*}
Claim: Let \( \mathbb{B} \{\vec{v_1},\dots,\vec{v_k}\} \) be a basis for
eigenspace \( E_{\lambda} \) of \( A \). Then \( \mathbb{B'} =
\{P^{-1}\vec{v_1},P^{-1}\vec{v_2},\dots,P^{-1}\vec{v_k}\} \) is a basis for
eigenspace \( E_{\lambda} \) of \( B \). Show that \( \mathbb{B'} \) is linearly
independent.
\begin{align*}
  \vec{0} &= \sum_{i=1}^{k}c_iP^{-1}(\vec{v_i}) \\
  &= P^{-1}(\sum_{i=1}^{k}c_i\vec{v_i}) \\
  \vec{0} &= P\vec{0} \\
  &= \sum_{i=1}^{k}c_i\vec{v_i} \\
  c_1 &= c_2 = \dots = c_k = 0
\end{align*}
Show \( span(\mathbb{B'}) = E_{\lambda} \) in \( B \). Take \( \vec{w}\in
E_{\lambda} \) of \( B \). Then \( \vec{w} = P^{-1}\vec{v} \) for some
\( \vec{v}\in\R^n \).
\[ \vec{w} = P^{-1}(\sum c_i\vec{v_i}) = \sum c_1P^{-1}(\vec{v_i})\in
  span(\mathbb{B'}) \]
\[ span(\mathbb{B'}) = E_{\lambda}~of~B \]

\subsubsection*{Example}
Let \( A,B \) be \( n\times n \) matrices with \( n \) distinct eigenvalues.
Prove that \( A \) and \( B \) have the same eigenvectors if and only if
\( AB = BA \). \par
Suppose \( A \) and \( B \) have the same eigenvectors. \( A,B \) having \( n \)
distinct eigenvectors implies that \( A,B \) are diagonalizable.
\begin{align*}
  P^{-1}AP = D_1 &\to A = PD_1P^{-1} \\
  P^{-1}BP = D_2 &\to B = PD_2P^{-1} \\
  AB &= (PD_1P^{-1})(PD_2P^{-1}) = PD_1D_2P^{-1} \\
  BA &= (PD_1P^{-1})(PD_1P^{-1}) = PD_2D_1P^{-1}
\end{align*}
Since \( D_1,D_2 \) are diagonal: \( D_1D_2 = D_2D_1 \). Assume \( AB = BA \).
Show that \( A \) and \( B \) have the same eigenvectors. Take \( \vec{v} \)
as an eigenvector of \( A \).
\[ A(B\vec{v}) = (AB)\vec{v} = B(A\vec{v}) = B(\lambda\vec{v}) =
  \lambda(B\vec{v}) \]
So \( B\vec{v} \) is an eigenvector of \( A \).
\[ span(\vec{v}) = E_{\lambda} \]
Since we know \( B\vec{v}\in span(\vec{v}) \) and \( B\vec{v} =
\lambda_i\vec{v} \) for some \( \lambda_i \). So \( \vec{v} \) is an eigenvector
of \( B \).

\begin{center}
  You can find all my notes at \url{http://omgimanerd.tech/notes}. If you have
  any questions, comments, or concerns, please contact me at
  alvin@omgimanerd.tech
\end{center}

\end{document}
