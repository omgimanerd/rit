\documentclass[letterpaper, 12pt]{math}

\usepackage{amsmath}
\usepackage{amssymb}
\usepackage{geometry}
\usepackage{tikz}

\geometry{letterpaper, margin=0.5in}

\title{Linear Algebra: Homework 2}
\author{Alvin Lin}
\date{August 2016 - December 2016}

\begin{document}

\maketitle

2) Complete the following exercises from Section 1.2: 48 - 61, 62 (a), 63, 64 (a), 65 (a), 66 - 69, 70 (a, b), 73, 74

(3) Complete the following exercises from Section 1.3: 1 - 23 (odd)

(4) Complete the following exercises from Section 2.1: 1 - 5 (odd), 11 - 39 (odd)

\subsubsection*{Problem 48}
Find all values of the scalar \( k \) for which the two vectors are orthogonal.
\[ \vec{u} = \begin{bmatrix}2 \\ 3\end{bmatrix} \quad
  \vec{v} = \begin{bmatrix}k+1 \\ k-1\end{bmatrix} \]
\begin{align*}
  \vec{u}\cdot\vec{v} &= 0 \\
  2(k+1)+3(k-1) &= 0 \\
  2k+2+3k-3 &= 0 \\
  5k &= 1 \\
  k &= \frac{1}{5}
\end{align*}

\subsubsection*{Problem 49}
Find all values of the scalar \( k \) for which the two vectors are orthogonal.
\[ \vec{u} = \begin{bmatrix}1 \\ -1 \\ 2\end{bmatrix} \quad
  \vec{v} = \begin{bmatrix}k^2 \\ k \\ -3\end{bmatrix} \]
\begin{align*}
  \vec{u}\cdot\vec{v} &= 0 \\
  k^2-k-6 &= 0 \\
  (k-3)(k+2) &= 0 \\
  k &= 3 \quad k = -2
\end{align*}

\subsubsection*{Problem 50}
Describe all vectors that are orthogonal to \( \vec{u} = \begin{bmatrix}3 \\ 1
\end{bmatrix} \).
\[ 3x+y = 0 \]
They are all parallel to the line described by \( y = -3x \).

\subsubsection*{Problem 51}
Describe all vectors that are orthogonal to \( \vec{u} = \begin{bmatrix}a \\ b
\end{bmatrix} \). \\
They are all parallel to the line described by \( ax+by = 0 \).

\subsubsection*{Problem 52}
Under what conditions are the following true for vectors \( \vec{u} \) and
\( \vec{v} \) in \( \R^2 \) or \( R^3 \)?
\begin{enumerate}
  \item \( \|\vec{u}+\vec{v}\| = \|\vec{u}\|+\|\vec{v}\| \):
    This is true when the vectors are parallel.
  \item \( \|\vec{u}+\vec{v}\| = \|\vec{u}\|-\|\vec{v}\| \):
    This is true when the vectors are antiparallel.
\end{enumerate}

\subsubsection*{Problem 53}
Prove Theorem 1.2(b).
\[ \vec{u}\cdot(\vec{v}+\vec{w}) = \vec{u}\cdot\vec{v}+\vec{u}\cdot\vec{w} \]
Proof:
\begin{align*}
  \vec{u}\cdot(\vec{v}+\vec{w}) &= \sum_{i=1}^{n}u_{i}(v_{i}+w_{i}) \\
  &= \sum_{i=1}^{n}(u_{i}v_{i}+u_{i}w_{i}) \\
  &= \sum_{i=1}^{n}u_{i}v_{i}+\sum_{i=1}^{n}u_{i}w_{i} \\
  &= \vec{u}\cdot\vec{v}+\vec{u}\cdot\vec{w}
\end{align*}

\subsubsection*{Problem 54}
Prove Theorem 1.2(d).
\[ \vec{u}\cdot\vec{u} \ge 0 \]
\[ \vec{u}\cdot\vec{u} = 0 \quad\text{iff}\quad \vec{u} = \vec{0} \]
Proof:
\begin{align*}
  \vec{u}\cdot\vec{u} &= \sum_{i=1}^{n}u_iu_i \\
  &= \sum_{i=1}^{n}(u_i)^{2}
\end{align*}
\( (u_i)^{2} \) is non-negative, therefore the summation must be greater than
or equal to 0, and only equal to 0 when \( \vec{u} = \vec{0} \).

\subsubsection*{Problem 55}
Prove the stated property of distance between vectors.
\[ d(\vec{u},\vec{v}) = d(\vec{v},\vec{u}) \]
Proof:
\begin{align*}
  d(\vec{u},\vec{v}) &= \sqrt{\sum_{i=1}^{n}(u_i-v_i)^2} \\
  &= \sqrt{\sum_{i=1}^{n}(-1)^2(v_i-u_i)^2} \\
  &= \sqrt{\sum_{i=1}^{n}(v_i-u_i)^2} \\
  &= d(\vec{v},\vec{u})
\end{align*}

\subsubsection*{Problem 56}
Prove the stated property of distance between vectors.
\[ d(\vec{u},\vec{w}) \le d(\vec{u},\vec{v})+d(\vec{v},\vec{w}) \]
Proof:
\begin{align*}
  d(\vec{u},\vec{w}) &\le d(\vec{u},\vec{v})+d(\vec{v},\vec{w}) \\
  \|\vec{u}-\vec{w}\| &\le \|\vec{u}-\vec{v}\|+\|\vec{v}-\vec{w}\| \\
  \|\vec{u}-\vec{w}\| &\le \|\vec{u}-\vec{v}+\vec{v}-\vec{w}\|
    \quad\text{(By the triangle inequality)} \\
  \|\vec{u}-\vec{w}\| &\le \|\vec{u}-\vec{w}\|
\end{align*}

\subsubsection*{Problem 57}
Prove the stated property of distance between vectors.
\[ d(\vec{u},\vec{v}) = 0 \quad\text{iff}\quad \vec{u} = \vec{v} \]
Proof:
\begin{align*}
  0 &= d(\vec{u},\vec{v}) \\
  0 &= \sqrt{\sum_{i=1}^{n}(u_i-v_i)^2} \\
  0 &= \sum_{i=1}^{n}(u_i-v_i)^2 \\
  0 &= \sum_{i=1}^{n}(u_i-v_i) \\
  0 &= \sum_{i=1}^{n}u_i-\sum_{i=1}^{n}v_i \\
  0 &= \vec{u}-\vec{v} \\
  \vec{v} &= \vec{u}
\end{align*}

\subsubsection*{Problem 58}
Prove that \( \vec{u}\cdot c\vec{v} = c(\vec{u}\cdot\vec{v}) \).
\begin{align*}
  \vec{u}\cdot c\vec{v} &= \sum_{i=1}^{n}u_{i}(cv_{i}) \\
  &= \sum_{i=1}^{n}c(u_{i}v_{i}) \\
  &= c\sum_{i=1}^{n}u_{i}v_{i} \\
  &= c(\vec{u}\cdot\vec{v})
\end{align*}

\subsubsection*{Problem 59}
Prove that \( \|\vec{u}-\vec{v}\| \ge \|\vec{u}\|-\|\vec{v}\| \).
\begin{align*}
  \|\vec{w}+\vec{v}\| &\le \|\vec{w}\|+\|\vec{v}\|
    \quad\text{(Triangle Inequality)} \\
  Let&: \vec{w} = \vec{u}-\vec{v} \\
  \|\vec{u}-\vec{v}+\vec{v}\| &\le \|\vec{u}-\vec{v}\|+\|\vec{v}\| \\
  \|\vec{u}\| &\le \|\vec{u}-\vec{v}\|+\|\vec{v}\| \\
  \|\vec{u}\|-\|\vec{v}\| &\le \|\vec{u}-\vec{v}\| \\
  \|\vec{u}-\vec{v}\| &\ge \|\vec{u}\|-\|\vec{v}\|
\end{align*}

\subsubsection*{Problem 60}
Suppose know that \( \vec{u}\cdot\vec{v} = \vec{u}\cdot\vec{w} \). Does it
follow that \( \vec{v} = \vec{w} \)? If it does, give a proof that is valid
in \( \R^n \). Otherwise, give a counterexample. \\
Suppose \( \vec{u} = \vec{0} \). \( \vec{v} \) and \( \vec{w} \) can be any
vector in \( \R^n \).
\[ \vec{0}\cdot\langle1,2\rangle = \vec{0}\cdot\langle3,4\rangle \]

\subsubsection*{Problem 61}
Prove that \( (\vec{u}+\vec{v})\cdot(\vec{u}-\vec{v}) =
\|\vec{u}\|^2-\|\vec{v}\|^2 \).
\begin{align*}
  (\vec{u}+\vec{v})\cdot(\vec{u}-\vec{v}) &= \vec{u}\cdot\vec{u}-
    \vec{u}\cdot\vec{v}+\vec{v}\cdot\vec{u}-\vec{v}\cdot\vec{v} \\
  &= \vec{u}\cdot\vec{u}-\vec{v}\cdot\vec{v} \\
  &= \|\vec{u}\|^2-\|\vec{v}\|^2
\end{align*}

\subsubsection*{Problem 62a}
Prove that \( \|\vec{u}+\vec{v}\|^2+\|\vec{u}-\vec{v}\|^2 = 2\|\vec{u}\|^2+
2\|\vec{v}\|^2 \).
\begin{align*}
  \|\vec{u}+\vec{v}\|^2+\|\vec{u}-\vec{v}\|^2 &=
    (\|\vec{u}\|^2+\|\vec{v}\|^2)+(\|\vec{u}\|^2+\|\vec{-v}\|^2)
    \quad\text{(Pythagorean Theorem)} \\
  &= \|\vec{u}\|^2+\|\vec{v}\|^2+\|\vec{u}\|^2+\|\vec{v}\|^2 \\
  &= 2\|\vec{u}\|^2+2\|\vec{v}\|^2
\end{align*}

\subsubsection*{Problem 63}
Prove that \( \vec{u}\cdot\vec{v} = \frac{1}{4}\|\vec{u}+\vec{v}\|^2-
\frac{1}{4}\|\vec{u}-\vec{v}\|^2 \).
\begin{align*}
  \frac{1}{4}\|\vec{u}+\vec{v}\|^2-\frac{1}{4}\|\vec{u}-\vec{v}\|^2 &=
    \frac{1}{4}\sqrt{(\vec{u}+\vec{v})\cdot(\vec{u}+\vec{v})}^2-
    \frac{1}{4}\sqrt{(\vec{u}-\vec{v})\cdot(\vec{u}-\vec{v})}^2 \\
  &= \frac{1}{4}(\vec{u}\cdot\vec{u}+2\vec{u}\cdot\vec{v}+\vec{v}\cdot\vec{v}-
    (\vec{u}\cdot\vec{u}+\vec{v}\cdot\vec{v})) \\
  &= \frac{1}{2}\vec{u}\cdot\vec{v}
\end{align*}

\subsubsection*{Problem 64a}
Prove that \( \|\vec{u}+\vec{v}\| = \|\vec{u}-\vec{v}\| \) if and only if
\( \vec{u} \) and \( \vec{v} \) are orthogonal.
\begin{align*}
  \|\vec{u}+\vec{v}\| &= \|\vec{u}-\vec{v}\| \\
  \sqrt{(\vec{u}+\vec{v})\cdot(\vec{u}+\vec{v})} &=
    \sqrt{(\vec{u}-\vec{v})\cdot(\vec{u}-\vec{v})} \\
  (\vec{u}+\vec{v})\cdot(\vec{u}+\vec{v}) &=
    (\vec{u}-\vec{v})\cdot(\vec{u}-\vec{v}) \\
  \vec{u}\cdot\vec{u}+2\vec{u}\cdot\vec{v}+\vec{v}\cdot\vec{v} &=
    \vec{u}\cdot\vec{u}+\vec{v}\cdot\vec{v} \\
  2\vec{u}\cdot\vec{v} = 0 \\
  \vec{u}\cdot\vec{v} = 0
\end{align*}
The dot product of two vectors is 0 if and only if the two vectors are
orthogonal.

\subsubsection*{Problem 65a}
Prove that \( \vec{u}+\vec{v} \) and \( \vec{u}-\vec{v} \) are orthogonal in
\( \R \) if and only if \( \|\vec{u}\| = \|\vec{v}\| \).
\begin{align*}
  (\vec{u}+\vec{v})\cdot(\vec{u}-\vec{v}) &= 0 \\
  \vec{u}\cdot\vec{u}-\vec{v}\cdot\vec{v} &= 0 \\
  \vec{u}\cdot\vec{u} &= \vec{v}\cdot\vec{v} \\
  \sqrt{\vec{u}\cdot\vec{u}} &= \sqrt{\vec{v}\cdot\vec{v}} \\
  \|\vec{u}\| &= \|\vec{v}\|
\end{align*}

\subsubsection*{Problem 66}
If \( \|\vec{u}\| = 2, \|\vec{v}\| = \sqrt{3} \), and \( \vec{u}\cdot\vec{v} =
1 \), find \( \|\vec{u}+\vec{v}\| \).
\begin{align*}
  \|\vec{u}+\vec{v}\| &= \sqrt{(\vec{u}+\vec{v})\cdot(\vec{u}+\vec{v})} \\
  &= \sqrt{\vec{u}\cdot\vec{u}+2\vec{u}\cdot\vec{v}+\vec{v}\cdot\vec{v}} \\
  &= \sqrt{\|\vec{u}\|^2+2(1)+\|\vec{v}\|^2} \\
  &= \sqrt{2^2+2+\sqrt{3}^2} \\
  &= \sqrt{4+2+3} \\
  &= \sqrt{9} = 3
\end{align*}

\subsubsection*{Problem 67}
Show that there are no vectors \( \vec{u} \) and \( \vec{v} \) such that
\( \|\vec{u}\| = 1, \|\vec{v}\| = 2 \), and \( \vec{u}\cdot\vec{v} = 3 \).
\begin{align*}
  \|\vec{u}+\vec{v}\| &\le \|\vec{u}\|+\|\vec{v}\| \\
  \sqrt{\|\vec{u}\|^2+2\vec{u}\cdot\vec{v}+\|\vec{v}\|^2} &\le
    \|\vec{u}\|+\|\vec{v}\| \\
  \sqrt{1^2+2(3)+2^2} &\le 1+2 \\
  \sqrt{1+6+4} &\le 3 \\
  \sqrt{11} &\le 3 \\
  3.316 &\le 3
\end{align*}
Since this case violates the triangle inequality, there can be no such vectors.

\subsubsection*{Problem 68}

\subsubsection*{Problem 69}
\subsubsection*{Problem 70a}
\subsubsection*{Problem 70b}
\subsubsection*{Problem 73}
\subsubsection*{Problem 74}

\begin{center}
  If you have any questions, comments, or concerns, please contact me at
  alvin@omgimanerd.tech
\end{center}

\end{document}
