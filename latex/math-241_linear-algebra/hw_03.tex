\documentclass{math}

\geometry{letterpaper, margin=0.5in}

\title{Linear Algebra: Homework 3}
\author{Alvin Lin}
\date{August 2016 - December 2016}

\begin{document}

\maketitle

\section*{Section 1.2}

\subsubsection*{Exercise 48}
Find all values of the scalar \( k \) for which the two vectors are orthogonal.
\[ \vec{u} = \begin{bmatrix}2 \\ 3\end{bmatrix} \quad
  \vec{v} = \begin{bmatrix}k+1 \\ k-1\end{bmatrix} \]
\begin{align*}
  \vec{u}\cdot\vec{v} &= 0 \\
  2(k+1)+3(k-1) &= 0 \\
  2k+2+3k-3 &= 0 \\
  5k &= 1 \\
  k &= \frac{1}{5}
\end{align*}

\subsubsection*{Exercise 49}
Find all values of the scalar \( k \) for which the two vectors are orthogonal.
\[ \vec{u} = \begin{bmatrix}1 \\ -1 \\ 2\end{bmatrix} \quad
  \vec{v} = \begin{bmatrix}k^2 \\ k \\ -3\end{bmatrix} \]
\begin{align*}
  \vec{u}\cdot\vec{v} &= 0 \\
  k^2-k-6 &= 0 \\
  (k-3)(k+2) &= 0 \\
  k &= 3 \quad k = -2
\end{align*}

\subsubsection*{Exercise 50}
Describe all vectors that are orthogonal to \( \vec{u} = \begin{bmatrix}3 \\ 1
\end{bmatrix} \).
\[ 3x+y = 0 \]
They are all parallel to the line described by \( y = -3x \).

\subsubsection*{Exercise 51}
Describe all vectors that are orthogonal to \( \vec{u} = \begin{bmatrix}a \\ b
\end{bmatrix} \). \\
They are all parallel to the line described by \( ax+by = 0 \).

\subsubsection*{Exercise 52}
Under what conditions are the following true for vectors \( \vec{u} \) and
\( \vec{v} \) in \( \R^2 \) or \( R^3 \)?
\begin{enumerate}
  \item \( \|\vec{u}+\vec{v}\| = \|\vec{u}\|+\|\vec{v}\| \):
    This is true when the vectors are parallel.
  \item \( \|\vec{u}+\vec{v}\| = \|\vec{u}\|-\|\vec{v}\| \):
    This is true when the vectors are antiparallel.
\end{enumerate}

\subsubsection*{Exercise 53}
Prove Theorem 1.2(b).
\[ \vec{u}\cdot(\vec{v}+\vec{w}) = \vec{u}\cdot\vec{v}+\vec{u}\cdot\vec{w} \]
Proof:
\begin{align*}
  \vec{u}\cdot(\vec{v}+\vec{w}) &= \sum_{i=1}^{n}u_{i}(v_{i}+w_{i}) \\
  &= \sum_{i=1}^{n}(u_{i}v_{i}+u_{i}w_{i}) \\
  &= \sum_{i=1}^{n}u_{i}v_{i}+\sum_{i=1}^{n}u_{i}w_{i} \\
  &= \vec{u}\cdot\vec{v}+\vec{u}\cdot\vec{w}
\end{align*}

\subsubsection*{Exercise 54}
Prove Theorem 1.2(d).
\[ \vec{u}\cdot\vec{u} \ge 0 \]
\[ \vec{u}\cdot\vec{u} = 0 \quad\text{iff}\quad \vec{u} = \vec{0} \]
Proof:
\begin{align*}
  \vec{u}\cdot\vec{u} &= \sum_{i=1}^{n}u_iu_i \\
  &= \sum_{i=1}^{n}(u_i)^{2}
\end{align*}
\( (u_i)^{2} \) is non-negative, therefore the summation must be greater than
or equal to 0, and only equal to 0 when \( \vec{u} = \vec{0} \).

\subsubsection*{Exercise 55}
Prove the stated property of distance between vectors.
\[ d(\vec{u},\vec{v}) = d(\vec{v},\vec{u}) \]
Proof:
\begin{align*}
  d(\vec{u},\vec{v}) &= \sqrt{\sum_{i=1}^{n}(u_i-v_i)^2} \\
  &= \sqrt{\sum_{i=1}^{n}(-1)^2(v_i-u_i)^2} \\
  &= \sqrt{\sum_{i=1}^{n}(v_i-u_i)^2} \\
  &= d(\vec{v},\vec{u})
\end{align*}

\subsubsection*{Exercise 56}
Prove the stated property of distance between vectors.
\[ d(\vec{u},\vec{w}) \le d(\vec{u},\vec{v})+d(\vec{v},\vec{w}) \]
Proof:
\begin{align*}
  d(\vec{u},\vec{w}) &\le d(\vec{u},\vec{v})+d(\vec{v},\vec{w}) \\
  \|\vec{u}-\vec{w}\| &\le \|\vec{u}-\vec{v}\|+\|\vec{v}-\vec{w}\| \\
  \|\vec{u}-\vec{w}\| &\le \|\vec{u}-\vec{v}+\vec{v}-\vec{w}\|
    \quad\text{(By the triangle inequality)} \\
  \|\vec{u}-\vec{w}\| &\le \|\vec{u}-\vec{w}\|
\end{align*}

\subsubsection*{Exercise 57}
Prove the stated property of distance between vectors.
\[ d(\vec{u},\vec{v}) = 0 \quad\text{iff}\quad \vec{u} = \vec{v} \]
Proof:
\begin{align*}
  0 &= d(\vec{u},\vec{v}) \\
  0 &= \sqrt{\sum_{i=1}^{n}(u_i-v_i)^2} \\
  0 &= \sum_{i=1}^{n}(u_i-v_i)^2 \\
  0 &= \sum_{i=1}^{n}(u_i-v_i) \\
  0 &= \sum_{i=1}^{n}u_i-\sum_{i=1}^{n}v_i \\
  0 &= \vec{u}-\vec{v} \\
  \vec{v} &= \vec{u}
\end{align*}

\subsubsection*{Exercise 58}
Prove that \( \vec{u}\cdot c\vec{v} = c(\vec{u}\cdot\vec{v}) \).
\begin{align*}
  \vec{u}\cdot c\vec{v} &= \sum_{i=1}^{n}u_{i}(cv_{i}) \\
  &= \sum_{i=1}^{n}c(u_{i}v_{i}) \\
  &= c\sum_{i=1}^{n}u_{i}v_{i} \\
  &= c(\vec{u}\cdot\vec{v})
\end{align*}

\subsubsection*{Exercise 59}
Prove that \( \|\vec{u}-\vec{v}\| \ge \|\vec{u}\|-\|\vec{v}\| \).
\begin{align*}
  \|\vec{w}+\vec{v}\| &\le \|\vec{w}\|+\|\vec{v}\|
    \quad\text{(Triangle Inequality)} \\
  Let&: \vec{w} = \vec{u}-\vec{v} \\
  \|\vec{u}-\vec{v}+\vec{v}\| &\le \|\vec{u}-\vec{v}\|+\|\vec{v}\| \\
  \|\vec{u}\| &\le \|\vec{u}-\vec{v}\|+\|\vec{v}\| \\
  \|\vec{u}\|-\|\vec{v}\| &\le \|\vec{u}-\vec{v}\| \\
  \|\vec{u}-\vec{v}\| &\ge \|\vec{u}\|-\|\vec{v}\|
\end{align*}

\subsubsection*{Exercise 60}
Suppose know that \( \vec{u}\cdot\vec{v} = \vec{u}\cdot\vec{w} \). Does it
follow that \( \vec{v} = \vec{w} \)? If it does, give a proof that is valid
in \( \R^n \). Otherwise, give a counterexample. \\
Suppose \( \vec{u} = \vec{0} \). \( \vec{v} \) and \( \vec{w} \) can be any
vector in \( \R^n \).
\[ \vec{0}\cdot\langle1,2\rangle = \vec{0}\cdot\langle3,4\rangle \]

\subsubsection*{Exercise 61}
Prove that \( (\vec{u}+\vec{v})\cdot(\vec{u}-\vec{v}) =
\|\vec{u}\|^2-\|\vec{v}\|^2 \).
\begin{align*}
  (\vec{u}+\vec{v})\cdot(\vec{u}-\vec{v}) &= \vec{u}\cdot\vec{u}-
    \vec{u}\cdot\vec{v}+\vec{v}\cdot\vec{u}-\vec{v}\cdot\vec{v} \\
  &= \vec{u}\cdot\vec{u}-\vec{v}\cdot\vec{v} \\
  &= \|\vec{u}\|^2-\|\vec{v}\|^2
\end{align*}

\subsubsection*{Exercise 62a}
Prove that \( \|\vec{u}+\vec{v}\|^2+\|\vec{u}-\vec{v}\|^2 = 2\|\vec{u}\|^2+
2\|\vec{v}\|^2 \).
\begin{align*}
  \|\vec{u}+\vec{v}\|^2+\|\vec{u}-\vec{v}\|^2 &=
    (\|\vec{u}\|^2+\|\vec{v}\|^2)+(\|\vec{u}\|^2+\|\vec{-v}\|^2)
    \quad\text{(Pythagorean Theorem)} \\
  &= \|\vec{u}\|^2+\|\vec{v}\|^2+\|\vec{u}\|^2+\|\vec{v}\|^2 \\
  &= 2\|\vec{u}\|^2+2\|\vec{v}\|^2
\end{align*}

\subsubsection*{Exercise 63}
Prove that \( \vec{u}\cdot\vec{v} = \frac{1}{4}\|\vec{u}+\vec{v}\|^2-
\frac{1}{4}\|\vec{u}-\vec{v}\|^2 \).
\begin{align*}
  \frac{1}{4}\|\vec{u}+\vec{v}\|^2-\frac{1}{4}\|\vec{u}-\vec{v}\|^2 &=
    \frac{1}{4}\sqrt{(\vec{u}+\vec{v})\cdot(\vec{u}+\vec{v})}^2-
    \frac{1}{4}\sqrt{(\vec{u}-\vec{v})\cdot(\vec{u}-\vec{v})}^2 \\
  &= \frac{1}{4}(\vec{u}\cdot\vec{u}+2\vec{u}\cdot\vec{v}+\vec{v}\cdot\vec{v}-
    (\vec{u}\cdot\vec{u}-2\vec{u}\cdot\vec{v}+\vec{v}\cdot\vec{v})) \\
  &= \vec{u}\cdot\vec{v}
\end{align*}

\subsubsection*{Exercise 64a}
Prove that \( \|\vec{u}+\vec{v}\| = \|\vec{u}-\vec{v}\| \) if and only if
\( \vec{u} \) and \( \vec{v} \) are orthogonal.
\begin{align*}
  \|\vec{u}+\vec{v}\| &= \|\vec{u}-\vec{v}\| \\
  \sqrt{(\vec{u}+\vec{v})\cdot(\vec{u}+\vec{v})} &=
    \sqrt{(\vec{u}-\vec{v})\cdot(\vec{u}-\vec{v})} \\
  (\vec{u}+\vec{v})\cdot(\vec{u}+\vec{v}) &=
    (\vec{u}-\vec{v})\cdot(\vec{u}-\vec{v}) \\
  \vec{u}\cdot\vec{u}+2\vec{u}\cdot\vec{v}+\vec{v}\cdot\vec{v} &=
    \vec{u}\cdot\vec{u}+\vec{v}\cdot\vec{v} \\
  2\vec{u}\cdot\vec{v} = 0 \\
  \vec{u}\cdot\vec{v} = 0
\end{align*}
The dot product of two vectors is 0 if and only if the two vectors are
orthogonal.

\subsubsection*{Exercise 65a}
Prove that \( \vec{u}+\vec{v} \) and \( \vec{u}-\vec{v} \) are orthogonal in
\( \R \) if and only if \( \|\vec{u}\| = \|\vec{v}\| \).
\begin{align*}
  (\vec{u}+\vec{v})\cdot(\vec{u}-\vec{v}) &= 0 \\
  \vec{u}\cdot\vec{u}-\vec{v}\cdot\vec{v} &= 0 \\
  \vec{u}\cdot\vec{u} &= \vec{v}\cdot\vec{v} \\
  \sqrt{\vec{u}\cdot\vec{u}} &= \sqrt{\vec{v}\cdot\vec{v}} \\
  \|\vec{u}\| &= \|\vec{v}\|
\end{align*}

\subsubsection*{Exercise 66}
If \( \|\vec{u}\| = 2, \|\vec{v}\| = \sqrt{3} \), and \( \vec{u}\cdot\vec{v} =
1 \), find \( \|\vec{u}+\vec{v}\| \).
\begin{align*}
  \|\vec{u}+\vec{v}\| &= \sqrt{(\vec{u}+\vec{v})\cdot(\vec{u}+\vec{v})} \\
  &= \sqrt{\vec{u}\cdot\vec{u}+2\vec{u}\cdot\vec{v}+\vec{v}\cdot\vec{v}} \\
  &= \sqrt{\|\vec{u}\|^2+2(1)+\|\vec{v}\|^2} \\
  &= \sqrt{2^2+2+\sqrt{3}^2} \\
  &= \sqrt{4+2+3} \\
  &= \sqrt{9} = 3
\end{align*}

\subsubsection*{Exercise 67}
Show that there are no vectors \( \vec{u} \) and \( \vec{v} \) such that
\( \|\vec{u}\| = 1, \|\vec{v}\| = 2 \), and \( \vec{u}\cdot\vec{v} = 3 \).
\begin{align*}
  \|\vec{u}+\vec{v}\| &\le \|\vec{u}\|+\|\vec{v}\| \\
  \sqrt{\|\vec{u}\|^2+2\vec{u}\cdot\vec{v}+\|\vec{v}\|^2} &\le
    \|\vec{u}\|+\|\vec{v}\| \\
  \sqrt{1^2+2(3)+2^2} &\le 1+2 \\
  \sqrt{1+6+4} &\le 3 \\
  \sqrt{11} &\le 3 \\
  3.316 &\le 3
\end{align*}
Since this case violates the triangle inequality, there can be no such vectors.

\subsubsection*{Exercise 68}
Prove that if \( \vec{u} \) is orthogonal to both \( \vec{v} \) and \( \vec{w}
\), then \( \vec{u} \) is orthogonal to \( \vec{v}+\vec{w} \).
\begin{align*}
  \vec{u}\cdot\vec{v} &= 0 \\
  \vec{u}\cdot\vec{w} &= 0 \\
  \vec{u}\cdot\vec{v}+\vec{u}\cdot\vec{w} &= 0 \\
  \vec{u}\cdot(\vec{v}+\vec{w}) &= 0
\end{align*}
Prove that if \( \vec{u} \) is orthogonal to both \( \vec{v} \) and \( \vec{w}
\), then \( \vec{u} \) is orthogonal to \( s\vec{v}+t\vec{w} \) for all
scalars \( s \) and \( t \).
\begin{align*}
  \vec{u}\cdot\vec{v} &= 0 \\
  \vec{u}\cdot\vec{w} &= 0 \\
  \vec{u}\cdot\vec{v}+\vec{u}\cdot\vec{w} &= 0 \\
  \vec{u}\cdot(\vec{v}+\vec{w}) &= 0 \\
  \sum_{i=1}^{n}u_i(v_i+w_i) &= 0 \\
  \sum_{i=1}^{n}u_1\sum_{i=1}^{n}(v_i+w_i) &= 0
\end{align*}

\subsubsection*{Exercise 69}
Prove that \( \vec{u} \) is orthogonal to
\( \vec{v}-\text{proj}_{\vec{u}}\vec{v} \)
for all vectors \( \vec{u} \) and \( \vec{v} \) in \( \R^n \), where
\( \vec{u}\ne\vec{0} \).
\begin{align*}
  \vec{u}\cdot(\vec{v}-\text{proj}_{\vec{u}}\vec{v}) &= \vec{u}\cdot(\vec{v}-
    \frac{\vec{u}\cdot\vec{v}}{\vec{u}\cdot\vec{u}}\vec{u}) \\
  &= \vec{u}\cdot\vec{v}-
    \frac{\vec{u}\cdot\vec{v}}{\vec{u}\cdot\vec{u}}\vec{u}\cdot\vec{u} \\
  &= \vec{u}\cdot\vec{v}-\vec{u}\cdot\vec{v} \\
  &= 0
\end{align*}

\subsubsection*{Exercise 70a}
Prove that \( \text{proj}_{\vec{u}}(\text{proj}_{\vec{u}}\vec{v}) =
\text{proj}_{\vec{u}}\vec{v} \).
\begin{align*}
  \vec{w} &= \text{proj}_{\vec{u}}\vec{v} \\
  &= \frac{\vec{u}\cdot\vec{v}}{\vec{u}\cdot\vec{u}}\vec{u} \\
  \text{proj}_{\vec{u}}(\text{proj}_{\vec{u}}\vec{v}) &= \text{proj}_{\vec{u}}
    \frac{\vec{u}\cdot\vec{v}}{\vec{u}\cdot\vec{u}}\vec{u} \\
  &= \text{proj}_{\vec{u}}\vec{w} \\
  &= \frac{\vec{u}\cdot\vec{w}}{\vec{u}\cdot\vec{u}}\vec{u} \\
  &= \frac{\vec{u}\cdot\frac{\vec{u}\cdot\vec{v}}{\vec{u}\cdot\vec{u}}\vec{u}}
    {\vec{u}\cdot\vec{u}}\vec{u} \\
  &= \frac{\vec{u}\cdot\vec{v}}{\vec{u}\cdot\vec{u}}
    \frac{\vec{u}\cdot\vec{u}}{\vec{u}\cdot\vec{u}}\vec{u} \\
  &= \frac{\vec{u}\cdot\vec{v}}{\vec{u}\cdot\vec{u}}\vec{u} \\
  &= \text{proj}_{\vec{u}}\vec{v}
\end{align*}

\subsubsection*{Exercise 70b}
Prove that \( \text{proj}_{\vec{u}}(\vec{v}-
\text{proj}_{\vec{u}}\vec{v}) = 0 \).
\begin{align*}
  \vec{w} &= \text{proj}_{\vec{u}}\vec{v} \\
  &= \frac{\vec{u}\cdot\vec{v}}{\vec{u}\cdot\vec{u}}\vec{u} \\
  \text{proj}_{\vec{u}}(\vec{v}-\text{proj}_{\vec{u}}\vec{v}) &=
    \text{proj}_{\vec{u}}(\vec{v}-\vec{w}) \\
  &= \frac{\vec{u}\cdot(\vec{v}-\vec{w})}{\vec{u}\cdot\vec{u}}\vec{u} \\
  &= \frac{\vec{u}\cdot\vec{v}-\vec{u}\cdot\vec{w}}
    {\vec{u}\cdot\vec{u}}\vec{u} \\
  &= \frac{\vec{u}\cdot\vec{v}-\vec{u}\cdot
    \frac{\vec{u}\cdot\vec{v}}{\vec{u}\cdot\vec{u}}\vec{u}}
    {\vec{u}\cdot\vec{u}}\vec{u} \\
  &= \left(\frac{\vec{u}\cdot\vec{v}}{\vec{u}\cdot\vec{u}}-
    \frac{\vec{u}\cdot\vec{v}}{\vec{u}\cdot\vec{u}}
    \frac{\vec{u}\cdot\vec{u}}{\vec{u}\cdot\vec{u}}\right)\vec{u} \\
  &= \left(\frac{\vec{u}\cdot\vec{v}}{\vec{u}\cdot\vec{u}}-
    \frac{\vec{u}\cdot\vec{v}}{\vec{u}\cdot\vec{u}}\right)\vec{u} \\
  &= 0\vec{u} = 0
\end{align*}

\subsubsection*{Exercise 73}
Use the fact that \( \text{proj}_{\vec{u}}\vec{v} = c\vec{u} \) for some scalar
\( c \) together with Figure 1.41 to find \( c \) and derive the formula for
\( \text{proj}_{\vec{u}}\vec{v} \).
\begin{align*}
  c\vec{u}\cdot(\vec{v}-c\vec{u}) &= 0 \\
  (c\vec{u}\cdot\vec{v})-(c\vec{u}\cdot c\vec{u}) &= 0 \\
  c(\vec{u}\cdot\vec{v}) &= |c|^2(\vec{u}\cdot\vec{u}) \\
  \frac{\vec{u}\cdot\vec{v}}{\vec{u}\cdot\vec{u}} &= |c| \\
  \text{proj}_{\vec{u}}\vec{v} &= c\vec{u} =
    \frac{\vec{u}\cdot\vec{v}}{\vec{u}\cdot\vec{u}}\vec{u}
\end{align*}

\subsubsection*{Exercise 74}
Using mathematical induction, prove the following generalization of the
Triangle Inequality:
\[ \|\vec{v_1}+\vec{v_2}+\dots+\vec{v_n}\| \le
  \|\vec{v_1}\|+\|\vec{v_2}\|+\dots+\|\vec{v_n}\| \]
Basis:
\[ \|\vec{u}+\vec{v}\| \le \|\vec{u}\|+\|\vec{v}\| \]
Assumption:
\[ \|\vec{v_1}+\vec{v_2}+\dots+\vec{v_n}\| \le
  \|\vec{v_1}\|+\|\vec{v_2}\|+\dots+\|\vec{v_n}\| \]
Induction:
\begin{align*}
  \|\vec{v_1}+\vec{v_2}+\dots+\vec{v_n}+\vec{v_{n+1}}\| &\le
    \|\vec{v_1}+\vec{v_2}+\dots+\vec{v_n}\|+\|\vec{v_{n+1}}\| \\
  &\le \|\vec{v_1}\|+\|\vec{v_2}\|+\dots+\|\vec{v_n}\|+\|\vec{v_{n+1}}\| \\
\end{align*}

\section*{Section 1.3}

\subsubsection*{Exercise 1}
Write the equation of the line passing through \( P \) with normal vector
\( \vec{n} \) in normal form and general form.
\[ P = (0,0) \quad \vec{n} = \begin{bmatrix}3 \\ 2\end{bmatrix} \]
\[ \begin{bmatrix}3 \\ 2\end{bmatrix}\cdot\begin{bmatrix}x \\ y\end{bmatrix} =
  \begin{bmatrix}0 \\ 0\end{bmatrix}\cdot\begin{bmatrix}3 \\ 2\end{bmatrix} \]
\[ 3x+2y = 0 \]

\subsubsection*{Exercise 3}
Write the equation of the line passing through \( P \) with direction vector
\( \vec{d} \) in vector form and parametric form.
\[ P = (1,0) \quad \vec{d} = \begin{bmatrix}-1 \\ 3\end{bmatrix} \]
\[ \begin{bmatrix}x \\ y\end{bmatrix} =
  \begin{bmatrix}1 \\ 0\end{bmatrix}+t\begin{bmatrix}-1 \\ 3\end{bmatrix} \]
\[ l = \begin{cases}
  x &= 1-t \\
  y &= 3t
\end{cases} \]

\subsubsection*{Exercise 5}
Write the equation of the line passing through \( P \) with direction vector
\( \vec{d} \) in vector form and parametric form.
\[ P = (0,1,0) \quad \vec{d} = \begin{bmatrix}1 \\ -1 \\ 4\end{bmatrix} \]
\[ \begin{bmatrix}x \\ y \\ z\end{bmatrix} =
  \begin{bmatrix}0 \\ 1 \\ 0\end{bmatrix}+
  t\begin{bmatrix}1 \\ -1 \\ 4\end{bmatrix} \]
\[ l = \begin{cases}
  x &= t \\
  y &= 1-t \\
  z &= 4t
\end{cases} \]

\subsubsection*{Exercise 7}
Write the equation of the line passing through \( P \) with normal vector
\( \vec{n} \) in normal form and general form.
\[ P = (0,1,0) \quad \vec{n} = \begin{bmatrix}3 \\ 2 \\ 1\end{bmatrix} \]
\[ \begin{bmatrix}3 \\ 2 \\ 1\end{bmatrix}\cdot
  \begin{bmatrix}x \\ y \\ z\end{bmatrix} =
  \begin{bmatrix}3 \\ 2 \\ 1\end{bmatrix}\cdot
  \begin{bmatrix}0 \\ 1 \\ 0\end{bmatrix} \]
\[ 3x+2y+z = 2 \]

\subsubsection*{Exercise 9}
Write the equation of the plane passing through \( P \) with vectors \( \vec{u}
\) and \( \vec{v} \) in vector form and parametric form.
\[ P = (0,0,0) \quad \vec{u} = \begin{bmatrix}2 \\ 1 \\ 2\end{bmatrix} \quad
  \vec{v} = \begin{bmatrix}-3 \\ 2 \\ 1\end{bmatrix} \]
\[ \begin{bmatrix}x \\ y \\ z\end{bmatrix} =
  \begin{bmatrix}0 \\ 0 \\ 0\end{bmatrix}+
  s\begin{bmatrix}2 \\ 1 \\ 2\end{bmatrix}+
  t\begin{bmatrix}-3 \\ 2 \\ 1\end{bmatrix} \]
\[ \Pi = \begin{cases}
  x &= 2s-t3 \\
  y &= s+2t \\
  z &= 2s+t
\end{cases} \]

\subsubsection*{Exercise 11}
Give the vector equation of the line passing through \( P \) and \( Q \).
\[ P = (1,-2) \quad Q = (3,0) \]
\[ \vec{d} = \begin{bmatrix}2 \\ 2\end{bmatrix} \]
\[ \begin{bmatrix}x \\ y\end{bmatrix} = \begin{bmatrix}3 \\ 0\end{bmatrix}+
  t\begin{bmatrix}2 \\ 2\end{bmatrix} \]

\subsubsection*{Exercise 13}
Give the vector equation of the plane passing through \( P, Q, R \).
\[ P = (1,1,1) \quad Q = (4,0,2) \quad R = (0,1,-1) \]
\begin{align*}
  \vec{u} &= \begin{bmatrix}3 \\ -1 \\ 1\end{bmatrix} \\
  \vec{v} &= \begin{bmatrix}-1 \\ 0 \\ -2\end{bmatrix} \\
  \begin{bmatrix}x \\ y \\ z\end{bmatrix} &=
    \begin{bmatrix}1 \\ 1 \\ 1\end{bmatrix}+
    s\begin{bmatrix}3 \\ -1 \\ 1\end{bmatrix}+
    t\begin{bmatrix}-1 \\ 0 \\ -2\end{bmatrix}
\end{align*}

\subsubsection*{Exercise 15}
Find parametric equations and an equation in vector form for the lines in
\( \R^2 \) with the following equations:
\[ y = 3x-1 \]
\[ l = \begin{cases}
  x &= t \\
  y &= 3t-1
\end{cases} \]
\[ \begin{bmatrix}x \\ y\end{bmatrix} = \begin{bmatrix}0 \\ -1\end{bmatrix}+
  t\begin{bmatrix}1 \\ 3\end{bmatrix} \]

\subsubsection*{Exercise 17}
Suggest a ``vector proof'' of the fact that in \( \R^2 \), two lines with slope
\( m_1 \) and \( m_2 \) are perpendicular if and only if \( m_1m_2 = -1 \).
\begin{align*}
  \vec{u} &= \langle m_1,1\rangle \\
  \vec{v} &= \langle m_2,1\rangle \\
  \vec{u}\cdot\vec{v} &= 0 \\
  m_1m_2+1 &= 0 \\
  m_1m_2 &= -1
\end{align*}

\subsubsection*{Exercise 19}
The plane \( P_1 \) has the equation \( 4x-y+5z = 2 \). For each
of the planes \( P \) in Exercise 18. determine whether \( P_1 \) and \( P \)
are parallel, perpendicular, or neither.
\begin{align*}
  \vec{n_1} &= \langle4,-1,5\rangle \\
  a: 2x+3y-z &= 1 \\
  \vec{n_a} &= \langle2,3,-1\rangle \\
  \vec{n_1} &\ne c\vec{n_a} \\
  \vec{n_1}\cdot\vec{n_a} &= 8+(-3)+(-5) = 0 \quad &\text{Perpendicular} \\
  b: 4x-y+5z &= 0 \\
  \vec{n_b} &= \langle4,-1,5\rangle \\
  \vec{n_1} &= c\vec{n_a} \quad &\text{Parallel} \\
  c: x-y-z &= 3 \\
  \vec{n_c} &= \langle1,-1,-1\rangle \\
  \vec{n_1} &\ne c\vec{n_c} \\
  \vec{n_1}\cdot\vec{n_c} &= 4+1+(-5) = 0 \quad &\text{Perpendicular} \\
  d: 4x+6y-2z &= 0 \\
  \vec{n_d} &= \langle4,6,-2\rangle = \langle2,3,-1\rangle \\
  \vec{n_1} &\ne c\vec{n_d} \\
  \vec{n_1}\cdot\vec{n_d} &= 8+(-3)+(-5) = 0 \quad &\text{Perpendicular}
\end{align*}

\subsubsection*{Exercise 21}
Find the vector form of the equation of the line in \( \R^2 \) that passes
through \( P = (2,-1) \) and is parallel to the line with general equation
\( 2x-3y = 1 \).
\[ \vec{d} = \langle2,3\rangle \]
\[ \begin{bmatrix}x \\ y\end{bmatrix} =
  \begin{bmatrix}2 \\ -1\end{bmatrix}+
  t\begin{bmatrix}2 \\ 3\end{bmatrix} \]

\subsubsection*{Exercise 23}
Find the vector form of the equation of the line in \( \R^3 \) that passes
through \( P = (-1,0,3) \) and is parallel to the line with parametric
equations:
\[ l = \begin{cases}
  x &= 1-t \\
  y &= 2+3t \\
  z &= -2-t
\end{cases} \]
\[ \begin{bmatrix}x \\ y \\ z\end{bmatrix} =
  \begin{bmatrix}-1 \\ 0 \\ 3\end{bmatrix}+
  t\begin{bmatrix}-1 \\ 3 \\ -1\end{bmatrix} \]

\section*{Section 2.1}

\subsubsection*{Exercise 1}
Determine which equations are linear in the variables \( x,y,z \).
\[ x-\pi y+\sqrt[3]{5}z = 0 \]
Linear.

\subsubsection*{Exercise 3}
Determine which equations are linear in the variables \( x,y,z \).
\[ x^{-1}+7y+z = \sin\frac{\pi}{9} \]
Nonlinear, \( x \) is not 1st degree.

\subsubsection*{Exercise 5}
\[ 3\cos(x)-4y+z = \sqrt{3} \]
Nonlinear, \( \cos(x) \) is a function of \( x \).

\subsubsection*{Exercise 11}
Find the solution set of each equation.
\[ 3x-6y = 0 \]
All points that lie on the line \( y = \frac{1}{2}x \).

\subsubsection*{Exercise 13}
Find the solution set of each equation.
\[ x+2y+3z = 4 \]
All points that lie on the plane \( x+2y+3z = 4 \) or \( [4-3t-2s,s,t] \).

\subsubsection*{Exercise 15}
Determine geometrically whether each system has a unique solution, infinitely
many solutions, or no solution. Then solve each system algebraically to confirm
your answer.
\[ x+y = 0 \quad 2x+y = 3 \]
Two intersecting lines, one solution at (3,-3).

\subsubsection*{Exercise 17}
Determine geometrically whether each system has a unique solution, infinitely
many solutions, or no solution. Then solve each system algebraically to confirm
your answer.
\[ 3x-6y = 3 \quad -x+2y = 1 \]
Two parallel lines, no solution.

\subsubsection*{Exercise 19}
Solve the given system by back substitution.
\begin{align*}
  x-2y &= 1 \\
  y &= 3 \\
  x &= 7
\end{align*}

\subsubsection*{Exercise 21}
Solve the given system by back substitution.
\begin{align*}
  x-y+z &= 0 \\
  2y-z &= 1 \\
  3z &= -1 \\
  z &= -\frac{1}{3} \\
  y &= \frac{1}{3} \\
  x &= \frac{2}{3}
\end{align*}

\subsubsection*{Exercise 23}
Solve the given system by back substitution.
\begin{align*}
  x_1+x_2-x_3-x_4 &= 1 \\
  x_2+x_3+x_4 &= 0 \\
  x_3-x_4 &= 0 \\
  x_4 &= 1 \\
  x_3 &= 1 \\
  x_2 &= -2 \\
  x_1 &= 1
\end{align*}

\subsubsection*{Exercise 25}
Solve these systems.
\begin{align*}
  x &= 2 \\
  2x+y &= -3 \\
  -3x-4y+z &= -1 \\
  y &= -7 \\
  z &= -23
\end{align*}

\subsubsection*{Exercise 27}
Find the augmented matrices of the linear systems.
\begin{align*}
  x-y &= 0 \\
  2x+y &= 3 \\
  A &= \left[\begin{array}{cc|c}
    1 & -1 & 0 \\
    2 & 1 & 3
  \end{array}\right]
\end{align*}

\subsubsection*{Exercise 29}
Find the augmented matrices of the linear systems.
\begin{align*}
  x+5y &= -1 \\
  -x+y &= -5 \\
  2x+4y &= 4 \\
  A &= \left[\begin{array}{cc|c}
    1 & 5 & -1 \\
    -1 & 1 & -5 \\
    2 & 4 & 4
  \end{array}\right]
\end{align*}

\subsubsection*{Exercise 31}
Find a system of linear equations that has the given matrix as its augmented
matrix.
\begin{align*}
  A &= \left[\begin{array}{ccc|c}
    0 & 1 & 1 & 1 \\
    1 & -1 & 0 & 1 \\
    2 & -1 & 1 & 1
  \end{array}\right] \\
  y+z &= 1 \\
  x-y &= 1 \\
  2x-y+z &= 1
\end{align*}

\subsubsection*{Exercise 33}
Solve the linear systems in the given equations.
\begin{align*}
  x-y &= 0 \\
  2x+y &= 3 \\
  A &= \left[\begin{array}{cc|c}
    1 & -1 & 0 \\
    2 & 1 & 3
  \end{array}\right] \\
  -2R_1+R_2 &\to R_2 \\
  A &= \left[\begin{array}{cc|c}
    1 & -1 & 0 \\
    0 & 3 & 3
  \end{array}\right] \\
  \frac{1}{3}R_2 &\to R_2 \\
  A &= \left[\begin{array}{cc|c}
    1 & -1 & 0 \\
    0 & 1 & 1
  \end{array}\right] \\
  R_2+R_1 &\to R_1 \\
  A &= \left[\begin{array}{cc|c}
    1 & 0 & 1 \\
    0 & 1 & 1
  \end{array}\right] \\
  x &= 1 \\
  y &= 1
\end{align*}

\subsubsection*{Exercise 35}
Solve the linear systems in the given equations.
\begin{align*}
  x+5y &= -1 \\
  -x+y &= -5 \\
  2x+4y &= 4 \\
  A &= \left[\begin{array}{cc|c}
    1 & 5 & -1 \\
    -1 & 1 & -5 \\
    2 & 4 & 4
  \end{array}\right] \\
  R_1+R_2 &\to R_2 \\
  -2R_1+R_3 &\to R_3 \\
  A &= \left[\begin{array}{cc|c}
    1 & 5 & -1 \\
    0 & 6 & -6 \\
    0 & -6 & 6
  \end{array}\right] \\
  \frac{1}{6}R_2 &\to R_2 \\
  \frac{1}{6}R_3 &\to R_3 \\
  A &= \left[\begin{array}{cc|c}
    1 & 5 & -1 \\
    0 & 1 & -1 \\
    0 & -1 & 1
  \end{array}\right] \\
  5R_3+R_1 &\to R_1 \\
  A &= \left[\begin{array}{cc|c}
    1 & 0 & 4 \\
    0 & 1 & -1 \\
    0 & -1 & 1
  \end{array}\right] \\
  x &= 4 \\
  y &= -1
\end{align*}

\subsubsection*{Exercise 37}
Solve the linear systems in the given equations.
\begin{align*}
  A &= \left[\begin{array}{ccc|c}
    0 & 1 & 1 & 1 \\
    1 & -1 & 0 & 1 \\
    2 & -1 & 1 & 1
  \end{array}\right] \\
  R_2 &\leftrightarrow R_1 \\
  &= \left[\begin{array}{ccc|c}
    1 & -1 & 0 & 1 \\
    0 & 1 & 1 & 1 \\
    2 & -1 & 1 & 1
  \end{array}\right] \\
  -2R_1+R_3 &\to R_3 \\
  &= \left[\begin{array}{ccc|c}
    1 & -1 & 0 & 1 \\
    0 & 1 & 1 & 1 \\
    0 & 1 & 1 & -1
  \end{array}\right] \\
  R_3-R_2 &\to R_3 \\
  &= \left[\begin{array}{ccc|c}
    1 & -1 & 0 & 1 \\
    0 & 1 & 1 & 1 \\
    0 & 0 & 0 & 2
  \end{array}\right] \\
  0 &= 2 \quad \text{(Inconsistent)}
\end{align*}

\subsubsection*{Exercise 39}
Find a system of two linear equations in the variables \( x \) and \( y \)
whose solution set is given by the parametric equations \( x = t \) and
\( y = 3-2t \).
\[ y = 3-2x \]
\[ 3x = 15 \]
Find another parametric solution to the system in the first part in which
the parameter is \( s \) and \( y = s \).
\[ y = s \]
\[ x = \frac{s-3}{-2} \]

\begin{center}
  If you have any questions, comments, or concerns, please contact me at
  alvin@omgimanerd.tech
\end{center}

\end{document}
