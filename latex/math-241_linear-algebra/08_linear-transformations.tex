\documentclass{math}

\title{Linear Algebra}
\author{Alvin Lin}
\date{August 2017 - December 2017}

\begin{document}

\maketitle

\section*{Linear Transformations}
\( T:\R^n\to\R^m \) is a transformation if for each \( \vec{x}\in\R^n~\exists!~
T(\vec{x})\in\R^m \). \( Range(T) = Image(T) = \{T(\vec{x})\mid\vec{x}\in\R^n\}
\). \( T:\R^n\to\R^m \) is one-to-one if \( \forall\vec{x},\vec{y}\in\R^n \),
\[ T(\vec{x}) = T(\vec{y}) \longrightarrow \vec{x} = \vec{u} \]
\[ \vec{x} \ne \vec{y} \longrightarrow T(\vec{x}) \ne \vec{y} \]
\( T:\R^n\to\R^m \) is \textbf{onto} (surjective) if \( Range(T) = Codomain(T)
\).

\subsection*{Properties of Linear Transformations}
Let \( T:\R^n\to\R^m \) be a transformation. \( T \) is a linear transformation
if for all \( \vec{u},\vec{v}\in\R^n \) and all scalars \( c,d \):
\begin{enumerate}
  \item \( T(\vec{u}+\vec{v}) = T(\vec{u})+T(\vec{v}) \)
  \item \( T(c\vec{u}) = cT(\vec{u}) \)
  \item \( T(c\vec{u}+d\vec{v}) = cT(\vec{u})+dT(\vec{u}) \)
\end{enumerate}

\subsubsection*{Example}
Suppose \( T:\R^2\to\R^2 \) is defined by:
\[ T\left(\begin{bmatrix}x \\ y\end{bmatrix}\right) = \begin{bmatrix}3y \\
  -4x\end{bmatrix} \]
Verify that \( T \) is linear. Let:
\[ \vec{u} = \begin{bmatrix}x_1 \\ y_1\end{bmatrix} \quad
  \vec{v} = \begin{bmatrix}x_2 \\ y_2\end{bmatrix} \]
\begin{align*}
  T(\vec{u}+\vec{v}) &=
    T\left(\begin{bmatrix}x_1+x_2 \\ y_1+y_2\end{bmatrix}\right) \\
  &= \begin{bmatrix}
    3(y_1+y_2) \\
    -4(x_1+x_2)
  \end{bmatrix} \\
  &= \begin{bmatrix}
    3y_1+3y_2 \\
    -4x_1-4x_2
  \end{bmatrix} \\
  &= \begin{bmatrix}
    3y_1 \\
    -4x_1
  \end{bmatrix}+\begin{bmatrix}
    3y_2 \\
    -4x_2
  \end{bmatrix} \\
  &= T(\vec{u})+T(\vec{v}) \\
  T(c\vec{u}) &= \left(\begin{bmatrix}cx \\ cy\end{bmatrix}\right) \\
  &= \begin{bmatrix}
    3(cy) \\
    -4(cx)
  \end{bmatrix} \\
  &= c\begin{bmatrix}
    3y \\
    -4x
  \end{bmatrix} \\
  &= cT(\vec{u})
\end{align*}
\( T \) satisfies both properties of a linear transformation. To prove a
transformation is not linear, one only needs to find a single counterexample.

\subsubsection*{Example}
Consider \( T:\R^n\to\R^m \) as a linear transformation. What is \( T(\vec{0})
\)?
\begin{align*}
  T(\vec{0}) &= T(\vec{0}+\vec{0}) \\
  &= T(\vec{0})+T(\vec{0}) \\
  \vec{0} &= T(\vec{0})
\end{align*}

\subsection*{Facts about Linear Transformations}
Suppose \( B = \{\vec{v_1},\vec{v_2},\dots,\vec{v_n}\} \) basis for \( R^n \).
Let \( \vec{x}\in\R^n \).
\begin{align*}
  \vec{x} &= \sum_{i=1}^{n}x_1\vec{v_1} \\
  T(\vec{x}) &= T\left(\sum_{i=1}^{n}x_i\vec{v_i}\right) \\
  &= \sum_{i=1}^{n}x_iT(\vec{v_i})
\end{align*}
Let the standard matrix \( A_T \) stand for the linear transformation \( T \):
\begin{align*}
  A_T &= [A] \\
  &= \begin{bmatrix}
    T(\vec{e_1}) & T(\vec{e_2}) & \dots & T(\vec{e_n})
  \end{bmatrix}
\end{align*}
Let \( S:\R^n\to\R^m \) and \( T:\R^m\to\R^k \). The composition of linear
transformations \( T\circ S:\R^n\to\R^k \) is also linear.
\begin{align*}
  [T] &= B \\
  [S] &= A \\
  (T\circ S)(\vec{x}) &= T(S(\vec{x})) \\
  &= T(A\vec{x}) \\
  &= B(A\vec{x}) \\
  &= (BA)\vec{x}
\end{align*}

\subsubsection*{Example}
Define \( \Pi_x:\R^2\to\R^2 \):
\[ \Pi_x\left(\begin{bmatrix}x \\ y\end{bmatrix}\right) = \begin{bmatrix}x \\
  0\end{bmatrix} \]
\( \Pi_x \) is linear. Find \( [\Pi_x] \):
\begin{align*}
  \Pi_x(\vec{e_1}) &= \begin{bmatrix}1 \\ 0\end{bmatrix} \\
  \Pi_x(\vec{e_2}) &= \begin{bmatrix}0 \\ 1\end{bmatrix} \\
  A &= [\Pi_x] = \begin{bmatrix}
    \Pi_x(\vec{e_1}) & \Pi_x(\vec{e_2})
  \end{bmatrix} \\
  &= \begin{bmatrix}
    1 & 0 \\
    0 & 0
  \end{bmatrix}
\end{align*}

\begin{center}
  You can find all my notes at \url{http://omgimanerd.tech/notes}. If you have
  any questions, comments, or concerns, please contact me at
  alvin@omgimanerd.tech
\end{center}

\end{document}
