\documentclass{math}

\title{Linear Algebra}
\author{Alvin Lin}
\date{August 2017 - December 2017}

\begin{document}

\maketitle

\section*{Final Exam Review}
\begin{enumerate}
  \item Solve a linear system
  \item Like \#8 from Exam 1. Given a 3 by 3 linear system with some
  coefficients \( k \), find all values of \( k \) such that the solution has
  no solutions, 1 solution, or infinitely many solution.
  \item Similar to Exam 2, \#3. A square matrix \( M \) satisfies a given
  polynomial equation. Find \( M^{-1}, rank(M), nullity(M) \).
  \item Given a vector space \( V \) and some subsets of \( V \), determine if
  those subsets are subspaces of \( V \).
  \item Like Exam 2, \#7. Given a 3 by 3 matrix \( A \), with some entries in
  terms of \( k \), find all values of \( k \) making \( A \) invertible.
  \item Given 2 vector spaces \( V \), determine if they are isomorphic.
  \item Compute some determinants.
  \item Similar to Exam 3 \#5
  \item Similar to Exam 3 \#4
  \item A question involving linear transformation \( T:V\to W \) and the
  rank-nullity theorem.
  \item Similar to questions from the Exam 4.
  \item Similar to questions from the Exam 4.
\end{enumerate}
\textbf{Bonus:} Given some vector space \( V \), compute its dimension. \\
\textbf{Bonus:} Given transformation \( T:V\to W \), find
\( [T]_{C\leftarrow B} \) where \( B \) is a basis for \( V \) and \( C \) is a
basis for \( W \).

\subsubsection*{Example}
Given a linear transformation \( T:V\to W \) with \( dim(V) = 5 \) and
\( dim(W) = 6 \). Are there any linear transformations \( T:V\to W \) that are
also onto? No, the rank-nullity theorem states that
\( dim(V) = rank(T)+nullity(T) \).
\[ rank(T) = dim(range(T)) \le dim(V) < dim(W) \]
\[ range(T) \ne W \]
So \( T \) is not onto. \par
Suppose \( dim(V) = 6, dim(W) = 5 \). Are there any one-to-one linear
transformations \( T:V\to W \)?
\begin{align*}
  rank(T) \le dim(W) &= 5 \\
  dim(V) &= rank(T)+nullity(T) \\
  \therefore nullity(T) &\ge 1 \\
  nullity(T) &\ne \{\vec{0}\}
\end{align*}
\( T \) is not one-to-one.

\subsubsection*{Example}
\( V \) is a vector space with subspaces \( U,W \). Prove \( U\cap W \) is a
subspace of \( V \).
\begin{enumerate}
  \item Is \( \vec{0}\in U\cap W \)? \( U \) is a subspace so
  \( \vec{0}\in U \). \( W \) is a subspace so \( \vec{0}\in W \). Therefore,
  \( \vec{0}\in U\cap W \).
  \item Let \( \vec{x},\vec{y}\in U\cap W \). So \( \vec{x},\vec{y}\in U \)
  and \( \vec{x},\vec{y}\in W \). Then \( \vec{x}+\vec{y}\in U \) and
  \( \vec{x}+\vec{y}\in W \). Thus, \( \vec{x}+\vec{y}\in U\cap W \).
  \item Let \( \vec{u}\in U\cap W \) and \( c \) be a scalar. \( \vec{u}\in U \)
  and \( \vec{u}\in W \). Then \( c\vec{u}\in U \) and \( c\vec{u}\in W \).
  Therefore, \( c\vec{u}\in U\cap W \).
\end{enumerate}

\subsubsection*{Example}
Let \( V \) be a vector space. Take \( U,W \) as subspaces of \( V \). Is it
true that \( U\cap W \) is a subspace of \( V \)? Let \( V = \R^2 \).
\( U = \{(x,0)\mid x\in\R\} \) and \( W = \{(0,y)\mid y\in\R\} \). \( U\cap W \)
is the union of the x and y axes, which is not closed under addition. For
\( U\cap W \) to be closed under addition, we need \( U\subseteq W \) or
\( W\subseteq U \).

\subsubsection*{Example}
Say \( B \) is a set of vectors in a vector space \( V \) with the property that
every \( \vec{v}\in V \) can be written uniquely as a linear combination of
elements of \( B \). Explain why \( B \) is a basis for \( V \). Show that the
\( span(B) = V \).
\begin{align*}
  \text{Take }\vec{v} &\in V \\
  \vec{v} &= \sum_{i=1}^{n}c_i\vec{v_i} \\
  span(B) = V
\end{align*}
Show that \( B \) is linearly independent.
\[ \sum_{i=1}^{n}c_i\vec{v_i} = \vec{0} = 0\vec{v_1}+0\vec{v_2}+\dots+
  0\vec{v_n} \]
By uniqueness, \( c_1 = c_2 = \dots = c_n = 0 \). Thus, \( B \) is linearly
independent, so \( B \) is a basis for \( V \).

\begin{center}
  You can find all my notes at \url{http://omgimanerd.tech/notes}. If you have
  any questions, comments, or concerns, please contact me at
  alvin@omgimanerd.tech
\end{center}

\end{document}
