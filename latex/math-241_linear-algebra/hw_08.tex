\documentclass{math}

\usepackage{enumerate}

\geometry{letterpaper, margin=0.5in}

\title{Linear Algebra: Homework 8}
\author{Alvin Lin}
\date{August 2016 - December 2016}

\begin{document}

\maketitle

\section*{Section 4.1}

\subsubsection*{Exercise 1}
Show that \( \vec{v} \) is an eigenvector of \( A \) and find the corresponding
eigenvalue.
\begin{align*}
  A &= \begin{bmatrix}0 & 3 \\ 3 & 0\end{bmatrix} \\
  \vec{v} &= \begin{bmatrix}1 \\ 1\end{bmatrix} \\
  A\vec{v} &= \begin{bmatrix}3 \\ 3\end{bmatrix} \\
  &= 3\vec{v} \\
  \lambda &= 3
\end{align*}

\subsubsection*{Exercise 3}
Show that \( \vec{v} \) is an eigenvector of \( A \) and find the corresponding
eigenvalue.
\begin{align*}
  A &= \begin{bmatrix}-1 & 1 \\ 6 & 0\end{bmatrix} \\
  \vec{v} &= \begin{bmatrix}1 \\ -2\end{bmatrix} \\
  A\vec{v} &= \begin{bmatrix}-3 \\ 6\end{bmatrix} \\
  &= -3\vec{v} \\
  \lambda &= -3
\end{align*}

\subsubsection*{Exercise 5}
Show that \( \vec{v} \) is an eigenvector of \( A \) and find the corresponding
eigenvalue.
\begin{align*}
  A &= \begin{bmatrix}
    3 & 0 & 0 \\
    0 & 1 & -2 \\
    1 & 0 & 1
  \end{bmatrix} \\
  \vec{v} &= \begin{bmatrix}2 \\ -1 \\ 1\end{bmatrix} \\
  A\vec{v} &= \begin{bmatrix}6 \\ -3 \\ 3\end{bmatrix} \\
  &= 3\vec{v} \\
  \lambda &= 3
\end{align*}

\subsubsection*{Exercise 7}
Show that \( \lambda \) is an eigenvalue of \( A \) and find one eigenvector
corresponding to this eigenvalue.
\begin{align*}
  A &= \begin{bmatrix}2 & 2 \\ 2 & -1\end{bmatrix} \\
  \lambda &= 3 \\
  A-\lambda I &= \begin{bmatrix}-1 & 2 \\ 2 & -4\end{bmatrix} \\
  \left[\begin{array}{cc|c}
    -1 & 2 & 0 \\
    2 & -4 & 0
  \end{array}\right] &= \begin{bmatrix}
    -1 & 2 & 0 \\
    0 & 0 & 0
  \end{bmatrix} \\
  x_1 &= 2x_2 \\
  \vec{x} &= span\left(\begin{bmatrix}2 \\ 1\end{bmatrix}\right)
\end{align*}
One possible eigenvector is \( \begin{bmatrix}2 \\ 1\end{bmatrix} \).

\subsubsection*{Exercise 9}
Show that \( \lambda \) is an eigenvalue of \( A \) and find one eigenvector
corresponding to this eigenvalue.
\begin{align*}
  A &= \begin{bmatrix}0 & 4 \\ -1 & 5\end{bmatrix} \\
  \lambda &= 1 \\
  A-\lambda I &= \begin{bmatrix}-1 & 4 \\ -1 & 4\end{bmatrix} \\
  \left[\begin{array}{cc|c}
    -1 & 4 & 0 \\
    -1 & 4 & 0
  \end{array}\right] &= \begin{bmatrix}
    -1 & 4 & 0 \\
    0 & 0 & 0
  \end{bmatrix} \\
  4x_2 &= x_1 \\
  \vec{x} &= span\left(\begin{bmatrix}4 \\ 1\end{bmatrix}\right)
\end{align*}
One possible eigenvector is \( \begin{bmatrix}4 \\ 1\end{bmatrix} \).

\subsubsection*{Exercise 11}
Show that \( \lambda \) is an eigenvalue of \( A \) and find one eigenvector
corresponding to this eigenvalue.
\begin{align*}
  A &= \begin{bmatrix}
    1 & 0 & 2 \\
    -1 & 1 & 1 \\
    2 & 0 & 1
  \end{bmatrix} \\
  \lambda &= -1 \\
  A-\lambda I &= \begin{bmatrix}
    2 & 0 & 2 \\
    -1 & 2 & 1 \\
    2 & 0 & 2
  \end{bmatrix} \\
  \left[\begin{array}{ccc|c}
    2 & 0 & 2 & 0 \\
    -1 & 2 & 1 & 0 \\
    2 & 0 & 2 & 0
  \end{array}\right] &= \begin{bmatrix}
    1 & 0 & 1 & 0 \\
    0 & 2 & 2 & 0 \\
    0 & 0 & 0 & 0
  \end{bmatrix} \\
  &= \begin{bmatrix}
    1 & 0 & 1 & 0 \\
    0 & 1 & 1 & 0 \\
    0 & 0 & 0 & 0
  \end{bmatrix} \\
  x_1 &= -x_3 = x_2 \\
  \vec{x} &= span\left(\begin{bmatrix}1 \\ 1 \\ -1\end{bmatrix}\right)
\end{align*}
One possible eigenvector is \( \begin{bmatrix}1 \\ 1 \\ -1\end{bmatrix} \).

\subsubsection*{Exercise 27}
Find all of the eigenvalues of the matrix \( A \) over the complex numbers
\( \mathbb{C} \). Give bases for each of the corresponding eigenspaces.
\begin{align*}
  A &= \begin{bmatrix}1 & 1 \\ -1 & 1\end{bmatrix} \\
  |A-\lambda I| &= \begin{vmatrix}
    1-\lambda & 1 \\
    -1 & 1-\lambda
  \end{vmatrix} \\
  &= (1-\lambda)(1-\lambda)+1 \\
  &= 1-2\lambda+\lambda^2+1 \\
  &= \lambda^2-2\lambda+2 = 0 \\
  \lambda &= \frac{2\pm\sqrt{4-8}}{2} \\
  &= 1\pm i \\
\end{align*}
\begin{align*}
  A-(1+i)I &= \begin{bmatrix} -i & 1 \\ -1 & i \end{bmatrix} \\
  \left[\begin{array}{cc|c}
    -i & 1 & 0 \\
    -1 & -i & 0
  \end{array}\right] &= \begin{bmatrix}
    -i & 1 & 0 \\
    -i & 1 & 0
  \end{bmatrix} \\
  x_2 &= ix_1 \\
  E_{1+i} &= span\left(\begin{bmatrix}i \\ 1\end{bmatrix}\right) \\
  A-(1-i)I &= \begin{bmatrix} i & 1 \\ -1 & i \end{bmatrix} \\
  \left[\begin{array}{cc|c}
    i & 1 & 0 \\
    -1 & i & 0
  \end{array}\right] &= \begin{bmatrix}
    i & 1 & 0 \\
    -i & -1 & 0
  \end{bmatrix} \\
  ix_1 &= -1x_2 \\
  E_{1-i} &= span\left(\begin{bmatrix}-i \\ 1\end{bmatrix}\right)
\end{align*}

\subsubsection*{Exercise 29}
Find all of the eigenvalues of the matrix \( A \) over the complex numbers
\( \mathbb{C} \). Give bases for each of the corresponding eigenspaces.
\begin{align*}
  A &= \begin{bmatrix}1 & i \\ i & 1\end{bmatrix} \\
  |A-\lambda I| &= \begin{vmatrix}
    1-\lambda & i \\
    i & 1-\lambda
  \end{vmatrix} \\
  &= (1-\lambda)^2-(-1) \\
  &= 1-2\lambda+\lambda^2+1 = 0 \\
  \lambda &= 1\pm i \\
  A-(1+i)I &= \begin{bmatrix}-i & i \\ i & -i\end{bmatrix} \\
  \left[\begin{array}{cc|c}
    -i & i & 0 \\
    i & -i & 0
  \end{array}\right] &= \begin{bmatrix}
    1 & -1 & 0 \\
    0 & 0 & 0
  \end{bmatrix} \\
  x_1 &= x_2 \\
  E_{1+i} &= span\left(\begin{bmatrix}1 \\ 1\end{bmatrix}\right) \\
  A-(1-i)I &= \begin{bmatrix}i & i \\ i & i\end{bmatrix} \\
  \left[\begin{array}{cc|c}
    i & i & 0 \\
    i & i & 0
  \end{array}\right] &= \begin{bmatrix}
    -1 & -1 & 0 \\
    0 & 0 & 0
  \end{bmatrix} \\
  x_1 &= -x_2 \\
  E_{1-i} &= span\left(\begin{bmatrix}-1 \\ 1\end{bmatrix}\right)
\end{align*}

\section*{Section 4.2}

\subsubsection*{Exercise 1}
Compute the determinant using cofactor expansion along the first row and column.
\begin{align*}
  \begin{vmatrix}
    1 & 0 & 3 \\
    5 & 1 & 1 \\
    0 & 1 & 2
  \end{vmatrix} &= 1\begin{vmatrix}1 & 1 \\ 1 & 2\end{vmatrix}-
    0+3\begin{vmatrix}5 & 1 \\ 0 & 1\end{vmatrix} \\
  &= 1(2-1)+3(5) \\
  &= 16 \\
  &= 1\begin{vmatrix}1 & 1 \\ 1 & 2\end{vmatrix}-
    5\begin{vmatrix}0 & 3 \\ 1 & 2\end{vmatrix}+0 \\
  &= 1(2-1)-5(0-3) \\
  &= 16
\end{align*}

\subsubsection*{Exercise 3}
Compute the determinant using cofactor expansion along the first row and column.
\begin{align*}
  \begin{vmatrix}
    1 & -1 & 0 \\
    -1 & 0 & 1 \\
    0 & 1 & -1
  \end{vmatrix} &= 1\begin{vmatrix}0 & 1 \\ 1 & -1\end{vmatrix}-
    (-1)\begin{vmatrix}-1 & 1 \\ 0 & -1\end{vmatrix}+0 \\
  &= 1(-1)+1(1) \\
  &= 0
\end{align*}
Since the matrix is symmetric, the expansion is the same along the first row and
column.

\subsubsection*{Exercise 5}
Compute the determinant using cofactor expansion along the first row and column.
\begin{align*}
  \begin{vmatrix}
    1 & 2 & 3 \\
    2 & 3 & 1 \\
    3 & 1 & 2
  \end{vmatrix} &= 1\begin{vmatrix}3 & 1 \\ 1 & 2\end{vmatrix}-
    2\begin{vmatrix}2 & 1 \\ 3 & 2\end{vmatrix}+
    3\begin{vmatrix}2 & 3 \\ 3 & 1\end{vmatrix} \\
  &= 1(6-1)-2(4-3)+3(2-9) \\
  &= 5-2-21 \\
  &= -18
\end{align*}
Since the matrix is symmetric, the expansion is the same along the first row and
column.

\subsubsection*{Exercise 7}
Compute the determinant using cofactor expansion along any row or column that
seems convenient.
\begin{align*}
  \begin{vmatrix}
    5 & 2 & 2 \\
    -1 & 1 & 2 \\
    3 & 0 & 0
  \end{vmatrix} &= 3\begin{vmatrix}2 & 2 \\ 1 & 2\end{vmatrix}-0+0 \\
  &= 3(4-2) \\
  &= 6
\end{align*}

\subsubsection*{Exercise 9}
Compute the determinant using cofactor expansion along any row or column that
seems convenient.
\begin{align*}
  \begin{vmatrix}
    -4 & 1 & 3 \\
    2 & -2 & 4 \\
    1 & -1 & 0
  \end{vmatrix} &= 1\begin{vmatrix}1 & 3 \\ -2 & 4\end{vmatrix}-
    (-1)\begin{vmatrix}-4 & 3 \\ 2 & 4\end{vmatrix}+0 \\
  &= 1(4+6)+1(-16-6) \\
  &= -12
\end{align*}

\subsubsection*{Exercise 11}
Compute the determinant using cofactor expansion along any row or column that
seems convenient.
\begin{align*}
  \begin{vmatrix}
    a & b & 0 \\
    0 & a & b \\
    a & 0 & b
  \end{vmatrix} &= a\begin{vmatrix}a & b \\ 0 & b\end{vmatrix}-
    b\begin{vmatrix}0 & b \\ a & b\end{vmatrix}+0 \\
  &= a(ab)-b(-ba) \\
  &= a^2b+ab^2 \\
  &= ab(a+b)
\end{align*}

\subsubsection*{Exercise 13}
Compute the determinant using cofactor expansion along any row or column that
seems convenient.
\begin{align*}
  \begin{vmatrix}
    1 & -1 & 0 & 3 \\
    2 & 5 & 2 & 6 \\
    0 & 1 & 0 & 0 \\
    1 & 4 & 2 & 1
  \end{vmatrix} &= 0-1\begin{vmatrix}
    1 & 0 & 3 \\
    2 & 2 & 6 \\
    1 & 2 & 1
  \end{vmatrix}+0-0 \\
  &= -1\left(1\begin{vmatrix}2 & 6 \\ 2 & 1\end{vmatrix}-0+
    3\begin{vmatrix}2 & 2 \\ 1 & 2\end{vmatrix}\right) \\
  &= -1(1(2-12)+3(4-2)) \\
  &= 4
\end{align*}

\subsubsection*{Exercise 15}
Compute the determinant using cofactor expansion along any row or column that
seems convenient.
\begin{align*}
  \begin{vmatrix}
    0 & 0 & 0 & a \\
    0 & 0 & b & c \\
    0 & d & e & f \\
    g & h & i & j
  \end{vmatrix} &= 0-0+0-a\begin{vmatrix}
    0 & 0 & b \\
    0 & d & e \\
    g & h & i
  \end{vmatrix} \\
  &= -a(0-0+b\begin{vmatrix}0 & d \\ g & h\end{vmatrix}) \\
  &= -ab(-dg) \\
  &= abdg
\end{align*}

\subsubsection*{Exercise 21}
Prove Theorem 4.2: The determinant of a triangular matrix is the product of the
entries on its main diagonal. Specifically, if \( A = [a_{ij}] \) is an
\( n\times n \) triangular matrix, then
\[ det(A) = a_{11}a_{22}\cdot\cdot\cdot a_{nn} \]
Base Case:
\[ |a_{11}| = a_{11} \]
Induction Hypothesis:
\[ det(A) = a_{11}a_{22}a_{33}\cdot\cdot\cdot a_{nn} \]
Induction:
\begin{align*}
  \begin{vmatrix}
    a_{11} & 0 & \dots & 0 \\
    a_{21} & a_{22} & \dots & 0 \\
    \vdots & \vdots & \vdots & 0 \\
    a_{(n+1)1} & a_{(n+1)2} & \dots & a_{(n+1)(n+1)}
  \end{vmatrix} &= a_{11}\begin{vmatrix}
    a_{22} & \dots & 0 \\
    \vdots & \vdots & 0 \\
    a_{(n+1)2} & \dots & a_{(n+1)(n+1)}
  \end{vmatrix} \\
  &= a_{11}a_{22}a_{33}\cdot\cdot\cdot a_{nn}a_{(n+1)(n+1)}
\end{align*}

\subsubsection*{Exercise 23}
Evaluate the given determinant using elementary row and/or column operations and
Theorem 4.3 to reduce the matrix to row echelon form.
\begin{align*}
  A &= \begin{bmatrix}
    -4 & 1 & 3 \\
    2 & -2 & 4 \\
    1 & -1 & 0
  \end{bmatrix} (\frac{1}{2}R_2\to R_2) \\
  &= \begin{bmatrix}
    -4 & 1 & 3 \\
    1 & -1 & 2 \\
    1 & -1 & 0
  \end{bmatrix} (R_3-R_2\to R_3)\quad (4R_2\to R_2) \\
  &= \begin{bmatrix}
    -4 & 1 & 3 \\
    4 & -4 & 8 \\
    0 & 0 & -2
  \end{bmatrix} (R_2+R_1\to R_2) \\
  &= \begin{bmatrix}
    -4 & 1 & 3 \\
    0 & -3 & 11 \\
    0 & 0 & -2
  \end{bmatrix} \\
  \left(\frac{1}{2}\right)(4)det(A) &= (-4)(-3)(-2) \\
  det(A) &= -12
\end{align*}

\subsubsection*{Exercise 25}
Evaluate the given determinant using elementary row and/or column operations and
Theorem 4.3 to reduce the matrix to row echelon form.
\begin{align*}
  A &= \begin{bmatrix}
    2 & 0 & 3 & -1 \\
    1 & 0 & 2 & 2 \\
    0 & -1 & 1 & 4 \\
    2 & 0 & 1 & -3
  \end{bmatrix} (R_1-2R_2\to R_1) \\
  &= \begin{bmatrix}
    0 & 0 & -1 & -5 \\
    1 & 0 & 2 & 2 \\
    0 & -1 & 1 & 4 \\
    2 & 0 & 1 & -3
  \end{bmatrix} (R_4-2R_2\to R_2) \\
  &= \begin{bmatrix}
    0 & 0 & -1 & -5 \\
    0 & 0 & -3 & -7 \\
    0 & -1 & 1 & 4 \\
    2 & 0 & 1 & -3
  \end{bmatrix} (R_2-3R_1\to R_1) \\
  &= \begin{bmatrix}
    0 & 0 & 0 & 8 \\
    0 & 0 & -3 & -7 \\
    0 & -1 & 1 & 4 \\
    2 & 0 & 1 & -3
  \end{bmatrix} \\
  (-2)(-3)det(A) &= (2)(-1)(-3)(8) \\
  det(A) &= 8
\end{align*}

\subsubsection*{Exercise 27}
Use properties of determinants to evaluate the given determinant by inspection.
Explain your reasoning.
\[ \begin{vmatrix}
  3 & 1 & 0 \\
  0 & -2 & 5 \\
  0 & 0 & 4
\end{vmatrix} = -24 \]
The matrix is upper triangular. The determinant of a triangular matrix is the
product of the elements on the diagonal.

\subsubsection*{Exercise 29}
Use properties of determinants to evaluate the given determinant by inspection.
Explain your reasoning.
\[ \begin{vmatrix}
  2 & 3 & -4 \\
  1 & -3 & -2 \\
  -1 & 5 & 2
\end{vmatrix} = 0 \]
Multiplying column 1 by -2 would makes it identical to column 3.

\subsubsection*{Exercise 31}
Use properties of determinants to evaluate the given determinant by inspection.
Explain your reasoning.
\[ \begin{vmatrix}
  4 & 1 & 3 \\
  -2 & 0 & -2 \\
  5 & 4 & 1
\end{vmatrix} = 0 \]
Subtracting the second column from the first column makes the first column
identical to the third column.

\subsubsection*{Exercise 33}
Use properties of determinants to evaluate the given determinant by inspection.
Explain your reasoning.
\[ \begin{vmatrix}
  0 & 2 & 0 & 0 \\
  -3 & 0 & 0 & 0 \\
  0 & 0 & 0 & 4 \\
  0 & 0 & 1 & 0
\end{vmatrix} = -24 \]
Swapping the first and second rows and swapping the third and fourth rows makes
it a triangular matrix. These two operations negate the determinant and cancel
out. The determinant of a triangular matrix is the product of the elements on
the diagonal.

\subsubsection*{Exercise 35}
Find the determinants assuming that
\[ \begin{vmatrix}
  a & b & c \\
  d & e & f \\
  g & h & i
\end{vmatrix} = 4 \]
\[ \begin{vmatrix}
  2a & 2b & 2c \\
  d & e & f \\
  g & h & i
\end{vmatrix} = 8 \]

\subsubsection*{Exercise 37}
Find the determinants assuming that
\[ \begin{vmatrix}
  a & b & c \\
  d & e & f \\
  g & h & i
\end{vmatrix} = 4 \]
\[ \begin{vmatrix}
  d & e & f \\
  a & b & c \\
  g & h & i
\end{vmatrix} = -4 \]

\subsubsection*{Exercise 39}
Find the determinants assuming that
\[ \begin{vmatrix}
  a & b & c \\
  d & e & f \\
  g & h & i
\end{vmatrix} = 4 \]
\[ \begin{vmatrix}
  2c & b & a \\
  2f & e & d \\
  2i & h & g
\end{vmatrix} = -8 \]

\subsubsection*{Exercise 45}
Use Theorem 4.6 to find all values of \( k \) for which \( A \) is invertible.
\begin{align*}
  A &= \begin{bmatrix}
    k & -k & 3 \\
    0 & k+1 & 1 \\
    k & -8 & k-1
  \end{bmatrix} \\
  0 &\ne \begin{vmatrix}
    k & -k & 3 \\
    0 & k+1 & 1 \\
    k & -8 & k-1
  \end{vmatrix} \\
  &\ne k\begin{vmatrix}k+1 & 1 \\ -8 & k-1\end{vmatrix}-0+
    k\begin{vmatrix}-k & 3 \\ k+1 & 1\end{vmatrix} \\
  &\ne k(k^2-1-(-8))+k(-k-(3k+3)) \\
  &\ne k^3+7k-k^2-3k^2-3k \\
  &\ne k^3-4k^2+4k \\
  &\ne k(k^2-4k+4) \\
  &\ne k(k-2)(k-2) \\
  k &\ne 0 \quad k\ne2
\end{align*}

\subsubsection*{Exercise 46}
Use Theorem 4.6 to find all values of \( k \) for which \( A \) is invertible.
\begin{align*}
  A &= \begin{bmatrix}
    k & k & 0 \\
    k^2 & 2 & k \\
    0 & k & k
  \end{bmatrix} \\
  0 &\ne \begin{vmatrix}
    k & k & 0 \\
    k^2 & 2 & k \\
    0 & k & k
  \end{vmatrix} \\
  &\ne k\begin{vmatrix}2 & k \\ k & k\end{vmatrix}-
    k\begin{vmatrix}k^2 & k \\ 0 & k\end{vmatrix}+0 \\
  &\ne k(2k-k^2)-k(k^3) \\
  &\ne 2k^2-k^3-k^4 \\
  &\ne -k^2(k^2+k-2) \\
  &\ne -k^2(k+2)(k-1) \\
  k &\ne 0 \quad k\ne-2 \quad k\ne1
\end{align*}

\subsubsection*{Exercise 47}
Assume that \( A \) and \( B \) are \( n\times n \) matrices with
\( det(A) = 3 \) and \( det(B) = -2 \). Find the indicated determinants.
\[ det(AB) = -6 \]

\subsubsection*{Exercise 48}
Assume that \( A \) and \( B \) are \( n\times n \) matrices with
\( det(A) = 3 \) and \( det(B) = -2 \). Find the indicated determinants.
\[ det(A^2) = 9 \]

\subsubsection*{Exercise 49}
Assume that \( A \) and \( B \) are \( n\times n \) matrices with
\( det(A) = 3 \) and \( det(B) = -2 \). Find the indicated determinants.
\[ det(B^{-1}A) = \frac{1}{-2}3 = -\frac{3}{2} \]

\subsubsection*{Exercise 50}
Assume that \( A \) and \( B \) are \( n\times n \) matrices with
\( det(A) = 3 \) and \( det(B) = -2 \). Find the indicated determinants.
\[ det(2A) = 2^n(3) \]

\subsubsection*{Exercise 51}
Assume that \( A \) and \( B \) are \( n\times n \) matrices with
\( det(A) = 3 \) and \( det(B) = -2 \). Find the indicated determinants.
\[ det(3B^T) = 3^n(-2) \]

\subsubsection*{Exercise 52}
Assume that \( A \) and \( B \) are \( n\times n \) matrices with
\( det(A) = 3 \) and \( det(B) = -2 \). Find the indicated determinants.
\[ det(AA^T) = 9 \]

\subsubsection*{Exercise 57}
Use Cramer's Rule to solve the given linear system.
\begin{align*}
  x+y &= 1 \\
  x-y &= 2 \\
  A &= \begin{bmatrix}1 & 1 \\ 1 & -1\end{bmatrix} \quad
    \vec{b} = \begin{bmatrix}1 \\ 2\end{bmatrix} \\
  det(A) &= -2 \\
  det(A_1\vec{b}) &= \begin{vmatrix}1 & 1 \\ 2 & -1\end{vmatrix} = -3 \\
  det(A_2\vec{b}) &= \begin{vmatrix}1 & 1 \\ 1 & 2\end{vmatrix} = 1 \\
  x &= \frac{-3}{-2} = \frac{3}{2} \\
  y &= \frac{1}{-2} = -\frac{1}{2}
\end{align*}

\subsubsection*{Exercise 58}
Use Cramer's Rule to solve the given linear system.
\begin{align*}
  2x-y &= 5 \\
  x+3y &= -1 \\
  A &= \begin{bmatrix}2 & -1 \\ 1 & 3\end{bmatrix} \quad
    \vec{b} = \begin{bmatrix}5 \\ -1\end{bmatrix} \\
  det(A) &= 7 \\
  det(A_1\vec{b}) &= \begin{vmatrix}5 & -1 \\ -1 & 3 \end{vmatrix} = 14 \\
  det(A_2\vec{b}) &= \begin{vmatrix}2 & 5 \\ 1 & -1\end{vmatrix} = -7 \\
  x &= \frac{14}{7} = 2 \\
  y &= \frac{-7}{7} = -1
\end{align*}

\subsubsection*{Exercise 59}
Use Cramer's Rule to solve the given linear system.
\begin{align*}
  2x+y+3z &= 1 \\
  y+z &= 1 \\
  z &= 1 \\
  A &= \begin{bmatrix}
    2 & 1 & 3 \\
    0 & 1 & 1 \\
    0 & 0 & 1
  \end{bmatrix}\quad \vec{b} = \begin{bmatrix}1 \\ 1 \\ 1\end{bmatrix} \\
  det(A) &= 2 \\
  det(A_1\vec{b}) &= \begin{vmatrix}
    1 & 1 & 3 \\
    1 & 1 & 1 \\
    1 & 0 & 1
  \end{vmatrix} = -2 \\
  det(A_2\vec{b}) &= \begin{vmatrix}
    2 & 1 & 3 \\
    0 & 1 & 1 \\
    0 & 1 & 1
  \end{vmatrix} = 0 \\
  det(A_3\vec{b}) &= \begin{vmatrix}
    2 & 1 & 1 \\
    0 & 1 & 1 \\
    0 & 0 & 1
  \end{vmatrix} = 2 \\
  x &= \frac{-2}{2} = -1 \\
  y &= \frac{0}{2} = 0 \\
  z &= 1
\end{align*}

\subsubsection*{Exercise 60}
Use Cramer's Rule to solve the given linear system.
\begin{align*}
  x+y-z &= 1 \\
  x+y+z &= 2 \\
  x-y &= 3 \\
  A &= \begin{bmatrix}
    1 & 1 & -1 \\
    1 & 1 & 1 \\
    1 & -1 & 0
  \end{bmatrix}\quad \vec{b} = \begin{bmatrix}1 \\ 2 \\ 3\end{bmatrix} \\
  det(A) &= 2+2 = 4 \\
  det(A_1\vec{b}) &= \begin{vmatrix}
    1 & 1 & -1 \\
    2 & 1 & 1 \\
    3 & -1 & 0
  \end{vmatrix} = 6+3 = 9 \\
  det(A_2\vec{b}) &= \begin{vmatrix}
    1 & 1 & -1 \\
    1 & 2 & 1 \\
    1 & 3 & 0
  \end{vmatrix} = 3-6 = -3 \\
  det(A_3\vec{b}) &= \begin{vmatrix}
    1 & 1 & 1 \\
    1 & 1 & 2 \\
    1 & -1 & 3
  \end{vmatrix} = 1+1+0 = 2 \\
  x &= \frac{9}{4} \\
  y &= \frac{-3}{4} \\
  z &= \frac{2}{4}
\end{align*}

\subsubsection*{Exercise 61}
Compute the inverse of the coefficient matrix.
\begin{align*}
  A &= \begin{bmatrix}1 & 1 \\ 1 & -1\end{bmatrix} \\
  C_{11} &= -1 \\
  C_{12} &= -1 \\
  C_{21} &= -1 \\
  C_{22} &= 1 \\
  A^{-1} &= \frac{1}{-2}\begin{bmatrix}-1 & -1 \\ -1 & 1\end{bmatrix} \\
  &= \begin{bmatrix}
    \frac{1}{2} & \frac{1}{2} \\[0.5em]
    \frac{1}{2} & -\frac{1}{2}
  \end{bmatrix}
\end{align*}

\subsubsection*{Exercise 62}
Compute the inverse of the coefficient matrix.
\begin{align*}
  A &= \begin{bmatrix}2 & -1 \\ 1 & 3\end{bmatrix} \\
  C_{11} &= 3 \\
  C_{12} &= -1 \\
  C_{21} &= 1 \\
  C_{22} &= 2 \\
  A^{-1} &= \frac{1}{7}\begin{bmatrix}3 & -1 \\ 1 & 2\end{bmatrix}^T \\
  &= \begin{bmatrix}
    \frac{3}{7} & \frac{1}{7} \\[0.5em]
    -\frac{1}{7} & \frac{2}{7}
  \end{bmatrix}
\end{align*}

\subsubsection*{Exercise 63}
Compute the inverse of the coefficient matrix.
\begin{align*}
  A &= \begin{bmatrix}
    2 & 1 & 3 \\
    0 & 1 & 1 \\
    0 & 0 & 1
  \end{bmatrix} \\
  adj(A) &= \begin{bmatrix}
    1 & 0 & 0 \\
    -1 & 2 & 0 \\
    -2 & -2 & 2
  \end{bmatrix}^T \\
  \frac{1}{det(A)} &= \frac{1}{2} \\
  A^{-1} &= \begin{bmatrix}
    \frac{1}{2} & -\frac{1}{2} & -1 \\
    0 & 1 & -1 \\
    0 & 0 & 1
  \end{bmatrix}
\end{align*}

\subsubsection*{Exercise 64}
Compute the inverse of the coefficient matrix.
\begin{align*}
  A &= \begin{bmatrix}
    1 & 1 & -1 \\
    1 & 1 & 1 \\
    1 & -1 & 0
  \end{bmatrix} \\
  adj(A) &= \begin{bmatrix}
    1 & 1 & -2 \\
    1 & 1 & 2 \\
    2 & -2 & 0
  \end{bmatrix}^T \\
  \frac{1}{det(A)} &= \frac{1}{4} \\
  A^{-1} &= \begin{bmatrix}
    \frac{1}{4} & \frac{1}{4} & \frac{1}{2} \\[0.5em]
    \frac{1}{2} & \frac{1}{4} & -\frac{1}{2} \\[0.5em]
    -\frac{1}{2} & \frac{1}{2} & 0
  \end{bmatrix}
\end{align*}

\section*{Section 4.3}

\subsubsection*{Exercise 1}
Compute the characteristic polynomial of \( A \), the eigenvalues of \( A \),
a basis for each eigenspace of \( A \), and the algebraic and geometric
multiplicity of each eigenvalue.
\begin{align*}
  A &= \begin{bmatrix}1 & 3 \\ -2 & 6\end{bmatrix} \\
  |A-\lambda I| &= \begin{vmatrix}
    1-\lambda & 3 \\
    -2 & 6-\lambda
  \end{vmatrix} = (6-\lambda)(1-\lambda)-(-6) \\
  &= 6-7\lambda+\lambda^2+6 = \lambda^2-7\lambda+12 \\
  &= (\lambda-3)(\lambda-4) = 0 \\
  \lambda &= 3 \quad \lambda = 4 \\
  [A-3I|0] &= \begin{bmatrix}
    -2 & 3 & 0 \\
    -2 & 3 & 0
  \end{bmatrix} \\
  3x_2 &= 2x_1 \\
  E_3 &= span\left(\begin{bmatrix}3 \\ 2\end{bmatrix}\right) \\
  [A-4I|0] &= \left[\begin{array}{cc|c}
    -3 & 3 & 0 \\
    -2 & 2 & 0
  \end{array}\right] = \begin{bmatrix}
    -1 & 1 & 0 \\
    0 & 0 & 0
  \end{bmatrix} \\
  x_1 &= x_2 \\
  E_4 &= span\left(\begin{bmatrix}1 \\ 1\end{bmatrix}\right)
\end{align*}
Each eigenvalue has algebraic and geometric multiplicity 1.

\subsubsection*{Exercise 3}
Compute the characteristic polynomial of \( A \), the eigenvalues of \( A \),
a basis for each eigenspace of \( A \), and the algebraic and geometric
multiplicity of each eigenvalue.
\begin{align*}
  A &= \begin{bmatrix}
    1 & 1 & 0 \\
    0 & -2 & 1 \\
    0 & 0 & 3
  \end{bmatrix} \\
  |A-\lambda I| &= \begin{vmatrix}
    1-\lambda & 1 & 0 \\
    0 & -2-\lambda & 1 \\
    0 & 0 & 3-\lambda
  \end{vmatrix} = (1-\lambda)(-2-\lambda)(3-\lambda) = 0 \\
  \lambda &= 1 \quad \lambda = -2 \quad \lambda = 3 \\
  [A-1I|0] &= \begin{bmatrix}
    0 & 1 & 0 & 0 \\
    0 & -3 & 1 & 0 \\
    0 & 0 & 2 & 0
  \end{bmatrix} = \begin{bmatrix}
    0 & 1 & 0 & 0 \\
    0 & 0 & 1 & 0 \\
    0 & 0 & 0 & 0
  \end{bmatrix} \\
  x_2 &= x_3 = 0 \\
  E_1 &= \begin{bmatrix}1 \\ 0 \\ 0\end{bmatrix}
\end{align*}
\begin{align*}
  [A+2I|0] &= \begin{bmatrix}
    3 & 1 & 0 & 0 \\
    0 & 0 & 1 & 0 \\
    0 & 0 & 5 & 0
  \end{bmatrix} = \begin{bmatrix}
    3 & 1 & 0 & 0 \\
    0 & 0 & 1 & 0 \\
    0 & 0 & 0 & 0
  \end{bmatrix} \\
  3x_1 &= -x_2 \\
  x_3 &= 0 \\
  E_{-2} &= span\left(\begin{bmatrix}-1 \\ 3 \\ 0\end{bmatrix}\right) \\
  [A-3I|0] &= \begin{bmatrix}
    -2 & 1 & 0 & 0 \\
    0 & -5 & 1 & 0 \\
    0 & 0 & 0 & 0
  \end{bmatrix} = \begin{bmatrix}
    10 & 0 & -1 & 0 \\
    0 & 5 & -1 & 0 \\
    0 & 0 & 0 & 0
  \end{bmatrix} \\
  10x_1 &= 5x_2 = x_3 \\
  E_3 &= span\left(\begin{bmatrix}1 \\ 2 \\ 10\end{bmatrix}\right)
\end{align*}
The algebraic and geometric multiplicities of the eigenvalues are 1.

\subsubsection*{Exercise 5}
Compute the characteristic polynomial of \( A \), the eigenvalues of \( A \),
a basis for each eigenspace of \( A \), and the algebraic and geometric
multiplicity of each eigenvalue.
\begin{align*}
  A &= \begin{bmatrix}
    1 & 2 & 0 \\
    -1 & -1 & 1 \\
    0 & 1 & 1
  \end{bmatrix} \\
  |A-\lambda I| &= \begin{vmatrix}
    1-\lambda & 2 & 0 \\
    -1 & -1-\lambda & 1 \\
    0 & 1 & 1-\lambda
  \end{vmatrix} = (1-\lambda)((1-\lambda)(-1-\lambda)-1)-2(-(1-\lambda)) \\
  &= (1-\lambda)(-1+\lambda^2-1)+2-2\lambda \\
  &= -2+\lambda^2+2\lambda-\lambda^3+2-2\lambda \\
  &= -\lambda^3+\lambda^2 \\
  &= -\lambda^2(\lambda-1) = 0 \\
  \lambda &= 0 \quad \lambda = 1 \\
  [A-0I|0] &= \begin{bmatrix}
    1 & 2 & 0 & 0 \\
    -1 & -1 & 1 & 0 \\
    0 & 1 & 1 & 0
  \end{bmatrix} = \begin{bmatrix}
    1 & 0 & -2 & 0 \\
    0 & 1 & 1 & 0 \\
    0 & 0 & 0 & 0
  \end{bmatrix} \\
  x_1 &= 2x_3 \\
  x_2 &= -x_3 \\
  E_0 &= span\left(\begin{bmatrix}2 \\ -1 \\ 1\end{bmatrix}\right)
\end{align*}
\begin{align*}
  [A-1I|0] &= \begin{bmatrix}
    0 & 2 & 0 & 0 \\
    -1 & -2 & 1 & 0 \\
    0 & 1 & 0 & 0
  \end{bmatrix} = \begin{bmatrix}
    1 & 0 & -1 & 0 \\
    0 & 1 & 0 & 0 \\
    0 & 0 & 0 & 0
  \end{bmatrix} \\
  x_1 &= x_3 \\
  x_2 &= 0 \\
  E_1 &= span\left(\begin{bmatrix}1 \\ 0 \\ 1\end{bmatrix}\right)
\end{align*}
\( \lambda = 0 \) has algebraic multiplicity 2 while \( \lambda = 1 \) has
algebraic multiplicity 1. They both have geometric multiplicity 1.

\subsubsection*{Exercise 7}
Compute the characteristic polynomial of \( A \), the eigenvalues of \( A \),
a basis for each eigenspace of \( A \), and the algebraic and geometric
multiplicity of each eigenvalue.
\begin{align*}
  A &= \begin{bmatrix}
    4 & 0 & 1 \\
    2 & 3 & 2 \\
    -1 & 0 & 2
  \end{bmatrix} \\
  |A-\lambda I| &= \begin{vmatrix}
    4-\lambda & 0 & 1 \\
    2 & 3-\lambda & 2 \\
    -1 & 0 & 2-\lambda
  \end{vmatrix} = (4-\lambda)((3-\lambda)(2-\lambda))+(0+(3-\lambda)) \\
  &= (4-\lambda)(6-5\lambda+\lambda^2)+3-\lambda \\
  &= 24-20\lambda+4\lambda^2-6\lambda+5\lambda^2-\lambda^3+3-\lambda \\
  &= -\lambda^3+9\lambda^2-27\lambda+27 = 0 \\
  \lambda &= 3 \\
  [A-3I|0] &= \begin{bmatrix}
    1 & 0 & 1 & 0 \\
    2 & 0 & 2 & 0 \\
    -1 & 0 & -1 & 0
  \end{bmatrix} = \begin{bmatrix}
    1 & 0 & 1 & 0 \\
    0 & 0 & 0 & 0 \\
    0 & 0 & 0 & 0
  \end{bmatrix} \\
  x_1 &= -x_3 \\
  E_3 &= \begin{bmatrix}x_1 & x_2 & -x_1\end{bmatrix} \\
  &= x_1\begin{bmatrix}1 \\ 0 \\ -1\end{bmatrix}+
    x_2\begin{bmatrix}0 \\ 1 \\ 0\end{bmatrix} \\
  &= span\left(\begin{bmatrix}1 \\ 0 \\ -1\end{bmatrix},
    \begin{bmatrix}0 \\ 1 \\ 0\end{bmatrix}\right)
\end{align*}
\( \lambda = 3 \) has algebraic multiplicity 3 and geometric multiplicity 2.

\subsubsection*{Exercise 9}
Compute the characteristic polynomial of \( A \), the eigenvalues of \( A \),
a basis for each eigenspace of \( A \), and the algebraic and geometric
multiplicity of each eigenvalue.
\begin{align*}
  A &= \begin{bmatrix}
    3 & 1 & 0 & 0 \\
    -1 & 1 & 0 & 0 \\
    0 & 0 & 1 & 4 \\
    0 & 0 & 1 & 1
  \end{bmatrix} \\
  |A-\lambda I| &= \begin{vmatrix}
    3-\lambda & 1 & 0 & 0 \\
    -1 & 1-\lambda & 0 & 0 \\
    0 & 0 & 1-\lambda & 4 \\
    0 & 0 & 1 & 1-\lambda
  \end{vmatrix} \\
  &= ((3-\lambda)(1-\lambda)+1)((1-\lambda)^2-4) \\
  &= \lambda^4-6\lambda^3+9\lambda^2+4\lambda-12 \\
  &= \lambda^2(\lambda^2-6\lambda+9)+4(\lambda-3) \\
  &= (\lambda-3)(\lambda^3-3\lambda^2+4) \\
  &= (\lambda-3)(\lambda-2)^2(\lambda+1) = 0 \\
  \lambda &= -1 \quad \lambda = 2 \quad \lambda = 3 \\
  [A+1I|0] &= \begin{bmatrix}
    4 & 1 & 0 & 0 & 0 \\
    -1 & 2 & 0 & 0 & 0 \\
    0 & 0 & 2 & 4 & 0 \\
    0 & 0 & 1 & 2 & 0
  \end{bmatrix} = \begin{bmatrix}
    1 & 0 & 0 & 0 & 0 \\
    0 & 1 & 0 & 0 & 0 \\
    0 & 0 & 1 & 2 & 0 \\
    0 & 0 & 0 & 0 & 0
  \end{bmatrix} \\
  x_1 &= x_2 = 0 \\
  x_3 &= -2x_4 \\
  E_{-1} &= span\left(\begin{bmatrix}0 \\ 0 \\ -2 \\ 1\end{bmatrix}\right) \\
  [A-2I|0] &= \begin{bmatrix}
    1 & 1 & 0 & 0 & 0 \\
    -1 & -1 & 0 & 0 & 0 \\
    0 & 0 & -1 & 4 & 0 \\
    0 & 0 & 1 & -1 & 0
  \end{bmatrix} = \begin{bmatrix}
    1 & 1 & 0 & 0 & 0 \\
    0 & 0 & 1 & 0 & 0 \\
    0 & 0 & 0 & 1 & 0 \\
    0 & 0 & 0 & 0 & 0
  \end{bmatrix} \\
  x_1 &= -x_2 \\
  x_3 &= x_4 = 0 \\
  E_2 &= span\left(\begin{bmatrix}-1 \\ 1 \\ 0 \\ 0\end{bmatrix}\right)
\end{align*}
\begin{align*}
  [A-3I|0] &= \begin{bmatrix}
    0 & 1 & 0 & 0 & 0 \\
    -1 & -2 & 0 & 0 & 0 \\
    0 & 0 & -2 & 4 & 0 \\
    0 & 0 & 1 & -2 & 0
  \end{bmatrix} = \begin{bmatrix}
    1 & 0 & 0 & 0 & 0 \\
    0 & 1 & 0 & 0 & 0 \\
    0 & 0 & 1 & -2 & 0 \\
    0 & 0 & 0 & 0 & 0
  \end{bmatrix} \\
  x_1 &= x_2 = 0 \\
  x_3 &= 2x_4 \\
  E_3 &= span\left(\begin{bmatrix}0 \\ 0 \\ 2 \\ 1\end{bmatrix}\right)
\end{align*}

\subsubsection*{Exercise 11}
Compute the characteristic polynomial of \( A \), the eigenvalues of \( A \),
a basis for each eigenspace of \( A \), and the algebraic and geometric
multiplicity of each eigenvalue.
\begin{align*}
  A &= \begin{bmatrix}
    1 & 0 & 0 & 0 \\
    0 & 1 & 0 & 0 \\
    1 & 1 & 3 & 0 \\
    -2 & 1 & 2 & -1
  \end{bmatrix} \\
  |A-\lambda I| &= \begin{vmatrix}
    1-\lambda & 0 & 0 & 0 \\
    0 & 1-\lambda & 0 & 0 \\
    1 & 1 & 3-\lambda & 0 \\
    -2 & 1 & 2 & -1-\lambda
  \end{vmatrix} \\
  &= (1-\lambda)(1-\lambda)(3-\lambda)(-1-\lambda) = 0 \\
  \lambda &= 1 \quad \lambda = 3 \quad \lambda = -1 \\
  [A-1I|0] &= \begin{bmatrix}
    0 & 0 & 0 & 0 & 0 \\
    0 & 0 & 0 & 0 & 0 \\
    1 & 1 & 2 & 0 & 0 \\
    -2 & 1 & 2 & -2 & 0
  \end{bmatrix} = \begin{bmatrix}
    0 & 0 & 0 & 0 & 0 \\
    0 & 0 & 0 & 0 & 0 \\
    1 & 1 & 2 & 0 & 0 \\
    3 & 0 & 0 & 2 & 0
  \end{bmatrix} \\
  x_1+x_2+2x_3 &= 0 \\
  3x_1+2x_4 &= 0 \\
  E_1 &= \begin{bmatrix}
    x_1 \\ -2x_3-x_1 \\ x_3 \\ \frac{-3x_1}{2}
  \end{bmatrix} =
    \begin{bmatrix}2x_1 \\ -4x_3-2x_1 \\ 2x_3 \\ -3x_1\end{bmatrix} =
    x_1\begin{bmatrix}2 \\ -2 \\ 0 \\ -3\end{bmatrix}+
    x_3\begin{bmatrix}0 \\ -4 \\ 2 \\ 0\end{bmatrix} \\
  &= span\left(\begin{bmatrix}2 \\ -2 \\ 0 \\ -3\end{bmatrix},
    \begin{bmatrix}0 \\ -2 \\ 1 \\ 0\end{bmatrix}\right)
\end{align*}

\subsubsection*{Exercise 13}
Prove Theorem 4.18b: Let \( A \) be a square matrix with eigenvalue
\( \lambda \) and corresponding eigenvector \( \vec{x} \). If \( A \) is
invertible, then \( \frac{1}{\lambda} \) is an eigenvalue of \( A^{-1} \) with
corresponding eigenvector \( \vec{x} \).
\begin{align*}
  A\vec{x} &= \lambda\vec{x} \\
  A^{-1}A\vec{x} &= A^{-1}(\lambda\vec{x}) \\
  \vec{x} &= \lambda A^{-1}\vec{x} \\
  A^{-1}\vec{x} &= \frac{1}{\lambda}\vec{x}
\end{align*}

\subsubsection*{Exercise 14}
Prove Theorem 4.18c: Let \( A \) be a square matrix with eigenvalue
\( \lambda \) and corresponding eigenvector \( \vec{x} \). If \( A \) is
invertible, then for any integer \( n \), \( \lambda^n \) is an eigenvalue of
\( A^n \) with corresponding eigenvector \( \vec{x} \). \\
Base Case \( n = 1 \): \\
\[ A^1\vec{x} = \lambda^1\vec{x} \]
Induction Hypothesis: \( A^n\vec{x} = \lambda^n\vec{x} \) \\
Induction Step:
\begin{align*}
  A^{n+1}\vec{x} &= A(A^n\vec{x}) \\
  &= A(\lambda^n\vec{x}) \\
  &= \lambda^n(A\vec{x}) \\
  &= \lambda^n(\lambda\vec{x}) \\
  &= \lambda^{n+1}\vec{x}
\end{align*}

\subsubsection*{Exercise 20}
Let \( A \) be a nilpotent matrix. Show that \( \lambda = 0 \) is the only
eigenvalue of \( A \).
\[ |A^m| = |0| = 0 \]
Thus \( \lambda = 0 \) is an eigenvalue. Show that there are no other
eigenvalues. \\
Suppose \( \lambda \) is another eigenvalue \( (\lambda\ne0) \).
Then \( \lambda^m \) is an eigenvalue of \( A^m \). This forces \( \lambda^m =
0 \therefore \lambda = 0 \).

\subsubsection*{Exercise 21}
Let \( A \) be an idempotent matrix. Show that \( \lambda = 0 \) and
\( \lambda = 1 \) are the only possible eigenvalues of \( A \). \\
Since \( A \) is idempotent, it means that \( A^2 = A \). Let \( \lambda \) be
an eigenvalue of \( A \), then \( \lambda^2 \) is an eigenvalue of
\( A^2 = A \).
\begin{align*}
  \lambda^2 &= \lambda \\
  \lambda^2-\lambda &= 0 \\
  \lambda(\lambda-1) &= 0 \\
  \lambda &= 0 \quad \lambda = 1
\end{align*}

\subsubsection*{Exercise 22}
If \( \vec{v} \) is an eigenvector of \( A \) with corresponding eigenvalue
\( \lambda \) and \( c \) is a scalar, show that \( \vec{v} \) is an eigenvector
of \( A-cI \) with corresponding eigenvalue \( \lambda-c \).
\begin{align*}
  B &= A-cI \\
  (A-\lambda I)\vec{v} &= 0 \\
  B-(\lambda-c)I &= A-cI-(\lambda-c)I \\
  &= A-cI-\lambda I+cI \\
  &= A-\lambda I \\
  (B-(\lambda-c)I)\vec{v} &= (A-\lambda I)\vec{v} = 0
\end{align*}

\section*{Section 4.4}

\subsubsection*{Exercise 1}
Show that \( A \) and \( B \) are not similar matrices.
\begin{align*}
  A &= \begin{bmatrix}4 & 1 \\ 3 & 1\end{bmatrix} \quad
    B = \begin{bmatrix}1 & 0 \\ 0 & 1\end{bmatrix} \\
  |A-\lambda I| &= \begin{vmatrix}
    4-\lambda & 1 \\
    3 & 1-\lambda
  \end{vmatrix} = (4-\lambda)(1-\lambda)-3 \\
  &= 4-5\lambda+\lambda^2-3 \\
  &= \lambda^2-5\lambda-7 \\
  |B-\lambda I| &= \begin{vmatrix}
    1-\lambda & 0 \\
    0 & 1-\lambda
  \end{vmatrix} = (1-\lambda)(1-\lambda)
\end{align*}
Since \( A \) and \( B \) do not have the same characteristic polynomial, they
are not similar.

\subsubsection*{Exercise 3}
Show that \( A \) and \( B \) are not similar matrices.
\begin{align*}
  A &= \begin{bmatrix}
    2 & 1 & 4 \\
    0 & 2 & 3 \\
    0 & 0 & 4
  \end{bmatrix} \\
  B &= \begin{bmatrix}
    1 & 0 & 0 \\
    -1 & 4 & 0 \\
    2 & 3 & 4
  \end{bmatrix} \\
  |A-\lambda I| &= (2-\lambda)(2-\lambda)(4-\lambda) \\
  |B-\lambda I| &= (1-\lambda)(4-\lambda)(4-\lambda)
\end{align*}
Since \( A \) and \( B \) do not have the same characteristic polynomial, they
are not similar.

\subsubsection*{Exercise 5}
A diagonalization of the matrix \( A \) is given in form \( P^{-1}AP = D \).
List the eigenvalues of \( A \) and bases from the corresponding eigenspaces.
\[ \begin{bmatrix}2 & -1 \\ -1 & 1\end{bmatrix}
  \begin{bmatrix}5 & -1 \\ 2 & 2\end{bmatrix}
  \begin{bmatrix}1 & 1 \\ 1 & 2\end{bmatrix} =
  \begin{bmatrix}4 & 0 \\ 0 & 3\end{bmatrix} \]
\begin{align*}
  \lambda &= 4 \quad \lambda = 3 \\
  E_4 &= span\left(\begin{bmatrix}1 \\ 1\end{bmatrix}\right) \\
  E_3 &= span\left(\begin{bmatrix}1 \\ 2\end{bmatrix}\right)
\end{align*}

\subsubsection*{Exercise 7}
A diagonalization of the matrix \( A \) is given in form \( P^{-1}AP = D \).
List the eigenvalues of \( A \) and bases from the corresponding eigenspaces.
\[ \begin{bmatrix}
    \frac{1}{8} & \frac{1}{8} & \frac{1}{8} \\[0.5em]
    -\frac{1}{4} & \frac{3}{4} & -\frac{1}{4} \\[0.5em]
    \frac{5}{8} & -\frac{3}{8} & -\frac{3}{8}
  \end{bmatrix}\begin{bmatrix}
    1 & 3 & 3 \\
    2 & 0 & 2 \\
    3 & 3 & 1
  \end{bmatrix}\begin{bmatrix}
    3 & 0 & 1 \\
    2 & 1 & 0 \\
    3 & -1 & -1
  \end{bmatrix} = \begin{bmatrix}
    6 & 0 & 0 \\
    0 & -2 & 0 \\
    0 & 0 & -2
  \end{bmatrix} \]
\begin{align*}
  \lambda &= 6 \quad \lambda = -2 \quad \lambda = -2 \\
  E_6 &= span\left(\begin{bmatrix}3 \\ 2 \\ 3\end{bmatrix}\right) \\
  E_{-2} &= span\left(\begin{bmatrix}0 \\ 1 \\ -1\end{bmatrix},
    \begin{bmatrix}1 \\ 0 \\ -1\end{bmatrix}\right)
\end{align*}

\subsubsection*{Exercise 9}
Determine whether \( A \) is diagonalizable and, if so, find an invertible
matrix \( P \) and a diagonal matrix \( D \) such that \( P^{-1}AP = D \).
\begin{align*}
  A &= \begin{bmatrix}-3 & 4 \\ -1 & 1\end{bmatrix} \\
  |A-\lambda I| &= \begin{vmatrix}
    -3-\lambda & 4 \\
    -1 & 1-\lambda
  \end{vmatrix} = (-3-\lambda)(1-\lambda)+4 \\
  &= -3+2\lambda+\lambda^2+4 = \lambda^2+2\lambda+1 \\
  &= (\lambda+1)^2 = 0 \\
\end{align*}
\( A \) is not diagonalizable since there cannot be two linearly independent
eigenvectors.

\subsubsection*{Exercise 11}
Determine whether \( A \) is diagonalizable and, if so, find an invertible
matrix \( P \) and a diagonal matrix \( D \) such that \( P^{-1}AP = D \).
\begin{align*}
  A &= \begin{bmatrix}
    1 & 0 & 1 \\
    0 & 1 & 1 \\
    1 & 1 & 0
  \end{bmatrix} \\
  |A-\lambda I| &= \begin{vmatrix}
    1-\lambda & 0 & 1 \\
    0 & 1-\lambda & 1 & 1 \\
    1 & 1 & -\lambda
  \end{vmatrix} = -\lambda^3+2\lambda^2+\lambda-2 \\
  &= -\lambda^2(\lambda-2)+1(\lambda-2) \\
  &= (1-\lambda)(1+\lambda)(\lambda-2) = 0 \\
  \lambda &= 1 \quad \lambda = -1 \quad \lambda = 2 \\
  [A-1I|0] &= \begin{bmatrix}
    0 & 0 & 1 & 0\\
    0 & 0 & 1 & 0 \\
    1 & 1 & -1 & 0
  \end{bmatrix} \\
  E_1 &= span\left(\begin{bmatrix}-1 \\ 1 \\ 0\end{bmatrix}\right) \\
  [A+1I|0] &= \begin{bmatrix}
    2 & 0 & 1 & 0 \\
    0 & 2 & 1 & 0 \\
    1 & 1 & 1 & 0
  \end{bmatrix} \\
  E_{-1} &= span\left(\begin{bmatrix}-1 \\ -1 \\ 2\end{bmatrix}\right) \\
  [A-2I|0] &= \begin{bmatrix}
    -1 & 0 & 1 & 0 \\
    0 & -1 & 1 & 0 \\
    1 & 1 & -2 & 0
  \end{bmatrix} \\
  E_2 &= span\left(\begin{bmatrix}1 \\ 1 \\ 1\end{bmatrix}\right)
\end{align*}
The algebraic multiplicities of the eigenvectors are equal to their geometric
multiplicities, so the matrix is diagonalizable. The following matrices \( P \)
and \( D \) satisfy \( P^{-1}AP = D \).
\[ P = \begin{bmatrix}
    -1 & -1 & 1 \\
    1 & -1 & 1 \\
    0 & 2 & 1
  \end{bmatrix} \quad D = \begin{bmatrix}
    1 & 0 & 0 \\
    0 & -1 & 0 \\
    0 & 0 & 2
  \end{bmatrix} \]

\subsubsection*{Exercise 13}
Determine whether \( A \) is diagonalizable and, if so, find an invertible
matrix \( P \) and a diagonal matrix \( D \) such that \( P^{-1}AP = D \).
\begin{align*}
  A &= \begin{bmatrix}
    1 & 2 & 1 \\
    -1 & 0 & 1 \\
    1 & 1 & 0
  \end{bmatrix} \\
  |A-\lambda I| &= \begin{vmatrix}
    1-\lambda & 2 & 1 \\
    -1 & -\lambda & 1 \\
    1 & 1 & -\lambda
  \end{vmatrix} = \lambda^2(\lambda-1) = 0 \\
  \lambda &= 0 \quad \lambda = 1 \\
  [A-0I|0] &= \begin{bmatrix}
    1 & 2 & 1 & 0 \\
    -1 & 0 & 1 & 0\\
    1 & 1 & 0 & 0
  \end{bmatrix} \\
  E_0 &= span\left(\begin{bmatrix}1 \\ -1 \\ 1\end{bmatrix}\right)
\end{align*}
Since the geometric multiplicity is not equal to the algebraic multiplicity for
\( \lambda = 0 \), this matrix is not diagonalizable.

\subsubsection*{Exercise 15}
Determine whether \( A \) is diagonalizable and, if so, find an invertible
matrix \( P \) and a diagonal matrix \( D \) such that \( P^{-1}AP = D \).
\begin{align*}
  A &= \begin{bmatrix}
    2 & 0 & 0 & 4 \\
    0 & 2 & 0 & 0 \\
    0 & 0 & -2 & 0 \\
    0 & 0 & 0 & -2
  \end{bmatrix} \\
  \lambda &= 2 \quad \lambda = 2 \\
  [A-2I|0] &= \begin{bmatrix}
    0 & 0 & 0 & 4 & 0 \\
    0 & 0 & 0 & 0 & 0 \\
    0 & 0 & -4 & 0 & 0 \\
    0 & 0 & 0 & -4 & 0
  \end{bmatrix} = \begin{bmatrix}
    0 & 0 & 1 & 0 & 0 \\
    0 & 0 & 0 & 1 & 0 \\
    0 & 0 & 0 & 0 & 0 \\
    0 & 0 & 0 & 0 & 0
  \end{bmatrix} \\
  x_3 &= x_4 = 0 \\
  E_2 &= span\left(\begin{bmatrix}1 \\ 0 \\ 0 \\ 0\end{bmatrix},
    \begin{bmatrix}0 \\ 1 \\ 0 \\ 0\end{bmatrix}\right)
\end{align*}
\begin{align*}
  [A+2I|0] &= \begin{bmatrix}
    4 & 0 & 0 & 4 & 0 \\
    0 & 4 & 0 & 0 & 0 \\
    0 & 0 & 0 & 0 & 0 \\
    0 & 0 & 0 & 0 & 0
  \end{bmatrix} = \begin{bmatrix}
    1 & 0 & 0 & 1 & 0 \\
    0 & 1 & 0 & 0 & 0 \\
    0 & 0 & 0 & 0 & 0 \\
    0 & 0 & 0 & 0 & 0
  \end{bmatrix} \\
  x_1 &= -x_4 \\
  x_2 &= 0 \\
  E_{-2} &= \begin{bmatrix}x_1 \\ 0 \\ x_3 \\ -x_1\end{bmatrix} =
    x_1\begin{bmatrix}1 \\ 0 \\ 0 \\ -1\end{bmatrix}+
    x_3\begin{bmatrix}0 \\ 0 \\ 1 \\ 0\end{bmatrix} \\
  &= span\left(\begin{bmatrix}1 \\ 0 \\ 0 \\ -1\end{bmatrix},
    \begin{bmatrix}0 \\ 0 \\ 1 \\ 0\end{bmatrix}\right)
\end{align*}
Since the algebraic multiplicities of the eigenvalues is equal to their
respective geometric multiplicities, the matrix is diagonalizable. The
following matrices \( P \) and \( D \) satisfy \( P^{-1}AP = D \).
\[ D = \begin{bmatrix}
    2 & 0 & 0 & 0 \\
    0 & 2 & 0 & 0 \\
    0 & 0 & -2 & 0 \\
    0 & 0 & 0 & -2
  \end{bmatrix} \quad P = \begin{bmatrix}
    1 & 0 & 1 & 0 \\
    0 & 1 & 0 & 0 \\
    0 & 0 & 0 & 1 \\
    0 & 0 & -1 & 0
  \end{bmatrix} \]

\subsubsection*{Exercise 17}
Use the method of Example 4.29 to compute the indicated power of the matrix.
\begin{align*}
  A &= \begin{bmatrix}-1 & 6 \\ 1 & 0\end{bmatrix} \\
  |A-\lambda I| &= \begin{vmatrix}
    -1-\lambda & 6 \\
    1 & -\lambda
  \end{vmatrix} \\
  &= -\lambda(-1-\lambda)-6 = \lambda^2+\lambda-6 \\
  &= (\lambda+3)(\lambda-2) = 0 \\
  \lambda &= -3 \quad \lambda = 2 \\
  [A+3I|0] &= \begin{bmatrix}
    2 & 6 & 0 \\
    1 & 3 & 0
  \end{bmatrix} = \begin{bmatrix}
    1 & 3 & 0 \\
    0 & 0 & 0
  \end{bmatrix} \\
  E_{-3} &= span\left(\begin{bmatrix}3 \\ -1\end{bmatrix}\right) \\
  [A-2I|0] &= \begin{bmatrix}
    -3 & 6 & 0 \\
    1 & -2 & 0
  \end{bmatrix} = \begin{bmatrix}
    -1 & 2 & 0 \\
    0 & 0 & 0
  \end{bmatrix} \\
  E_2 &= span\left(\begin{bmatrix}2 \\ 1\end{bmatrix}\right) \\
  D &= \begin{bmatrix}2 & 0 \\ 0 & -3\end{bmatrix} \\
  P &= \begin{bmatrix}2 & 3 \\ 1 & -1\end{bmatrix} \\
  P^{-1} &= -1\begin{bmatrix}-1 & -3 \\ -1 & 2\end{bmatrix} \\
  P^{-1}AP &= D \\
  P^{-1}A^{10}P &= D^{10} \\
  A^{10} &= PD^{10}P^{-1} \\
  &= \begin{bmatrix}2 & 3 \\ 1 & -1\end{bmatrix}
    \begin{bmatrix}2^{10} & 0 \\ 0 & (-3)^{10}\end{bmatrix}
    \begin{bmatrix}1 & 3 \\ 1 & -2\end{bmatrix} \\
  &= \begin{bmatrix}
    179195 & -348150 \\
    -58025 & 121170
  \end{bmatrix}
\end{align*}

\subsubsection*{Exercise 19}
\subsubsection*{Exercise 21}
\subsubsection*{Exercise 23}
\subsubsection*{Exercise 25}
\subsubsection*{Exercise 27}
\subsubsection*{Exercise 29}

\begin{center}
  If you have any questions, comments, or concerns, please contact me at
  alvin@omgimanerd.tech
\end{center}

\end{document}
