\documentclass[letterpaper, 12pt]{math}

\geometry{letterpaper, margin=0.5in}

\title{Linear Algebra: Homework 5}
\author{Alvin Lin}
\date{August 2016 - December 2016}

\begin{document}

\maketitle

\section*{Section 2.3}

\subsubsection*{Exercise 1}
Determine if the vector \( \vec{v} \) is a linear combination of the remaining
vectors.
\[ \vec{v} = \begin{bmatrix}1 \\ 2\end{bmatrix} \quad
  \vec{u_1} = \begin{bmatrix}1 \\ -1\end{bmatrix} \quad
  \vec{u_2} = \begin{bmatrix}2 \\ -1\end{bmatrix} \]
\begin{align*}
  A &= \begin{bmatrix}
    1 & 2 & 1 \\
    -1 & -1 & 2
  \end{bmatrix} \\
  &= \begin{bmatrix}
    0 & 1 & 3 \\
    -1 & -1 2
  \end{bmatrix} \\
  &= \begin{bmatrix}
    0 & 1 & 3 \\
    -1 & 0 5
  \end{bmatrix} \\
  & \text{Yes}
\end{align*}

\subsubsection*{Exercise 3}
Determine if the vector \( \vec{v} \) is a linear combination of the remaining
vectors.
\[ \vec{v} = \begin{bmatrix}1 \\ 2 \\ 3\end{bmatrix} \quad
  \vec{u_1} = \begin{bmatrix}1 \\ 1 \\ 0\end{bmatrix} \quad
  \vec{u_2} = \begin{bmatrix}0 \\ 1 \\ 1\end{bmatrix} \]
\begin{align*}
  A &= \begin{bmatrix}
    1 & 0 & 1 \\
    1 & 1 & 2 \\
    0 & 1 & 3
  \end{bmatrix} \\
  &= \begin{bmatrix}
    1 & 0 & 1 \\
    1 & 0 & -1 \\
    0 & 1 & 3
  \end{bmatrix} \\
  & \text{No}
\end{align*}

\subsubsection*{Exercise 5}
Determine if the vector \( \vec{v} \) is a linear combination of the remaining
vectors.
\[ \vec{v} = \begin{bmatrix}1 \\ 2 \\ 3\end{bmatrix} \quad
  \vec{u_1} = \begin{bmatrix}1 \\ 1 \\ 0\end{bmatrix} \quad
  \vec{u_2} = \begin{bmatrix}0 \\ 1 \\ 1\end{bmatrix} \quad
  \vec{u_3} = \begin{bmatrix}1 \\ 0 \\ 1\end{bmatrix} \]
\begin{align*}
  A &= \begin{bmatrix}
    1 & 0 & 1 & 1 \\
    1 & 1 & 0 & 2 \\
    0 & 1 & 1 & 3
  \end{bmatrix} \\
  &= \begin{bmatrix}
    1 & 0 & 1 & 1 \\
    0 & 1 & -1 & 1 \\
    0 & 1 & 1 & 3
  \end{bmatrix} \\
  &= \begin{bmatrix}
    1 & 0 & 1 & 1 \\
    0 & 2 & 0 & 4 \\
    0 & 1 & 1 & 3
  \end{bmatrix} \\
  &= \begin{bmatrix}
    1 & 0 & 1 & 1 \\
    0 & 1 & 0 & 2 \\
    0 & 0 & 1 & 1
  \end{bmatrix} \\
  & \text{Yes}
\end{align*}

\subsubsection*{Exercise 7}
Determine if the vector \( \vec{v} \) is in the span of the columns of the
matrix \( A \).
\[ A = \begin{bmatrix}1 & 2 \\ 3 & 4\end{bmatrix} \quad
  \vec{b} = \begin{bmatrix}5 \\ 6\end{bmatrix} \]
\begin{align*}
  A|\vec{b} &= \begin{bmatrix}
    1 & 2 & 5 \\
    3 & 4 & 6
  \end{bmatrix} \\
  &= \begin{bmatrix}
    1 & 2 & 5 \\
    1 & 0 & -4
  \end{bmatrix} \\
  &= \begin{bmatrix}
    0 & 2 & 9 \\
    1 & 0 & -4
  \end{bmatrix} \\
\end{align*}
There is a unique solution for:
\begin{align*}
  1a+2b &= 5 \\
  3a+4b &= 6
\end{align*}
\[ \begin{bmatrix}a \\ b\end{bmatrix} =\
  \begin{bmatrix}-1 \\ \frac{9}{2}\end{bmatrix} \]

\subsubsection*{Exercise 9}
Show that \( \R^2 = span\left(\begin{bmatrix}1 \\ 1\end{bmatrix},
\begin{bmatrix}1 \\ -1\end{bmatrix}\right) \)
\begin{align*}
  \begin{bmatrix}
    1 & 1 & a \\
    1 & -1 & b
  \end{bmatrix} &= \begin{bmatrix}
    1 & 1 & a \\
    2 & 0 & a+b
  \end{bmatrix} \\
  &= \begin{bmatrix}
    1 & 1 & a \\
    1 & 0 & \frac{a+b}{2}
  \end{bmatrix} \\
  &= \begin{bmatrix}
    0 & 1 & a-\frac{a+b}{2} \\
    1 & 0 & \frac{a+b}{2}
  \end{bmatrix} \\
  x &= a-\frac{a+b}{2} \\
  y &= \frac{a+b}{2}
\end{align*}
Any point in \( \R^2 \) can be described by a linear combination of the two
vectors using the constants described above.

\subsubsection*{Exercise 11}
Show that \( \R^3 = span\left(\begin{bmatrix}1 \\ 0 \\ 1\end{bmatrix},
\begin{bmatrix}1 \\ 1 \\0\end{bmatrix},\begin{bmatrix}0 \\ 1 \\ 1\end{bmatrix}
\right) \)
\begin{align*}
  \begin{bmatrix}
    1 & 1 & 0 & a \\
    0 & 1 & 1 & b \\
    1 & 0 & 1 & c
  \end{bmatrix} &= \begin{bmatrix}
    1 & 0 & -1 & a-b \\
    0 & 1 & 1 & b \\
    1 & 0 & 1 & c
  \end{bmatrix} \\
  &= \begin{bmatrix}
    2 & 0 & 0 & a-b+c \\
    0 & 1 & 1 & b \\
    1 & 0 & 1 & c
  \end{bmatrix} \\
  &= \begin{bmatrix}
    1 & 0 & 0 & \frac{a-b+c}{2} \\
    0 & 1 & 1 & b \\
    0 & 0 & 1 & c-\frac{a-b+c}{2}
  \end{bmatrix} \\
  &= \begin{bmatrix}
    1 & 0 & 0 & \frac{a-b+c}{2} \\
    0 & 1 & 0 & b-(c-\frac{a-b+c}{2}) \\
    0 & 0 & 1 & c-\frac{a-b+c}{2}
  \end{bmatrix} \\
  x &= \frac{a-b+c}{2} \\
  y &= b-(c-\frac{a-b+c}{2}) \\
  z &= c-\frac{a-b+c}{2}
\end{align*}
Any point in \( \R^3 \) can be described by a linear combination of the two
vectors using the constants described above.

\subsubsection*{Exercise 13b}
Describe the span of the given vectors algebraically.
\[ \begin{bmatrix}2 \\ -4\end{bmatrix},\begin{bmatrix}-1 \\ 2\end{bmatrix} \]
\begin{align*}
  s\begin{bmatrix}2 \\ -4\end{bmatrix}+t\begin{bmatrix}-1 \\ 2\end{bmatrix} &=
    \begin{bmatrix}x \\ y\end{bmatrix} \\
  2s-t &= x\\
  -4s+2t &= y \\
  \begin{bmatrix}
    2 & -1 & x \\
    -4 & 2 & y
  \end{bmatrix} &= \begin{bmatrix}
    2 & -1 & x \\
    0 & 0 & y+2x
  \end{bmatrix} \\
  y+2x &= 0 \\
  y &= -2x
\end{align*}
The span of the two vectors describes the set of all vectors parallel and
antiparallel to the given vectors, which line on the line \( y = -2x \).

\subsubsection*{Exercise 15b}
Describe the span of the given vectors algebraically.
\[ \begin{bmatrix}1 \\ 2 \\ 0\end{bmatrix},
  \begin{bmatrix}3 \\ 2 \\ -1\end{bmatrix} \]
\begin{align*}
  s\begin{bmatrix}1 \\ 2 \\ 0\end{bmatrix}+
    t\begin{bmatrix}3 \\ 2 \\ -1\end{bmatrix} &=
    \begin{bmatrix}x \\ y \\ z\end{bmatrix} \\
  s+3t &= x \\
  2s+2t &= y \\
  -t &= z \\
  \begin{bmatrix}
    1 & 3 & x \\
    2 & 2 & y \\
    0 & -1 & z
  \end{bmatrix} &= \begin{bmatrix}
    1 & 0 & x+3z \\
    2 & 0 & y+2z \\
    0 & -1 & z
  \end{bmatrix} \\
  &= \begin{bmatrix}
    1 & 0 & x+3z \\
    0 & 0 & y+2z-2(x+3z) \\
    0 & -1 & z
  \end{bmatrix} \\
  0 &= y+2z-2x-6z \\
  &= -2x+y-4z
\end{align*}
The span of the vectors describes a plane with the equation \( 0 = -2x+y-4z \).

\subsubsection*{Exercise 17}
The general equation of the plane that contains the points (1,0,3),(-1,1,-3),
and the origin is of the form \( ax+by+cz = 0 \). Solve for \( a, b, c \).
\begin{align*}
  \begin{bmatrix}
    1 & -1 & 0 & x \\
    0 & 1 & 0 & y \\
    3 & -3 & 0 & z
  \end{bmatrix} &= \begin{bmatrix}
    1 & 0 & 0 & x+y \\
    0 & 1 & 0 & y \\
    1 & -1 & 0 & \frac{z}{3}
  \end{bmatrix} \\
  &= \begin{bmatrix}
    1 & 0 & 0 & x+y \\
    0 & 1 & 0 & y \\
    1 & 0 & 0 & \frac{z}{3}+y
  \end{bmatrix} \\
  &= \begin{bmatrix}
    1 & 0 & 0 & x+y \\
    0 & 1 & 0 & y \\
    0 & 0 & 0 & \frac{z}{3}+y-(x+y)
  \end{bmatrix} \\
  0 &= \frac{z}{3}+y-x-y \\
  &= -x+\frac{z}{3} \\
  &= -3x+z \\
  a &= -3 \\
  b &= 0 \\
  c &= 1
\end{align*}

\subsubsection*{Exercise 18}
Prove that \( \vec{u},\vec{v},\vec{w} \) are all in
\( span(\vec{u},\vec{v},\vec{w}) \).
\begin{align*}
  \vec{u} &= 1\vec{u}+0\vec{v}+0\vec{w} \\
  \vec{v} &= 0\vec{u}+1\vec{v}+0\vec{w} \\
  \vec{w} &= 0\vec{u}+0\vec{v}+1\vec{w} \\
  \vec{u},\vec{v},\vec{w} &\in span(\vec{u},\vec{v},\vec{w})
\end{align*}

\subsubsection*{Exercise 19}
\begin{align*}
  \vec{u} &= 1\vec{u}+0(\vec{u}+\vec{v})+0(\vec{u}+\vec{v}+\vec{w}) \\
  \vec{v} &= -1\vec{u}+1(\vec{u}+\vec{v})+0(\vec{u}+\vec{v}+\vec{w}) \\
  \vec{w} &= 0\vec{u}-1(\vec{u}+\vec{v})+1(\vec{u}+\vec{v}+\vec{w}) \\
  \vec{u},\vec{v},\vec{w} &\in
    span(\vec{u},\vec{u}+\vec{v},\vec{u}+\vec{v}+\vec{w})
\end{align*}

\subsubsection*{Exercise 20}
Prove that if \( \vec{u_1},\dots,\vec{u_m}\in\R^n \) where \( S = \{\vec{u_1},
\dots,\vec{u_k}\} \) and \( T = \{\vec{u_1},\dots,\vec{u_k},\vec{u_{k+1}},\dots,
\vec{u_m}\} \) that \( span(S)\subseteq span(T) \).
\begin{align*}
  c_1\vec{u_1}+c_2\vec{u_2}+\dots+c_k\vec{u_k} &\in span(S) \\
  c_1\vec{u_1}+c_2\vec{u_2}+\dots+c_k\vec{u_k} &=
    c_1\vec{u_1}+\dots+c_k\vec{u_k}+0\vec{u_{k+1}}+\dots+0\vec{u_m} \\
  c_1\vec{u_1}+\dots+c_k\vec{u_k}+0\vec{u_{k+1}}+\dots+0\vec{u_m} &\in
    span(T) \\
  span(S) &\subseteq span(T) \\
\end{align*}
Deduce if \( \R^n = span(S) \), then \( \R^n = span(T) \).
\begin{align*}
  \R^n &= span(S) \\
  span(S) &\subseteq span(T) \\
  span(T) &\supseteq \R^n \\
  span(T) &= \R^n \\
\end{align*}

\subsubsection*{Exercise 21}
Suppose \( \vec{w} \) is a linear combination of \( \vec{u_1},\vec{u_2},\dots,
\vec{u_k} \) and each \( \vec{u_1} \) is a linear combination of \( \vec{v_1},
\dots,\vec{v_m} \). Prove \( \vec{w} \) is a linear combination of \( \vec{v_1},
\dots,\vec{v_m} \).
\begin{align*}
  \vec{w} &= \sum_{i=1}^{k}c_iu_i \\
  &= \sum_{i=1}^{k}c_i\left(\sum_{j=1}^{m}d_{ij}\vec{v_j}\right) \\
  &= \sum_{i=1}^{k}\left(\sum_{j=1}^{m}c_id_{ij}\vec{v_j}\right) \\
  &= \sum_{j=1}^{m}\left(\sum_{i=1}^{k}c_id_{ij}\right)\vec{v_i} \\
  w &\in span(\vec{v_1},\dots,\vec{v_m})
\end{align*}
Also suppose each \( \vec{v_j} \) is a linear combination of \( \vec{u_1},\dots,
\vec{u_k} \). Prove \( span(\vec{u_1},\dots,\vec{u_k}) =
span(\vec{v_1},\dots,\vec{v_m}) \). Above we proved:
\[ span(\vec{u_1},\dots,\vec{u_k}) \subseteq span(\vec{v_1},\dots,\vec{v_m}) \]
Therefore:
\[ span(\vec{u_1},\dots,\vec{u_k}) \supseteq span(\vec{v_1},\dots,\vec{v_m}) \]
Use the result above to prove that:
\[ \R^3 = span\left(\begin{bmatrix}1 \\ 0 \\ 0\end{bmatrix},
  \begin{bmatrix}1 \\ 1 \\ 0\end{bmatrix},
  \begin{bmatrix}1 \\ 1 \\ 1\end{bmatrix}\right) \]
\begin{align*}
  \R^3 &= span(\vec{e_1},\vec{e_2},\vec{e_3}) \\
  \begin{bmatrix}1 \\ 0 \\ 0\end{bmatrix} &= \vec{e_1} \\
  \begin{bmatrix}1 \\ 1 \\ 0\end{bmatrix} &= \vec{e_1}+\vec{e_2} \\
  \begin{bmatrix}1 \\ 1 \\ 1\end{bmatrix} &= \vec{e_1}+\vec{e_2}+\vec{e_3} \\
  span(\vec{e_1},\vec{e_2},\vec{e_3}) &\subseteq span\left(
    \begin{bmatrix}1 \\ 0 \\ 0\end{bmatrix},
    \begin{bmatrix}1 \\ 1 \\ 0\end{bmatrix},
    \begin{bmatrix}1 \\ 1 \\ 1\end{bmatrix}\right) \\
  \R^3 &= span\left(
    \begin{bmatrix}1 \\ 0 \\ 0\end{bmatrix},
    \begin{bmatrix}1 \\ 1 \\ 0\end{bmatrix},
    \begin{bmatrix}1 \\ 1 \\ 1\end{bmatrix}\right)
\end{align*}

\subsubsection*{Exercise 23}
Determine if the sets of vectors are linearly independent.
\[ \begin{bmatrix}1 \\ 1 \\ 1\end{bmatrix},
  \begin{bmatrix}1 \\ 2 \\ 3\end{bmatrix},
  \begin{bmatrix}1 \\ -1 \\ 2\end{bmatrix} \]
\begin{align*}
  \begin{bmatrix}
    1 & 1 & 1 & 0 \\
    1 & 2 & -1 & 0 \\
    1 & 3 & 2 & 0
  \end{bmatrix} &= \begin{bmatrix}
    0 & 1 & 3 & 0 \\
    1 & 0 & -2 & 0 \\
    0 & 0 & -5 & 0
  \end{bmatrix} \\
  &= \begin{bmatrix}
    1 & 0 & 0 & 0 \\
    0 & 1 & 0 & 0 \\
    0 & 0 & 1 & 0
  \end{bmatrix}
\end{align*}
Linearly independent.

\subsubsection*{Exercise 25}
Determine if the sets of vectors are linearly independent.
\[ \begin{bmatrix}0 \\ 1 \\ 2\end{bmatrix},
  \begin{bmatrix}2 \\ 1 \\ 3\end{bmatrix},
  \begin{bmatrix}2 \\ 0 \\ 1\end{bmatrix} \]
\[ \begin{bmatrix}0 \\ 1 \\ 2\end{bmatrix}+
  \begin{bmatrix}2 \\ 0 \\ 1\end{bmatrix} =
  \begin{bmatrix}2 \\ 1 \\ 3\end{bmatrix} \]
Linearly dependent.

\subsubsection*{Exercise 27}
Determine if the sets of vectors are linearly independent.
\[ \begin{bmatrix}3 \\ 4 \\ 5\end{bmatrix},
  \begin{bmatrix}6 \\ 7 \\ 8\end{bmatrix},
  \begin{bmatrix}0 \\ 0 \\ 0\end{bmatrix} \]
Linearly dependent, \( \vec{0} \) is linearly dependent to all other vectors.

\subsubsection*{Exercise 29}
Determine if the sets of vectors are linearly independent.
\[ \begin{bmatrix}1 \\ -1 \\ 1 \\ 0\end{bmatrix},
  \begin{bmatrix}-1 \\ 1 \\ 0 \\ 1\end{bmatrix},
  \begin{bmatrix}1 \\ 0 \\ 1 \\ -1\end{bmatrix},
  \begin{bmatrix}0 \\ 1 \\ -1 \\ 1\end{bmatrix} \]
\begin{align*}
  \begin{bmatrix}
    1 & -1 & 1 & 0 & 0 \\
    -1 & 1 & 0 & 1 & 0 \\
    1 & 0 & 1 & -1 & 0 \\
    0 & 1 & -1 & 1 & 0
  \end{bmatrix} &= \begin{bmatrix}
    1 & 0 & 1 & -1 & 0 \\
    0 & 1 & 0 & -1 & 0 \\
    0 & 0 & 1 & 1 & 0 \\
    0 & 0 & 0 & -1 & 0
  \end{bmatrix} \\
  &= \begin{bmatrix}
    1 & 0 & 0 & 0 & 0 \\
    0 & 1 & 0 & 0 & 0 \\
    0 & 0 & 1 & 0 & 0 \\
    0 & 0 & 0 & 1 & 0
  \end{bmatrix} \\
\end{align*}
Linearly independent.

\subsubsection*{Exercise 31}
Determine if the sets of vectors are linearly independent.
\[ \begin{bmatrix}3 \\ -1 \\ 1 \\ -1\end{bmatrix},
  \begin{bmatrix}-1 \\ 3 \\ 1 \\ -1\end{bmatrix},
  \begin{bmatrix}1 \\ 1 \\ 3 \\ 1\end{bmatrix},
  \begin{bmatrix}-1 \\ -1 \\ 1 \\ 3\end{bmatrix} \]
\begin{align*}
  \begin{bmatrix}
    3 & -1 & 1 & -1 & 0 \\
    -1 & 3 & 1 & -1 & 0 \\
    1 & 1 & 3 & 1 & 0 \\
    -1 & -1 & 1 & 3 & 0
  \end{bmatrix} &= \begin{bmatrix}
    2 & 2 & 2 & -2 & 0 \\
    0 & 4 & 4 & 0 & 0 \\
    0 & 0 & 4 & 4 & 0 \\
    -1 & -1 & 1 & 3 & 0
  \end{bmatrix} \\
  &= \begin{bmatrix}
    0 & 0 & 4 & 4 & 0 \\
    0 & 4 & 4 & 0 & 0 \\
    0 & 0 & 4 & 4 & 0 \\
    -1 & -1 & 1 & 3 & 0
  \end{bmatrix} \\
  &= \begin{bmatrix}
    -1 & -1 & 1 & 3 & 0 \\
    0 & 1 & 1 & 0 & 0 \\
    0 & 0 & 1 & 1 & 0 \\
    0 & 0 & 0 & 0 & 0 \\
  \end{bmatrix}
\end{align*}
Linearly dependent, infinite solutions.

\subsubsection*{Exercise 42a}
If the columns of an \( n\times n \) matrix \( A \) are linearly independent
as vectors in \( \R^n \), what is the rank of \( A \)? Explain.
The rank of \( A \) is \( n \) because there exists a non-trivial solution. It
is possible to reduce \( A \) to reduced row echelon form with no row of 0s
at the bottom.

\subsubsection*{Exercise 44}
Prove two vectors are linearly dependent if and only if one is a scalar multiple
of the other.
\begin{align*}
  \vec{v_1} &= c\vec{v_2} \\
  1\vec{v_1}-c\vec{v_2} &= \vec{0}
\end{align*}

\subsubsection*{Exercise 46}
Prove that every subset of a linearly independent set is linearly independent.
Let \( \vec{v_1},\vec{v_2},\dots,\vec{v_n} \) be a linearly independent set.
\[ c_1\vec{v_1}+c_2\vec{v_2}+\dots+c_n\vec{v_n} \ne \vec{0} \]
By Theorem 2.5, \( \vec{v},\vec{v_2},\dots,\vec{v_n} \) are linearly dependent
if and only if at least one of the vectors can be expressed as a linear
combination of the others. If no such set of scalars exist, then no subset
of \( \vec{v_1},\vec{v_2},\dots,\vec{v_n} \) can be linearly dependent. Thus,
every subset of a linearly independent set is linearly independent.

\subsubsection*{Exercise 47}
Suppse that \( S = \{\vec{v_1},\dots,\vec{v_k},\vec{v}\} \) is a set of vectors
in some \( \R^n \) and that \( \vec{v} \) is a linear combination of
\( \vec{v_1},\dots,\vec{v_k} \). If \( S' = \{\vec{v_1},\dots,\vec{v_k}\} \),
prove that \( span(S) = span(S') \).
\[ \vec{v} = c_1\vec{v_1}+\dots+c_k\vec{v_k} \]
If a vector \( \vec{w}\in span(S) \):
\begin{align*}
  \vec{w} &= d_1\vec{v_1}+\dots+d_k\vec{v_k}+d\vec{v} \\
  &= d_1\vec{v_1}+\dots+d_k\vec{v_k}+d(c_1\vec{v_1}+\dots+c_k\vec{v_k}) \\
  &= d_1\vec{v_1}+\dots+d_k\vec{v_k}+dc_1\vec{v_1}+\dots+dc_k\vec{v_k}) \\
  &= (d_1+dc_1)\vec{v_1}+\dots+(d_k+dc_k)\vec{v_k} \\
  \therefore \vec{w} &\in span(S') \\
  span(S) &\subseteq span(S')
\end{align*}
If a vector \( \vec{x}\in span(S') \):
\begin{align*}
  \vec{x} &= a_1\vec{v_1}+\dots+a_k\vec{v_k}+0\vec{v} \\
  \therefore \vec{x}&\in span(S) \\
  span(S') &\subseteq span(S) \\
  \therefore span(S) &= span(S')
\end{align*}

\subsubsection*{Exercise 48}
Let \( \{\vec{v_1},\dots,\vec{v_k}\} \) be a linearly independent set of vectors
in \( \R^n \), and let \( \vec{v} \) be a vector in \( \R^n \). Suppose that
\( \vec{v} = c_1\vec{v_1}+c_2\vec{v_2}+\dots+c_k\vec{v_k} \) with \( c_1 \ne 0
\). Prove that \( \{\vec{v},\vec{v_2},\dots,\vec{v_k}\} \) is linearly
independent.
\[ ? \]

\section*{Section 3.1}
Let:
\[ A = \begin{bmatrix}3 & 0 \\ -1 & 5\end{bmatrix},
  B = \begin{bmatrix}4 & -2 & 1 \\ 0 & 2 & 3\end{bmatrix},
  C = \begin{bmatrix}1 & 2 \\ 3 & 4 \\ 5 & 6\end{bmatrix} \]
\[ D = \begin{bmatrix}0 & -3 \\ -2 & 1\end{bmatrix},
  E = \begin{bmatrix}4 & 2\end{bmatrix},
  F = \begin{bmatrix}-1 \\ 2\end{bmatrix} \]

\subsubsection*{Exercise 1}
Compute the indicated matrices.
\begin{align*}
  A+2D &= \begin{bmatrix}
    3 & 0 \\
    -1 & 5
  \end{bmatrix}+2\begin{bmatrix}
    0 & -3 \\
    -2 & 1
  \end{bmatrix} \\
  &= \begin{bmatrix}
    3 & -6 \\
    -5 & 7
  \end{bmatrix}
\end{align*}

\subsubsection*{Exercise 3}
Compute the indicated matrices.
\begin{align*}
  B-C
\end{align*}
Undefined.

\subsubsection*{Exercise 5}
Compute the indicated matrices.
\begin{align*}
  AB &= \begin{bmatrix}
    12 & -6 & 3 \\
    -4 & 12 & 14
  \end{bmatrix}
\end{align*}

\subsubsection*{Exercise 7}
Compute the indicated matrices.
\begin{align*}
  D+BC &= \begin{bmatrix}
    0 & -3 \\
    -2 & 1
  \end{bmatrix}+\begin{bmatrix}
    3 & 6 \\
    21 & 26
  \end{bmatrix} \\
  &= \begin{bmatrix}
    3 & 3 \\
    19 & 27
  \end{bmatrix}
\end{align*}

\subsubsection*{Exercise 9}
Compute the indicated matrices.
\begin{align*}
  E(AF) &= \begin{bmatrix}
    4 & 2
  \end{bmatrix}\begin{bmatrix}
    -3 \\ 11
  \end{bmatrix} \\
  &= \begin{bmatrix}10\end{bmatrix}
\end{align*}

\subsubsection*{Exercise 11}
Compute the indicated matrices.
\begin{align*}
  FE &= \begin{bmatrix}
    -4 & -2 \\
    8 & 4
  \end{bmatrix}
\end{align*}

\subsubsection*{Exercise 13}
Compute the indicated matrices.
\begin{align*}
  B^TC^T-(CB)^T &= \begin{bmatrix}
    4 & 0 \\
    -2 & 2 \\
    1 & 3
  \end{bmatrix}\begin{bmatrix}
    1 & 3 & 5 \\
    2 & 4 & 6
  \end{bmatrix}-\begin{bmatrix}
    4 & 2 & 7 \\
    12 & 2 & 15 \\
    20 & 2 & 23
  \end{bmatrix}^T \\
  &= \begin{bmatrix}
    4 & 12 & 20 \\
    2 & 2 & 2 \\
    7 & 15 & 23
  \end{bmatrix}-\begin{bmatrix}
    4 & 12 & 20 \\
    2 & 2 & 2 \\
    7 & 15 & 23
  \end{bmatrix} \\
  &= \begin{bmatrix}
    0 & 0 & 0 \\
    0 & 0 & 0 \\
    0 & 0 & 0
  \end{bmatrix}
\end{align*}

\subsubsection*{Exercise 15}
Compute the indicated matrices.
\begin{align*}
  A^3 &= \begin{bmatrix}
    9 & 0 \\
    -8 & 25
  \end{bmatrix}A \\
  &= \begin{bmatrix}
    27 & 0 \\
    -49 & 125
  \end{bmatrix}
\end{align*}

\subsubsection*{Exercise 17}
Given an example of a nonzero \( 2\times2 \) matrix \( A \) such that
\( A^2 = 0 \).
\[ \begin{bmatrix}1 & 1 \\ -1 & -1\end{bmatrix} \]

\subsubsection*{Exercise 21}
Write the given system of linear equations as a matrix equation of the form
\( A\vec{x} = \vec{b} \).
\begin{align*}
  x_1-2x_2+3x_3 &= 0 \\
  2x_1+x_2-5x_3 &= 4 \\
  \begin{bmatrix}
    1 & -2 & 3 \\
    2 & 1 & -5
  \end{bmatrix}\begin{bmatrix}
    x_1 \\ x_2 \\ x_3
  \end{bmatrix} &= \begin{bmatrix}
    0 \\ 4
  \end{bmatrix}
\end{align*}

\subsubsection*{Exercise 22}
Write the given system of linear equations as a matrix equation of the form
\( A\vec{x} = \vec{b} \).
\begin{align*}
  -x_1 + 2x_3 &= 1 \\
  x_1-x_2 &= -2 \\
  x_2+x_3 &= -1 \\
  \begin{bmatrix}
    -1 & 0 & 2 \\
    1 & -1 & 0 \\
    0 & 1 & 1
  \end{bmatrix}\begin{bmatrix}
    x_1 \\ x_2 \\ x_3
  \end{bmatrix} &= \begin{bmatrix}
    1 \\ -2 \\ -1
  \end{bmatrix}
\end{align*}

\subsubsection*{Exercise 35}
Let \( A = \begin{bmatrix}0 & 1 \\ -1 & 1\end{bmatrix} \). Compute \( A^2,A^3,
\dots,A^7 \). What is \( A^{2015} \)?
\begin{align*}
  A^2 &= \begin{bmatrix}-1 & 1 \\ -1 & 0\end{bmatrix} \\
  A^3 &= \begin{bmatrix}-1 & 0 \\ 0 & -1\end{bmatrix} \\
  A^4 &= \begin{bmatrix}0 & -1 \\ 1 & -1\end{bmatrix} \\
  A^5 &= \begin{bmatrix}1 & -1 \\ 1 & 0\end{bmatrix} \\
  A^6 &= \begin{bmatrix}1 & 0 \\ 1 & 0\end{bmatrix} = I_2 \\
  A^7 &= A^1 \\
  A^{2015} &= A^{6(370)+5} \\
  &= (A^6)^{370}A^5 \\
  &= (I_2)^{370}A^5 \\
  &= \begin{bmatrix}1 & -1 \\ 1 & 0\end{bmatrix}
\end{align*}

\subsubsection*{Exercise 36}
Let:
\[ B = \begin{bmatrix}
  \frac{1}{\sqrt{2}} & -\frac{1}{\sqrt{2}} \\
  \frac{1}{\sqrt{2}} & \frac{1}{\sqrt{2}}
\end{bmatrix} \]
Find \( B^{2015} \).
\begin{align*}
  B^2 &= \begin{bmatrix}0 & -1 \\ 1 & 0\end{bmatrix} \\
  B^3 &= \begin{bmatrix}
    -\frac{1}{\sqrt{2}} & -\frac{1}{\sqrt{2}} \\
    \frac{1}{\sqrt{2}} & -\frac{1}{\sqrt{2}}
  \end{bmatrix} \\
  B^4 &= \begin{bmatrix}
    -\frac{1}{\sqrt{2}} & -\frac{1}{\sqrt{2}} \\
    \frac{1}{\sqrt{2}} & -\frac{1}{\sqrt{2}}
  \end{bmatrix}
\end{align*}

\subsubsection*{Exercise 37}
Let \( A = \begin{bmatrix}1 & 1 \\ 0 & 1\end{bmatrix} \). Find a formula for
\( A^n(n\ge1) \) and verify your formula using mathematical induction.

\subsubsection*{Exercise 38}
Let \( A = \begin{bmatrix}\cos\theta & -\sin\theta \\ \sin\theta & \cos\theta
\end{bmatrix} \). Show that \( A^2 = \begin{bmatrix}\cos2\theta & -\sin2\theta
\\ \sin2\theta & \cos2\theta\end{bmatrix} \).
\begin{align*}
  A\times A &= \begin{bmatrix}
    \cos\theta & -\sin\theta \\
    \sin\theta & \cos\theta
  \end{bmatrix}\begin{bmatrix}
    \cos\theta & -\sin\theta \\
    \sin\theta & \cos\theta
  \end{bmatrix} \\
  &= \begin{bmatrix}
    \cos^2\theta-\sin^2\theta & -\cos\theta\sin\theta-\sin\theta\cos\theta \\
    \sin\theta\cos\theta+\cos\theta\sin\theta & -\sin^2\theta+\cos^2\theta
  \end{bmatrix} \\
  &= \begin{bmatrix}
    \cos(\theta+\theta) & -\sin(\theta+\theta) \\
    \sin(\theta+\theta) & \cos(\theta+\theta)
  \end{bmatrix} \\
  &= \begin{bmatrix}
    \cos(2\theta) & -\sin(2\theta) \\
    \sin(2\theta) & \cos(2\theta)
  \end{bmatrix}
\end{align*}
Prove by mathematical induction that:
\[ A^n = \begin{bmatrix}
  \cos(n\theta) & -\sin(n\theta) \\
  \sin(n\theta) & \cos(n\theta)
\end{bmatrix} \quad \text{ for } n\ge 1 \]
Base Case (\( n = 1 \)): Obvious. \\
Induction Step: Assume it is true for \( n \). Prove it is true for \( n+1 \):
\begin{align*}
  A^{n+1} &= A^nA \\
  &= \begin{bmatrix}
    \cos(n\theta) & -\sin(n\theta) \\
    \sin(n\theta) & \cos(n\theta)
  \end{bmatrix}\begin{bmatrix}
    \cos\theta & -\sin\theta \\
    \sin\theta & \cos\theta
  \end{bmatrix} \\
  &= \begin{bmatrix}
    \cos(n\theta)\cos(\theta)-\sin(n\theta)\sin(\theta) &
      \cos(n\theta)(-\sin(\theta))-\sin(n\theta)\cos(\theta) \\
    \sin(n\theta)\cos(\theta)+\cos(n\theta)\sin(\theta) &
      -\sin(n\theta)\sin(\theta)+\cos(n\theta)\cos(\theta)
  \end{bmatrix} \\
  &= \begin{bmatrix}
    \cos(n\theta+\theta) & -\sin(n\theta+\theta) \\
    \sin(n\theta+\theta) & \cos(n\theta+\theta)
  \end{bmatrix} \\
  &= \begin{bmatrix}
    \cos((n+1)\theta) & -\sin((n+1)\theta) \\
    \sin((n+1)\theta) & \cos((n+1)\theta)
  \end{bmatrix}
\end{align*}

\begin{center}
  If you have any questions, comments, or concerns, please contact me at
  alvin@omgimanerd.tech
\end{center}

\end{document}
