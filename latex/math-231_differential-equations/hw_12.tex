\documentclass{math}

\usepackage{tikz}
\usetikzlibrary{arrows.meta}

\geometry{letterpaper, margin=0.5in}

\title{Differential Equations: Homework 12}
\author{Alvin Lin}
\date{January 2018 - May 2018}

\begin{document}

\maketitle
\clearpage

\section*{Section 7.6}

\subsubsection*{Exercise 2}
Sketch the graph of the given function and determine its Laplace transform.
\[ u(t-1)-u(t-4) \]
\begin{center}
  \begin{tikzpicture}[shorten >=-2.5pt,shorten <=-2.5pt]
    \draw[<->] (0,-2) -- (0,2);
    \draw[->] (0,0) -- (5,0);
    \draw[thick,-{Circle[open]}] (0,0) -- (1,0);
    \draw[thick,{Circle[open]}-{Circle[open]}] (1,1) -- (4,1);
    \draw[thick,{Circle[open]}-] (4,0) -- (5,0);
  \end{tikzpicture}
\end{center}
\begin{align*}
  \laplace{u(t-1)-u(t-4)} &= \laplace{u(t-1)}-\laplace{u(t-4)} \\
  &= \frac{\e^{-s}}{s}-\frac{\e^{-4s}}{s}
\end{align*}

\subsubsection*{Exercise 5}
Express the given function using window and step functions and compute its
Laplace transform.
\begin{align*}
  g(t) &= \begin{cases}
    0, &\quad 0<t<1 \\
    2, &\quad 1<t<2 \\
    1, &\quad 2<t<3 \\
    3, &\quad 3<t
  \end{cases}
\end{align*}
\begin{center}
  \begin{tikzpicture}[shorten >=-2.5pt,shorten <=-2.5pt]
    \draw[<->] (0,-2) -- (0,2);
    \draw[->] (0,0) -- (5,0);
    \draw[thick,-{Circle[open]}] (0,0) -- (1,0);
    \draw[thick,{Circle[open]}-{Circle[open]}] (1,2) -- (2,2);
    \draw[thick,{Circle[open]}-{Circle[open]}] (2,1) -- (3,1);
    \draw[thick,{Circle[open]}-] (3,3) -- (5,3);
  \end{tikzpicture}
\end{center}
\begin{align*}
  g(t) &= 0+2u(t-1)-u(t-2)+2u(t-3) \\
  \laplace{g(t)} &= 2\laplace{u(t-1)}-\laplace{u(t-2)}+2\laplace{u(t-3)} \\
  &= \frac{2\e^{-s}}{s}-\frac{\e^{-2s}}{s}+\frac{2\e^{-3s}}{s}
\end{align*}

\subsubsection*{Exercise 6}
Express the given function using window and step functions and compute its
Laplace transform.
\begin{align*}
  g(t) &= \begin{cases}
    0, &\quad 0<t<2 \\
    t+1, &\quad 2<t
  \end{cases} \\
  &= (t+1)u(t-2) \\
  \laplace{g(t)} &= \laplace{(t+1)u(t-2)} \\
  &= \e^{-2s}\laplace{(t+1)+2} \\
  &= \e^{-2s}\laplace{t+3} \\
  &= \e^{-2s}(\laplace{t}+3\laplace{1}) \\
  &= \e^{-2s}(\frac{1}{s^2}+\frac{3}{s})
\end{align*}

\subsubsection*{Exercise 11}
Determine an inverse Laplace transform of the given function.
\begin{align*}
  \ilaplace{\frac{\e^{-2s}}{s-1}} &= \ilaplace{\e^{-as}F(s)} \\
  &= f(t-a)u(t-a) \\
  a &= 2 \\
  F(s) &= \frac{1}{s-1} \\
  f(t) &= \e^t \\
  f(t-a)u(t-a) &= \e^{t-2}u(t-2)
\end{align*}

\subsubsection*{Exercise 12}
Determine an inverse Laplace transform of the given function.
\begin{align*}
  \ilaplace{\frac{\e^{-3s}}{s^2}} &= \ilaplace{\e^{-as}F(s)} \\
  &= f(t-a)u(t-a) \\
  a &= 3 \\
  F(s) &= \frac{1}{s^2} \\
  f(t) &= t \\
  f(t-a)u(t-a) &= (t-3)u(t-3)
\end{align*}

\subsubsection*{Exercise 13}
Determine an inverse Laplace transform of the given function.
\begin{align*}
  \ilaplace{\frac{\e^{-2s}-3\e^{-4s}}{s+2}} &=
    \ilaplace{\frac{\e^{-2s}}{s+2}}-3\ilaplace{\frac{\e^{-4s}}{s+2}} \\
  a_1 &= 2 \\
  F(s) &= \frac{1}{s+2} \\
  f(t) &= \e^{-2t} \\
  \ilaplace{\frac{\e^{-2s}}{s+2}} &= f(t-a_1)u(t-a_1) \\
  &= \e^{-2(t-2)}u(t-2) \\
  a_2 &= 4 \\
  \ilaplace{\frac{\e^{-4s}}{s+2}} &= f(t-a_2)u(t-a_2) \\
  &= \e^{-2(t-4)}u(t-4) \\
  \ilaplace{\frac{\e^{-2s}}{s+2}}-3\ilaplace{\frac{\e^{-4s}}{s+2}} &=
    \e^{-2t+4}u(t-2)-3\e^{-2t+8}u(t-4)
\end{align*}

\subsubsection*{Exercise 14}
Determine an inverse Laplace transform of the given function.
\begin{align*}
  \ilaplace{\frac{\e^{-3s}}{s^2+9}} &= \ilaplace{\e^{-as}F(s)} \\
  &= f(t-a)u(t-a) \\
  a &= 3 \\
  F(s) &= \frac{1}{s^2+9} \\
  &= \frac{1}{3}\frac{3}{s^2+9} \\
  f(t) &= \frac{1}{3}\sin(3t) \\
  \ilaplace{\e^{-as}F(s)} &= f(t-a)u(t-a) \\
  &= \frac{1}{3}\sin(3(t-3))u(t-3) \\
  &= \frac{1}{3}\sin(3t-9)u(t-3)
\end{align*}

\subsubsection*{Exercise 17}
Determine an inverse Laplace transform of the given function.
\begin{align*}
  \ilaplace{\frac{\e^{-3s}(s-5)}{(s+1)(s+2)}} &= \ilaplace{\e^{-as}F(s)} \\
  &= f(t-a)u(t-a) \\
  a &= 3 \\
  F(s) &= \frac{s-5}{(s+1)(s+2)} \\
  \frac{A}{s+1}+\frac{B}{s+2} &= \frac{s-5}{(s+1)(s+2)} \\
  A(s+2)+B(s+1) &= s-5 \\
  Let: s = -2 \quad B &= 7 \\
  Let: s = -1 \quad A &= -6 \\
  F(s) &= \frac{-6}{s+1}+\frac{7}{s+2} \\
  f(t) &= \ilaplace{F(s)} \\
  &= -6\ilaplace{\frac{1}{s+1}}+7\ilaplace{\frac{1}{s+2}} \\
  &= -6\e^{-t}+7\e^{-2t} \\
  f(t-a)u(t-a) &= (-6\e^{-(t-3)}+7\e^{-2(t-3)})u(t-3) \\
  &= (-6\e^{-t+3}+7\e^{-2t+6})u(t-3)
\end{align*}

\subsubsection*{Exercise 21}
Solve the given initial value problem using the method of Laplace transforms.
Sketch the graph of the solution.
\begin{align*}
  y''+y &= u(t-3) \quad y(0) = 0 \quad y'(0) = 1 \\
  \laplace{y''+y} &= \laplace{u(t-3)} \\
  s^2Y(s)-sy(0)-y'(0)+Y(s) &= \frac{\e^{-3s}}{s} \\
  s^2Y(s)-1+Y(s) &= \frac{\e^{-3s}}{s} \\
  Y(s)(s^2+1)-1 &= \frac{\e^{-3s}}{s} \\
  Y(s)(s^2+1) &= \frac{\e^{-3s}}{s}+1 \\
  Y(s) &= \frac{\e^{-3s}}{s(s^2+1)}+\frac{1}{s^2+1} \\
  y(t) &= \ilaplace{Y(s)} \\
\end{align*}

\subsubsection*{Exercise 29}
Solve the given initial value problem using the method of Laplace transforms.
Sketch the graph of the solution.
\begin{align*}
  y''+4y &= g(t) \quad y(0) = 1 \quad y'(0) = 3 \\
  g(t) &= \begin{cases}
    \sin(t), &\quad 0\le t\le 2\pi \\
    0, &\quad 2\pi<t
  \end{cases}
\end{align*}

\section*{Section 7.9}

\subsubsection*{Exercise 5}
Evaluate the given integral:
\begin{align*}
  \int_{0}^{\infty}\e^{-2t}\delta(t-1)\diff{t}
\end{align*}

\subsubsection*{Exercise 7}
Determine the Laplace transform of the given generalized function.
\begin{align*}
  \laplace{\delta(t-1)-\delta(t-3)}
\end{align*}

\subsubsection*{Exercise 8}
Determine the Laplace transform of the given generalized function.
\begin{align*}
  \laplace{\delta(t-1)}
\end{align*}

\subsubsection*{Exercise 13}
Solve the given symbolic initial value problem.
\begin{align*}
  w''+w &= \delta(t-\pi) \quad w(0) = 0 \quad w'(0) = 0
\end{align*}

\subsubsection*{Exercise 15}
Solve the given symbolic initial value problem.
\begin{align*}
  y''+2y'-3y &= \delta(t-1)-\delta(t-2) \quad y(0) = 2 \quad y'(0) = -2
\end{align*}

\subsubsection*{Exercise 16}
Solve the given symbolic initial value problem.
\begin{align*}
  y''-2y'-3y &= 2\delta(t-1)-\delta(t-3) \quad y(0) = 2 \quad y'(0) = 2
\end{align*}

\begin{center}
  If you have any questions, comments, or concerns, please contact me at
  alvin@omgimanerd.tech
\end{center}

\end{document}
