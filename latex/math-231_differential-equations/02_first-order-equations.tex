\documentclass{math}

\usepackage{tikz}

\title{Differential Equations}
\author{Alvin Lin}
\date{January 2018 - May 2018}

\begin{document}

\maketitle

\section*{Separable Equations}
Consider:
\[ \ddiff{y}{x} = p(y)g(x) \]
In this case, we can rewrite it as:
\[ \frac{\diff{y}}{p(y)} = g(x)\diff{x} \]
If we let \( h(y) = \frac{1}{p(y)} \):
\begin{align*}
  h(y)\diff{y} &= g(x)\diff{x} \\
  \int h(y)\diff{y} &= \int g(x)\diff{x} \\
  H(y) &= G(x)+c
\end{align*}
This yields an implicit solution. Separable equations can be linear or
non-linear.

\subsubsection*{Example}
Solve the following:
\[ \ddiff{y}{x} = 8x^3\e^{-2y} \]
\begin{align*}
  \ddiff{y}{x} &= 8x^3\e^{-2y} \\
  \frac{\diff{y}}{\e^{-2y}} &= 8x^3\diff{x} \\
  \e^{2y}\diff{y} &= 8x^3\diff{x} \\
  \int\e^{2y}\diff{y} &= \int 8x^3\diff{x} \\
  \frac{1}{2}\e^{2y}+c_1 &= 2x^4+c_2 \\
  \frac{1}{2}\e^{2y} &= 2x^4+c
\end{align*}
We can also solve this explicitly:
\begin{align*}
  \frac{1}{2}\e^{2y} &= 2x^4+c \\
  \ln(\e^{2y}) &= \ln(4x^4+c) \\
  2y &= \ln(4x^4+c) \\
  y &= \frac{\ln(4x^4+c)}{2}
\end{align*}

\subsubsection*{Example}
Solve the following initial value problem:
\[ \ddiff{y}{x} = (1+y^2)\tan(x) \quad y(0) = \sqrt{3} \]
This is a non-linear separable equation.
\begin{align*}
  \ddiff{y}{x} &= (1+y^2)\tan(x) \\
  \frac{\diff{y}}{1+y^2} &= \tan(x)\diff{x} \\
  \int\frac{\diff{y}}{1+y^2} &= \int\tan(x)\diff{x} \\
  \arctan(y) &= \ln|\sec(x)|+c
\end{align*}
Using our initial value \( y(0) = \sqrt{3} \):
\begin{align*}
  \arctan(\sqrt{3}) &= \ln|\sec(0)|+c \\
  \arctan(\sqrt{3}) &= \ln|1|+c \\
  \arctan(\sqrt{3}) &= c \\
  c &= \frac{\pi}{3} \text{ over } \left(\frac{-\pi}{2},\frac{\pi}{2}\right)
\end{align*}
We can now solve explicitly for \( y \):
\begin{align*}
  \arctan(y) &= \ln|\sec(x)|+\frac{\pi}{3} \\
  \tan(\arctan(y)) &= \tan\bigg(\ln|sec(x)|+\frac{\pi}{3}\bigg) \\
  y &= \tan\bigg(\ln|sec(x)|+\frac{\pi}{3}\bigg) \\
\end{align*}

\subsubsection*{Example}
Solve the following initial value problem over \( [0,\infty) \):
\[ \sqrt{y}\diff{x}+(1+x)\diff{y} = 0 \quad y(0) = 1 \]
\begin{align*}
  \sqrt{y}\diff{x}+(1+x)\diff{y} &= 0 \\
  (1+x)\diff{y} &= -\sqrt{y}\diff{x} \\
  \ddiff{y}{x} &= (-\sqrt{y})\frac{1}{1+x} \\
  \frac{\diff{y}}{\sqrt{y}} &= -\frac{\diff{x}}{1+x} \\
  \int y^{-\frac{1}{2}}\diff{y} &= -\int\frac{\diff{x}}{1+x} \\
  2y^{\frac{1}{2}} &= -\ln|1+x|+c
\end{align*}
Using our initial value \( y(0) = 1 \):
\begin{align*}
  2 &= -\ln|1|+c \\
  c &= 2
\end{align*}
We can now solve explicitly for \( y \):
\begin{align*}
  2\sqrt{y} &= 2-\ln|1+x| \\
  \sqrt{y} &= \frac{2-\ln|1+x|}{2} \\
  y &= \bigg[\frac{2-\ln|1+x|}{2}\bigg]^2
\end{align*}

\subsubsection*{Example}
Solve the initial value problem:
\[ x^2\ddiff{y}{x} = y-xy \quad y(1) = 1 \]
\begin{align*}
  x^2\ddiff{y}{x} &= y-xy \\
  x^2\ddiff{y}{x} &= y(1-x) \\
  \ddiff{y}{x} &= (y)\bigg(\frac{1-x}{x^2}\bigg) \\
  \frac{\diff{y}}{y} &= \frac{1-x}{x^2}\diff{x} \\
  \int\frac{\diff{y}}{y} &= \int\bigg[\frac{1}{x^2}-\frac{1}{x}\bigg]\diff{x} \\
  \ln|y| &= -\frac{1}{x}-\ln|x|+c \\
  \e^{\ln|y|} &= \e^{-\frac{1}{x}-\ln|x|+c} \\
  |y| &= \e^{-\frac{1}{x}}\e^{-\ln|x|}\e^c \\
  &= \e^{-\frac{1}{x}}\frac{1}{|x|}\e^c \\
  Let: k &= \pm\e^c \\
  y &= \frac{k\e^{-\frac{1}{x}}}{x}
\end{align*}
Using our initial value \( y(1) = 1 \):
\begin{align*}
  1 &= \frac{k\e^{-1}}{1} \\
  k &= \e
\end{align*}
We can now solve explicitly for \( y \):
\begin{align*}
  y &= \frac{k\e^{-\frac{1}{x}}}{x} \\
  y &= \frac{\e^{1-\frac{1}{x}}}{x}
\end{align*}

\subsubsection*{Example}
Solve the initial value problem:
\[ \frac{1}{\theta}\ddiff{y}{\theta} = \frac{y\sin\theta}{y^2+1} \quad
  y(\pi) = 1 \]
\begin{align*}
  \frac{1}{\theta}\ddiff{y}{\theta} &= \frac{y\sin\theta}{y^2+1} \\
  \ddiff{y}{\theta} &= (\theta\sin\theta)(\frac{y}{y^2+1}) \\
  \frac{y^2+1}{y}\diff{y} &= \theta\sin\theta\diff{\theta} \\
  \int\frac{y^2+1}{y}\diff{y} &= \int\theta\sin\theta\diff{\theta} \\
  \frac{y^2}{2}+\ln|y|+c &= -\theta\cos\theta-\int(-\cos\theta\diff{\theta}) \\
  &= -\theta\cos\theta+\sin\theta+c
\end{align*}
Using our initial value \( y(\pi) = 1 \):
\begin{align*}
  \frac{1}{2} &= -\pi\cos(\pi)+\sin(\pi)+c \\
  \frac{1}{2} &= -\pi(-1)+c \\
  c &= \frac{1}{2}-\pi
\end{align*}
We can now solve explicitly for \( y \):
\[ \frac{y^2}{2}+\ln|y| = -\theta\cos\theta+\sin\theta+(\frac{1}{2}-\pi) \]

\section*{First Order Linear Equations}
General form:
\[ a_1(x)\ddiff{y}{x}+a_0(x)y = g(x) \]
Let's consider:
\[ x\ddiff{y}{x}+y = x^2 \]
By the product rule we can see that the left hand side is equal to
\( \ddiff{y}{x}\bigg[xy\bigg] \):
\[ \ddiff{}{x}\bigg[xy\bigg] = x^2 \]
Integrating both sides:
\begin{align*}
  \int\ddiff{}{x}\bigg[xy\bigg]\diff{x} &= \int x^2\diff{x} \\
  xy &= \frac{x^3}{3}+c \\
  y &= \frac{x^2}{3}+\frac{c}{x}
\end{align*}

\subsection*{Standard Form}
Consider:
\[ a_1(x)\ddiff{y}{x}+a_0(x)y = g(x) \]
Write it in standard form:
\[ \ddiff{y}{x}+\frac{a_0(x)}{a_1(x)}y = \frac{g(x)}{a_1(x)} \]
Let \( P(x) = \frac{a_0(x)}{a_1(x)} \) and \( Q(x) = \frac{g(x)}{a_1(x)} \):
\[ \ddiff{y}{x}+P(x)y = Q(x) \]
This is a first order linear equation in standard form. In order to solve this,
we need to determine a standard function of \( x \) called \( \mu \). We
multiply the equation by \( \mu \):
\[ \mu(x)\ddiff{y}{x}+\mu(x)P(x)y = \mu(x)Q(x) \]
We require that that \( \mu'(x) = \mu(x)P(x) \). To find this function
\( \mu \):
\begin{align*}
  \frac{\mu'(x)}{\mu(x)} &= P(x) \\
  \int \frac{\mu'(x)}{\mu(x)} &= \mu P(x) \\
  \ln|\mu(x)| &= \int P(x)\diff{x}+c \\
  c &= 0 \\
  \e^{\ln|\mu|} &= \e^{\int P(x)\diff{x}} \\
  |\mu(x)| &= \e^{\int P(x)\diff{x}}
\end{align*}
This is called an integrating factor. From this, we can determine that:
\begin{align*}
  \ddiff{}{x}\bigg[\mu(x)y\bigg] &= \mu(x)Q(x) \\
  \int\ddiff{}{x}\bigg[\mu(x)y\bigg] &= \int\mu(x)Q(x) \\
  \mu(x)y &= \int\mu(x)Q(x)\diff{x} \\
  y &= \frac{1}{\mu(x)}\bigg[\mu(x)Q(x)\diff{x}\bigg] \\
  \mu(x) &= \e^{\int P(x)\diff{x}}
\end{align*}

\subsubsection*{Example}
\[ \ddiff{y}{x}-3y = 6 \]
This is a first order linear equation already in standard form. It is also
separable, but we will solve using the method for first order linear equations.
\begin{enumerate}
  \item Identify \( P(x) \):
  \[ P(x) = -3 \quad Q(x) = 6 \]
  \item Find \( \mu(x) \):
  \[ \mu(x) = \e^{\int-3\diff{x}} = \e^{-3x} \]
  \item Multiply the given equation by \( \mu(x) \):
  \begin{align*}
    \e^{-3x}\ddiff{y}{x}-3\e^{-3x}y &= 6\e^{-3x} \\
    \ddiff{}{x}\bigg[\e^{-3x}y\bigg] &= 6\e^{-3x} \\
    \int\ddiff{}{x}\bigg[\e^{-3x}y\bigg]\diff{x} &= \int 6\e^{-3x}\diff{x} \\
    \e^{-3x}y &= 6(-\frac{1}{3}\e^{-3x})+c \\
    &= -2\e^{-3x}+c \\
    y &= -2+C\e^{3x}
  \end{align*}
\end{enumerate}

\subsubsection*{Example}
\[ x\ddiff{y}{x}-4y = x^6\e^x \]
\begin{enumerate}
  \item Write it in standard form:
  \[ \ddiff{y}{x}-\frac{4}{x}y = x^5\e^x \]
  \item Identify \( P(x) \):
  \[ P(x) = \frac{-4}{x} \]
  \item Find \( \mu(x) \):
  \[ \mu(x) = \e^{\int\frac{-4}{x}\diff{x}} = \e^{-4\ln|x|} =
    \e^{\ln\frac{1}{x^4}} = \frac{1}{x^4} \]
  \item Multiply the given equation by \( \mu(x) \):
  \begin{align*}
    \frac{1}{x^4}\ddiff{y}{x}-\frac{4}{x^5}y &= x\e^x \\
    \ddiff{}{x}\bigg[\frac{1}{x^4}y\bigg] &= x\e^x \\
    \frac{1}{x^4}y &= \int x\e^x\diff{x}
      = x\e^x-\int\e^x\diff{x}
      = x\e^x-\e^x+c \\
    y &= x^5\e^x-x^4\e^x+Cx^4
  \end{align*}
\end{enumerate}

\subsubsection*{Example}
Solve the following over \( (0,\infty) \):
\begin{align*}
  x\ddiff{y}{x}+3(y+x^2) = \frac{\cos(x)}{x^2} \\
  x\ddiff{y}{x}+3y &= \frac{\cos(x)}{x^2}-3x^2 \\
  \ddiff{y}{x}+\frac{3}{x}y &= \frac{\cos(x)}{x^3}-3x \\
  P(x) &= \frac{3}{x} \\
  \mu(x) &= \e^{\int\frac{3}{x}\diff{x}}
    = \e^{3\ln(x)} = \e^{\ln(x^3)} = x^3 \\
  x^3\ddiff{y}{x}+3x^2y &= \cos(x)+3x^4 \\
  \ddiff{}{x}\bigg[x^3y\bigg] &= \cos(x)+3x^4 \\
  \int\ddiff{}{x}\bigg[x^3y\bigg]\diff{x} &= \int\cos(x)+3x^4\diff{x} \\
  x^3y &= \sin(x)+\frac{3x^5}{5}+c \\
  y &= \frac{\sin(x)}{x^3}-\frac{3x^2}{5}+\frac{c}{x^3}
\end{align*}

\subsubsection*{Example}
Solve the following initial value problem over \( (-\frac{\pi}{2},\frac{\pi}{2})
\) given \( y(0) = 2 \):
\begin{align*}
  \cos(x)\ddiff{y}{x}+\sin(x)y &= 1 \\
  \ddiff{y}{x}+\frac{\sin(x)}{\cos(x)}y &= \frac{1}{\cos(x)} \\
  \ddiff{y}{x}+\tan(x)y &= \sec(x) \\
  \mu(x) &= \e^{\int P(x)\diff{x}} = \e^{\int\tan(x)\diff{x}} =
    \e^{\ln|\sec(x)|} = \sec(x) \\
  \sec(x)\ddiff{y}{x}+\sec(x)\tan(x)y &= \sec^2(x)
\end{align*}
\begin{align*}
  \ddiff{}{x}\bigg[\sec(x)y\bigg] &= \sec^2(x) \\
  \int\ddiff{}{x}\bigg[\sec(x)y\bigg]\diff{x} &= \int\sec^2(x)\diff{x} \\
  \sec(x)y &= \tan(x)+c \\
  y &= \frac{\tan(x)}{\sec(x)}+\frac{c}{\sec(x)} \\
  &= \sin(x)+c\cos(x)
\end{align*}
Using our initial value:
\begin{align*}
  y &= \sin(x)+c\cos(x) \\
  2 &= \sin(0)+c\cos(0) \\
  &= 0+c \\
  c &= 2 \\
  y &= \sin(x)+2\cos(x)
\end{align*}

\section*{Substitutions and Transformations}
Consider:
\[ \ddiff{y}{x}+P(x)y = Q(x)y^n \]
where \( n \) is any real number. This is called a \textbf{Bernoulli Equation}.
We use the substitution \( v = y^{1-n} \) to transform the equation into a new
variable:
\[ \ddiff{v}{x} = (1-n)y^{-n}\ddiff{y}{x} \]
This gives a linear equation in \( v \):
\[ \ddiff{y}{x} = \frac{1}{1-n}y^n\ddiff{v}{x} \]
We divide the equation by \( y^n \):
\[ y^{-n}\ddiff{y}{x}+P(x)y(y^{-n}) = Q(x) \]
We can now substitute for \( \ddiff{y}{x} \):
\begin{align*}
  \frac{1}{1-n}y^n\ddiff{v}{x}y^{-n}+P(x)y^{1-n} &= Q(x) \\
  \frac{1}{1-n}\ddiff{v}{x}+P(x)v &= Q(x) \\
  \ddiff{v}{x}+(1-n)P(x)v &= (1-n)Q(x)
\end{align*}
Using this substitution allows us to change our equation into a first-order
linear equation in standard form, which we can now solve.

\subsubsection*{Example}
Solve the following over \( (0,\infty) \):
\begin{align*}
  \ddiff{y}{x}+\frac{y}{x} &= x^2y^2 \\
  \ddiff{y}{x}+\frac{1}{x}y &= x^2y^2 \\
  v &= y^{1-2} = y^{-1} \\
  \ddiff{v}{x} &= -y^{-2}\ddiff{y}{x} \\
  \ddiff{y}{x} &= -y^2\ddiff{v}{x} \\
  \frac{1}{y^2}\bigg(\ddiff{y}{x}+\frac{1}{x}y &= x^2y^2\bigg) \\
  y^{-2}\ddiff{y}{x}+\frac{1}{x}y(y^{-2}) &= x^2 \\
  y^{-2}(-y^2\ddiff{v}{x})+\frac{1}{x}y^{-1} &= x^2 \\
  -\ddiff{v}{x}+\frac{1}{x}v &= x^2 \\
  \ddiff{v}{x}-\frac{1}{x}v &= -x^2
\end{align*}
\begin{align*}
  \ddiff{v}{x}-\frac{1}{x}v &= -x^2 \\
  \mu(x) &= \e^{\int-\frac{\diff{x}}{x}} = \e^{-\ln|x|} =
    \e^{\ln|\frac{1}{x}|} = \frac{1}{x} \\
  \frac{1}{x}\ddiff{v}{x}-\frac{1}{x^2}v &= -x \\
  \int\ddiff{}{x}\bigg[\frac{1}{x}v\bigg]\diff{x} &= -\int x\diff{x} \\
  \frac{1}{x}v &= -\frac{x^2}{2}+c \\
  v &= -\frac{x^3}{2}+cx \\
  \frac{1}{y} &= -\frac{x^3}{2}+cx
\end{align*}

\subsubsection*{Example}
Solve the following initial value problem given \( y(1) = 1 \):
\begin{align*}
  x^2\ddiff{y}{x}-2xy &= 3y^4 \\
  \ddiff{y}{x}-\frac{2}{x}y &= \frac{3y^4}{x^2} \\
  n &= 4 \quad v = y^{1-4} = y^{-3} \\
  \ddiff{v}{x} &= -3y^{-4}\ddiff{y}{x} \\
  \ddiff{y}{x} &= -\frac{1}{3}y^4\ddiff{v}{x} \\
  \frac{1}{y^4}\bigg(\ddiff{y}{x} &= -\frac{1}{3}y^4\ddiff{v}{x}\bigg) \\
  y^{-4}\ddiff{y}{x}-\frac{2}{x}y(y^{-4}) &= \frac{3}{x^2} \\
  y^{-4}(-\frac{1}{3}y^4\ddiff{v}{x})-\frac{2}{x}v &= \frac{3}{x^2} \\
  -\frac{1}{3}\ddiff{v}{x}-\frac{2}{x}v &= \frac{3}{x^2} \\
  \ddiff{v}{x}+\frac{6}{x}v &= -\frac{9}{x^2}
\end{align*}
This equation is now a first order linear equation in \( v \).
\begin{align*}
  \ddiff{v}{x}+\frac{6}{x}v &= -\frac{9}{x^2} \\
  \mu(x) &= \e^{\int\frac{6}{x}\diff{x}} = \e^{6\ln|x|} = \e^{\ln|x^6|} = x^6 \\
  x^6\ddiff{v}{x}+\frac{6x^5}v &= -9x^4 \\
  \ddiff{}{x}\bigg[x^6v\bigg] &= -9x^4 \\
  \int\ddiff{}{x}\bigg[x^6v\bigg]\diff{x} &= \int-9x^4\diff{x} \\
  x^6v &= -\frac{9}{5}x^5+c \\
  v &= -\frac{9}{5x}+\frac{c}{x^6} \\
  \frac{1}{y^3} &= -\frac{9}{5x}+\frac{c}{x^6}
\end{align*}
Using our initial value:
\begin{align*}
  \frac{1}{y^3} &= -\frac{9}{5x}+\frac{c}{x^6} \\
  \frac{1}{1^3} &= -\frac{9}{5}+\frac{c}{1^6} \\
  c &= \frac{14}{5} \\
  \frac{1}{y^3} &= -\frac{9}{5}\frac{1}{x}+\frac{14}{5}\frac{1}{x^6}
\end{align*}

\section*{Exact Equations}
Consider:
\[ z = f(x,y) \quad \text{ vs } \quad y = f(x) \]
\( z = f(x,y) \) is a surface in 3D, while \( y = f(x) \) is a curve in 2D.
Recall that when differentiating \( y = f(x) \), \( \ddiff{y}{x} = f'(x) \)
and \( \diff{y} = f'(x)\diff{x} \). To differentiate \( z = f(x,y) \) we need
to find partial derivatives. These are computed by treating \( y \) or \( x \)
as a constant and taking the derivative with respect to other variable.
\begin{align*}
  x &= f(x,y) = x^3y^4-2x^3y^5 \\
  f_x &= \pdiff{f}{x} = 3x^2y^4-6x^2y^5 \\
  f_y &= \pdiff{f}{y} = 4x^3y^3-10x^3y^4 \\
  f_{xy} &= \frac{\partial^2f}{\partial{y}\partial{x}} = 12x^2y^3-30x^2y^4 \\
  f_{yx} &= \frac{\partial^2f}{\partial{y}\partial{x}} = 12x^2y^3-30x^2y^4 \\
\end{align*}
We can also integrate \( f(x,y) \) with respect to \( x \) or with respect to
\( y \).
\[ \int xy^3\diff{y} = x\int y^2\diff{y} = \frac{xy^3}{3}+c \]
\( \diff{y} \) tells us the variable of integration, with \( x \) being constant
with respect to \( y \).
\[ \int xy^2\diff{x} = \frac{x^2y^2}{2}+c \]
\( \diff{x} \) tells us the variable of integration, with \( y \) being constant
with respect to \( x \). \\
Consider again \( z = f(x,y) \), the total differential is:
\[ \diff{z} = \pdiff{f}{x}\diff{x}+\pdiff{f}{y}\diff{y} \]
This is a first order equation. This equation is called an exact equation if:
\[ \pdiff{}{y}\bigg[\pdiff{f}{x}\bigg] = \pdiff{}{x}\bigg[\pdiff{f}{y}\bigg] \]
We can write this as:
\[ M(x,y)\diff{x}+N(x,y)\diff{y} = 0 \]
where \( M = \pdiff{f}{x} \) and \( N = \pdiff{f}{y} \). We want to find
\( f(x,y) = c \), geometrically represented as a level curve, or a slice of the
curve \( z = f(x,y) \). Initially:
\[ f(x,y) = \int M(x,y)\diff{x}+h(y) \]
We need to find \( h(y) \) by differentiating with respect to \( y \):
\begin{align*}
  \pdiff{}{y}\bigg[\int M(x,y)\diff{y}\bigg] &= h'(y) \\
  \int h'(y)\diff{y} &= h(y)+c \\
  f(x,y) &= \int M(x,y)\diff{x}+h(y)+c = 0
\end{align*}

\subsubsection*{Example}
Solve the initial value problem given \( y(1) = 1 \):
\begin{align*}
  (x^2y^3)\diff{x}+(x^3y^2)\diff{y} &= 0 \\
  M(x,y) &= x^2y^3 \quad N(x,y) = x^3y^2 \\
  \pdiff{M}{y} &= 3x^2y^2 \quad \pdiff{N}{x} = 3x^2y^2
\end{align*}
From this, we can determine that it is an exact equation.
\begin{align*}
  f(x,y) &= \int x^2y^3\diff{x}+h(y) \\
  &= \frac{x^3y^3}{3}+h(y) \\
  \pdiff{f}{y} &= x^3y^2+h'(y) = N(x,y) = x^3y^2 \\
  \therefore h'(y) &= 0 \\
  h(y) &= k \\
  f(x,y) &= c = \frac{x^3y^3}{3}+k \\
  \frac{x^3y^3}{3} &= c
\end{align*}
Using our initial value:
\begin{align*}
  \frac{x^3y^3}{3} &= c \\
  \frac{(1)(1)}{3} &= c \\
  \frac{x^3y^3}{3} &= \frac{1}{3}
\end{align*}

\subsubsection*{Example}
\begin{align*}
  (\e^{2y}-y\cos(xy))\diff{x}+(2x\e^{2y}-x\cos(xy)+2y)\diff{y} &= 0 \\
  M(x,y) = \e^{2y}-y\cos(xy) \quad N(x,y) &= 2x\e^{2y}-x\cos(xy)+2y \\
  \pdiff{M}{y} &= 2\e^{2y}-(\cos(xy)+y(-\sin(xy)x)) \\
  &= 2\e^{2y}-\cos(xy)+xy\sin(xy) \\
  \pdiff{N}{y} &= 2\e^{2y}-(\cos(xy)+x(-\sin(xy))y) \\
  &= 2\e^{2y}-\cos(xy)+xy\sin(xy) \\
  f(x,y) &= \int M(x,y)+h(y) \\
  &= \int(\e^{2y}-y\cos(xy))\diff{x}+h(y) \\
  &= x\e^{2y}-(y+\sin(xy)\frac{1}{y})+h(y) \\
  &= x\e^{2y}-\sin(xy)+h(y) \\
  \pdiff{f}{y} &= 2x\e^{2y}-\cos(xy)x+h'(y) = N(x,y) \\
  h'(y) &= 2y \\
  h(y) &= y^2+c \\
  \therefore f(x,y) &= x\e^{2y}-\sin(xy)+y^2+c
\end{align*}

\begin{center}
  You can find all my notes at \url{http://omgimanerd.tech/notes}. If you have
  any questions, comments, or concerns, please contact me at
  alvin@omgimanerd.tech
\end{center}

\end{document}
