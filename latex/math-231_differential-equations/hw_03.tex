\documentclass{math}

\usepackage{pgfplots}
\pgfplotsset{compat=1.13}

\geometry{letterpaper, margin=0.5in}

\title{Differential Equations: Homework 3}
\author{Alvin Lin}
\date{January 2018 - May 2018}

\begin{document}

\maketitle

\section*{Section 2.3}

\subsubsection*{Exercise 1}
Determine whether the given equation is separable, linear, neither, or both.
\[ x^2\ddiff{y}{x}+\sin(x)-y = 0 \]
Not separable and linear.

\subsubsection*{Exercise 3}
Determine whether the given equation is separable, linear, neither, or both.
\[ (t^2+1)\ddiff{y}{t} = yt-y \]
Separable and non-linear.

\subsubsection*{Exercise 5}
Determine whether the given equation is separable, linear, neither, or both.
\[ x\ddiff{x}{t}+t^2x = \sin(t) \]
Not separable and non-linear.

\subsubsection*{Exercise 7}
Obtain the general solution to the equation:
\begin{align*}
  \ddiff{y}{x}-y-\e^{3x} &= 0 \\
  \ddiff{y}{x}-y &= \e^{3x} \\
  \mu(x) &= \e^{\int-1\diff{x}} = \e^{-x} \\
  \e^{-x}\ddiff{y}{x}-y\e^{-x} &= \e^{3x}\e^{-x} \\
  \int\bigg(\e^{-x}\ddiff{y}{x}-y\e^{-x}\bigg)\diff{x} &=
    \int\e^{3x}\e^{-x}\diff{x} \\
  y\e^{-x} &= \frac{1}{2}\e^{2x}+c
\end{align*}

\subsubsection*{Exercise 8}
Obtain the general solution to the equation:
\begin{align*}
  \ddiff{y}{x} &= \frac{y}{x}+2x+1 \\
  \ddiff{y}{x}-\frac{1}{x}y &= 2x+1 \\
  \mu(x) &= \e^{\int\frac{-1}{x}\diff{x}} = \e^{-\ln(x)} = \frac{1}{x} \\
  \frac{1}{x}\ddiff{y}{x}-y &= 2+\frac{1}{x} \\
  \int\bigg(\frac{1}{x}\ddiff{y}{x}-\frac{1}{x^2}y\bigg)\diff{x} &=
    \int2+\frac{1}{x}\diff{x} \\
  \frac{y}{x} &= 2x-\ln(x)+c \\
  y &= 2x^2-x\ln(x)+cx
\end{align*}

\subsubsection*{Exercise 9}
Obtain the general solution to the equation:
\begin{align*}
  \ddiff{r}{\theta}+r\tan\theta &= \sec\theta \\
  \mu(\theta) &= \e^{\int\tan\theta\diff{\theta}} = \e^{\ln(\sec\theta)} =
    \sec\theta \\
  \sec(x)\ddiff{r}{\theta}+r\tan\theta\sec\theta &= \sec^2\theta \\
  \int\bigg(\sec(x)\ddiff{r}{\theta}+r\tan\theta\sec\theta\bigg)\diff{\theta} &=
    \int\sec^2\theta\diff{\theta} \\
  r\sec\theta &= \tan\theta+c \\
  r &= \sin\theta+c\cos\theta \\
\end{align*}

\subsubsection*{Exercise 10}
Obtain the general solution to the equation:
\begin{align*}
  x\ddiff{y}{x}+2y &= x^{-3} \\
  \ddiff{y}{x}+\frac{2}{x}y &= x^{-4} \\
  \mu(x) &= \e^{\int\frac{2}{x}\diff{x}} = \e^{2\ln(x)} = x^2 \\
  x^2\ddiff{y}{x}+2xy &= x^{-2} \\
  \int\bigg(x^2\ddiff{y}{x}+2xy\bigg)\diff{x} &= \int x^{-2}\diff{x} \\
  yx^2 &= -\frac{1}{x}+c \\
  y &= -\frac{1}{x^3}+\frac{c}{x^2}
\end{align*}

\subsubsection*{Exercise 11}
Obtain the general solution to the equation:
\begin{align*}
  (t+y+1)\diff{t}-\diff{y} = 0 \\
  \ddiff{y}{t} &= t+y+1 \\
  \ddiff{y}{t}-y &= t+1 \\
  \mu(t) &= \e^{\int-1\diff{t}} = \e^{-t} \\
  \e^{-t}\ddiff{y}{t}-y\e^{-t} &= t\e^{-t}+\e^{-t} \\
  \int\bigg(\e^{-t}\ddiff{y}{t}-y\e^{-t}\bigg) &=
    \int t\e^{-t}+\e^{-t}\diff{t} \\
  y\e^{-t} &= -e^{-t}(t+1)-\e^{-t}+c \\
  y &= -t-1-1+c\e^t \\
  &= -t-2+c\e^t
\end{align*}

\subsubsection*{Exercise 15}
Obtain the general solution to the equation:
\begin{align*}
  (x^2+1)\ddiff{y}{x}+xy-x &= 0 \\
  \ddiff{y}{x}+\frac{x}{x^2+1}y &= \frac{x}{x^2+1} \\
  \mu(x) &= \e^{\int\frac{x}{x^2+1}\diff{x}} = \e^{\frac{1}{2}\ln(x^2+1)} =
    \sqrt{x^2+1} \\
  \sqrt{x^2+1}\ddiff{y}{x}+\frac{x}{\sqrt{x^2+1}}y &= \frac{x}{\sqrt{x^2+1}} \\
  \int\bigg(\sqrt{x^2+1}\ddiff{y}{x}+\frac{x}{\sqrt{x^2+1}}y\bigg)\diff{x} &=
    \int\frac{x}{\sqrt{x^2+1}}\diff{x} \\
  y\sqrt{x^2+1} &= \sqrt{x^2+1}+c \\
  y &= 1+\frac{c}{\sqrt{x^2+1}}
\end{align*}

\subsubsection*{Exercise 17}
Solve the initial value problem.
\begin{align*}
  \ddiff{y}{x}-\frac{y}{x} &= x\e^x \quad y(1) = \e-1 \\
  \mu(x) &= \e^{\int-\frac{1}{x}\diff{x}} = \e^{-\ln(x)} = \frac{1}{x} \\
  \frac{1}{x}\ddiff{y}{x}-\frac{1}{x^2}y &= \e^x \\
  \int\bigg(\frac{1}{x}\ddiff{y}{x}-\frac{1}{x^2}y\bigg)\diff{x} &=
    \int\e^x\diff{x} \\
  \frac{y}{x} &= \e^x+c \\
  y &= x\e^x+cx
\end{align*}
\begin{align*}
  \e-1 &= 1\e^1+c1 \\
  c &= -1 \\
  y &= x\e^x-x
\end{align*}

\subsubsection*{Exercise 18}
Solve the initial value problem.
\begin{align*}
  \ddiff{y}{x}+4y-\e^{-x} &= 0 \quad y(0) = \frac{4}{3} \\
  \ddiff{y}{x}+4y &= \e^{-x} \\
  \mu(x) &= \e^{\int4\diff{x}} = \e^{4x} \\
  \e^{4x}\ddiff{y}{x}+4y\e^{4x} &= \e^{3x} \\
  \int\bigg(\e^{4x}\ddiff{y}{x}+4y\e^{4x}\bigg)\diff{x} &=
    \int\e^{3x}\diff{x} \\
  y\e^{4x} &= \frac{1}{3}\e^{3x}+c \\
  y &= \frac{1}{3\e^x}+\frac{c}{\e^{4x}} \\
  \frac{4}{3} &= \frac{1}{3\e^0}+\frac{c}{\e^0} \\
  c &= 1 \\
  y &= \frac{1}{3\e^x}+\frac{1}{\e^{4x}}
\end{align*}

\subsubsection*{Exercise 20}
Solve the initial value problem.
\begin{align*}
  \ddiff{y}{x}+\frac{3y}{x}+2 &= 3x \quad y(1) = 1 \\
  \ddiff{y}{x}+\frac{3}{x}y &= 3x-2 \\
  \mu(x) &= \e^{\int\frac{3}{x}\diff{x}} = \e^{3\ln(x)} = x^3 \\
  x^3\ddiff{y}{x}+3x^2y &= 3x^4-2x^3 \\
  \int\bigg(x^3\ddiff{y}{x}+3x^2y\bigg)\diff{x} &= \int3x^4-2x^3\diff{x} \\
  yx^3 &= \frac{3}{5}x^5-\frac{2}{4}x^4+c \\
  y &= \frac{3}{5}x^2-\frac{1}{2}x^2+\frac{c}{x^3} \\
  1 &= \frac{3}{5}-\frac{1}{2}+c \quad c = \frac{9}{10} \\
  y &= \frac{3}{5}x^2-\frac{1}{2}x^2+\frac{9}{10x^3} \\
\end{align*}

\section*{Section 2.6}

\subsubsection*{Exercise 21}
Use the method discussed under ``Bernoulli Equations'' to solve:
\begin{align*}
  \ddiff{y}{x}+\frac{y}{x} &= x^2y^2 \\
  v &= y^{-1} \\
  \ddiff{v}{x} &= -y^{-2}\ddiff{y}{x} \\
  \frac{1}{y^2}\ddiff{y}{x}+\frac{1}{y^2}\frac{y}{x} &= \frac{1}{y^2}x^2y^2 \\
  \frac{1}{y^2}\ddiff{y}{x}+\frac{1}{xy} &= x^2 \\
  -\ddiff{v}{x}+\frac{v}{x} &= x^2 \\
  \ddiff{v}{x}-\frac{v}{x} &= -x^2 \\
  \mu(x) &= \e^{\int-\frac{1}{x}\diff{x}} = \e^{-\ln(x)} = \frac{1}{x} \\
  \frac{1}{x}\ddiff{v}{x}-\frac{v}{x^2} &= -x^2 \\
  \int\bigg(\frac{1}{x}\ddiff{v}{x}-\frac{v}{x^2}\bigg)\diff{x} &= \int-x^2\diff{x} \\
  \frac{v}{x} &= -\frac{x^3}{3}+c \\
  \frac{1}{xy} &= -\frac{x^3}{3}+c
\end{align*}

\subsubsection*{Exercise 22}
Use the method discussed under ``Bernoulli Equations'' to solve:
\begin{align*}
  \ddiff{y}{x}-y &= \e^{2x}y^3 \\
  P(x) = -1 \quad Q(x) = \e^{2x} \quad n = 3 \quad
    v &= y^{-2} \quad \ddiff{v}{x} = -\frac{1}{2}y^{-3}\ddiff{y}{x} \\
  \ddiff{v}{x}+(1-n)P(x)v &= (1-n)Q(x) \\
  \ddiff{v}{x}+2v &= -2\e^{2x} \\
  \mu(x) &= \e^{\int2\diff{x}} = \e^{2x} \\
  \e^{2x}\ddiff{v}{x}+2v\e^{2x} &= -2\e^{4x} \\
  \int\bigg(\e^{2x}\ddiff{v}{x}+2v\e^{2x}\bigg)\diff{x} &=
    \int-2\e^{4x}\diff{x} \\
  v\e^{2x} &= \frac{-2}{4}\e^{4x}+c \\
  \frac{\e^{2x}}{y^2} &= \frac{-\e^{4x}}{2}+c
\end{align*}

\subsubsection*{Exercise 23}
Use the method discussed under ``Bernoulli Equations'' to solve:
\begin{align*}
  \ddiff{y}{x} &= \frac{2y}{x}-x^2y^2 \\
  \ddiff{y}{x}-\frac{2y}{x} &= -x^2y^2 \\
  v &= y^{-1} \quad \ddiff{v}{x} = -\frac{1}{y^2}\ddiff{y}{x} \\
  \frac{1}{y^2}\ddiff{y}{x}-\frac{1}{y^2}\frac{2y}{x} &= -\frac{1}{y^2}x^2y^2 \\
  \frac{1}{y^2}\ddiff{y}{x}-\frac{2}{xy} &= -x^2 \\
  -\ddiff{v}{x}-\frac{2v}{x} &= -x^2 \\
  \ddiff{v}{x}+\frac{2v}{x} &= x^2 \\
  \mu(x) &= \e^{\int\frac{2}{x}\diff{x}} = \e^{2\ln(x)} = x^2 \\
  x^2\ddiff{v}{x}+2vx &= x^4 \\
  \int\bigg(x^2\ddiff{v}{x}+2vx\bigg)\diff{x} &= \int x^4\diff{x} \\
  vx^2 &= \frac{x^5}{5}+c \\
  \frac{x^2}{y} &= \frac{x^5}{5}+c
\end{align*}

\subsubsection*{Exercise 24}
Use the method discussed under ``Bernoulli Equations'' to solve:
\begin{align*}
  \ddiff{y}{x}+\frac{y}{x-2} &= 5(x-2)y^{\frac{1}{2}} \\
  v &= y^{\frac{1}{2}} \quad \ddiff{v}{x} = -\frac{1}{\sqrt{y}}\ddiff{y}{x} \\
  \frac{1}{\sqrt{y}}\ddiff{y}{x}+\frac{1}{\sqrt{y}}\frac{y}{x-2} &=
    \frac{1}{\sqrt{y}}5(x-2)\sqrt{y} \\
  \frac{1}{\sqrt{y}}\ddiff{y}{x}+\frac{\sqrt{y}}{x-2} &= 5x-10 \\
  -\ddiff{v}{x}+\frac{v}{x-2} &= 5x-10 \\
  \ddiff{v}{x}-\frac{v}{x-2} &= 10-5x \\
  \mu(x) &= \e^{\int\frac{-1}{x-2}\diff{x}} = \e^{-\ln(x-2)} = \frac{1}{x-2} \\
  (x-2)\ddiff{v}{x}-\frac{v}{(x-2)^2} &= \frac{10-5x}{x-2}
\end{align*}
\begin{align*}
  \int\bigg((x-2)\ddiff{v}{x}-\frac{v}{(x-2)^2}\bigg)\diff{x} &=
    \int\frac{10-5x}{x-2}\diff{x} \\
  v(x-2) &= -5x+c \\
  (x-2)\sqrt{y} &= -5x+c
\end{align*}

\subsubsection*{Exercise 25}
Use the method discussed under ``Bernoulli Equations'' to solve:
\begin{align*}
  \ddiff{x}{t}+tx^3+\frac{x}{t} &= 0 \\
  \ddiff{x}{t}+\frac{x}{t} &= -x^3 \\
  v &= x^{-2} \quad \ddiff{v}{t} = -\frac{1}{2}x^{-3} \\
  \frac{1}{x^3}\ddiff{x}{t}+\frac{1}{x^3}\frac{x}{t} &= -\frac{1}{x^3}{x^3} \\
  \frac{1}{x^3}\ddiff{x}{t}+\frac{1}{x^2t} &= -1 \\
  -\ddiff{v}{t}+\frac{v}{t} &= -1 \\
  \ddiff{v}{t}-\frac{v}{t} &= 1 \\
  \mu(t) &= \e^{\int-\frac{1}{t}\diff{t}} = \e^{-\ln(t)} = \frac{1}{t} \\
  \frac{1}{t}\ddiff{v}{t}-\frac{v}{t^2} &= \frac{1}{t} \\
  \int\bigg(\frac{1}{t}\ddiff{v}{t}-\frac{v}{t^2}\bigg)\diff{t} &=
    \int\frac{1}{t}\diff{t} \\
  \frac{v}{t} &= \ln(t)+c \\
  \frac{1}{tx^2} &= \ln(t)+c
\end{align*}

\subsubsection*{Exercise 26}
Use the method discussed under ``Bernoulli Equations'' to solve:
\begin{align*}
  \ddiff{y}{x}+y &= \e^xy^{-2} \\
  v &= y^{3} \quad \ddiff{v}{x} = 3y^2\ddiff{y}{x} \\
  y^2\ddiff{y}{x}+y(y^2) &= \e^xy^{-2}(y^2) \\
  \frac{1}{3}\ddiff{v}{x}+v &= \e^x \\
  \ddiff{v}{x}+3v &= 3\e^x \\
  \mu(x) &= \e^{\ln3\diff{x}} = \e^{3x} \\
  \e^{3x}\ddiff{v}{x}+3v\e^{3x} &= 3\e^{4x}
\end{align*}
\begin{align*}
  \int\bigg(\e^{3x}\ddiff{v}{x}+3v\e^{3x}\bigg)\diff{x} &=
    \int3\e^{4x}\diff{x} \\
  v\e^{3x} &= \frac{3}{4}\e^{4x}+c \\
  y^{3}\e^{3x} &= \frac{3}{4}\e^{4x}+c \\
\end{align*}

\subsubsection*{Exercise 27}
Use the method discussed under ``Bernoulli Equations'' to solve:
\begin{align*}
  \ddiff{r}{\theta} &= \frac{r^2+2r\theta}{\theta^2} \\
  &= \frac{r^2}{\theta^2}+\frac{2r}{\theta} \\
  \ddiff{r}{\theta}-\frac{2r}{\theta} &= \frac{r^2}{\theta^2} \\
  v &= r^{-1} \quad \ddiff{v}{\theta} = -\frac{1}{r^2}\ddiff{r}{\theta} \\
  \frac{1}{r^2}\ddiff{r}{\theta}-\frac{1}{r^2}\frac{2r}{\theta} &=
    \frac{1}{r^2}\frac{r^2}{\theta^2} \\
  -\ddiff{v}{\theta}-\frac{2v}{\theta} &= \frac{1}{\theta^2} \\
  \ddiff{v}{\theta}+\frac{2v}{\theta} &= -\frac{1}{\theta^2} \\
  \mu(\theta) &= \e^{\int\frac{2}{\theta}\diff{\theta}} = \e^{2\ln(\theta)} =
    \theta^2 \\
  \theta^2\ddiff{v}{\theta}+2v\theta &= -1 \\
  \int\bigg(\theta^2\ddiff{v}{\theta}+2v\theta\bigg)\diff{\theta} &=
    \int-1\diff{\theta} \\
  v\theta^2 &= -\theta+c \\
  \frac{\theta^2}{r} &= -\theta+c
\end{align*}

\subsubsection*{Exercise 28}
Use the method discussed under ``Bernoulli Equations'' to solve:
\begin{align*}
  \ddiff{y}{x}+y^3x+y &= 0 \\
  \ddiff{y}{x}+y &= -xy^3 \\
  v &= y^{-2} \quad \ddiff{v}{x} = -\frac{1}{2}y^{-3}\ddiff{y}{x} \\
  \frac{1}{y^3}\ddiff{y}{x}+\frac{1}{y^3}y &= \frac{1}{y^3}-xy^3 \\
  -2\ddiff{v}{x}+v &= -x
\end{align*}
\begin{align*}
  \ddiff{v}{x}-\frac{v}{2} &= \frac{x}{2} \\
  \mu(x) &= \e^{\int-\frac{1}{2}\diff{x}} = \e^{-\frac{x}{2}} \\
  \e^{-\frac{x}{2}}\ddiff{v}{x}-\e^{-\frac{x}{2}}\frac{v}{2} &=
    \frac{x\e^{\frac{x}{2}}}{2} \\
  \int\bigg(\e^{-\frac{x}{2}}\ddiff{v}{x}-\frac{v\e^{-\frac{x}{2}}}{2}\bigg)
    \diff{x} &= \int\frac{x\e^{-\frac{x}{2}}}{2}\diff{x} \\
  v\e^{-\frac{x}{2}} &= -\e^{-\frac{x}{2}}(x-2)+c \\
  \frac{1}{y^2} &= 2-x+c
\end{align*}

\begin{center}
  If you have any questions, comments, or concerns, please contact me at
  alvin@omgimanerd.tech
\end{center}

\end{document}
