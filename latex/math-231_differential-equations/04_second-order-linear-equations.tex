\documentclass{math}

\usepackage{tikz}

\title{Differential Equations}
\author{Alvin Lin}
\date{January 2018 - May 2018}

\begin{document}

\maketitle

\section*{Homogeneous Linear Equations}
Consider:
\[ ay''+by'+cy = f(t) \]
where \( a,b,c \) are constants and \( a \ne 0 \). If \( f(t) = 0 \):
\[ ay''+by'+cy = 0 \]
And this is a second order linear homogeneous equation. Consider \( y(t) =
\e^{rt} \) is a solution:
\begin{align*}
  y'(t) &= r\e^{rt} \\
  y''(t) &= r^2\e^{rt}
\end{align*}
If we substitute:
\begin{align*}
  ar^2\e^{rt}+br\e^{rt}+c\e^{rt} &= 0 \\
  \e^{rt}\bigg[ar^2+br+c\bigg] &= 0
\end{align*}
To satisfy this equation, we have \( ar^2+br+c = 0 \). This is called the
\textbf{characteristic equation} or the \textbf{auxiliary equation}. We can
determine the roots of \( ar^2+bc+r = 0 \) using the quadratic formula.
\[ r = \frac{-b\pm\sqrt{b^2-4ac}}{2a} \]
Using this equation, there are three cases:
\begin{enumerate}
  \item When \( b^2-4ac > 0 \), the roots are real and distinct.
  \item When \( b^2-4ac = 0 \), the root \( r = \frac{-b}{2a} \) is a repeated
  root.
  \item When \( b^2-4ac < 0 \), the roots are complex conjugates \( r = \alpha
  \pm i\beta \) where \( i = \sqrt{-1} \).
\end{enumerate}

\subsection*{Case 1: Real and Distinct Roots}
We can use the roots to solve the differential equation.
\begin{align*}
  y''+y'-2y &= 0 \\
  r^2+r-2 &= 0 \\
  (r+2)(r-2) &= 2 \\
  r_1 = -2 \quad r_1 &= 1
\end{align*}
Using this characteristic equation, we can determine \( y_1 = \e^{-2t} \) and
\( y_2 = \e^t \). We can verify \( y_1 \) and \( y_2 \) are both solutions.
\begin{align*}
  y_1 &= \e^{-2t} &\quad y_2 &= \e^t \\
  y_1' &= -2\e^{-2t} &\quad y_2' &= \e^t \\
  y_1'' &= 4\e^{-2t} &\quad y_2'' &= \e^t
\end{align*}
\begin{align*}
  y''+y'-2y &= 0 \\
  4\e^{-2t}-2\e^{-2t}-2\e^{-2t} &= 0 \\
  \e^t+\e^t-2\e^t &= 0
\end{align*}
Furthermore, \( y = y_1+y_2 \) also satisfies the equation.
\begin{align*}
  y &= \e^{-2t}+\e^t \\
  y' &= -2\e^{-2t}+\e^t \\
  y'' &= 4\e^{-2t}+\e^t \\
  4\e^{-2t}+\e^t-2\e^{-2t}-2(\e^{-2t}+\e^t) &= 0
\end{align*}
Also, note that \( y = c_1\e^{r_1t}+c_2\e^{r_2t} \) also satisfies
\[ ay''+by'+cy = 0 \]
This is a general solution since it contains two arbitrary constants. If we
are given initial conditions \( y(x_0) = y_0 \) and \( y'(x_0) = y_1 \), then
we can solve for \( c_1 \) and \( c_2 \) and treat it as an initial value
problem. For example:
\begin{gather*}
  y''+y' = 0 \\
  y(0) = 2 \quad y'(0) = 1
\end{gather*}
The characteristic equation is \( r^2+r = 0 \), giving us roots \( r_1 = 0 \)
and \( r_2 = -1 \). Thus we have the general solution
\[ y = c_1\e^{0t}+c_2\e^{-t} = c_1+c_2\e^{-t} \]
We can use our initial conditions to solve for an explicit solution:
\begin{align*}
  y &= c_1+c_2\e^{-t} \\
  y' &= -c_2\e^{-t} \\
  y(0) &= 2 = c_1+c_2 \\
  y'(0) &= 1 = -c_2 \\
  c_2 &= -1 \quad c_1 = 3 \\
  y &= 3-\e^{-t}
\end{align*}

\subsubsection*{Example}
Solve the initial value problem given \( y(0) = 0 \) and \( y'(0) = 1 \):
\begin{align*}
  y''+2y'-y &= 0 \\
  r^2+2r-1 &= 0 \\
  r &= \frac{-2\pm\sqrt{4-4(-1)}}{2} \\
  &= -1\pm\sqrt{2} \\
  y &= c_1\e^{(-1+\sqrt{2})t}+c_2\e^{(-1-\sqrt{2})t} \\
  y' &= (-1+\sqrt{2})c_1\e^{(-1+\sqrt{2})t}+
    (-1-\sqrt{2})c_2\e^{(-1-\sqrt{2})t} \\
  y(0) &= 0 = c_1+c_2 \\
  y'(0) &= 1 = (-1+\sqrt{2})c_1+(-2-\sqrt{2})c_2 \\
  c_1 &= \frac{1}{2\sqrt{2}} \quad c_2 = -\frac{1}{2\sqrt{2}}
\end{align*}

\subsubsection*{Existence and Uniqueness}
If we have existence and uniqueness for the solution to the initial value
problem:
\begin{gather*}
  ay''+by'+cy = 0 \\
  y(x_0) = y_0 \\
  y'(x_0) = y_1
\end{gather*}
then the solution is valid of \( (-\infty,\infty) \). If \( y_1(t) \) and
\( y_2(t) \) are any two solutions that are linearly independent over
\( (-\infty,\infty) \), then \( y = c_1y_1+c_2y_2 \) form a fundamental set of
solutions. If two functions \( y_1 \) and \( y_2 \) exist such that one is not
a constant multiple of the other, then \( y_1 \) and \( y_2 \) are linearly
independent. For example, \( y_1 = \cos(x) \) and \( y_2 = \sin(x) \) are
linearly independent. \( y_1 = \e^{2t} \) and \( y_2 = \e^t \) are also
linearly independent. Consider the Wronskian, \( W \):
\[ W(y_1,y_2) = \begin{vmatrix}
  y_1 & y_2 \\
  y_1' & y_2'
\end{vmatrix} = y_1y_2'-y_2y_1' \]
If \( W = 0 \), then \( y_1 \) and \( y_2 \) are linearly dependent. Otherwise,
\( y_1 \) and \( y_2 \) are linearly independent.

\subsubsection*{Example}
\begin{align*}
  y_1 &= \cos(x) \\
  y_2 &= \sin(x) \\
  W(\cos(x),\sin(x)) &= \begin{vmatrix}
    \cos(x) & \sin(x) \\
    -\sin(x) & \cos(x)
  \end{vmatrix} \\
  &= \cos^2(x)-(-\sin^2(x)) = 1 \ne 0
\end{align*}

\subsubsection*{Example}
\begin{align*}
  y_1 &= 3\e^{2t} \\
  y_2 &= \e^{2t} \\
  W(y_1,y_2) &= \begin{vmatrix}
    3\e^{2t} & \e^{2t} \\
    6\e^{2t} & 2\e^{2t}
  \end{vmatrix} \\
  &= 6\e^{2t}-6\e^{2t} = 0
\end{align*}

\subsection*{Case 2: Real and Repeated Roots}
If the roots of the auxiliary equation are real and repeated (\( r =
\frac{-b}{2a} \)), then \( y = c_1\e^{rt}+c_2\e^{rt} \). Note that
\( y_1 = c_1\e^{rt} \) and \( y_2 = c_2\e^{rt} \) will be linearly dependent.
We want another linearly independent solution to
\[ ay''+by'+cy = 0 \]
If \( y_1 = c_1\e^{rt} \), then \( y_2 = c_2t\e^{rt} \) gives a second solution
when \( y_1 \) and \( y_2 \) are linearly independent. We append a factor of
\( t \) to \( y_1 = c_1\e^{rt} \). For this case, our general solution is of
the form
\[ y = c_1\e^{rt}+c_2t\e^{rt} \]

\subsubsection*{Example}
Solve the initial value problem given \( y(0) = 1 \) and \( y'(0) = 3 \):
\begin{align*}
  y''+10y'+25 &= 0 \\
  r^2+10r+25 &= 0 \\
  (r+5) &= 0 \\
  r &= -5 \\
  y &= c_1\e^{-5t}+c_2t\e^{-5t} \\
  y(0) &= 1 = c_1 \\
  y' &= -5c_1\e^{-5t}+c_2\e^{-5t}-5c_2t\e^{-5t} \\
  y'(0) &= 3 = -5c_1+c_2 \\
  c_2 &= 8 \\
  y &= \e^{-5t}+8t\e^{-5t}
\end{align*}

\subsection*{Case 3: Complex Roots}
If the roots of the auxiliary equation are complex terms \( r = \alpha\pm
i\beta \) where \( i = \sqrt{-1} \), then as in previous cases:
\[ y = \e^{(\alpha+i\beta)t} \]
We can start by using the rules of exponents to split the complex term in the
exponent.
\[ y = \e^{\alpha t}\e^{i\beta t} \]
We can use Euler's formula to resolve this:
\[ \e^{it} = \cos\theta+i\sin\theta \]
If we let \( \theta = \beta t \), then we have:
\[ y = \e^{\alpha t}\bigg[\cos(\beta t)+i\sin(\beta t)\bigg] \]
We can see from this point that the solution is going to oscillate because
cosine is a sinusoidal wave. From this, one way to write the general form of
the solution is:
\[ y = \e^{\alpha t}\bigg[c_1\cos(\beta t)+ic_2\sin(\beta t)\bigg] \]
Given an initial value problem, this would solve it, but going through it would
involve complex arithmetic. Typically, the constants \( c_1 \) and \( c_2 \) are
complex values themselves. However, we can avoid complex arithmetic in this
particular situation. The equation we are trying to solve is:
\[ ay''+by'+cy = 0 \]
Consider a complex valued function \( z = u+iv \) that is a solution to 1.
\begin{align*}
  z &= u+iv \\
  z' &= u'+iv' \\
  z'' &= u''+iv''
\end{align*}
If we substitute this into the function:
\begin{align*}
  a\bigg[u''+iv''\bigg]+b\bigg[u'+iv'\bigg]+c\bigg[u+iv\bigg] &= 0 \\
  \bigg[au''+bu'+cu\bigg]+i\bigg[av''+bv'+cv\bigg] &= 0
\end{align*}
If \( s+it = 0 \), then \( s = t = 0 \). Therefore, both the real and imaginary
parts of the equation above are zero.
\begin{align*}
  au''+bu'+cu &= 0 \\
  av''+bv'+cv &= 0
\end{align*}
Since we have manipulated \( u \) and \( v \) into the forms of our original
equation, we can conclude that both \( u \) and \( v \) are solutions to our
differential equation. We can write a general solution as:
\[ y(t) = \e^{\alpha t}\bigg[c_1\cos(\beta t)+c_2\sin(\beta t)\bigg] \]
where \( r = \alpha\pm i\beta \) are complex valued roots.

\subsubsection*{Example}
Consider this example:
\begin{align*}
  y''+2y'+2y &= 0 \\
  r^2+2r+2 &= 0 \\
  r &= \frac{-2\pm\sqrt{4-(4)(2)}}{2} \\
  r &= -1\pm i \\
  \alpha &= -1 \quad \beta = 1 \\
  y(t) &= \e^{-t}\bigg[c_1\cos(t)+c_2\sin(t)\bigg]
\end{align*}
If we have the initial values \( y(0) = 1 \) and \( y'(0) = 1 \):
\begin{align*}
  y(t) &= \e^{-t}\bigg[c_1\cos(t)+c_2\sin(t)\bigg] \\
  y'(t) &= \e^{-t}\bigg[-c_1\sin(t)+c_2\cos(t)\bigg]-
    e^{-t}\bigg[c_1\cos(t)+c_2\sin(t)\bigg] \\
  y(0) &= 1 = \e^{0}\bigg[c_1\cos(0)+c_2\sin(0)\bigg] \\
  &= c_1 \\
  y'(0) &= 1 = \e^{0}\bigg[-c_1\sin(0)+c_2\cos(0)\bigg]-
    \e^{0}\bigg[c_1\cos(0)+c_2\sin(0)\bigg] \\
  &= c_2-1(c_1) \\
  c_1 &= 1 \quad c_2 = 2 \\
  y &= \e^{-t}\bigg[\cos(t)+2\sin(t)\bigg]
\end{align*}

\clearpage
\section*{The Method of Undetermined Coefficients}
Consider:
\[ ay''+by'+cy = f(t) \ne 0 \]
This is a second order non-homogeneous equation with constant coefficients. The
method of undetermined coefficients applies to functions that are exponentials,
polynomials, sines and consines, or products of these types of functions. For
example:
\[ y''-3y'+2y = 5\e^{3t} \]
We seek a particular solution, \( y_p \). We can try \( y_p = A\e^{3t} \) where
\( A \) is a constant.
\begin{align*}
  y_p &= A\e^{3t} \\
  y_p' &= 3A\e^{3t} \\
  y_p'' &= 9A\e^{3t} \\
  9A\e^{3t}-3(3A\e^{3t})+2A\e^{3t} &= 5\e^{3t} \\
  9A\e^{3t}-9A\e^{3t}+2A\e^{3t} \\
  2A &= 5\e^{3t} \\
  2A &= 5 \\
  y_p &= \frac{5}{2}\e^{3t}
\end{align*}

\subsubsection*{Example}
Consider another example:
\begin{align*}
  y''-3y'+2y &= 5\e^{2t} \\
  y_p &= A\e^{2t} \\
  y_p' &= 2A\e^{2t} \\
  y_p'' &= 4A\e^{2t} \\
  4A\e^{2t}-3(2A\e^{2t})+2A\e^{2t} &= 0 \ne 5\e^{2t}
\end{align*}
The key is to consider the associated auxiliary equation.
\begin{align*}
  r^2-3r+2 &= 0 \\
  (r-2)(r-1) &= 0 \\
  r_1 &= 1 \quad r_2 = 2
\end{align*}
In this case, \( r = 2 \) matched one of the roots of \( r^2-3r+2 = 0 \). This
means that \( y = c\e^{2t} \) is a solution to the associated homogeneous
equation, and not the given non-homogeneous equation. We append a factor of
\( t \):
\[ y_p = At\e^{2t} \]
If we try this form:
\begin{align*}
  y_p &= At\e^{2t} \\
  y_p' &= A\e^{2t}+2At\e^{2t} \\
  y_p'' &= 2A\e^{2t}+2A\e^{2t}+4At\e^{2t} \\
  &= 4A\e^{2t}+4At\e^{2t}
\end{align*}
\begin{align*}
  (4A\e^{2t}+4At\e^{2t})-3(A\e^{2t}+2At\e^{2t})+2At\e^{2t} &= 5\e^{2t} \\
  4A\e^{2t}+4At\e^{2t}-A\e^{2t}-6At\e^{2t}+2At\e^{2t} &= 5\e^{2t} \\
  A\e^{2t} &= 5\e^{2t} \\
  A &= 5 \\
  y_p &= 5\e^{2t}
\end{align*}

\subsubsection*{Example}
\begin{align*}
  y''+3y'+2y &= 3\e^{0t} \\
  r^2+3r+2 &= 0 \\
  (r+1)(r+2) &= 0 \\
  r_1 &= -1 \quad r_2 = -2
\end{align*}
Neither of these solutions match \( r = 0 \), so we do not need to append a term
\( t \). Consider \( y_p \) to be a polynomial of degree 1.
\begin{align*}
  y_p &= At+B \\
  y_p' &= A \\
  y_p'' &= 0
\end{align*}
Substituting these into our differential equation allows us to solve for an
explicit solution:
\begin{align*}
  0+3A+2(At+B) &= 3t \\
  (3A)t+(3A+2B) &= 3t \\
  2A &= 3 \quad 3A+2B = 0 \\
  B &= -\frac{9}{4} \\
  y_p &= \frac{3}{2}t-\frac{9}{4}
\end{align*}

\subsubsection*{Example}
\begin{align*}
  y''-y &= 10\e^{-t} \\
  r &= -1 \\
  r^2-1 &= 0 \\
  r_1 &= 1 \quad r_2 = -1
\end{align*}
Since we have a matching root, we append a factor of \( t \) to the solution:
\begin{align*}
  y_p &= At\e^{-t} \\
  y_p' &= A\e{-t}-At\e^{-t} \\
  y_p'' &= -A\e^{-t}-(A\e^{-t}-At\e^{-t}) \\
  &= -2A\e^{-t}+At\e^{-t} \\
  (-2A\e^{-t}+At\e^{-t})-(At\e^{-t}) &= 10\e^{-t} \\
  -2A\e^{-t} &= 10\e^{-t} \\
  -2A &= 10 \\
  A &= 5 \\
  y_p &= -5t\e^{-t}
\end{align*}

\subsubsection*{Example}
\begin{align*}
  y''+17y &= -7\e^{0t} \\
  r &= 0 \\
  r^2+17 &= 0 \\
  r_1 &= 0+17i \quad r_2 = 0-17i \\
  y_p &= A \\
  y_p' &= y_p'' = 0 \\
  0+17A &= -7 \\
  A &= -\frac{7}{17} \\
  y_p &= -\frac{7}{17}
\end{align*}

\subsection*{Methodology}
The method of undetermined coefficients is a method to find a particular
solution \( y_p \) to \( ay''+by'+cy = f(t) \ne 0 \).
\begin{enumerate}
  \item When \( f(t) = Ct^m\e^{rt} \) where \( C \) is any constant, \( r \) is
  a real number, and \( m \) is a non-negative integer, the general form of
  \( y_p \) will be:
  \[ y_p(t) = t^s\bigg[A_mt^m+A_{m-1}t^{m-1}+\dots+A_1t+A_0\bigg]\e^{rt} \]
  where \( s = 0 \) if \( r \) doesn't match the roots of the characteristic
  equation, \( s = 1 \) if \( r \) is a match to the roots of the characteristic
  equation and the roots are non-repeated, and \( s = 2 \) if \( r \) is a
  repeated real root of the corresponding auxiliary equation.
  \item When \( f(t) = Ct^m\e^{\alpha t}\cos(\beta t) \) or \( f(t) =
  Ct^m\e^{\alpha t}\sin(\beta t) \) where \( m \) is a positive integer, \( C \)
  is any constant, \( \alpha \) is real, and \( \beta \) is a nonzero real
  number, the form of \( y_p \) will be:
  \begin{align*}
    y_p(t) =\ & t^s\bigg[A_mt^m+A_{m-1}t^{m-1}+\dots+A_1t+A_0\bigg]\e^{\alpha t}
      \cos(\beta t)+ \\
    & t^s\bigg[B_mt^m+B_{m-1}t^{m-1}+\dots+B_1t+B_0\bigg]\e^{\alpha t}
      \sin(\beta t)
  \end{align*}
  where \( s = 0 \) if \( r = \alpha\pm i\beta \) is a root of the
  characteristic equation and there is no match with \( \alpha \) and
  \( \beta \), or \( s = 1 \) if \( r = \alpha\pm i\beta \) is a root of the
  characteristic equation.
\end{enumerate}
Note that these two cases only cover functions \( f(t) \) that can be rewritten
as the forms above. If \( f(t) \) cannot be expressed in the forms above, then
the problem cannot be solved using the method of undetermined coefficients.

\subsubsection*{Example}
\begin{align*}
  y''+3y'+2y &= 5\sin(t)t^0\e^0t \\
  r^2+3r+2 &= (r+2)(r+1) = 0 \\
  r &= -1 \quad r = -2 \\
  y_p &= A\cos(t)+B\sin(t) \\
  y_p' &= -A\sin(t)+B\cos(t) \\
  y_p'' &= -A\cos(t)-B\sin(t)
\end{align*}
Since the roots are real, there is no match to \( \alpha \) or \( \beta \) when
forming the particular solution.
\begin{align*}
  y''+3y'+2y &= 5\sin(t) \\
  -A\cos(t)-B\sin(t)+3\bigg[-A\sin(t)+B\cos(t)\bigg]+
    2\bigg[A\cos(t)+B\sin(t)\bigg] &= 5\sin(t) \\
  -3A\sin(t)+B\sin(t)-A\cos(t)+3B\cos(t) &= 5\sin(t) \\
  (-3A+B)\sin(t)+(-A+3B)\cos(t) &= 5\sin(t) \\
  -3A+B = 5 \quad -A+3B &= 0 \\
  A = -\frac{3}{2} \quad B &= \frac{1}{2}
\end{align*}
By equating the like coefficients, we can find an explicit solution:
\[ y_p = -\frac{3}{2}\cos(t)+\frac{1}{2}\sin(t) \]

\subsubsection*{Example}
Find the general form of a particular solution \( y_p \).
\begin{align*}
  y''+8y'-9y &= 10t^2\e^t\cos(2t) \\
  r^2+8r-9 &= (r+9)(r-1) = 0 \\
  r &= -9 \quad r = 1 \quad s = 0 \\
  \e^t &= \e^{1t} \quad \alpha = 1 \quad \beta = 2 \\
  y_p &= \bigg[At^2+Bt+C\bigg]\e^t\cos(2t)+\bigg[Dt^2+Et+F\bigg]\e^t\sin(2t)
\end{align*}

\clearpage
\section*{Superposition}
Let \( y_1 \) be a solution to \( ay''+by'+cy = f_1(t) \) and let \( y_2 \) be
a solution to \( ay''+by'+cy = f_2(t) \). Then \( y = k_1y_1+k_2y_2 \) is a
solution to \( ay''+by'+cy = k_1f_1(t)+k_2f_2(t) \). If \( f_1(t) = 0 \), then
we can solve the homogeneous equation:
\[ y_h = c_1y_2+c_2y_2 \]
and \( ay''+by'+cy = f_2(t) \ne 0 \). We find a particular solution to the
second equation, \( y_p \), and the general solution to \( ay''+by'+cy = f(t) \)
will be \( y = y_h+y_p \), where \( y_h \) is the general solution to the
corresponding homogeneous equation.

\subsubsection*{Example}
Solve the following initial value problem.
\[ y''-y = -11t+1 \quad y(0) = -1 \quad y'(0) = 1 \]
We can use the method of superposition to solve this as:
\[ y = y_h+y_p \]
We first need to find the solution to the corresponding homogeneous equation
\( y_h \):
\begin{align*}
  y''-y &= 0 \\
  r^2-1 &= 0 \\
  r &= \pm1 \\
  y_h &= c_1\e^t+c_2\e^{-t}
\end{align*}
Now we need to find a particular solution \( y_p \):
\begin{align*}
  y_p &= At+B \\
  y_p' &= A \\
  y_p'' &= 0 \\
  0-(At+B) &= -11t+1 \\
  -At-B &= -11t+1 \\
  -A &= 11 \quad -B = 11 \\
  y_p &= 11t-1
\end{align*}
Now we have both solutions and we can solve for the arbitrary constants:
\begin{align*}
  y &= y_h+y_p = c_1\e^t+c_2\e^{-t}+11t-1 \\
  y(0) &= -1 = c_1+c_2-1 \\
  c_1 &= -c_2 \\
  y' &= c_1\e^t-c_2\e^{-t}+11 \\
  y'(0) &= 1 = c_1-c_2+11 \\
  -1 &= -2c_2 \\
  c_2 &= 5 \quad c_1 = -5 \\
  y &= -5\e^t+5\e^{-t}+11t-1
\end{align*}

\subsubsection*{Example}
Solve the following initial value problem.
\[ y''+9y = 27 \quad y(0) = 4 \quad y'(0) = 6 \]
Homogeneous solution:
\begin{align*}
  y''+9y &= 27 \\
  y &= y_h+y_p \\
  r^2+9 &= 0 \\
  r &= 0\pm3i \quad \alpha = 0 \quad \beta = 3 \\
  y_h &= \e^{0t}\bigg[c_1\cos(3t)+c_2\sin(3t)\bigg]
\end{align*}
Particular solution:
\begin{align*}
  y_p &= A \\
  y_p' &= y_p'' = 0 \\
  0+9(A) &= 27 \quad A = 3 \\
  y_p &= 3
\end{align*}
Solve for arbitrary constants:
\begin{align*}
  y &= y_h+y_p = c_1\cos(3t)+c_2\sin(3t)+3 \\
  y(0) &= 4 = c_1+3 \\
  y' &= -3c_1\sin(3t)+3c_2\cos(3t) \\
  y'(0) &= 6 = 3c_2 \quad c_2 = 2 \\
  y &= \cos(3t)+2\sin(3t)+3
\end{align*}

\subsubsection*{Example}
Find the general solution to the following differential equation.
\[ y''+9y = \cos(t)+1 \]
Homogeneous solution:
\begin{align*}
  y''+9y &= 0 \\
  r^2+9 &= 0 \\
  r &= 0\pm3i \quad \alpha = 0 \quad \beta = 3 \\
  y_h &= c_1\cos(3t)+c_2\sin(3t)
\end{align*}
Particular solution:
\begin{align*}
  y_p &= A\cos(t)+B\sin(t)+C \\
  y_p' &= -A\sin(t)+B\cos(t) \\
  y_p'' &= -A\cos(t)-B\sin(t) \\
  -A\cos(t)-B\sin(t)+9\bigg[A\cos(t)+B\sin(t)+C\bigg] &= \cos(t)+1 \\
  8A\cos(t)+8B\sin(t)+9C &= \cos(t)+1 \\
  8A &= 1 \quad 8B = 0 \quad 9C = 1 \\
  A &= \frac{1}{8} \quad B = 0 \quad C = \frac{1}{9} \\
  y_p &= \frac{\cos(t)}{8}+\frac{1}{9}
\end{align*}
General solution:
\[ y = c_1\cos(3t)+c_2\sin(3t)+\frac{\cos(t)}{8}+\frac{1}{9} \]

\subsubsection*{Example}
Solve the following initial value problem.
\[ y''+4y = \sin(t)-\cos(t) \quad y(0) = 1 \quad y'(0) = 0 \]
Homogeneous solution:
\begin{align*}
  y''+4y &= 0 \\
  r^2+4 &= 0 \\
  r &= 0\pm2i \quad \alpha = 0 \quad \beta = 2
\end{align*}
Particular solution:
\begin{align*}
  y_p &= A\cos(t)+B\sin(t) \\
  y_p' &= -A\sin(t)+B\cos(t) \\
  y_p'' &= -A\cos(t)-B\sin(t) \\
  -A\cos(t)-B\sin(t)+4\bigg[A\cos(t)+B\sin(t)\bigg] &= \sin(t)-\cos(t) \\
  3A\cos(t)+3B\sin(t) &= \sin(t)-\cos(t) \\
  3A &= -1 \quad A = -\frac{1}{3} \\
  3B &= 1 \quad B = \frac{1}{3} \\
  y_p &= -\frac{1}{3}\cos(t)+\frac{1}{3}\sin(t)
\end{align*}
Solve for arbitrary constants:
\begin{align*}
  y &= y_h+y_p \\
  y &= c_1\cos(2t)+c_2\sin(2t)-\frac{1}{3}\cos(t)+\frac{1}{3}\sin(t) \\
  y(0) &= 1 = c_1-\frac{1}{3} \\
  c_1 &= \frac{4}{3} \\
  y' &= -2c_1\sin(2t)+2c_2\cos(2t)+\frac{1}{3}\sin(t)+\frac{1}{3}\cos(t) \\
  y'(0) &= 0 = 2c_2+\frac{1}{3} \\
  c_2 &= -\frac{1}{6} \\
  y &= \frac{4}{3}\cos(2t)-\frac{1}{6}\sin(2t)-\frac{1}{3}\cos(t)+
    \frac{1}{3}\sin(t)
\end{align*}

\clearpage
\section*{Variation of Parameters}
This technique is used to find a particular solution, \( y_p \), to
\[ ay''+by'+cy = f(t) \ne 0 \]
if the method of undetermined coefficients is not applicable. Given this
equation:
\begin{align*}
  y &= y_h+y_p \\
  y_h &= c_1y_1+c_2y_2 \\
  y_p &= v_1y_1+v_2y_2 \\
  v_1 &= -\int\frac{f(t)y_2}{aW[y_1,y_2]}\diff{t} \\
  v_2 &= \int\frac{f(t)y_1}{aW[y_1,y_2]}\diff{t}
\end{align*}

\subsubsection*{Example}
Find the general solution to the following differential equation.
\[ y''+2y'+y = \e^{-t}\ln(t) \]
Homogeneous Solution:
\begin{align*}
  r^2+2r+1 &= 0 \\
  (r+1)^2 &= 0 \\
  r &= -1 \\
  y_h &= c_1\e^{-t}+c_2t\e^{-t}
\end{align*}
Particular Solution:
\begin{align*}
  y_h &= c_1\e^{-t}+c_2t\e^{-t} \\
  y_p &= v_1y_1+v_2y_2 \\
  y_1 &= \e^{-t} \quad y_2 = t\e^{-t} \\
  v_1 &= -\int\frac{\e^{-t}\ln(t)t\e^{-t}}{W[\e^{-t},t\e^{-t}]}\diff{t} \\
  W[\e^{-t},t\e^{t}] &= \begin{vmatrix}
    \e^{-t} & t\e^{-t} \\
    -\e^{-t} & \e^{-t}-t\e^{-t}
  \end{vmatrix} = \e^{-2t} \\
  v_1 &= -\int\frac{\e^{-2t}t\ln(t)}{\e^{-2t}}\diff{t} =
    \frac{t^2}{4}-\frac{t^2}{2}\ln(t)
\end{align*}
\begin{align*}
  v_2 &= \int\frac{\e^{-t}\ln(t)}{\e^{-2t}} \\
  &= \int\ln(t)\diff{t} \\
  &= t(\ln(t)-1) \\
  y &= y_h+y_p \\
  y &= c_1\e^{-t}+c_2t\e^{-t}+
    \bigg[\frac{t^2}{4}-\frac{t^2}{2}\ln(t)\bigg]\e^{-t} +
    \bigg[t(\ln(t)-1)\bigg]t\e^{-t} \\
\end{align*}

\subsubsection*{Example}
Find the general solution to the following differential equation over
\( (-\frac{\pi}{2},\frac{\pi}{2}) \).
\[ y''+y = \tan(t) \]
Homogeneous Solution:
\begin{align*}
  r^2+1 &= 0 \\
  r^2 &= -1 \\
  r &= 0\pm i \quad \alpha = 0 \quad \beta = 1 \\
  y_h &= c_1\cos(t)+c_2\sin(t)
\end{align*}
Particular Solution:
\begin{align*}
  y_h &= c_1\cos(t)+c_2\sin(t) \\
  y_p &= v_1y_1+v_2y_2 \\
  v_1 &= -\int\frac{\tan(t)\sin(t)}{W[\cos(t),\sin(t)]}\diff{t} \\
  W[\cos(t),\sin(t)] &= 1 \\
  v_1 &= -\int\frac{\sin(t)\sin(t)}{\cos(t)}\diff{t} =
    -\int\frac{\sin^2(t)}{\cos(t)}\diff{t} \\
  &= -\int\frac{1-\cos^2(t)}{\cos(t)}\diff{t} \\
  &= -\bigg[\int\frac{1}{\cos(t)}\diff{t}-\int\cos(t)\diff{t}\bigg] \\
  &= \sin(t)-\ln|\sec(t)+\tan(t)|
\end{align*}
\begin{align*}
  v_2 &= \int\tan(t)\cos(t)\diff{t} \\
  &= \int\sin(t)\diff{t} \\
  &= -\cos(t) \\
  y &= y_h+y_p \\
  y &= c_1\cos(t)+c_2\sin(t)+\bigg[\sin(t)-\ln|\sec(t)+\tan(t)|\bigg]\cos(t)-
    \cos(t)\sin(t)
\end{align*}

\clearpage
\section*{Mass Spring Systems}

\subsection*{Free Mechanical Vibrations}
The governing equation for mass spring systems is given by
\[ my''+by'+ky = f(t) \]
where \( m \) is mass in kilograms, \( b \) is a damping coefficient (friction)
in Newton-seconds per meter, \( k \) is the spring constant (stiffness)
in Newtons per meter, and \( f \) is the external force. For free mechanical
vibrations, \( f(t) = 0 \).

\subsection*{Simple Harmonic Motion}
When we deal with simple harmonic motion, there is no external force, and
friction is negligible. Thus, our equation becomes:
\[ my''+ky = 0 \]
We can solve this equation using the techniques above.
\begin{align*}
  y''+\frac{k}{m}y &= 0 \\
  Let: \omega &= \sqrt{\frac{k}{m}} \\
  \omega^2 &= \frac{k}{m} \\
  y''+\omega^2y &= 0 \\
  r^2+\omega^2 &= 0 \\
  r^2 &= -\omega^2 \\
  r &= 0\pm\omega i \quad \alpha = 0 \quad \beta = \omega \\
  y &= c_1\cos(\omega t)+c_2\sin(\omega t)
\end{align*}
Given an initial value problem, we can solve for \( c_1 \) and \( c_2 \). We can
express \( y \) as the following equation:
\[ y = A\sin(\omega t+\phi) \]
where \( A = \sqrt{c_1^2+c_2^2} \) is the amplitude, \( \phi \) is a phase
angle, \( \omega \) is the angular frequency in radians per second, \( T =
\frac{2\pi}{\omega} \) is the period in seconds, and \( \frac{\omega}{2\pi} \)
is the natural frequency in cycles per second. To find \( \phi \) and \( A \)
in terms of \( c_1 \) and \( c_2 \), consider the following trigonometric
identity:
\begin{align*}
  \sin(\alpha+\beta) &= \cos\alpha\sin\beta+\cos\beta\sin\alpha \\
  \sin(\omega t+\phi) &= \cos(\omega t)\sin\phi+\cos\phi\sin(\omega t) \\
  A\bigg[\sin(\omega t+\phi)\bigg] &=
    A\cos(\omega t)\sin\phi+A\cos\phi\sin(\omega t) \\
  y = A\sin(\omega t+\phi) &= c_1\cos(\omega t)+c_2\sin(\omega t) \\
  A\sin\phi &= c_1 \quad c_1 = A\sin\phi \\
  A\cos\phi &= c_2 \quad c_2 = A\cos\phi \\
  A &= \sqrt{c_1^2+c_2^2} \\
  \tan\phi &= \frac{c_1}{c_2} \quad \phi = \arctan(\frac{c_1}{c_2})
\end{align*}
The signs of \( c_1 \) and \( c_2 \) will determine what quadrant \( \phi \)
lies in.

\subsubsection*{Example}
A 1kg mass is attached to a spring with stiffness \( k = 64\frac{N}{m} \). The
mass is stretched in the positive direction from the equilibrium position
\( y = 0 \) by \( \frac{2}{3}m \), and given an initial velocity of
\( -\frac{4}{3}\frac{m}{s} \) in the opposite direction. Neglecting external
forces and friction:
\begin{enumerate}
  \item Find the equation of motion.
  \begin{align*}
    y''+64y &= 0 \\
    y(0) &= \frac{2}{3} \quad y'(0) = -\frac{4}{3} \\
    r^2+64 &= 0 \quad r = 0\pm8i \\
    y(t) &= c_1\cos(8t)+c_2\sin(8t) \\
    y'(t) &= -8c_1\sin(8t)+8c_2\cos(8t)
  \end{align*}
  \begin{align*}
    y(0) &= \frac{2}{3} = c_1 \\
    y'(0) &= -\frac{4}{3} = 8c_2 \\
    y &= \frac{2}{3}\cos(8t)-\frac{1}{6}\sin(8t) \\
    A &= \sqrt{\bigg(\frac{2}{3}\bigg)^2+\bigg(-\frac{1}{6}\bigg)^2} =
      \frac{\sqrt{17}}{6} \\
    \tan\phi &= \frac{c_1}{c_2} = -4
  \end{align*}
  From the signs of \( c_1 \) and \( c_2 \), we know \( \phi \) lies in quadrant
  II.
  \begin{align*}
    \phi &= \arctan(-4) \approx (-1.326+\pi)~rad \approx 1.816~rad \\
    y(t) &= \frac{\sqrt{17}}{6}\sin(8t+1.816)
  \end{align*}
  \item Find the first time that the mass passes over the equilibrium position.
  \begin{align*}
    y(t) &= 0 = \frac{\sqrt{17}}{6}\sin(8t+1.816) \\
    \sin(8t+1.816) &= 0 \\
    8t+1.816 &= n\pi, \quad n\in\Z \\
    8t &= n\pi-1.816 \\
    t &= \frac{n\pi-1.816}{8} \\
    t &= \frac{\pi-1.816}{8}
  \end{align*}
\end{enumerate}

\subsection*{Under-damped Motion}
When the damping coefficient exists but is small relative to the spring's
stiffness, oscillation occurs and eventually flattens out. The roots of the
characteristic equation will typically be complex.

\subsubsection*{Example}
A 1kg mass is attached to a spring with constant \( k = 10\frac{N}{m} \) and a
damping coefficient of \( b = 2\frac{N\cdot s}{m} \). Find the equation of
motion given the initial conditions \( y(0) = -1 \) and \( y'(0) = 0 \).
\begin{align*}
  my''+by'+ky &= 0 \\
  y''+2y'+10 &= 0 \\
  r &= \frac{-2\pm\sqrt{4-(4)(10)}}{2} \\
  &= -1\pm3i \quad \alpha = -1 \quad \beta = 3 \\
  y &= \e^{-t}\bigg[c_1\cos(3t)+c_2\sin(3t)\bigg] \\
  y' &= \e^{-t}\bigg[-3c_1\sin(3t)+3c_2\cos(3t)\bigg]-
    \e^{-t}\bigg[c_1\cos(3t)+c_2\sin(3t)\bigg] \\
  y(0) &= -1 = c_1 \\
  y'(0) &= 0 = 3c_2-c_1 \\
  c_2 &= -\frac{1}{3} \\
  y &= \e^{-t}\bigg[-\cos(3t)-\frac{1}{3}\sin(3t)\bigg]
\end{align*}
Now we need to rewrite this in the form \( y =
A\e^{-t}\bigg[\sin(3t+\phi)\bigg] \):
\begin{align*}
  A &= \sqrt{(-1)^2+(-\frac{1}{3})^2} = \frac{\sqrt{10}}{3} \\
  \tan\phi &= \frac{c_1}{c_2} = 3 \\
\end{align*}
From the signs of \( c_1 \) and \( c_2 \), we know \( \phi \) lies in quadrant
III.
\begin{align*}
  \phi &= \arctan(-4) \approx (1.25+\pi)~rad \approx 4.39~rad \\
  y(t) &= \frac{\sqrt{10}}{3}\e^{-t}\bigg[\sin(3t+4.39)\bigg]
\end{align*}
Note the following property of this equation of motion:
\[ \lim_{t\to\infty}y(t) = 0 \]
This is due the the amplitude also being a variable function of time that
approaches 0 as time goes to infinity. This is not a periodic function. A
``quasi-period`` \( \frac{2\pi}{\beta} = \frac{2\pi}{3} \) exists, as well
as a ``quasi-frequency'' \( \frac{\beta}{2\pi} = \frac{3}{2\pi} \).

\subsection*{Over-damped Motion}
This can occur when the damping coefficient is large compared to the spring
stiffness. The roots of the characteristic equation will typically be real
and distinct.

\subsubsection*{Example}
Find the equation of motion for \( y''+3y'+2y = 0 \) given the initial
conditions \( y(0) = 1 \) and \( y'(0) = 0 \).
\begin{align*}
  y''+3y'+2y &= 0 \\
  r^2+3r+2 &= (r+1)(r+2) = 0 \\
  r &= -1 \quad r = -2 \\
  y(t) &= c_1\e^{-t}+c_2\e^{-2t} \\
  y'(t) &= -c_1\e^{-t}-2c_2\e^{-t} \\
  y(0) &= 1 = c_1+c_2 \quad c_1 = 1-c_2 \\
  y'(0) &= 0 = -c_1-c_2 \\
  c_1 &= 2 \quad c_2 = -1 \\
  y(t) &= 2\e^{-t}-\e^{-2t}
\end{align*}
Notice here that \( \lim_{t\to\infty}y(t) = 0 \) and there is no oscillation.

\subsection*{Critically-damped Systems}
Suppose we a spring system defined by \( y''+10y'+25y = 0 \) with the initial
conditions \( y(0) = 1 \) and \( y'(0) = 0 \).
\begin{align*}
  y''+10y'+25y &= 0 \\
  r^2+10r+25 &= (r+5)^2 = 0 \\
  r &= -5 \\
  y(t) &= c_1\e^{-5t}+c_2t\e^{-5t} \\
  y'(t) &= -5c_1\e^{-5t}+c_2\e^{-5t}-5c_2t\e^{-5t} \\
  y(0) &= 1 = c_1 \\
  y'(0) &= 0 = -5+c_2 \quad c_2 = 5 \\
  y(t) &= \e^{-5t}+5t\e^{-5t}
\end{align*}
Note that this system also will not oscillate as \( \lim_{t\to\infty}y(t) =
0 \).

\begin{center}
  You can find all my notes at \url{http://omgimanerd.tech/notes}. If you have
  any questions, comments, or concerns, please contact me at
  alvin@omgimanerd.tech
\end{center}

\end{document}
