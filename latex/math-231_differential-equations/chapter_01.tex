\documentclass{math}

\title{Differential Equations}
\author{Alvin Lin}
\date{January 2018 - May 2018}

\begin{document}

\maketitle

\section*{Chapter 1}

\subsubsection*{Concept Review}
\begin{align*}
  \int\frac{\diff{x}}{1+x^2} &\ne \ln(1+x^2)+c \\
  \int\frac{\diff{x}}{1+x^2} &= \arctan(x)+c
\end{align*}

\subsubsection*{Example}
\begin{align*}
  \int\frac{x}{1+x^2}\diff{x} \\
  Let: u &= x^2+1 \\
  \diff{u} &= 2x\diff{x} \\
  \frac{\diff{u}}{2} = x\diff{x} \\
  \int\frac{x}{1+x^2}\diff{x} &= \frac{1}{2}\int\frac{\diff{u}}{u} \\
  &= \frac{1}{2}\ln|u|+c \\
  &= \frac{1}{2}\ln|x^2+1|+c
\end{align*}

\subsubsection*{Example}
\begin{align*}
  \int\e^{2x}\diff{x} &\ne \frac{\e^{2x+1}}{2x+1}+c \\
  &= \frac{1}{2}\e^{2x}+c
\end{align*}

\subsubsection*{Example}
\begin{align*}
  \int\ln(x)\diff{x} \\
  Let: u &= \ln(x) \quad \diff{v} = \diff{x} \\
  \diff{u} &= \diff{x} \quad v = x \\
  &= x\ln(x)-\int{x\diff{x}}
\end{align*}

\subsubsection*{Properties of Logs}
\begin{enumerate}
  \item \( \ln(ab) = \ln(a)+\ln(b) \) given \( a,b\ne0 \)
  \item \( \ln(\frac{a}{b}) = \ln(a)-\ln(b) \)
  \item \( r\ln(a) = \ln(a^r) \)
\end{enumerate}

\subsubsection*{Rules of Exponents}
\begin{enumerate}
  \item \( \e^{x+y} = \e^x\e^y \)
  \item \( \e^{x-y} = \frac{\e^x}{\e^y} \)
  \item \( \e^{x^y} = \e^{xy} \)
\end{enumerate}

\section*{Solutions and Initial Value Problems}
A differential equation is an equation that contains one or more derivatives of
some unknown function.
\[ y''-\frac{2}{x^2}y = 0 \]
Using Leibniz's Notation:
\[ \ddiff{^2y}{x^2}-\frac{2}{x^2}y(x) = 0 \]
Another example:
\[ y''+3y'+2y = 0 \]
A function \( \phi \) defined on some interval \( I \) having at least \( n \)
continuous derivatives on \( I \), is an explicit solution over \( I \) if it
satisfies the equation on \( I \).

\subsection*{Classification by Order and Linearity}
The order of a differential equation is the order of the highest derivative that
appears in the equation.
\[ y'''+2y'+y = \e^x \]
is a third order equation. With respect to linearity, consider the following:
\[ a_n(x)y^{(n)}+a_{n-1}y^{(n-1)}+\dots+a_1(x)y'+a_0(x)y = g(x) \]
where \( a_n(x) \) is a function of the independent variable only,
\( y^{(n)} \) is a derivative to the n\textsuperscript{th} power and \( g(x) \)
is a function of \( x \) only.
\begin{itemize}
  \item \( \ddiff{^2y}{x^2} = -\cos(y) \) is a non-linear, second-order
  differential equation.
  \item \( y''+\ln(y)y'-5y = \e^x \) is a non-linear, second-order differential
  equation.
  \item \( y''+\ln(x)y-5y = \e^x \) is a linear, second-order differential
  equation.
\end{itemize}

\subsection*{Example}
Show that \( y = \phi(x) = x^2-\frac{1}{x} \) is an explicit solution to
\( y''-\frac{2}{x^2}y = 0 \).
\begin{align*}
  y &= x^2-\frac{1}{x} \\
  y' &= 2x+x^{-2} \\
  y'' &= 2-2x^{-3} \\
  y''-2\frac{x^2}y &= 2-2x^{-3}-\frac{2}{x^2}(x^2-\frac{1}{x}) \\
  &= 2-\frac{2}{x^3}-2+\frac{2}{x^3} \\
  &= 0
\end{align*}
Hence, \( y \) satisfies this differential equation.

\subsection*{Example}
Verify that \( y(t)=\e^{-2t}\sin(4t) \) is a solution to \( y''+4y'+20y = 0 \).
\begin{align*}
  y &= \e^{-2t}\sin(4t) \\
  y' &= -2\e^{-2t}\sin(4t)+\e^{-2t}(4)\cos(4t) \\
  y'' &= 4\e^{-2t}\sin(4t)+(-2\e^{-2t})(4)\cos(4t)+(4)(-2\e^{-2t})\cos(4t)+
    \e^{-2t}(-16)\sin(4t) \\
  y''+4y'+20y &= 4\e^{-2t}\sin(4t)-8\e^{-2t}\cos(4t)-8\e^{-2t}\cos(4t)- \\
    &~~~~ 16\e^{-2t}\sin(4t)+
      4\bigg(-2\e^{-2t}\sin(4t)+4\e^{-2t}\cos(4t)\bigg)+ \\
    &~~~~ 20\e^{-2t}\sin(4t) \\
  &= 24\e^{-2t}\sin(4t)-24\e^{-2t}\sin(4t)+
    16\e^{-2t}\cos(4t)-16\e^{-2t}\cos(4t) \\
  &= 0
\end{align*}

\begin{center}
  You can find all my notes at \url{http://omgimanerd.tech/notes}. If you have
  any questions, comments, or concerns, please contact me at
  alvin@omgimanerd.tech
\end{center}

\end{document}
