\documentclass{math}

\geometry{letterpaper, margin=0.5in}

\title{Differential Equations: Homework 5}
\author{Alvin Lin}
\date{January 2018 - May 2018}

\begin{document}

\maketitle
\clearpage

\section*{Section 3.5}

\subsubsection*{Exercise 1}
An RL circuit with a \( 5-\Omega \) resistor and a 0.05-H inductor carries a
current of 1 A at \( t = 0 \), at which time a voltage source
\( E(t) = 5\cos(120t)V \) is added. Determine the subsequent inductor current
and voltage.
\begin{align*}
  L\ddiff{i}{t}+Ri &= E(t) \\
  \ddiff{i}{t}+\frac{R}{L}i &= \frac{E(t)}{L} \\
  \ddiff{i}{t}+\frac{5}{0.05}i &= \frac{5\cos(120t)}{0.05} \\
  \ddiff{i}{t}+100i &= 100\cos(120t) \\
  \mu(t) &= \e^{\int100\diff{t}} = \e^{100t} \\
  \e^{100t}\ddiff{i}{t}+\e^{100t}100i &= 100\cos(120t)\e^{100t} \\
  \int\bigg(\e^{100t}\ddiff{i}{t}+\e^{100t}100i\bigg)\diff{t} &=
    \int100\cos(120t)\e^{100t}\diff{t} \\
  i\e^{100t}+c &= \int100\cos(120t)\e^{100t}\diff{t} \\
  i &= \e^{-100t}\bigg(\int100\cos(120t)\e^{100t}\diff{t}+c\bigg) \\
  &= \frac{5}{61}(6\sin(120t)+5\cos(120t))+c\e^{-100t} \\
  1 &= \frac{5}{61}(6\sin(0)+5\cos(0))+c \\
  c &= 1-\frac{25}{61} \\
  i(t) &= \frac{30}{61}\sin(120t)+\frac{25}{61}\cos(120t)+\frac{36}{61}\e^{-100t} \\
\end{align*}

\subsubsection*{Exercise 7}
An industrial electromagnet can be modeled as an RL circuit, while it is being
energized with a voltage source. If the inductance is 10H and the wire windings
contain \( 3\Omega \) of resistance, how long does it take a constant applied
voltage to energize the electromagnet to within 90\% of its final value (that
is, the current equals 90\% of its asymptotic value)?
\begin{align*}
  \ddiff{i}{t}+\frac{R}{L}i &= \frac{E}{L} \\
  \mu(t) &= \e^{\int\frac{R}{L}\diff{t}} = \e^{\frac{Rt}{L}} \\
  \e^{\frac{Rt}{L}}\ddiff{i}{t}+\e^{\frac{Rt}{L}}\frac{R}{L}i &=
    \e^{\frac{Rt}{L}}\frac{E}{L} \\
  ie^{\frac{R}{L}t}+c &= \int\e^{\frac{Rt}{L}}\frac{E}{L}\diff{t} \\
  i &= \frac{E}{R}+c\e^{-\frac{Rt}{L}}
\end{align*}
\begin{align*}
  0 &= \frac{E}{R}+c\e^0 \\
  c &= -\frac{E}{R} \\
  i &= \frac{E}{R}(1-\e^{-\frac{Rt}{L}}) \\
  \lim_{t\to\infty}i &= \frac{E}{R} \\
  0.9\frac{E}{R} &= i = \frac{E}{R}(1-\e^{-\frac{Rt}{L}}) \\
  0.9 &= 1-\e^{-\frac{Rt}{L}} \\
  \e^{-\frac{Rt}{L}} &= 0.1 \\
  -\frac{Rt}{L} &= \ln(0.1) \\
  t &= \frac{-L\ln(0.1)}{R} \\
  &= \frac{-10\ln(0.1)}{3} \approx 7.67
\end{align*}

\section*{Section 4.2}

\subsubsection*{Exercise 1}
Find a general solution to the given differential equation.
\begin{align*}
  2y''+7y'-4y &= 0 \\
  y(t) &= \e^{rt} \\
  y'(t) &= r\e^{rt} \\
  y''(t) &= r^2\e^{rt} \\
  2r^2\e^{rt}+7r\e^{rt}-4\e^{rt} &= 0 \\
  2r^2+7r-4 &= 0 \\
  (2r-1)(r+4) &= 0 \\
  r &= \frac{1}{2} \quad r = -4 \\
  y &= \e^{\frac{1}{2}t} \quad y = \e^{-4t} \\
  y &= c_1\e^{\frac{1}{2}t}+c_2\e^{-4t}
\end{align*}

\subsubsection*{Exercise 2}
Find a general solution to the given differential equation.
\begin{align*}
  y''+6y'+9y &= 0 \\
  r^2+6r+9 &= (r+3)(r+3) = 0 \\
  r &= -3 \\
  y &= c_1\e^{-3t}+c_2t\e^{-3t}
\end{align*}

\subsubsection*{Exercise 3}
Find a general solution to the given differential equation.
\begin{align*}
  y''+5y'+6y &= 0 \\
  r^2+5r+6 &= (r+3)(r+2) = 0 \\
  r &= -3 \quad r = -2 \\
  y &= \e^{-3t} \quad y = \e^{-2t} \\
  y &= c_1\e^{-3t}+c_2\e^{-2t}
\end{align*}

\subsubsection*{Exercise 4}
Find a general solution to the given differential equation.
\begin{align*}
  y''-y'-2y &= 0 \\
  r^2-r-2 &= (r-2)(r+1) = 0 \\
  r &= 2 \quad r = -1 \\
  y &= \e^{2t} \quad y = \e^{-t} \\
  y &= c_1\e^{2t}+c_2\e^{-t}
\end{align*}

\subsubsection*{Exercise 7}
Find a general solution to the given differential equation.
\begin{align*}
  6y''+y'-2y &= 0 \\
  6r^2+r-2 &= (2r-1)(3r+2) = 0 \\
  r &= \frac{1}{2} \quad r = -\frac{2}{3} \\
  y &= \e^{\frac{1}{2}t} \quad y = \e^{-\frac{2}{3}t} \\
  y &= c_1\e^{\frac{1}{2}t}+c_2\e^{-\frac{2}{3}t}
\end{align*}

\subsubsection*{Exercise 8}
Find a general solution to the given differential equation.
\begin{align*}
  z''+z'-z &= 0 \\
  r^2+r-1 &= 0 \\
  r &= \frac{-1\pm\sqrt{5}}{2} \\
  z &= \e^{\frac{-1+\sqrt{5}}{2}t} \quad z = \e^{\frac{-1-\sqrt{5}}{2}t} \\
  z &= c_1\e^{\frac{-1+\sqrt{5}}{2}t}+c_2\e^{\frac{-1-\sqrt{5}}{2}t}
\end{align*}

\subsubsection*{Exercise 11}
Find a general solution to the given differential equation.
\begin{align*}
  4w''+20w'+25w &= 0 \\
  4r^2+20r+25 &= (2r+5)^2 = 0 \\
  r &= -\frac{5}{2} \\
  w &= c_1\e^{-\frac{5}{2}t}+c_2t\e^{-\frac{5}{2}t}
\end{align*}

\subsubsection*{Exercise 13}
Solve the given initial value problem.
\begin{align*}
  y''+2y'-8y &= 0 \\
  r^2+2r-8 &= (r+4)(r-2) = 0 \\
  r &= -4 \quad r = 2 \\
  y &= \e^{-4t} \quad y = \e^{2t} \\
  y &= c_1\e^{-4t}+c_2\e^{2t}
\end{align*}
Using the initial values of \( y(0) = 3 \) and \( y'(0) = -12 \):
\begin{align*}
  y &= c_1\e^{-4t}+c_2\e^{2t} \\
  3 &= c_1+c_2 \\
  y' &= -4c_1\e^{-4t}+2c_2\e^{2t} \\
  -12 &= -4c_1+2c_2 \\
  c_1 &= 3 \quad c_2 = 0 \\
  y &= 3\e^{-4t}
\end{align*}

\subsubsection*{Exercise 14}
Solve the given initial value problem.
\begin{align*}
  y''+y' &= 0 \\
  r^2+r &= r(r+1) = 0 \\
  r &= 0 \quad r = -1 \\
  y &= \e^{0t} = 1 \quad y = \e^{-t} \\
  y &= c_1+c_2\e^{-t}
\end{align*}
Using the initial values of \( y(0) = 2 \) and \( y'(0) = 1 \):
\begin{align*}
  y &= c_1+c_2\e^{-t} \\
  2 &= c_1+c_2 \\
  y' &= -c_2\e^{-t} \\
  1 &= -c_2 \\
  c_1 &= 3 \quad c_2 = -1 \\
  y &= -\e^{-3}+3
\end{align*}

\subsubsection*{Exercise 15}
Solve the given initial value problem.
\begin{align*}
  y''-4y'+3y &= 0 \\
  r^2-4r+3 &= (r-3)(r-1) = 0 \\
  r &= 3 \quad r = 1 \\
  y &= \e^{3t} \quad y = \e^t \\
  y &= c_1\e^{3t}+c_2\e^t
\end{align*}
Using the initial value of \( y(0) = 1 \) and \( y'(0) = \frac{1}{3} \):
\begin{align*}
  y &= c_1\e^{3t}+c_2\e^t \\
  1 &= c_2+c_2 \\
  y' &= 3c_1\e^{3t}+c_2\e^t \\
  \frac{1}{3} &= 3c_1+c_2 \\
  c_1 &= -\frac{1}{3} \quad c_2 = \frac{4}{3} \\
  y &= -\frac{1}{3}\e^{3t}+\frac{4}{3}\e^t
\end{align*}

\subsubsection*{Exercise 17}
Solve the given initial value problem.
\begin{align*}
  y''-6y'+9y &= 0 \\
  r^2-6r+9 &= (r-3)^2 = 0 \\
  r &= 3 \\
  y &= \e^{3t} \\
  y &= c_1\e^{3t}+c_2t\e^{3t}
\end{align*}
Using the initial values of \( y(0) = 2 \) and \( y'(0) = \frac{25}{3} \):
\begin{align*}
  y &= c_1\e^{3t}+c_2t\e^{3t} \\
  2 &= c_1 \\
  y' &= 3c_1\e^{3t}+c_2e^{3t}(3t+1) \\
  \frac{25}{3} &= 3c_1+c_2 \\
  c_1 &= 2 \quad c_2 = \frac{7}{3} \\
  y &= 2\e^{3t}+\frac{7}{3}t\e^{3t}
\end{align*}

\subsubsection*{Exercise 18}
Solve the given initial value problem.
\begin{align*}
  z''-2z'-2z &= 0 \\
  r^2-2r-2 &= 0 \\
  r &= \frac{2\pm\sqrt{4+8}}{2} = \frac{2\pm2\sqrt{3}}{2} = 1\pm\sqrt{3} \\
  z &= c_1\e^{t(1+\sqrt{3})}+c_2\e^{t(1-\sqrt{3})}
\end{align*}
Using the initial values of \( z(0) = 0 \) and \( z'(0) = 3 \):
\begin{align*}
  z &= c_1\e^{t(1+\sqrt{3})}+c_2\e^{t(1-\sqrt{3})} \\
  0 &= c_1+c_2 \\
  z' &= (1+\sqrt{3})c_1\e^{t(1+\sqrt{3})}+(1-\sqrt{3})\e^{t(1-\sqrt{3})} \\
  3 &= (1+\sqrt{3})c_1+(1-\sqrt{3})c_2 \\
  c_1 &= \frac{\sqrt{3}}{2} \quad c_2 = -\frac{\sqrt{3}}{2} \\
  z &= \frac{\sqrt{3}}{2}\e^{t(1+\sqrt{3})}-\frac{\sqrt{3}}{2}\e^{t(1-\sqrt{3})}
\end{align*}

\subsubsection*{Exercise 19}
Solve the given initial value problem.
\begin{align*}
  y''+2y'+y &= 0 \\
  r^2+2r+1 &= (r+1)^2 = 0 \\
  r &= -1 \\
  y &= c_1\e^{-t}+c_2t\e^{-t}
\end{align*}
Using the initial values of \( y(0) = 1 \) and \( y'(0) = -3 \):
\begin{align*}
  y &= c_1\e^{-t}+c_2t\e^{-t} \\
  1 &= c_1 \\
  y' &= -c_1\e^{-t}-c_2e^{-t}(t-1) \\
  -3 &= -c_1+c_2 \\
  c_1 &= 1 \quad c_2 = -2 \\
  y &= \e^{-t}-2t\e^{-t}
\end{align*}

\subsubsection*{Exercise 29}
Use Definition 1 to determine whether the functions \( y_1 \) and \( y_2 \) are
linearly dependent on the interval \( (0,1) \):
\[ y_1(t) = t\e^{2t} \quad y_2(t) = \e^{2t} \]
\( y_1 \) and \( y_2 \) are linearly independent because they are not
\textit{constant multiples} of one another.

\subsubsection*{Exercise 30}
Use Definition 1 to determine whether the functions \( y_1 \) and \( y_2 \) are
linearly dependent on the interval \( (0,1) \):
\[ y_1(t) = t^2\cos(\ln(t)) \quad y_2(t) = t^2\sin(\ln(t)) \]
\( y_1 \) and \( y_2 \) are linearly independent on the interval \( (0,1) \)
because they are not \textit{constant multiples} of one another. They are
only constant multiples when \( t = 0 \).

\subsubsection*{Exercise 32}
Use Definition 1 to determine whether the functions \( y_1 \) and \( y_2 \) are
linearly dependent on the interval \( (0,1) \):
\[ y_1(t) = 0 \quad y_2(t) = \e^t \]
Since \( y_1(t) = 0 \), it's only possible multiple is 0, therefore the two
functions are linearly independent.

\begin{center}
  If you have any questions, comments, or concerns, please contact me at
  alvin@omgimanerd.tech
\end{center}

\end{document}
