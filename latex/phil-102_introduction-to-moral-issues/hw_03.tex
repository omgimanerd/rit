\documentclass{article}

\usepackage{enumerate}
\usepackage{soul}

\title{Introduction to Moral Issues: Homework 3}
\author{Alvin Lin}
\date{January 2018 - May 2018}

\begin{document}

\maketitle

\section*{MLK on activism}
\begin{enumerate}
  \item What is King's reply to those who criticize his presence in Birmingham and call him an outside agitator?
  \par Martin Luther King cites his organization ties and the presence of injustice as his reason for being in Birmingham. He likens his mission to that of the Christian prophets when they traveled to spread their gospel. King mentions that the effects of racial injustice in Birmingham affect everyone, not just those in Birmingham. He makes the comparision that no one who lives inside the United States can be considered an outsider because they are involved and affected by its issues.
  \item What is King's reply to those who suggest that he should prefer negotiation instead of direct action (for example, organizing marches)?
  \par Martin Luther King replies that negotiation is unproductive because the promises made during negotiation are quickly broken. He cites the example of negotiating for merchants who promised to remove humiliating racial signs, but quickly reinstated them weeks after the negotiation. Since the promises from negotiation are not kept, King states that direct action is the only alternative. He also argues that the purpose of direct action is to call for meaningful negotiation by dramatizing the issue so that the other party is forced to confront it. Direct action is needed so that the results of negotiating are fulfilled instead of being ignored.
  \item What is King's reply to those who argue that it is best to wait for reforms instead of fighting for civil rights now?
  \par In the past, many demonstrations were postponed for the sake of elections and other such events so that they could not be politicized and used to cloud the argument. The community had to endure endless delays for the sake of waiting for reforms. Since nothing was getting done, fighting through direct action became the only course of action to work towards their goal.
  \par King also notes that the people in power are still segregationists and will not produce change voluntarily. Fighting for civil rights through direction action puts pressure on those in power to create change. He argues that the oppressed have suffered long enough and that ``justice too long delayed is justice denied''.
  \par In a later section, King argues that waiting accomplishes nothing because time itself is a neutral force that does nothing to solve the problem of injustice. Change and human progress are only accomplished through the sweat and blood of people who make the effort to produce change.
  \item What is King's reply to those who argue that he contradicts himself when he advocates breaking some laws and obeying others?
  \par King makes the distinction that there are just law and unjust laws. While it is one's moral responsibility to obey just laws, one also has a responsiblity to disobey unjust laws. King uses the law of God (which has its own shaky foundations) as the basis for just law, and juxtaposes unjust law as anything opposing it. He argues that segregation is unjust because it ``distorts the soul and damages the personality'' by giving the oppressor a false sense of superiority while giving the oppressed a false sense of inferiority. Through this logic, he argues that segregation ordinances must be disobeyed because of their unjust nature.
  \par One of Martin Luther King's most important distinctions is that his civil disobedience is performed ``openly, lovingly, and with a willingness to accept the penalty''. His defiance of the law is done with the purpose of raising awareness of its injustice and not the spread of anarchy.
  \item What is the difference between just laws and unust laws \ul{in terms of the way they are applied to minority groups and majority groups}?
  \par King expands his argument further to saw that many of the discriminatory laws are themselves contradictory because they are not applied and enforced ubiquitously as laws should be. They are constructed by those in power to deny rights to the minority, and are enforced on a racial basis. He makes the case that they are not democratically supported because the majority of African Americans have been systematically denied the right to vote, even though they sometimes constitute a majority of the population.
  \par King cites the example of his arrest for parading with a permit during his demonstrations. While there is nothing wrong with the existence of that law for the purpose of societal order, it becomes unjust when the law selectively enforced to deny his First Amendment privilege of peaceful assembly and protest.
  \item What does King mean by ``white moderate''?
  \par King notes that the white moderate oppose his demonstrations because of their fear of change. They do not want justice if it requires upsetting social order, and prefer peace even if it comes with some injustice. Though King does not explicitly say it, he notes that the white moderate cannot understand the struggle he is facing since they have never experienced the oppression that he has, and thus do not want to forsake their quality of life. He argues that law and order are good in a just society, but their enforcement now is a roadblock for social progress and the pursuit of equality. He compares his demonstrations to the opening of a boil by saying that it may be temporarily discomforting, but it exposes the ugliness of injustice and cures it.
  \item What is King's reply to the argument that even though direction action might be peaceful, it must be condemned because it precipitates violence?
  \par King's argument is that his actions precipitate violence against him and his demonstrators and frames the accusation as a form of victim blaming. Since his actions themself are nonviolent, the only forms of violence that can arise are those perpetrated against him. He argues that this condemnation makes no sense because it blames him for the actions of others and the violence enacted by other people. Any evil that arises as a result of his peaceful demonstration is the conscious result of another person's action.
  \item Part of King's reply to the argument that he is an extremist is that he actually stands between two extremes or ``forces''. Briefly describe the two extremes or options he is against.
  \par King describes the two extremes as complacency and hatred. One on end, African Americans living in oppression have become so drained and tired that they have adjusted to segregation and simply live with it. Some have gotten lucky and achieved middle class status with some economic security and are insensitive to the problems caused by segregation because they benefit from it in some ways. On the other end, discrimination and segregation have instilled a feeling of bitterness and hatred, causing the rise of black nationalist groups who wanted to fight back with violence. Both are in strict opposition to Martin Luther King's doctrine, since he wants to push back through direct action in the form peaceful protest and demonstration.
  \item Another part of King's reply to the charge of extremism is that his accusers are asking the wrong question. What is the question he thinks we should ask about extremism?
  \par Instead of rejecting the label of extremist, Martin Luther King accepts the label and instead reframes the question as a matter of whether or not he is an extremist for hate or for love. He compares his actions to those of Amos, the apostle Paul, and Jesus Christ, nothing that all three were peaceful extremists in pursuit of justice and love. King concludes this argument by saying that the South is in ``dire need of creative extremists''.
\end{enumerate}

\section*{Narratives on race}
\begin{enumerate}
  \item From the section on ``Blaming the black victom'': Choose any point or example that alled your attention and write a brief comment about it. For example, explain why you found it interesting, or surprising, or outrageous, or weak, or strong, etc.
  \par One of the more nonsensical defenses of the actions of Officer Wilson is that ``Michael Brown is dead because he had failing parents, who were not together and raised him in the right way''. In many conservative news outlets, the nature of criminals is often blamed on ``failing parents'', or ``parents who were not together'', which is a egregious non sequitur. It further places negative connotations and stereotypes on single parents, many of which work hard and can are successful in raising children. The misconception that single or separated parents are a negative force is heavily rooted in evangelical Christian ideology and is flawed for many reasons, particularly since the intention of these news outlets is to imply that only traditional nuclear families with a man and a woman can be successful.
  \item From the section on ``Blaming the black community'': Choose any point or example that called your attention and write a brief comment about it.
  \par In this section, Laura Ingraham makes the comment that ``the non-minority working class is like 'wait a second, I'm just trying to get by every day okay' '' in order to absolve the white working class of blame during a discussion on Fox News. One of the failings of these commentaries is that they do not address the reality of privilege in society. While it is true, many working class white families ``are just trying to get by'', their path to ``getting by'' has significantly fewer potholes. This dismissal of racism fails to acknowledge many facets of the argument.
  \par She also comments that ``I think a lot of white people are sick of it and a lot of black people are sick of it'', which has an odd irony to it since she clearly does not understand what black people are sick of.
\end{enumerate}

\section*{Inner cities and segregation}
\begin{enumerate}
  \item Many European and Asian immigrants eventually moved out of the ghettos in which they originally lived. Why didn't black migrants from the South do the same?
  \par Many immigrants and former slaves moved north to urban ghettos near factories to find work. Black workers were increasingly displaced from the South due to mechanized agriculural revolutions. Other immigrant groups and workers from white ethnic ghettos eventually were able to move out as they gained wealth over time, but due to segregation laws, black people were unable to move out even if they did become wealthy. Segregation left them with far fewer options than ghetto residents of other races.
  \item Why were most black people not eligible for Social Security and unemployment insurance during the Great Depression?
  \par During the Great Depression, the Social Security policies and unemployment insurance were only for people who weren't domestics or agricultural workers. A large majority of black people, especially from the South, fell into these categories and were subsequently excluded from the benefits. This allowed many other groups in poverty to escape their situation.
  \item How did the FHA help families accumulate wealth? How did the FHA's policies end up excluding black people?
  \par The Federal Housing Administration instituted a program that guaranteed mortgages for home purchases. Families who were homeowners were more likely to accumulate wealth, create jobs, and provide community investment. However, the FHA cited most black neighborhoods as financial risks and refused to guarantee mortgages to most of their residents. These policies eventually became adopted by private lenders and are unofficially used today to exclude poor neighborhoods.
  \par Because of zoning requirements, many poor neighborhoods were labeled as industrial, and thus could not undergo residential construction or even renovation.
  \item Mention two economic factors that caused the loss of jobs that paid a living wage?
  \par Prior to World War II, many manufacturing jobs in the North were available to large numbers of African Americans, allowing many to escape poverty. As Europe and Japan rebuilt during the aftermath of World War II, they began to industrialize and compete with American companies. As a result, many companies chose to globalize and move their factories overseas to Third World countries with cheap labor and no regulations.
  \par Many manufacturing jobs disappeared, and with increasing mechanization and computerization, more educated workers were needed. A higher threshold for education meant wages for jobs in the service sector and manufacturing subsequently dropped. Both of these factors caused the loss of many jobs that paid a living wage in the United States.
\end{enumerate}

\section*{Mobility}
\begin{enumerate}
  \item From Section 4.1: What was Dudley Duncan's hypothesis about how to eliminate racial inequality?
  \par Duncan's hypothesis was that if society could eliminate the negative perception of race and the resulting discrimination, racial inequality would eventually disappear by itself since it would have no basis. While some data from the 1960s and 70s support his hypothesis, recent data has found little to no improvement to racial minorities even with the expansion of civil rights.
  \item Mention one way in which Duncan's vision hasn't become true.
  \par Even with the expansion of civil rights, black children on average have significantly lower incomes as adults compared to white children who grew in similar conditions and economic statuses. The data from the studies done shows that black children growing up in poor neighborhoods tend to stay trapped in those neighborhoods from a cycle of poverty and as a result have lower incomes compared to white children from similar backgrounds.
  \item From Section 4.2: Explain the concept of ``economic mobility'' as used in the reading.
  \par Economic mobility in this reading is described as the probability that the children and parents have similar economic status. If poor children tend to stay poor, then the society they live in is characterized as having low economic mobility. In contrast, if there is little correlation between the economic status of the parents and that of the children (whether positive or negative), then the society is said to be economically mobile.
  \item From Section 4.2: Some processes (or mechanisms) of mobility are incompatible with ideas of fairness. Explain why using an example from the reading.
  \par The American Dream is based on the idea of fairness since everyone has an equal opportunity for advancement and self-improvement. High levels of inequality are therefore acceptable if everyone can move up. However, if the forces behind economic mobility are applied differently to everyone depending on race or economic status, then this conflicts with the idea of fairness despite inequality. Americans would be unhappy if the system provided more opportunity for rich Americans and less for lower income workers.
  \item From Section 4.3: Summarize the results of the Gautreaux Assisted Housing program.
  \par The Gautreaux Assisted Housing Program was a court ordered program that provided housing subsidies for low-income residents in Chicago to move out in order to desegregate neighborhoods. It was found that families moved to wealthy suburban neighborhoods fared significantly better than those families living in segregated neighborhoods in the city. Furthermore, it was found that the relocated children who grew up in the wealthy suburban neighborhoods had a higher probability of economic sucess, high school completion, and college attendance. This program demonstrated the correlation between the economic status of the neighborhood and the economic status of children who grew up in the neighborhood.
\end{enumerate}

\end{document}
