\documentclass{article}

\usepackage{enumerate}

\title{Introduction to Moral Issues: Exam}
\author{Alvin Lin}
\date{January 2018 - May 2018}

\begin{document}

\maketitle

\section*{Husak on criminalization}

\begin{enumerate}
  \item \textbf{Section I: Drugs and Health} \\
  Husak considers several arguments for the criminalization and gives a number of replies against each of those arguments. Summarize two of the arguments he discusses in that section and at least one reply he gives to each of those arguments.
  \par The first argument for drug criminalization is that they are bad for health, and thus the state has an obligation to protect us from it. Husak begins to address this by first pointing out the counterproductiveness of this method because punishment is being applied to the people that the state is trying to protect (drug users). He points out that this logic is applied to no other health risk. Companies that sell tainted food are punishable by law, but the people who eat the food cannot and should not be punished. Aside from drugs, there are many other public health risks such as unhealthy foods, and it makes no sense to treat the recreational consumption of unhealthy foods as a criminal offense. Husak points out that the intent of this law is to protect the health of drug users, but sending them to prison continues to expose them to drugs and is more hazardous to their health than the drugs themselves.
  \par A second argument for drug criminalization seeks to defend the position that criminalization acts as a deterrent against the use of drugs. Husak points out the weakness of this argument by pointing out the years of drug abuse by millions of Americans despite the existence of criminal drug penalties. Clearly, this argument is weak since it is evidently not true, but also does not account for the fact that the criminalization of illicit drugs does not prevent people from abusing legal drugs. Husak points out that the abuse of tobacco and alcohol lead to far more deaths and negative effects than the usage of drugs.
  \par (Towards the end of this section, Husak argues that if we should criminalize health risks if they are for the purpose of recreation by comparing the pursuit of professional sports to the usage of drugs. It should be noted that this is a false equivalence and a weak counterargument)
  \item \textbf{Section II: Drugs and Children} \\
  Summarize two of the arguments in this section against decriminalizing drugs and at least one reply per argument.
  \par One of the arguments against the decriminalization of drugs in this section revolves around the idea that society wants what is best for the children, and ``protecting them from the evils of drugs'' is in our best interests. Husak responds to this by asking why we stop wanting the best as soon as our children begins to use drugs. No tolerance policies, along with the criminalization of drugs, demonize drug users and seek to prosecute and sentence adolescent drug abusers. If we are concerned for our children, then why do we stop being concerned once they start using drugs? He argues that our concern for child welfare is contrary to the idea of punishing them and that criminalizing drugs is detrimental to child welfare.
  \par No matter what we do, children are bound to get their hands on illicit substances. A second argument Husak addresses is the idea that the criminalization of drugs prevents them from leaking from adults to children. Husak uses this to point out the decriminalization is only with regard to the demand side of the equation in that only drug usage need be decriminalized while drug production and distribution can still be treated as criminal offenses. He points out the statistics of drug acquisition, noting that over 56\% of teenagers report drugs as being ``easy to obtain'', debunking the effectiveness of drug criminalization in keeping them away from children.
  \item \textbf{From Section III: Drugs and Crime} \\
  What is systemic crime? What is Husak's argument against the view that criminalization decreases systemic crime?
  \par Systemic crime is crime that occurs because the criminalization of drugs forces them to be bought and sold in the black market. Competition and disputes between drug buyers and sellers who participate in black market sales cannot be resolved in a court of law, so they are usually resolved through violence.
  \par Husak argues that criminalization increases systemic crime because it creates a vicious cycle in which black market drug disputes lead to a call for stricter enforcement of drug criminalization laws, thereby making it more profitable to sell drugs on the black market and creating more systemic crime. Most drug-related crimes are systemic in nature, so decriminalization would drastically reduce the rate of drug-related crimes.
  \item \textbf{From Section III: Drugs and Crime} \\
  What is economic crime? Why does Husak believe that legalization might reduce economic crime?
  \par Economic crime is a type of crime committed by addicts who need the money to purchase illegal drugs. Due to their addiction and the high price of illicit drugs, many drug abusers will resort to various forms of economic crime in order to satisfy their addiction.
  \par The high price of drugs is partly due to the high demand and low supply created by its illegal status. Husak argues that the legalization of drugs could reduce the price of drugs like heroin to 2\% of what it is sold for on the black market. This would reduce the amount of economic crime if drugs were cheaper. He acknowledges that cheaper drugs which are easier to acquire could tempt more people into usage, but points to the fact that alcoholics and tobacco addicts rarely have to steal to satisfy their addictions.
  \item \textbf{From Section III: Drugs and Crime} \\
  What is psycho-pharmacological crime? Summarize Husak's argument about this type of crime and the effects of marijuana, heroine, cocaine, and alcohol?
  \par Psycho-pharmacological crime is any crime that occurs due to the influence of a drug on a person’s inhibitions or mental state. Due to the fact that many drugs are taken for their mind-altering properties, proponents for criminalization aruge that this leads to more crime since people act in ``chilling and horrible ways'' while under the influence of drugs.
  \par Husak argues that no evidence supports this hypothesis that people under the influence of marijuana or heroin are more likely to become aggressive and violent. In fact, marijuana is used for its properties to pacify and relax the user. He points out the glaring hole in this argument in that alcohol is far less regulated and has no criminal status, but is far more likely to encourage violent and aggressive behavior. In terms of psychopharmacological crime, alcohol is a far more obvious culprit.
  \item \textbf{From Section IV: Drugs and Immorality} \\
  Choose any argument or statement in this section that called your attention and write a brief assessment of it. For example, explain why you found it relevant (or irrelevant), or weak (or string), convincing (or incomplete) etc. Explain how the author uses the definition you are discussing and explain if you find those definitions useful.
  \par In this section, Husak dissects the argument that drugs are immoral by separating it into its two logical parts: one, the criminal law should punish people who behave immorally, and two, illicit drug use for recreational purposes is immoral. Though it is only necessary to disprove one in order to invalidate the syllogism, Husak argues that both parts of this argument are invalid.
  \par He points out that the morality of legal and illegal drug use has little difference if the argument for the immorality of cocaine is that it ``alters one’s soul''. One particularly interesting point of argument that Husak makes is the selective nature of legal moralism. We punish drug use as immoral, but fail to acknowledge the use of alcohol, caffine, tobacco, or other immoral behaviors such as lying.
  \par Husak notes that any arguments made using morality heavily resemble an argument on religious grounds. Proponents for criminalization should note that once religion is brought into the argument, any hypotheses and claims made will become highly fallacious and untenable. The capricious, interpretive, and indoctrinative nature of religion makes it a poor tool for decision making or justification with regard to drug policy (or any policy as a matter of fact).
\end{enumerate}

\end{document}
