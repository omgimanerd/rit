\documentclass{article}

\usepackage{enumerate}

\title{Introduction to Moral Issues: Homework 1}
\author{Alvin Lin}
\date{January 2018 - May 2018}

\begin{document}

\maketitle

\section*{Relativism}
\begin{enumerate}[a)]
  \item Summarize two negative consequences of Cultural Relativism.
  \par Cultural Relativism allows for societal standards and traditions to dictate what is morally right or wrong. Without an extrinsic standard, societal practices are absolved of wrongdoing due to the fact that their society's standards are different than our own. Cultural Relativism argues against a universal metric by which actions can be compared.
  \par Additionally, the lack of a standard removes any need for moral progress in any form. It becomes impossible to judge any change as improvement because no change can be judged as better or worse without a standard that applies across all time periods. One would not be able to judge the barbaric ways of the past because of the reasoning that ``it was a different time''.
  \item Why does Rachels think that there is less disagreement among cultures than it seems? What distinction does he use to explain the apparent differences?
  \par Rachels argues that fundamentally, all cultures agree share similar beliefs about what is morally right or wrong. However, the environment and many other physical factors influence the customs and practices of that culture, which may lead to interpretations and belief systems that execute the values in ways that seem contrary. Rachels uses the argument that some cultures will not eat cows because they believe the souls of humans inhabit the bodies of animals after death. While we may share the fundamental value that we shouldn't eat another human being, the difference is in the religious beliefs of that culture which state that human beings may inhabit the bodies of cows. The disagreement is merely over the fact that the cow may or may not be another human being.
  \item Mention two values or principles that all cultures must have in common according to Rachels. Why do we need those values and principles?
  \par Rachels posits that all cultures must share the value of caring for the young, and also that all societies should discourage murder. Some principles are necessary in societies and thus all societies share them because some principles are necessary for a cooperative society to function. A group that does not care for its young has no means of maintaining its population, and thus would die out eventually. A society that did not discourage murder would also discourage cooperation. Any sort of large scale human effort would disintegrate since every human would avoid the others out of a desire to survive. Even if subgroups or subsocieties formed, they would have to indirectly acknowledge that murder was detrimental to their progress.
\end{enumerate}

\section*{Consequentialism}
\begin{enumerate}[a)]
  \setcounter{enumi}{3}
  \item What is the main idea of utilitarianism?
  \par Utilitarianism is the idea that actions should be judged purely on their consequences without regard to any absolute standard of right or wrong. In order to judge an action, the amount of ``utility'' it yielded was used as a metric to compare different consequences of different actions. Consequences which resulted in higher utility were judged to be better.
  \item How does classical economics define ``happiness''?
  \par Since happiness was the metric by which consequences were judged, classical economics defined happiness in the context of utilitarianism as the ``measure of satisfaction of [one's] desires''. The amount of utility one received upon satisfying a desire was directly related to the strength of the desire.
  \item What is the consequentialist approach to principles such as ``You should not lie''?
  \par Consequentialists would evaluate this according to the results of whatever happened from lying. Without an absolute standard, lying would be right in some scenarios and wrong in others if the outcome resulted in a greater net gain in human happiness. Absolutists might agree that some scenarios it would be ``just'' to lie, but consequentialists would say that one could lie without regret and it would be the right action to take if it yielded greater utility.
  \item What are negative and positive rights?
  \par Positive rights are rights that oblige other entities to help a person who chooses to pursue some action. They give someone the ability and assistance to pursue education, life, happiness, etc. Negative rights take away from a person's ability to hinder someone else's rights. Someone who has the right to do something also has the right not to be obstructed by another individual.
  \item What is one of the main challenges or problems for rights theorists?
  \par Rights impart duties upon an individual in order for that right to be upheld. These duties may conflict with other rights in extreme situations such as war. Both sides have a right to life and have a duty to protect their own lives, but must abridge that right from a soldier on the opposing side.
\end{enumerate}

\section*{Social Contract}
\begin{enumerate}[a)]
  \setcounter{enumi}{8}
  \item What is Binmore's definition of ``social contract''?
  \par Binmore defines a social contract as any stable social convention involving the coordination of behavior. He quantifies interactions that are stable, efficient, and fair as part of valid social contracts. Social contracts exist because its participants have no other better alternative than to cooperate, thus making it self-regulatory.
  \item Explain why the notion of ``willing consent'' needs to be established more closely.
  \par Binmore argues that social contracts exist and are self-enforcing because its participants have no better alternative and thus willingly consent to be a part of it. However, there are many societal forces and nuanced issues that may encourage an individual to participate in the social contract, past the point at which it can be considered willing consent. These may include threats, lack of reasonable alternatives, lack of resources, or misinformation, which all can lead to an individual participating in a social contract ``willingly''.
  \item What are the main goals of what Corning calls a ``biosocial contract''?
  \par Corning focuses more on the idea of fairness and its role in an altruistic human society in a biosocial contract. Embedded in his biosocial contract is the core idea that a human society exists to improve survival and provide for one another. This is his first precept, best summarized as the concept of equality, where goods and services should be distributed equally and satisfy society's basic needs. However, Corning's biosocial contract theory also includes the idea of fairness by distributing surplus resources according to the meritability of the individual and the societal contribution of the individual. Known as reciprocity, participants on his biosocial contract are obligated to provide for the survival of the society if they are able to.
\end{enumerate}

\section*{Virtue Ethics}
\begin{enumerate}[a)]
  \setcounter{enumi}{11}
  \item Does Aristotle argue that a good-tempered person should never get angry? Explain why or why not.
  \par Aristotle argues that a good-tempered person should not be led by passion. They should be angry at the proper things, in the right way, and for the right amount of time. A person who is angry too often exists in an extreme of ``irascibility'', while a person who is angry too little becomes ``inirascible''. Aristotle argues that both extremes are bad since they lead to vengeful retaliation or passive indifference.
  \item McDonald argues that there is a ``healthy, objective way'' of finding a balance between caring too much and being indifferent. What is his argument for this claim? In other words, what are the pros and cons of each extreme?
  \par McDonald's argument is that living too much on either extreme takes away from one's experiences. Caring too much about others can cause one to get lost in the emotions and struggles of others, draining time, resources, and vitality. While this may seem selfless and noble, it is exhausting and the quality of care will inevitably decrease due to it. In contrast, caring too little allows one to focus on oneself, but strips away the experience of helping others in a meaningful way. It leads to an apathetic emptiness in one's life without the opportunity to help others.
  \item McDonald gives three general pieces of advice: (1) Put yourself first; (2) Give others what they need, not what they want; (3) Question everything. Choose any of these and write a brief critique of it. You don’t need to reject the advice as a whole; you may identify parts of his advice that need to be articulated more precisely to avoid negative implications or simplistic solutions.
  \par While McDonald's advice to ``put yourself first'' is admirable, very specific lines need to be drawn in order to prevent this advice from causing one to fall into the trap of caring too little, which is the very thing that McDonald is trying to avoid. One specific bit of advice that McDonald gives is that putting yourself first ``not giving a shit about the old friends that don't serve you''. While his intent may have been different, this piece of advice has a cutthroat intent that implies one should not take relationships or friendships that don't serve an end goal or don't benefit oneself in some way. This falls dangerously close to the edge of caring too little. It is important to draw a line between helping others in a meaningful way and helping others to help oneself. This is a necessary and important criteria in determining when it is appropriate to put oneself first and when to put others first.
\end{enumerate}

\end{document}
