\documentclass{article}

\usepackage{enumerate}

\title{Introduction to Moral Issues: Homework 2}
\author{Alvin Lin}
\date{January 2018 - May 2018}

\begin{document}

\maketitle

\section*{Living Wage}
\begin{enumerate}[a)]
  \item From the concept of ``just wage'' discussed in medieval times: Explain the relationship between a just wage, subsistence, and virtue.
  \par Discussed in the context of the medieval times, a just wage is one that has been knowingly and voluntarily agreed upon through fair bargaining. At the very minimum, a just wage must be one that allows for workers to subsist and acquire basic necessities such as food and shelter. Medieval scholars recognized that low wages were related to virtue because low wage rates would increase the likelihood that workers would turn to crime to fulfill their basic needs, therefore making low wages unjust.
  \item From the contemporary accounts of the living wage: Explain why the model of ``perfect competition'' is unrealistic.
  \par The model of ``perfect competition'' begins with the assumption that workers and employers have equal power when bargaining for wages. Under this assumption, wages eventually become ``a market estimation of what a worker adds to the production of goods and services that society wants''. This conclusion is unrealistic because the premise it is based on is not true in today's world, since employers today have much more power than workers. The employer's position holds much more power since they are the ones offering the jobs and can exert their power through other ways by influencing politics and society with their wealth.
  \item From the section on capabilities: Summarize Ryan's early conceptualization of the concept of ``capability''.
  \par The concept of capability is an extension of the concept of basic needs with regard to a living wage. Ryan's idea is that a living wage should not only cover a worker's basic needs (food, clothing, shelter), but also empower them to ``develop within reasonable limits all [their] faculties''. Put simply, a living wage should allow workers to pursue personal development within reasonable limits so that they can function as a member of society and live a meaningful life. The idea of capability is analogous to the idea of self-actualization in Abraham Maslow's philosophy on the hierachy of human needs. Ryan argues that work should be a medium through which people gain and develop capability in addition to basic subsistence.
  \item From the section of externalities: Explain how in contemporary settings low wages pose a cost to society.
  \par From the point of view of externalities, employers who pay their workers low wages may be making more profit, but the cost of the labor is deferred to society since the exploitation of the workforce leads a decline in energy, character, intelligence, and productivity. Ryan also makes an argument in terms of economics: businesses who exploit workers do not pay the full cost of labor, thus producing more product at lower costs and inefficiently allocating resources. In contemporary settings, low-paid workers tend to be supported by various welfare programs, meaning employers are also imposing costs on the taxpayers in a society.
\end{enumerate}

\section*{Competition and Harm}
\begin{enumerate}[a)]
  \setcounter{enumi}{4}
  \item From Section 1: Why does Wolff think that an argument for competition based on liberty alone is not enough?
  \par As a baseline, Wolff defines liberty as the right for an individual to act as they wish provided that they do not harm others. This is applied to all other walks of life, allowing people to engage in society and go about their lives pursuing their interests as long as they do not harm others. The economic harm suffered from an abuse of the liberty to compete is as dangerous as all other forms of harm, but it is permitted under the defense of liberty. Wolff argues that using liberty to defend competition is not enough because certain forms of economic competition are used to hurt other people in a way that would be forbidden in any other context.
  \item From Section 3: What is Wolff's definition of ``exploitation''? Is it merely using another person for your own ends?
  \par Wolff's defines exploitation as the wrongful misuse of another person for one's own benefit. The distinction is made however, that using others for personal gain is permissible if the person being used is not vulnerable in some way. Generally, the exploitation of another person is only possible if they are vulnerable in some way, such as being ``poor, ignorant, dependent [...]'', etc. Additionally, Wolff requires that the act of using another person without regard for their well-being is only be classified as exploitation if the act of using them is necessary for achieving a goal.
  \item From Section 5: A competition held for the ``side-effect of activity`` is not exploitative in the case of economic competition because: The competitors voluntarily enter the competition. What is Wolff's reply to this argument?
  \par Wolff relates the argument of voluntary participation to his definition of exploitation by noting that the victims of exploitation are generally always ``willing'' because they are in a vulnerable state. The choice to participate is considered voluntary because the victim made the choice. However, with proper contextual analysis, the victims often have few other choices and no better alternative. Wolff argues that the consent of the exploited is not a sufficient argument because their consent carries little weight since they are both vulnerable and unable to make other choices.
  \item From Section 5: A competition held for the ``side-effect of activity`` is not exploitative in the case of economic competition because: The interests of the victims are taken into account to a sufficient degree. What is Wolff's reply to this argument?
  \par Wolff considers a system in which the interests of the victims are taken into account as inherently not exploitative, but considers the people who benefit from the system as exploiters. Wolff's point of view is that certain arguments in defense of exploitative systems merely defer the blame to some other party and that the victims are still being unfairly treated for their labor. While the system may not be exploitative, the workers are still being exploited and the people who stand to gain from the unfair usage of labor should be labeled as exploiters.
\end{enumerate}

\section*{Free Markets and Choice}
\begin{enumerate}[a)]
  \setcounter{enumi}{8}
  \item From Section 3: Greenfield argues that ``if markets have their say, your wage would depend on [...] how much the company would have to pay your replacement''. Briefly explain what this means.
  \par Greenfield argues that markets detract from the idea of merit based compensation because employers would end up paying as little as possible for workers. Under the direction of pure market forces, the only thing determining a worker's pay would be the amount that another worker would be willing to work for. In essence, prices and earnings are dependent on the value placed on it by other people, which limits choices to what one can afford.
  \item From Section 3: The situation of many working families in both developing countries and rich countries challenges the notion that free markets by themselves raise people's living standards. Mention two specific examples or statistics that challenge that notion.
  \par Greenfield argues that markets provide choice only if a participant has the money to afford those choices. Without money, a person cannot raise their standard of living or improve their standing. Greenfield cites the example of American workers today earning salaries comparable to salaries paid in the 1970s, which limits their choices in the market and thus prevents them from raising their standard of living. He also states the most obvious example of the billions of people living in other countries who do not benefit from the markets that reach them through globalist expansion. Even though they have access to markets, they do not have access to the choices that the market offers because they do not have the money to afford those choices.
  \item From Section 5: Mention one advantage and one disadvantage of having ``cognitive shortcuts`` or subconscious mechanisms of decision making.
  \par Cognitive shortcuts allow us to make decisions in a world with complex and often unfamiliar rules and relationships. This is an advantage that allows us to reasonably function in society by simplifying market choice to ones that we think would be best for us. However, the disadvantage is that marketers and advertisers who are conscious of this can hijack this against us by pushing us to make rash and emotionally driven decisions instead of rational ones when purchasing products. Smart producers can manufacture our desire for a product in addition to the product itself by taking advantage of the cognitive shortcuts in our subconscious. Greenfield uses this to argue that markets involve very little choice today despite their philosophy because advertisers and marketers create predispositions in their consumers to affect their market choices.
  \item From Section 6: What is the ``collective action`` problem? Explain it using an example from the reading.
  \par Markets place an emphasis on individual decision making through the mechanism of ``voting with dollars''. A consumer purchases the products that they believe best suits their needs and producers need to accommodate by making desired products to satisfy those needs. A purchase of a product can often have hidden consequences however, such as in the example mentioned by Greenfield of a Walmart superstore opening in his home town. People who ``chose'' to purchase products from Walmart inadvertently crippled the business and small stores in the town. Because of the focus on individual choice, it becomes hard for consumers to collectively object to the practices of a producer.
  \par Markets make problems of ``collective action'' difficult to handle because the only mechanism by which a participant has a voice is through their dollars. This is evident because services like clean air, infrastructure, public transportation, and access for the disabled only exist because of government regulation and intervention. The market does not facilitate these services simply because they cost money or don't make enough money.
  \item Mention two examples of things that have been commodified. What conclusion does Greenfield obtain from these examples?
  \par Greenfield notes the example of how a farmer in rural Pakistan sold his kidney for money to pay bills to demonstrate how markets lead to things like illegal organ trafficking. In addition to human organs, markets commidify anything that has value, including children, as evident by rampant prostitution rings run in Malaysia. He concludes that free markets commodify anything that has value since someone will monetize it in order to make money. The only criterion an object has with regard to its monetization is whether or not there is a demand for it and how much it is worth to other people. Obviously, things like children and human organs cannot be evaluated using only these metrics. Therefore, the law must step in where markets fall short by limiting what commodities can be bought and sold in markets.
\end{enumerate}

\section*{Poverty}
\begin{enumerate}[a)]
  \setcounter{enumi}{13}
  \item Ehrenreich admits that poverty is complex and there are many things that society could do about it. However, she argues that there are at least two things that we should do now: stop stealing from the poor and stop kicking people who are already down. Briefly explain what she means and given two examples of each.
  \par Ehrenreich argues that helping the poor is important, but as a society, we need to first do no harm. In its current state, our society has many restrictions and laws that make it difficult for poor people to get out of poverty, trapping them in and endless cycle of debt. She argues that wage theft is stealing from the poor, who often have no way of fighting it because of the expense of hiring lawyers and the lengthy process needed to do so. States that allow employer discrimination by only hiring employed candidates prevent people who need jobs the most from getting them.
  \par With regard to kicking people who are already down, many states in the US have laws and fines that further penalize people for the consequences of being poor. Employment discrimination based on poor credit is one example, as poor credit is usually an indicator of financial struggle. Adding fines on top of that is clearly nonsensical as the victim doesn't have the resources to pay the fine. The laws regarding the homeless are also examples of kicking people who are already down, since they allow the police to arrest and fine the homeless, who obviously cannot pay the fine.
\end{enumerate}

\end{document}
