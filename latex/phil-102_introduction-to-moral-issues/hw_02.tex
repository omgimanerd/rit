\documentclass{article}

\usepackage{enumerate}

\title{Introduction to Moral Issues: Homework 1}
\author{Alvin Lin}
\date{January 2018 - May 2018}

\begin{document}

\maketitle

\section*{Living Wage}
\begin{enumerate}[a)]
  \item From the concept of ``just wage'' discussed in medieval times: Explain the relationship between a just wage, subsistence, and virtue.
  \item From the contemporary accounts of the living wage: Explain why the model of ``perfect competition'' is unrealistic.
  \item From the section on capabilities: Summarize Ryan's early conceptualization of the concept of ``capability''.
  \item From the section of externalities: Explain how in contemporary settings low wages pose a cost to society.
\end{enumerate}

\section*{Competition and Harm}
\begin{enumerate}[a)]
  \setcounter{enumi}{4}
  \item From Section 1: Why does Wolff think that an argument for competition based on liberty alone is not enough?
  \item From Section 3: What is Wolff's definition of ``exploitation''? Is it merely using another person for your own ends?
  \item From Section 5: A competition held for the ``side-effect of activity`` is not exploitative in the case of economic competition because: The competitors voluntarily enter the competition. What is Wolff's reply to this argument?
  \item From Section 5: A competition held for the ``side-effect of activity`` is not exploitative in the case of economic competition because: The interests of the victims are taken into account to a sufficient degree. What is Wolff's reply to this argument?
\end{enumerate}

\section*{Free Markets and Choice}
\begin{enumerate}[a)]
  \setcounter{enumi}{8}
  \item From Section 3: Greenfield argues that ``if markets have their say, your wage would depend on [...] how much the company would have to pay your replacement''. Briefly explain what this means.
  \item From Section 3: The situation of many working families in both developing countries and rich countries challenges the notion that free markets by themselves raise people's living standards. Mention two specific examples or statistics that challenge that notion.
  \item From Section 5: Mention one advantage and one disadvantage of having ``cognitive shortcuts`` or subconscious mechanisms of decision making.
  \item From Section 6: What is the ``collective action`` problem? Explain it using an example from the reading.
  \item Mention two examples of things that have been commodified. What conclusion does Greenfield obtain from these examples?
\end{enumerate}

\section*{Poverty}
\begin{enumerate}[a)]
  \setcounter{enumi}{13}
  \item Ehrenreich admits that poverty is complex and there are many things that society could do about it. However, she argues that there are at least two things that we should do now: stop stealing from the poor and stop kicking people who are already down. Briefly explain what she means and given two examples of each.
\end{enumerate}

\end{document}
