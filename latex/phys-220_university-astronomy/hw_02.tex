\documentclass{math}

\usepackage{enumerate}

\title{University Astronomy: Homework 2}
\author{Alvin Lin}
\date{January 2019 - May 2019}

\begin{document}

\maketitle

\subsubsection*{Question 2.1}
Over the course of the year, which gets more hours of daylight, the Earth's
north pole or south pole? \par
The north pole gets more daylight over the course of the year because the south
pole is closest to the Sun during perihelion, meaning it sweeps through that
portion of the Earth's orbit around the Sun faster and spends less time in
daylight.

\subsubsection*{Question 2.2}
On 2003 August 27, Mars was in opposition as seen from the Earth. On 2005 July
14 (687 days later), Mars was seen in western quadrature as seen from the Earth.
What was the distance of Mars from the Sun on these dates, measured in
astronomical units (AU)? Is this greater than or less than the semimajor axis
length of the Martian orbit? You may assume the Earth's orbit is a perfect
circle. \par
After the 687 days (1.882 years), Mars is in the same position it was in before.
However, the Earth has completed 1.882 rotations, meaning it is
\( (2-1.882)\times360^{\circ} \) behind Mars (the Mars-Sun-Earth angle). Since
Mars is at quadrature, the Mars-Earth-Sun angle is \( 90^{\circ} \). This forms
a triangle and the Mars-Sun distance \( d \) can be described with the following
relation:
\[ d = \frac{1~AU}{\cos((2-1.882)360^{\circ})} \approx 1.36~AU \]
According to Kepler's third law, we can calculate the length of the semimajor
axis length of the Martian orbit using \( P^2 = Ka^3 \).
\begin{align*}
  P^2 &= Ka^3 \\
  (1.882~yr)^2 &= \frac{1~yr^2}{AU^{-3}}a^3 \\
  3.54~yr^2 &= \frac{1~yr^2}{AU^{-3}}a^3 \\
  a^3 &= 3.54\frac{yr^2AU^3}{yr^2} \\
  a &= \sqrt[3]{3.54~AU^3} \\
  &= 1.52~AU
\end{align*}
This is greater than the calculated Mars-Sun distance.

\subsubsection*{Question 2.3}
In the 1670s, the astronomer Ole R{\o}mer observed eclipses of the Galilean
satellite Io as it plunged through Jupiter's shadow once per orbit. He noticed
that the time between observed eclipses became shorter as Jupiter came closer to
the Earth and longer as Jupiter moved away from the. R{\o}mer calculated that
the eclipses were observed 17 minutes earlier when Jupiter was in opposition
compared to when it was close to conjunction. This was attributed by R{\o}mer to
the finite speed of light. From R{\o}mer's data, compute the speed of light,
first in AU \( \text{min}^{-1} \), then in m \( \text{s}^{-1} \). \par
Let \( b \) be the Jupiter-Earth distance at conjunction and \( a \) be the
Jupiter-Earth distance at opposition and \( c \) be the speed of light.
\begin{align*}
  b &= a+2~AU \\
  a+c(17~min) &= b \\
  c &= \frac{b-a}{17~min} \\
  &= \frac{2~AU}{17~min} \\
  &= 0.118\frac{AU}{min} \\
  &= 0.118\frac{AU}{min}\times\frac{1~min}{60~s}\times
    \frac{149597780.7~km}{1~AU}\times
    \frac{1000~m}{1~km} \\
  &= 294209145.71\frac{m}{s}
\end{align*}

\subsubsection*{Question 2.4}
In addition to aberration of starlight due to the Earth's orbital motion around
the Sun, there should also be diurnal aberration due to the Earth's rotation.
Where on the Earth is this effect the largest, and what is the amplitude? \par
This effect would be largest at the Earth's equator, and would result in an
angular correction of
\[ \theta \approx \frac{v}{c} \approx
  \frac{0.465~km~s^{-1}}{3\times10^5~km~s^{-1}} \approx 0.23'' \]

\subsubsection*{Question 2.5}
A light-year is defined as the distance traveled by light in a vacuum during one
tropical year. How many light-years are in a parsec?
\[ 1~parsec\times\frac{206,205~AU}{1~parsec}\times
  \frac{1~light-year}{63241.1~AU} = 3.26~light-years \]

\subsubsection*{Question 2.6}
The planets of the solar system all orbit the Sun in the same sense;
counterclockwise as seen from above the Earth's north pole. Imagine a
``wrong-way'' planet orbiting the Sun in the opposite (clockwise) sense, on an
orbit of semimajor axis length \( a = 1.3~AU \). What would the sidereal period
of this planet be? What would its synodic period be as seen from the Earth?
What would its synodic period be as seen from Mars?
\begin{align*}
  P^2 &= Ka^3 \\
  &= 1\frac{yr^2}{AU^{-3}}(1.3~AU)^3 \\
  P &= \sqrt{1.3^3~yr^2} \\
  &= 1.3^{\frac{3}{2}}~yr \\
  &= 1.482~yr
\end{align*}
The sidereal period of this planet would be 1.482 years. This planet would
be considered a superior planet to Earth, reversing the angular velocity
equation.
\begin{align*}
  \frac{1}{P_p} &= \frac{1}{P_E}+\frac{1}{P_{syn}} \\
  P_{syn} &= \left[\frac{1}{P_p}+\frac{1}{P_E}\right]^{-1} \\
  &= \left[\frac{1}{1.482~yr}+\frac{1}{1~yr}\right]^{-1} \\
  &= 0.597~yr
\end{align*}
Relative to Earth, this planet would have a synodic period of 0.597 years.
Relative to Mars, this would be an inferior planet.
\begin{align*}
  \frac{1}{P_p} &= \frac{1}{P_{Mars}}+\frac{1}{P_{syn}} \\
  P_{syn} &= \left[\frac{1}{P_{Mars}}+\frac{1}{P_p}\right]^{-1} \\
  &= \left[\frac{1}{1.882~yr}+\frac{1}{1.482~yr}\right]^{-1} \\
  &= 0.829~yr
\end{align*}
As observed from Mars, this planet would have a synodic period of 0.829 years.

\subsubsection*{Question 2.7}
Consider a football thrown directly northward at a latitude \( 40^{\circ} \)N.
The distance of the quarterback from the receiver is 20 yards (18.5m), and the
speed of the thrown ball is 25m \( s^{-1} \). Does the Coriolis effect deflect
the ball to the right or to the left? By what amount (in meters) is the ball
deflected? Does the receiver need to worry about correcting for the deflection,
or should he be more worried about being nailed by the free safety? \par
The ball will be deflected to the right. It will have a flight time
\[ \Delta t = \frac{18.5~m}{25~ms^{-1}} = 0.74~s \]
Given this information:
\begin{align*}
  \Delta d &\approx \frac{1}{2}vw(\delta t)^2 \\
  &\approx \frac{1}{2}\frac{25~m}{s}(\frac{1}{14000~s})(0.74~s)^2 \\
  &\approx 0.000489~m \\
  &\approx 4.89\times10^{-4}~m
\end{align*}
The amount of deflection is negligible to the receiver.

\begin{center}
  You can find all my notes at \url{http://omgimanerd.tech/notes}. If you have
  any questions, comments, or concerns, please contact me at
  alvin@omgimanerd.tech
\end{center}

\end{document}
