\documentclass{math}

\usepackage{enumerate}

\title{University Astronomy: Homework 9}
\author{Alvin Lin}
\date{January 2019 - May 2019}

\begin{document}

\maketitle

\subsubsection*{Question 15.1}
What is the rate (in kilograms per second) at which the Sun is currently
converting hydrogen to helium?
\begin{align*}
  \Delta m &= 4m_H-m_{He} \\
  &= 4.83\times10^{-29}kg \\
  E &= mc^2 \\
  &= 4.347\times10^{-12}J \\
  n_{reactions/s} &= \frac{L_{\odot}}{E} \\
  &= 8.83\times10^{37}\frac{reactions}{s} \\
  \delta M &= mn_{reactions/s} \\
  &= 4.27\times10^9\frac{kg}{s}
\end{align*}

\subsubsection*{Question 15.2}
How much energy, in MeV, is produced per proton in the PP chain?
\begin{align*}
  \Delta m &= 4m_p-(m_{He}+2m_{e^-}) \\
  &= 4.26\times10^{-29}kg \\
  E &= mc^2 \\
  &= 3.83\times10^{-12}J = 23.92 MeV \\
  \frac{23.92 MeV}{4p} &= 6.98\frac{MeV}{p}
\end{align*}

\subsubsection*{Question 15.3}
Approximately half the original hydrogen in the Sun's core has now been
converted to helium. Compute the mean molecular mass \( \mu \)
\begin{enumerate}
  \item at the surface of the sun, given standard abundances (\( X_{\odot} =
    0.734, Y_{\odot} = 0.250, Z_{\odot} = 0.016 \))
  \begin{align*}
    \mu &= (2X+\frac{3}{4}Y+\frac{1}{2}Z)^{-1} \\
    &= 0.6011
  \end{align*}
  \item at the center of the Sun
  \begin{align*}
    \mu &= (2X+\frac{3}{4}Y+\frac{1}{2}Z)^{-1} \\
    X &= \frac{X_{\odot}}{2} \quad Y = Y_{\odot}+\frac{X_{\odot}}{2} \quad
      Z = Z_{\odot} \\
    \mu &= 0.825
  \end{align*}
\end{enumerate}

\subsubsection*{Question 15.4}
If a star has \( M = 100M_{\odot} \) and \( L = 10^6L_{\odot} \), how long can
it shine at that luminosity if it started as pure hydrogen and is able to
convert all its H to He? If a star has \( M = 0.5M_{\odot} \) and
\( L = 0.1L_{\odot} \), how long can it shine under the same conditions?
\begin{align*}
  n_H &= \frac{100M_{\odot}}{m_p} \\
  &= 3.055\times10^{55} \\
  E &= n_H\Delta E \\
  &= 1.2526\times10^{47}J \\
  t &= \frac{E}{L} \\
  &= 3.2628\times10^{14}s \approx 10.3MYr
\end{align*}
A star with \( M = 100M_{\odot} \) and \( L = 10^6L_{\odot} \) can shine for
approximately 10 million years.
\begin{align*}
  E &= \frac{0.5M_{\odot}}{m_p}\Delta E \\
  &= 6.2640\times10^{44}J \\
  t &= \frac{E}{L} \\
  &= 5.17\times10^{-11}\text{ years}
\end{align*}

\subsubsection*{Question 15.5}
Consider the Sun to be a sphere of uniform density that derives its luminosity
from steady contraction. What fractional decrease in the Sun's radius,
\( \frac{\delta R}{R} \), would be required over historical times (say, the
last 6000 years) to account for the Sun's constant luminosity over that period
of time?

\subsubsection*{Question 15.6}
Suppose that the Sun is 100\% carbon (coal, for instance) and that burning this
can extract 3 eV per carbon nucleus. How long, assuming an inexhaustible
supply of oxygen from outside, could burning carbon maintain the Sun's current
luminosity?
\begin{align*}
  E &= \frac{M_{\odot}}{12m_p}(3 eV) \\
  &= 4.8948\times10^{37}J \\
  t &= \frac{E}{L} \\
  &= 1.2750\times10^{-11}s = 4043.1\text{ years}
\end{align*}

\begin{center}
  You can find all my notes at \url{http://omgimanerd.tech/notes}. If you have
  any questions, comments, or concerns, please contact me at
  alvin@omgimanerd.tech
\end{center}

\end{document}
