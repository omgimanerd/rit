\documentclass{math}

\usepackage{enumerate}

\title{University Astronomy: Homework 3}
\author{Alvin Lin}
\date{January 2019 - May 2019}

\begin{document}

\maketitle

\subsubsection*{Question 6.1}
Assume that your vision is diffraction limited at \( \lambda = 5000\angstrom \)
and that the diameter of the pupil of your eye is \( D = 8 mm \). What angular
resolution can you achieve with your unaided eye? How does this compare with the
maximum angular size of Venus and Jupiter as seen from Earth?
\[ \theta = 1.22\frac{5000\angstrom}{8 mm}\times\frac{1000mm}{1m}\times
  \frac{1 m}{10^{10}\angstrom}\times\frac{360^{\circ}}{2\pi~rad}\times
  \frac{3600''}{1^{\circ}} = 15.73'' \]
The maximum angular size of Jupiter is 50.1'' and the maximum angular size of
Venus is 1'6'', larger than our minimum achieved angular resolution. We would
be able to perceive both Venus and Jupiter with just our eye.

\subsubsection*{Question 6.2}
\begin{enumerate}[(a)]
  \item The Hiltner Telescope at the MDM Observatory (on Kitt Peak, Arizona)
    has an aperture \( D = 2.4m \). Its Cassegrain focus has an f-number
    \( f/7 \). What is the focal length \( F \) and plate scale \( s \).
    \begin{align*}
      \frac{F}{2.4} &= 7 \\
      F &= 16.8m \\
      s &= \frac{206.265}{16.8m} = 12.28\frac{arcsec}{mm}
    \end{align*}
  \item The Mayall Telescope at the Kitt Peak National Observatory (also on
    Kitt Peak) has an aperture \( D = 4.0m \). Its prime focus has an f-number
    \( f/2.7 \), its Cassegrain focus has \( f/8 \), and its Coud\'{e} focus
    has \( f/160 \). What is the focal length and plate scale for each of these
    three foci?
    \begin{align*}
      F_1 &= Df_1 = (4)(2.7) = 10.8m \\
      s_1 &= \frac{206.265}{F} = \frac{206.265}{10.8} = 19.1\frac{arcsec}{mm} \\
      F_2 &= Df_2 = (4)(8) = 32m \\
      s_2 &= \frac{206.265}{F} = \frac{206.265}{32} = 6.4\frac{arcsec}{mm} \\
      F_3 &= Df_3 = (4)(160) = 320m \\
      s_3 &= \frac{206.265}{F} = \frac{206.265}{320} = 0.64\frac{arcsec}{mm}
    \end{align*}
  \item The Keck Telescope (on Mauna Kea, Hawaii) has an aperture
    \( D = 10.0m \). Its Cassegrain focus has \( f/15 \). What is the focal
    length and plate scale?
    \begin{align*}
      F &= Df = (10)(15) = 150m \\
      s &= \frac{206.265}{F} = \frac{206.265}{150} = 1.38\frac{arcsec}{mm}
    \end{align*}
\end{enumerate}

\subsubsection*{Question 6.3}
With the \( D = 2.4m \) telescope at the MDM Observatory, I can obtain a
spectrum of a particular star with signal-to-noise ratio \( S/N = 100 \) in
\( t = 20 \) minutes when the atmospheric seeing is average \( \theta = 1'' \).
How long would it take me to obtain the same data with the Keck Telescope
(\( D = 10.0m \)) with excellent seeing (\( \theta = 0.4'' \))? \par
Since the aperture size is \( 4.17 \) times larger and the seeing is \( 2.5 \)
times better, the time would decrease proportionally so. It would take us
\( \frac{20}{4.17\times2.5} = 1.92 \) minutes to obtain the same data.

\subsubsection*{Question 6.4}
A charge-coupled device (CCD) detector is mounted at the focus of an \( f/7 \)
reflecting telescope with \( D = 50cm \) mirror. The CCD chip contains
\( 1024\times1024 \) pixels, with each square pixel being \( 10\mu m \) on a
side.
\begin{enumerate}[(a)]
  \item What is the area (in square arcseconds) of the sky that is imaged on
    a single pixel?
  \begin{align*}
    s &= \frac{206.265}{7\times0.5} = 58.93\frac{arcsec}{mm} \\
    w &= \frac{\sqrt{a}}{58.93} = 0.01mm \\
    a &\approx (0.01\times58.93)^2 = 0.347\text{ square arcseconds}
  \end{align*}
  \item What is the area (in square arcminutes) of the sky that is imaged on
    the entire chip? Would the image of the full Moon fit into the chip?
  \begin{align*}
    a &\approx 1024^2\times0.347\times\left(\frac{1'}{60''}\right)^2 \\
    &= 101.15\text{ square arcminutes}
  \end{align*}
  The maximum size of the Moon is 34 arcminutes, so the entire image would
  fit onto the chip.
  \item How many separate exposures would be required to cover the entire
    celestial sphere (\( 4\pi \) steradians)?
  \[ \frac{4\pi}{101.15\text{ square arcminutes}}\times
    (\frac{180^{\circ}}{\pi})^2\times(\frac{60'}{1^{\circ}})^2 \approx
    1468966.8 \]
\end{enumerate}

\subsubsection*{Question 6.5}
Suppose that you want to see stars that are as faint as possible in the
background limited case. The Astronomy Fairy gives you a choice: \textit{either}
she can increase the quantum efficiency of your retina from \( q = 0.1 \) to
\( q = 1 \), \textit{or} she can double the maximum pupil size of your eye while
guaranteeing diffraction-limited angular resolution. Which of these choices
would produce a lower limited flux \( F_{\lambda} \)? Explain your choice.
\begin{align*}
  t &\propto \left(\frac{\theta}{F_{\lambda}D\phi_a}\right)^2
    \frac{S_{\lambda}}{\Delta\lambda\phi_tq} \\
  \sqrt{t} &\propto \frac{\theta}{F_{\lambda}D\phi_a}
    \sqrt{\frac{S_{\lambda}}{\Delta\lambda\phi_tq}} \\
  F_{\lambda} &\propto \frac{\theta}{D\phi_a}\sqrt{
    \frac{S_{\lambda}}{\Delta\lambda\phi_tq}
    \frac{1}{t}}
\end{align*}
Doubling the maximum pupil size \( D \) would result in a proportional halving
in \( F_{\lambda} \) (a factor of 0.5). Increasing the quantum efficiency from
0.1 to 1 would decrease \( F_{\lambda} \) by the square root of the change
(\( \frac{1}{\sqrt{10}} = 0.316 \)). Increasing the quantum efficiency of your
eye would produce a lower limited flux.

\subsubsection*{Question 6.6}
The Atacama Large Millimeter/Submillimeter Array (ALMA) is designed to operate
over the wavelength range \( \lambda = 0.3\to9.6mm \). It will consist of 80
independent 12m telescopes with a maximum baseline of 18km.
\begin{enumerate}[(a)]
  \item What is the highest angular resolution achievable with ALMA?
  \begin{align*}
    \theta &= 1.22\frac{\lambda}{D} \\
    &= 1.22\frac{9.6mm}{18000m}\times\frac{1m}{1000mm}\times
      \frac{180^{\circ}}{rad}\times\frac{3600''}{1^{\circ}} \\
    &= 0.421''
  \end{align*}
  \item How large would a single-dish antenna have to be to have the same
    collecting area as ALMA?
  \begin{align*}
    A_{total} &= 80\pi(\frac{12}{2})^2 = 9043.2m^2 \\
    &= \pi(\frac{D}{2})^2 \\
    D &= 2\sqrt{\frac{A_{total}}{\pi}} = 107.3m
  \end{align*}
\end{enumerate}

\subsubsection*{Question 6.7}
Prove that equation (6.18) is correct for a Poisson probability distribution.
\begin{align*}
  \langle x^2\rangle &= \sum_{x=0}^{\infty}x^2P(x,\mu) \\
  &= \sum_{x=1}^{\infty}x^2\frac{\mu^x}{x!}\e^{-\mu} \\
  &= \sum_{x=1}^{\infty}x\frac{\mu^x\e^{-\mu}}{(x-1)!} \\
  z &= z-1 \quad x = z+1 \\
  &= \sum_{z=0}^{\infty}(z+1)\frac{\mu^{z+1}\e^{-\mu}}{z!} \\
  &= \sum_{z=0}z\mu\frac{\mu^z\e^{-\mu}}{z!}+
    \sum_{z=0}^{\infty}\mu\frac{\mu^z\e^{-\mu}}{z!} \\
  &= \mu\sum_{z=0}z\frac{\mu^z\e^{-\mu}}{z!}+\mu \quad
    (\text{definition of } \langle x\rangle)\\
  \sum_{z=0}z\frac{\mu^z\e^{-\mu}}{z!} &= \mu \quad (\text{by analogy}) \\
  &= \mu(\mu)+\mu \\
  &= \mu^2+\mu
\end{align*}

\begin{center}
  You can find all my notes at \url{http://omgimanerd.tech/notes}. If you have
  any questions, comments, or concerns, please contact me at
  alvin@omgimanerd.tech
\end{center}

\end{document}
