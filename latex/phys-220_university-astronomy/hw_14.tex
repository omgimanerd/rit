\documentclass{math}

\usepackage{enumerate}

\title{University Astronomy: Homework 14}
\author{Alvin Lin}
\date{January 2019 - May 2019}

\begin{document}

\maketitle

\subsubsection*{Question 20.1}
At what distance (and at what redshift) does an object 1kpc across subtend
an angle of 1 arcsecond?
\begin{align*}
  \theta'' &= \frac{206265d}{D} \\
  1 &= \frac{206265\times1000}{D} \\
  D &= 206265000pc \\
  cz &= H_0d \\
  z &= \frac{H_0d}{c} \\
  &= 48128.5\times10^6
\end{align*}

\subsubsection*{Question 20.2}
The \( \text{Ca}_{II} \) H and K lines have rest wavelengths of \( \lambda_0 =
3968.5\textup{\AA} \) and \( 3933.6\textup{\AA} \), respectively. In the
spectrum of a galaxy in the cluster Abell 2065 (aka the Corona Borealis
Cluster), the observed wavelengths of the two lines are \( \lambda = 4255.0
\textup{\AA} \) and \( 4217.6\textup{\AA} \).
\begin{enumerate}[(a)]
  \item What is the redshift \( z \) of the galaxy?
  \begin{align*}
    z &= \frac{\Delta\lambda}{\lambda} \\
    &= \frac{4255\textup{\AA}-3968.5\textup{\AA}}{4255\textup{\AA}} \\
    &= 0.0673
  \end{align*}
  \item What is the distance to the galaxy?
  \begin{align*}
    cz &= H_0d \\
    d &= \frac{cz}{H_0} \\
    &= 288.42 Mpc
  \end{align*}
  \item What is the distance modulus of the galaxy?
  \begin{align*}
    M &= 5\log\left(\frac{d}{10}\right) \\
    &= 5\log\left(\frac{388.42\times10^6pc}{10pc}\right) \\
    &= 37.3
  \end{align*}
\end{enumerate}

\subsubsection*{Question 20.3}
Rewrite the relation for the distance modulus (equation 13.25) in terms of the
redshift \( z \) rather than the distance \( d \).
\begin{align*}
  M &= 5\log\left(\frac{d}{10}\right) \\
  &= 5\log\left(\frac{cz}{H_0}\frac{1}{10}\right) \\
  &= 5\log\left(\frac{cz}{10H_0}\right)
\end{align*}

\subsubsection*{Question 20.6}
Consider a black hole of mass \( M = 10^8M_{\odot} \). What is the maximum
distance at which its radius of influence could be resolved using the
\textit{Hubble Space Telescope} at a wavelength \( \lambda \approx 1\mu m \)?
\begin{align*}
  \log(\sigma_*) &\approx 2.2 \\
  \sigma_* &\approx 10^{2.2}\frac{km}{s} = 10^{5.2}\frac{m}{s} \\
  r_{bh} &= \frac{GM_{bh}}{\sigma_*^2} \\
  &= 5.284\times10^{17}m \\
  \theta &= \frac{r}{d} = 1.22\frac{\lambda}{D} \\
  d &= \frac{rD}{1.22\lambda} \\
  &= 1.04\times10^{24}m
\end{align*}

\subsubsection*{Question 21.1}
The quasar PDS 456 has a redshift \( z = 0.184 \) and an apparent magnitude
\( m_V = 14.0 \).
\begin{enumerate}[(a)]
  \item What is the distance to this quasar?
  \begin{align*}
    cz &= H_0d \\
    d &= \frac{cz}{H_0} \\
    &= 788.57 Mpc
  \end{align*}
  \item What is its absolute magnitude, \( M_V \)?
  \begin{align*}
    m_V-M_V &= 5\log\left(\frac{d}{10}\right) \\
    M_V &= m_V-5\log\left(\frac{d}{10}\right) \\
    &= -25.5
  \end{align*}
\end{enumerate}

\begin{center}
  You can find all my notes at \url{http://omgimanerd.tech/notes}. If you have
  any questions, comments, or concerns, please contact me at
  alvin@omgimanerd.tech
\end{center}

\end{document}
