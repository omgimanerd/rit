\documentclass{math}

\usepackage{enumerate}

\title{University Astronomy: Homework 3}
\author{Alvin Lin}
\date{January 2019 - May 2019}

\begin{document}

\maketitle

\subsubsection*{Question 3.2}
The asteroid Eros is seen in opposition from the Earth once every 847 days. What
is the sidereal orbital period of Eros? What is the length \( a \) of the
semimajor axis of Eros' orbit? The eccentricity of the orbit of Eros is
\( e = 0.223 \). Does Eros ever come within 1 AU of the Sun?
\begin{align*}
  \frac{1}{P} &= \frac{1}{P_E}-\frac{1}{P_{Eros}} \\
  &= \frac{1}{365}-\frac{1}{847} \\
  P &= 625~days = 1.71 yr \\
  P^2 &= Ka^3 \\
  a &= \sqrt[3]{\frac{P^2}{K}} \\
  &= \sqrt[3]{\frac{1.71^2}{1yr^2AU^{-3}}} \\
  &= 1.431~AU
\end{align*}
\begin{align*}
  e &= (1-\frac{b^2}{a^2})^{\frac{1}{2}} \\
  e^2 &= 1-\frac{b^2}{a^2} \\
  \frac{b^2}{a^2} &= 1-e^2 \\
  b &= \sqrt{a^2(1-e^2)} \\
  &= \sqrt{(1.431^2)(1-0.223^2)} \\
  &= 1.395~AU
\end{align*}
Eros does not come within 1 AU of the Sun.

\subsubsection*{Question 8.1}
What is the mean mass density \( \bar{\rho} \) of Saturn's largest satellite,
Titan? What does this suggest about the composition of Titan?
\begin{align*}
  \rho &= \frac{3M}{3\pi R^3} \\
  &= \frac{3(1346\times10^{20}~kg)}{4\pi(2575000~m)^3} \\
  &\approx 1882.97\frac{kg}{m^3}
\end{align*}
This is typical of a Jovian planet, suggesting Titan is mostly gas and ice.

\subsubsection*{Question 8.2}
Radioactive decay of elements in the Earth's interior results in a mean heat
flux through the Earth's surface of \( 5\times10^{-2}Wm^{-2} \). What is this
flux expressed as a fraction of the energy flux due to thermal re-radiation of
absorbed solar energy? If radioactive decay were the \textit{only} heat source
for the Earth, what would the Earth's surface temperature be?
\begin{align*}
  F &= \frac{L_{Sun}}{4\pi r^2}(1-A) \\
  &= \frac{3.839\times10^{26}~W}{4\pi(1.496\times10^8~km)^2}(1-0.4) \\
  &= 8.19\times10^8Wm^{-2} \\
  \frac{5\times10^{-2}Wm^{-2}}{8.19\times10^8Wm^{-2}} &=
    6.10500611\times10^{-11} \\
  L_p &= 4\pi R^2\sigma_{SB}T_p^4 \\
  T_p &= \sqrt[4]{\frac{L_p}{4\pi R^2\sigma_{SB}}} \\
  &= \sqrt[4]{\frac{5\times10^{-2}Wm^{-2}}
    {4\pi(6378000m)^2(5.670\times10^{-8}Wm^{-2}K^{-4})}} \\
  &= 0.0064K
\end{align*}

\subsubsection*{Question 8.4}
Pure, solid water ice has an albedo of \( A \approx 0.35 \). What is the
minimum distance from the Sun at which a rapidly rotating ice cube would
remain frozen? Between which two planets does this distance lie?
\begin{align*}
  T_p &\approx 395K(1-A)^{\frac{1}{4}}(\frac{r}{1AU})^{-\frac{1}{2}} \\
  273.15K &= 395K(0.65)^{\frac{1}{4}}(\frac{r}{1AU})^{-\frac{1}{2}} \\
  \frac{273.15K}{395K(0.65)^{\frac{1}{4}}} &= \sqrt{\frac{1AU}{r}} \\
  \frac{1}{0.77} &= \frac{1}{r}AU \\
  r &= 0.77AU
\end{align*}
This distance lies between Venus and the Earth.

\subsubsection*{Question 10.5}
How often does an observer at the Sun's location see the rings of Saturn exactly
edge-on? \par
An observer at the Sun's location observes the rings of Saturn exactly edge-on
twice per revolution of Saturn around the Sun (every 15 years).

\subsubsection*{Slide Question 1}
What is the approximate energy of the Gamow peak? \par
300 keV

\subsubsection*{Slide Question 2}
What is the process that generates energy for a star that has a central
temperature of 25 million Kelvin? \par
The CNO cycle

\subsubsection*{Slide Question 3}
How much gravitational energy will be generated as kinetic energy when the Sun
shrinks to a white dwarf (roughly the size of the Earth)?
\begin{align*}
  U &= \frac{3}{5}\frac{GM^2}{R} \\
  &= \frac{3}{5}\frac{(6.673\times10^{-11}m^3kg^{-1}s^{-2})
    (1.989\times10^{30}kg)^2}{6378km} \\
  &= 2.48\times10^{46}J
\end{align*}
Approximately half of this is converted to kinetic energy, so
\( \sim1.24\times10^{46}J \).

\subsubsection*{Slide Question 4}
What is the temperature in a shock induced by a relativistic jet of electrons
accelerated to the speed of light?
\begin{align*}
  T &= \frac{3}{16}\frac{mc^2}{k} \\
  &= \frac{3}{16}\frac{(9.109\times10^{-31}kg)(2.998\times10^8ms^{-1})^2}
    {1.381\times10^{-23}m^2kgs^{-2}K^{-1}} \\
  &= 1.11\times10^9 K
\end{align*}

\begin{center}
  You can find all my notes at \url{http://omgimanerd.tech/notes}. If you have
  any questions, comments, or concerns, please contact me at
  alvin@omgimanerd.tech
\end{center}

\end{document}
