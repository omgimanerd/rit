\documentclass{math}

\usepackage{enumerate}

\title{University Astronomy}
\author{Alvin Lin}
\date{January 2019 - May 2019}

\begin{document}

\maketitle

\subsubsection*{Question 1.2}
The right ascension and declination of the seven stars of the Big Dipper are
given below [omitted]. For what range of latitudes are all the stars of the Big
Dipper circumpolar? What is the southernmost latitude from which all the stars
of the Big Dipper can be seen? For what range of latitudes are \textit{none} of
the stars of the Big Dipper ever seen above the horizon? \par
All the stars are circumpolar for observers north of \( 90^{\circ}-
61^{\circ}45' = 29^{\circ}15' \) N latitude. All the stars are visible up to
\( 90^{\circ}-61^{\circ}45' = 20^{\circ}15' \) S latitude (below the
equator). None of the stars are visible past \( 90^{\circ}-49^{\circ}19' =
41^{\circ}41' \) S latitude to the south pole.

\subsubsection*{Question 1.3}
Columbus, Ohio, is in the Eastern Time Zone, for which the civil time is equal
to the mean solar time along the \( 75^{\circ} \) W meridian of longitude.
\begin{enumerate}[(a)]
  \item Ignoring daylight saving time for the moment, are there any days of the
    year when civil noon (as shown by a clock) is the same as apparent local
    noon (as shown by the Sun) in the city of Columbus? If so, what day or days
    are they? \par
    The days for which civil noon matches apparent local noon are April 15th,
    June 14th, September 2nd, and December 25th according to zero point on the
    Sun's analemma when plotting its declination as a function of the equation
    of time.
  \item Daylight saving time advances the clock by one hour from the second
    Sunday in March to the first Sunday in November. When daylight saving time
    is in effect, are there any days of the year when civil noon is the same as
    apparent local noon in the city of Columbus? If so, what day or days are
    they? \par
    Civil noon will match local noon only on December 25th.
\end{enumerate}

\subsubsection*{Question 1.4}
Suppose you've been granted access to a large telescope during the last week in
September. One of the two objects you want to observe is in the constellation
Virgo; the other is in the constellation Pisces. You only have time to observe
one object: which should you choose? Please explain your answer. \par
During this time, the Sun would appear to pass directly through Libra, meaning
it would be between the Earth and Virgo. It would be easiest to observe Pisces,
which is on the opposite side of the Earth.

\subsubsection*{Question 1.5}
In \textit{The Old Man and the Sea}, Hemingway described the old man lying in
his boat off the coast of Cuba, looking up at the sky just after sunset: ``It
was dark now as it becomes dark quickly after the Sun sets in September. He lay
against the worn wood of the bow and rested all that he could. The first stars
were out. He did not know the name of Rigel but he saw it and knew soon they
would all be out and he would have all his distant friends.'' Explain what is
astronomically incorrect about this passage. \par
Rigel is located at RA 5 hours, 14 minutes, and 32 seconds, and declination
\( -8^{\circ}12'6'' \). During a September night in Cuba, it would be on the
other side of the Earth.

\subsubsection*{Question 1.6}
\begin{enumerate}[(a)]
  \item Consider two points on the Earth's surface that are separated by 1
    arcsecond as seen from the center of the (assumed to be transparent) Earth.
    What is the physical distance between the two points?
    \[ \frac{24901~miles}{360^{\circ}}\times\frac{1^{\circ}}{3600~arcminutes} =
      0.019~miles \]
  \item Consider two points on the Earth's equator that are separated by 1
    second of time. What is the physical distance between the two points?
    \[ \frac{24901~miles}{24~hours}\times\frac{1~hour}{3600~seconds} =
      0.288~miles\]
\end{enumerate}

\subsubsection*{Question 1.7}
The bright star Mintaka (also known as \( \delta \) Orionis, the westernmost
star of Orion's belt) is extremely close to the celestial equator. Amateur
astronomers can determine the field of view of their telescope (that is, the
angular width of the region that they can see through the telescope) by timing
how long it takes Mintaka to drift through their field of view when the
telescope is held stationary in hour angle. How long does it take Mintaka to
drift through a \( 1^{\circ} \) field of view?
\[ 1^{\circ}\times\frac{24~hours}{360^{\circ}} = 0.066~hours = 4~minutes \]

\begin{center}
  You can find all my notes at \url{http://omgimanerd.tech/notes}. If you have
  any questions, comments, or concerns, please contact me at
  alvin@omgimanerd.tech
\end{center}

\end{document}
