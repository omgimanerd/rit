\documentclass{math}

\usepackage{enumerate}

\title{University Astronomy: Homework 3}
\author{Alvin Lin}
\date{January 2019 - May 2019}

\begin{document}

\maketitle

\subsubsection*{Question 5.1}
Verify that the Maxwell-Boltzmann distribution has its maximum at a speed
\[ \nu_p = \left(\frac{2kT}{m}\right)^{\frac{1}{2}} \]
\begin{align*}
  F(v)\diff{v} &= 4\pi\left(\frac{m}{2\pi kT}\right)^{\frac{3}{2}}v^2
    \e^{-\frac{mv^2}{2kT}}\diff{v} \\
  \ddiff{}{v}F(v) &= 0 \\
  \ddiff{}{v}4\pi\left(\frac{m}{2\pi kT}\right)^{\frac{3}{2}}v^2
    \e^{-\frac{mv^2}{2kT}} &= 0 \\
  Let: \alpha = -\frac{m}{2kT} \\
  \ddiff{}{v}v^2\e^{\alpha v^2} &= 0 \\
  v^2\e^{\alpha v^2}2\alpha v +2v\e^{\alpha v^2} &= 0 \\
  \alpha v^2+1 &= 0 \\
  v &= \pm\sqrt{-\frac{1}{\alpha}} \\
  &= \pm\sqrt{-(-\frac{2kT}{m})} \\
  v_{max} &= \sqrt{\frac{2kT}{m}} = \left(\frac{2kT}{m}\right)^{\frac{1}{2}}
\end{align*}

\subsubsection*{Question 5.5}
\begin{enumerate}[(a)]
  \item A neutral sodium atom has ionization potential of \( \chi = 5.1eV \).
    What is the speed of a free electron that has just barely enough kinetic
    energy to collisionally ionize a sodium atom in its ground state? What is
    the speed of a free \textit{proton} with just enough kinetic energy to
    collisionally ionize this atom?
    \begin{align*}
      \frac{1}{2}m_ev_e^2 &\ge \chi \\
      v_e &\ge \sqrt{\frac{2\chi}{m_e}} \\
      &\ge 1342278.88\frac{m}{s} \\
      \frac{1}{2}m_pv_p^2 &\ge \chi \\
      v_p &\ge \sqrt{\frac{2\chi}{m_p}} \\
      &\ge 31329.96\frac{m}{s}
    \end{align*}
  \item What is the temperature \( T \) of a gas in which the average particle
    kinetic energy is just barely sufficient to ionize a sodium atom in its
    ground state?
    \begin{align*}
      \langle E\rangle &= \frac{3}{2}kT \\
      T &= \frac{2\chi}{3k} \\
      &= 39443.15 K
    \end{align*}
  \item At the temperature \( T \) computed in part (b), what is the expected
    thermal Doppler broadening, \( \frac{\Delta\lambda}{\lambda} \), of a sodium
    spectral line? (Hint: the only stable isotope of sodium has mass number
    A = 23).
    \begin{align*}
      \frac{\Delta\lambda}{\lambda} &\approx
        (3\times10^{-7})\left(\frac{T}{1K}\right)^{\frac{1}{2}}
        \mu^{-\frac{1}{2}} \\
      &\approx (3\times10^{-7})\sqrt{\frac{39433.15K}{1K}}(23)^{-\frac{1}{2}} \\
      &\approx 1.242\times10^{-5}
    \end{align*}
\end{enumerate}

\subsubsection*{Question 5.8}
If an incandescent light bulb has a luminosity \( L = 60W \) and a filament
temperature of \( T = 2900K \), what must be the surface area of its filament?
If the filament consists of a cylindrical wire with diameter
\( d = 4.6\times10^{-5}m \) (as in a standard incandescent 60 watt, 120 volt
bulb), what is the length of the wire?
\begin{align*}
  L &= A\sigma_{SB}T^4 \\
  A &= \frac{L}{\sigma_{SB}T^4} \\
  &= \frac{60W}{5.67\times10^{-8}Wm^{-2}K^{-4}(2900K)^4} \\
  &= 1.496\times10^{-5}m^2 \\
  &= \pi(\frac{d}{2})^2l \\
  l &= \frac{A}{\pi(\frac{d}{2})^2} \\
  &= \frac{1.496\times10^{-5}}{\pi(\frac{4.6\times10^{-5}}{2})^2} \\
  &= 9002.65m
\end{align*}

\subsubsection*{Slide Question 1}
What is the signal to noise ratio for a source that generates 100 electrons in
a detector if the observation is limited by shot noise from the source?
\[ \frac{S}{N} = \frac{100}{\sqrt{100}} = \frac{100}{10} = 10 \]

\subsubsection*{Slide Question 2}
What is the equivalent width of the following absorption line?
[diagram not shown]
\begin{align*}
  W_{\lambda} &= \int(1-\frac{F_{\lambda}}{F_0})\diff{\lambda} \\
  &= \int_{1.5\mu m}^{1.6\mu m}(1-\frac{0.5}{1})\diff{\lambda} \\
  &= \int_{1.5\mu m}^{1.6\mu m}0.5\diff{\lambda} \\
  &= 0.25\mu m
\end{align*}

\subsubsection*{Slide Question 3}
What is the wavelength of the Bracket-gamma (HI 7-4) line? \par
1458.93 nanometers.

\subsubsection*{Slide Question 4}
In which waveband does the Bracket-gamma line appear? \par
Infrared.

\subsubsection*{Slide Question 5}
At which wavelength, observed by the DIRBE instrument on COBE, does zodiacal
emission appear most prominent? \par
Around 10 \(\mu m \) (\( \sim9.66\mu m \)), or infrared wavelengths.

\subsubsection*{Slide Question 6}
At which wavelength is the maximum flux for a blackbody of temperature 1000K?
\begin{align*}
  \lambda_{max}T &= 2898\mu m K \\
  \lambda_{max} &= \frac{2898\mu m}{1000} \\
  &= 2.898\mu m = 2898nm
\end{align*}
(Infrared)

\subsubsection*{Slide Question 7}
What is the wavelength of a photon emitted by a hydrogen atom with the electron
in the ground state when it flips from a spin that is aligned with the spin of
the proton to a spin that is anti-aligned with the spin of the proton? \par
The photon emitted has a 21 centimeter wavelength.

\begin{center}
  You can find all my notes at \url{http://omgimanerd.tech/notes}. If you have
  any questions, comments, or concerns, please contact me at
  alvin@omgimanerd.tech
\end{center}

\end{document}
