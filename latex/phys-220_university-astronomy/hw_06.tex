\documentclass{math}

\usepackage{enumerate}

\title{University Astronomy: Homework 6}
\author{Alvin Lin}
\date{January 2019 - May 2019}

\begin{document}

\maketitle

\subsubsection*{Question 1}
Approximately what fraction of the total mass in the Solar System is in
the Sun? \par
\begin{center}
  \begin{tabular}{|c|c|}
    \hline
    Planet & Mass \( (M_{\oplus}) \) \\
    \hline
    Mercury & 0.0553 \\
    Venus & 0.8150 \\
    Earth & 1.0000 \\
    Mars & 0.1074 \\
    Jupiter & 317.8 \\
    Saturn & 95.16 \\
    Uranus & 14.50 \\
    Neptune & 17.20 \\
    Total & 446.6377 \\
    \hline
  \end{tabular}
\end{center}
\[ \frac{446.6377M_{\oplus}}{M_{\odot}} = 0.0013 \]

\subsubsection*{Question 2}
How old are comets compared to the age of the Solar System? \par
Comets are about the same age as the solar system on the cosmic scale as they
are formed from leftover material during the solar system formation event.

\subsubsection*{Question 3}
What is the source of elements heavier than helium in the Sun? \par
The CNO fusion cycle in stars yields elements heavier than helium based on the
temperature of the star.

\subsubsection*{Question 4}
Calculate the approximate maximum distance from the Sun for a rapidly
rotating planet to have liquid water, assuming the planet is a blackbody.
\begin{align*}
  T_p &\approx 279K(1-A)^{\frac{1}{4}}
    \left(\frac{r}{1AU}\right)^{-\frac{1}{2}} \\
  273K &\approx 279K(1-0)^{\frac{1}{4}}
    \left(\frac{r}{1AU}\right)^{-\frac{1}{2}} \\
  \frac{273}{279} &\approx \sqrt{\frac{1AU}{r}} \\
  0.978^2 &\approx \frac{1}{r}AU \\
  r &\approx 1.045AU
\end{align*}

\begin{center}
  You can find all my notes at \url{http://omgimanerd.tech/notes}. If you have
  any questions, comments, or concerns, please contact me at
  alvin@omgimanerd.tech
\end{center}

\end{document}
