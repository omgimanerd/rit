\documentclass{math}

\usepackage{enumerate}

\title{University Astronomy: Homework 11}
\author{Alvin Lin}
\date{January 2019 - May 2019}

\begin{document}

\maketitle

\subsubsection*{Question 17.4}
In this problem, you will estimate the duration of the horizontal branch
phase in a \( 1M_{\odot} \) star.
\begin{enumerate}
  \item Compute the energy released in the net triple alpha reaction
    \( 3^4\text{He}\to ^{12}\text{C} \). The masses of \( ^4\text{He} \)
    and \( ^{12}\text{C} \) are 4.0026 amu and 12.0000 amu, respectively,
    where 1 amu (atomic mass unit) = \( 1.6606\times10^{-27}\text{kg} \).
  \begin{align*}
    E_1 &= (\Delta m)c^2 \\
    &= (3\times m_{He}+m_{C})c^2 \\
    &= 1.166\times10^{-12}J
  \end{align*}
  \item Assume that at the beginning of the horizontal branch phase,
    10\% of the original mass of the star is in the form of
    \( ^{4}\text{He} \) in the stellar core. Estimate the total energy
    released by fusing this amount of helium into carbon via the triple
    alpha process.
  \begin{align*}
    E_2 &= \frac{M_{\odot}}{10}\times\frac{E_1}{3m_{He}} \\
    &= 1.163\times10^{43}J
  \end{align*}
  \item Assume that during the horizontal branch phase,
    \( L = 100L_{\odot} \). If all this luminosity is provided by fusion of
    helium to carbon in the core, how long will the horizontal branch phase
    last?
  \begin{align*}
    L &= \frac{E_2}{t} \\
    t &= \frac{E_2}{L} \\
    &= 3.029\times10^{14}s = 9.604\times10^{6}yr
  \end{align*}
\end{enumerate}

\subsubsection*{Question 18.1}
What would the rotation period of the Sun if it collapsed to a radius
\( R = 6000km \) without losing angular momentum?
\begin{align*}
  P &= \frac{2\pi R}{0.1c} \\
  &= 1.26s
\end{align*}

\subsubsection*{Question 18.2}
What is the radius of a \( 1.5M_{\odot} \) neutron star, expressed as a
fraction of its Schwarzschild radius?
\begin{align*}
  R_{ns} &= 11km\left(\frac{M_{ns}}{1.4M_{\odot}}\right)^{-\frac{1}{3}} \\
  &= 10749.91m \\
  R_{Sch} &= 3km\left(\frac{M_{ns}}{M_{\odot}}\right) \\
  &= 4500m \\
  \frac{R_{ns}}{R_{Sch}} &= 2.366
\end{align*}

\subsubsection*{Question 18.4}
What is the mean density of a \( 1.5M_{\odot} \) neutron star? A carbon
nucleus has a radius \( r\approx3\times10^{-15}m \); what is its density?
What is the ratio of the two densities?
\begin{align*}
  R_{ns} &= 10749.91m \\
  \rho_{ns} &= \frac{M_{ns}}{V_{ns}} =
    \frac{M_{ns}}{\frac{4}{3}\pi(R_{ns})^3} \\
  &= 5.734\times10^{17}\frac{kg}{m^3} \\
  \rho_{C} &= \frac{M_C}{V_C} =
    \frac{M_C}{\frac{4}{3}\pi(r_C)^3} \\
  &= 1.762\times10^{17}\frac{kg}{m^3} \\
  \frac{\rho_{ns}}{\rho_{C}} &= 3.25
\end{align*}

\subsubsection*{Slide Question}
How many stars with masses between 1 and 2 solar masses form in a star
formation event if there are 10 stars with masses between 100 and 101 solar
masses, assuming a Salpeter initial mass function?
\begin{align*}
  10 &= A\int_{100}^{101}m^{-2.35}\diff{m} \\
  &= A(1.972\times10^{-5}) \\
  A &= \frac{10}{1.972\times10^{-5}} \\
  &= 5.071\times10^5 \\
  N &= A\int_{1}^{2}m^{-2.35}\diff{m} \\
  &= 2.283\times10^5
\end{align*}

\begin{center}
  You can find all my notes at \url{http://omgimanerd.tech/notes}. If you have
  any questions, comments, or concerns, please contact me at
  alvin@omgimanerd.tech
\end{center}

\end{document}
