\documentclass{math}

\title{Principles of Data Management}
\author{Alvin Lin}
\date{August 2018 - December 2018}

\begin{document}

\maketitle

\section*{Databases}
A \textbf{database management system (DBMS)} consists of:
\begin{itemize}
  \item a collection of interrelated data
  \item programs to access that data
  \item and an environment that makes it easy to use
\end{itemize}
Things that use databases:
\begin{itemize}
  \item social networks
  \item retail
  \item university systems
  \item banks
  \item manufacturing
\end{itemize}
Before databases, things were stored in flat files. This had a few issues:
\begin{itemize}
  \item data redundancy
  \item data access was difficult
  \item data isolation, data is in different formats
  \item integrity problems
  \item lack of atomicity
  \item problems with concurrency
  \item no built-in security
\end{itemize}
Database levels of extraction:
\begin{itemize}
  \item Physical - how it's stored
  \item Logical - relationships of data
  \item View - limit of the data
\end{itemize}
Instances and schemas:
\begin{itemize}
  \item Physical schema - how the data is physically stored
  \item Logical schema - restrictions applied to the data
  \item Instance - value of the data at a point in time
  \item Physical data dependence - we do not care how we store the data,
    changing the physical schema should not affect the logical schema.
\end{itemize}

\subsection*{Data Model}
A data model is a collection of tools to describe data, the relationships in the
data, the semantics of the data, and the constraints applied to the data. There
are a few key models:
\begin{itemize}
  \item Relational Model - data represents objects from the program
  \item Entity Relationship Model
  \item Semi-Structured Model - includes things like XML, there is an overall
    structure to the document, but it is not concrete
  \item Object Based Model
\end{itemize}
Spreadsheets are similar to relational models, the columns represent the type
of the data stored and the rows hold related data. \par
A \textbf{data definition language (DDL)} is used to define the database
structure, and is stored in the data dictionary. A data manipulation language
(DML) is used to access and modify data in the database, often called the query
language. Data manipulation languages come in two classes:
\begin{itemize}
  \item Pure (relational algebra, domain relational calculus, tuple relational
    calculus)
  \item Commercial (SQL)
\end{itemize}
SQL is the most common query language. It is not a turing complete language, but
there exist extensions to make it so. Generally, a higher level language is used
for complex operations.

\subsection*{Designing a database}
\begin{itemize}
  \item Logical - what is a good collection of data and how does it relate?
  \item Business Decisions - what do we need to record?
  \item Technical - where does that data go and how do we represent it?
  \item Physical Design - database engineers decide how stored data will be
    written to the disk.
\end{itemize}

\subsection*{Database Engine}
\begin{enumerate}
  \item \textbf{Storage Manager}: handles file organization, indexing/hashing,
    and manages where the data is physically stored.
  \item \textbf{Query Processor}: takes your query and sends it to a parser and
    translator, which is then converted to relational algebra. This is
    sent to an optimizer, which pulls metadata from the data and uses it to
    optimize the query. After the query it optimized, it is sent to an
    execution plan and then executed by pulling data from the appropriate
    databases. From there, we get output (most often in the form of a table).
  \item \textbf{Transaction Manager}: deals with worst case scenarios if the
    system fails. A transaction is defined by multiple physical operations
    treated as a one logical operation. This part handles transaction and
    concurrency management.
\end{enumerate}

\subsection*{Users}
\begin{itemize}
  \item Naive users: uses an application that uses a database, without
    knowledge of it.
  \item Application programmers: know about the database and the tables that
    exist to some degree. They may not have access to all data or understand
    the underlying infrastructure.
  \item Sophisticated user: uses tools to manipulate the database.
  \item Administrator: creates and manages the structure and indices of the
    database.
\end{itemize}

\subsection*{Database Architecture}
\begin{itemize}
  \item Centralized: all data is localized in one central database.
  \item Client-Server: a single point of entry into the database, with multiple
    backends to manage queries.
  \item Parallel: can be either centralized or client-server, running on
    multiple processors.
  \item Distributed: data is distributed among multiple servers.
\end{itemize}
Timeline of database usage:
\begin{itemize}
  \item 1950s: data processing done with tapes, input on punch cards, access was
    sequential.
  \item 1960-70s: hard disks allowed for random access, network models and the
    relational data model were invented, as well as transaction processing.
  \item 1980s: SQL was invented, parallel and distributed systems and object
    oriented databases were conceptualized.
  \item 1990s: data mining systems and decision support systems became
    popularized, web commerce grew, and databases approached the scale of
    terabytes.
  \item 2000s: XML and XQuery were heavily used for data, document storage, and
    document sending.
  \item Recently: Big Data, BigTable, large scale data analytics into the scale
    of petabytes and exabytes.
\end{itemize}

\begin{center}
  You can find all my notes at \url{http://omgimanerd.tech/notes}. If you have
  any questions, comments, or concerns, please contact me at
  alvin@omgimanerd.tech
\end{center}

\end{document}
