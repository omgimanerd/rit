\documentclass{math}

\title{Principles of Data Management}
\author{Alvin Lin}
\date{August 2018 - December 2018}

\begin{document}

\maketitle

\section*{Databases}
A \textbf{database management system (DBMS)} consists of:
\begin{itemize}
  \item a collection of interrelated data
  \item programs to access that data
  \item and an environment that makes it easy to use
\end{itemize}
Things that use databases:
\begin{itemize}
  \item social networks
  \item retail
  \item university systems
  \item banks
  \item manufacturing
\end{itemize}
Before databases, things were stored in flat files. This had a few issues:
\begin{itemize}
  \item data redundancy
  \item data access was difficult
  \item data isolation, data is in different formats
  \item integrity problems
  \item lack of atomicity
  \item problems with concurrency
  \item no built-in security
\end{itemize}
Database levels of extraction:
\begin{itemize}
  \item Physical - how it's stored
  \item Logical - relationships of data
  \item View - limit of the data
\end{itemize}
Instances and schemas:
\begin{itemize}
  \item Physical schema - how the data is physically stored
  \item Logical schema - restrictions applied to the data
  \item Instance - value of the data at a point in time
  \item Physical data dependence - we do not care how we store the data,
    changing the physical schema should not affect the logical schema.
\end{itemize}

\subsection*{Data Model}
A data model is a collection of tools to describe data, the relationships in the
data, the semantics of the data, and the constraints applied to the data. There
are a few key models:
\begin{itemize}
  \item Relational Model - data represents objects from the program
  \item Entity Relationship Model
  \item Semi-Structured Model - includes things like XML, there is an overall
    structure to the document, but it is not concrete
  \item Object Based Model
\end{itemize}
Spreadsheets are similar to relational models, the columns represent the type
of the data stored and the rows hold related data. \par
A \textbf{data definition language (DDL)} is used to define the database
structure, and is stored in the data dictionary. A data manipulation language
(DML) is used to access and modify data in the database, often called the query
language. Data manipulation languages come in two classes:
\begin{itemize}
  \item Pure (relational algebra, domain relational calculus, tuple relational
    calculus)
  \item Commercial (SQL)
\end{itemize}
SQL is the most common query language. It is not a turing complete language, but
there exist extensions to make it so. Generally, a higher level language is used
for complex operations.

\subsubsection*{Designing a database}
\begin{itemize}
  \item Logical - what is a good collection of data and how does it relate?
  \item Business Decisions - what do we need to record?
  \item Technical - where does that data go and how do we represent it?
  \item Physical Design - database engineers decide how stored data will be
    written to the disk.
\end{itemize}

\begin{center}
  You can find all my notes at \url{http://omgimanerd.tech/notes}. If you have
  any questions, comments, or concerns, please contact me at
  alvin@omgimanerd.tech
\end{center}

\end{document}
