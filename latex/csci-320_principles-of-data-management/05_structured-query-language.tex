\documentclass{math}

\usepackage{listings}

\title{Principles of Data Management}
\author{Alvin Lin}
\date{August 2018 - December 2018}

\begin{document}

\lstset{basicstyle=\ttfamily\footnotesize,breaklines=true}
\maketitle

\section*{Structured Query Language}
Structured Query Language (SQL) was created at IBM in the 80s:
\begin{itemize}
  \item SQL-86 (first standard)
  \item SQL-89
  \item SQL-92 (what all SQL compliant DBs use)
  \item SQL-1999
  \item SQL-2003
\end{itemize}
SQL is a commercial language that is intended to be a data manipulation
language, but supports features of data definition languages. It supports the
following features:
\begin{itemize}
  \item Schema
  \item Domain
  \item Integrity Constraints
  \item Indicies
  \item Security
  \item Physical Storage
\end{itemize}

\subsection*{Domain Types}
\begin{itemize}
  \item \texttt{char(n)} - n number of characters
  \item \texttt{int} - numbers
  \item \texttt{smallint} - smaller value number
  \item \texttt{numeric(p,d)} - precision decimal
  \item \texttt{real,double,float} - ``infinite'' precision decimals
  \item \texttt{varchar(n)} - variable length characters
\end{itemize}

\subsection*{Create Table}
\begin{lstlisting}
  create table r (A1 D1, A2 D2, ... ,An Dn) (integrity constraints);
\end{lstlisting}
Example:
\begin{lstlisting}
  create table dog(
    ID int,
    Name varchar(20),
    Salary numeric(8,2),
    goodboy smallint
  );
\end{lstlisting}
Example:
\begin{lstlisting}
  create table p(
    ID int,
    Name varchar(4) not null,
    DOB varchar(10),
    dog_id int,
    primary key(ID, Name),
    foreign key(dog_id) references dog(ID)
  );
\end{lstlisting}

\subsection*{CRUD}
\begin{lstlisting}
  insert into r values (v1,v2,vn);
  insert into dog values (1,"cat",10.00,0);
  delete from dog;
  drop table r;
  alter table dog add iscat int not null default 0;
  alter table dog drop iscat;
\end{lstlisting}

\subsection*{Select}
\begin{lstlisting}
  select a1,a2,a3,an from r1,r2,rn where predicate;
  select name from dog;
  select distinct name from dog;
  select all name from dog;
  select "joe" as name;
  select name,"lab" as breed from dog;
  select * from dog;
\end{lstlisting}
Explicitly specify what attributes to return, \texttt{SELECT *} should be
avoided with the exception of manually viewing data.
\begin{lstlisting}
  select id,name,salary as pay from dog where salary > 80000;
\end{lstlisting}
The \texttt{where} keyword allows for conditions by which we can filter the
returned rows. We can specify modifications to the returned attribute and use
boolean conditions to create compound conditionals.
\begin{lstlisting}
  select id,name,salary/26 as pay from dog where salary > 80000 and breed="lab";
\end{lstlisting}
The \texttt{between} modifier allows us to select a range as a predicate.
\begin{lstlisting}
  select * from dog where age between 6 and 9;
\end{lstlisting}
Selecting from multiple tables using \texttt{SELECT *} returns a Cartesian
product.
\begin{lstlisting}
  select * from dog,cat;
\end{lstlisting}
This is not particularly useful, it is much more useful to use a predicate to
match the data.
\begin{lstlisting}
  select * from dog,cat where dog.name = cat.name and dog.age = 4;
  select * from dog as D, cat as C where D.age = C.age;
\end{lstlisting}

\subsubsection*{Examples}
\begin{center}
  \begin{tabular}{|c|c|}
    \hline
    \multicolumn{2}{|c|}{table dog} \\
    \hline
    name & owner \\
    \hline
    Charlie & Snoopy \\
    Woodstock & Snoopy \\
    Lucy & Ricky \\
    Snoopy & Sam \\
    \hline
  \end{tabular}
\end{center}
Finding all the owners with dogs named Woodstock:
\begin{lstlisting}
  select d2.owner from dog where name = "woodstock";
\end{lstlisting}
Finding all the dogs owned by Snoopy:
\begin{lstlisting}
  select name from dog where owner = "snoopy";
\end{lstlisting}
Selecting the owner's owner of Charlie:
\begin{lstlisting}
  select owner from dog d1, dog d2 where d1.name = "charlie" and
      d1.owner = d2.name;
\end{lstlisting}

\subsection*{String Operations}
The percent sign (\%) is used for partial substring matches while the
underscore (\_) is used for single character matches.
\begin{lstlisting}
  select * from dog where name like "%py";
  select * from dog where name like 'Sn\'' escape '\';
\end{lstlisting}
Concatenation is expressed with double pipes (\textbar\textbar).
\begin{lstlisting}
  select * from fname||lname like "%foo";
\end{lstlisting}

\subsection*{Ordering}
The \texttt{order by} keyword allows for queries to be sorted. Specifying more
than one sort option allows for sorting by multiple fields.
\begin{lstlisting}
  select name,lname from dog order by name asc;
  select name,lname from dog order by name,lname desc;
  select name,lname from dog order by name desc,lname;
\end{lstlisting}

\subsection*{Set Operations}
Set operation keywords are valid in SQL.
\begin{lstlisting}
  (select name from dog) union (select name from cat);
  (select name from dog) intersect (select name from cat);
  (select name from dog) except (select name from cat);
\end{lstlisting}

\subsection*{Null Values}
\begin{lstlisting}
  select field from table where age is null
\end{lstlisting}
Boolean evaluation:
\begin{enumerate}
  \item True or null evaluates to true
  \item False or null evaluates to null
  \item Null or null evaluates to null
  \item True and null evaluates to null
  \item False and null evaluates to false
  \item Null and null evaluates to null
\end{enumerate}

\subsection*{Aggregate Functions}
There are \( 4\frac{1}{2} \) aggregate functions: \texttt{avg, min, max, sum,
count}. Count does not entirely count as an aggregate function. We can use
this to count the number of non-null rows:
\begin{lstlisting}
  select count(*) from penguin;
\end{lstlisting}
Because no rows can be non-null, this just returns the number of rows in your
database. We can also select the number of non-null values for a field:
\begin{lstlisting}
  select count(age) from penguin;
\end{lstlisting}
To count distinct values in a field:
\begin{lstlisting}
  select count(distinct name) from dog;
\end{lstlisting}
The other aggregate functions can be used in a similar way:
\begin{lstlisting}
  select avg(salary) from dog where state="CA";
\end{lstlisting}
If there are no dogs, then 0 is returned. If some dogs have a null salary, then
null is returned.
\begin{lstlisting}
  select avg(salary) as avgs,state from dog group by state;
  select avg(salary) as avgs,state,name from dog group by state,name;
\end{lstlisting}
Results like this can be filtered with the \texttt{having} clause:
\begin{lstlisting}
  select avg(salary) as avgs,state from dog group by state having avgs > 5;
\end{lstlisting}

\subsection*{Subqueries}
Subquery expressions can be used for set membership, comparisons, and
cardinality expressions.
\begin{lstlisting}
  select name,age from dog where age in (select age from cat where
    state="sleeping");
  select name,age from dog where (age,legs) in (select age,legs from cat where
    state="sleeping");
  select name from dog where age > some(select age from cat where state="CA");
  select name from dog where age > all(select age from cat where state="CA");
  select name from dog where age > (select min(age) from cat where state="CA");
\end{lstlisting}
Because nested subqueries can get very messy, we use the \texttt{with} clause
as a way of defining a temporary relation that only exists in the clause in
which the \texttt{with} clause occurs.
\begin{lstlisting}
  with dept_total(dept_name, value) as
    (select dept_name, sum(salary) from instructor group by dept_name),
    dept_total_avg(value) as (select avg(value) from dept_total)
  select dept_name from dept_total,dept_total_avg
    where dept_total.value > dept_total_avg.value;
\end{lstlisting}
Scalar subqueries must be used where only a single value is expected.
\begin{lstlisting}
  select dept_name,
    (select count(*) from instructor
      where department.dept_name=instructor.dept_name) as num_instructors
    from department;
\end{lstlisting}

\subsection*{Database Modification}
Deleting records (also accepts a valid where clause):
\begin{lstlisting}
  delete from instructor;
  delete from instructor where dept_name='Finance';
  delete from instructor where dept_name in
    (select dept_name from department where building='Watson');
  delete from instructor where salary < (select avg(salary) from instructor);
\end{lstlisting}
Records can be inserted into databases:
\begin{lstlisting}
  insert into course(course_id, title, dept_name, credits)
    values('CS-437', 'Database Systems', 'Comp Sci', 4);
  insert into course values('CS-437', 'Database Systems', 'Comp Sci', 4);
  insert into student select ID, name, dept_name, 0 from instructor;
\end{lstlisting}
Records can be updated in databases:
\begin{lstlisting}
  update instructor
    set salary = salary * 1.03
    where salary > 100000;
  update instructor
    set salary = salary * 1.05
    where salary <= 100000;
\end{lstlisting}
As an aside, since we have no guarantee to the order of execution of the
previous statements, some instructors might get more than one raise. We can
use the \texttt{case} statement to solve this.
\begin{lstlisting}
  update instructor set salary = case
    when salary <= 100000 then salary * 1.05
    else salary * 1.03
  end
\end{lstlisting}
Updates can also take scalar subqueries:
\begin{lstlisting}
  update student S
    set total_credits = (select sum(credits) from takes,course
      where takes.course_id = course.course_id and
        S.ID = takes.ID and
        takes.grade <> 'F' and
        takes.grade is not null);
\end{lstlisting}

\subsection*{Database Joins}
Join types:
\begin{itemize}
  \item \texttt{inner}
  \item \texttt{left outer}
  \item \texttt{right outer}
  \item \texttt{full outer}
\end{itemize}
Conditions:
\begin{itemize}
  \item \texttt{natural}
  \item \texttt{on \textless predicate\textgreater, using (attributes)}
\end{itemize}
Natural joins match on all attributes with the same name in both tables. Inner
joins only select rows on the matching attribute (essentially enforcing that
no row in the output has null values) while full outer joins preserve all
information in both tables.
\begin{center}
  \begin{tabular}{|c|c|}
    \hline
    \multicolumn{2}{|c|}{Dog} \\
    \hline
    Name & Age \\
    \hline
    A & 5 \\
    B & 10 \\
    C & 7 \\
    \hline
  \end{tabular}
  \begin{tabular}{|c|c|c|}
    \hline
    \multicolumn{3}{|c|}{Cat} \\
    \hline
    ID & Name & State \\
    \hline
    1 & C & VT \\
    2 & D & IA \\
    \hline
  \end{tabular}
\end{center}
\begin{lstlisting}
  select * from dog natural left outer join cat;
\end{lstlisting}
\begin{center}
  \begin{tabular}{|c|c|c|c|}
    \hline
    Name & Age & ID & State \\
    \hline
    A & 5 & NULL & NULL \\
    B & 10 & NULL & NULL \\
    C & 7 & 1 & VT \\
    \hline
  \end{tabular}
\end{center}
\begin{lstlisting}
  select * dog natural right outer join cat;
\end{lstlisting}
\begin{center}
  \begin{tabular}{|c|c|c|c|}
    \hline
    Name & Age & ID & State \\
    \hline
    C & 7 & 1 & VT \\
    D & NULL & 2 & IA \\
    \hline
  \end{tabular}
\end{center}
\begin{lstlisting}
  select * from dog natural full outer join cat;
\end{lstlisting}
\begin{center}
  \begin{tabular}{|c|c|c|c|}
    \hline
    Name & Age & ID & State \\
    \hline
    A & 5 & NULL & NULL \\
    B & 10 & NULL & NULL \\
    C & 7 & 1 & VT \\
    D & NULL & 2 & IA \\
    \hline
  \end{tabular}
\end{center}
\begin{lstlisting}
  select * from dog inner join cat on cat.name=dog.name;
\end{lstlisting}
\begin{center}
  \begin{tabular}{|c|c|c|c|}
    \hline
    Name & Age & ID & State \\
    \hline
    C & 7 & 1 & VT \\
    \hline
  \end{tabular}
\end{center}
Suppose we have two cats with the same name:
\begin{center}
  \begin{tabular}{|c|c|}
    \hline
    \multicolumn{2}{|c|}{Dog} \\
    \hline
    Name & Age \\
    \hline
    A & 5 \\
    B & 10 \\
    C & 7 \\
    \hline
  \end{tabular}
  \begin{tabular}{|c|c|c|}
    \hline
    \multicolumn{3}{|c|}{Cat} \\
    \hline
    ID & Name & State \\
    \hline
    1 & C & UT \\
    2 & D & IA \\
    3 & C & MI \\
    \hline
  \end{tabular}
\end{center}
\begin{lstlisting}
  select * from dog natural full outer join cat;
\end{lstlisting}
\begin{center}
  \begin{tabular}{|c|c|c|c|}
    \hline
    Name & Age & ID & State \\
    \hline
    A & 5 & NULL & NULL \\
    B & 10 & NULL & NULL \\
    C & 7 & 1 & VT \\
    D & NULL & 2 & IA \\
    C & 7 & 3 & MI \\
    \hline
  \end{tabular}
\end{center}

\subsection*{Views}
Views are used to restrict access to attributes or rows. It represents a virtual
table.
\begin{lstlisting}
  create view old_dogs as (select name,age from dog where age > 10);
\end{lstlisting}
Executing any query on old\_dogs is equivalent to using the view as a subquery.
This is generally used to restrict access to the table, prevent users from
querying all dogs or querying certain attributes of dogs. We can also insert
into views.
\begin{lstlisting}
  insert into old_dogs values("Jeff", 13) values("Cheese", 8);
\end{lstlisting}
While this might work, you will never see Cheese by querying the view because
the view restricts access to dogs whose age is over 10.
\begin{lstlisting}
  create view dogs2 as (select name from dog,cat where dog.lives_in=cat.state);
\end{lstlisting}
Inserting into this view would lead to ambiguity, so in general, there are
certain restrictions for inserting into views.
\begin{enumerate}
  \item Only a single table is present in the from clause.
  \item No aggregations, distinct, etc.
  \item No group by or having clauses.
  \item Any attribute not in the selection can be set to null.
\end{enumerate}
There exist the concept of materialized views, which are physical tables that
contain the results of a view. It is a cache of data used for long running
queries that needs to be updated.

\subsection*{Transactions}
In most flavors of SQL, transactions begin with most statements and usually end
with a commit or rollback. You can specify for a group of statements to be
atomic however.
\begin{lstlisting}
  begin atomic;
    select * from dog;
    insert into cat ...
    update turtle ...
  commit;
\end{lstlisting}

\subsection*{Integrity Constraints}
Integrity contraints for a single table include the following:
\begin{itemize}
  \item \texttt{not null}
  \item \texttt{unique}
  \item \texttt{primary key}: not null, unique, indexed
  \item \texttt{check \textless predicate\textgreater}
\end{itemize}
\begin{lstlisting}
  create table dog (
    id int primary key,
    name varchar(30) not null,
    salary int,
    check (salary > 5),
    check (age < 10),
    check (breed in ("Lab", "Dalmatian", "GSD"))
  )
\end{lstlisting}
Integrity constraints can apply across multiple tables. Referential integrity
uses foreign keys to ensure some condition.
\begin{lstlisting}
  create table dog (
    id int primary key,
    name varchar(20) not null,
    foreign key(owner) references person(id)
    on update cascade
  )
\end{lstlisting}
If the referenced person is deleted or updated in database, we specify how the
database handles the constraint. \texttt{on update cascade} makes any changes
from the person table cascade into the dog table to update its value.
Alternatively, we can specify \texttt{on delete set null} to set the value to
null if the referenced person is deleted. Another option is to specify
\texttt{set default 0} to specify a default value if the referenced person is
deleted. Complex check statements can also be used like this, but it is not
usually recommended since there are better options.
\begin{lstlisting}
  create table dog (
    check owner_id in (select id from person)
  )
\end{lstlisting}

\subsection*{Additional Data Types}
\begin{itemize}
  \item Date - yyyy-mm-dd
  \item Time - HH:MM:SS
  \item Timestamp - Date and time
  \item Interval - the resulting of subtracting a date/time related data type
  \item Blob - binary large object (cannot be indexed)
  \item Clob - character large object (cannot be indexed)
\end{itemize}
Blobs and clobs are referenced by a pointer stored in the database. Because
they are not indexed, relevant metadata needs to be stored as well.

\subsection*{Indices}
Suppose have a dog table with the attributes id, name, age, and state:
\begin{lstlisting}
  create index dog_index on dog(state, age);
\end{lstlisting}
Creating an index affects the structure of how the data is stored on disk to
allow efficient access. Queries that use an index must use all the
attributes that the index applies to. An index creates a table of all the
distinct values for an attribute currently in the table. Each entry in the
index holds a list of pointers to all matching entries in the table. The index
does not have to be specified in order for it to be used to optimize a query.
After it is created, the database will use automatically for any matching
predicates.

\begin{center}
  You can find all my notes at \url{http://omgimanerd.tech/notes}. If you have
  any questions, comments, or concerns, please contact me at
  alvin@omgimanerd.tech
\end{center}

\end{document}
