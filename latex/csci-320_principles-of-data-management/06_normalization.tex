\documentclass{math}

\usepackage{listings}

\title{Principles of Data Management}
\author{Alvin Lin}
\date{August 2018 - December 2018}

\begin{document}

\lstset{basicstyle=\ttfamily\footnotesize,breaklines=true}
\maketitle

\section*{Normalization}
Normalization is a way to make bad data good by dealing with anomalies.
Anomalies occur when data is inserted, deleted, or updated in an inconsistent
way, causing rows to contain different data formats. This is usually solved
by decomposition. If many rows of a table are going to the same data, we can
decompose the table into two tables. For example, instead of:
\begin{lstlisting}
  Game(1, RDR2, M, \$59.99, RPG)
  Game(18, RDR2, M \$59.99, RPG)
\end{lstlisting}
We can decompose this into two tables and store RDR2 in its own Game table as
a template which we can update to reflect the change across all instances.
\begin{lstlisting}
  Game(RDR2, M \$59.99, RPG)
  GameInstance(18, RDR2)
  GameInstance(1, RDR2)
\end{lstlisting}
Decomposition can be either lossy or lossless, depending on whether or not the
original structure of the data can be recovered. In this case, the decomposition
is lossless since we can obtain the original records by joining the tables on
the game title.

\subsection*{Function Dependencies}
Certain fields can map to other fields. For example, ZIP codes can map to cities
and states. We can make certain attributes in a table dependent on other
attributes. Armstrong's Axioms:
\begin{itemize}
  \item if \( X\supseteq Y \) then \( X\to Y \) is reflexivity
  \item if \( X\to Y \) then \( XZ\to YZ \) is augmentation
  \item if \( X\to Y \) and \( Y\to Z \), then \( X\to Z \) is transitivity
\end{itemize}
We use Armstrong's Axioms to infer all functional dependencies and get the
minimal set of functional dependencies that include all possible attributes.
\begin{itemize}
  \item First normal form: all attributes are atomic.
  \item Second normal form: must be in first normal form, no partial key
    dependencies.
  \item Third normal form: must be in second normal form, no transitive
    dependencies.
  \item Boyce-Codd normal form (BCNF): stronger form of third normal form, no
    non-candidate key dependencies.
\end{itemize}

\subsection*{Denormalization}
In order to denormalize, you must have normalized data to start with. This is
usually performed as a method of combining tables. This reduces the number of
tables and reduces the number of necessary join operations, which are often
expensive. As a result, create, update, and delete operations are slowed down
but read operations are sped up. Redundant data will occur and there will be
less foreign keys.

\subsection*{Partitioning}
Horizontal and vertical partitioning is a form of potentially lossy
decomposition. Suppose we have a table with attributes:
\begin{center}
  \begin{tabular}{|c|c|c|c|c|}
    \hline
    ID & name & age & street & city \\
    \hline
  \end{tabular}
\end{center}
Vertical partitioning:
\begin{center}
  \begin{tabular}{|c|c|}
    \hline
    ID & name \\
    \hline
  \end{tabular} \quad
  \begin{tabular}{|c|c|c|c|}
    \hline
    ID & age & street & city \\
    \hline
  \end{tabular}
\end{center}
Horizontal partitioning involves putting the rows of a large database into
multiple databases. For example, users with \( age > 30 \) will be put into one
database and users with \( age \le 30 \) will be put into another database with
a view containing the union of both for access to all users. This is done for
faster access, especially since loading large tables into memory is expensive
and slow.

\begin{center}
  You can find all my notes at \url{http://omgimanerd.tech/notes}. If you have
  any questions, comments, or concerns, please contact me at
  alvin@omgimanerd.tech
\end{center}

\end{document}
