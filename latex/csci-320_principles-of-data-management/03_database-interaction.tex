\documentclass{math}

\usepackage{listing}

\title{Principles of Data Management}
\author{Alvin Lin}
\date{August 2018 - December 2018}

\begin{document}

\maketitle

\section*{Database Interaction}
SQL alone is not enough to interact with a database. It cannot display a GUI
since it has no non-declarative actions. There are a few options:
\begin{itemize}
  \item Embedded SQL
  \item Dynamic SQL: ORM/interface
\end{itemize}
We can connect to a database via a direct connection, Open Database Connection
(ODBC), or Java Database Connection (JDBC). Both ODBC and JDBC add a level of
abstraction by acting as a proxy between the application and the database.
While it may slow things down by adding a layer of complexity, it allows you to
switch between databases seamlessly. The application developer can interface
with the ODBC/JDBC in a database agnostic manner. ODBC started out as a
Windows-specific platform but it has grown to almost every platform. Because
these interfaces cater to the lowest common denominator, they do not support
database-specific functionality and are primarily used for simple databases.


\begin{center}
  You can find all my notes at \url{http://omgimanerd.tech/notes}. If you have
  any questions, comments, or concerns, please contact me at
  alvin@omgimanerd.tech
\end{center}

\end{document}
