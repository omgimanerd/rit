\documentclass{math}

\usepackage{forest}
\usepackage{listings}

\title{Principles of Data Management}
\author{Alvin Lin}
\date{August 2018 - December 2018}

\begin{document}

\lstset{basicstyle=\ttfamily\footnotesize,breaklines=true}
\maketitle

\section*{Indexes}
Indexes serve as search keys for content. There are ordered indices and hashed
indices, and since hashed indices function similar to a hash table, we will
simply cover ordered indices.

\subsection*{Evaluation Metrics}
\begin{itemize}
  \item Access types supported efficiently (records with attributes, range of
    values)
  \item Access time
  \item Insertion time
  \item Deletion time
  \item Space overhead
\end{itemize}

\subsection*{Types of Indices}
With an ordered index, search keys are sorted. With a \textbf{primary index},
records are sorted in sequential order of the file. This is also called a
clustering index and is specified by the primary key. With a \textbf{secondary
index}, an order is specified other than that of the primary. This is also
known as a non-clustering index. With \textbf{dense index files}, there exists
an index entry for every search key. In contrast, a \textbf{sparse index file}
only has index entries for some of the values in the table. \par
A secondary index will typically point to a primary index which has pointers to
values in the table. Thus, it must be a dense index because the values are not
ordered. There is some overhead on insertions and deletions since both primary
and secondary indices must both be updated. \par
Disks are slow, so what often happens in a database is that multilevel indices
are used. There is data stored in blocks on disk that we need to read, for
which access time is slow. An inner index will store information about different
blocks on disk. There may be one or more inner indices. A single outer index
will point to the inner indices to determine where data should be accessed.
The outer index would be stored in memory while the inner indices are stored
on disk like the data. \par
Indices managed on linear files generally degrade in performance as the index
grows due to the constant reordering needed as data is inserted. B+ trees
support local reorganization and have a fixed height and are much better for
index organization.

\subsection*{B+ Trees}
B+ trees have extra delection time, insertion time, and space overhead.
Generally for databases though, the advantages of a B+ tree far outweigh the
disadvantages.
\begin{center}
  \scalebox{0.8}{
  \begin{forest}
    [Mozart
      [Einstein|Gold
        [Brandt|Cali|Crick]
        [Einstein|ElSaid]
        [Gold|Katz|Kim]
      ]
      [Snu
        [Mozart|Singh]
        [Snu|Wu]
      ]
    ]
  \end{forest}}
\end{center}
Each of the elements in the leaf nodes point to sections of the data. They also
contain linear pointers to the next leaf node on the same level. The height of
the B+ tree is fixed. All paths from root to leaf are the same length.
\begin{itemize}
  \item Each node internal node has \( \lceil\frac{n}{2}\rceil \) to \( n \)
    children.
  \item Each leaf node has \( \lceil\frac{n-1}{2}\rceil \) to \( n-1 \) values.
  \item The root, if it is not a leaf, has at least 2 children.
  \item The root, if it is a leaf, has 0 to \( n-1 \) values.
\end{itemize}
Given \( k \) search keys, the tree height is
\( \lceil\log_{\lceil\frac{n}{2}\rceil}(k)\rceil \).

\subsubsection*{Node Insertion}
An insertion of the value \texttt{Adams} adds a new leaf node and propagates up
the tree.
\begin{center}
  \scalebox{0.8}{
  \begin{forest}
    [Mozart
      [Cali|Einstein|Gold
        [Adams|Brandt]
        [Cali|Crick]
        [Einstein|ElSaid]
        [Gold|Katz|Kim]
      ]
      [Snu
        [Mozart|Singh]
        [Snu|Wu]
      ]
    ]
  \end{forest}}
\end{center}
An insertion of the value \texttt{Lamport} adds a new leaf node, which adds a
new internal node.
\begin{center}
  \scalebox{0.8}{
  \begin{forest}
    [Mozart
      [Cali|Einstein
        [Adams|Brandt]
        [Cali|Crick]
        [Einstein|ElSaid]
      ]
      [Kim
        [Gold|Katz]
        [Kim|Lamport]
      ]
      [Snu
        [Mozart|Singh]
        [Snu|Wu]
      ]
    ]
  \end{forest}}
\end{center}
Deleting \texttt{Snu} from the rightmost leaf causes the leaf node to be merged
left. This makes the right internal node invalid, so it will merge left as well.
\begin{center}
  \scalebox{0.8}{
  \begin{forest}
    [Mozart
      [Cali|Einstein
        [Adams|Brandt]
        [Cali|Crick]
        [Einstein|ElSaid]
      ]
      [Kim|Mozart
        [Gold|Katz]
        [Kim|Lamport]
        [Mozart|Singh|Wu]
      ]
    ]
  \end{forest}}
\end{center}

\begin{center}
  You can find all my notes at \url{http://omgimanerd.tech/notes}. If you have
  any questions, comments, or concerns, please contact me at
  alvin@omgimanerd.tech
\end{center}

\end{document}
