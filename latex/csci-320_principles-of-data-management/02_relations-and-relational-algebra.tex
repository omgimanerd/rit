\documentclass{math}

\title{Principles of Data Management}
\author{Alvin Lin}
\date{August 2018 - December 2018}

\begin{document}

\maketitle

\section*{Relations and Relational Algebra}
\begin{center}
  \begin{tabular}{|c|c|c|c|}
    \hline
    id & name & DOB & height \\
    \hline
    1 & Jane & 12/1 & 5'8" \\
    \hline
    2 & Joe & 7/13 & null \\
    \hline
    &&& \\
    \hline
  \end{tabular}
\end{center}
In this table, the rows (tuples) and columns (attributes) contain data and
type information about the data. A database domain defines the attributes of the
database. It is atomic and every domain can have a null value. \par
Suppose \( A_1,A_2,\dots,A_n \) are attributes. A relation R is a subset of
domains. For example:
\[ Dog = (breed, color, good boy?, name, gender, weight, DOB, owner, address) \]
Relation items are unordered. An instance of a relation is defined by a table
and an element of the instance is a row.

\subsubsection*{Keys}
Keys are defined as a subset of attributes from a relation. The following keys
are all single relation keys.
\begin{itemize}
  \item super key: any combination of attributes that uniquely identify a row.
  \item candidate key: minimal super key
  \item primary key: one of the candidate keys
\end{itemize}
Foreign keys allow for referencing relations. A value for a given foreign key
attribute must appear in another table.

\subsection*{Relational Algebra}
\begin{itemize}
  \item Select Operation (\( \sigma \)): if we have a table R with the
    following attributes:
    \begin{center}
      \begin{tabular}{|c|c|c|c|}
        \hline
        A & B & C & D \\
        \hline
        x & x & 1 & 7 \\
        x & y & 5 & 7 \\
        y & y & 13 & 5 \\
        y & y & 30 & 4 \\
        \hline
      \end{tabular}
    \end{center}
    If we select on this relation using the condition A = B (denoted
    \( \sigma_{A=B}(R) \)), we get the last two rows.
    \[ \sigma_{A=B}(R) \Rightarrow
      \begin{tabular}{|c|c|c|c|}
        \hline
        A & B & C & D \\
        \hline
        x & x & 1 & 7 \\
        y & y & 13 & 5 \\
        y & y & 30 & 4 \\
        \hline
      \end{tabular}
    \]
    Any number of conditions
    can be used in a select operation.
  \item Projection (\( \Pi \)): limits the number of columns selected
    \begin{center}
      \begin{tabular}{|c|c|c|c|}
        \hline
        A & B & C \\
        \hline
        x & 1 & 0 \\
        y & 2 & 1 \\
        x & 3 & 0 \\
        y & 4 & 2 \\
        \hline
      \end{tabular}
    \end{center}
    \[ \Pi_{A,C}(R) \Rightarrow
      \begin{tabular}{|c|c|c|c|}
        \hline
        A & C \\
        \hline
        x & 0 \\
        y & 1 \\
        x & 0 \\
        y & 2 \\
        \hline
      \end{tabular}
    \]
    Because this has duplicates, this would reduce to:
    \begin{center}
      \begin{tabular}{|c|c|c|c|}
        \hline
        A & C \\
        \hline
        x & 0 \\
        y & 1 \\
        y & 2 \\
        \hline
      \end{tabular}
    \end{center}
  \item Union: merging two relations that have the same attributes
    \[ R = \begin{tabular}{|c|c|}
      \hline
      A & B \\
      \hline
      x & 1 \\
      y & 2 \\
      z & 3 \\
      \hline
    \end{tabular} \]
    \[ S = \begin{tabular}{|c|c|}
      \hline
      A & B \\
      \hline
      m & 5 \\
      n & 6 \\
      \hline
    \end{tabular} \]
    \[ T = \sigma_{B>5}(R\cup S) = \begin{tabular}{|c|c|}
      \hline
      A & B \\
      \hline
      x & 1 \\
      y & 2 \\
      z & 3 \\
      m & 5 \\
      n & 6 \\
      \hline
    \end{tabular} \]
\end{itemize}

\begin{center}
  You can find all my notes at \url{http://omgimanerd.tech/notes}. If you have
  any questions, comments, or concerns, please contact me at
  alvin@omgimanerd.tech
\end{center}

\end{document}
