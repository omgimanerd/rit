\documentclass{math}

\title{Principles of Data Management}
\author{Alvin Lin}
\date{August 2018 - December 2018}

\begin{document}

\maketitle

\section*{Relations and Relational Algebra}
\begin{center}
  \begin{tabular}{|c|c|c|c|}
    \hline
    id & name & DOB & height \\
    \hline
    1 & Jane & 12/1 & 5'8" \\
    \hline
    2 & Joe & 7/13 & null \\
    \hline
    &&& \\
    \hline
  \end{tabular}
\end{center}
In this table, the rows (tuples) and columns (attributes) contain data and
type information about the data. A database domain defines the attributes of the
database. It is atomic and every domain can have a null value. \par
Suppose \( A_1,A_2,\dots,A_n \) are attributes. A relation R is a subset of
domains. For example:
\[ Dog = (breed, color, good boy?, name, gender, weight, DOB, owner, address) \]
Relation items are unordered. An instance of a relation is defined by a table
and an element of the instance is a row.

\subsubsection*{Keys}
Keys are defined as a subset of attributes from a relation. The following keys
are all single relation keys.
\begin{itemize}
  \item super key: any combination of attributes that uniquely identify a row.
  \item candidate key: minimal super key
  \item primary key: one of the candidate keys
\end{itemize}
Foreign keys allow for referencing relations. A value for a given foreign key
attribute must appear in another table.

\subsection*{Relational Algebra}
\begin{itemize}
  \item Select Operation (\( \sigma \)): if we have a table R with the
    following attributes:
    \[ \begin{tabular}{|c|c|c|c|}
      \hline
      A & B & C & D \\
      \hline
      x & x & 1 & 7 \\
      x & y & 5 & 7 \\
      y & y & 13 & 5 \\
      y & y & 30 & 4 \\
      \hline
    \end{tabular} \]
    If we select on this relation using the condition A = B (denoted
    \( \sigma_{A=B}(R) \)), we get the last two rows.
    \[ \sigma_{A=B}(R) \Rightarrow \begin{tabular}{|c|c|c|c|}
      \hline
      A & B & C & D \\
      \hline
      x & x & 1 & 7 \\
      y & y & 13 & 5 \\
      y & y & 30 & 4 \\
      \hline
    \end{tabular} \]
    Any number of conditions
    can be used in a select operation.
  \item Projection (\( \Pi \)): limits the number of columns selected
    \[ \begin{tabular}{|c|c|c|c|}
      \hline
      A & B & C \\
      \hline
      x & 1 & 0 \\
      y & 2 & 1 \\
      x & 3 & 0 \\
      y & 4 & 2 \\
      \hline
    \end{tabular} \quad\Pi_{A,C}(R) \Rightarrow
    \begin{tabular}{|c|c|c|c|}
      \hline
      A & C \\
      \hline
      x & 0 \\
      y & 1 \\
      x & 0 \\
      y & 2 \\
      \hline
    \end{tabular} \]
    Because this has duplicates, this would reduce to:
    \[ \begin{tabular}{|c|c|c|c|}
      \hline
      A & C \\
      \hline
      x & 0 \\
      y & 1 \\
      y & 2 \\
      \hline
    \end{tabular} \]
  \item Union: merging two relations that have the same attributes
    \[ R = \begin{tabular}{|c|c|}
      \hline
      A & B \\
      \hline
      x & 1 \\
      y & 2 \\
      z & 3 \\
      \hline
    \end{tabular} \quad
    S = \begin{tabular}{|c|c|}
      \hline
      A & B \\
      \hline
      m & 5 \\
      n & 6 \\
      \hline
    \end{tabular} \quad
    T = \sigma_{B>5}(R\cup S) = \begin{tabular}{|c|c|}
      \hline
      A & B \\
      \hline
      x & 1 \\
      y & 2 \\
      z & 3 \\
      m & 5 \\
      n & 6 \\
      \hline
    \end{tabular} \]
  \item Set Difference: all tuples in \( R \) not in \( S \)
    \[ R = \begin{tabular}{|c|c|}
      \hline
      A & B \\
      \hline
      a & 1 \\
      a & 2 \\
      b & 1 \\
      \hline
    \end{tabular} \quad
    S = \begin{tabular}{|c|c|}
      \hline
      A & B \\
      \hline
      a & 2 \\
      a & 3 \\
      \hline
    \end{tabular} \quad
    R-S = \begin{tabular}{|c|c|}
      \hline
      A & B \\
      \hline
      a & 1 \\
      b & 1 \\
      \hline
    \end{tabular} \]
  \item Set Intersection: set of all tuples in both \( R \) and \( S \), can
    also be rewritten using set difference.
    \[ R\cup S = R-(R-S) \]
  \item Cross/Cartesian Product: maps everything in \( R \) to everything in
    \( S \).
    \[ R = \begin{tabular}{|c|c|}
      \hline
      A & B \\
      \hline
      a & 1 \\
      b & 1 \\
      \hline
    \end{tabular} \quad
    S = \begin{tabular}{|c|c|c|}
      \hline
      C & D & E \\
      \hline
      x & a & 1 \\
      y & c & 5 \\
      z & f & 7 \\
      \hline
    \end{tabular} \quad
    R\times S = \begin{tabular}{|c|c|c|c|c|}
      \hline
      A & B & C & D & E \\
      \hline
      a & 1 & x & a & 1 \\
      a & 1 & y & c & 5 \\
      a & 1 & z & f & 7 \\
      b & z & x & a & 1 \\
      b & z & y & c & 5 \\
      b & z & z & f & 7 \\
      \hline
    \end{tabular} \]
  \item Renaming: not really a real operation, but allows us to resolve
    attribute name conflicts.
    \[ R = \begin{tabular}{|c|c|}
      \hline
      A & B \\
      \hline
      1 & 2 \\
      3 & 4 \\
      \hline
    \end{tabular} \quad
    R\times P_S(R) = \begin{tabular}{|c|c|c|c|}
      \hline
      RA & RB & SA & SB \\
      \hline
      1 & 2 & 1 & 2 \\
      1 & 2 & 3 & 4 \\
      3 & 4 & 1 & 2 \\
      3 & 4 & 3 & 4 \\
      \hline
    \end{tabular} \]
  \item Natural Join: given two relations \( R \) and \( S \), where \( R \) and
    \( S \) have attributes with the same name for every tuple, if the values
    of those attributes have the same name, join them.
    \[ R = \begin{tabular}{|c|c|c|c|}
      \hline
      A & B & C & D \\
      \hline
      a & 1 & a & a \\
      b & 2 & c & a \\
      c & 4 & b & b \\
      a & 1 & c & a \\
      d & 2 & b & b \\
      \hline
    \end{tabular} \quad
    S = \begin{tabular}{|c|c|c|}
      \hline
      B & D & G \\
      \hline
      1 & a & a \\
      3 & a & b \\
      1 & a & c \\
      2 & b & d \\
      3 & b & e \\
      \hline
    \end{tabular} \quad
    R \Join S = \begin{tabular}{|c|c|c|c|c|}
      \hline
      A & B & C & D & E \\
      \hline
      a & 1 & a & a & a \\
      a & 1 & a & a & c \\
      a & 1 & c & a & a \\
      a & 1 & c & a & c \\
      d & 2 & b & b & d \\
      \hline
    \end{tabular} \]
    Note that this is logically equivalent to:
    \[ \Pi_{A,R.B,C,R.D,E}(\sigma_{R.B=S.B\wedge R.D=S.D}(R\times S)) \]
    but it cannot be generalized like set intersection because one needs to know
    the common attributes of both tables.
\end{itemize}
Relational algebra is not Turing-complete because it has no persistent storage,
and each statement always terminates with the output of a table.

\subsubsection*{Example}
Get the names and ages of all dogs who weight more than 15 pounds:
\[ \Pi_{name,age}(\sigma_{weight>15}(Dog)) \]
or in relational calculus:
\[ \{P\mid\exists D\in Dog(D.weight>15\wedge P.name=D.name\wedge
  P.age=D.age)\} \]

\begin{center}
  You can find all my notes at \url{http://omgimanerd.tech/notes}. If you have
  any questions, comments, or concerns, please contact me at
  alvin@omgimanerd.tech
\end{center}

\end{document}
