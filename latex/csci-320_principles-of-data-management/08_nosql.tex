\documentclass{math}

\usepackage{forest}
\usepackage{listings}

\title{Principles of Data Management}
\author{Alvin Lin}
\date{August 2018 - December 2018}

\begin{document}

\lstset{basicstyle=\ttfamily\footnotesize,breaklines=true}
\maketitle

\section*{NoSQL}
The term NoSQL refers to any non-relational non-row oriented database.
\begin{itemize}
  \item With column oriented databases, you can only query on certain columns
    (indexed), and you do not receive the data in the entire row. There are
    many implementations of column oriented databases that behave slightly
    differently, some may return a pointer to the row. Column oriented
    databases make it easy to retrieve attributes in a single column without
    the need to load unnecessary attributes.
  \item Another type of database is the key-value database, where values are
    stored based on a key. You can only query on the key, and operations may
    be extremely expensive.
  \item Document based databases are similar to key-value databases except
    that you can query on any attribute. You can also index on any attribute,
    thus adding performance to any query. Every document can also have different
    attributes, so it is up to the application inserting data to enforce
    attribute limitations. The most popular one is MongoDB.
\end{itemize}

\begin{center}
  You can find all my notes at \url{http://omgimanerd.tech/notes}. If you have
  any questions, comments, or concerns, please contact me at
  alvin@omgimanerd.tech
\end{center}

\end{document}
