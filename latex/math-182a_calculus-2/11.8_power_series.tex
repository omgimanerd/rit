\documentclass[letterpaper, 12pt]{article}
\usepackage{amsmath}
\usepackage{amssymb}

\renewcommand*{\arcsin}{\sin^{-1}}
\renewcommand*{\arccos}{\cos^{-1}}
\renewcommand*{\arctan}{\tan^{-1}}
\newcommand*{\arccot}{\cot^{-1}}
\newcommand*{\arcsec}{\sec^{-1}}
\newcommand*{\arccsc}{\csc^{-1}}
\newcommand*{\diff}{\mathrm{d}}
\newcommand*{\ddiff}[2]{\frac{\diff{#1}}{\diff{#2}}}
\newcommand*{\Diff}[1]{\mathrm{d^#1}}
\newcommand*{\e}{\mathrm{e}}

\title{Power Series}
\author{Alvin Lin}
\date{Calculus II: August 2016 - December 2016}

\begin{document}

\maketitle

\section*{Power Series}
A power series is any series that can be written in the form:
\[ \sum_{n=0}^{\infty}c_{n}(x-a)^{n} \]
where \( a \) is any number and \( c \) is a coefficient of the series.
There exists a number \( R \) such that the power series converges for
\( |x-a|< R \) and diverges for \( |x-a|>R \). This is called the radius
of convergence.
We are always guaranteed convergence for \( x = a \) because:
\[ \sum_{n=0}^{\infty}c_{n}(a-a)^{n} = \sum_{n=0}^{\infty}c_{n}(0)^{n} =
   c_{0}(0)^{n}+\sum_{n=1}^{\infty}c_{n}(0)^{n} = c_{0}+\sum_{n=1}^{\infty}0 =
   c_{0} \]
Let \( a = 0 \):
\[ \sum_{n=0}^{\infty}c_{n}x^{n} =
   c_{0}+c_{1}x+c_{2}x^{2}+...+c_{n}x^{n}+... \]
This can converge for a single value of \( x \), for multiple values of \( x \),
or for all \( x \).

\subsection*{Example}
\[ \sum_{n=1}^{\infty}\frac{x^{n}}{\sqrt{n}} \]
\begin{align*}
  a_{n} &= \frac{x^{n}}{n} \\
  a_{n+1} &= \frac{x^{n+1}}{\sqrt{n+1}} \\
  \lim_{n\to\infty}|\frac{a_{n+1}}{a_{n}}| &=
    \lim_{n\to\infty}|\frac{\frac{x^{n+1}}{\sqrt{n+1}}}{\frac{x^{n}}{n}}| \\
  &= \lim_{n\to\infty}|x|\frac{\sqrt{n+1}}{\sqrt{n}} \\
  &= \lim_{n\to\infty}|x|\sqrt{1+\frac{1}{n}} \\
  &= |x|
\end{align*}
\( \sum\frac{x^{n}}{\sqrt{n}} \) converges when \( |x| < 1 \). Thus (-1, 1)
is the radius of convergence.

\subsection*{Example 2}
\[ \sum_{n=1}^{\infty}\frac{x^{n}}{n!} \]
\begin{align*}
  a_{n} &= \frac{x^{n}}{n!} \\
  a_{n+1} &= \frac{x^{n+1}}{(n+1)!} \\
  \lim_{n\to\infty}|\frac{a_{n+1}}{a_{n}}| &=
    \lim_{n\to\infty}|\frac{\frac{x^{n+1}}{(n+1)!}}{\frac{x^{n}}{n!}}| \\
  &= \lim_{n\to\infty}|x|(\frac{n!}{(n+1)!}) \\
  &= \lim_{n\to\infty}|x|\frac{1}{n+1} \\
  &= 0
\end{align*}
This series converges for all \( x \in \mathbb{R} \). Thus the radius of
convergence is \( \infty \).

\subsection*{Practice Problem 4}
\[ \sum_{n=0}^{\infty}(-1)^{n}\frac{x^{n}}{n+1} \]
\begin{align*}
  a_{n} &= (-1)^{n}\frac{x^{n}}{n+1} \\
  a_{n+1} &= (-1)^{n+1}\frac{x^{n+1}}{n+2} \\
  \lim_{n\to\infty}|\frac{a_{n+1}}{a_{n}}| &=
    |\frac{(-1)^{n}\frac{x^{n}}{n+1}}{(-1)^{n+1}\frac{x^{n+1}}{n+2}}| \\
  &= \lim_{n\to\infty}|x|(\frac{n+1}{n+2}) \\
  &= \lim_{n\to\infty}|x|(\frac{n}{n})(\frac{1+\frac{1}{n}}{1+\frac{2}{n}}) \\
  &= |x|
\end{align*}
\( \sum\frac{x^{n}}{n+1} \) converges when \( |x| < 1 \). Thus (-1, 1)
is the radius of convergence.

\begin{center}
  If any errors are found, please contact me at alvin.lin.dev@gmail.com
\end{center}

\end{document}
