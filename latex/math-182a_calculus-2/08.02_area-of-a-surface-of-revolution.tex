\documentclass[letterpaper, 12pt]{math}

\usepackage{pgfplots}

\title{Area of a Surface of Revolution}
\author{Alvin Lin}
\date{Calculus II: August 2016 - December 2016}

\begin{document}

\maketitle

\section*{Area of a Surface of Revolution}
Finding the surface area of a solid of revolution follows a similar process as
finding its volume. Instead of integrating volumes of cross sections, we divide
the solid of revolution into frustums and use the arc length formula to
integrate the surface areas of the frustums.
\begin{center}
  \begin{tikzpicture}
    \begin{scope}
      \clip (-4,0) rectangle (4,2);
      \draw[dashed] (0,0) circle (4 and 1);
    \end{scope}
    \begin{scope}
      \clip (-4,0) rectangle (4,-2);
      \draw (0,0) circle (4 and 1);
    \end{scope}
    \draw (0,4) circle (2 and 0.5);
    \draw (-4,0) -- (-2,4);
    \draw (4,0) -- (2, 4);
    \draw[dashed] (4,0) -- node[above] {\( r_{1} \)}
                           coordinate[](a)(0,0)
                        -- node[left] {\( h \)}
                           coordinate[](b)(0,4)
                        -- node[above] {\( r_{2} \)}
                           coordinate[](c)(2,4)
                        -- node[right] {\( l \)}
                           coordinate[](d)(4,0);
  \end{tikzpicture}
\end{center}
\[ A = 2\pi r l \quad \mathrm{where} \quad r = \frac{r_{1}+r_{2}}{2} \]
If we have a function \( y=f(x) \) being revolved around the x-axis and split
into frustums, then:
\[ r_{1} = f(x_{i}) \]
\[ r_{2} = f(x_{i-1}) \]
\[ l = S \quad \mathrm(arc\ length) = \sqrt{1+(f'(x_{i}))^{2}}\Delta x \]
We can assume that \( \Delta x \) is infinitely small and \( f(x) \) is
continuous, thus both \( f(x_{i}) \) and \( f(x_{i-1}) \) converge to some
point \( f(x) \). Therefore:
\[ r = \frac{f(x_{i})+f(x_{i-1})}{2} \approx f(x) \]
\[ A = 2\pi f(x)\sqrt{1+(f'(x))^{2}}\Delta x \]
If we integrate to find the surface area of the entire solid, it follows that:
\begin{align*}
  A &= \int_{a}^{b}{2\pi f(x)\sqrt{1+(f'(x))^{2}}\diff{x}} \\
  &= \int_{a}^{b}{2\pi y\sqrt{1+(f'(x))^{2}}\diff{x}}
\end{align*}
This general form can also be modified for rotating a function about the y-axis
and also with a function in terms of y.

\subsection*{Rotating about the x and y axes}
About x-axis:
\[ S = \int{(2\pi y)\diff{S}} \]
About y-axis:
\[ S = \int{(2\pi x)\diff{S}} \]
If you have \( y = f(x) \quad a \leq x \leq b \):
\[ \diff{S} = \int_{a}^{b}{\sqrt{1+(\frac{\diff{y}}{\diff{x}})^{2}}\diff{x}} \]
If you have \( x = g(y) \quad c \leq y \leq d \):
\[ \diff{S} = \int_{c}^{d}{\sqrt{1+(\frac{\diff{x}}{\diff{y}})^{2}}\diff{y}} \]

\subsection*{Example 1}
\[ y = x^{4} \quad 0 \leq x \leq 1 \]
\begin{align*}
  S &= 2\pi\int_{0}^{1}{y\sqrt{1+(y')^{2}}\diff{x}} \\
  &= 2\pi\int_{0}^{1}{x^{4}\sqrt{1+16x^{6}}\diff{x}}
\end{align*}

\subsection*{Example 2}
\[ y = x^{3} \quad 0 \leq x \leq 2 \]
\begin{align*}
  S &= 2\pi\int_{0}^{2}{y\sqrt{1+(y')^{2}}\diff{x}} \\
  &= 2\pi\int_{0}^{2}{x^{3}\sqrt{1+9x^{4}}\diff{x}} \\
  & Let: u = 1+9x^{4} \quad \diff{u} = 36x^{3}\diff{x} \\
  &= 2\pi\int{\sqrt{u}\frac{\diff{u}}{36}} \\
  &= \frac{\pi}{18}\int{u^{\frac{1}{2}}\diff{u}} \\
  &= \frac{\pi}{18}\frac{2u^{\frac{3}{2}}}{3} \\
  &= \frac{\pi}{27}u^{\frac{3}{2}} \\
  &= \frac{\pi}{27}\bigg[(1+9x^{4})^{\frac{3}{2}}\bigg]_{0}^{2}
\end{align*}

\subsection*{Example 3}
\[ y = \arctan(x) \quad 0 \leq x \leq 1 \]
\[ \frac{\diff{y}}{\diff{x}} = \frac{1}{1+x^{2}} \]
\begin{center}
  Rotating about the x-axis:
\end{center}
\begin{align*}
  S &= \int_{0}^{1}{2\pi y\sqrt{1+(\frac{\diff{y}}{\diff{x}})^{2}}\diff{x}} \\
  &= \int_{0}^{1}{
    2\pi\arctan(x)(1+(\frac{1}{1+x^{2}})^{2})^{\frac{1}{2}}\diff{x}} \\
\end{align*}
\begin{center}
  Rotating about the y-axis:
\end{center}
\begin{align*}
  S &= \int_{0}^{1}{2\pi x\sqrt{1+(\diff{y}/\diff{x})^{2}}\diff{x}} \\
  &= \int_{0}^{1}{2\pi x(1+\frac{1}{(1+x^{2})^{2}})^{\frac{1}{2}}\diff{x}}
\end{align*}

\subsection*{Example 4}
Rotated about y-axis:
\[ f(x) = y = \frac{x^{3}}{6}+\frac{1}{2x} \quad \frac{1}{2} \leq x \leq 1 \]
\[ S = \int{2\pi y\diff{S}} \]
\[ S = \int_{\frac{1}{2}}^{1}{2\pi
  y\sqrt{1+(\frac{\diff{y}}{\diff{x}})^{2}}\diff{x}} \]
\[ y = \frac{x^{3}}{6}+\frac{1}{2x} \quad
   y' = \frac{3x^{2}}{6}-\frac{1}{2x^{2}} = \frac{x^{2}}{2}-\frac{1}{2x^{2}} \]
\begin{align*}
  1+(\frac{\diff{y}}{\diff{x}})^{2} &=
    1+(\frac{x^{2}}{2}-\frac{1}{2x^{2}})^{2} \\
  &= 1+\frac{x^{4}}{4}+\frac{1}{4x^{4}}-2(\frac{x^{2}}{2})(\frac{1}{2x^{2}}) \\
  &= 1+\frac{x^{4}}{4}+\frac{1}{4x^{4}}-\frac{1}{2} \\
  &= (\frac{x^{2}}{2})^{2}+(\frac{1}{2x^{2}})+\frac{1}{2} \\
  &= (\frac{x^{2}}{2})^{2}+(\frac{1}{2x^{2}})^{2}+
    2(\frac{x^{2}}{2})(\frac{1}{2x^{2}}) \\
  &= (\frac{x^{2}}{2}+\frac{1}{2x^{2}})^{2} \\
  \mathrm{Hence:} \\
  S &= \int_{\frac{1}{2}}^{1}{2\pi(\frac{x^{3}}{6}+\frac{1}{2x})
    (\frac{x^{2}}{2}+\frac{1}{2x^{2}})\diff{x}} \\
  &= 2\pi\int_{\frac{1}{2}}^{1}{
    \bigg[\frac{x^{5}}{12}+\frac{x}{12}+
      \frac{x}{4}+\frac{1}{4x^{3}}\bigg]\diff{x}} \\
  &= 2\pi\int_{\frac{1}{2}}^{1}{
    \bigg[\frac{x^{5}}{12}+\frac{x}{3}+\frac{1}{4x^{3}}\bigg]\diff{x}} \\
  &= 2\pi\bigg[\frac{x^{6}}{72}+\frac{x^{2}}{6}+
    \frac{1}{-8x^{2}}\bigg]_{\frac{1}{2}}^{1} \\
  &= \frac{263\pi}{256}
\end{align*}

\subsection*{Example 5}
Rotated about y-axis:
\[ x = \sqrt{a^{2}-y^{2}} \quad 0 \leq y \leq \frac{a}{2} \]
\[ S = \int{2\pi x\diff{S}} \]
\[ S = \int_{0}^{a/2}{
   \sqrt{a^{2}-y^{2}}\sqrt{1+(\frac{\diff{x}}{\diff{y}})^{2}}\diff{x}} \]
\[ x = \sqrt{a^{2}-y^{2}} \quad
   \frac{\diff{x}}{\diff{y}} = \frac{1}{2}(a^{2}-y^{2})^{\frac{-1}{2}}(-2y) \]
\[ \frac{\diff{x}}{\diff{y}} = \frac{-y}{\sqrt{a^{2}-y^{2}}} \]
\begin{align*}
  1+(\frac{\diff{y}}{\diff{x}})^{2} &=
    1+\frac{y^{2}}{a^{2}-y^{2}} \\
  &= \frac{a^{2}-y^{2}+y^{2}}{a^{2}-y^{2}} \\
  &= \frac{a^{2}}{a^{2}-y^{2}} \\
  \mathrm{Hence:} \\
  S &= \int_{0}^{a/2}{
    2\pi\sqrt{a^{2}-y^{2}}(\frac{a^{2}}{a^{2}-y^{2}})\diff{y}} \\
  &= \int_{0}^{a/2}{
    2\pi\sqrt{a^{2}-y^{2}}(\frac{\sqrt{a^{2}}}{\sqrt{a^{2}-y^{2}}})\diff{y}} \\
  &= \int_{0}^{a/2}{2\pi a\diff{y}} \\
  &= 2\pi a\bigg[y\bigg]_{0}^{a/2} \\
  &= 2\pi a(\frac{a}{2}-0) \\
  &= \pi a^{2}
\end{align*}

\begin{center}
  You can find all my notes at \url{http://omgimanerd.tech/notes}. If you have
  any questions, comments, or concerns, please contact me at
  alvin@omgimanerd.tech
\end{center}

\end{document}
