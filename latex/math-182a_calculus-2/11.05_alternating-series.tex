\documentclass[letterpaper, 12pt]{math}

\title{Alternating Series}
\author{Alvin Lin}
\date{Calculus II: August 2016 - December 2016}

\begin{document}

\maketitle

\section*{Alternating Series}
\[ \sum_{n=1}^{\infty}(-1)^{n}b_{n} = b_{1}-b_{2}+b_{3}-b_{4}... \]
If the following is true:
\begin{enumerate}
  \item \( b_{n+1} \leq b_{n} \quad \forall \quad n \)
  \item \( \lim_{n\to\infty}b_{n} \to 0 \)
\end{enumerate}
Then:
\[ \sum_{n=1}^{\infty}(-1)^{n}b_{n} \quad \mathrm{converges} \]

\subsection*{Example 1}
\[ \sum_{n=1}^{\infty}\frac{(-1)^{n-1}}{2n+1} \]
\[ b_{n} = \frac{1}{2n+1} \]
\[ b_{n+1} \leq b_{n} \quad \forall \quad n \]
\[ \lim_{n\to\infty}b_{n} = \lim_{n\to\infty}\frac{1}{2n+1} = 0 \]
Therefore:
\[ \sum_{n=1}^{\infty}\frac{(-1)^{n-1}}{2n+1} \quad \mathrm{converges} \]

\subsection*{Example 2}
\[ \sum_{n=1}^{\infty}\frac{(-1)^{n-1}}{\ln(n+4)} \]
\[ b_{n} = \frac{1}{\ln(n+4)} \]
\[ b_{n+1} \leq b_{n} \quad \forall \quad n \]
\[ \lim_{n\to\infty}\frac{1}{\ln(n+4)} = 0 \]
Therefore:
\[ \sum_{n=1}^{\infty}\frac{(-1)^{n-1}}{\ln(n+4)} \quad \mathrm{converges} \]

\subsection*{Example 3}
\[ \sum_{n=1}^{\infty}(-1)^{n-1}\frac{3n-1}{2n+1} \]
\[ b_{n} = \frac{3n-1}{2n+1} \]
\[ \lim_{n\to\infty}b_{n} = \frac{3}{2} \neq 0 \]
Therefore:
\[ \sum_{n=1}^{\infty}(-1)^{n-1}\frac{3n-1}{2n+1} \quad \mathrm{diverges} \]

\subsection*{Practice Problem 11}
\[ \sum_{n=1}^{\infty}(-1)^{n-1}\frac{n^{2}}{n^{3}+4} \]
\[ b_{n} = \frac{n^{2}}{n^{3}+4} \]
\[ b_{n+1} \leq b_{n} \quad \forall \quad n \]
\[ \lim_{n\to\infty}b_{n} = 0 \]
Therefore:
\[ \sum_{n=1}^{\infty}(-1)^{n-1}\frac{n^{2}}{n^{3}+4} \quad
   \mathrm{converges} \]

\subsection*{Left Comparison Test}
\[ \lim_{n\to\infty}\frac{a_{n}}{b_{n}} = c \]
\[ 0 < c < \infty \]
Therefore, both sequences either converge or diverge.

\begin{center}
  You can find all my notes at \url{http://omgimanerd.tech/notes}. If you have
  any questions, comments, or concerns, please contact me at
  alvin@omgimanerd.tech
\end{center}

\end{document}
