\documentclass[letterpaper, 12pt]{article}
\usepackage{amsmath}
\usepackage{amssymb}

\renewcommand*{\arcsin}{\sin^{-1}}
\renewcommand*{\arccos}{\cos^{-1}}
\renewcommand*{\arctan}{\tan^{-1}}
\newcommand*{\arccot}{\cot^{-1}}
\newcommand*{\arcsec}{\sec^{-1}}
\newcommand*{\arccsc}{\csc^{-1}}
\newcommand*{\diff}{\mathrm{d}}
\newcommand*{\ddiff}[2]{\frac{\diff{#1}}{\diff{#2}}}
\newcommand*{\Diff}[1]{\mathrm{d^#1}}
\newcommand*{\e}{\mathrm{e}}

\title{Absolute Convergence and the Ratio and Root Tests}
\author{Alvin Lin}
\date{Calculus II: August 2016 - December 2016}

\begin{document}

\maketitle

\section*{Absolute Convergence and the Ratio and Root Tests}
\textbf{Definition}: \( \sum{a_{n}} \) is absolutely convergent if
\( \sum|a_{n}| \) converges. If \( \sum|a_{n}| \) diverges but
\( \sum{a_{n}} \) converges, then \( \sum{a_{n}} \) is conditionally convergent.

\subsection*{The Ratio Test}
Let \( \sum{a_{n}} \) be given series.
\( \lim_{n\to\infty}|\frac{a_{n+1}}{a_{n}}| = L \)
\begin{enumerate}
  \item If \( L < 1 \), then \( \sum{a_{n}} \) converges absolutely.
  \item If \( L > 1 \), then \( \sum{a_{n}} \) diverges.
  \item If \( L = 1 \), then the test fails.
\end{enumerate}

\subsubsection*{Example 1}
\[ \sum\frac{n^{2}}{2^{n}} \]
\begin{align*}
  a_{n} &= \frac{n^{2}}{2^{n}} \\
  a_{n+1} &= \frac{(n+1)^{2}}{2^{n+1}} \\
  \lim_{n\to\infty}|\frac{a_{n+1}}{a_{n}}| &=
    \lim_{n\to\infty}\frac{\frac{(n+1)^{2}}{2^{n+1}}}{\frac{n^{2}}{2^{n}}} \\
  &= \lim_{n\to\infty}\frac{n^{2}+2n+1}{2n^{2}} \\
  &= \lim_{n\to\infty}(\frac{n^{2}}{n^{2}})
    (\frac{1+\frac{2}{n}+\frac{1}{n^{2}}}{2}) \\
  &= \frac{1}{2} \\
\end{align*}
Since \( \frac{1}{2} < 1 \), \( \sum{a_{n}} \) is absolutely convergent.

\subsubsection*{Example 2}
\[ \sum\frac{(-10)^{n}}{n!} \]
\begin{align*}
  a_{n} &= \frac{(-10)^{n}}{n!} \\
  a_{n+1} &= \frac{(-10)^{n+1}}{(n+1)!} \\
  \lim_{n\to\infty}|\frac{a_{n+1}}{a_{n}}| &=
    \lim_{n\to\infty}\frac{\frac{(-10)^{n+1}}{(n+1)!}}{\frac{(-10)^{n}}{n!}} \\
  &= \lim_{n\to\infty}\frac{-10}{n+1} \\
  &= 0
\end{align*}
Since \( 0 < 1 \), \( \sum{a_{n}} \) is absolutely convergent.

\subsubsection*{Example 3}
\[ \sum{a_{n}} = 1-\frac{1\times 3}{3!}+\frac{1\times 3\times 5}{5!}-
   \frac{1\times 3\times 5\times 7}{7!}+... \]
\begin{align*}
  a_{n} &= (-1)^{n-1}\frac{1\times 3\times 5\times...(2n-1)}{(2n-1)!} \\
  a_{n+1} &= (-1)^{n}\frac{1\times 3\times 5\times...(2n-1)\times(2n+1)}
    {(2n+1)!} \\
  \lim_{n\to\infty}|\frac{a_{n+1}}{a_{n}}| &=
    \frac{1\times 3\times 5\times...(2n-1)\times(2n+1)}{(2n+1)!}\times
    \frac{(2n-1)!}{1\times 3\times 5\times...(2n-1)} \\
  &= \lim_{n\to\infty}\frac{2n+1}{(2n)(2n+1)} \\
  &= \lim_{n\to\infty}\frac{1}{2n} \\
  &= 0
\end{align*}
Since \( 0 < 1 \), \( \sum{a_{n}} \) is absolutely convergent.

\subsection*{The Root Test}
Given \( \sum{a_{n}} \):
\[ \lim_{n\to\infty}\sqrt[n]{a_{n}} = L \]
\begin{enumerate}
  \item If \( L < 1 \), the series is absolutely convergent.
  \item If \( L > 1 \), the series diverges.
  \item If \( L = 1 \), the test fails.
\end{enumerate}

\subsubsection*{Example 1}
\[ \sum_{n=1}^{\infty}(\frac{n^{2}+1}{2n^{2}+1})^{n} \]
\begin{align*}
  a_{n} &= (\frac{n^{2}+1}{2n^{2}+1})^{n} \\
  \lim_{n\to\infty}\sqrt[n]{(\frac{n^{2}+1}{2n^{2}+1})^{n}} &=
    \lim_{n\to\infty}\frac{n^{2}+1}{2n^{2}+1} \\
  &= \lim_{n\to\infty}
    (\frac{n^{2}}{n^{2}})(\frac{1+\frac{1}{n^{2}}}{2+\frac{1}{n^{2}}}) \\
  &= \frac{1}{2}
\end{align*}
Since \( \frac{1}{2} < 1 \), \( \sum{a_{n}} \) is absolutely convergent.

\subsubsection*{Example 2}
\[ \sum_{n=1}^{\infty}(1+\frac{1}{n})^{n^{2}} \]
\begin{align*}
  a_{n} &= (1+\frac{1}{n})^{n^{2}} \\
  \lim_{n\to\infty}\sqrt[n]{(1+\frac{1}{n})^{n^{2}}} &=
    \lim_{n\to\infty}(1+\frac{1}{n})^{n} \\
  &= \e
\end{align*}
Since \( \e > 1 \), \( \sum{a_{n}} \) diverges.

\subsubsection*{Example 3}
\[ \sum_{n=1}^{\infty}(\frac{-2n}{n+2})^{7n} \]
\begin{align*}
  a_{n} &= (\frac{-2n}{n+2})^{7n} \\
  \lim_{n\to\infty}\sqrt[n]{(\frac{-2n}{n+2})^{7n}} &=
    \lim_{n\to\infty}(\frac{2n}{n+2})^{7} \\
  &= \lim_{n\to\infty}(\frac{n}{n})^{7}(\frac{2}{1+\frac{2}{n}})^{7} \\
  &= 2^{7}
\end{align*}
Since \( 2^{7} > 1 \), \( \sum{a_{n}} \) diverges.

\begin{center}
  If any errors are found, please contact me at alvin.lin.dev@gmail.com
\end{center}

\end{document}
