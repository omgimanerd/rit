\documentclass[letterpaper, 12pt]{article}
\usepackage{amsmath}
\usepackage{amssymb}

\renewcommand*{\arcsin}{\sin^{-1}}
\renewcommand*{\arccos}{\cos^{-1}}
\renewcommand*{\arctan}{\tan^{-1}}
\newcommand*{\arccot}{\cot^{-1}}
\newcommand*{\arcsec}{\sec^{-1}}
\newcommand*{\arccsc}{\csc^{-1}}
\newcommand*{\diff}{\mathrm{d}}
\newcommand*{\ddiff}[2]{\frac{\diff{#1}}{\diff{#2}}}
\newcommand*{\Diff}[1]{\mathrm{d^#1}}
\newcommand*{\e}{\mathrm{e}}

\title{Absolute Convergence and the Ratio and Root Tests}
\author{Alvin Lin}
\date{Calculus II: August 2016 - December 2016}

\begin{document}

\maketitle

\section*{Absolute Convergence and the Ratio and Root Tests}
\textbf{Definition}: \( \sum{a_{n}} \) is absolutely convergent if
\( \sum|a_{n}| \) converges. If \( \sum|a_{n}| \) diverges but
\( \sum{a_{n}} \) converges, then \( \sum{a_{n}} \) is conditionally convergent.

\subsection*{The Ratio Test}
Let \( \sum{a_{n}} \) be given series.
\( \lim_{n\to\infty}|\frac{a_{n+1}}{a_{n}}| = L \)
\begin{enumerate}
  \item If \( L < 1 \), then \( \sum{a_{n}} \) converges absolutely.
  \item If \( L > 1 \), then \( \sum{a_{n}} \) diverges.
  \item If \( L = 1 \), then the test fails.
\end{enumerate}

\subsection*{Example 1}
\[ \sum\frac{n^{2}}{2^{n}} \]
\[ a_{n} = \frac{n^{2}}{2^{n}} \]
\[ a_{n+1} = \frac{(n+1)^{2}}{2^{n+1}} \]
\[ |\frac{a_{n+1}}{a_{n}}| =
   \frac{\frac{(n+1)^{2}}{2^{n+1}}}{\frac{n^{2}}{2^{n}}} \]
\[ \lim_{n\to\infty}|\frac{a_{n}+1}{a_{n}}| = \frac{1}{2} \]
Since \( \frac{1}{2} < 1 \), \( \sum{a_{n}} \) is absolutely convergent.

\subsection*{Example 2}
\[ \sum\frac{(-10)^{n}}{n!} \]
\[ a_{n} = \frac{(-10)^{n}}{n!} \]
\[ a_{n+1} = \frac{(-10)^{n+1}}{(n+1)!} \]
\[ |\frac{a_{n+1}}{a_{n}}| =
   \frac{\frac{(-10)^{n+1}}{(n+1)!}}{\frac{(-10)^{n}}{n!}} \]
\[ \lim_{n\to\infty}|\frac{a_{n}+1}{a_{n}}| = 0 \]
Since \( 0 < 1 \), \( \sum{a_{n}} \) is absolutely convergent.

\begin{center}
  If any errors are found, please contact me at alvin.lin.dev@gmail.com
\end{center}

\end{document}
