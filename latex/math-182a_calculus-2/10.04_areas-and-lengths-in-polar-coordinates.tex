\documentclass[letterpaper, 12pt]{math}

\usepackage{pgfplots}



\title{Areas and Lengths in Polar Coordinates}
\author{Alvin Lin}
\date{Calculus II: August 2016 - December 2016}

\begin{document}

\maketitle

\section*{Areas and Lengths in Polar Coordinates}

\subsection*{Discrete Area}
\begin{center}
  \begin{tikzpicture}
    \draw (0,0) -- node[below] {\( r \)}
                   coordinate[](a)(4,2)
                -- node[right] {}
                   coordinate[](b)(2,4)
                -- node[above] {\( \bar{r} \)}
                   coordinate[](c)(0,0);
    \draw (0.5,0.1) node[label={\( \theta \)}]{};
  \end{tikzpicture}
\end{center}
\begin{align*}
  A_{i} &= \frac{1}{2}r^{2}\theta \\
  A &= \sum_{i=i}^{n}{A_{i}} \\
  &= \sum_{i=1}^{n}{\frac{1}{2}r^{2}\Delta\theta} \\
  &= \frac{1}{2}\int_{a}^{b}{r^{2}\diff{\theta}}
\end{align*}

\subsection*{Arc Length}
\[ L = \int_{a}^{b}{\sqrt{x^{2}+y^{2}}\diff{\theta}} \]
\[ r = f(\theta) \quad \mathrm{compared\ to\ } \quad y=f(x) \]
\[ x = r\cos(\theta) = f(\theta)\cos(\theta) \Longrightarrow
   \ddiff{x}{\theta} = -f(\theta)\sin(\theta)+f'(\theta)\cos(\theta) \]
\[ y = r\sin(\theta) = f(\theta)\sin(\theta) \Longrightarrow
   \ddiff{y}{\theta} = -f'(\theta)\sin(\theta)+f(\theta)\cos(\theta) \]
\begin{align*}
  x^{2}+y^{2} &= (-r\sin(\theta)+\ddiff{r}{\theta}\cos(\theta))^{2}
    +(\ddiff{r}{\theta}\sin(\theta)+r\cos(\theta))^{2} \\
  &= r^{2}\sin^{2}(\theta)+(\ddiff{r}{\theta})^{2}\cos^{2}(\theta)-
    2r\ddiff{r}{\theta}\sin(\theta)\cos(\theta)+ \\
  & \quad (\ddiff{r}{\theta})^{2}\sin^{2}(\theta)+r^{2}\cos^{2}(\theta)+
    2r\ddiff{r}{\theta}\sin(\theta)\cos(\theta) \\
  &= r^{2}(\cos^{2}(\theta)+\sin^{2}(\theta))+
    (\ddiff{r}{\theta})^{2}(\cos^{2}(\theta)+\sin^{2}(\theta)) \\
  &= r^{2}+(\ddiff{r}{\theta})^{2} \\
  L &= \int_{a}^{b}{\sqrt{r^{2}+(\ddiff{r}{\theta})^{2}}\diff{\theta}}
\end{align*}

\subsection*{Example 1}
\[ r = 2-\sin(\theta) \]
\begin{align*}
  A &= \frac{1}{2}\int_{0}^{2\pi}{r^{2}\diff{\theta}} \\
  &= \frac{1}{2}\int_{0}^{2\pi}{(2-\sin(\theta))^{2}\diff{\theta}} \\
  &= \frac{1}{2}\int_{0}^{2\pi}
    {4+\sin^{2}(\theta)-4\sin(\theta)\diff{\theta}} \\
  &= \frac{1}{2}\int_{0}^{2\pi}
    {4+(\frac{1-\cos(2\theta)}{2})-4\sin(\theta)\diff{\theta}} \\
  &= \frac{1}{2}\bigg[4\theta+\frac{\theta}{2}-
    \frac{\sin(2\theta)}{4}+4\cos(\theta)\bigg]_{0}^{2\pi} \\
  &= \frac{1}{2}\bigg[\frac{9}{2}{2\pi}-\frac{\sin{4\pi}}{4}+4\cos(2\pi)-
    (-\frac{\sin(\theta)}{4}+4\cos(\theta))\bigg] \\
  &= \frac{1}{2}\bigg[9\pi+4-4\bigg] \\
  &= \frac{9\pi}{2}
\end{align*}

\subsection*{Problem 30}
Find the integral for the area in common between the two functions:
\[ r = 1+\cos(\theta) \quad r = 1-\cos(\theta) \]
\begin{center}
  \begin{tikzpicture}
    \draw[thick,->,>=latex] (-4,0)--(4,0) node[above] {\( x \)};
    \draw[thick,->,>=latex] (0,-4)--(0,4) node[left] {\( y \)};
    \draw[domain=0:2*pi, samples=250] plot ({deg(\x)}:{1+cos(deg(\x))});
    \draw[domain=0:2*pi, samples=250] plot ({deg(\x)}:{1-cos(deg(\x))});
  \end{tikzpicture}
\end{center}
\[ A = (4)\frac{1}{2}\int_{0}^{\frac{\pi}{2}}
   {(1-\cos(\theta))^{2}\diff{\theta}} \]

\subsection*{Problem 35}
Find the arc length:
\[ r = 3\sin(\theta) \quad \theta\in\bigg[0,\frac{\pi}{3}\bigg] \]
\begin{center}
  \begin{tikzpicture}
    \draw[thick,->,>=latex] (-4,0)--(4,0) node[above] {\( x \)};
    \draw[thick,->,>=latex] (0,-4)--(0,4) node[left] {\( y \)};
    \draw[domain=0:pi/3, samples=250] plot ({deg(\x)}:{3*sin(deg(\x))});
  \end{tikzpicture}
\end{center}
\begin{align*}
  L &= \int_{a}^{b}{\sqrt{r^{2}+(\ddiff{r}{\theta})^{2}}\diff{\theta}} \\
  &= \int_{0}^{\frac{\pi}{3}}
    {\sqrt{(3\sin(\theta))^{2}+(3\cos(\theta))^{2}}\diff{\theta}} \\
  &= \int_{0}^{\frac{\pi}{3}}
    {\sqrt{9(sin^{2}(\theta)+\cos^{2}(\theta))}\diff{\theta}} \\
  &= \int_{0}^{\frac{\pi}{3}}{3\diff{\theta}} \\
  &= \bigg[3\theta\bigg]_{0}^{\frac{\pi}{3}} \\
  &= \pi
\end{align*}

\subsection*{Practice Problem 46}
\[ r = \e^{2\theta} \quad \theta\in\bigg[0,2\pi\bigg] \]
\begin{align*}
  L &= \int_{a}^{b}{\sqrt{r^{2}+(\ddiff{r}{\theta})^{2}}\diff{\theta}} \\
  &= \int_{0}^{2\pi}
    {\sqrt{(e^{2\theta})^{2}+(2\e^{2\theta})^{2}}\diff{\theta}} \\
  &= \int_{0}^{2\pi}{\sqrt{\e^{4\theta}+4\e^{4\theta}}\diff{\theta}} \\
  &= \int_{0}^{2\pi}{\e^{2\theta}\sqrt{5}\diff{\theta}} \\
  &= \bigg[\sqrt{5}\frac{e^{2\theta}}{2}\bigg]_{0}^{2\pi} \\
  &= \frac{\e^{4\pi}\sqrt{5}}{2}-\frac{\sqrt{5}}{2}
\end{align*}

\begin{center}
  You can find all my notes at \url{http://omgimanerd.tech/notes}. If you have
  any questions, comments, or concerns, please contact me at
  alvin@omgimanerd.tech
\end{center}

\end{document}
