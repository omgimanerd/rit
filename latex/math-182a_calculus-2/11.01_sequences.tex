\documentclass[letterpaper, 12pt]{article}
\usepackage{amsmath}
\usepackage{amssymb}
\usepackage{pgfplots}

\pgfplotsset{compat=1.10}

\renewcommand*{\arcsin}{\sin^{-1}}
\renewcommand*{\arccos}{\cos^{-1}}
\renewcommand*{\arctan}{\tan^{-1}}
\newcommand*{\arccot}{\cot^{-1}}
\newcommand*{\arcsec}{\sec^{-1}}
\newcommand*{\arccsc}{\csc^{-1}}
\newcommand*{\diff}{\mathrm{d}}
\newcommand*{\ddiff}[2]{\frac{\diff{#1}}{\diff{#2}}}
\newcommand*{\Diff}[1]{\mathrm{d^#1}}
\newcommand*{\e}{\mathrm{e}}

\title{Sequences}
\author{Alvin Lin}
\date{Calculus II: August 2016 - December 2016}

\begin{document}

\maketitle

\section*{Sequences}
A list of numbers written in a definite order. Example:
\[ \bigg\{1,19,11,12,19,13,...\frac{1}{11},19^{2}\bigg\} \]
\[ \bigg\{1,\frac{1}{2},\frac{1}{3},\frac{1}{4},...\bigg\} \]
Define a mapping:
\[ f:N \rightarrow R \]
\[ f(n) = a_{n} \]
\[ \bigg\{a_{1},a_{2},a_{3},a_{4},...a_{n}\bigg\} \]
\( a_{n} \) is the generic term where \( a_{n} = \frac{1}{n} \).

\subsubsection*{Definition}
A sequence \( \{a_{n}\} \) has the limit \( L \) if we can make the terms
as close to \( L \) as we want by choosing a sufficiently large \( n \). If
\( a_{n} \) has a limit \( L \), we say that \( \{a_{n}\} \) converges to
\( L \). We denote this by \( a_{n} \rightarrow L \). \par
If there is no real number \( L \) such that \( a_{n} \rightarrow L \),
then \( \{a_{n}\} \) diverges.
\newline
Example:
\[ a_{n} = \frac{1}{n} \]
\[ a_{n} \to 0 \]
\[ \lim_{n \to \infty}{a_{n}} = L \]
\begin{itemize}
  \item \( a_{n} \) is increasing iff
    \( a_{n}<a_{n+1} \quad \forall \quad n \).
  \item \( a_{n} \) is decreasing iff
    \( a_{n}>a_{n+1} \quad \forall \quad n \).
  \item \( a_{n} \) is monotonic if it is either increase or decreasing.
  \item \( a_{n} \) is bounded from above iff
    \( a_{n} \leq M \quad \forall \quad n \).
  \item \( a_{n} \) is bounded from below iff
    \( a_{n} \geq M \quad \forall \quad n \).
  \item \( r^{n} \) is convergent is \( -1 < r \leq 1 \) and divergent
    for all other \( r \)'s.
\end{itemize}

\subsubsection*{Identities}
\begin{itemize}
  \item \( \lim{ca_{n}} = c\lim{a_{n}} \)
  \item \( \lim{a_{n} \pm b_{n}} = \lim{a_{n}}+\lim{b_{n}} \)
  \item \( \lim{\frac{a_{n}}{b_{n}}} = \frac{\lim{a_{n}}}{\lim{b_{n}}} \)
  \item \( \lim{a_{n}b_{n}} = (\lim{a_{n}})(\lim{b_{n}}) \)
  \item \( \lim{((a_{n})^{p})} = (\lim{a_{n}})^{p} \quad (p>0,a_{n}>0) \)
\end{itemize}

\subsubsection*{Theorems}
\begin{itemize}
  \item If \( \lim_{x \to \infty}{f(x)} = L \), then \( a_{n} \rightarrow L \)
    where \( a_{n} = f(n) \).
  \item If \( \lim{|a_{n}|} = 0\), then \( \lim{a_{n}} = 0 \).
  \item Every bounded monotonic sequence is convergent. (Bounded from above and
    below). \( |a_{n}| < M \).
  \item The Squeeze Theorem:
    \[ a_{n} \leq b_{n} \leq c_{n} \quad n \geq n_{o} \]
    \[ \lim_{n\to\infty}{a_{n}} = \lim_{n\to\infty}{c_{n}} = L \]
    \[ \rightarrow \lim_{n\to\infty}{b_{n}} = L \]
\end{itemize}

\subsection*{Practice Problem 17}
\[ a_{n} = 1-(0.2)^{n} \]
\begin{align*}
  \lim_{n\to\infty}{a_{n}} &= \lim_{n\to\infty}{(1-(0.2)^{n})} \\
  &= 1-\lim_{n\to\infty}{(0.2)^{n}} \\
  &= 1-0 \\
  &= 1
\end{align*}
\[
  \lim_{n\to\infty}{r^{n}} =
  \begin{cases}
      0 & -1 \leq r \leq 1 \\
      1 & r = 1 \\
   \end{cases}
\]
Any limit of this form converges when \( -1 < r \leq 1 \) otherwise it
diverges.

\subsection*{Practice Problem 18}
\[ a_{n} = \frac{n^{3}}{n^{3}+1} \]
\begin{align*}
  \lim_{n\to\infty}{a_{n}} &= \lim_{n\to\infty}{\frac{n^{3}}{n^{3}+1}} \\
  &= \lim_{n\to\infty}
    {\frac{n^{3}}{n^{3}+1}\frac{\frac{1}{n^{3}}}{\frac{1}{n^{3}}}} \\
  &= \lim_{n\to\infty}{\frac{1}{1+\frac{1}{n^{3}}}} \\
  &= \frac{1}{1+0} \\
  &= 1
\end{align*}

\subsection*{Practice Problem 21}
\[ a_{n} = \e^{\frac{1}{n}} \]
\begin{align*}
  \lim_{n\to\infty}{a_{n}} &= \lim_{n\to\infty}{\e^{\frac{1}{n}}} \\
  &= \e^{\lim_{n\to\infty}{\frac{1}{n}}} \\
  &= \e^{0} \\
  &= 1
\end{align*}

\subsection*{Practice Problem 25}
\[ a_{n} = \frac{(-1)^{n}n}{n^{2}+1} \]
\begin{align*}
  \lim_{n\to\infty}{a_{n}} &= \lim_{n\to\infty}{\frac{(-1)^{n}n}{n^{2}+1}} \\
  \lim_{n\to\infty}{|a_{n}|} &= \lim_{n\to\infty}{\frac{n}{n^{2}+1}} \\
  &= \lim_{n\to\infty}
    {\frac{n}{n^{2}+1}\frac{\frac{1}{n^{2}}}{\frac{1}{n^{2}}}} \\
  &= \lim_{n\to\infty}
    {\frac{\frac{1}{n}}{1+\frac{1}{n^{2}}}} \\
  &= \frac{0}{1+0} \\
  &= 0
\end{align*}
\[ \lim_{n\to\infty}{|a_{n}|} = 0 \quad \therefore \quad
   \lim_{n\to\infty}{a_{n}} = 0 \]
\begin{center}
  (Theorem)
\end{center}

\subsection*{Practice Problem 29}
\[ a_{n} = \frac{(2n-1)!}{(2n+1)!} \]
\begin{align*}
  \lim_{n\to\infty}{a_{n}} &= \lim_{n\to\infty}{\frac{(2n-1)!}{(2n+1)!}} \\
  &= \lim_{n\to\infty}{\frac{(2n-1)!}{(2n+1)(2n)(2n-1)!}} \\
  &= \lim_{n\to\infty}{\frac{1}{(2n+1)(2n)}} \\
  &= 0
\end{align*}

\subsection*{Practice Problem 36}
\[ a_{n} = \ln(n+1)-\ln(n) \]
\begin{align*}
  \lim_{n\to\infty}{a_{n}} &= \lim_{n\to\infty}{\ln(n+1)-\ln(n)} \\
  &= \lim_{n\to\infty}{\ln(\frac{n+1}{n})} \\
  &= \lim_{n\to\infty}{\ln(1+\frac{1}{n})} \\
  &= \ln(1) \\
  &= 0
\end{align*}

\subsection*{Practice Problem 37}
\[ a_{n} = n\sin(\frac{1}{n}) \]
\begin{align*}
  \lim_{n\to\infty}{a_{n}} &= \lim_{x\to\infty}{x\sin(\frac{1}{x})} \\
  &= \lim_{x\to\infty}{\frac{\sin(\frac{1}{x})}{\frac{1}{x}}} \\
  &= \lim_{y\to0}{\frac{\sin(y)}{y}} \\
  & \mathrm{(L'Hopital's\ Rule)} \\
  &= \lim_{y\to0}{\frac{\cos(y)}{1}} \\
  &= \lim_{t\to0}{\cos(y)} \\
  &= 1
\end{align*}

\subsection*{Practice Problem 38}
\[ a_{n} = \sqrt[n]{2^{1+3n}} \]
\begin{align*}
  \lim_{n\to\infty}{a_{n}} &= \lim_{n\to\infty}{2^{\frac{1+3n}{n}}} \\
  &= \lim_{n\to\infty}{2^{\frac{1}{n}+3}} \\
  &= 2^{0+3} \\
  &= 8
\end{align*}

\subsection*{Practice Problem 39}
\[ a_{n} = (1+\frac{2}{n})^{n} \]
\[ \lim_{n\to\infty}{(1+\frac{2}{n})^{n}} \]
\[ y = y(x) = (1+\frac{2}{x})^{x} \]
\[ \ln(y) = x\ln(1+\frac{2}{x}) \]
\[ \ln(y) = \frac{\ln(1+\frac{2}{x})}{\frac{1}{x}} \]
By L'Hopital's Rule:
\[ \lim_{x\to\infty}{\ln(y)} =
   \lim_{x\to\infty}{\frac{\ln(1+\frac{2}{x})}{\frac{1}{x}}} = 2 \]
\[ \ln(y) = 2 \]
\[ y = \e^{2} \]
\[ \lim_{n\to\infty}{a_{n}} = \e^{2} \]

\subsection*{Practice Problem 40}
\[ a_{n} = \frac{\sin(2n)}{1+\sqrt{n}} \]
\[ 0 \leq a_{n} \leq \frac{1}{1+\sqrt{n}} \]
By the Squeeze Theorem, since 0 and \( \frac{1}{1+\sqrt{n}} \) converge to 0,
\( a_{n} \) must converge to 0.

\subsection*{Practice Problem 66}
\[ a_{n} = n+\frac{1}{n} \]
Either \( a_{n} \geq a_{n+1} \) or \( a_{n+1} \geq a_{n} \) for all n.
\[ f(x) = x+\frac{1}{x} \]
\[ f'(x) = 1+(-x^{2}) = 1-\frac{1}{x^{2}} \]
\[ f'(x) > 0 \]
We can conclude that \( a_{n} \) is increasing and has a lower bound of 0.
Thus, \( a_{n} \) diverges.

\subsection*{Practice Problem 68}
\[ a_{1} = \sqrt{2} \]
\[ a_{2} = \sqrt{2+\sqrt{2}} \]
\[ a_{3} = \sqrt{2+\sqrt{2+\sqrt{2}}} \]
\[ a_{n+1} = \sqrt{2+a_{n}} \]
\[ a_{n+2} = \sqrt{2+a_{n+1}} \]
By the Monotonic Convergence Theorem:
\[ \lim_{n\to\infty}{a_{n}} = \lim_{n\to\infty}{a_{n+1}} = L \]
\[ \lim_{n\to\infty}{a_{n+1}} = \sqrt{2+\lim_{n\to\infty}{a_{n}}} =
   \sqrt{2+L} \]
\[ L^{2} = 2+L \]
\[ L^{2}-L-2 = 0 \]
\[ (L-2)(L+1) = 0 \]
\[ L = 2 \quad L = -1 \]
We can eliminate -1 since we know the bounds of the sequence.
\[ \lim_{n\to\infty}{a_{n}} = 2 \]

\begin{center}
  If any errors are found, please contact me at alvin.lin.dev@gmail.com
\end{center}

\end{document}
