\documentclass[letterpaper, 12pt]{article}
\usepackage{amsmath}
\usepackage{amssymb}

\renewcommand*{\arcsin}{\sin^{-1}}
\renewcommand*{\arccos}{\cos^{-1}}
\renewcommand*{\arctan}{\tan^{-1}}
\newcommand*{\arccot}{\cot^{-1}}
\newcommand*{\arcsec}{\sec^{-1}}
\newcommand*{\arccsc}{\csc^{-1}}
\newcommand*{\diff}{\mathrm{d}}
\newcommand*{\ddiff}[2]{\frac{\diff{#1}}{\diff{#2}}}
\newcommand*{\Diff}[1]{\mathrm{d^#1}}
\newcommand*{\e}{\mathrm{e}}

\title{Taylor and Maclaurin Series}
\author{Alvin Lin}
\date{Calculus II: August 2016 - December 2016}

\begin{document}

\maketitle

\section*{Taylor and Maclaurin Series}
Here is a function \( f(x) \) that can be represented as a power series:
\begin{align*}
  f(x) &= \sum_{n=0}^{\infty}c_{n}(x-a)^{n} \\
  &= c_{0}+c_{1}(x-a)+c_{2}(x-a)^{2}+c_{3}(x-a)^{3}+c_{4}(x-a)^{4}+...
\end{align*}
If we assume that the derivatives of \( f(x) \) in every order exist, then we
can solve for the coefficients \( c_{n} \). For example:
\[ f(a) = c_{0} \]
Using the first derivative:
\begin{align*}
  f'(x) &= c_{1}+2c_{2}(x-a)+3c_{3}(x-a)^{2}+4c_{4}(x-a)^{3} \\
  f'(a) &= c_{1}
\end{align*}
Using the second derivative:
\begin{align*}
  f''(x) &= 2c_{2}+(3)(2)c_{3}(x-a)+(4)(3)(x-a)^{2}+(5)(4)(x-a)^{3} \\
  f''(a) &= 2c_{2} \\
  c_{2} &= \frac{f''(a)}{2}
\end{align*}
Using the third derivative:
\begin{align*}
  f'''(x) &= (3)(2)(1)c_{3}+(4)(3)(2)c_{4}(x-a)+(5)(4)(3)c_{5}(x-a) \\
  f'''(a) &= (3)(2)c_{3} \\
  c_{3} &= \frac{f'''(a)}{(3)(2)}
\end{align*}
It follows that a general form of the coefficient \( c_{n} \) is:
\[ c_{n} = \frac{f^{(n)}(a)}{n!} \]
Provided a power series representation for the function \( f(x) \) about \( x = a \) exists, the Taylor
Series for \( f(x) \) about \( x = a \) is:
\begin{align*}
  f(x) &= \sum_{n=0}^{\infty}\frac{f^{(n)}(a)}{n!}(x-a)^{n} \\
  &= f(a)+f'(a)(x-a)+\frac{f'(a)(x-a)^{2}}{2!}+\frac{f''(a)(x-a)^{3}}{3!}+...
\end{align*}
If we use \( a = 0 \), then the Taylor Series is known as the Maclaurin Series
for \( f(x) \):
\begin{align*}
  f(x) &= \sum_{n=0}^{\infty}\frac{f^{(n)}(0)}{n!}x^{n} \\
  &= f(0)+f'(0)x+\frac{f''(0)x^{2}}{2!}+\frac{f'''(0)x^{3}}{3!}+...
\end{align*}

\subsection*{Example 1}
Find the Taylor Series for \( f(x)= \e^{x} \) about \( x = 0 \):
\[ f(x) = \e^{x} = \sum_{n=0}^{\infty}\frac{f^{(n)}(0)}{n!}x^{n} \]
\begin{align*}
  f^{(n)}(x) &= \e^{x} \\
  f^{(n)}(0) &= \e^{x} \\
  f(x) &= \sum_{n=0}^{\infty}\frac{x^{n}}{n!} \\
  &= 1+x+\frac{x^{2}}{2!}+\frac{x^{3}}{3!}+...
\end{align*}
We can actually derive the value of \( \e \) by taking \( f(0) \):
\[ f(0) = \e = 1+1+\frac{1}{2}+\frac{1}{6}+... \]

\subsection*{Example 2}
\begin{align*}
  \sin(x) &= f(0)+\frac{f'(0)}{1!}x+\frac{f''(0)}{2!}x^{2}+
    \frac{f'''(0)}{3!}x^{3}+... \\
  &= 0+x-\frac{x^{3}}{3!}+\frac{x^{5}}{5!}-\frac{x^{7}}{7!}+... \\
  \cos(x) &= f(0)+\frac{f'(0)}{1!}x+\frac{f''(0)}{2!}x^{2}+
    \frac{f'''(0)}{3!}x^{3}+... \\
  &= 1-\frac{x^{2}}{2!}+\frac{x^{4}}{4!}-\frac{x^{6}}{6!}+... \\
  \e^{ix} &= \cos(x)+i\sin(x)
\end{align*}

\subsection*{Binomial Series}
\[ (1+x)^{k} \quad |x| < 1 \]
\begin{align*}
  (1+x)^{k} &= \sum_{n=0}^{\infty}c_{n}x^{n} \\
  &= \sum_{n=0}^{\infty}\binom{k}{n}x \\
  &= 1+\frac{kx}{1!}+\frac{k(k-1)x^{2}}{2!}+\frac{k(k-1)(k-2)x^{3}}{3!}+
    \frac{k(k-1)(k-2)(k-3)x^{4}}{4!}+... \\
  & +\frac{k(k-1)...(k-(n-1))x^{n}}{n!}
\end{align*}
\[ \frac{1}{1+x} = 1-x+x^{2}-x^{3}+... \]
\[ \frac{1}{1-x} = 1+x+x^{2}+x^{3}+... \]

\subsection*{Example 3}
\[ f(x) = x^{4}-3x^{2}+1 \]
At \( n = 0 \):
\[ c_{n} = f(1) = -1 \]
At \( n = 1 \):
\[ c_{n} = f'(1) = -2 \]
At \( n = 2 \):
\[ c_{n} = f''(1) = 6 \]
At \( n = 3 \):
\[ c_{n} = f'''(1) = 24 \]
At \( n = 4 \):
\[ c_{n} = f^{(4)}(1) = 24 \]
\[ f(x) = c_{0}+c_{1}(x-a)+c_{2}(x-a)^{2}+c_{3}(c-a)^{3}+... \]
\[ = -1+(-2)(x-a)+6(x-a)^{2}+24(x-a)^{3}+... \]

\begin{center}
  If any errors are found, please contact me at alvin.lin.dev@gmail.com
\end{center}

\end{document}
