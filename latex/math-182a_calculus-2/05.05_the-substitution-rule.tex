\documentclass{math}

\title{The Substitution Rule}
\author{Alvin Lin}
\date{Calculus II: August 2016 - December 2016}

\begin{document}

\maketitle

\section*{The Substitution Rule}
\[ \int{\sin(t)\sqrt{1+\cos(t)}\diff{t}} \]
Substitute unsolvable terms with u to reduce the equation to something easier
to integrate.
\[ Let: \quad u = 1+\cos(t) \]
\[ \diff{u} = -\sin(t)\diff{t} \]
\[ \diff{t} = \frac{-\diff{u}}{\sin(t)} \]
Substitute your new terms back into the original equation. The goal is to get
a friendlier integral in terms of another variable. If the new integral has
both \( x \) and \( u \) in it, then it may be advisable to try a different method
of integration or substituting a different term.
\[ \int{\sin(t)\sqrt{u}(\frac{-\diff{u}}{\sin(t)})} \]
One we simplify the integral after substituting everything, it becomes much
easier to integrate.
\[ -\int{u^{\frac{1}{2}}\diff{u}} \]
\[ -(\frac{u^{\frac{3}{2}}}{\frac{3}{2}})+C \]
\[ = -\frac{2}{3}(1+\cos(t))^{\frac{3}{2}}+C \]

\subsection*{Practice problem 30}
\[ \int{\frac{sec^{2}(x)}{\tan^{2}(x)}\diff{x}} \]
\[ Let: \quad u = \tan(x) \]
\[ \diff{u} = sec^{2}(x)\diff{x} \]
\[ \int{\frac{\diff{u}}{u^{2}}} \]
\[ \int{u^{-2}\diff{u}} \]
\[ \frac{u^{-1}}{-1}+C \]
\[ = -\frac{1}{\tan(x)}+C \]

\subsection*{Practice problem 31}
\[ \int{\frac{\arctan(x^{2})}{1+x^{2}}\diff{x}} \]
\[ Let: \quad u = \arctan(x) \]
\[ \diff{u} = \frac{1}{1+x^{2}}\diff{x} \]
\[ \int{u^{2}\diff{u}} \]
\[ = \frac{u^{2}}{3}+C \]

\subsection*{Practice problem 42}
\[ \int{\frac{\cos(ln(t))}{t}\diff{t}} \]
\[ Let: \quad u = ln(t) \]
\[ \diff{u} = \frac{1}{t}\diff{t} \]
\[ \int{\cos(u)\diff{u}} \]
\[ = \sin(u)+C \]

\begin{center}
  You can find all my notes at \url{http://omgimanerd.tech/notes}. If you have
  any questions, comments, or concerns, please contact me at
  alvin@omgimanerd.tech
\end{center}

\end{document}
