\documentclass[letterpaper, 12pt]{article}

\renewcommand*{\arcsin}{\sin^{-1}}
\renewcommand*{\arccos}{\cos^{-1}}
\renewcommand*{\arctan}{\tan^{-1}}
\newcommand*{\arccot}{\cot^{-1}}
\newcommand*{\arcsec}{\sec^{-1}}
\newcommand*{\arccsc}{\csc^{-1}}
\newcommand*{\diff}{\mathrm{d}}
\newcommand*{\Diff}[1]{\mathrm{d^#1}}
\newcommand*{\e}{\mathrm{e}}

\title{Average Value of a Function}
\author{Alvin Lin}
\date{Calculus II: August 2016 - December 2016}

\begin{document}

\maketitle

\section*{Average Value of a Function}
Recall the notion of averaging. Given:
\[ x_{1}, x_{2}, x_{3} \]
\[ x_{average} = \frac{x_{1}+x_{2}+x_{3}}{3} \]
The average value of a function from \( x = a \) to \( x = b \) is:
\[ f_{av} = \frac{1}{b-a}\int_{a}^{b}{f(x)\diff{x}} \]
This is a continuous version of the standard average formula, which uses
discrete points instead.

\subsection*{The Mean Value Theorem of Integrals}
If \( f \) is continuous on [a, b], then there exists \( c \) in [a, b] such
that
\[ f(c) = f_{av} = \frac{1}{b-a}\int_{a}^{b}{f(x)\diff{x}} \]
Therefore:
\[ \int_{a}^{b}{f(x)\diff{x}} = (b-a)f(c) \]

\subsection*{Practice Problem 9}
\[ f(x) = (x-3)^{2} \quad [2, 5] \]
\[ f_{av} = \frac{1}{b-a}\int_{a}^{b}{f(x)\diff{x}} \]
\[ \frac{1}{5-2}\int_{2}^{5}{(x-3)^{2}\diff{x}} \]
\[ \frac{1}{3}\bigg[\frac{(x-3)^{3}}{3}\bigg]_{2}^{5} \]
\[ \frac{1}{9}\bigg[(x-3)^{3}]_{2}^{5} \]
\[ = \frac{1}{9}(8+1) = 1 \]
\[ f(c) = f_{av} = 1 \]
\[ (c-3)^{3} = 1 \]
\[ c^{2}-6c+9 = 1 \]
\[ c^{2}-6c+8 = 1 \]
\[ (c-2)(c-4) = 1 \]

\begin{center}
  If any errors are found, please contact me at alvin.lin.dev@gmail.com
\end{center}

\end{document}
