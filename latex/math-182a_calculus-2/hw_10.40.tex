\documentclass[letterpaper, 12pt]{math}

\usepackage{pgfplots}

\geometry{letterpaper, margin=1in}

\title{Section 10.4}
\author{Alvin Lin}
\date{Calculus II: August 2016 - December 2016}

\begin{document}

\maketitle

\section*{Exercise 5}
\[ r^{2} = \sin(2\theta) \]
\begin{align*}
  A &= \frac{1}{2}\int_{a}^{b}{r^{2}\diff{\theta}} \\
  &= \frac{1}{2}\int_{0}^{\pi/2}{\sin(2\theta)\diff{\theta}} \\
  &= \frac{1}{2}\bigg[\frac{-\cos(2\theta)}{2}\bigg]_{0}^{\pi/2} \\
  &= \frac{1}{2}
\end{align*}

\section*{Exercise 7}
\[ r = 4+3\sin(\theta) \]
\begin{align*}
  A &= \frac{1}{2}\int_{a}^{b}{r^{2}\diff{\theta}} \\
  &= \frac{1}{2}\int_{-\pi/2}^{\pi/2}{(4+3\sin(\theta))^{2}\diff{\theta}} \\
  &= \frac{1}{2}\int_{-\pi/2}^{\pi/2}
    {9\sin^{2}(\theta)+24\sin(\theta)+16\diff{\theta}} \\
  &= \frac{1}{2}\int_{-\pi/2}^{\pi/2}
    {\frac{9(1-\cos(2\theta))}{2}+24\sin(\theta)+16\diff{\theta}} \\
  &= \frac{1}{2}\int_{-\pi/2}^{\pi/2}
    {\frac{-\cos(2\theta)}{2}+24\sin(\theta)+\frac{41}{2}\diff{\theta}} \\
  &= \frac{1}{2}\bigg[
    \frac{-9\sin(2\theta)}{4}-24\cos(\theta)+
      \frac{41}{2}\theta\bigg]_{-\pi/2}^{\pi/2}
  &= \frac{41\pi}{2}
\end{align*}

\section*{Exercise 17}
\[ r = 4\cos(3\theta) \]
\begin{center}
  \begin{tikzpicture}
    \draw[thick,->,>=latex] (-5,0)--(5,0) node[above] {\( x \)};
    \draw[thick,->,>=latex] (0,-5)--(0,5) node[left] {\( y \)};
    \draw[domain=0:2*pi, samples=250] plot ({deg(\x)}:{4*cos(3*deg(\x))});
  \end{tikzpicture}
\end{center}
\begin{align*}
  A &= \frac{1}{2}\int_{a}^{b}{r^{2}\diff{\theta}} \\
  &= \frac{1}{2}\int_{-\pi/6}^{\pi/6}{(4\cos(3\theta))^{2}\diff{\theta}} \\
  &= 8\int_{-\pi/6}^{\pi/6}{\cos^{2}(3\theta)\diff{\theta}} \\
  &= 8\int_{-\pi/6}^{\pi/6}{\frac{1+\cos(6\theta)}{2}\diff{\theta}} \\
  &= 4\int_{-\pi/6}^{\pi/6}{1+\cos(6\theta)\diff{\theta}} \\
  &= 4\bigg[\theta+\frac{\sin(6\theta)}{6}]_{-\pi/6}^{\pi/6} \\
  &= \frac{4\pi}{3}
\end{align*}

\section*{Exercise 23}
\[ r_{1} = 2 \quad r_{2} = 4\sin(\theta) \]
\begin{center}
  \begin{tikzpicture}
    \draw[thick,->,>=latex] (-5,0)--(5,0) node[above] {\( x \)};
    \draw[thick,->,>=latex] (0,-5)--(0,5) node[left] {\( y \)};
    \draw[domain=0:2*pi, samples=250] plot ({deg(\x)}:{4*sin(deg(\x))});
    \draw[domain=0:2*pi, samples=250] plot ({deg(\x)}:{2});
  \end{tikzpicture}
\end{center}
\begin{align*}
  A &= \frac{1}{2}\int_{a}^{b}{(r_{2})^{2}-(r_{1})^{2}\diff{\theta}} \\
  &= \frac{1}{2}\int_{\pi/6}^{5\pi/6}{(4\sin(\theta))^{2}-2^{2}\diff{\theta}} \\
  &= \frac{1}{2}\int_{\pi/6}^{5\pi/6}{16\sin^{2}(\theta)-4\diff{\theta}} \\
  &= 2\int_{\pi/6}^{5\pi/6}{4\sin^{2}(\theta)-1\diff{\theta}} \\
  &= 2\int_{\pi/6}^{5\pi/6}{4\sin^{2}(\theta)-2+1\diff{\theta}} \\
  &= 2\int_{\pi/6}^{5\pi/6}{2(2\sin^{2}(\theta)-1)+1\diff{\theta}} \\
  &= 2\int_{\pi/6}^{5\pi/6}{-2\cos(2\theta)+1\diff{\theta}} \\
  &= 2\bigg[-\sin(2\theta)+\theta\bigg]_{\pi/6}^{5\pi/5} \\
  &= 2\bigg(\big[-\sin(\frac{5\pi}{3})+\frac{5\pi}{6}\big]-
    \big[-\sin(\frac{\pi}{3})+\frac{\pi}{6}\big]\bigg) \\
  &= 2\sqrt{3}+\frac{4\pi}{3}
\end{align*}

\section*{Exercise 29}
\[ r = 3\cos(\theta) \quad r = 3\sin(\theta) \]
\begin{center}
  \begin{tikzpicture}
    \draw[thick,->,>=latex] (-4,0)--(4,0) node[above] {\( x \)};
    \draw[thick,->,>=latex] (0,-4)--(0,4) node[left] {\( y \)};
    \draw[domain=0:2*pi, samples=250] plot ({deg(\x)}:{3*sin(deg(\x))});
    \draw[domain=0:2*pi, samples=250] plot ({deg(\x)}:{3*cos(deg(\x))});
  \end{tikzpicture}
\end{center}
\begin{align*}
  A &= \frac{1}{2}\int_{a}^{b}{r^{2}\diff{\theta}} \\
  &= \frac{1}{2}(2)\int_{0}^{\pi/4}{(3\sin(\theta))^{2}\diff{\theta}} \\
  &= \int_{0}^{\pi/4}{9\sin^{2}(\theta)\diff{\theta}} \\
  &= \int_{0}^{\pi/4}{9(\frac{1-\cos(2\theta)}{2})\diff{\theta}} \\
  &= \frac{9}{2}\int_{0}^{\pi/4}{1-\cos(2\theta)\diff{\theta}} \\
  &= \frac{9}{2}\bigg[\theta-\frac{\sin(2\theta)}{2}\bigg]_{0}^{\pi/4} \\
  &= \frac{9}{2}\bigg(\big[\frac{\pi}{4}-\frac{\sin(\frac{\pi}{2})}{2}\big]-
    \big[0-\frac{\sin(0)}{2}\big]\bigg) \\
  &= \frac{9\pi}{8}-\frac{9}{4}
\end{align*}

\section*{Exercise 45}
\[ r = 2\cos(\theta) \quad 0 \leq \theta \leq \pi \]
\begin{align*}
  \ddiff{r}{\theta} &= -2\sin(\theta) \\
  L &= \int_{a}^{b}{\sqrt{r^{2}+(\ddiff{r}{\theta})^{2}}\diff{\theta}} \\
  &= \int_{0}^{\pi}
    {\sqrt{(2\cos(\theta))^{2}+(-2\sin(\theta))^{2}}\diff{\theta}} \\
  &= \int_{0}^{\pi}
    {\sqrt{4\cos^{2}(\theta)+4\sin^{2}(\theta)}\diff{\theta}} \\
  &= \int_{0}^{\pi}
    {\sqrt{4(\cos^{2}(\theta)+\sin^{2}(\theta))}\diff{\theta}} \\
  &= \int_{0}^{\pi}{\sqrt{4}\diff{\theta}} \\
  &= \int_{0}^{\pi}{2\diff{\theta}} \\
  &= \bigg[2\theta\bigg]_{0}^{\pi} \\
  &= 2\pi
\end{align*}

\begin{center}
  If you have any questions, comments, or concerns, please contact me at
  alvin@omgimanerd.tech
\end{center}

\end{document}
