\documentclass[letterpaper, 12pt]{math}

\usepackage{listings}

\geometry{letterpaper, margin=1in}

\title{Section 7.7}
\author{Alvin Lin}
\date{Calculus II: August 2016 - December 2016}

\begin{document}

\maketitle

\section*{Note}
The following calculations were done with a Python program I wrote, available
here:
\newline
\url{https://github.com/omgimanerd/experimental/blob/master/calc.py}

\subsection*{Source code:}
\begin{lstlisting}
#!/usr/bin/python

from numpy import arange, linspace

def delta_x(a, b, n):
    return (b - a) / n

def left_riemann_sum(f, a, b, n):
    interval = list(linspace(a, b, n, endpoint=False))
    return (interval, delta_x(a, b, n) * sum([f(x) for x in interval]))

def right_riemann_sum(f, a, b, n):
    interval = list(linspace(a, b, n + 1)[1:])
    return (interval, delta_x(a, b, n) * sum([f(x) for x in interval]))

def midpoint_approximation(f, a, b, n):
    delta = delta_x(a, b, n)
    interval = list(arange(a + (delta / 2), b, delta))
    return (interval, delta * sum([f(x) for x in interval]))

def trapezoidal_approximation(f, a, b, n):
    interval = list(linspace(a, b, n + 1))
    return (interval,
            (left_riemann_sum(f, a, b, n)[1] + right_riemann_sum(f, a, b, n)[1]) / 2)

def simpsons_rule_approximation(f, a, b, n):
    series = linspace(a, b, n + 1)
    interval = list(linspace(a, b, n + 1))
    for index, term in enumerate(series):
        if index == 0 or index == len(series) - 1:
            series[index] = f(term)
        elif not(index % 2): # even
            series[index] = 2 * f(term)
        else:
            series[index] = 4 * f(term)
    return (interval, (delta_x(a, b, n) / 3) * sum(series))

if __name__ == '__main__':
    from math import *
    exec("f = lambda x: {}".format(input("Input your function: lambda x: ")))
    a = int(input("Lower limit: "))
    b = int(input("Upper limit: "))
    n = int(input("n: "))
    print("-----")
    print("Interval: {0}\nLeft Riemann Sum Approximation: {1}\n".format(
        *left_riemann_sum(f, a, b, n)))
    print("Interval: {0}\nRight Riemann Sum Approximation: {1}\n".format(
        *right_riemann_sum(f, a, b, n)))
    print("Interval: {0}\nMidpoint Sum Approximation: {1}\n".format(
        *midpoint_approximation(f, a, b, n)))
    print("Interval: {0}\nTrapezoidal Sum Approximation: {1}\n".format(
        *trapezoidal_approximation(f, a, b, n)))
    print("Interval: {0}\nSimpson's Rule Approximation: {1}\n".format(
        *simpsons_rule_approximation(f, a, b, n)))
\end{lstlisting}

\section*{Exercise 7c}
\[ \int_{1}^{2}{\sqrt{x^{3}-1}\diff{x}} \quad n = 10 \]
\[ \Delta x = \frac{b-a}{n} = \frac{2-1}{10} = 0.1 \]
\[ S_{n} \approx 1.511519 \]

\section*{Exercise 11c}
\[ \int_{0}^{4}{x^{3}\sin{x}\diff{x}} \quad n = 8 \]
\[ \Delta x = \frac{b-a}{n} = \frac{4-0}{8} = 0.5 \]
\[ S_{n} \approx -5.605350 \]

\section*{Exercise 15c}
\[ \int_{0}^{1}{\frac{x^{2}}{1+x^{4}}\diff{x}} \quad n = 10 \]
\[ \Delta x = \frac{b-a}{n} = \frac{1-0}{10} = 0.1 \]
\[ S_{n} \approx 0.243751 \]

\begin{center}
  If you have any questions, comments, or concerns, please contact me at
  alvin@omgimanerd.tech
\end{center}

\end{document}
