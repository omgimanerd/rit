\documentclass[letterpaper, 12pt]{article}
\usepackage{geometry}
\usepackage{polynom}

\geometry{letterpaper, margin=1in}

\renewcommand*{\arcsin}{\sin^{-1}}
\renewcommand*{\arccos}{\cos^{-1}}
\renewcommand*{\arctan}{\tan^{-1}}
\newcommand*{\arccot}{\cot^{-1}}
\newcommand*{\arcsec}{\sec^{-1}}
\newcommand*{\arccsc}{\csc^{-1}}
\newcommand*{\diff}{\mathrm{d}}
\newcommand*{\Diff}[1]{\mathrm{d^#1}}
\newcommand*{\e}{\mathrm{e}}

\title{Section 7.4}
\author{Alvin Lin}
\date{Calculus II: August 2016 - December 2016}

\begin{document}

\maketitle

\section*{Problem 7}
\[ \int{\frac{x^{4}}{x-1}\diff{x}} \]
\begin{center}
  \polylongdiv{x^4}{x-1}
\end{center}
\[ \int{\frac{x^{4}}{x-1}\diff{x}} =
   \int{x^{3}+x^{2}+x+1+\frac{1}{x-1}\diff{x}} \]
\[ = \frac{x^{4}}{4}+\frac{x^{3}}{3}+\frac{x^{2}}{2}+x+\ln|x+1|+C \]

\section*{Problem 11}
\[ \int_{0}^{1}{\frac{2}{2x^{2}+3x+1}\diff{x}} \]
\[ \frac{2}{2x^{2}+3x+1} = \frac{2}{(2x+1)(x+1)} =
   \frac{A}{2x+1}+\frac{B}{x+1} \]
\[ (2x+1)(x+1)\frac{2}{(2x+1)(x+1)} = \frac{A}{2x+1}+\frac{B}{x+1}(2x+1)(x+1) \]
\[ 2 = A(x+1)+B(2x+1) \]
\[ 2 = Ax + A + 2Bx + B = (A+2B)x+(A+B) \]
\[ A+2B = 0 \quad A+B = 2 \]
\[ A+2(2-A) = A+4-2A = 0 \]
\[ A = 4 \quad B = -2 \]
\[ \frac{2}{2x^{2}+3x+1} = \frac{4}{2x+1}-\frac{2}{x+1} \]
Solve it as an indefinite integral:
\[ \int{\frac{2}{2x^{2}+3x+1}\diff{x}} =
   \int{\frac{4}{2x+1}-\frac{2}{x+1}\diff{x}} \]
\[ 2\int{\frac{2}{2x+1}\diff{x}}-2\int{\frac{1}{x+1}\diff{x}} \]
\[ 2\int{\frac{2}{2x+1}\diff{x}}-2\int{\frac{1}{x+1}\diff{x}} \]
\[ 2\ln|2x+1|-2\ln|x+1|+C \]
Now we apply the original limits of integration:
\[ \bigg[2\ln|2x+1|-2\ln|x+1|\bigg]_{0}^{1} \]
\[ 2\ln|2(1)+1|-2\ln|1+1|-(2\ln|2(0)+1|-2\ln|0+1|) \]
\[ 2\ln|3|-2\ln|2|-(0) \]
\[ = \ln|\frac{9}{4}| \]

\section*{Problem 19}
\[ \int_{0}^{1}{\frac{x^{2}+x+1}{(x+1)^{2}(x+2)}\diff{x}} \]
\[ \frac{x^{2}+x+1}{(x+1)^{2}(x+2)} = \frac{Ax+B}{(x+1)^{2}}+\frac{D}{x+2} \]
\[ (x+1)^{2}(x+2)\frac{x^{2}+x+1}{(x+1)^{2}(x+2)} =
   \frac{Ax+B}{(x+1)^{2}}+\frac{D}{x+2}(x+1)^{2}(x+2) \]
\[ x^{2}+x+1 = (Ax+B)(x+2)+(D)(x^{2}+2x+1) \]
\[ x^{2}+x+1 = Ax^{2}+2Ax+Bx+2B+Dx^{2}+2Dx+D \]
\[ x^{2}+x+1 = (A+D)x^{2}+(2A+B+2D)x+(2B+D) \]
\[ A+D = 1 \quad 2A+B+2D = 1 \quad 2B+D = 1 \]
\[ 2(1-D)+\frac{1-D}{2}+2D = 1 \]
\[ 4-4D+1-D+4D = 2 \]
\[ A = -2 \quad B = -1 \quad D = 3 \]
\[ \frac{x^{2}+x+1}{(x+1)^{2}(x+2)} = \frac{-2x-1}{(x+1)^{2}}+\frac{3}{x+2} \]
Solve it as an indefinite integral:
\[ \int{\frac{-2x-1}{(x+1)^{2}}+\frac{3}{x+2}\diff{x}} \]
\[ -\int{\frac{2x+1}{(x+1)^{2}}\diff{x}}+\int{\frac{3}{x+2}\diff{x}} \]
\[ 3\ln|x+2|-\int{\frac{2x+1}{(x+1)^{2}}\diff{x}} \]
\[ Let: u = x+1 \quad x = u-1 \quad \diff{u}=\diff{x} \]
\[ 3\ln|x+2|-\int{\frac{2(u-1)+1}{u^{2}}\diff{u}} \]
\[ 3\ln|x+2|-\int{\frac{2u-1}{u^{2}}\diff{u}} \]
\[ 3\ln|x+2|-\int{\frac{2}{u}\diff{u}}-\int{u^{-2}\diff{u}} \]
\[ 3\ln|x+2|-2\ln|u|+\frac{1}{u}+C \]
\[ 3\ln|x+2|-2\ln|x+1|+\frac{1}{x+1}+C \]
Now we apply the original limits of integration:
\[ \bigg[3\ln|x+2|-2\ln|x+1|+\frac{1}{x+1}\bigg]_{0}^{1} \]
\[ 3\ln|1+2|-2\ln|1+1|+\frac{1}{1+1}-(3\ln|0+2|-2\ln|0+1|+\frac{1}{0+1}) \]
\[ 3\ln|3|-2\ln|2|+\frac{1}{2}-(3\ln|2|-0+1) \]
\[ 3\ln|3|-2\ln|2|-3\ln|2|-\frac{1}{2} \]

\section*{Problem 23}
\[ \int{\frac{10}{(x-1)(x^{2}+9)}} \]
\[ \frac{10}{(x-1)(x^{2}+9)} = \frac{A}{x-1}+\frac{Bx+D}{x^{2}+9} \]
\[ (x-1)(x^{2}+9)\frac{10}{(x-1)(x^{2}+9)} =
   \frac{A}{x-1}+\frac{Bx+D}{x^{2}+9}(x-1)(x^{2}+9) \]
\[ 10 = A(x^{2}+9)+(Bx+D)(x-1) \]
\[ 10 = Ax^{2}+9A+Bx^{2}-Bx+Dx-D \]
\[ 10 = (A+B)x^{2}+(D-B)x+(9A-D) \]
\[ A+B = 0 \quad D-B = 0 \quad 9A-D = 10 \]
\[ D = 9A-10 \quad 9A-10-B = 0 \quad 9A-10-(-A) = 10 \]
\[ 10A = 20 \]
\[ A = 2 \quad B = -2 \quad D = 8 \]
\[ \frac{10}{(x-1)(x^{2}+9)} = \frac{2}{x-1}+\frac{-2x+8}{x^{2}+9} \]
\[ \int{\frac{10}{(x-1)(x^{2}+9)}\diff{x}} =
   \int{\frac{2}{x-1}\diff{x}}-\int{\frac{2x-8}{x^{2}+9}\diff{x}} \]
\[ 2\int{\frac{1}{x-1}\diff{x}}-\int{\frac{2x}{x^{2}+9}\diff{x}}+
   \int{\frac{8}{x^{2}+9}\diff{x}} \]
\[ 2\ln|x-1|-\ln|x^{2}+9|+\frac{8}{3}\int{\frac{3}{x^{2}+9}\diff{x}} \]
\[ 2\ln|x-1|-\ln|x^{2}+9|+\frac{8}{3}\arctan(\frac{x}{3})+C \]

\section*{Problem 39}
\[ \int{\frac{\sqrt{x+1}}{x}\diff{x}} \]
\[ Let: u = \sqrt{x+1} \]
\[ x = u^{2}-1 \]
\[ \diff{u} = \frac{1}{2\sqrt{x+1}}\diff{x} \]
\[ 2\diff{u}\sqrt{x+1} = \diff{x} \]
\[ 2u\diff{u} = \diff{x} \]
\[ \int{\frac{u}{u^{2}-1}2u\diff{u}} \]
\[ 2\int{\frac{u^{2}}{u^{2}-1}\diff{u}} \]
\[ 2\int{\frac{u^{2}-1+1}{u^{2}-1}\diff{u}} \]
\[ 2\int{\frac{u^{2}-1}{u^{2}-1}+\frac{1}{u^{2}-1}\diff{u}} \]
\[ 2\int{1\diff{u}}+2\int{\frac{1}{(u-1)(u+1)}\diff{u}} \]
\[ 2u+C+2\int{\frac{1}{(u-1)(u+1)}\diff{u}} \]
\[ \frac{1}{(u-1)(u+1)} = \frac{A}{u-1}+\frac{B}{u+1} \]
\[ (u-1)(u+1)\frac{1}{(u-1)(u+1)} = \frac{A}{u-1}+\frac{B}{u+1}(u-1)(u+1) \]
\[ 1 = A(u+1)+B(u-1) \]
\[ 1 = Au+A+Bu-B \]
\[ 1 = (A+B)u+(A-B) \]
\[ A+B = 0 \quad A-B = 1 \]
\[ (1+B)+B = 0 \quad B = -\frac{1}{2} \quad A = \frac{1}{2} \]
\[ \frac{1}{(u-1)(u+1)} = \frac{1}{2(u-1)}-\frac{1}{2(u+1)} \]
\[ \int{\frac{1}{(u-1)(u+1)}\diff{u}} =
   \int{\frac{1}{2(u-1)}-\frac{1}{2(u+1)}\diff{u}} \]
\[ \frac{1}{2}\int{\frac{1}{u-1}-\frac{1}{u+1}\diff{u}} \]
\[ \frac{1}{2}(\ln|u-1|-\ln|u+1|)+C \]
\[ 2u+C+2\int{\frac{1}{(u-1)(u+1)}\diff{u}} =
   2u+2(\frac{1}{2}(\ln|u-1|-\ln|u+1|))+C \]
\[ 2u+\ln|u-1|-\ln|u+1|+C \]

\begin{center}
  If any errors are found, please contact me at alvin.lin.dev@gmail.com
\end{center}

\end{document}
