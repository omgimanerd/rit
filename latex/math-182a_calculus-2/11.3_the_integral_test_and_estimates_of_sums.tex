\documentclass[letterpaper, 12pt]{article}
\usepackage{amsmath}
\usepackage{amssymb}

\renewcommand*{\arcsin}{\sin^{-1}}
\renewcommand*{\arccos}{\cos^{-1}}
\renewcommand*{\arctan}{\tan^{-1}}
\newcommand*{\arccot}{\cot^{-1}}
\newcommand*{\arcsec}{\sec^{-1}}
\newcommand*{\arccsc}{\csc^{-1}}
\newcommand*{\diff}{\mathrm{d}}
\newcommand*{\ddiff}[2]{\frac{\diff{#1}}{\diff{#2}}}
\newcommand*{\Diff}[1]{\mathrm{d^#1}}
\newcommand*{\e}{\mathrm{e}}

\title{The Integral Test and Esimates of Sums}
\author{Alvin Lin}
\date{Calculus II: August 2016 - December 2016}

\begin{document}

\maketitle

\section*{The Integral Test and Estimates of Sums}

\subsection*{The Integral Test}
If \( f \) is continuous, positive, and decreasing, and let \( f(n) = a_{n} \).
Then:
\begin{enumerate}
  \item If \( \int_{1}^{\infty}f(x)\diff{x} \) converges, then
    \( \sum{a_{n}} \) converges.
  \item If \( \int_{1}^{\infty}f(x)\diff{x} \) diverges, then
    \( \sum{a_{n}} \) diverges.
\end{enumerate}

\subsection*{Example}
\[ \sum_{n=1}^{\infty}\frac{1}{\sqrt{n+4}} \]
\[ f(x) = \frac{1}{\sqrt{x+4}} \]
\begin{align*}
  \int_{1}^{\infty}f(x)\diff{x} &=
    \lim_{b\to\infty}\int_{1}^{b}f(x)\diff{x} \\
  &= \lim_{b\to\infty}\int_{1}^{b}\frac{1}{\sqrt{x+4}}\diff{x} \\
  &= \lim_{b\to\infty}\int_{1}^{b}(x+4)^{-\frac{1}{2}}\diff{x} \\
  &= \lim_{b\to\infty}\bigg[
    \frac{(x+4)^{\frac{1}{2}}}{\frac{1}{2}}\bigg]_{1}^{b} \\
  &= \lim_{b\to\infty}\bigg[2(b+4)^{\frac{1}{2}}-2(5)^{\frac{1}{2}}] \\
  &= \infty \\
  & \therefore \sum_{n=1}^{\infty}\frac{1}{\sqrt{n+4}}\mathrm{(DIVERGES)}
\end{align*}

\subsection*{Example 2}
\[ \sum_{n=1}^{\infty}n\e^{-n} \]
\[ f(x) = x\e^{-x} \]
\begin{align*}
  \int_{1}^{\infty}f(x)\diff{x} &=
    \lim_{b\to\infty}\int_{1}^{b}x\e^{-x}\diff{x} \\
  &= \lim_{b\to\infty}\bigg(\big[-x\e^{-x}\big]_{1}^{b}-
    \int_{1}^{b}-\e^{-x}\diff{x}\bigg) \\
  &= \lim_{b\to\infty}\bigg[\frac{-b}{\e^{b}}+\frac{1}{\e}\bigg]-
    \lim_{b\to\infty}\bigg[\e^{-b}-\e^{-1}\bigg] \\
  &= 0+\frac{1}{\e}+\frac{1}{\e} \\
  & \therefore \sum_{n=1}^{\infty}n\e^{-n} \mathrm{converges}
\end{align*}

\subsection*{Example 3}
\[ \sum_{n=1}^{\infty}\frac{12}{-5^{n}} \]
\[ a_{1}+a_{2}+a_{3}+... =
   \frac{12}{-5^{1}}+\frac{12}{-5^{2}}+\frac{12}{-5^{3}}+... \]
\[ \frac{a_{2}}{a_{1}} = \frac{a_{3}}{a_{2}} = r \]
\[ r = \frac{\frac{12}{-5^{2}}}{\frac{12}{-5}} = \frac{-1}{5}; |r| < 1 \]
\[ \sum_{n=1}^{\infty}\frac{12}{-5^{n}} = \frac{a}{1-r} =
   \frac{\frac{12}{-5}}{1-\frac{-1}{5}} = -2 \]

\subsection*{Example 4}
\[ \sum_{n=2}^{\infty}\frac{1}{(1+c)^{n}} = 2 \]
\[ \frac{1}{(1+c)^{2}}+\frac{1}{(1+c)^{3}}+... = 2 \]
\[ \frac{a}{1-r} = 2 \]
\[ \frac{\frac{1}{(1+c)^{2}}}{1-\frac{1}{1+c}} = 2 \]
\[ \frac{\frac{1}{1+c}}{c} = 2 \]
\[ \frac{1}{c(1+c)} = 2 \]
\[ 2c+2c^{2} = 1 \]
\[ 2c^{2}+2c-1 = 0 \]
\[ c = \frac{-2\pm\sqrt{4-4(2)(-1)}}{4} = \frac{-1\pm\sqrt{3}}{2} \]

\subsection*{Example 5}
\[ 1+\frac{1}{3}+\frac{1}{5}+\frac{1}{2}+... =
   \sum_{n=1}^{\infty}\frac{1}{2n-1} \]
\[ f(x) = \frac{1}{2n-1} \]
\begin{align*}
  \lim_{b\to\infty}\int_{1}^{b}f(x)\diff{x} &=
    \lim_{b\to\infty}\int_{1}^{b}\frac{1}{2x-1}\diff{x} \\
  &= \lim_{b\to\infty}\bigg[\frac{\ln(2x-1)}{2}\bigg]_{1}^{b} \\
  &= \lim_{b\to\infty}\bigg[\frac{\ln(2b-1)}{2}-\frac{\ln(1)}{2}\bigg] \\
  &= \infty \\
  & \therefore \sum_{n=1}^{\infty}\frac{1}{2n-1} \mathrm{diverges}
\end{align*}

\begin{center}
  If any errors are found, please contact me at alvin.lin.dev@gmail.com
\end{center}

\end{document}
