\documentclass{math}

\title{Trigonometric Integrals}
\author{Alvin Lin}
\date{Calculus II: August 2016 - December 2016}

\begin{document}

\maketitle

\section*{Trigonometric Integrals}
Using concepts from Trigonometric Substitution, we can solve integrals with
trigonometric functions by separating out even powers and using the following
identities:
\[ \sin^{2}(\theta)+\cos^{2}(\theta) = 1 \]
\[ 1+\tan^{2}(\theta) = \sec^{2}(\theta) \]
\[ 1+\cot^{2}(\theta) = \csc^{2}(\theta) \]
If we can isolate an even power of sin, we can substitute:
\[ \sin^{2}(\theta) = 1-\cos^{2}(\theta) \]
If we can isolate an even power of cos, we can substitute:
\[ \cos^{2}(\theta) = 1-\sin^{2}(\theta) \]
The same rules apply for even powers of tan, sec, cot, and csc.

\subsection*{Practice Problem 2}
\[ \int{\sin^{3}(\theta)\cos^{4}(\theta)\diff{\theta}} \]
\[ \int{\sin^{2}(\theta)\sin(\theta)\cos^{3}(\theta)\diff{\theta}} \]
\[ Let: x = \cos(\theta) \]
\[ \diff{x} = -\sin(\theta)\diff{\theta} \]
\[ -\int{(1-\cos^{2}(\theta))\cos^{4}(\theta)\diff{x}} \]
\[ -\int{(1-x^{2})x^{4}\diff{x}} \]
\[ -\int{x^{4}-x^{6}\diff{x}} \]
\[ -\frac{1}{5}x^{5}+\frac{1}{7}x^{7}+C \]
\[ = -\frac{1}{5}\cos^{5}(\theta)+\frac{1}{7}\cos^{7}(\theta)+C \]

\subsection*{Practice Problem 4}
\[ \int_{0}^{\pi/2}{\sin^{5}(x)\diff{x}} \]
\[ \int_{0}^{\pi/2}{\sin^{4}(x)\sin(x)\diff{x}} \]
\[ \int_{0}^{\pi/2}{(1-\cos^{2}(x))^{2}\sin(x)\diff{x}} \]
\[ \int_{0}^{\pi/2}{(\cos^{4}(x)-2\cos^{2}(x)+1)\sin(x)\diff{x}} \]
\[ Let: t = \cos(x) \]
\[ \diff{t} = -\sin(x)\diff{x} \]
The limits from 0 to \( \pi/2 \) become 1 to 0.
\( \bigg(\int_{0}^{\pi/2}\rightarrow\int_{1}^{0}\bigg) \)
\[ -\int_{1}^{0}{t^{4}-2t^{2}+1\diff{t}} \]
We can flip the integral and negate it to remove the minus sign.
\[ \int_{0}^{1}{t^{4}-2t^{2}+1\diff{t}} \]
\[ \bigg[\frac{1}{5}t^{5}-\frac{2}{3}t^{3}+t\bigg]_{0}^{1} \]
\[ \frac{1}{5}-\frac{2}{3}+1 = \frac{3-10+15}{15} \]
\[ = \frac{8}{15} \]

\subsection*{Practice Problem 13}
\[ \int{\sqrt{cos(\theta)}\sin^{3}(\theta)\diff{\theta}} \]
\[ \int{\sqrt{cos(\theta)}(1-\sin^{2}(\theta))\sin(\theta)\diff{\theta}} \]
\[ Let: t = \cos(\theta) \]
\[ \diff{t} = -\sin(\theta)\diff{\theta} \]
\[ -\int{t^{\frac{1}{2}}(1-t^{2})\diff{t}} \]
\[ \int{-t^{\frac{1}{2}}+t^{\frac{5}{2}}\diff{t}} \]
\[ -\frac{2}{3}t^{\frac{3}{2}}+\frac{2}{7}t^{\frac{7}{2}}+C \]
\[ -\frac{2}{3}\cos^{\frac{3}{2}}(\theta)+
   \frac{2}{7}\cos^{\frac{7}{2}}(\theta)+C \]

\subsection*{Practice Problem 21}
\[ \int{\tan(x)\sec^{3}(x)\diff{x}} \]
\[ \int{\tan(x)\sec(x)\sec^{2}(x)\diff{x}} \]
\[ Let: t = \sec(x) \]
\[ \diff{t} = \sec(x)\tan(x)\diff{x} \]
\[ \int{t^{2}\diff{t}} \]
\[ \frac{1}{3}t^{3}+C \]
\[ = \frac{1}{3}\sec^{3}(x)+C \]

\subsection*{Practice Problem 22}
\[ \int{\tan^{2}(\theta)\sec^{4}(\theta)\diff{\theta}} \]
\[ \int{\tan^{2}(\theta)\sec^{2}(\theta)\sec^{2}(\theta)\diff{\theta}} \]
\[ \int{\tan^{2}(\theta)(1+\tan^{2}(\theta))\sec^{2}(\theta)\diff{\theta}} \]
\[ Let: t = \tan(\theta) \]
\[ \diff{t} = \sec^{2}(\theta) \]
\[ \int{t^{2}(1+t^{2})\diff{t}} \]
\[ \int{t^{2}+t^{6}\diff{t}} \]
\[ \frac{1}{3}t^{3}+\frac{1}{5}t^{5}+C \]
\[ \frac{1}{3}\tan^{3}(\theta)+\frac{1}{5}\tan^{5}(\theta)+C \]

\begin{center}
  You can find all my notes at \url{http://omgimanerd.tech/notes}. If you have
  any questions, comments, or concerns, please contact me at
  alvin@omgimanerd.tech
\end{center}

\end{document}
