\documentclass{math}

\usepackage{enumerate}

\title{Personal Financial Management}
\author{Alvin Lin}
\date{January 2019 - May 2019}

\begin{document}

\maketitle

\section*{HW 2}

\subsubsection*{Question 4}
For each of the following situations, compute the missing amount.
\begin{enumerate}
  \item Assets \$45,000; liabilities \$12,600; net worth \$32,400
  \item Assets \$78,980; liabilities \$65,280; net worth \$13,700
  \item Assets \$44,280; liabilities \$12,965; net worth \$31,315
  \item Assets \$92,140; liabilities \$38,345; net worth \$53,795
\end{enumerate}

\subsubsection*{Question 1}
Ross Martin arrived at the following tax information: \\
\begin{center}
  \begin{tabular}{ll}
    Gross salary & \$56,145 \\
    Interest earnings & \$205 \\
    Dividend income & \$65 \\
    One personal exemption & \$4,050 \\
    Itemized deductions & \$11,250 \\
    Adjustments to income & \$1,200
  \end{tabular}
\end{center}
What amount would Ross report as taxable income? \par
\[ 56,145+205+65-1,200-4,050-11,250 = 39,915 \]

\subsubsection*{Question 2}
If Lola Harper had the following itemized deductions, should she use Schedule
A or the standard deduction? The standard deduction for her tax situation is
\$6,300.
\begin{center}
  \begin{tabular}{ll}
    Donations to church and other charities & \$2,050 \\
    Medical and dental expenses exceeding 10 percent of adjusted gross income &
      \$400 \\
    State income tax & \$690 \\
    Job-related expenses exceeding 2 percent of adjust gross income & \$2,010 \\
  \end{tabular}
\end{center}
She should use the standard deduction since she will only have \$5,150 in
deductions under Schedule A.

\subsubsection*{Question 5}
What would the average tax rate for a person who paid taxes of \$5,490 on a
taxable income of \$41,670?
\[ \frac{5490}{41670} = 0.1317 = 13.17\% \]

\subsubsection*{Question 6}
Based on the following data, would Ann and Carl Wilton receive a refund or owe
additional taxes?
\begin{center}
  \begin{tabular}{ll}
    Adjusted gross income & \$66,686 \\
    Itemized deductions & \$14,900 \\
    Child care tax credit & \$100 \\
    Federal income tax withheld & \$5,490 \\
    Amount for personal exemptions & \$12,150 \\
    Tax rate on taxable income & 15 percent
  \end{tabular}
\end{center}
Tax liability: \( 0.15(66686 - 14900 - 12150) - 100 = 5845.4 \) \\
Based on this information, Ann and Carl Wilton would owe an additional \$355.40.

\begin{center}
  You can find all my notes at \url{http://omgimanerd.tech/notes}. If you have
  any questions, comments, or concerns, please contact me at
  alvin@omgimanerd.tech
\end{center}

\end{document}
