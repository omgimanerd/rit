\documentclass{math}

\title{Personal Financial Management}
\author{Alvin Lin}
\date{January 2019 - May 2019}

\begin{document}

\maketitle

\section*{HW 3}
Ever since his wife's death, Erci Stanford has faced difficult personal and
financial circumstances. His job provides him with a good income but keeps him
away from his daughters, ages 8 and 10, nearly 20 days a month. This requires
him to use in-home child care services that consume a large portion of his
income. Since the Stanfords live in a small apartment, this arrangement has been
very inconvenient. \par
Due to the costs of caring for his children, Eric has only a minimal amount
withheld from his salary for federal income taxes. This makes more money
available during the year, but for the last few years he has to make large
payments in April - another financial burdern. \par
Although Eric has created an investment fund for his daughters' college
education and for his retirement, he has not sought to look at several aspects
of his tax planning activities to find strategies that will best serve his
current and future financial needs. \par
Eric has assembled the following information for the current tax year:
\begin{center}
  \begin{tabular}{|c|c|}
    \hline
    Earnings from wages & \$102,170 \\
    \hline
    Interest earned on savings & \$150 \\
    \hline
    IRA deduction & \$5,500 \\
    \hline
    Checking account interest & \$80 \\
    \hline
    Three exemptions & \$4,050 each \\
    \hline
    Current standard deduction for filing status & \$9,300 \\
    \hline
    Amount withheld for federal income tax & \$10,178 \\
    \hline
    Tax credit for child care & \$600 \\
    \hline
    Filing status & Head of household \\
    \hline
  \end{tabular}
\end{center}

\subsubsection*{Question 1}
What are Eric's major financial concerns in his current situation? \par
Eric needs to have money available to pay the difference in his tax liability
in April. He needs to save money throughout the year in order to make up the
difference.

\subsubsection*{Question 2}
In what ways might Eric improve his tax planning efforts? \par
It may be a smarter choice to withhold more from his paycheck so that he does
not have to make a large payment in April.

\subsubsection*{Question 3}
Is Eric typical of many people in our society with regard to his tax planning?
Why or why not? \par
Most people are usually in Eric's situation since a standard family will have
two sources of income filing jointly. The death of Eric's wife is an unfortunate
event that increases the amount of financial burden on the household.

\subsubsection*{Question 4}
What additional actions might Eric investigate with regard to taxes and
personal financial planning? \par
Getting married (to someone else) would bring in an additional source of income
and shift them into a different income bracket. Eric might also want to consider
opening a 401k to save for his future and obtain more tax deductions to itemize.

\subsubsection*{Question 5}
Calculate the following.
\begin{enumerate}
  \item What is Eric's taxable income?
  \[ 102170+150+80-5500-4050-4050-4050 = \$84750 \]
  \item What is his total tax liability? What is his average tax rate?
  \[ (1855+8512.50+0.25(84750-75000))-600 = \$12205 \]
  \[ \frac{12205}{84750} = 14.4\% \]
  \item Based on his withholding, will Eric receive a refund or owe additional
    tax? What is the amount? \par
    Based on this information, Eric will owe an additional \$2027 in April.
\end{enumerate}

\begin{center}
  You can find all my notes at \url{http://omgimanerd.tech/notes}. If you have
  any questions, comments, or concerns, please contact me at
  alvin@omgimanerd.tech
\end{center}

\end{document}
