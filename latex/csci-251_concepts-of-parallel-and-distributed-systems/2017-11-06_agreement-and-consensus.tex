\documentclass{math}

\usepackage{tikz}

\title{CSCI 251: Concepts of Parallel and Distributed Systems}
\author{Alvin Lin}
\date{November 1st, 2017}

\begin{document}

\maketitle

\section*{Agreement and Consensus}
Topics:
\begin{itemize}
  \item Agreement Algorithms:
    \begin{itemize}
      \item Consensus
      \item Byzantine Generals
      \item Interactive Consistence
    \end{itemize}
  \item Transactions
\end{itemize}

\subsection*{Consensus}
Suppose there are decision variables. At the termination of the agreement
algorithm, all the processes should agree on the decision variable. For
processes \( p_1 \) to \( p_n \), each process proposes a value \( v_1 \) to
\( v_n \). At the end of the consensus algorithm, all the processes should
agree on one decision value. This can be done as either the majority value,
the max value, the min value, or the average.

\subsubsection*{Byzantine General's Problem}
There is a distinguished process which decides the decision variable and every
process sets their decision value to the one suggested by the distinguished
process.
\begin{center}
  \begin{tikzpicture}
    \draw (0,4) circle (1cm) node {\( p_1 \)};
    \draw (-3,0) circle (1cm) node {\( p_2 \)};
    \draw (3,0) circle (1cm) node {\( p_3 \)};
    \draw[->] (1,3) -- (2.25,1) node[pos=0.5,right] {\( p_1,v \)};
    \draw[->] (-1,3) -- (-2.25,1) node[pos=0.5,left] {\( p_1,v \)};
    \draw[->] (-1.75,0.25) -- (1.75,0.25) node[pos=0.5,above] {\( p_2,p_1,v \)};
    \draw[->] (1.75,-0.25) -- (-1.75,-0.25)
      node[pos=0.5,below] {\( p_3,p_1,v \)};
  \end{tikzpicture}
\end{center}
In the case where a propagated decision variable is faulty or corrupted, this
degrades into making a decision via consensus instead. It is impossible to
identify the number of faulty processes \( f \) out of \( n \) processes if:
\[ n\le3f \]

\subsubsection*{Interactive Consistency}
In interactive consistency, each process proposes a value, which is its
contribution to the vector of decision values. At the termination of the
algorithm, all processes agree on the vector of values.

\subsection*{Transactions}
Transactions are series of a read/write operations on a server. A transaction
has to satisfy four properties.
\begin{itemize}
  \item Atomicity - all or nothing, either the operation completes or it does
  not complete, without a middle state.
  \item Consistency - the transaction process moves from one consistent state to
  another consistent state.
  \item Isolation - there should be no interference from other transactions.
  \item Durability - transaction objects must be recoverable and stored in
  permanent memory.
\end{itemize}

\subsubsection*{Example}
Banking Accounts A, B, C:
\[ A = \$100; \quad B = \$200; \quad C = \$300 \]
\begin{center}
  \begin{tabular}{l|l}
    \textbf{Transaction T} & \textbf{Transaction U} \\
    \hline
    \texttt{balance = b.getBalance()}; \$200 & \\
    & \texttt{balance = b.getBalance()}; \$200 \\
    & \texttt{b.setBalance(balance * 1.1)}; \$220 \\
    \texttt{b.setBalance(balance * 1.1)}; \$220 &
  \end{tabular}
\end{center}
A situation like this is problematic without ACID since transaction U's update
is lost.
\[ A = B = \$200; \]
\begin{center}
  \begin{tabular}{l|l}
    \textbf{Transaction V} & \textbf{Transaction W} \\
    \hline
    \texttt{a.withdraw(100)}; \$100 & \\
    & \texttt{total = a.getBalance()}; \$100 \\
    & \texttt{total += b.getBalance()}; \$300 \\
    \texttt{b.deposit(100)}; \$300 &
  \end{tabular}
\end{center}
In this situation, \( W \) does not see the deposit done by \( V \) and only
sees the withdrawal action. These issues are solved by:
\begin{itemize}
  \item Serialization - each transaction should appear as if it is the only
  transaction on the server.
  \item Locking - Inefficient and creates a bottleneck.
  \item Optimistic Concurrency - Most efficient, uses multiple phases to resolve
  transactions.
  \item Timestamping - Resolves transactions in time stamp order.
\end{itemize}

\section*{Reminders}
Check MyCourses for details on Project 2. \\
\noindent Professor Mohan Kumar: \\
\url{mjkvcs@rit.edu} \\
\url{https://cs.rit.edu/~mjk} \\

\noindent Rahul Dashora (TA): \\
\url{rd5476@mail.rit.edu} \\

\begin{center}
  You can find all my notes at \url{http://omgimanerd.tech/notes}. If you have
  any questions, comments, or concerns, please contact me at
  alvin@omgimanerd.tech
\end{center}

\end{document}
