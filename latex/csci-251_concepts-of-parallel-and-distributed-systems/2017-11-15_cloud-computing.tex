\documentclass{math}

\usepackage{tikz}

\title{CSCI 251: Concepts of Parallel and Distributed Systems}
\author{Alvin Lin}
\date{November 15th, 2017}

\begin{document}

\maketitle

\section*{Cloud Computing}
Cloud computing defines a mechanism for hosting and delivering services to
applications over the Internet. Cloud services are sold on a ``pay as you go''
model to give the perception of infinite computing resources. Services
typically offered are storage and processing. The cloud consists of a data
center's hardware and software.

\subsection*{Services}
A service is self-contained functional unit that accepts request and returns
responses through a well-defined interface. They can be globally distributed
across organizations and are typically reuseable. Examples:
\begin{itemize}
  \item Find account balance
  \item Credit card validation
  \item Account lookup
\end{itemize}

\subsection*{Definitions}
Grid Computing:
\begin{itemize}
  \item For non-interactive but intensive workloads
  \item The computers are interconnected over the network, possibly remotely
  \item Middleware-based approach
  \item Started by the High Performance Computing community
\end{itemize}
Cluster Computing:
\begin{itemize}
  \item Loosely or tightly connected computers in a fast local area network
  \item Each node in the cluster uses similar hardware, OS, and software
\end{itemize}
Cloud Computing:
\begin{itemize}
  \item Provides on-demand network access to a shared pool of computing
  resources
  \item Computing resources can be quickly commissioned or decommissioned
  \item Greater flexibility
  \item Started by the business community
\end{itemize}
SaaS: Software as a Service \\
IaaS: Infrastructure as a Service \\
PaaS: Platform as a Service

\subsection*{Cloud Computing's Motivation}
Cloud computing provides a way to save in capital costs through on-demand
operational costs. It provides a robustness to failure and allows for complex
resource requirements.

\subsection*{Eucalyptus Architecture}
Elastic Utility Computing Architecture for Linking Your Program To Useful
Systems
\begin{itemize}
  \item Allows sites with existing clusters to host a cloud that is interface
  compatible with Amazon's AWS.
  \item Enables users to explore new cloud computing functionality with no
  impact on their existing application development software.
\end{itemize}

\subsubsection*{Commercial Products}
\begin{itemize}
  \item Amazon EC2
  \item Windows Azure
  \item Google App Engine
  \item Dropbox (wtf for real my dude?)
\end{itemize}

\subsection*{Research Challenges}
\begin{itemize}
  \item Automated service provisioning
  \item Virtual machine migration
  \item Server consolidation
  \item Energy management
  \item Traffic management and analysis
  \item Data security
  \item Software frameworks
  \item Storage technologies and data management
  \item Novel cloud architectures
\end{itemize}

\subsection*{Virtualization}
Virtualization is a computer architectural technique where multiple virtual
machines take turns to use the same hardware architecture. Hardware or software
resources can be virtualized. Virtualization enhances resource sharing and
improves computer performance, resource utilization, and application
flexibility.

\subsection*{Virtualized Systems}
Traditionally, computers run with an operating system that was specially
tailored for its hardware architecture. In a virtualized system, one or more
guest operating systems run on the same hardware. A virtualization layer
known as the hypervisor or virtual machine monitor separates the host system
from the virtual ones. \par
The operating system is an abstract layer created between the host OS and the
user applications. Virtualization layers live inside the OS, where resources
are partitioned. Virtual machines at the OS level have minimal startup/shutdown
costs, low resource requirements, and high scalability. Synchronizing state
changes is possible when needed. The only disadvantage is the all virtual
machines must run on the same OS.

\section*{Reminders}
Project 2 details. Let Professor Kumar know if you are working solo or in a
group by Monday, November 20th.

\noindent Professor Mohan Kumar: \\
\url{mjkvcs@rit.edu} \\
\url{https://cs.rit.edu/~mjk} \\

\noindent Rahul Dashora (TA): \\
\url{rd5476@mail.rit.edu} \\

\begin{center}
  You can find all my notes at \url{http://omgimanerd.tech/notes}. If you have
  any questions, comments, or concerns, please contact me at
  alvin@omgimanerd.tech
\end{center}

\end{document}
