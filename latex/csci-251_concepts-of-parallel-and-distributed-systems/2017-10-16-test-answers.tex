\documentclass[letterpaper, 12pt]{math}

\usepackage{listings}
\lstset{basicstyle=\ttfamily\footnotesize,breaklines=true}

\title{CSCI 251: Concepts of Parallel and Distributed Systems}
\author{Alvin Lin}
\date{October 16th, 2017}

\begin{document}

\maketitle

\section*{Test Answers}

\subsubsection*{Question 1}
In this problem, we were given a \( 4\times4 \) matrix multiplication operation
to parallelize across a \( 2\times2 \) grid. We can look at these as four
individual processors to split the data among, or a \( 2\times2 \) grid on which
we can implement Canon's algorithm. For this problem, the serial computation
time \( T_s \) is \( O(n^3) \), or 64 in this specific case. The parallel
computation time is \( T_p \) is 16 units at minimum plus the communication
costs. Depending on the implementation and initial assumptions, the
communication costs can vary.

\subsubsection*{Question 2}
In this problem, we had a tree of \( P \) processor nodes, of which the
\( k \) leaves must perform 1-to-all personalized communication. The
generalized expression for the cost of this \( k \)-to-all communication is:
\[ k\sum_{i=0}^{\log_2(P)}2^i\times i \]

\subsubsection*{Question 3}
For this problem, we had to implement a parallelized bitonic sort, merge sort,
odd-even sort, or quicksort. The only thing to note is that those who
implemented bitonic sort needed to implement the correct \texttt{compare\_split}
and \texttt{compare\_exchange} operation as discussed in class. For bitonic
sort in the case of \( N > P \), the order complexity becomes:
\[ \frac{N}{P}\log(\frac{N}{P})+\frac{N}{P}\log^2(P) \]

\section*{Project 2}
Project 2 will involve implementing bitonic sort. Since it requires message
passing, the use of MPI is suggested. Please review the \texttt{compare\_split}
operation. With MPI, the process with rank 0 can be treated as the master
process. The data can then be scattered across the other processes for
manipulation.

\section*{Reminders}
\noindent Professor Mohan Kumar: \\
\url{mjkvcs@rit.edu} \\
\url{https://cs.rit.edu/~mjk} \\

\noindent Rahul Dashora (TA): \\
\url{rd5476@mail.rit.edu} \\

\begin{center}
  You can find all my notes at \url{http://omgimanerd.tech/notes}. If you have
  any questions, comments, or concerns, please contact me at
  alvin@omgimanerd.tech
\end{center}

\end{document}
