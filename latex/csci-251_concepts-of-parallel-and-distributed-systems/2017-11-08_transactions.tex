\documentclass{math}

\usepackage{tikz}

\title{CSCI 251: Concepts of Parallel and Distributed Systems}
\author{Alvin Lin}
\date{November 1st, 2017}

\begin{document}

\maketitle

\section*{Transactions}
Transactions are created and managed by a coordinator interface. It involves a
client program, an object, and a coordinator. Example operations:
\begin{itemize}
  \item \texttt{openTransaction()}: starts a new transaction and delivers a
  unique transaction ID. This identifier is used by other operations in the
  transaction.
  \item \texttt{closeTransaction(identifier)}: ends a transaction by
  committing it.
  \item \texttt{abortTransaction(identifier)}: ends a transaction by aborting
  it.
\end{itemize}
A transaction may be aborted by the client or the server, in which case nothing
about that transaction is stored in permanent memory.

\subsection*{Failure model}
Server actions:
\begin{itemize}
  \item Server crash: new process aborts uncommitted transactions and recovers
  values produced by committed transactions.
  \item Client crash: uses expiry time to abort transactions.
\end{itemize}
Client actions:
\begin{itemize}
  \item Server crash: operations return an exception after timeout. If they are
  replaced by another process, uncommitted transactions will not be valid.
  \item Client crash: decision at the client's end.
\end{itemize}

\subsection*{Concurrency Control}
\begin{itemize}
  \item Locks
  \item Optimistic concurrency
  \item Timestamp ordering
\end{itemize}

\subsection*{Serial Equivalence}
\textbf{Serially Equivalent Interleaving:} the combined effect is the same as if
the transactions had been performed one at a time in some order. Serial
equivalence prevents lost updates and inconsistent retrievals.

\subsection*{Conflicting Operations}
A conflict when the combined effect of a pair of operations depends on the order
in which they are executed. Serial equivalence is a necessary and sufficient
condition where all pairs of conflicting operations are executed in the same
order at all objects they both access. \\
\textbf{Isolation Property:} The uncommitted state of one transaction should not
be visible to another transaction.

\subsection*{Nested Transactions}
A top level transaction may open child transactions which can run concurrently
along with their descendants. They can commit or abort independently of one
another. A transaction can commit or abort only after its child transactions
have completed. When a parent transaction aborts, all of its child transactions
must abort. When a parent transaction commits, all the child transactions can
commit, provided none of their ancestors have aborted.

\subsection*{Locks}
Serialization is achieved by the use of exclusive locks, which allow only one
transaction to access an object until it closes. When an operation accesses an
object within a transaction:
\begin{itemize}
  \item If the object is not already locked, the operations acquires a lock and
  the operation proceeds.
  \item If the object has a conflicting lock set by another transaction, the
  current transaction must wait until it is unlocked.
  \item If the object has a non-conflicting lock set by another transaction, the
  lock is shared and the operation proceeds.
  \item If the lock has already been locked in the same transaction, the lock
  will be promoted if necessary and the operation proceeds. When promotion is
  prevented by a conflicting lock, the second rule applies.
\end{itemize}
When a transaction closes, it unlocks all objects it locked for the transaction.
Locks reduce concurrency in general because they are not released until the end
of the transaction. Lock maintenance is also an overhead; deadlock prevention
and detection is expensive and reduces concurrency.

\subsection*{Optimistic Concurrency Control}
The likelihood of two transactions occurring at the same time is very low.
Transactions are allowed to proceed as if there are no conflicts until it is
closed. If a conflict is detected, the transaction may be aborted. Transactions
occur in three phases:
\begin{enumerate}
  \item \textbf{Working Phase}
  \item \textbf{Validation Phase}
  \item \textbf{Update Phase}
\end{enumerate}

\section*{Reminders}
Check MyCourses for details on Project 2. \\
\noindent Professor Mohan Kumar: \\
\url{mjkvcs@rit.edu} \\
\url{https://cs.rit.edu/~mjk} \\

\noindent Rahul Dashora (TA): \\
\url{rd5476@mail.rit.edu} \\

\begin{center}
  You can find all my notes at \url{http://omgimanerd.tech/notes}. If you have
  any questions, comments, or concerns, please contact me at
  alvin@omgimanerd.tech
\end{center}

\end{document}
