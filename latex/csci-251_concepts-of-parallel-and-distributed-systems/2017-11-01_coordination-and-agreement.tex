\documentclass{math}

\usepackage{tikz}

\title{CSCI 251: Concepts of Parallel and Distributed Systems}
\author{Alvin Lin}
\date{November 1st, 2017}

\begin{document}

\maketitle

\section*{Coordination and Agreement}
Topics:
\begin{itemize}
  \item Election Algorithm
  \item Multicast
  \item Agreement
\end{itemize}

\subsection*{Bully Algorithm}
A process is elected as the leader/coordinator based on its number/label.
Any process can start the election process.

\subsection*{Multicast}
The message \( m \) is sent to a group \( g \) of processes. The basic multicast
involves a process performing a multicast to all connected processes. Those
processes will propagate the message to all of its connected processes. This
is inefficient since a network of processes can send and acknowledge duplicate
messages. Reliable multicast is a variation that ensures some other basic
properties are satisfied.
\begin{itemize}
  \item Integrity: A correct process delivers \( m \) exactly once.
  \item Validity: If a correct process multicasts \( m \), then it will
  eventually deliver \( m \).
  \item Agreement: If a correct process delivers \( m \), then every correct
  process in the group \( g \) will also deliver \( m \).
\end{itemize}

\subsection*{Piggyback Acknowledgements}
Along with \( (m,g) \), a process lets other processes know the sequence
number of the multicast message for a process \( \langle(m,g),S_g^p\rangle \).
A message \( \langle q,R_g^q\rangle \) is sent to each process such that
\( q\in g, q\ne p \). \( R_g^q \) is nothing more than the sequence number of
the last multicast message from \( q \) that was delivered at \( p \). For a
receiving process \( r \), \( R_g^p \) is the last multicast message from
\( p \) that \( r \) has delivered, and \( R_g^q \) is the last multicast
message from \( q \) that \( r \) has delivered.
\begin{itemize}
  \item \( S = R_g^p+1 \), deliver \( m \)
  \item \( S \le R_g^p \), ignore \( m \) since it already has been delivered
  \item \( S > R_g^p+1 \), process \( r \) has missed a previous message, so the
  current message is placed in a holdback queue and negative acknowledgement is
  sent to \( p \)
\end{itemize}

\subsection*{Ordering of Multicast Message}
FIFO: if a process sends \( (m,g) \) before \( (m',g) \), then \( d(m) \)
happens before \( d(m') \). \\
Causal Ordering: multicast message \( (m,g) \) is the cause of another
multicast message \( (m',g) \) and therefore \( d(m) \) happens before
\( d(m') \).

\section*{Reminders}
Check MyCourses for details on Project 2. \\
\noindent Professor Mohan Kumar: \\
\url{mjkvcs@rit.edu} \\
\url{https://cs.rit.edu/~mjk} \\

\noindent Rahul Dashora (TA): \\
\url{rd5476@mail.rit.edu} \\

\begin{center}
  You can find all my notes at \url{http://omgimanerd.tech/notes}. If you have
  any questions, comments, or concerns, please contact me at
  alvin@omgimanerd.tech
\end{center}

\end{document}
