\documentclass[letterpaper, 12pt]{math}

\usepackage{listings}
\lstset{basicstyle=\ttfamily\footnotesize,breaklines=true}

\title{CSCI 251: Concepts of Parallel and Distributed Systems}
\author{Alvin Lin}
\date{October 4th, 2017}

\begin{document}

\maketitle

\subsection*{Practice Problem}
Illustrate the user of recursive locks using a binary tree search algorithm.
The program takes in a large list of numbers. The list is divided across
multiple threads. Each thread tries to insert its elements into the tree by
using a single lock associated with the tree.
\begin{enumerate}
  \item Show that the single lock becomes a bottleneck even for a moderate
  number of threads. \\
  (This problem is obvious, a single thread inserting an element locks the
  entire tree and prevents any other thread from writing.)
  \item Improve the above program by associating a lock with each node in the
  tree (as opposed to a single lock with the entire tree). A thread locks a node
  when it reads or writes it. Examine the performance properties of the
  implementation.
\end{enumerate}

\subsection*{Practice Problem}
Implement a threaded hash table in which a single lock is associated with a
block of \( k \) hash table entries. Threads attempting to read/write an element
in a block must first lock the corresponding block. How does the performance
of your algorithm vary as a function of \( k (k<n) \), where \( n \) is the
total number of entries in the hash table.

\subsection*{Practice Problem}
The odd-even transposition algorithm sorts \( N \) elements in \( N \) phases
where \( N \) is even. The algorithm alternates between two phases called odd
and even phases. Each phase requires \( \frac{N}{2} \) compare-exchange
operations. During the odd phase, elements with odd indices are compared with
their right neighbors and they are exchanged if they are out of sequence.
Similarly, during the even phase, elements with even indices are compared with
their right neighbors, and they are exchanged if they are out of sequence. After
\( N \) phases of odd-even exchanges, the sequence is sorted.
\begin{enumerate}
  \item Assume availability of \( P = N \) processes. Show a parallel
  formulation of the problem and compute the parallel time and speedup.
  \item Assume availability of \( P << N \) processes. Show a parallel
  formulation of the problem and compute the parallel time and speedup.
\end{enumerate}

\subsection*{Practice Problem}
Develop a parallel formulation of the quicksort algorithm. Provide analysis of
your algorithm. Show the parallel time and speedup. Discuss the implementation
of the algorithm on shared memory and message passing systems.

\section*{Reminders}
The midterm is on October 11th.
Refer to MyCourses for details on Project 1, which is due Friday,
October 6th. \\

\noindent Professor Mohan Kumar: \\
\url{mjkvcs@rit.edu} \\
\url{https://cs.rit.edu/~mjk} \\

\noindent Rahul Dashora (TA): \\
\url{rd5476@mail.rit.edu} \\

\begin{center}
  You can find all my notes at \url{http://omgimanerd.tech/notes}. If you have
  any questions, comments, or concerns, please contact me at
  alvin@omgimanerd.tech
\end{center}

\end{document}
