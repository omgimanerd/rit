\documentclass[letterpaper, 12pt]{math}

\title{CSCI 251: Concepts of Parallel and Distributed Systems}
\author{Alvin Lin}
\date{October 18th, 2017}

\begin{document}

\maketitle

\section*{Distributed Systems}
Topics:
\begin{itemize}
  \item OS and distributed OS concepts
  \item Interprocess communications
  \item Clocks and global states
  \item Coordination
  \item Distributed Transactions
  \item Blockchain
  \item RMI and RPC
  \item Request-Reply protocol
  \item Marshalling and data representation
\end{itemize}

\subsection*{Distributed Computing}
\begin{itemize}
  \item Concurrent components
  \begin{itemize}
    \item Independent
    \item Communicate and coordinate through message passing or shared memory
  \end{itemize}
  \item Lack of a global clock
  \begin{itemize}
    \item Each component has its own clock
    \item Clock synchonization is a huge issue
  \end{itemize}
  \item Independent failure of components
  \begin{itemize}
    \item Components are isolated from one another for fault tolerance
    \item Redundancy
  \end{itemize}
\end{itemize}

\subsection*{Operation System Layer}
\begin{itemize}
  \item Middleware is on top of OS layers
  \item OS on a particular node provides abstraction of local resources, such
  as processing, storage, and communications.
  \item Middleware is across all notes to utilize combinations of local and
  remote resources. It controls the interactions among objects/processes.
\end{itemize}

\subsection*{OS Mechanisms}
\begin{itemize}
  \item Encapsulation: OS kernel and server processes provide server interfaces
  to clients to meet their needs but hide the system details.
  \item Protection: protection from illegitimate accesses, protect kernel
  addressses and system registers.
  \item Concurrent processing: sharing of resources among processes using
  communication and scheduling for efficient utilization.
\end{itemize}

\subsection*{Core OS Components}
\begin{itemize}
  \item Process manager: handles creation and operations, resource management
  for address space and threads.
  \item Thread manager: handles creation, synchronization, and scheduling.
  Threads are attached to processes.
  \item Communication manager: handles communication between threads on the same
  processor and threads on remote processors.
  \item Memory manager: handles physical memory, virtual memory, and cache.
  \item Supervisor: handles interrupts, system calls, and exceptions.
\end{itemize}
The time it takes to fetch data from the disk \( t_d \), memory \( t_m \), and
cache \( t_c \) have the following relation:
\[ t_d > t_m > t_c \]

\subsection*{Protection}
The operation system is expected to protect from illegitimate accesses. It
maintains access privileges between users and supervisors and separates the
address spaces between the user processes. During exceptions, control is
transferred from the user to the kernel, but switching takes time and memory.

\subsection*{Levels of Communication}
Five Levels:
\begin{itemize}
  \item Message Passing: communication primitives, send and receive, direct,
  consists of one direct logical communication path from a sending process to a
  receiving process.
  \item Request/Reply
  \item Transactions
  \item Transport Connection
  \item Packet Switching
\end{itemize}

\subsection*{Message Synchronization}
\begin{itemize}
  \item Nonblocking send: source is released after copying message to buffer.
  \item Blocking send: source is released after transmitting message to network.
  \item Reliable blocking send: source is released after the message is
  received by the receiver's buffer.
  \item Explicit blocking send: source is released after the message has been
  received by the receiving process.
  \item Request and reply: the sender is released after the message is received
  by the received, a service is executed at the received, and a response is
  returned to the sender.
\end{itemize}
Buffer spaces at both kernels allow for asynchronous operations to reduce
blocking and minimize deadlocks.

\subsubsection*{Pipe}
Pipes are APIs in Windows and Unix systems that allow for the transfer of
uninterrupted byte sequences. Named pipes can be shared among disjoint
processes.

\subsubsection*{Sockets and Ports}
A socket is a communication end point managed by transport services. But TCP
and UDP use socket abstraction, messages are sent to an IP address and port
number. The socket of a receiving process must be bound to a port and the IP
address of the computer. A process can use multiple ports to receive messages.

\subsection*{Connection-oriented Client/Server Socket Communication}
\begin{enumerate}
  \item Server and client issue socket calls separately.
  \item Server issues a bind call.
  \item Listen socket call: the server is willing to accept a connection.
  \item Client issues a connect call to the server.
  \item The server will reply with an accept call.
  \item The server reads the client's request and responds.
\end{enumerate}

\subsection*{Data Representation and Marshalling}
\begin{itemize}
  \item Data in ``data structures'' must be converted to a sequence of bytes.
  \item Data representation is different in different computers. This requires
  the use of an agreed upon format for transmission, since conversion will
  happen at the sender as well as the receiver.
  \item The sender's format should provide details to the recipient.
  \item Marshalling is the process of assembling data items for transmission.
\end{itemize}

\subsection*{Remote Procedure Call (RPC)}
\begin{itemize}
  \item RPC is the most popular request/reply communication process.
  \item RPCs are very similar to procedure calls and provide access
  transparency and communication transparency. It can be considered as an API
  for the transport service.
  \item RPC introduces vulnerabilities, so security features have to be built on
  top of secure RPC. This usually involves an authenticated protocol for RPC
  for mutual authentication, message integrity, confidentiality, and
  originality.
\end{itemize}

\section*{Reminders}
\noindent Professor Mohan Kumar: \\
\url{mjkvcs@rit.edu} \\
\url{https://cs.rit.edu/~mjk} \\

\noindent Rahul Dashora (TA): \\
\url{rd5476@mail.rit.edu} \\

\begin{center}
  You can find all my notes at \url{http://omgimanerd.tech/notes}. If you have
  any questions, comments, or concerns, please contact me at
  alvin@omgimanerd.tech
\end{center}

\end{document}
