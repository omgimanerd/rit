\documentclass[letterpaper, 12pt]{math}

\usepackage{listings}
\lstset{basicstyle=\ttfamily\footnotesize,breaklines=true}

\title{CSCI 251: Concepts of Parallel and Distributed Systems}
\author{Alvin Lin}
\date{September 18th, 2017}

\begin{document}

\maketitle

\section*{Topics}
\begin{itemize}
  \item Complete Bitonic Sort
  \item Shared Memory
  \item Processes and Threads
  \item POSIX (P. Threads)
  \item Project 1 Problem
\end{itemize}

\subsection*{Shared Memory}
The idea with shared memory is that you allow multiple processes to share the
same memory. Computers may have separate memory process access, but use
a logically shared address space. Concurrently, all processes can read from the
memory space with no problems, but processes that write to the memory space must
write atomically and serially. A mutual exclusion (mutex) lock is necessary for
this to happen. A semaphore is a generalized mutex that can be used to control
access to memory by processes.

\subsubsection*{Processes and Threads}
Processes:
\begin{itemize}
  \item A sequence of instructions
  \item Process ID, User ID
  \item Program instructions and PC
  \item Stack
  \item Heap
  \item Registers
  \item File descriptors
  \item Libraries
  \item Mechanisms for message passing
\end{itemize}
Threads:
\begin{itemize}
  \item Created by a process
  \item Shares process resources (file descriptors, libraries)
  \item Stack pointer
  \item Registers
  \item Lightweight
  \item Can be executed independently
  \item Scheduled for execution by the OS
\end{itemize}
Example code:
\begin{lstlisting}
  for (row = 0; row < n; row++)
    for (col = 0; col < n; col++)
      c[row][col] = dot_product(get_row(a, row), get_col(b, col))
      # it is feasible to create a thread here to do this
\end{lstlisting}

\subsubsection*{Project 1}
Refer to MyCourses for details on Project 1

\section*{Reminders}
Professor Mohan Kumar: \\
\url{mjkvcs@rit.edu} \\
\url{https://cs.rit.edu/~mjk} \\

\noindent Rahul Dashora (TA): \\
\url{rd5476@mail.rit.edu} \\

\begin{center}
  You can find all my notes at \url{http://omgimanerd.tech/notes}. If you have
  any questions, comments, or concerns, please contact me at
  alvin@omgimanerd.tech
\end{center}

\end{document}
