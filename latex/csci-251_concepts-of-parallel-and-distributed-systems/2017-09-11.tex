\documentclass[letterpaper, 12pt]{math}

\usepackage{amsmath}
\usepackage{hyperref}
\usepackage{tikz}

\title{CSCI 251: Concepts of Parallel and Distributed Systems}
\author{Alvin Lin}
\date{September 11th, 2017}

\begin{document}

\maketitle

\section*{Topics}
\begin{itemize}
  \item Matrix Multiplication
  \item Communication/Shared Memory Costs
  \item Solving Systems of Linear Equations
  \item Bitonic Sort
\end{itemize}

\subsection*{Matrix Multiplication}
We discussed the case where the number of processors is equal to the number of
rows in the matrix. In the case where the number of rows \( N \) is greater
than the number of processors \( P \), we divide \( N \) by \( P \) and assign
\( \frac{N}{P} \) rows to each process.
\[ \begin{bmatrix}A\end{bmatrix}\times\begin{bmatrix}B\end{bmatrix} =
  \begin{bmatrix}C\end{bmatrix} \]
We compute \( \frac{N}{P} \) rows of the result matrix \( C \) on each
processor. This operation is \( O(n^3) \) but parallelizing it brings its
efficiency close to 1. With \( P \) processes, the \( B \) matrix must be
available to each process in order for the computation to be performed.
\[ \begin{bmatrix}
  a_{11} & a_{12} & a_{13} & a_{14} \\
  a_{21} & a_{22} & a_{23} & a_{24} \\
  a_{31} & a_{32} & a_{33} & a_{34} \\
  a_{41} & a_{42} & a_{43} & a_{44}
\end{bmatrix}\times\begin{bmatrix}
  b_{11} & b_{12} & b_{13} & b_{14} \\
  b_{21} & b_{22} & b_{23} & b_{24} \\
  b_{31} & b_{32} & b_{33} & b_{34} \\
  b_{41} & b_{42} & b_{43} & b_{44}
\end{bmatrix} \]
To compute the first row of the result matrix, we only need the first row of
\( A \), but we need the entire \( B \) matrix. Row \( A_i \) is given to
process \( P_i \), but \( B \) must be distributed all the processes, which is
the cost of this computation.

\subsection*{Communication between Processes}
Suppose we have processors \( P_i \) and \( P_j \), between which data much be
shared. Physically, they can be separate processes on two computers that send
messages to each other. This can also be done with shared memory, a region to
which both \( P_i \) and \( P_j \) have read/write access. An important note
is that the shared memory access must be synchronized (atomic). There are many
types of interprocess communications and these are just a few. Processors can
communicate single messages, broadcast one-to-many, or communicate many-to-one.
\subsubsection*{Message Passing}
\begin{itemize}
  \item Limited by network bandwidth
  \item Limited by network topology
\end{itemize}
\subsubsection*{Shared Memory}
\begin{itemize}
  \item Limited by synchronization
  \item Limited by memory latency
\end{itemize}

\subsection*{System of Linear Equations}
\begin{align*}
  a_{11}x_1+a_{12}x_2+a_{13}x_3+a_{14}x_4 &= b_1 \\
  a_{21}x_1+a_{22}x_2+a_{23}x_3+a_{24}x_4 &= b_2 \\
  a_{31}x_1+a_{32}x_2+a_{33}x_3+a_{34}x_4 &= b_3 \\
  a_{41}x_1+a_{42}x_2+a_{43}x_3+a_{44}x_4 &= b_4
\end{align*}
This can be represented as the following matrix:
\[ \begin{bmatrix}
  a_{11} & a_{12} & a_{13} & a_{14} \\
  a_{21} & a_{22} & a_{23} & a_{24} \\
  a_{31} & a_{32} & a_{33} & a_{34} \\
  a_{41} & a_{42} & a_{43} & a_{44}
\end{bmatrix}\times\begin{bmatrix}
  x_1 \\ x_2 \\ x_3 \\ x_4
\end{bmatrix} = \begin{bmatrix}
  b_1 \\ b_2 \\ b_3 \\ b_4
\end{bmatrix} \]
\[ Ax = b \]
We can reduce this matrix to an upper triangular matrix for parallelization.
\[ \begin{bmatrix}
  1 & u_{12} & u_{13} & u_{14} \\
  0 & 1 & u_{23} & u_{24} \\
  0 & 0 & 1 & u_{34} \\
  0 & 0 & 0 & 1
\end{bmatrix}\times\begin{bmatrix}
  x_1 \\ x_2 \\ x_3 \\ x_4
\end{bmatrix} = \begin{bmatrix}
  y_1 \\ y_2 \\ y_3 \\ y_4
\end{bmatrix} \]
\begin{align*}
  x_1+u_{12}x_2+u_{13}x_3+u_{14}x_4 &= y_1 \\
  x_2+u_{23}x_3+u_{24}x_4 &= y_2 \\
  x_3+u_{34}x_4 &= y_4 \\
  x_4 &= y_4
\end{align*}
From this, we know what \( x_4 \) is and we can solve the systems of equations.
To compute the triangular matrix we do the following operation:
\[ A[i,j] := A[i,j]-A[i,k]\times A[k,j] \]

\subsubsection*{Example}
\[ \begin{bmatrix}
  3 & 4 & 2 & 7 \\
  2 & 5 & 1 & 2 \\
  1 & 2 & 3 7 8 \\
  4 & 1 & 4 & 3
\end{bmatrix}\times\begin{bmatrix}
  x_1 \\ x_2 \\ x_3 \\ x_4
\end{bmatrix} = \begin{bmatrix}
  20 \\ 14 \\ 10 \\ 8
\end{bmatrix} \]
Convert to an upper triangular matrix:
\[ \begin{bmatrix}
  1 & \frac{4}{3} & \frac{2}{3} & \frac{7}{3} \\
  0 & 1 & \dots & \dots \\
  \dots & \dots & \dots & \dots \\
  \dots & \dots & \dots & \dots
\end{bmatrix}\times\begin{bmatrix}
  x_1 \\ x_2 \\ x_3 \\ x_4
\end{bmatrix} = \begin{bmatrix}
  \frac{20}{3} \\ \dots \\ \dots \\ \dots
\end{bmatrix} \]

\subsection*{Sorting}
The basic part of a sorting algorithm is a \textbf{comparator}. A comparator,
given two inputs \( a \) and \( b \), can give either the minimum or maximum
of \( a \) and \( b \).
\begin{center}
  \begin{tikzpicture}
    \node (A) at  (0,2) {\( a \)};
    \node (A2) at (2,2) {};
    \node (A3) at (4,2) {};
    \node (A4) at (6,2) {\( Min(a,b) \)};
    \node (B) at  (0,0) {\( b \)};
    \node (B2) at (2,0) {};
    \node (B3) at (4,0) {};
    \node (B4) at (6,0) {\( Max(a,b) \)};
    \node (PLUS) at (3,1) {\( + \)};
    \draw (A2) -- (A3) -- (B3) -- (B2) -- (A2);
    \draw[->] (A) -- (A2);
    \draw[->] (B) -- (B2);
    \draw[->] (A3) -- (A4);
    \draw[->] (B3) -- (B4);
  \end{tikzpicture}
\end{center}
\begin{center}
  \begin{tikzpicture}
    \node (A) at  (0,2) {\( a \)};
    \node (A2) at (2,2) {};
    \node (A3) at (4,2) {};
    \node (A4) at (6,2) {\( Max(a,b) \)};
    \node (B) at  (0,0) {\( b \)};
    \node (B2) at (2,0) {};
    \node (B3) at (4,0) {};
    \node (B4) at (6,0) {\( Min(a,b) \)};
    \node (PLUS) at (3,1) {\( - \)};
    \draw (A2) -- (A3) -- (B3) -- (B2) -- (A2);
    \draw[->] (A) -- (A2);
    \draw[->] (B) -- (B2);
    \draw[->] (A3) -- (A4);
    \draw[->] (B3) -- (B4);
  \end{tikzpicture}
\end{center}

\section*{Reminders}
Professor Mohan Kumar: \\
\url{mjkvcs@rit.edu} \\
\url{https://cs.rit.edu/~mjk} \\

\noindent Rahul Dashora (TA): \\
\url{rd5476@mail.rit.edu} \\

We will discuss POSIX next week and the project will be handed out.

\begin{center}
  You can find all my notes at \url{http://omgimanerd.tech/notes}. If you have
  any questions, comments, or concerns, feel free to contact me at
  alvin@omgimanerd.tech
\end{center}

\end{document}
