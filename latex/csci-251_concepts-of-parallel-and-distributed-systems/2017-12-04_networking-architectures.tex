\documentclass{math}

\usepackage{tikz}

\title{CSCI 251: Concepts of Parallel and Distributed Systems}
\author{Alvin Lin}
\date{December 4th, 2017}

\begin{document}

\maketitle

\section*{Networking Architectures}
Topics:
\begin{enumerate}
  \item Principles of network applications
  \item Web and HTTP
  \item Email: SMTP, POP3, IMAP
  \item DNS
  \item P2P applications
  \item Video streaming and CDNs
  \item Socket programming with UDP and TCP
\end{enumerate}

\subsection*{Networked Applications}
Creating a networked app requires writing a program that runs on different end
systems and communicate over a network. Network core devices do not need this
software since they do not run the end user applications.

\subsubsection*{Client-server Architecture}
A client server architecture involves an always-on host with a permanent IP
address and uses data centers for scaling. Clients communicate with the server,
may be intermittently connected, may have dynamic IP addresses, and do not
communicate directly with each other.

\subsubsection*{P2P architecture}
Peer to peer architecture does not have an always on server. An arbitrary number
of end streams directly communicate with each other. Peers request and provide
service to other peers. This architecture has the property of self-scalability,
new peers bring new service capacity and meet the new service demands. The peers
may be intermittently connected and change IP addresses.

\section*{Processes}
Processes are programs running within a host. Within the same host, two
processes communicate using inter-process communication. Processes in
different hosts communicate by exchanging messages. Generally, the client
process is the one that initiates communication while the server process waits
to be contacted. Applications with P2P architectures have client and server
processes.

\subsubsection*{Sockets}
A process sends and receives messages to and from its socket. A socket is
analogous to a door: the sending process shoves a message out the door and
relies on the underlying transport infrastructure on the other side of the
door to deliver the message.

\subsubsection*{Addressing Processes}
To receive messages, a process must have an identifier. A host device has a
unique 32-bit IP address. Many processes can be running on the same host, so
the identifier includes both the IP address and port numbers associated with
the process on the host.

\subsubsection*{Messages}
The application layer protocol defines the types of messages exchanged, the
message syntax, message semantics, and rules for when and how processes send
and respond to messages. Open protocols like those defined in the IETF's
RFCs, HTTP, and SMTP allow for application interoperability.

\section*{Transport Services}
\begin{itemize}
  \item Data integrity: some applications (file transfer, web transactions)
  require 100\% reliable data transfer. Audio streaming applications, however,
  can tolerate some loss.
  \item Timing: some applications (Internet telephony, multiplayer games)
  require low delays in order to be effective.
  \item Throughput: some apps (multimedia) require a higher minimum throughput
  in order to be effective. Other apps may make use of whatever throughput they
  get.
  \item Security: some applications require encryption and data integrity.
\end{itemize}

\subsubsection*{TCP}
\begin{itemize}
  \item reliable transport between sending and receiving process
  \item flow control: sending won't overwhelm receiver
  \item congestion control: throttle sender when the network is overloaded
  \item does not provide timing, minimum throughput guarantee, and security
  \item connection-oriented: setup required between client and server processes
  \item bottlenecked by acknowledgement
\end{itemize}
For security, applications use SSL libraries that talk to the TCP layer. This
provides an encrypted TCP connection, data integrity, and end-point
authentication.

\subsubsection*{UDP}
\begin{itemize}
  \item unreliable data transfer between sending and receiving process
  \item does not provide reliability, flow control, congestion control, timing,
  throughput guarantee, security, or connection setup
\end{itemize}

\section*{HTTP/Web}
Refer to the slides on MyCourses.

\section*{Reminders}
Work on Project 2.

\noindent Professor Mohan Kumar: \\
\url{mjkvcs@rit.edu} \\
\url{https://cs.rit.edu/~mjk} \\

\noindent Rahul Dashora (TA): \\
\url{rd5476@mail.rit.edu} \\

\begin{center}
  You can find all my notes at \url{http://omgimanerd.tech/notes}. If you have
  any questions, comments, or concerns, please contact me at
  alvin@omgimanerd.tech
\end{center}

\end{document}
