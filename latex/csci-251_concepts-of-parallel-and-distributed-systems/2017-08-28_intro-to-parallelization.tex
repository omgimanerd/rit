\documentclass[letterpaper, 12pt]{math}

\title{CSCI 251: Concepts of Parallel and Distributed Systems}
\author{Alvin Lin}
\date{August 28th, 2017}

\begin{document}

\maketitle

\section*{Course Objectives}
\begin{itemize}
  \item Understand parallelism and concurrency
  \item Recognize inherent parallelism that exists in computational problems.
  \item Exploit machine parallelism and learn to write efficient and effective
    parallel programs
  \item Learn concurrent programming techniques for distributed computing
    systems
  \item Learn basics of routing and networking to better utilize parallel and
    distributed computing resources
  \item Understand basics of network security and cloud computing
\end{itemize}

\subsection*{Grading}
\begin{itemize}
  \item Quizzes (20\%)
  \begin{itemize}
    \item 10 quizzes, one every week.
    \item The best 8 quizzes will be taken into account with each counting
      2.25\%.
    \item They will be short answer, true/false, or multiple choice.
    \item May have negative grading to discourage guessing.
  \end{itemize}
  \item Projects (40\%)
  \begin{itemize}
    \item 4 projects
    \item 3 programming and one problem solving
    \item Each project is worth 10\%
  \end{itemize}
  \item Exams (40\%)
  \begin{itemize}
    \item 2 Exams (midterm and final)
    \item Each exam is worth 20\%
  \end{itemize}
  \item Homeworks (not graded directly)
  \begin{itemize}
    \item Impacts performance on quizzes and exams
    \item Discussion in MyCourses
  \end{itemize}
\end{itemize}

\subsection*{Organization}
\begin{itemize}
  \item Topics
  \begin{itemize}
    \item Main Topics: parallelism, concurrency, parallel computer systems,
      distributed computer systems, parallel programming, concurrent
      programming
    \item Additional topics: Principles of computer networks, packet routing,
      TCP/UDP, network security, cloud computing and virtualization
  \end{itemize}
  \item Teaching
  \begin{itemize}
    \item Mondays: Lecture
    \item Wednesday: Lecture, review, problem solving, questions, homeworks,
      quizzes, midterm exam
  \end{itemize}
  \item MyCourses
  \begin{itemize}
    \item Course material, important dates, schedule, news, events, homework,
      sample questions, grades, discussions, etc.
  \end{itemize}
\end{itemize}

\subsection*{General Suggestions and Advice}
\begin{itemize}
  \item Think parallel
  \item Think concurrent
  \item Attend classes
  \item Visit MyCourses
  \item Read and digest course materials
  \item Do homeworks
  \item Solve problems
  \item Do not postpone
  \item Do not carry doubts
  \item Ask questions to the instructor and fellow students
  \item Make notes
  \item Write the algorithm first, and then code
  \item Discuss with fellow students and the instructor
  \item Write your own solutions and answers
\end{itemize}

\subsection*{Parallelism}
\begin{itemize}
  \item What is parallelism?
  \begin{itemize}
    \item The objective is to execute tasks faster
    \item Processes are executed simultaneously on parallel computing elements
  \end{itemize}
  \item Data parallelism
  \begin{itemize}
    \item Data to be processed exhibits parallelism (matrix operations, image
      processing)
    \item The same task is performed on the same or different set/stream of data
  \end{itemize}
  \item Instruction parallelism
  \begin{itemize}
    \item Processors with multiple execution units
    \item Execute multiple instructions through pipelining
  \end{itemize}
  \item Task parallelism
  \begin{itemize}
    \item Parallel tasks performed, each on a different computing element
  \end{itemize}
  \item Device parallelism
  \begin{itemize}
    \item Relates to hardware
    \item Multiple cores, GPUs, parallel computers, clusters, grids, etc.
  \end{itemize}
\end{itemize}

\subsection*{Applications}
At any given moment:
\begin{itemize}
  \item How many Visa credit cards are processed?
  \item How many aircrafts are in the air?
  \item How many students are trying to enroll into a class?
  \item How many phone calls are active?
  \item How many homes/people are downloading the same movie from Netflix?
  \item How many smart phones are using the same app?
  \item How many sensors are collecting sensory information?
  \item How many apps are active on your smart phone?
\end{itemize}

\subsection*{Example: Matrix Multiplication}
\[
  \begin{bmatrix}
    a_{11} & a_{12} & a_{13} & a_{14} \\
    \dots & \dots & \dots & \dots \\
    \dots & \dots & \dots & \dots \\
    \dots & \dots & \dots & \dots
  \end{bmatrix}
  \times
  \begin{bmatrix}
    b_{11} & \dots & \dots & \dots \\
    b_{21} & \dots & \dots & \dots \\
    b_{31} & \dots & \dots & \dots \\
    b_{41} & \dots & \dots & \dots
  \end{bmatrix}
  =
  \begin{bmatrix}
    \dots & \dots & \dots & \dots \\
    \dots & \dots & \dots & \dots \\
    \dots & \dots & \dots & \dots \\
    \dots & \dots & \dots & \dots
  \end{bmatrix}
\]
This operation can be heavily parallelized.

\subsection*{MIPS and FLOPS}
\begin{itemize}
  \item MIPS - million instructions per second, CPU power to execute
    instructions
  \item MFLOPS - million floating point operations per second, used for
    computations, involves the whole system and not just the CPU. Involves
    memory, DISK, cache, bandwidth, I/O for high precision scientific
    calculations.
\end{itemize}

\subsection*{Why parallel computing?}
Single processor systems are inadequate to complete certain compute intensive
tasks, on time.
\subsubsection*{CPUs in computers}
\begin{itemize}
  \item Intel 386 DX (1985): 11 MIPS @ 33 MHz
  \item Intel Pentium Pro (1996): 541 MIPS @ 200 MHz
  \item Intel Core 2 X6800 (2006): 27079 MIPS @ 2.93 GHz
  \item Intel i7 5960X 8 Core (2015): 238,310 MIPS @ 3 GHz
\end{itemize}
\subsubsection*{Applications demand}
\begin{itemize}
  \item Several Tera/Peta FLOPS
  \item Sunway TaihLight, National Super Computer Center, China: 10,649,600
    cores, 93 Peta FLOPS
  \item Titan Cray XK7, ORNL: 560,640 cores, 17.5 Peta FLOPS
\end{itemize}

\subsection*{Concurrency}
What is concurrency? Concurrency is the potential for parallelism. It provides
resource access to multiple processes and threads. It involves coordination,
sharing, and synchronization. \textbf{Concurrency control} deals with correct
and efficient access by multiple threads to shared resources because
concurrently executing programs need to coordinate.

\subsection*{Distributed Computing}
\begin{itemize}
  \item Concurrent components are independent and communicate and coordinate
    through message passing or shared memory.
  \item The lack of a global clock causes problems because each component has
    its own clock and synchronization is a huge issue.
  \item Components fail independently. They are isolated from one another and
    redudant, and should not be visible to the end user.
\end{itemize}
\subsubsection*{Characteristics}
\begin{itemize}
  \item Heterogeneous
  \item Transparent
  \item Secure
  \item Privacy preserving
  \item Scalable
  \item Fault-tolerant
  \item Concurrent
\end{itemize}
\subsubsection*{Examples}
\begin{itemize}
  \item The Internet
  \item Sensor Systems
  \item P2P Systems
  \item Airlines
  \item Aircraft
  \item Cars
\end{itemize}

\subsection*{Reminders}
Professor Mohan Kumar (Professor and Chair, CS Department): \\
\url{mjkvcs@rit.edu} \\
\url{https://cs.rit.edu/~mjk} \\
Include ``CSCI 251'' in your message header. \\

\noindent Jennifer Burt (Additional Contact): \\
\url{jennifer@cs.rit.edu} \\

There is no single textbook for this course. Check MyCourses for appropriate
texts and references.

\subsection*{Homework}
Identify parallelism in different daily activities and read the
``Dining Philosophers'' problem.

\begin{center}
  You can find all my notes at \url{http://omgimanerd.tech/notes}. If you have
  any questions, comments, or concerns, please contact me at
  alvin@omgimanerd.tech
\end{center}

\end{document}
