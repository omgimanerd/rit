\documentclass{math}

\title{CSCI 251: Concepts of Parallel and Distributed Systems}
\author{Alvin Lin}
\date{September 13th, 2017}

\begin{document}

\maketitle

\section*{Gaussian Elimination Problem}
Gaussian elimination uses the elementary array operations to turn a matrix
into row echelon form.
\[ \begin{bmatrix}
  4 & 3 & 2 & 5 \\
  6 & 7 & 8 & 2 \\
  3 & 1 & 5 & 6 \\
  7 & 8 & 3 & 4
\end{bmatrix} \]
\[ \frac{1}{4}R_1\to R_1 \]
\[ \begin{bmatrix}
  1 & \frac{3}{4} & \frac{2}{4} & \frac{5}{4} \\
  6 & 7 & 8 & 2 \\
  3 & 1 & 5 & 6 \\
  7 & 8 & 3 & 4
\end{bmatrix} \]
\[ -6R_1+R_2\to R_2 \]
\[ \begin{bmatrix}
  1 & \frac{3}{4} & \frac{2}{4} & \frac{5}{4} \\
  0 & 7-\frac{18}{4} & 8-\frac{12}{4} & 2-\frac{30}{4} \\
  3 & 1 & 5 & 6 \\
  7 & 8 & 3 & 4
\end{bmatrix} \]
\[ \begin{bmatrix}
  1 & \frac{3}{4} & \frac{2}{4} & \frac{5}{4} \\
  0 & \frac{5}{2} & 5 & \frac{-11}{2} \\
  3 & 1 & 5 & 6 \\
  7 & 8 & 3 & 4
\end{bmatrix} \]

\section*{Bitonic Sort}
\textbf{Compare Exchange}: With two processors \( P_i \) and \( P_j \), this
operation causes them to exchange their respective values \( x_i \) and
\( x_j \) with \( P_i \) maintaining \( max(x_i,x_j) \) and \( P_j \)
maintaining \( min(x_i,x_j) \). \\
\textbf{Compare Split}: With two processors \( P_i \) and \( P_j \), the
two processors take \( \frac{N}{P} \) elements to sort individually, and
exchange them at the end of their respective sorts. In the case where
\( N >> P \), this has a complexity \( O(\frac{N}{P}\log_2\frac{N}{P}) \).

\subsection*{Example}
Suppose you have two processors \( P_i \) and \( P_j \) and in the splitting
the input, \( P_i \) receives [2, 8, 6, 4] and \( P_j \) receives [7, 3, 1, 5].
They will each respectively sort their sublists:
\[ P_i: [2, 4, 6, 8] \]
\[ P_j: [1, 3, 5, 7] \]
And then perform a compare split:
\[ P_i: [1, 2, 3, 4] \]
\[ P_j: [5, 6, 7, 8] \]

\subsection*{Bitonic Sequence}
A bitonic sequence is a sequence containing an ascending sequence followed by a
descending sequence.
\[ a_1,a_2,a_3,a_4,\dots,a_{\frac{N}{2}} \]
\[ a_{\frac{N}{2}+1},a_{\frac{N}{2}+2},\dots,a_{N} \]
You can merge elements of a bitonic sequence in \( \log_2 n \) steps to get a
sorted list. Each step has \( \frac{N}{2} \) swaps.
\[ [18,36,72,90,52,39,26,13] \]
In the above example, the data size is 8. \( N = 8 = 2^k, k = 3 \). We compare
element \( i \) with element \( i+2^{k-1} \) from 1 to \( 2^{k-1} \).
If element \( i \) is less than element \( i+2^{k-1} \), we exchange them.
This results in:
\[ [18,36,26,13] [52,39,72,90] \]
Between the two groups on the two processors, we compare element \( i \) with
\( i+2^{k-2} \), resulting in:
\[ [18,13] [26,36] [52,39] [72,90] \]
We then compare element \( i \) with element \(i+2^{k-2} \):
\[ [13,18,26,36,39,52,72,90] \]
Note that this split-comparison process can be done recursively in \( \log n \)
time. This is better than doing an \( O(n) \) merge on the bitonic sequence.

\subsubsection*{Example}
Turning an array of numbers into a bitonic sequence:
\[ \begin{array}{lclclcl}
  98  & \to & 28  & \to & 28  & \to & 28  \\
  28  & \to & 98  & \to & 56  & \to & 56  \\
  70  & \to & 70  & \to & 70  & \to & 70  \\
  56  & \to & 56  & \to & 98  & \to & 98  \\
  42  & \to & 42  & \to & 84  & \to & 112 \\
  14  & \to & 14  & \to & 112 & \to & 84  \\
  112 & \to & 84  & \to & 42  & \to & 42  \\
  86  & \to & 112 & \to & 14  & \to & 14  \\
\end{array} \]

\section*{Reminders}
Professor Mohan Kumar: \\
\url{mjkvcs@rit.edu} \\
\url{https://cs.rit.edu/~mjk} \\

\noindent Rahul Dashora (TA): \\
\url{rd5476@mail.rit.edu} \\

\begin{center}
  You can find all my notes at \url{http://omgimanerd.tech/notes}. If you have
  any questions, comments, or concerns, please contact me at
  alvin@omgimanerd.tech
\end{center}

\end{document}
