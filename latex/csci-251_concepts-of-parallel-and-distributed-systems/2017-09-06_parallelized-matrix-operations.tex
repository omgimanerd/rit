\documentclass{math}

\title{CSCI 251: Concepts of Parallel and Distributed Systems}
\author{Alvin Lin}
\date{September 6th, 2017}

\begin{document}

\maketitle

\section*{Topics}
\begin{itemize}
  \item Speedup, Efficiency, Scalability
  \item Matrix-Vector Operations
  \item Matrix-Matrix Operations
  \item Solving Systems of Linear Equations
\end{itemize}

\subsection*{Speedup, Efficiency, Scalability}
These equations model parallel addition:
\[ T_p = \frac{N}{P}+2\log P \]
\( 2\log P \) represents the cost of communication.
\[ \text{Speedup} = \frac{N}{\frac{N}{P}+2\log_2 P} \]
\[ \text{Efficiency} = \frac{\text{Speedup}}{P} =
  \frac{1}{1+\frac{2P\log_2 P}{N}} \]

\subsubsection*{Examples}
Suppose we are adding 64 data units with 4 processors:
\[ E = \frac{1}{1+\frac{(2)(4)(2)}{64}} = \frac{1}{1.25} = 0.8 \]
If we keep the data size constant but 16 processors:
\[ E = \frac{1}{1+\frac{(2)(16)(4)}{64}} = \frac{1}{3} = 0.333 \]
Efficiency does not increase proportionally to the number of processors if the
data size remains the same. Suppose \( P = 4 \) and \( N = 128 \):
\[ E = \frac{1}{1+\frac{(2)(4)(2)}{128}} = \frac{1}{1.125} = 0.8888 \]
To find the optimal number of processors for the optimal \( T_p \):
\[ \ddiff{T_p}{P} = 0 \]
\[ P = \frac{N}{2} \]

\subsection*{Matrix-Vector Operations}
Let \( A \) be a matrix of size \( N\times N \), \( x \) is a vector of size
\( N \), and \( y \) be the resultant vector of size \( N \).
\[ Ax = y \]
\[ \begin{bmatrix}
    a_{11} & a_{12} & a_{13} & \dots & a_{1N} \\
    a_{21} & a_{22} & a_{23} & \dots & a_{2N} \\
    \vdots & \vdots & \vdots & \vdots & \vdots \\
    a_{N1} & a_{N2} & a_{N3} & \dots & a_{NN}
  \end{bmatrix}\times\begin{bmatrix}
    x_1 \\ x_1 \\ \vdots \\ x_N
  \end{bmatrix} = \begin{bmatrix}
    y_1 \\ y_1 \\ \vdots \\ y_N
  \end{bmatrix}
\]
\[ y_i = \sum_{i,j=1}^{N}(a_{ij})(x_j) \]
\begin{align*}
  a_{11}x_1+a_{12}x_2+a_{13}x_3+\dots+a_{1N}x_N &= y_1
    \quad \Theta(N)\text{ multiplication} \\
  \dots &= y_2 \\
  \dots &= y_2 \\
  \vdots \\
  \dots &= y_N
\end{align*}
This entire process is \( \Theta(N^2) \). To compute each \( y_i \), we have
\( \Theta(N) \) multiplications and \( N-1 \) additions, but we will not be
taking into account the \( N-1 \) additions because they are outshadowed by the
computation time of the multiplications (which are simply repeated additions).
To compute the vector \( y \), we need \( \Theta(N^2) \) multiplications.

\subsubsection*{Example}
If we consider the case where \( N = P \). We assign one row of \( A \) to one
processor.
\[ Row_i = P_i \]
Each \( P_i \) will compute their corresponding \( y_i \). All \( P_i \)'s will
work in parallel here, making \( T_p = \Theta(N) \) and speedup
\( = \frac{\Theta(N^2)}{\Theta(N)} = \Theta(N) \)
This does not take into account the cost of communication. For this example,
each processor needs to have a copy of vector \( x \). It needs to be broadcast
to all other processes.

\subsubsection*{Example}
Suppose however, that \( N \) is greater than \( P \). Suppose we have 16 rows
and 4 processors, \( P_1 \) will have \( \frac{N}{P} \) vectors to compute,
making this process \( \Theta(N\frac{N}{P}) \).

\subsection*{Matrix-Matrix Operations}
\[ \begin{bmatrix}A\end{bmatrix}\times
  \begin{bmatrix}B\end{bmatrix} = \begin{bmatrix}C\end{bmatrix} \]
Let \( A \) and \( B \) be matrices of size \( N\times N \)
\[ C_{ij} = \sum_{i,j=0}^{N}a_{ij}b_{ji} \]
This is a computation that is \( \Theta(N^3) \). The matrix \( B \) must be
broadcast to all processors.

\section*{Reminders}
Professor Mohan Kumar: \\
\url{mjkvcs@rit.edu} \\
\url{https://cs.rit.edu/~mjk} \\

\noindent Rahul Dashora (TA): \\
\url{rd5476@mail.rit.edu}

\begin{center}
  You can find all my notes at \url{http://omgimanerd.tech/notes}. If you have
  any questions, comments, or concerns, please contact me at
  alvin@omgimanerd.tech
\end{center}

\end{document}
