\documentclass{math}

\usepackage{tikz}

\title{CSCI 251: Concepts of Parallel and Distributed Systems}
\author{Alvin Lin}
\date{November 29th, 2017}

\begin{document}

\maketitle

\section*{Distributed Systems}
Outline:
\begin{itemize}
  \item Bitcoin and blockchain
  \item No block, no chain - directed acyclic graph
  \item Distributed consensus algorithm: proof-of-stake,
  proof-of-space-and-time, proof-of-retrievability
  \item Smart contracts
\end{itemize}

\subsection*{How to Use Bitcoins}
\begin{enumerate}
  \item Download software to your computer or phone to set up a Bitcoin wallet.
  This gives you the basic facilities to send, receive, and store Bitcoins.
  \item Your software will generate a unique sting of letters and numbers: your
  Bitcoin address. The address isn't tied to your name or any other personal
  data, but it identifies you to the Bitcoin network. Give this address to
  anyone who needs to pay you.
  \item Buy Bitcoins with a standard offline currency, either from another user
  or through a dedicated Bitcoin exchange. Your new digital funds are added to
  your wallet.
  \item The Bitcoin network authenticates transactions by recording them in the
  `block chain' - the underlying code that preserves the integrity of the
  currency.
  \item Use your software to send payments to other addresses. Divisions as
  small as 100,000,000th of a Bitcoin are possible - a unit called a
  `Satoshi', after the currency's enigmatic inventor.
\end{enumerate}

\subsection*{Problems of Digital Tokens}
\begin{itemize}
  \item Ownership: who owns the digital currency?
  \item Authorization: who authorizes a transaction?
  \item Double Spending: what prevents a token from being transferred to
  multiple owners by erasing a transaction?
\end{itemize}

\subsection*{The Bitcoin Transaction Life Cycle}
\begin{enumerate}
  \item Person A opens his bitcoin wallet, scans/copies Person B's address,
  and sends a specific amount in a transaction.
  \item The wallet signs the transaction using Person A's private key.
  \item The transaction is propagated and validated by the network nodes.
  \item Miners include the transaction in the next block to be mined.
  \item The miner who solves the Proof of Work propagates the new block to the
  network.
  \item The nodes verify the result and propagate the block.
  \item Person B sees the first confirmation.
  \item New confirmations appear with each new block that is created.
\end{enumerate}

\subsection*{Key Elements}
\begin{itemize}
  \item Cryptographic Techniques: hash function (SHA256), public and private
  keys; the hash of the private key is the address for receiving money;
  pseudo-identity; digital signature/certificate
  \item Data structure to record ``logs''
  \item Consensus algorithms: proof-of-work
\end{itemize}

\subsection*{Proof of Work}
A nonce is appended to the previous hash to produce a new hash. A miner that
guesses the correct nonce can broadcast the new block to the chain.

\section*{Reminders}
Work on Project 2.

\noindent Professor Mohan Kumar: \\
\url{mjkvcs@rit.edu} \\
\url{https://cs.rit.edu/~mjk} \\

\noindent Rahul Dashora (TA): \\
\url{rd5476@mail.rit.edu} \\

\begin{center}
  You can find all my notes at \url{http://omgimanerd.tech/notes}. If you have
  any questions, comments, or concerns, please contact me at
  alvin@omgimanerd.tech
\end{center}

\end{document}
