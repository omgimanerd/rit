\documentclass{math}

\usepackage{listings}
\lstset{basicstyle=\ttfamily\footnotesize,breaklines=true}

\title{CSCI 251: Concepts of Parallel and Distributed Systems}
\author{Alvin Lin}
\date{September 18th, 2017}

\begin{document}

\maketitle

\section*{Topics}
\begin{itemize}
  \item Race conditions and mutexes
  \item Overheads
  \item Load balancing
\end{itemize}

\subsection*{Race Conditions}
Suppose we are finding the number of primes in a given range. If we are
checking if a number \( n \) is prime, we can check for factors up to
\( \sqrt{n} \) as a simple optimization. If we have a function
\texttt{isPrime(n)} to be distributed across 2 threads, we can divide \( n \)
into two ranges for checking, one range being 3 to \( \frac{n}{2} \), and the
second range being \( \frac{n}{2} \) to \( n \).
\begin{lstlisting}
#include <pthread.h>

pthread_t t1,t1;
struct prime_arg t1arg = {3,n/2};
struct prime_arg t2arg = {n/2+1,n};
pthread_create(&t1, NULL, prime_thread, &t1arg);
pthread_create(&t2, NULL, prime_thread, &t2arg);
...
pthread_join(t1, NULL);
pthread_join(t2, NULL);
\end{lstlisting}
Suppose these threads increment a \texttt{pcount} variable to count primes,
a race condition can occur if two threads are trying to read and write to a
single piece of memory at the same time (namely, the \texttt{pcount} variable).
Race conditions cause unpredictable behavior.

\subsection*{Mutex Locks}
A mutex lock ensures that two threads cannot simultaneously attempt to update
the \texttt{pcount} variable. When a thread acquires a lock, it ensures that
operation to be performed is atomic.
\begin{lstlisting}
p_thread_mutex_lock(&mutex);
pcount++;
p_thread_mutex_unlock(&mutex);
\end{lstlisting}
This is a problem however, since the mutex becomes a bottleneck for the threads.
We can remove the need for a mutex lock by having each thread maintain a local
count and accumulating at the end.

\subsection*{Load Balancing}
In the previous example, dividing the input range \( n \) among two threads is
a naive solution since one thread may finish first. This solution is not load
balanced since one thread will be idle at the end of its set of operations.

\subsection*{Programming Guidelines}

\subsubsection*{Memory}
Every memory location in your program must meet at least one
of the following criteria:
\begin{itemize}
  \item Only one thread can access it.
  \item It is immutable.
  \item It is synchronized (locks are used to avoid race conditions).
\end{itemize}

\subsubsection*{Synchronization}
\begin{itemize}
  \item Avoid data races, use locks to prevent it.
  \item Use consistent locking, each shared location that needs synchronization
    should have a corresponding lock.
  \item Coarse grained vs fine grained locks, you can lock memory locations
    individually or lock the entire block, depending on your needs.
  \item Critical sections, the cost of the critican section is given by the
    time to execute the critical section and the resources used by the OS.
\end{itemize}

\section*{Reminders}
The midterm is on October 11th.
Refer to MyCourses for details on Project 1. \\

\noindent Professor Mohan Kumar: \\
\url{mjkvcs@rit.edu} \\
\url{https://cs.rit.edu/~mjk} \\

\noindent Rahul Dashora (TA): \\
\url{rd5476@mail.rit.edu} \\

\begin{center}
  You can find all my notes at \url{http://omgimanerd.tech/notes}. If you have
  any questions, comments, or concerns, please contact me at
  alvin@omgimanerd.tech
\end{center}

\end{document}
