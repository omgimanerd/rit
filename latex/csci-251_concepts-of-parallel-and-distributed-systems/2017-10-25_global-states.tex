\documentclass{math}

\title{CSCI 251: Concepts of Parallel and Distributed Systems}
\author{Alvin Lin}
\date{October 23rd, 2017}

\begin{document}

\maketitle

\section*{Global States}
A cut is nothing more than a snapshot of a distributed system. For a cut to be
consistent, the events and messages must be a in state such that there is no
state that depends on a state not inside the cut.

\subsection*{Monitor Process}
Assumtions:
\begin{itemize}
  \item During the snapshot operation, there are no faults.
  \item There is a finite time for messages to flow from one process to another.
  \item Any process can start the monitoring process.
\end{itemize}

\subsection*{Chandy \& Lampon Algorithm for Monitoring}
\begin{enumerate}
  \item Marker message is sent by the initiating process.
  \item The marker process records its state and sends a marker message
  \( m \) on every outgoing channel
\end{enumerate}

\section*{Reminders}
Check MyCourses for details on Project 2. \\
\noindent Professor Mohan Kumar: \\
\url{mjkvcs@rit.edu} \\
\url{https://cs.rit.edu/~mjk} \\

\noindent Rahul Dashora (TA): \\
\url{rd5476@mail.rit.edu} \\

\begin{center}
  You can find all my notes at \url{http://omgimanerd.tech/notes}. If you have
  any questions, comments, or concerns, please contact me at
  alvin@omgimanerd.tech
\end{center}

\end{document}
