\documentclass[letterpaper, 12pt]{math}

\usepackage{listings}
\lstset{basicstyle=\ttfamily\footnotesize,breaklines=true}
\usepackage{tikz}

\title{CSCI 251: Concepts of Parallel and Distributed Systems}
\author{Alvin Lin}
\date{September 27th, 2017}

\begin{document}

\maketitle

\section*{Topics}
\begin{itemize}
  \item Exercise Problems
  \item Challenges in Parallel Programming
\end{itemize}

\subsection*{Costs of Communication}
We want to minimize the idling time of a processor. The cost of communication
can depend on the size of the message being passed. We need to keep in mind
the concept of granularity in our programs, whether or not we have coarse grain
parallelism or fine grain parallelism. \par
\textbf{Fine grained parallelism} involves a low ratio of computation to
communication, with very high levels of parallelism. Often, this entails small
tasks distributed and communicated between many processors. \par
\textbf{Coarse grained parallelism} involves a high ratio of computation to
communication. This will have very low overhead since each processor will tend
to do large amounts of computation and less communication. \par
The measure of the ratio between communication and computation is relative since
it depends on the CPU and communication network architecture. One also needs to
take into account data transmission latency and bandwidths as an additional
bottleneck when analyzing parallel programs. \par
\textbf{In degree} refers to how long a processor has to wait before it starts
computing. \par
\textbf{Out degree} refers to how much is spent in terms of communication.

\section*{Reminders}
The midterm is on October 11th.
Refer to MyCourses for details on Project 1. \\

\noindent Professor Mohan Kumar: \\
\url{mjkvcs@rit.edu} \\
\url{https://cs.rit.edu/~mjk} \\

\noindent Rahul Dashora (TA): \\
\url{rd5476@mail.rit.edu} \\

\begin{center}
  You can find all my notes at \url{http://omgimanerd.tech/notes}. If you have
  any questions, comments, or concerns, please contact me at
  alvin@omgimanerd.tech
\end{center}

\end{document}
