\documentclass{math}

\title{Analysis of Algorithms}
\author{Alvin Lin}
\date{August 2017 - December 2017}

\begin{document}

\maketitle

\section*{Participation in a Programming Contest}
\textbf{Contest:} Google Foobar Challenge \\
\textbf{Date:} 12/2/2017 \\
\textbf{Website:} \url{https://google.com/foobar}
(Only accessible if prompted) \\
The Google Foobar challenge is a challenge that comes up if you frequently
search for code related material on google. While not strictly a programming
contest, you are presented with 5 problems of increasing difficulty on a secret
portal accessible only through the search prompt. The legends say that if you
complete all 5 challenges, Google will extend a job offer to you. \par
During the weekend, I completed the first two challenges, which were relatively
simple data structure questions designed to filter out beginner coders. While
not difficult, the second problem forced me to think a bit because it seemed
naively easy but had a execution time constraint. The third problem involved
XORing a specific sequence of numbers together, which upfront also seemed
naively easy. This problem also had a time constraint on it to prevent you from
solving it the naive way. Unlike the algorithms questions in class, this problem
required you to know clever math tricks and properties of XOR so that you could
deterministically calculate parts of the problem instead of using a loop to XOR
all the numbers of the sequence together. I had a general idea of how to do it,
but ultimately ran out of time for the problem because of homework and
obligations to other classes. Having participated in CTFs before (crypto-themed
programming contests), I had a vague idea of the scope and objective of this
challenge. Overall, it was quite fun and gave me an excellent excuse to
procrastinate, so I'll probably retry the third problem in the future.

\begin{center}
  You can find all my notes at \url{http://omgimanerd.tech/notes}. If you have
  any questions, comments, or concerns, please contact me at
  alvin@omgimanerd.tech
\end{center}

\end{document}
