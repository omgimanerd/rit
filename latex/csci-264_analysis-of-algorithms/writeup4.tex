\documentclass{math}

\usepackage{listings}
\lstset{basicstyle=\ttfamily\footnotesize,breaklines=true}
\title{Analysis of Algorithms}
\author{Alvin Lin (axl1439) and William Leuschner (wel2138)}
\date{August 2017 - December 2017}

\begin{document}

\maketitle

\section*{Problem 2}

\subsubsection*{Algorithm Description}
For this problem, we implemented a dynamic programming algorithm to find the
maximum cost attainable with a sample of items placed in two backpacks. The
heart of the solution is described as follows:
\begin{align*}
  S[k][w_1][w_2] =~& \text{the maximum cost of a subset of the first } \\
  & k \text{ items where the weight of items in the subset is} \\
  & \text{distributed in two backpacks with weight limits } w_1w_2 \\[3em]
  S[k][w_1][w_2] =& \begin{cases}
    0 & if~k = 0, w_1 = 0, w_2 = 0 \\[1em]
    \begin{aligned}
      \texttt{max}(&S[k-1][w_1][w_2], \\
      & c_k+S[k-1][w_1-w_k][w_2], \\
      & c_k+S[k-1][w_1][w_2-w_k])
    \end{aligned} & if~w_k\le w_1\wedge w_k\le w_2 \\[2em]
    \begin{aligned}
      \texttt{max}(&S[k-1][w_1][w_2], \\
      & c_k+S[k-1][w_1-w_k][w_2])
    \end{aligned} & if~w_k\le w_1 \\[2em]
    \begin{aligned}
      \texttt{max}(&S[k-1][w_1][w_2], \\
      & c_k+S[k-1][w_1][w_2-w_k])
    \end{aligned} & if~w_k\le w_2 \\[2em]
    S[k-1][w_1][w_2] & otherwise \\
  \end{cases}
\end{align*}
With \( n \) items and respective weight limits of \( W_1,W_2 \), the solution
is thus yielded by:
\[ S[n][W_1][W_2] \]
This algorithm works by trying to place the item into the two bags. If it fits
in both, then it will use the maximum cost attained by placing it either, or
not placing it at all. If it only fits it one, then it takes the maximum cost
attained by placing it in that bag or not placing it in the bag.

\subsubsection*{Running Time Estimate}
Since the dynamic programming solution matrix is a three dimensional array with
dimensions \( n\times W_1\times W_2 \), our running time is \( O(nW_1W_2) \)
since we need to populate the entire solution matrix.

\subsubsection*{Psuedocode}
\begin{lstlisting}
def solve(items, W1, W2):
  let n = items.length
  let s = matrix of size (n+1) x (W1+1) x (W2+1)
  for k = 0 to n + 1:
    for w1 = 0 to W1:
      for w2 = 0 to W2:
        if k == 0:
          s[k][w1][w2] = 0
        else:
          let item = items[k - 1]
          let previous = s[k - 1]
          if item.weight <= w1 and item.weight <= w2:
            solution[k][w1][w2] = max(
              previous[w1][w2],
              item.cost + previous[w1 - item.weight][w2],
              item.cost + previous[w1][w2 - item.weight]
            )
          else if item.weight <= w1:
            solution[k][w1][w2] = max(
              previous[w1][w2],
              item.cost + previous[w1 - item.weight][w2]
            )
          else if item.weight <= w2:
            solution[k][w1][w2] = max(
              previous[w1][w2],
              item.cost + previous[w1][w2 - item.weight]
            )
          else:
            solution[k][w1][w2] = previous[w1][w2]
  return solution[n][W1][W2]
\end{lstlisting}

\section*{Problem 3}

\subsubsection*{Algorithm Description}
For this problem, we extended it off of problem 2's dynamic programming solution
matrix. We iterate backwards over the items array and pull out the item's
weight and cost and the current value in the solution matrix for that item. We
can subtract the value in the solution matrix from the previous value in the
solution matrix to determine if the item was included in the cost calculation
during the construction of the solution matrix, and if it was, we can add it to
the knapsack as a item included in the valid solution.

\subsubsection*{Argument of Correctness}
This algorithm reverses the construction of the optimal cost entry in the
solution matrix, so we are guaranteed that the items it produces is part of the
optimal set of items to maximize cost. By subtracting entries in the solution
matrix, we can determine the cost of the item that was added.

\subsubsection*{Running Time Estimate}
Since this algorithm relies on the solution matrix to construct the solution,
it is \( O(nW_1W_2) \). The solution construction simply needs to go through
every item, so the construction is \( O(n) \). The entire runtime is upper
bounded by the time it takes to generate the solution matrix.

\subsubsection*{Pseudocode}
\begin{lstlisting}
def solve(solution, items, W1, W2):
  let bag1 = []
  let bag2 = []
  let w1 = W1
  let w2 = W2
  for i = items.length - 1 to 0:
    let item = items[i]
    let current = solution[i + 1][w1][w2]
    let prev = solution[i]
    if w1 >= item.weight and
        current == item.cost + prev[w1 - item.weight][w2]:
      bag1.append(i + 1)
      w1 -= item.weight
    elif w2 >= item.weight and
        current == item.cost + prev[w1][w2 - item.weight]:
      bag2.append(i + 1)
      w2 -= item.weight
  return [bag1, bag2]
\end{lstlisting}

\section*{Problem 4}

\subsubsection*{Algorithm Description}
For this problem, we implemented a dynamic programming algorithm to solve for
the minimum number of operations needed to multiply the optimally parenthesized
groups of matrices. While populating the solution array, we also stored the
groups of parenthesized matrices so that we could recursively reconstruct the
solution. The heart of the solution is as follows:
\begin{align*}
  S[l][r] =~ & \text{the minimum number of steps required to multiply the} \\
  & \text{matrices from index } l \text{ to } r \text{ and the grouping that} \\
  & \text{yielded the resulting minimum number of steps}\\
  S[l][r] =~ &
    min(S[l,k] + S[k+1][r] + a_{l-1}a_ka_r)~\forall{k} (1\le k\le r-1)
    \text{and} [l, k, r] \\
\end{align*}
The final result for the minimum number of steps is given by:
\[ S[1][n] \]
We can take the \( [l,k,r] \) grouping stored at \( S[1][n] \) and recursively
expand it into solution string.

\subsubsection*{Running Time Estimate}
Populating the solution matrix requires \( O(n^3) \) time, but recursively
reconstructing the solution only requires \( O(n) \) time because a recursive
expansion can occur at most \( n \) times.

\subsubsection*{Pseudocode}
\begin{lstlisting}
def solve(dimensions):
  let n = dimensions.length - 1
  let solution = list of length n
  let groupings = matrix of size n x 3
  for d = 1 to n:
    for l = 0 to n - d:
      let r = l + d
      solution[l][r] = infinity
      for k = l to r:
        let tmp = solution[l][k] + solution[k + 1][r] +
          dimensions[l] * dimensions[k + 1] * dimensions[r + 1]
        if tmp < solution[l][r]:
          solution[l][r] = tmp
          groupings[l][r] = [l, k, r]
  return solution, groupings

def reconstruct(groupings):
  let n = groupings.length
  return reconstructRecursive(groupings, groupings[0][n - 1])

def reconstructRecursive(groupings, solutionGroup):
  let l, k, r = solutionGroup
  if l + 1 == r:
    return "( A%d x A %d )".format(l + 1, r + 1)
  elif l == k:
    return "( A%d x %s )".format(l + 1,
      reconstructRecursive(groupings[k + 1][r]))
  elif k + 1 == r:
    return "( %s x A%d )".format(
      reconstructRecursive(groupings[l][k]), r + 1)
  else:
    return "( %s x %s )".format(
      reconstructRecursive(groupings[l][k]),
      reconstructRecursive(groupings[k + 1][r]))
\end{lstlisting}

\end{document}
