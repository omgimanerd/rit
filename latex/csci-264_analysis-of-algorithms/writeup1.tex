\documentclass[letterpaper, 12pt]{math}

\usepackage{amsmath}
\usepackage{amssymb}
\usepackage{hyperref}
\usepackage{listings}
\lstset{basicstyle=\ttfamily\footnotesize,breaklines=true}

\title{Analysis of Algorithms}
\author{Alvin Lin and William Leuschner}
\date{September 10th, 2017}

\begin{document}

\maketitle

\section*{Problem 1}
Least to greatest:
\begin{itemize}
  \item Constant Time:
  \[ 3 \quad n^{\frac{1}{\log n}} \]
  \item Logarithmic:
  \[ \quad \log\log n \quad \log_8 n \quad \log_2 n \quad \log_2 n^3 \quad
    (\log n)^2 \]
  \item Sub-linear:
  \[ n^{\frac{1}{2}} \quad \frac{n}{\log n} \]
  \item Linear:
  \[ n-\log n \quad n\log n \]
  \item Polynomial:
  \[ n^2 \quad n^2\log n \quad n^{\log\log n} \quad (\log n)^{\log n} \quad
    n^2+10^{100}n\log n \quad n^4 \]
  \item Exponential:
  \[ 4^{\log n} \quad 4^{2n} \quad n! \quad 8^{n-1} \quad n^n\quad 8^n \quad
    4^{n^2} \quad \]
\end{itemize}

\section*{Problem 2}
\begin{lstlisting}[mathescape=true]
let n1 = the list of she-aliens
let n2 = the list of he-aliens
let c = n1 $\cup$ n2
let d = []
for alien in c:
    if alien.preferences is empty:
        remove alien from c
while c is not empty:
    let current = min(c, by length of alien.preferences)
    if current.preferences is empty:
        remove current from c
    let match = random.choice(current.preferences)
    add (current, match) to d
    remove current from c
    remove match from c
    for alien in c:
        remove current from alien.preferences
        remove match from alien.preferences
\end{lstlisting}
This function is \( O(n^2) \) because while iterating through the list of
aliens to find matches, we must go through all other aliens if we find a match
in order to remove the matches from the other aliens' preferences.

\section*{Problem 3}

\subsubsection*{Algorithm Description}
For this problem, we needed fast lookup and insertion. A self-balancing binary
search tree was the best choice for this, so we implemented an AVLTree.
While iterating through the array of numbers, we look in our AVLTree for both
numbers that would result in a difference of \( t \), that is, we look for
\( a_i+t \) and \( a_i-t \).

\subsubsection*{Argument of Correctness}
Our algorithm examines every element in the input and calculates every
possible pairing that could produce \( t \) to search for them.

\subsubsection*{Running Time Estimate} Since iteration through the numbers is
\( O(n) \) and we must perform an insertion and 2 lookups for each number, our
execution time is \( n\times3\log(n) \), which is \( O(n\log n) \).

\subsubsection*{Pseudocode}
\begin{lstlisting}
let avlt = an new AVLTree (self-balancing)
let count = 0
for i=0 to the length of the input:
    let possibility1 = avlt.search(input[i]+t)
    let possibility2 = avlt.search(input[i]-t)
    if possibility1 exists:
        count += possibility1.frequency
    if possiblilty2 exists and t != 0:
        count += possibility2.frequency
    let currentValue = avlt.search(input[i])
    if currentValue exists:
        currentValue.frequency += 1
    else:
        avlt.insert(currentValue)
return count
\end{lstlisting}

\section*{Problem 4}

\subsubsection*{Algorithm Description}
For this problem, we used the AVLTree implemented in problem 3 to keep track
of all the \( t \) differences. We computed the difference \( t \) between every
pair of input values and stored each \( t \) difference and its frequency as
a tuple inside the tree.

\subsubsection*{Argument of Correctness}
This algorithm finds every possible \( t \) difference by computing the
difference between every single pair of inputs. It uses an AVLTree to store the
differences for fast lookup and update. Because we have every possible \( t \)
difference, we can find the most frequently occurring one.

\subsubsection*{Running Time Estimate}
Finding every possible \( t \) difference is \( O(n^2) \). Storing them in the
tree requires \( O(\log n) \) time, and finding the maximum requires \( O(n) \)
time. This makes our complexity \( n^2*2\log(n)+n \), which is
\( O(n^2\log n) \).

\subsubsection*{Pseudocode}
\begin{lstlisting}
let avlt = an new AVLTree (self-balancing)
for i=0 to the length of the input:
    for j=i to the length of the input:
        let t = input[i] - input[j]
        let possibility = avlt.search(t)
        if possibility exists:
            possibility.frequency += 1
        else:
            avlt.insert(t)
let max = 0
let stack = a new stack
stack.push(avlt.root)
while stack is not empty:
    let current = stack.pop()
    if current.frequency > max.frequency:
        max = current
    stack.push(current.left)
    stack.push(current.right)
return max.value
\end{lstlisting}

\section*{Problem 5}

\subsubsection*{Algorithm Description}
For this problem, we first found the lexical difference between the two strings
as an array of differences between each letter. This requires only a single
pass. This list of differences can be represented as a series of
heights/depths in a ``mountain range''. We will refer to the letter distances
as heights, and the contiguous groups of letters as ``mountains'' or
``valleys''. For example:
\[ d = [1\ 2\ 3\ 3\ 2\ 1\ 0\ 3\ 4\ 3\ 3\ 2\ 3\ 0\ -1\ -2\ -3\ -2\ -3\ 0] \]
In this example, our lexical difference array contains two mountains and one
valley. The two mountains contain a valley, and the valley itself contains a
mountain. We then recursively chunk the mountains and valleys into groups and
cut away the base height or depth shared in common by the mountain/valley.
For example, the \( d \) array above would be split into three ``chunks''.
\[ \text{mountains} = \begin{bmatrix}
  [1\ 2\ 3\ 3\ 2\ 1] \\
  [3\ 4\ 3\ 3\ 2\ 3] \\
  [-1\ -2\ -3\ -2\ -3]
\end{bmatrix} \]
The first mountain has a height 1 base that we can cut off, the second mountain
has a height 2 base that we can cut off, and the valley has a depth 1 base that
we can fill upwards. Each of these counts as a shift as defined in the problem
statement, for a total of 4 shifts during this operation. After the cutoff,
the arrays become:
\[ \text{mountains} = \begin{bmatrix}
  [0\ 1\ 2\ 2\ 1\ 0] \\
  [1\ 2\ 1\ 1\ 0\ 1] \\
  [0\ -1\ -2\ -1\ -2]
\end{bmatrix} \]
We then run the chunking function on each of these entries and add the results
back into the array. Note that the second mountain, after cutoff, has split into
two mountains.
\[ \text{mountains} = \begin{bmatrix}
  [1\ 2\ 2\ 1] \\
  [1\ 2\ 1\ 1] \\
  [1] \\
  [-1\ -2\ -1\ -2]
\end{bmatrix} \]
We can cut off the common bases on these mountains/valleys again and add the
shifts required to the total. The cut mountains/valleys are rechunked and cut
again recursively until they have all been flattened out, at which point the
number of operations accumulated from each mountain cut is our result.

\subsubsection*{Argument of Correctness}
This algorithm minimizes the number of shifts needed to accomplish the goal by
recursively finding the greatest common contiguous segment of letters that can
be shifted in a single direction. This prevents any unnecessary shifts from
being made if the same operation can be done on a bigger segment of letters.

\subsubsection*{Running Time Estimate}
This algorithm is \( O(n) \), chunking the lexical difference array is linear,
while calculating the shift of a mountain is constant. In the worst case, the
maximum number of times we need to recursively chunk a mountain with a valley
in it is upper bounded by \( n \), so our algorithm runs in constant \( n \)
time.

\subsubsection*{Pseudocode}
\begin{lstlisting}
def stringDistance(s1, s2):
    result = []
    for c1 in s1, c2 in s2:
        result.append(c2 - c1)
    return result

def chunk(segment):
    let mountains = []
    let mountain = []
    let lastHeight = segment[0]
    for i=1 to segment.length:
        let currentHeight = segment[i]
        if lastHeight == 0 and currentHeight == 0:
            pass
        else if lastHeight == 0 and currentHeight != 0:
            mountain.append(currentHeight)
        else if lastHeight != 0 and currentHeight = 0:
            mountains.append(mountain)
            mountain = [currentHeight]
        else if lastHeight is opposite sign of currentHeight:
            mountains.append(mountain)
            mountain = [currentHeight]
        else:
            mountain.append(currentHeight)
        lastHeight = currentHeight
    mountains.append(mountain)

def chopMin**(chunk):
    let minValue = min*(chunk)
    for i=0 to chunk.length:
        chunk[i] -= minValue
    return minValue

let queue = new queue
let chunks = chunk(stringDistance(input1, input2))
let totalShifts = 0
for i=0 to chunks.length:
    queue.push(chunk[i])
while queue is not empty:
    let currentMountain = queue.pop()
    shifts += chopMin(currentMountain)
    queue.push(chunk(currentMountain))
return totalShifts
\end{lstlisting}
*: Note that this min function finds the element in chunk that is closest to 0,
so it is more of an ``absolute value min'' function. \\
**: The chopMin() function mutates the input array. \\

\end{document}
