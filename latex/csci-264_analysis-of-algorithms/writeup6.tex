\documentclass{math}

\usepackage{enumerate}
\usepackage{hyperref}
\usepackage{listings}
\usepackage{xcolor}
\usepackage{pgfplots}
\usepackage{subcaption}
\pgfplotsset{compat=1.8}
\lstset{basicstyle=\ttfamily\footnotesize,breaklines=true}

\title{Analysis of Algorithms}
\author{Alvin Lin (axl1439) and William Leuschner (wel2138)}
\date{August 2017 - December 2017}

\begin{document}

\maketitle

\section*{Problem 1}

\subsubsection*{Algorithm Description}
The Bellman-Ford algorithm for finding the shortest path in a graph with
negative edge weights conveniently encounters an error if there is a negative
weight cycle.  In order to determine if the graph has a negative weight cycle,
we simply run Bellman-Ford and return \texttt{true} if Bellman-Ford errors out,
and \texttt{false} if it succeeds.

\subsubsection*{Running Time Estimate}
Since we're merely running Bellman-Ford, plus a very short constant time boolean
value inversion at the end, our algorithm runs in the same time bounds as
Bellman-Ford, namely \( O(n*m) \).  When we originally wrote this code, we
figured, ``Oh, \( m \) is a function of \( n \), so it'll all work out fine in
the end.''  Regrettably, we failed to account for the fact that, in the worst
case, \( m \) is not a \emph{linear} function of \( n \).  In a fully-connected
graph, the number of edges per node is \( (n - 1) \), making the number of edges
in the graph \( (n - 1) * n \), and forcing our time complexity up to \( n * ((n
- 1) * n)) \).

We believe this running time is at least partially justifiable, since fully
connected graphs are rare, and we couldn't think of a faster way to do it.

\subsubsection*{Pseudocode}

\begin{lstlisting}[mathescape=true]
def bellmanFord(edges, vtxCount, source):
  distances = []
  let every element of distances = $\infty$
  let distances[source] = 0
  for i = 0 to vtxCount:
    for edge in edges:
      let newDist = distances[edge.src] + edge.weight
      if distances[edge.dst] < newDist:
        distances[edge.dst] = newDist
  for edge in edges:
    if distances[edge.dst] > distances[edge.src] + edge.weight:
      return None
  return distances

let E be the set of input edges
let |V| be the number of input vertices
if bellmanFord(E, |V|, 1) is None:
  output YES
else:
  output NO
\end{lstlisting}

\section*{Problem 2}
For these diagrams (Figures 1--5), the residual graphs specify backward edges
in red, where the weight of the backward edge is the capacity used on that
edge. Completely used-up forward edges are not shown. The augmenting path
graphs specify augmenting paths in blue, where the weight of the edge describes
the flow of commodity that is passing through that edge. Weights on forward
edges depict the remaining usable capacity of that edge.

\usetikzlibrary{positioning}
\usetikzlibrary{arrows.meta}
\begin{figure}[p]
  \centering
  \begin{tikzpicture}[
      node distance=5mm and 5mm,
      vertex/.style={
          circle,
          minimum size=3mm,
          thick,
          draw=black
      }]
    \node (1) [vertex] {1};
    \node (2) [vertex,above right=of 1] {2};
    \node (3) [vertex,below right=of 1] {3};
    \node (4) [vertex,right=of 2] {4};
    \node (5) [vertex,right=of 3] {5};
    \node (6) [vertex,below right=of 4] {6};

    \draw [-{Latex}] (1) -- (2) node [midway,above left] {1000};
    \draw [-{Latex}] (1) -- (3) node [midway,below left] {1000};
    \draw [-{Latex}] (2) -- (4) node [midway,above=2mm] {1000};
    \draw [-{Latex}] (3) -- (5) node [midway,below=2mm] {1000};
    \draw [-{Latex}] (4) -- (6) node [midway,above right] {1000};
    \draw [-{Latex}] (5) -- (6) node [midway,below right] {1000};
    \draw [-{Latex}] (3) -- (4) node [midway,above left] {1};
  \end{tikzpicture}
  \caption{Residual graph before running the algorithm}
  \label{fig:before}
\end{figure}
\begin{figure}[p]
  \centering
  \begin{subfigure}[h]{0.49\textwidth}
    \centering
    \begin{tikzpicture}[
        node distance=5mm and 5mm,
        vertex/.style={ circle, minimum size=3mm, thick, draw=black },
        fwd/.style={ -{Latex}, fill=black, draw=black },
        bkwd/.style={ fill=red, draw=red }]

      \node (1) [vertex] {1};
      \node (2) [vertex,above right=of 1] {2};
      \node (3) [vertex,below right=of 1] {3};
      \node (4) [vertex,right=of 2] {4};
      \node (5) [vertex,right=of 3] {5};
      \node (6) [vertex,below right=of 4] {6};

      \draw [fwd] (1) -- (2) node [midway,above left] {1000};
      \draw [fwd] (1) -- (3) node [midway,below left] {999};
      \draw [fwd] (2) -- (4) node [midway,above=2mm] {1000};
      \draw [fwd] (3) -- (5) node [midway,below=2mm] {1000};
      \draw [fwd] (4) -- (6) node [midway,above right] {999};
      \draw [fwd] (5) -- (6) node [midway,below right] {1000};

      \draw [bkwd] (6) edge[bend left] node [midway, below left] {1} (4);
      \draw [bkwd] (3) edge[bend right]  node [midway,above right] {1} (1);
      \draw [bkwd] (4) -- (3) node [midway,above left] {1};
    \end{tikzpicture}
    \caption{Residual}
  \end{subfigure}
  \begin{subfigure}[h]{0.49\textwidth}
    \centering
    \begin{tikzpicture}[
        node distance=5mm and 5mm,
        vertex/.style={ circle, minimum size=3mm, thick, draw=black },
        fwd/.style={ -{Latex}, fill=black, draw=black },
        aug/.style={ -{Latex}, fill=blue, draw=blue }]

      \node (1) [vertex] {1};
      \node (2) [vertex,above right=of 1] {2};
      \node (3) [vertex,below right=of 1] {3};
      \node (4) [vertex,right=of 2] {4};
      \node (5) [vertex,right=of 3] {5};
      \node (6) [vertex,below right=of 4] {6};

      \draw [fwd] (1) -- (2) node [midway,above left] {1000};
      \draw [fwd] (1) -- (3) node [midway,below left] {999};
      \draw [fwd] (2) -- (4) node [midway,above=2mm] {1000};
      \draw [fwd] (3) -- (5) node [midway,below=2mm] {1000};
      \draw [fwd] (4) -- (6) node [midway,above right] {999};
      \draw [fwd] (5) -- (6) node [midway,below right] {1000};

      \draw [aug] (4) edge[bend right] node [midway, below left] {1} (6);
      \draw [aug] (1) edge[bend left]  node [midway,above right] {1} (3);
      \draw [aug] (3) -- (4) node [midway,above left] {1};
    \end{tikzpicture}
    \caption{Augmenting}
  \end{subfigure}
  \caption{Residual graph and augmenting path after 1 step of the algorithm.
  Flow $f = 1$.}
  \label{fig:step1}
\end{figure}

\begin{figure}[p]
  \centering
  \begin{subfigure}[h]{0.49\textwidth}
    \centering
    \begin{tikzpicture}[
        node distance=5mm and 5mm,
        vertex/.style={ circle, minimum size=3mm, thick, draw=black },
        bkwd/.style={ -{Latex}, fill=red, draw=red },
        fwd/.style={ -{Latex}, fill=black, draw=black }]

      \node (1) [vertex] {1};
      \node (2) [vertex,above right=of 1] {2};
      \node (3) [vertex,below right=of 1] {3};
      \node (4) [vertex,right=of 2] {4};
      \node (5) [vertex,right=of 3] {5};
      \node (6) [vertex,below right=of 4] {6};

      \draw [fwd] (1) -- (2) node [midway,above left] {1};
      \draw [fwd] (1) -- (3) node [midway,below left] {999};
      \draw [fwd] (2) -- (4) node [midway,above=2mm] {1};
      \draw [fwd] (3) -- (5) node [midway,below=2mm] {1000};
      \draw [fwd] (5) -- (6) node [midway,below right] {1000};

      \draw [bkwd] (2) edge[bend right=90] node [midway, above left] {999} (1);
      \draw [bkwd] (4) edge[bend right=90] node [midway, above] {999} (2);
      \draw [bkwd] (3) edge[bend right]  node [midway,above right] {1} (1);
      \draw [bkwd] (4) -- (3) node [midway,above left] {1};
      \draw [bkwd] (6) -- (4) node [midway, above right] {1000};
    \end{tikzpicture}
    \caption{Residual}
  \end{subfigure}
  \begin{subfigure}[h]{0.49\textwidth}
    \centering
    \begin{tikzpicture}[
        node distance=5mm and 5mm,
        vertex/.style={ circle, minimum size=3mm, thick, draw=black },
        fwd/.style={ -{Latex}, fill=black, draw=black },
        aug/.style={ -{Latex}, fill=blue, draw=blue }]

      \node (1) [vertex] {1};
      \node (2) [vertex,above right=of 1] {2};
      \node (3) [vertex,below right=of 1] {3};
      \node (4) [vertex,right=of 2] {4};
      \node (5) [vertex,right=of 3] {5};
      \node (6) [vertex,below right=of 4] {6};

      \draw [fwd] (1) -- (2) node [midway,above left] {1};
      \draw [fwd] (1) -- (3) node [midway,below left] {999};
      \draw [fwd] (2) -- (4) node [midway,above=2mm] {1};
      \draw [fwd] (3) -- (5) node [midway,below=2mm] {1000};
      \draw [fwd] (5) -- (6) node [midway,below right] {1000};

      \draw [aug] (1) edge[bend left=90] node [midway, above left] {999} (2);
      \draw [aug] (2) edge[bend left=90] node [midway, above] {999} (4);
      \draw [aug] (4) -- (6) node [midway, above right] {999};
    \end{tikzpicture}
    \caption{Augmenting}
  \end{subfigure}
  \caption{Residual graph and augmenting path after 2 steps of the algorithm.
  Flow $f = 1000$}
  \label{fig:step2}
\end{figure}

\begin{figure}[p]
  \centering
  \begin{subfigure}[h]{0.49\textwidth}
    \centering
    \begin{tikzpicture}[
        node distance=5mm and 5mm,
        vertex/.style={ circle, minimum size=3mm, thick, draw=black },
        bkwd/.style={ -{Latex}, fill=red, draw=red },
        fwd/.style={ -{Latex}, fill=black, draw=black }]

      \node (1) [vertex] {1};
      \node (2) [vertex,above right=of 1] {2};
      \node (3) [vertex,below right=of 1] {3};
      \node (4) [vertex,right=of 2] {4};
      \node (5) [vertex,right=of 3] {5};
      \node (6) [vertex,below right=of 4] {6};

      \draw [fwd] (1) -- (2) node [midway,above left] {1};
      \draw [fwd] (2) -- (4) node [midway,above=2mm] {1};
      \draw [fwd] (3) -- (5) node [midway,below=2mm] {1};
      \draw [fwd] (5) -- (6) node [midway,below right] {1};

      \draw [bkwd] (2) edge[bend right=90] node [midway, above left] {999} (1);
      \draw [bkwd] (4) edge[bend right=90] node [midway, above] {999} (2);
      \draw [bkwd] (6) edge[bend left=90] node [midway, below right] {999} (5);
      \draw [bkwd] (5) edge[bend left=90] node [midway, below] {999} (3);
      \draw [bkwd] (4) -- (3) node [midway,above left] {1};
      \draw [bkwd] (6) -- (4) node [midway, above right] {1000};
      \draw [bkwd] (3) -- (1) node [midway, below left] {1000};
    \end{tikzpicture}
    \caption{Residual}
  \end{subfigure}
  \begin{subfigure}[h]{0.49\textwidth}
    \centering
    \begin{tikzpicture}[
        node distance=5mm and 5mm,
        vertex/.style={ circle, minimum size=3mm, thick, draw=black },
        fwd/.style={ -{Latex}, fill=black, draw=black },
        aug/.style={ -{Latex}, fill=blue, draw=blue }]

      \node (1) [vertex] {1};
      \node (2) [vertex,above right=of 1] {2};
      \node (3) [vertex,below right=of 1] {3};
      \node (4) [vertex,right=of 2] {4};
      \node (5) [vertex,right=of 3] {5};
      \node (6) [vertex,below right=of 4] {6};

      \draw [fwd] (1) -- (2) node [midway,above left] {1};
      \draw [fwd] (2) -- (4) node [midway,above=2mm] {1};
      \draw [fwd] (3) -- (5) node [midway,below=2mm] {1};
      \draw [fwd] (5) -- (6) node [midway,below right] {1};

      \draw [aug] (5) edge[bend right=90] node [midway, below right] {999} (6);
      \draw [aug] (3) edge[bend right=90] node [midway, below] {999} (5);
      \draw [aug] (1) -- (3) node [midway, below left] {999};
    \end{tikzpicture}
    \caption{Augmenting}
  \end{subfigure}
  \caption{Residual graph and augmenting path after 3 steps of the algorithm.
  Flow $f = 1999$}
  \label{fig:step3}
\end{figure}

\begin{figure}[p]
  \centering
  \begin{subfigure}[h]{0.49\textwidth}
    \centering
    \begin{tikzpicture}[
        node distance=5mm and 5mm,
        vertex/.style={ circle, minimum size=3mm, thick, draw=black },
        bkwd/.style={ -{Latex}, fill=red, draw=red },
        fwd/.style={ -{Latex}, fill=black, draw=black }]

      \node (1) [vertex] {1};
      \node (2) [vertex,above right=of 1] {2};
      \node (3) [vertex,below right=of 1] {3};
      \node (4) [vertex,right=of 2] {4};
      \node (5) [vertex,right=of 3] {5};
      \node (6) [vertex,below right=of 4] {6};

      \draw [fwd] (3) -- (4) node [midway, above left] {1};

      \draw [bkwd] (2) -- (1) node [midway, above left] {1000};
      \draw [bkwd] (3) -- (1) node [midway, below left] {1000};
      \draw [bkwd] (4) -- (2) node [midway, above=2mm] {1000};
      \draw [bkwd] (5) -- (3) node [midway, below=2mm] {1000};
      \draw [bkwd] (6) -- (4) node [midway, above right] {1000};
      \draw [bkwd] (6) -- (5) node [midway, below right] {1000};
    \end{tikzpicture}
    \caption{Residual}
  \end{subfigure}
  \begin{subfigure}[h]{0.49\textwidth}
    \centering
    \begin{tikzpicture}[
        node distance=5mm and 5mm,
        vertex/.style={ circle, minimum size=3mm, thick, draw=black },
        aug/.style={ -{Latex}, fill=blue, draw=blue },
        fwd/.style={ -{Latex}, fill=black, draw=black }]

      \node (1) [vertex] {1};
      \node (2) [vertex,above right=of 1] {2};
      \node (3) [vertex,below right=of 1] {3};
      \node (4) [vertex,right=of 2] {4};
      \node (5) [vertex,right=of 3] {5};
      \node (6) [vertex,below right=of 4] {6};

      \draw [aug] (4) -- (3) node [midway, above left] {1};

      \draw [aug] (1) -- (2) node [midway, above left] {1};
      \draw [aug] (2) -- (4) node [midway, above=2mm] {1};
      \draw [aug] (3) -- (5) node [midway, below=2mm] {1};
      \draw [aug] (5) -- (6) node [midway, below right] {1};
    \end{tikzpicture}
    \caption{Augmenting}
  \end{subfigure}
  \caption{Residual graph and augmenting path after 4 steps of the algorithm.
  Flow $f = 2000$}
  \label{fig:step4}
\end{figure}


\section*{Problem 3}

\subsubsection*{Algorithm Description}
The key to this problem is that the only place an edge can be added is across
the minimum \emph{s-t} cut of the graph.  In order to locate the minimum
\emph{s-t} cut, we apply the Edmonds-Karp algorithm to reduce the graph to a
state where all of the existing augmenting paths from the server (source) to the
client (sink) have been removed.  Once the graph is in this state, we can then
transpose it and build a breadth-first traversal of the transpose.  This tree
contains all of the nodes on the \emph{t} side of the \emph{s-t} cut.  We then
search this tree for the node with the lowest index (lexicographically first),
and draw a new edge between that node and the source node.

Although we cannot determine why this algorithm fails the sixth test case, we
have a hypothesis: Our constraints are wrong and we are excluding nodes that
we shouldn't be excluding during the search for a node to connect the source
node to. This could be due to the fact that our algorithm for calculating the
predecessor nodes in the breadth-first traversal is excluding the leaf node at
the end of every branch of the tree, or at least some of them.

\subsubsection*{Running Time Estimate}
This algorithm is just Edmonds-Karp with some extra stuff at the end.
Edmonds-Karp runs on \( O(m^2n) \) time because finding a single augmenting path
requires \( O(m) \) time, and at worst, this needs to be done \( mn \) times.

\pagebreak

\subsubsection*{Pseudocode}
\begin{lstlisting}[mathescape=true]
def getBandwidthBottleneckEdge(graph, source, sink):
  copy = deepCopy(graph)
  predecessors, visited = bfs(graph, source)
  while visited[sink]:
    minFlow = $\infty$
    currentNode = sink
    while currentNode != source:
      predecessor = predecessors[currentNode]
      minFlow = min(minFlow, graph[predecessor][currentNode])
      currentNode = predecessor
    currentNode = sink
    while currentNode != source:
      predecessor = predecessors[currentNode]
      graph[predecessor][currentNode] -= minFlow
      graph[currentNode][predecessor] += minFlow
      currentNode = predecessor
    predecessors, visited = b{figure}fs(graph, source)

  predecessors, visited = bfs(transpose(graph), sink)
  nodes = filter(lambda x: x != -1, sorted(predecessors))
  for node in nodes:
    if copy[source][node] == 0:
      return (source, node)
  return "NO"
\end{lstlisting}


\section*{Problem 4}

\subsection*{Reducing \( P \) to \( Q_1 \)}
Because the longest path algorithm is capped at returning a path through every
node, we can simply run our graph through it unchanged and check if the output
length is equal to a path through every node.
\begin{enumerate}[a.]
  \item Construct \( G_1, s_1, and t_1 \) by setting them equal to their
  subscript-less counterparts.
  \item Given \( \ell \), \( G \) will have a Hamiltonian path if and only
  if \( \ell \) is equal to \( n - 1 \), where \( n \) is the number of
  vertices in \( G \).
\end{enumerate}

\subsection*{Reducing \( P \) to \( Q_2 \)}
This problem is slightly more difficult, but not by much.  Becuase the black-box
shortest path algorithm accepts negative weights, we can turn it into a
longest-path algorithm by assigning every edge in \( G \) a weight of \( -1 \).
\begin{enumerate}[a.]
  \item Construct \( G_1 \) by assigning every edge a weight of \( -1 \).
  Construct $s_1$ and $t_1$ by assigning to them $s$ and $t$,
  respectively.
  \item Given \( \ell \), \( G \) will have a Hamiltonian path if and only if
  \( \|\ell\| = n - 1 \), where \( n \) is the number of vertices in \( G \).
\end{enumerate}

\end{document}
