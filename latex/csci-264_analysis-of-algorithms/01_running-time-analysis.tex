\documentclass[letterpaper, 12pt]{math}

\usepackage{amsmath}
\usepackage{amssymb}
\usepackage{hyperref}

\title{Analysis of Algorithms}
\author{Alvin Lin}
\date{August 2017 - December 2017}

\begin{document}

\maketitle

\section*{Big-Oh Running Time (asymptotic upper bound)}
\( f(n) = O(g(n)) \) if there exists a constant \( c > 0 \) and a constant
\( n_{\circ} \) such that for every \( n\geq n_{\circ} \) we have
\( f(n)\leq cg(n) \).

\subsection*{Example}
Prove that \( n^3 = O(7n^2+\frac{n^3}{3}) \):
We need to find \( c,n_{\circ} \) such that \( \forall n\geq n_{\circ} \):
\[ n^3\leq c(7n^2-\frac{n^3}{3}) \]
\[ c = 3 \]

\subsection*{Example}
Prove that \( n^3 = O(\frac{n^3}{3}-7n^2) \):
We need to find \( c,n_{\circ} \) such that \( \forall n\geq n_{\circ} \):
\[ n^3\geq c(\frac{n^3}{3}-7n^2) \]
We can take \( c = 6 \), then we want to show that:
\begin{align*}
  n^3 &\leq 6(\frac{n^3}{3}-7n^2) \quad \forall n\geq n_{\circ} \\
  n^3 &\leq 2n^3-42n^2 \\
  42n^2 &\leq n^3 \\
  42 &\leq n
\end{align*}
Thus, this inequality holds for \( n_{\circ} = 42 \).

\subsection*{Example}
Prove that \( \log_{10}n = O(\log n) \) and that \( \log n = O(\log_{10}n) \):
We need to find \( c,n_{\circ} \) such that \( \forall n\geq n_{\circ} \):
\begin{align*}
  \log_{10}n &\leq c\log_{2}n \\
  \log_{10}n &= \frac{\log_{2}n}{\log_{2}10}
\end{align*}
We can take \( c = \frac{1}{\log_{2}10} \) and \( n_{\circ} = 1 \).

\subsection*{Example}
What about \( 3^n \) and \( 2^n \)?
\[ 2^n = O(3^n) \]
\[ 2^n \neq O(2^n) \]
For this case, we need to show that for every possible \( c \) and every
possible \( n_{0} \), the equality is false. By contradiction, assume that
\( 3^n = O(2^n) \). If this is the case, then there exists constants \( c \)
and \( n_{\circ} \) such that \( 3^n \) is upper bounded by \( c\cdot2^n \).
\[ 3^n \leq c\cdot2^n \quad \forall n\geq n_{\circ} \]
Divide both sides by \( 2^n \):
\[ (\frac{3}{2})^n \leq c \]
This cannot possibly hold as \( n\to\infty \) because \( \frac{3}{2} > 1 \)
and the function is exponential.

\section*{Big-Omega Running Time (asymptotic lower bound)}
\( f(n) = \Omega(g(n)) \) if there exists a constant \( c > 0 \) and a constant
\( n_{\circ} \) such that for every \( n \geq n_{\circ} \) we have
\( f(n) \geq cg(n) \). This is the reverse of Big-Oh and determines a lower
bound. Claim: any comparison-based sort of \( n \) numbers needs at least
\( \Omega(n\log n) \) comparisons.

\section*{Theta Running Time (asymptotically tight bound)}
\( f(n) = \Theta(g(n)) \) if there exists constants \( c_{1},c_{2} > 0 \) and
a constant \( n_{\circ} \) such that for every \( n \geq n_{\circ} \) we have
\( c_{1}g(n) \leq f(n) \leq c_{2}g(n) \).
\[ \log n = \Theta(1+\log n) \]
\[ n^3 = \Theta(7n^2+\frac{n^3}{3}) \]

\begin{center}
  You can find all my notes at \url{http://omgimanerd.tech/notes}. If you have
  any questions, comments, or concerns, feel free to contact me at
  alvin@omgimanerd.tech
\end{center}

\end{document}
