\documentclass[letterpaper, 12pt]{math}

\usepackage{listings}
\lstset{basicstyle=\ttfamily\footnotesize,breaklines=true}

\title{Analysis of Algorithms}
\author{Alvin Lin (axl1439) and William Leuschner (wel2138)}
\date{August 2017 - December 2017}

\begin{document}

\maketitle

\section*{Problem 1}

\subsubsection*{Algorithm Description}
For this problem, we used two passes over the array of heights to determine
the maximum possible volume of the reservoir. One pass would start left to
right, and the other pass would sweep right to left. Each pass kept track of
the highest point that it passed over and added the difference between the
subsequent lower points to the volume. If the pass reached a point higher than
the apex that it was tracking, it designated the end of a reservoir, and the
summed volume would be added to a heap for tracking.

\subsubsection*{Argument of Correctness}
There are three possible ways for a reservoir to exist, the left wall can be
higher than the right, the left wall can be lower than the right, or they can
be equal. A left to right pass of this algorithm is able to determine the
volume of all the reservoirs as long as the right side of the reservoir is
higher than or equal to the left side. A right to left pass has the opposite
condition, so by making two passes, we are able to determine the sizes of all
the reservoirs.

\subsubsection*{Running Time Estimate}
This algorithm requires two passes through the heights, so the execution time is
a constant \( 2n \), or \( O(n) \).

\subsubsection*{Pseudocode}
\begin{lstlisting}
def getMaxReservoirVolume(data):
  let highestVolume = 0
  let currentVolumeLTR = 0
  let currentVolumeRTL = 0
  let currentApexLTR = data[0]
  let currentApexRTL = data[data.length - 1]
  for i=1 to data.length:
    let LTRelement = data[i]
    let RTLelement = data[data.length - i - 1]
    if LTRelement > currentApexLTR:
      currentApexLTR = LTRelement
      highestVolume = max(highestVolume, currentVolumeLTR)
      currentVolumeLTR = 0
    else:
      currentVolumeLTR += currentApexLTR - LTRelement
    if RTLelement > currentApexRTL:
      currentApexRTL = RTLelement
      highestVolume = max(highestVolume, currentVolumeRTL)
      currentVolumeRTL = 0
    else:
      currentVolumeRTL += currentApexRTL - RTLelement
  return highestVolume
\end{lstlisting}

\section*{Problem 2}
Consider the divide-and-conquer algorithm in Figure 1 that assumes access to
a global string \( s \) of lower-case letters, where \( s[i] \) refers to
the \( i \)-th character of \( s \).
\begin{enumerate}
  \item State the recurrence for \( T(n) \) that captures the running time of
  the algorithm as closely as possible.
  \[ O(n) \]
  \item Use the ``unrolling the recurrence'' or the mathematical induction
  technique to find a tight bound on \( T(n) \).
  \begin{align*}
    T(1) &= 1 \\
    T(n) &= 2T\left(\frac{n}{2}\right)+c \\
    &= 2\left(2T\left(\frac{n}{4}\right)+c\right)+c \\
    &= 2\left(2\left(2T\left(\frac{n}{8}\right)+c\right)+c\right)+c \\
    &= 2^kT\left(\frac{n}{2^k}\right)+kc \\
    & \text{stops when } \frac{n}{2^k} = 1 \equiv k = \log(n) \\
    &= 2^{\log(n)}T(1)+c\log(n) \\
    &= n+c\log(n) \\
    &= O(n+\log(n)) \\
    &= O(n)
  \end{align*}
  \item What does the algorithm do? \\
  This algorithm calculates four statistics about the number and placement of
  vowels and consonants in a string. Those four statistics are:
  \begin{itemize}
    \item \texttt{maxdif}, the maximum of \texttt{leftalignedmaxdif} and
    \texttt{rightalignedmaxdif} (usually, although we couldn't figure out the
    special cases where it wasn't)
    \item \texttt{leftalignedmaxdif} is the maximum difference between the
    number of consonants and the number of vowels, in the left half of the
    input.
    \item \texttt{rightalignedmaxdif} is the maximum difference between the
    number of consonants and the number of vowels, in the right half of the
    input.
    \item \texttt{dif} contains the difference between the number of consonants
    and the number of vowels in string \( s \). (This is the only one we're
    completely sure about.)
  \end{itemize}
\end{enumerate}

\section*{Problem 3}

\subsubsection*{Algorithm Description}
For this problem, we represented each number in the array as a tuple containing
the number's value and the index of its original position. We then ran a merge
sort on the array. Each merge operation was a subproblem that could be handled
the same way in order to count the weight of the inversions. When merge sorting
two lists, hereafter referred to as \( left \) and \( right \), pulling a number
from the left list into the result means that there was no inversion. Pulling a
number from the right list means that number has an inversion with every number
on the left list. Essentially, the number of inversions is incremented by the
size of the left list each time we pull from the right list for every merge
operation. \par
We made a modification to this algorithm to count the weight of the inversions
instead of the number of inversions by precomputing the sum of all the elements
in the left list. For this problem, we needed to find the difference in indices
between the right element and ALL the elements in the left list every time we
pulled from the left list. Let \( l_1,l_2,\dots,l_n \) be the original indices
of the elements in the left list, and let \( r \) be the original index of the
element in the right list which we are pulling out. Each time we pull out an
element from the right, we need to compute the following in order to get the
total weight all the inversions:
\begin{align*}
  (r-l_1)+(r-l_2)+\dots+(r-l_n) &= \sum_{i=1}^{n}r-l_n \\
  &= \sum_{i=1}^{n}r-\sum_{i=1}^{n}l_n \\
  &= rn-\sum_{i=1}^{n}l_n
\end{align*}
Before we begin the merging, we can compute \( \sum_{i=1}^{n}l_n \) simply by
summing up all the numbers in the left list. Each time we pull from the right
list, we multiply the right number's original index by the size of the
left list and subtract our precomputed sum to get the weight of all the
inversions. Each time we pull from the list list, we subtract that number's
value from the precomputed sum so that it remains equal to the sum of all
the numbers in the left list. By making this calculation during each merge
operation and summing up all the results, we will have the total weight of
all the inversions when the list has been fully sorted.

\subsubsection*{Argument of Correctness}
During each merge operation, we are guaranteed that the left and right lists
are sorted due to the recursive nature of the merge sort algorithm. Since a
merge sort is also stable, this allows us to keep track of the number of times
a number is ``swapped'' when we pull from the right list. Each time a number
is swapped, we know it is inverted with every element in the left list, and
we can use this to sum up whatever information we need about the inversions.

\subsubsection*{Running Time Estimate}
The merge sort is \( O(n\log n) \), and since we are precomputing the
sum of the left side before each merge operation, we can compute the weight
of the inversions during the merge in constant time. This makes the entire
algorithm \( O(n\log n) \).

\subsubsection*{Pseudocode}
\begin{lstlisting}
def calculateWeightedInversions(data):
  let inversions = 0

  def merge(left, right):
    let leftSum = sum([e.index for e in left])
    let result = []
    while left.length + right.length < result.length:
      if left.length == 0:
        result.push(right.pop())
      elif right.length == 0:
        result.push(left.pop())
      elif left.peek().index > right.peek().index:
        inversions += left.length * right.peek().index - leftSum
        result.push(right.pop())
      else:
        leftSum -= left.peek().value
        result.push(left.pop())
    return result

  def mergeSort(data):
    if data.length == 1:
      return data
    let middle = data.length / 2
    return merge(data[:middle], data[middle + 1:])

  mergeSort(data)
  return inversions
\end{lstlisting}

\section*{Problem 4}
For each the following recurrences, use the Master theorem to express \( T(n) \)
as a Theta of a simple function. State what the corresponding values of \( a \),
\( b \), and \( f(n) \) are and how you determined which case of the theorem
applies. Do not worry about the base case or rounding.
\begin{enumerate}
  \item \( T(n) = 9T(\frac{n}{3})+n^3 \)
  \begin{align*}
    a &= 9 \\
    b &= 3 \\
    f(n) &= n^3 \\
    n^3 &= O(n^{\frac{\log9}{\log3}-\epsilon})\quad \epsilon > 0 \\
    \therefore T(n) &= \Theta(n^{\frac{\log9}{\log3}}\log n) \\
    &= \Theta(n^2\log n)
  \end{align*}
  \item \( T(n) = \frac{3}{2}T(\frac{2n}{3})+3 \)
  \begin{align*}
    a &= \frac{3}{2} \\
    b &= \frac{3}{2} \\
    f(n) &= 3 \\
    f(n) &= O(n) \\
    \therefore T(n) &= \Theta(n)
  \end{align*}
  \item \( T(n) = 2T(\frac{n}{4}) + \sqrt{n} \)
  \begin{align*}
    a &= 2 \\
    b &= 4 \\
    f(n) &= \sqrt{n} \\
    \therefore T(n) &= \Theta(\sqrt{n}\log n)
  \end{align*}
\end{enumerate}

\section*{Problem 5}

\subsubsection*{Algorithm Description}
For this problem, we implemented a radix sort with a variable base. By computing
an optimal base to convert all the numbers into, we are able to sort all the
numbers in linear time.

\subsubsection*{Argument of Correctness}
Given \( n \) numbers of which the maximum is \( n^2-1 \), a base 10 radix sort
would have a runtime of \( n\log_{10}(n^2-1) \), which is:
\[ O(n\log(n)) \]
In general, for a radix sort of base \( b \), the runtime is \( b\times d \)
where \( d \) is the number of digits in the largest number. Since we are given
the constraint that the largest number can be upper bounded by \( n^2 \),
\( d \) can be expressed in terms of \( n \) as \( \log_b(n^2) = d \). \par
Note that as long as \( b \) is a function of \( n \) of the form \( cn \), then
this runtime is equivalent to linear time.
\begin{align*}
  Let: b &= cn \\
  bd &= b\log_b(n^2) \\
  &= cn\log_{cn}(n^2) \\
  &= 2cn\log_{cn}(n) \\
  &= 2cn\frac{\log(n)}{\log(cn)} \\
  \lim_{n\to\infty}\frac{\log(n)}{\log(cn)} &= 1
\end{align*}
From the definitions above:
\begin{align*}
  \log_b(m) &= d \\
  \log_b(n^2) &= d \\
  n^2 &= b^d
\end{align*}
Logically, this shows that we could do a sort in one pass if we let \( d = 1 \)
and \( b = n^2 \). This will have \( O(n^2) \) space complexity as it is
basically just creating an array of size \( n^2 \) and placing the numbers at
the index they correspond to. It makes the most sense to choose a \( b \) as
close to the resultant \( d \) as possible because this minimizes the space
complexity.

\subsubsection*{Running Time Estimate}
From the calculations above, the running time of this optimized radix sort is
linear, or \( O(n) \).

\subsubsection*{Pseudocode}
\begin{lstlisting}
def getOptimalBaseDigits(max):
  for base = 2 to max:
    let digits = ceil(log(max) / log(base))
    if digits < base:
      return (base, digits)

def convertFromBase10(number, base, digits):
  let result = []
  while number != 0:
    result.pushFirst(number % base)
    number /= base
  return result

def radixSort(data, base, digits):
  let buckets1 = [[] for i in base]
  let buckets2 = [[] for i in base]
  for i = 0 to data.length:
    buckets1[data[i][0]].pushLast(data[i])
  let empty = buckets2
  let filled = buckets1
  for digit = 1 to digits:
    for bucket in filled:
      while not bucket.isEmpty():
        let num = bucket.popFirst()
        empty[num[digit]].pushLast(num)
    filled, empty = empty, filled
  let result = []
  for bucket in filled:
    result.pushLast(bucket.popFirst())

def main(input):
  let base, digits = getOptimalBaseDigits(max(input))
  let matrix = [convertFromBase10(d, base, digits) for d in data]
  matrix = radixSort(matrix)
  return matrix
\end{lstlisting}

\end{document}
