\documentclass{math}

\title{Programming Language Concepts}
\author{Alvin Lin}
\date{January 2018 - May 2018}

\begin{document}

\maketitle

\section*{Scheme}
Scheme was invented at MIT in the 1970's and originally called Schemer.
CommonLISP was developed in the 1980's after a merger between LISP and Scheme
failed. We will use DrRacket in class. Since we are doing functional
programming, we will not be using any \texttt{do} or \texttt{begin} commands or
any functions ending with \texttt{!}.

\subsection*{Scheme: an example of REPL}
A REPL is a read-evaluate-print-loop. In the console, expressions entered are
evaluated one at a time.

\subsection*{Scheme Expression Types}
\begin{itemize}
  \item \textbf{Atoms}: numbers, strings, characters, symbols
  \item \textbf{Symbols}: Scheme's identifiers. Symbols can be bound to values.
  \item \textbf{Lists}: zero or more expressions enclosed in parentheses
\end{itemize}

\subsection*{Evaluation Rules}
\begin{itemize}
  \item Literals (numbers, strings, characters) evaluate to themselves.
  \item Symbols evaluate to the values to which they are bound. They are both
  names and values.
  \item Lists are treated as function calls.
\end{itemize}
\texttt{(* 9 8)}
\begin{itemize}
  \item The symbol \texttt{*} is bound to the function's definition.
  \item The two arguments to the function are 9 and 8, which are evaluated left
  to right.
  \item The values of the arguments are passed to the function bound to
  \texttt{*} as a list.
  \item Function \texttt{*} returns the value 72.
\end{itemize}
\texttt{(/ (+ 5 3) (- 4 2))} evaluates to 4.

\subsection*{\texttt{define}}
We can use the \texttt{define} function to set the values of global variables.
Example: \texttt{(define x 3.14)}. Note that the first argument is not
evaluated. This is normally only used for functions, because global variables
are bad.

\subsection*{Delayed Evaluation}
The quote operator has a special function. It takes a symbol and returns it
verbatim.

\subsection*{Lists}
Lists are linked, using 2-part \textit{cons cells}. The \texttt{cons} function
builds cells from two values. The null value \texttt{()} is a proper list.
\begin{itemize}
  \item \texttt{(cons 4 (cons 3 '()))}
  \item \texttt{(cons 4 otherlist)}
\end{itemize}
An unquoted list is treated as an expression: evaluation of a list returns the
value returned from the function call. Thus we use the function \texttt{list}
to build lists.

\subsection*{Pulling Lists Apart}
\texttt{car} (Contents of Address Register) extracts the first element of a
list. \texttt{cdr} (Contents of Decrement Register) extracts everything after
the first element.

\subsection*{Booleans}
\begin{itemize}
  \item \texttt{\#f} and \texttt{'()} are false.
  \item \texttt{\#t} and everything else are true.
  \item Predicate functions: \texttt{list?}, \texttt{number?}, etc
  \item Boolean functions: \texttt{not}, \texttt{and}, and \texttt{or}
  \item Number comparision: \texttt{=}, \texttt{<=}, etc
\end{itemize}

\subsection*{Alternation}
\texttt{if} is another special function.
\begin{center}
  \texttt{(if then-expression else-expression)}
\end{center}
\texttt{cond} allows for multi-way conditionals.
\begin{center}
  \texttt{(cond
    [(text-expression) (then-body)] [(test-expression) (then-body)] ...)}
\end{center}

\subsection*{Functions}
The \texttt{lambda} function allows you to declare an anonymous function, which
can be bound to a symbol using \texttt{define}. Example: \\
\texttt{(lambda (args..) expr)} \\
\texttt{(lambda (x y) (+ x y))}

\subsection*{IO (non-functional)}
For input there is the functions \texttt{read readline}, and for output there
is \texttt{display newline}.

\subsection*{Acceptable Variable Binding}
\texttt{(let ((symbol1 expr1) (symbol2 expr2)) expr...)} \\
Binds a variable locally within \texttt{expr}.

\subsection*{Command Line Programming}
\texttt{racket [-i]}: Loads standard library; runs REPL \\
\texttt{racket -f file}: Loads standard library and \texttt{file}; exits \\
\texttt{racket -if file}: Loads standard library and \texttt{file}; runs REPL \\
The comment character is \texttt{;}, which comments to the end of the line.
Loaded programs just execute everything in them. In REPL mode, each line you
type is loaded, and the result is displayed in the console.

\subsection*{High Order Functions}
Scheme treats functions like data. \\
\texttt{(map car '((1 2) (2 3))) -> (1 3)}

\begin{center}
  You can find all my notes at \url{http://omgimanerd.tech/notes}. If you have
  any questions, comments, or concerns, please contact me at
  alvin@omgimanerd.tech
\end{center}

\end{document}
