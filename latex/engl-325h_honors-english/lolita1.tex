\documentclass[letterpaper, 12pt]{article}

\title{Lolita: Assignment 1}
\author{Alvin Lin}
\date{August 2016 - December 2016}

\begin{document}

\maketitle

\subsubsection*{Throughout the narrative as throughout their relationship HH
claims to maintain perfect control. As readers, we tend to disagree. Where in
the text can we find evidence to the fact that HH’s grip on his narrative and
his captive as not as strong as he might think?}

Throughout Humbert Humbert's narrative, we see his use of bribery through gifts
to maintain control of Dolores Haze (Lolita). He uses veiled threats and fear to
maintain control by describing what her life would be without him. He offers no
other alternative to her other than to stay with him and thus controls her
actions. ``In plainer words, if we two are found out, you will be analyzed and
institutionalized, my pet, \textit{c'est tout}. You will dwell, my Lolita ...
with thirty-nine other dopes in a dirty dormitory ... under the supervision of
hideous matrons'' (151). However, Humbert Humbert does not have complete control
over Lolita's mindset. He is unable to do anything about her boredom when
traveling with him and cannot maintain her satisfaction with him. This is
evident in his own confession. ``By rubbing all this in, I succeeded in
terrorizing Lo, who despite a certain brash alertness of manner and spurts of
wit was not as intelligent a child as her I.Q. might suggest. But if I managed
to establish that bakcground of shared secrecy and shared guilt, I was much
less successful in keeping her in good humor'' (151). By his own admission,
Humbert's control slips when it comes to her state of mind. He is also unable
to keep Lolita from flirting with others. ``Oh, I had to keep a very sharp eye
on Lo, little limp Lo! ... For little Lo was aware of that glow of hers, and I
would often catch her \textit{coulant un regard} in the direction of some
amiable male, some grease monkey, with a sinewy golden-brown forearm and
watch-braceleted wrist, and hardly had I turned my back to go and buy this very
Lo a lollipop, than I would hear her and the fair mechanic burst into a perfect
love song of wisecracks'' (159). This also shows that Humbert's bribery is
losing its effect. It shows how Humberts control over her is slowly ebbing away.

\end{document}
