\documentclass{article}

\usepackage{hyperref}

\title{Basics of Economics}
\author{Alvin Lin}
\date{Principles of Microeconomics: August 2016 - December 2016}

\begin{document}

\maketitle

\section{Supply and Demand}
A market is any arrangement that enables buyers and sellers to get information
and do business with each other. One solution to the question of how scarce
resources should be allocated is the mechanism of a competitive market. \par
Markets have two parts, the \textbf{demand side}, which entails consumers
that way to buy, and the \textbf{supply side}, the sellers and producers.

\subsection{Demand}
You demand something if you want it, can afford it, and have definite plans to
buy it. The \textbf{quantity demanded} of a good or service is the amount that
consumers plan to buy during a particular time period at a particular price.

\subsubsection{The Law of Demand}
The Law of Demand states that, ceteris paribus, the higher the price
of a good, the smaller the quantity demanded. The lower the price of a good,
the larger the quantity desired. \textbf{Demand} refers to the entire
relationship between the price of a good and the quantity demanded of that
good. It is determined by preferences, income, and the price of other goods.
Two  factors for the Law of Demand are:

\paragraph{The Substitution Effect:}
When the price of a good increases, people seek substitutes for it, so the
quantity demanded decreases.

\paragraph{The Income Effect:}
When the price of a good increases relative to income, people cannot afford all
of the goods they could previously, so the quantity demanded of the good
decreases.

\subsubsection{The Demand Curve}
The demand curve shows the relationship between the quantity of a good and
its price when all other influences on a consumer's planned purchases remain
the same. It represents the willingness to pay for each quantity. The price
along the demand curve is the highest price a consumer is willing to pay
for that unit (Marginal Benefit Curve).

\subsection{Supply}
A firm supplies a good or service if it has the resources and technology to
produce that good, can make a profit from producing that good, and has
definite plans to produce and sell it. The \textbf{quantity supplied} of good
or service is the amount producers plan to sell during a particular time
period at a particular price. For any given price, the quantity supplied
tells us how much firms are willing to sell at that price.

\subsubsection{The Law of Supply}
The Law of Supply states that, ceteris paribus, producers are willing to supply
a good if they can at least cover their marginal costs. This results from the
general tendency for the marginal cost of producing a good to increase as the
quantity supplied increases.

\subsubsection{The Supply Curve and Supply Schedule}
The supply curve shows the relationship between the quantity supplied of a good
and its price when all other influences on a producer's plans remain the same.
Example supply curve:
\begin{center}
  \begin{tabular}{|c|c|}
    \hline
    Price of iPhones & Quantity Supplied (in thousands) \\ \hline
    0   & 0  \\ \hline
    150 & 10 \\ \hline
    300 & 20 \\ \hline
    450 & 30 \\ \hline
    600 & 40 \\ \hline
  \end{tabular}
\end{center}

\subsection{Market Equilibrium}
Equilibrium is a situation in which opposing forces balance each other. A
market equilibrium occurs when the price balances the plans of buyers and
sellers.

\subsubsection{Equilibrium Price \( P^{*} \)}
\[ Q_{S}\ at\ P^{*} = Q_{D}\ at\ P^{*} \]
The equilibrium price is the price at which the quantity demanded equals the
quanitity supplied. The equilibrium quantity is the quantity bought and sold
at that price. At prices below the equilibrium price there is excess demand,
or shortage. At prices above the equilibrium price there is excess supply,
or surplus. \par
At prices \textit{above} the equilibrium, a surplus forces prices down. At
prices \textit{below} the equilibrium, a shortage forces prices up. The
equilibrium point is stable, as there is no pressure on prices to change.

\subsubsection{Predicting Changes in Price and Quantity}
The demand curve shifts constantly. At the original price, there is either
a surplus or shortage. The price rises or falls to restore the equilibrium
and the quantity supplied moves along the supply curve. \par
The supply curve also shifts. At the original price, there can also be a
surplus or shortage. The price rises or falls to restore the equilibrium and
quantity demanded moves along the demand curve.

\subsubsection{Changes in Demand}
Changes in demand occur when factors other than the price influence buying
plans.

\paragraph{Prices of Related Goods:}
Prices of related goods can affect demand. A \textbf{substitute} is a good
that can be used in place of another good. A \textbf{complement} is a good
that is used in conjunction with another good.
\begin{itemize}
  \item A rise in the price of a substitute increases demand for the good.
  \item A decrease in the price of a substitute decreases demand for the good.
  \item A rise in the price of a complement decreases the demand for a good.
  \item A decrease in the price of a complement increases the demand for a good.
\end{itemize}

\paragraph{Expected Future Prices:}
Expected future prices also affect demand. If people expect the price to
increase in the future, they will demand more now. Conversely, if people
expect the price to decrease in the future, they will demand less now.

\paragraph{Income:}
For some goods, demand increases when income increases, but not all goods.
The demand of a \textbf{Normal Good} increases when income increases. The
demand of an \textbf{Inferior Good} decreases when income increases.

\paragraph{Expected Future Income and Credit:}
Demand can increase if people expect to earn more income in the future.

\paragraph{Population:} The larger the population is, the larger the demand is
for all goods.

\paragraph{Preferences:} If preferences change, demand will shift.

\subsubsection{Changes in Supply}
Changes in supply occur when any factor other than the price influences selling plans.

\paragraph{Prices of Factors of Production:}
If the price of a factor of production rises, the minimum price that a producer
is willing to accept for producing each quantity must also rise.

\paragraph{Prices of Related Goods:}
Prices of related goods can affect supply. A \textbf{subsitute in production}
is a good that can be produced using the same resources. A \textbf{complement
in production} is a good that must be produced in conjunction with another
good.
\begin{itemize}
  \item An increase in the price of a substitute in production increases the
        supply of a good.
  \item A decrease in the price of a substitute in production decreases the
        supply of a good.
  \item An increase in the price of a complement in production increases the
        supply of a good.
  \item A decrease in the price of a complement in production decreases the
        supply of a good.
\end{itemize}

\paragraph{Expected Future Prices:}
If the price of a good is expected to increase in the future, the supply of it
today decreases.

\paragraph{Number of Suppliers:}
The larger the number of suppliers of a good, the greater the supply is of the
good.

\paragraph{Technology:}
Advances in technology increase supply.

\paragraph{State of Nature:}
A natural disaster can decrease supply.

\begin{center}
  You can find all my notes at \url{http://omgimanerd.tech/notes}. If you have
  any questions, comments, or concerns, please contact me at
  alvin@omgimanerd.tech
\end{center}

\end{document}
