\documentclass[letterpaper, 12pt]{article}

\title{Basics of Economics}
\author{Alvin Lin}
\date{Principles of Microeconomics: August 2016 - December 2016}

\begin{document}

\maketitle

\section{Key Terms and Ideas}
\textbf{Absolute Advantage (AA)}: A person (or country) has an absolute
advantage if they are more productive than others.
\newline
\textbf{Allocative Efficiency}: An allocation of goods that maximizes benefit
(total benefit - total cost).
\newline
\textbf{Comparative Advantage (CA)}: A person (or country) has a comparative
advantage in an activity if they can perform that activity at a lower
opportunity cost that everyone else.
\newline
\textbf{Demand}: The entire relationship between the price of a good and the
quantity demanded of that good.
\newline
\textbf{Diminishing Marginal Benefit}: The more of a good an individual has,
the less the individual is willing to give up for one more unit of that good.
\newline
\textbf{Function}: The relationship between two quantities.
\newline
\textbf{Gains From Trade}: The net benefit to agents who conduct trade of
specific goods.
\newline
\textbf{Marginals}: The variance in the output of a function as the input value
changes by a given amount. Analogous to the derivative of a function.
\newline
\textbf{Marginal Benefit}: The total amount of a good that an individual is
willing to give up for ONE unit of another good.
\newline
\textbf{Marginal Cost}: The opportunity cost of producing one additional unit
of some good. Ex: ``I give up 3 cookies to produce the 6th unit of cat food''.
\newline
\textbf{Marginal Revenue}: The change in total revenue obtained by a seller
as a result of a change in the total quantity sold.
\newline
\textbf{Marginal Utility}: The change in total utility that results from a
one unit increase in the quantity of a good consumed.
\newline
\textbf{Minimum Wage}: A government regulation making it illegal to pay a
worker less than a specified wage.
\newline
\textbf{Normative Statement}: An expression of how things ought to be.
\newline
\textbf{Opportunity Cost (OC)}: The highest valued alternative for the action.
\newline
\textbf{Pareto Efficiency}: An allocation is pareto efficient if there is no
way to make an individual better off without making someone else worse off.
\newline
\textbf{Positive Statement}: An expression of how things are.
\newline
\textbf{Price Ceiling}: A regulation making it illegal to charge more than
a specified price.
\newline
\textbf{Price Elasticity of Demand}: A measure of the responsiveness of the
demand for a good to a change in its \textit{price}, ceteris paribus.
\newline
\textbf{Price Floor}: A regulation making it illegal to charge less than
a specified price.
\newline
\textbf{Production Efficiency}: Goods are produced at the lowest possible
opportunity cost.
\newline
\textbf{Production Possibilities Frontier (PPF)}: The set of all possible goods
that can be produced at a given point in time.
\newline
\textbf{Subsidies}: A payment made by the government to a producer or a
consumer.
\newline
\textbf{Supply}: The entire relationship between the quantity supplied of a
good and its price.
\newline
\textbf{Tax}: A levy imposed on an individual or legal entity by the state
whose failure to be paid can be punishable by law.
\newline
\textbf{Tax Incidence}: The division of the burden of a tax between buyers and
sellers.
\newline
\textbf{Total Benefit}: The total amount of a good that an individual is
willing to give up for some amount of another good.
\newline
\textbf{Total Fixed Cost}: The sum of all fixed costs in the production
process.
\newline
\textbf{Total Revenue}: The price of a good times the quantity sold.
\newline
\textbf{Total Utility}: The total benefit or satisfaction a person gets from
consuming goods.
\newline
\textbf{Total Variable Cost}: The sum of all variable costs in the production
process at a certain level \( p \).

\begin{center}
  You can find all my notes at \url{http://omgimanerd.tech/notes}. If you have
  any questions, comments, or concerns, please contact me at
  alvin@omgimanerd.tech
\end{center}

\end{document}
