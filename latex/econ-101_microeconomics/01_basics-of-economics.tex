\documentclass[letterpaper, 12pt]{article}

\title{Basics of Economics}
\author{Alvin Lin}
\date{Principles of Microeconomics: August 2016 - December 2016}

\begin{document}

\maketitle

\section*{Basics of Economics}
Economics is a social science that studies how people relate to each other in
a society. It studies how societies allocate scarce resources. If resources were
unlimited and freely available, the problem would be trivial.

\subsection*{Fundamental Assumption}
The Fundamental Assumption in economics is that people face the problem
of scarcity. Scarcity arises when people have unlimited wants, but there are not
enough resources to satisfy those wants. Unlimited wants is not required for
a problem to arise, the problem exists when the total amount of a good desired
exceeds the total amount that is available.

\subsection*{Rule of Thumb}
A good is scarce if whenever the price of it is zero there is not enough of it
provided to satisfy everyone's demand. Rarity is not the same as scarcity.

\subsection*{The Fundamental Problem of Economics}
How \textit{do} societies deal with problems of scarcity?
\begin{itemize}
  \item We can find some answers by looking out into the world.
\end{itemize}
How \textit{should} societies deal with the problem of scarcity?
\begin{itemize}
  \item Should we train more nurses, or more lawyers?
  \item More food, or more weapons?
  \item Who gets the goods eventually?
  \item Should this be based on need or whether not the person can afford the
        good?
\end{itemize}
Scarcity involves tradeoffs. A society can invest more in education children,
or in national defense, but perhaps not both. A individual can fix her car, or
take a vacation, but not both.

\subsection*{6 Key Ideas}
\begin{itemize}
  \item \textbf{Tradeoffs}: Individuals face tradeoffs when making choices.
  This is because of scarcity, something must be given up in order to get
  something else.
  \item \textbf{Benefits}: The benefit to a person of an activity is the most
  that a person is willing to give up in order to undertake that activity.
  \item \textbf{Costs}: The opportunity cost is what a person actually
  sacrifices when choosing one option over the other.
  \item \textbf{Choosing at the Margin}: Choosing at the margin involves
  the cost-benefit analysis of an activity to determine how much of it one
  should undertake.
  \item \textbf{People Make Rational Choices}: The idea that people make cold
  rational calculations is an assumption of our economics model.
  \item \textbf{Choices Respond to Incentives}: Individuals take into account
  costs and benefits when making choices.
\end{itemize}

\begin{center}
  If any errors are found, please contact me at alvin.lin.dev@gmail.com
\end{center}

\end{document}
