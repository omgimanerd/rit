\documentclass[letterpaper, 12pt]{article}
\usepackage{amsmath}

\title{Basics of Economics}
\author{Alvin Lin}
\date{Principles of Microeconomics: August 2016 - December 2016}

\begin{document}

\maketitle

\section{Perfect Competition}
A perfectly competitive market involves many firms selling an identical product
to many buyers, no restrictions on entry, and no advantage to established firms
where buyers and sellers are informed about prices.

\subsection{Marginal Revenue (MR)}
Marginal revenue is the change in total revenue obtained by a seller as a
result of a change in the total quantity sold.

\subsection{The Rule for Maximizing Profit}
If a perfectly competitive firm produces, it choose a quantity such that
\( MC = MR \). Since \( MR = p \) for a perfectly competitive firm,
\( p = MC \) gives the profit maximizing quantity for a perfectly competitive
firm.

\subsection{Supply Curve of the Firm}
There exists a point called the shutdown point where \( p < AVC \). The firm
shuts down at this point. The supply curve of the firm is the \( MC \) curve
lying above the \( AVC \) curve. \par
If a firm chooses \( q^{*} > 0 \), then:
\begin{align*}
  \mathrm{profit} &= p \times q^{*}-TC(q^{*}) \\
  &= p \times q^{*}-TVC(q^{*})-TFC \\
  &= p \times q^{*}-q^{*}AVC(q^{*})-TFC \\
  &= q^{*}(p-AVC(q^{*}))-TFC
\end{align*}

\begin{center}
  If any errors are found, please contact me at alvin.lin.dev@gmail.com
\end{center}

\end{document}
