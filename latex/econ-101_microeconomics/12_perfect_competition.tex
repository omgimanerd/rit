\documentclass[letterpaper, 12pt]{article}
\usepackage{amsmath}

\title{Basics of Economics}
\author{Alvin Lin}
\date{Principles of Microeconomics: August 2016 - December 2016}

\begin{document}

\maketitle

\section{Perfect Competition}
A perfectly competitive market involves many firms selling an identical product
to many buyers, no restrictions on entry, and no advantage to established firms
where buyers and sellers are informed about prices.

\subsection{Marginal Revenue (MR)}
Marginal revenue is the change in total revenue obtained by a seller as a
result of a change in the total quantity sold.

\subsection{The Rule for Maximizing Profit}
If a perfectly competitive firm produces, it choose a quantity such that
\( MC = MR \). Since \( MR = p \) for a perfectly competitive firm,
\( p = MC \) gives the profit maximizing quantity for a perfectly competitive
firm.

\subsection{Supply Curve of the Firm}
There exists a point called the shutdown point where \( p < AVC \). The firm
shuts down at this point. The supply curve of the firm is the \( MC \) curve
lying above the \( AVC \) curve. \par
If a firm chooses \( q^{*} > 0 \), then:
\begin{align*}
  \mathrm{profit} &= p \times q^{*}-TC(q^{*}) \\
  &= p \times q^{*}-TVC(q^{*})-TFC \\
  &= p \times q^{*}-q^{*}AVC(q^{*})-TFC \\
  &= q^{*}(p-AVC(q^{*}))-TFC
\end{align*}

\subsection{Possible Outcomes in the Short Run}
\begin{itemize}
  \item If \( p < AVC \), the firm chooses \( q = 0 \) and loses its total
    fixed cost.
  \item If \( AVC \leq p < ATC \), the firm operates at a loss, but covers a
    portion of the fixed cost.
  \item If \( p = ATC \), the firm operates with no net gain or loss.
  \item If \( p > ATC \), the firm receives a net gain from operation.
\end{itemize}
In the long run, firms that fall under the first two cases tend to drop out of
the market.

\subsection{Economic versus Accounting Profit}

\subsubsection{Accounting Profit}
Accounting profit for an enterprise is represented by the explicit monetary
costs and opportunity costs of operating business.

\subsubsection{Economic Profit}
Economic profit for a given enterprise is the accounting profit of the enter-
prise minus the accounting profit from the next highest perfect competition
enterprise.

\subsection{Market Short Run Supply Curve}
The market short run supply curve is the horizontal sum of all short
run supply curves in a given market, much like market consumer demand
or market supply. This short run equilibrium will be dependent on the
operational choices of firms, which are, in turn, dependent on the whether or
not the firm is operating with a net gain.

\subsection{Market Long Run Supply Curve}
If firms are operating at a loss during the transition from short run to long
run, they will not be as viable as other perfectly competitive industries and
drop out of the market. \par
Firms operating at a profit will continue to do so.
The market long run supply curve will shift left according to the number of
units lost by the firms dropping out and bring the price up. The price will
stop increasing and firms will stop dropping out the market price is equal to
the average total cost for all firms. \par
If firms are universally making a profit in the short run, the market is more
attractive as it transitions to the long run. Firms will join the market,
increasing supply and decreasing the price until it reaches the average total
cost. \par
Essentially, all firms will eventually make enough profit to break even in the
long run, regardless of the short run situation.

\begin{center}
  If any errors are found, please contact me at alvin.lin.dev@gmail.com
\end{center}

\end{document}
