\documentclass{math}

\geometry{letterpaper, margin=0.5in}

\title{Multivariable and Vector Calculus: Homework 7}
\author{Alvin Lin}
\date{August 2016 - December 2016}

\begin{document}

\maketitle

\section*{Section 15.3}

\subsubsection*{Exercise 9}
Evaluate the given integral by changing to polar coordinates.
\[ \iint_{R}\sin(x^2+y^2)\diff{A} \]
where \( R \) is the region in the first quadrant between the circles with
center the origin and radii 1 and 3.
\begin{align*}
  \iint_R\sin(x^2+y^2)\diff{A} &=
    \int_{0}^{\frac{\pi}{2}}\int_{1}^{3}\sin(r^2)r\diff{r}\diff{\theta} \\
  &= \int_{0}^{\frac{\pi}{2}}\bigg[-\frac{\cos(r^2)}{2}\bigg]_{1}^{3}
    \diff{\theta} \\
  &= \int_{0}^{\frac{\pi}{2}}\frac{1}{2}(-\cos(9)+\cos(1))\diff{\theta} \\
  &= \frac{1}{2}\bigg[\cos(1)\theta-\cos(9)\theta\bigg]_{0}^{\frac{\pi}{2}} \\
  &= \frac{\pi(\cos(1)-\cos(9))}{4}
\end{align*}

\subsubsection*{Exercise 15}
Use a double integral to find the area of one loop of the rose
\( r = \cos(3\theta) \).
\begin{align*}
  \iint_{D}\diff{A} &= \int_{-\frac{\pi}{4}}^{\frac{\pi}{4}}
    \int_{0}^{\cos(3\theta)}r\diff{r}\diff{\theta} \\
  &= \int_{-\frac{\pi}{4}}^{\frac{\pi}{4}}
    \bigg[\frac{r^2}{2}\bigg]_{0}^{\cos(3\theta)}\diff{\theta} \\
  &= \frac{1}{2}\int_{-\frac{\pi}{4}}^{\frac{\pi}{4}}
    \cos^2(3\theta)\diff{\theta} \\
  &= \frac{1}{2}\int_{-\frac{\pi}{4}}^{\frac{\pi}{4}}
    \frac{(1+\cos(6\theta))}{2}\diff{\theta} \\
  &= \frac{1}{4}\bigg[
    \theta-\frac{\sin(6\theta)}{6}\bigg]_{-\frac{\pi}{4}}^{\frac{\pi}{4}} \\
  &= \frac{1}{4}\left(\frac{\pi}{4}-\frac{1}{6}\sin(\frac{6\pi}{4})+
    \frac{\pi}{4}+\frac{1}{6}\sin(-\frac{6\pi}{4})\right) \\
  &= \frac{\pi}{8}
\end{align*}

\subsubsection*{Exercise 17}
Use a double integral to find the area of the region inside the circle
\( (x-1)^2+y^2 = 1 \) and outside the circle \( x^2+y^2 = 1 \).
\begin{align*}
  (x-1)^2+y^2 &= 1 \\
  x^2-2x+1+y^2 &= 1 \\
  x^2+y^2-2x &= 0 \\
  r^2-2r\cos\theta &= 0 \\
  r &= 2\cos\theta \\
  x^2+y^2 &= 1 \\
  r &= 1 \\
  r &= 1 = 2\cos\theta \\
  \theta &= -\frac{\pi}{3} \quad \theta = \frac{\pi}{3} \\
  \iint_{D}\diff{A} &= \int_{-\frac{\pi}{3}}^{\frac{\pi}{3}}
    \int_{1}^{2\cos\theta}r\diff{r}\diff{\theta} \\
  &= 2\int_{0}^{\frac{\pi}{3}}
    \bigg[\frac{r^2}{2}\bigg]_{1}^{2\cos\theta}\diff{\theta} \\
  &= \int_{0}^{\frac{\pi}{3}}4\cos^2\theta-1\diff{\theta} \\
  &= \int_{0}^{\frac{\pi}{3}}4\frac{\cos2\theta-1}{2}-1\diff{\theta} \\
  &= \int_{0}^{\frac{\pi}{3}}2\cos2\theta-3\diff{\theta} \\
  &= \bigg[\bigg]_{0}^{\frac{\pi}{3}}
\end{align*}

\subsubsection*{Exercise 21}
\subsubsection*{Exercise 29}
\subsubsection*{Exercise 31}

\section*{Section 15.4}

\section*{Section 15.6}


15.3     page 1014    problems
15.4     page 1024    problems   7,15
15.6     page 1029   problems   11,13,21,33​

\begin{center}
  If you have any questions, comments, or concerns, please contact me at
  alvin@omgimanerd.tech
\end{center}

\end{document}
