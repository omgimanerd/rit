\documentclass{math}

\geometry{letterpaper, margin=0.5in}

\title{Multivariable and Vector Calculus: Homework 8}
\author{Alvin Lin}
\date{August 2016 - December 2016}

\begin{document}

\maketitle

\section*{Section 15.3}

\subsubsection*{Exercise 9}
Evaluate the given integral by changing to polar coordinates.
\[ \iint_{R}\sin(x^2+y^2)\diff{A} \]
where \( R \) is the region in the first quadrant between the circles with
center the origin and radii 1 and 3.
\begin{align*}
  \iint_R\sin(x^2+y^2)\diff{A} &=
    \int_{0}^{\frac{\pi}{2}}\int_{1}^{3}\sin(r^2)r\diff{r}\diff{\theta} \\
  &= \int_{0}^{\frac{\pi}{2}}\bigg[-\frac{\cos(r^2)}{2}\bigg]_{1}^{3}
    \diff{\theta} \\
  &= \int_{0}^{\frac{\pi}{2}}\frac{1}{2}(-\cos(9)+\cos(1))\diff{\theta} \\
  &= \frac{1}{2}\bigg[\cos(1)\theta-\cos(9)\theta\bigg]_{0}^{\frac{\pi}{2}} \\
  &= \frac{\pi(\cos(1)-\cos(9))}{4} \\
  &= \frac{\pi}{4}(\cos(1)-\cos(9))
\end{align*}

\subsubsection*{Exercise 15}
Use a double integral to find the area of one loop of the rose
\( r = \cos(3\theta) \).
\begin{align*}
  \iint_{D}\diff{A} &= \int_{-\frac{\pi}{4}}^{\frac{\pi}{4}}
    \int_{0}^{\cos(3\theta)}r\diff{r}\diff{\theta} \\
  &= \int_{-\frac{\pi}{4}}^{\frac{\pi}{4}}
    \bigg[\frac{r^2}{2}\bigg]_{0}^{\cos(3\theta)}\diff{\theta} \\
  &= \frac{1}{2}\int_{-\frac{\pi}{4}}^{\frac{\pi}{4}}
    \cos^2(3\theta)\diff{\theta} \\
\end{align*}
\begin{align*}
  &= \frac{1}{2}\int_{-\frac{\pi}{4}}^{\frac{\pi}{4}}
    \frac{(1+\cos(6\theta))}{2}\diff{\theta} \\
  &= \frac{1}{4}\bigg[
    \theta-\frac{\sin(6\theta)}{6}\bigg]_{-\frac{\pi}{4}}^{\frac{\pi}{4}} \\
  &= \frac{1}{4}\left(\frac{\pi}{4}-\frac{1}{6}\sin(\frac{6\pi}{4})+
    \frac{\pi}{4}+\frac{1}{6}\sin(-\frac{6\pi}{4})\right) \\
  &= \frac{\pi}{8}
\end{align*}

\subsubsection*{Exercise 17}
Use a double integral to find the area of the region inside the circle
\( (x-1)^2+y^2 = 1 \) and outside the circle \( x^2+y^2 = 1 \).
\begin{align*}
  (x-1)^2+y^2 &= 1 \\
  x^2-2x+1+y^2 &= 1 \\
  x^2+y^2-2x &= 0 \\
  r^2-2r\cos\theta &= 0 \\
  r &= 2\cos\theta \\
  x^2+y^2 &= 1 \\
  r &= 1 \\
  r &= 1 = 2\cos\theta \\
  \theta &= -\frac{\pi}{3} \quad \theta = \frac{\pi}{3} \\
  \iint_{D}\diff{A} &= \int_{-\frac{\pi}{3}}^{\frac{\pi}{3}}
    \int_{1}^{2\cos\theta}r\diff{r}\diff{\theta} \\
  &= 2\int_{0}^{\frac{\pi}{3}}
    \bigg[\frac{r^2}{2}\bigg]_{1}^{2\cos\theta}\diff{\theta} \\
  &= \int_{0}^{\frac{\pi}{3}}4\cos^2\theta-1\diff{\theta} \\
  &= \int_{0}^{\frac{\pi}{3}}4\frac{\cos2\theta+1}{2}-1\diff{\theta} \\
  &= \int_{0}^{\frac{\pi}{3}}2\cos2\theta-1\diff{\theta} \\
  &= \bigg[-\sin(2\theta)-\theta\bigg]_{0}^{\frac{\pi}{3}} \\
  &= \sin(\frac{2\pi}{3}-\frac{\pi}{3}+\sin(0)) \\
  &= \frac{\sqrt{3}}{2}-\frac{\pi}{3}
\end{align*}

\subsubsection*{Exercise 21}
Use polar coordinates to find the volume below the plane \( 2x+y+z = 4 \) and
above the disk \( x^2+y^2\le1 \).
\begin{align*}
  x(x,y) &= 4-2x-y \\
  \iint_{D}4-2x-y\diff{A} &= \int_{0}^{2\pi}\int_{0}^{1}
    4-2(r\cos\theta)-(r\sin\theta)r\diff{r}\diff{\theta} \\
  &= \int_{0}^{2\pi}\bigg[2r-\frac{2r^2\sin\theta}{3}+\frac{r^2\cos\theta}{3}
    \bigg]_{0}^{1}\diff{\theta} \\
  &= \int_{0}^{2\pi}
    2-\frac{2}{3}\sin\theta+\frac{1}{3}\cos\theta\diff{\theta} \\
  &= \bigg[
    2\theta+\frac{2}{3}\cos\theta-\frac{1}{3}\sin\theta\bigg]_{0}^{2\pi} \\
  &= 4\pi+\frac{2}{3}\cos2\pi-\frac{1}{2}\sin2\pi-
    0-\frac{2}{3}\cos0+\frac{1}{3}\sin0 \\
  &= 4\pi
\end{align*}

\subsubsection*{Exercise 29}
Evaluate the iterated integral by converting it to polar coordinates.
\begin{align*}
  \int_{0}^{2}\int_{0}^{\sqrt{4-x^2}}\e^{-x^2-y^2}\diff{y}\diff{x} &=
    \int_{0}^{\frac{\pi}{2}}\int_{0}^{2}\e^{(-1)r^2}r\diff{r}\diff{\theta} \\
  &= \int_{0}^{\frac{\pi}{2}}
    \bigg[-\frac{\e^{-r^2}}{2}\bigg]_{0}^{2}\diff{\theta} \\
  &= \int_{0}^{\frac{\pi}{2}}\frac{1}{2}-\frac{\e^{-4}}{2}\diff{\theta} \\
  &= \frac{1}{2}\bigg[\theta-\frac{\theta}{\e^4}\bigg]_{0}^{\frac{\pi}{2}} \\
  &= \frac{\pi}{4}-\frac{\pi}{4\e^4}
\end{align*}

\subsubsection*{Exercise 31}
Evaluate the iterated integral by converting it to polar coordinates.
\begin{align*}
  \int_{0}^{\frac{1}{2}}\int_{\sqrt{3}y}^{\sqrt{1-y^2}}xy^2\diff{x}\diff{y} &=
    \int_{0}^{\frac{\pi}{6}}\int_{0}^{1}
      (r\cos\theta)(r\sin\theta)^2r\diff{r}\diff{\theta} \\
  &= \int_{0}^{\frac{\pi}{6}}\int_{0}^{1}
    r^4\cos\theta\sin^2\theta\diff{r}\diff{\theta} \\
  &= \int_{0}^{\frac{\pi}{6}}
    \bigg[\frac{r^5\cos\theta\sin^2\theta}{5}\bigg]_{0}^{1}\diff{\theta} \\
  &= \frac{1}{5}\int_{0}^{\frac{\pi}{6}}
    \cos\theta\sin^2\theta\diff{\theta} \\
  &= \frac{1}{5}\bigg[\frac{\sin^3\theta}{3}\bigg]_{0}^{\frac{\pi}{6}} \\
  &= \frac{1}{15}\bigg(\sin^3(\frac{\pi}{6})-\sin^3(0)\bigg) \\
  &= \frac{1}{15}(\frac{1^3}{2^3}-0) \\
  &= \frac{1}{120}
\end{align*}

\section*{Section 15.4}

\subsubsection*{Exercise 7}
Find the mass and center of mass of the lamina that occupies the region \( D \)
and has the given density function \( \rho \).
\[ D \text{ is bounded by } y = 1-x^2 \text{ and } y = 0;\quad\rho(x,y) = ky \]
\begin{align*}
  0 &= 1-x^2 \\
  x &= \pm1 \\
  mass &= \iint_{D}\rho(x,y)\diff{A} \\
  &= \int_{-1}^{1}\int_{0}^{1-x^2}ky\diff{y}\diff{x} \\
  &= 2\int_{0}^{1}\bigg[\frac{ky^2}{2}\bigg]_{0}^{1-x^2}\diff{x} \\
  &= k\int_{0}^{1}1-2x^2+x^4\diff{x} \\
  &= k\bigg[x-\frac{2x^3}{3}+\frac{x^5}{5}\bigg]_{0}^{1} \\
  &= k(1-\frac{2}{3}+\frac{1}{5}) \\
  &= \frac{8k}{15}
\end{align*}
\begin{align*}
  x_{CoM} &= \frac{1}{mass}\iint_{D}x\rho(x,y)\diff{A} \\
  &= \frac{15}{8k}\int_{-1}^{1}\int_{0}^{1-x^2}xky\diff{y}\diff{x} \\
  &= k\frac{15}{8k}\int_{-1}^{1}\int_{0}^{1-x^2}xy\diff{y}\diff{x} \\
  &= \frac{15}{8}\int_{-1}^{1}\bigg[\frac{xy^2}{2}\bigg]_{0}^{1-x^2}\diff{x} \\
  &= \frac{15}{16}\int_{-1}^{1}x(1-2x^2+x^4)\diff{x} \\
  &= \frac{15}{16}\int_{-1}^{1}x-2x^3+x^5\diff{x} \\
  &= \frac{15}{16}
    \bigg[\frac{x^2}{2}-\frac{2x^4}{4}+\frac{x^6}{6}\bigg]_{-1}^{1} \\
  &= 0
\end{align*}
\begin{align*}
  y_{CoM} &= \frac{1}{mass}\iint_{D}y\rho(x,y)\diff{A} \\
  &= \frac{15}{8k}\int_{-1}^{1}\int_{0}^{1-x^2}ky^2\diff{y}\diff{x} \\
  &= \frac{15}{8}\int_{-1}^{1}\bigg[\frac{y^3}{3}\bigg]_{0}^{1-x^2}\diff{y} \\
  &= \frac{5}{8}\int_{-1}^{1}(1-x^2)^3\diff{x} \\
  &= \frac{5}{8}\int_{-1}^{1}1-x^6-3x^2+3x^4\diff{x} \\
  &= \frac{5}{8}\bigg[x-\frac{x^7}{7}-x^3+\frac{3x^5}{5}\bigg]_{-1}^{1} \\
  &= \frac{5}{8}(1-\frac{1}{7}-1+\frac{3}{5}+1-\frac{1}{7}-1+\frac{3}{5}) \\
  &= \frac{5}{8}(-\frac{2}{7}+\frac{6}{5}) \\
  &= \frac{5}{8}\frac{32}{35} \\
  &= \frac{4}{7}
\end{align*}
Center of mass: \( \langle0,\frac{4}{7}\rangle \)

\subsubsection*{Exercise 15}
Find the center of mass of a lamina in the shape of an isosceles right triangle
with equal sides of length \( a \) is the density at any point is proportional
to the square of the distance from the vertex opposite the hypotenuse. \\
Treat the origin as the right angle with the legs \( a \) extending along the x
and y axes.
\[ \rho(x,y) = x^2+y^2 \]
\begin{align*}
  mass &= \iint_{D}\rho(x,y)\diff{A} \\
  &= \int_{0}^{a}\int_{0}^{a-y}x^2+y^2\diff{x}\diff{y} \\
  &= \int_{0}^{a}\bigg[\frac{x^3}{3}+xy^2\bigg]_{0}^{a-y}\diff{y} \\
  &= \int_{0}^{a}\frac{(a-y)^3}{3}+y^2(a-y)\diff{y} \\
  &= \int_{0}^{a}
    \frac{1}{3}(a^3-3a^2y+3ay^2-y^3)+\frac{1}{3}(3ay^2-3y^3)\diff{y} \\
  &= \frac{1}{3}\int_{0}^{a}a^3-3a^2y+6ay^2-4y^3\diff{y} \\
  &= \frac{1}{3}\bigg[a^3y-\frac{3a^2y^2}{2}+2ay^3-y^4\bigg]_{0}^{a} \\
  &= \frac{1}{3}(a^4-\frac{3a^4}{2}+2a^4-a^4) \\
  &= \frac{a^4}{6} \\
  x_{CoM} = y_{CoM} &= \frac{1}{mass}\iint_{D}y\rho(x,y)\diff{A} \\
  &= \frac{6}{a^4}\int_{0}^{a}\int_{0}^{a-y}y(x^2+y^2)\diff{x}\diff{y} \\
  &= \frac{6}{a^4}\int_{0}^{a}
    \bigg[\frac{x^3y}{3}+xy^3\bigg]_{0}^{a-y}\diff{y} \\
  &= \frac{6}{a^4}\int_{0}^{a}
    \frac{y}{3}(a^3-3a^2y+3ay^2-y^3)+y^3(a-y)\diff{y} \\
  &= \frac{6}{a^4}\int_{0}^{a}
    \frac{a^3y}{3}-a^2y^2+ay^3-\frac{y^4}{3}+ay^3-y^4\diff{y} \\
  &= \frac{6}{a^4}\int_{0}^{a}
    \frac{a^3y}{3}-a^2y^2+2ay^3-\frac{4y^4}{3}\diff{y} \\
  &= \frac{6}{a^4}\bigg[\frac{a^3y^2}{6}-\frac{a^2y^3}{3}+\frac{2ay^4}{4}-
    \frac{4y^5}{15}\bigg]_{0}^{a} \\
  &= \frac{6}{a^4}\bigg[
    \frac{a^5}{6}-\frac{a^5}{3}+\frac{a^5}{2}-\frac{4a^5}{15}] \\
  &= a-2a+3a-\frac{8a}{5} \\
  &= \frac{2a}{5}
\end{align*}
Center of mass: \( \langle\frac{2a}{5},\frac{2a}{5}\rangle \)

\section*{Section 15.6}

\subsection*{Exercise 11}
Evaluate the triple integral.
\begin{align*}
  \iiint_{E}\frac{z}{x^2+z^2}\diff{V} &=
    \int_{1}^{4}\int_{y}^{4}\int_{0}^{z}
    \frac{z}{x^2+z^2}\diff{x}\diff{z}\diff{y} \\
  &= \int_{1}^{4}\int_{y}^{4}
    z\bigg[\frac{1}{z}\arctan(\frac{x}{z})]_{0}^{z}\diff{z}\diff{y} \\
  &= \int_{1}^{4}\int_{y}^{4}\frac{\pi}{4}\diff{z}\diff{y} \\
  &= \int_{1}^{4}\bigg[\frac{\pi z}{4}\bigg]_{y}^{4}\diff{y} \\
  &= \int_{1}^{4}\pi-\frac{\pi y}{4}\diff{y} \\
  &= \bigg[\pi y-\frac{\pi y^2}{8}\bigg]_{1}^{4} \\
  &= 4\pi-2\pi-(\pi-\frac{\pi}{8}) \\
  &= \frac{9\pi}{8}
\end{align*}

\subsection*{Exercise 13}
Evaluate the triple integral \( \iiint_{E}6xy\diff{V} \) where \( E \) lies
below the plane \( z = 1+x+y \) and above the region in the xy-plane bounded by
the curves \( y = \sqrt{x}, y = 0, x = 1 \).
\begin{align*}
  \iiint_{E}6xy\diff{V} &=
    \int_{0}^{1}\int_{0}^{\sqrt{x}}\int_{0}^{1+x+y}
    6xy\diff{z}\diff{y}\diff{x} \\
  &= \int_{0}^{1}\int_{0}^{\sqrt{x}}
    \bigg[6xyz\bigg]_{0}^{1+x+y}\diff{y}\diff{x} \\
  &= \int_{0}^{1}\int_{0}^{\sqrt{x}}6xy(1+x+y)\diff{y}\diff{x} \\
  &= \int_{0}^{1}\bigg[3xy^2+3x^2y^2+2xy^3\bigg]_{0}^{\sqrt{x}}\diff{x} \\
  &= \int_{0}^{1}3x^2+3x^3+2x^{\frac{5}{2}}\diff{x} \\
  &= \bigg[x^3+\frac{3x^4}{4}+\frac{4}{7}x^{\frac{4}{7}}\bigg]_{0}^{1} \\
  &= 1+\frac{3}{4}+\frac{4}{7} \\
  &= \frac{28}{28}+\frac{21}{28}+\frac{16}{28} \\
  &= \frac{65}{28}
\end{align*}

\subsection*{Exercise 21}
Use a triple integral to find the volume of the solid enclosed by the cylinder
\( y = x^2 \) and the planes \( z = 0 \) and \( y+z = 1 \).
\begin{align*}
  \{ (x,y,z)&\in E\mid -1\le x\le 1, x^2\le y\le 1, 0\le z\le1-y\} \\
  \iiint_{E}\diff{V} &= \int_{-1}^{1}\int_{x^2}^{1}\int_{0}^{1-y}
    \diff{z}\diff{y}\diff{x} \\
  &= \int_{-1}^{1}\int_{x^2}^{1}\bigg[z\bigg]_{0}^{1-y}\diff{y}\diff{x} \\
  &= \int_{-1}^{1}\int_{x^2}^{1}1-y\diff{y}\diff{x} \\
  &= \int_{-1}^{1}\bigg[y-\frac{y^2}{2}\bigg]_{x^2}^{1}\diff{x} \\
  &= \int_{-1}^{1}1-\frac{1}{2}-x^2+\frac{x^4}{2}\diff{x} \\
  &= \bigg[\frac{x^5}{10}-\frac{x^3}{3}+\frac{x}{2}\bigg]_{-1}^{1} \\
  &= \frac{1}{10}-\frac{1}{3}+\frac{1}{2}+
    \frac{1}{10}-\frac{1}{3}+\frac{1}{2} \\
  &= \frac{1}{5}-\frac{2}{3}+1 \\
  &= \frac{8}{15}
\end{align*}

\subsection*{Exercise 33}
The figure shows the region of integration for the integral
\[ \int_{0}^{1}\int_{\sqrt{x}}^{1}\int_{0}^{1-y}
  f(x,y,z)\diff{z}\diff{y}\diff{x} \]
Rewrite this integral as an equivalent iterated integral in the five other
orders.
\begin{align*}
  \int_{0}^{1}\int_{\sqrt{x}}^{1}\int_{0}^{1-y}
    f(x,y,z)\diff{z}\diff{y}\diff{x} &=
  \int_{0}^{1}\int_{0}^{1-y}\int_{0}^{y^2}\diff{x}\diff{y}\diff{z} \\
  &= \int_{0}^{1}\int_{0}^{1-z}\int_{0}^{y^2}\diff{x}\diff{z}\diff{y} \\
  &= \int_{0}^{1}\int_{0}^{(1-z)^2}\int_{\sqrt{x}}^{1-z}
    \diff{y}\diff{x}\diff{z} \\
  &= \int_{0}^{1}\int_{0}^{1-\sqrt{x}}\int_{\sqrt{x}}^{1-z}
    \diff{y}\diff{z}\diff{x} \\
  &= \int_{0}^{1}\int_{0}^{y^2}\int_{0}^{1-y}\diff{z}\diff{x}\diff{y} \\
\end{align*}

\begin{center}
  If you have any questions, comments, or concerns, please contact me at
  alvin@omgimanerd.tech
\end{center}

\end{document}
