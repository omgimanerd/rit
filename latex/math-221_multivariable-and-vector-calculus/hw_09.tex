\documentclass{math}

\geometry{letterpaper, margin=0.5in}

\title{Multivariable and Vector Calculus: Homework 9}
\author{Alvin Lin}
\date{August 2016 - December 2016}

\begin{document}

\maketitle

\section*{Section 15.7}

\subsubsection*{Exercise 7}
Identify the surface whose equation is given.
\[ r^2+z^2 = 4 \]
\[ x^2+y^2+z^2 = 4 \]
A sphere centered at the origin with radius 2.

\subsubsection*{Exercise 8}
Identify the surface whose equation is given.
\begin{align*}
  r &= 2\sin\theta \\
  r^2 &= 2r\sin\theta \\
  x^2+y^2 &= 2y \\
  x^2+y^2-2y &= 0 \\
  x^2+y^2-2y+1 &= 1 \\
  x^2+(y-1)^2 &= 1
\end{align*}
A cylinder with radius 1 centered at (0,1).

\subsubsection*{Exercise 17}
Use cylindrical coordinates to evaluate \( \iiint_{E}\sqrt{x^2+y^2}\diff{V} \)
where \( E \) is the region that lies inside the cylinder \( x^2+y^2 = 16 \) and
between the planes \( z = -5 \) and \( z = 4 \).
\begin{align*}
  \iiint_{E}\sqrt{x^2+y^2}\diff{V} &=
    \int_{-5}^{4}\int_{0}^{4}\int_{0}^{2\pi}\sqrt{r^2}r
    \diff{z}\diff{r}\diff{\theta} \\
  &= \int_{-5}^{4}\int_{0}^{4}\bigg[zr^2\bigg]_{0}^{2\pi}
    \diff{r}\diff{\theta} \\
  &= \int_{-5}^{4}\int_{0}^{4}2\pi r^2\diff{r}\diff{\theta} \\
  &= \int_{-5}^{4}\bigg[\frac{2\pi r^3}{3}\bigg]_{0}^{4}\diff{\theta} \\
  &= \frac{2\pi}{3}\int_{-5}^{4}64\diff{\theta} \\
  &= \frac{128\pi}{3}\bigg[\theta\bigg]_{-5}^{4} \\
  &= \frac{128\pi(9)}{3} \\
  &= 384\pi
\end{align*}

\subsubsection*{Exercise 21}
Use cylindrical coordinates to evaluate \( \iiint_{E}x^2\diff{V} \) where
\( E \) is the solid that lies within the cylinder \( x^2+y^2 = 1 \), above
the plane \( z = 0 \) and below the cone \( z^2 = 4x^2+4y^2 \).
\begin{align*}
  \iiint_{E}x^2\diff{V} &=
    \int_{0}^{2\pi}\int_{0}^{1}\int_{0}^{\sqrt{4r^2}}
    r^2\sin^2(\theta)r\diff{z}\diff{r}\diff{\theta} \\
  &= \int_{0}^{2\pi}\int_{0}^{1}\bigg[zr^3\cos^2\theta\bigg]_{0}^{2r}
    \diff{r}\diff{\theta} \\
  &= \int_{0}^{2\pi}\int_{0}^{1}2r^4\cos^2\theta\diff{r}\diff{\theta} \\
  &= 2\int_{0}^{2\pi}\cos^2\theta\bigg[\frac{r^5}{5}\bigg]_{0}^{1}
    \diff{\theta} \\
  &= \frac{2}{5}\int_{0}^{2\pi}\frac{1+\cos(2\theta)}{2}\diff{\theta} \\
  &= \frac{1}{5}\bigg[\theta+\frac{\sin(2\theta)}{2}\bigg]_{0}^{2\pi} \\
  &= \frac{1}{2}\left(2\pi+\frac{\sin(4\pi)}{2}-0-\frac{\sin(0)}{2}\right) \\
  &= \frac{2\pi}{5}
\end{align*}

\subsubsection*{Exercise 29}
Evaluate the integral by changing to cylindrical coordinates.
\begin{align*}
  \int_{-2}^{2}\int_{-\sqrt{4-y^2}}^{\sqrt{4-y^2}}\int_{\sqrt{x^2+y^2}}^{2}
    xz\diff{z}\diff{x}\diff{y} &= \int_{0}^{2\pi}\int_{0}^{2}\int_{r}^{2}
    r\cos\theta zr\diff{z}\diff{r}\diff{\theta} \\
  &= \int_{0}^{2\pi}\int_{0}^{2}r^2\cos\theta\bigg[z\bigg]_{r}^{2}
    \diff{r}\diff{\theta} \\
  &= \int_{0}^{2\pi}\cos\theta\int_{0}^{2}r^2(2-r)\diff{r}\diff{\theta} \\
  &= \int_{0}^{2\pi}\cos\theta\int_{0}^{2}2r^2-r^3\diff{r}\diff{\theta} \\
  &= \int_{0}^{2\pi}\cos\theta\bigg[\frac{2r^3}{3}-\frac{r^4}{4}\bigg]_{0}^{2}
    \diff{\theta} \\
  &= \int_{0}^{2\pi}\cos\theta\left(\frac{16}{3}-4\right)\diff{\theta} \\
  &= \frac{4}{3}\bigg[-\sin\theta\bigg]_{0}^{2\pi} \\
  &= \frac{4}{3}\bigg(-\sin(2\pi)+\sin(0)\bigg) \\
  &= 0
\end{align*}

\section*{Section 15.8}

\subsubsection*{Exercise 7}
Identify the surface whose equation is given.
\[ \rho\cos\phi = 1 \]
\[ z = 1 \]
The plane \( z = 1 \).

\subsubsection*{Exercise 13}
Sketch the solid described by the given inequalities.
\[ 2\le\rho\le4, 0\le\phi\le\frac{\pi}{3}, 0\le\theta\le\pi \]
The region between half-spheres of radius of 2 and 4 contained between 0 and
\( \frac{\pi}{3} \).

\subsubsection*{Exercise 17}
Sketch the solid whose volume is given by the integral and evaluate the
integral. The slice of a quarter sphere of radius 3 contained between 0 and
\( \frac{\pi}{6} \).
\begin{align*}
  \int_{0}^{\frac{\pi}{6}}\int_{0}^{\frac{\pi}{2}}\int_{0}^{3}
    \rho^2\sin\phi\diff{\rho}\diff{\theta}\diff{\phi} &=
    \int_{0}^{\frac{\pi}{6}}\sin\phi\int_{0}^{\frac{\pi}{2}}
    \bigg[\frac{\rho^3}{3}\bigg]_{0}^{3}\diff{\theta}\diff{\phi} \\
  &= 9\int_{0}^{\frac{\pi}{6}}\sin\phi\int_{0}^{\frac{\pi}{2}}
    \diff{\theta}\diff{\phi} \\
  &= 9\int_{0}^{\frac{\pi}{6}}\sin\phi\bigg[\theta\bigg]_{0}^{\frac{\pi}{2}}
    \diff{\phi} \\
  &= \frac{9\pi}{2}\int_{0}^{\frac{\pi}{6}}\sin\phi\diff{\phi} \\
  &= \frac{9\pi}{2}\bigg[-\cos\phi\bigg]_{0}^{\frac{\pi}{6}} \\
  &= \frac{9\pi}{2}\left(-\cos(\frac{\pi}{6})+\cos(0)\right) \\
  &= \frac{9\pi}{2}\left(-\frac{\sqrt{3}}{2}+1\right) \\
  &= \frac{18\pi-9\pi\sqrt{3}}{4}
\end{align*}

\subsubsection*{Exercise 21}
Use spherical coordinates to evaluate \( \iiint_{B}(x^2+y^2+z^2)^2\diff{V} \)
where \( B \) is the ball with center the origin and radius 5.
\begin{align*}
  \iiint_{B}(x^2+y^2+z^2)^2\diff{V} &=
    \int_{0}^{\pi}\int_{0}^{2\pi}\int_{0}^{5}(\rho^2)^2\rho^2\sin\phi
    \diff{\rho}\diff{\theta}\diff{\phi} \\
  &= \int_{0}^{\pi}\sin\phi\int_{0}^{2\pi}
    \bigg[\frac{\rho^7}{7}\bigg]_{0}^{5}\diff{\theta}\diff{\phi} \\
  &= \frac{78125}{7}\int_{0}^{\pi}\sin\phi
    \bigg[\theta\bigg]_{0}^{2\pi}\diff{\phi} \\
  &= \frac{78125}{7}\int_{0}^{\pi}2\pi\sin\phi\diff{\phi} \\
  &= \frac{2\pi(78125)}{7}\bigg[-\cos\phi\bigg]_{0}^{\pi} \\
  &= \frac{2\pi(78125)}{7}\bigg(-\cos\pi+\cos(0)\bigg) \\
  &= \frac{312500\pi}{7}
\end{align*}

\subsubsection*{Exercise 23}
Use spherical coordinates to evaluate \( \iiint_{E}(x^2+y^2)\diff{V} \), where
\( E \) lies between the spheres \( x^2+y^2+z^2 = 4 \) and
\( x^2+y^2+z^2 = 9 \).
\begin{align*}
  \iiint_{E}(x^2+y^2)\diff{V} &= \int_{0}^{\pi}\int_{0}^{2\pi}\int_{2}^{3}
    (\rho\sin\phi\cos\theta)^2+(\rho\sin\phi\sin\theta)^2\rho^2\sin\phi
    \diff{\rho}\diff{\theta}\diff{\phi} \\
  &= \int_{0}^{\pi}\int_{0}^{2\pi}\int_{2}^{3}
    \rho^2\sin^2\phi(\cos^2\theta+\sin^2\theta)\rho^2\sin\phi
    \diff{\rho}\diff{\theta}\diff{\phi} \\
  &= \int_{0}^{\pi}\sin^3\phi\int_{0}^{2\pi}
    \bigg[\frac{\rho^5}{5}\bigg]_{2}^{3}\diff{\theta}\diff{\phi} \\
  &= \frac{211}{5}\int_{0}^{\pi}\sin^3\phi\int_{0}^{2\pi}
    \diff{\theta}\diff{\phi} \\
  &= \frac{422\pi}{5}\int_{0}^{\pi}\sin^3\phi\diff{\phi} \\
  &= \frac{422\pi}{5}\int_{0}^{\pi}\sin\phi(1-\cos^2\phi)\diff{\phi} \\
  &= \frac{422\pi}{5}\int_{0}^{\pi}\sin\phi-\sin\phi\cos^2\phi\diff{\phi} \\
  &= \frac{422\pi}{5}\bigg[-\cos\phi+\frac{1}{3}\cos^3\phi\bigg]_{0}^{\pi} \\
  &= \frac{422\pi}{5}
    \bigg(-\cos\pi+\frac{1}{3}\cos^3\pi+\cos(0)-\frac{1}{3}\cos^3(0)\bigg) \\
  &= \frac{422\pi}{5}\bigg(1-\frac{1}{3}+1-\frac{1}{3}\bigg) \\
  &= \frac{422\pi}{5}\frac{4}{3} \\
  &= \frac{1688\pi}{15}
\end{align*}

\subsubsection*{Exercise 25}
Use spherical coordinates to evaluate \( \iiint_{E}x\e^{x^2+y^2+z^2}\diff{V} \),
where \( E \) is the portion of the unit ball \( x^2+y^2+z^2\le1 \) that lies in
the first octant.
\begin{align*}
  \iiint_{E}x\e^{x^2+y^2+z^2}\diff{V} &=
    \int_{0}^{\frac{\pi}{2}}\int_{0}^{\frac{\pi}{2}}\int_{0}^{1}
    \rho\sin\phi\cos\theta\e^{\rho^2}\rho^2\sin\phi
    \diff{\rho}\diff{\theta}\diff{\phi} \\
  &= \int_{0}^{\frac{\pi}{2}}\sin^2\phi\int_{0}^{\frac{\pi}{2}}\cos\theta
    \int_{0}^{1}\rho^3\e^{\rho^2}\diff{\rho}\diff{\theta}\diff{\phi} \\
  &= \frac{1}{2}\int_{0}^{\frac{\pi}{2}}\sin^2\phi
    \int_{0}^{\frac{\pi}{2}}\cos\theta
    \bigg[\e^{x^2}x^2-\e^{x^2}\bigg]_{0}^{1}\diff{\theta}\diff{\phi} \\
  &= \frac{1}{2}\int_{0}^{\frac{\pi}{2}}\sin^2\phi
    \int_{0}^{\frac{\pi}{2}}\cos\theta(\e-\e+1)\diff{\theta}\diff{\phi} \\
  &= \frac{1}{2}\int_{0}^{\frac{\pi}{2}}\sin^2\phi
    \bigg[\sin\theta\bigg]_{0}^{\frac{\pi}{2}}\diff{\phi} \\
  &= \frac{1}{2}\int_{0}^{\frac{\pi}{2}}\sin^2\phi\diff{\phi} \\
  &= \frac{1}{4}\int_{0}^{\frac{\pi}{2}}\phi\diff{\phi} \\
\end{align*}

\subsubsection*{Exercise 27}
Use spherical coordinates to find the volume of the part of the ball \( \rho\le
a \) that lies between the cones \( \phi = \frac{\pi}{6} \) and \( \phi =
\frac{\pi}{3} \).
\begin{align*}
  \int_{\frac{\pi}{6}}^{\frac{\pi}{3}}\int_{0}^{2\pi}\int_{0}^{a}\rho^2\sin\phi
    \diff{\rho}\diff{\theta}\diff{\phi} &=
    \int_{\frac{\pi}{6}}^{\frac{\pi}{3}}\sin\phi\int_{0}^{2\pi}
    \int_{0}^{a}\rho^2\diff{\rho}\diff{\theta}\diff{\phi} \\
  &= \int_{\frac{\pi}{6}}^{\frac{\pi}{3}}\sin\phi\int_{0}^{2\pi}
    \bigg[\frac{\rho^3}{3}\bigg]_{0}^{a}\diff{\theta}\diff{\phi} \\
  &= \frac{a^3}{3}\int_{\frac{\pi}{6}}^{\frac{\pi}{3}}\sin\phi\int_{0}^{2\pi}
    \diff{\theta}\diff{\phi} \\
  &= \frac{2\pi a^3}{3}
    \int_{\frac{\pi}{6}}^{\frac{\pi}{3}}\sin\phi\diff{\phi} \\
  &= \frac{2\pi a^3}{3}\bigg[-\cos\phi\bigg]_{\frac{\pi}{6}}^{\frac{\pi}{3}} \\
  &= \frac{\pi a^3\sqrt{3}-\pi a^3}{3}
\end{align*}

\subsubsection*{Exercise 29a}
Use spherical coordinates to find the volume of the solid that lies above the
cone \( \phi = \frac{\pi}{3} \) and below the sphere \( \rho = 4\cos\phi \).
\begin{align*}
  \int_{0}^{\frac{\pi}{3}}\int_{0}^{2\pi}\int_{0}^{4\cos\phi}\rho^2\sin\phi
    \diff{\rho}\diff{\theta}\diff{\phi} &=
    \int_{0}^{\frac{\pi}{3}}\sin\phi\int_{0}^{2\pi}
    \bigg[\frac{\rho^{3}}{3}\bigg]_{0}^{4\cos\phi}
    \diff{\theta}\diff{\phi} \\
  &= \frac{64}{3}\int_{0}^{\frac{\pi}{3}}\sin\phi\cos^3\phi\int_{0}^{2\pi}
    \diff{\theta}\diff{\phi} \\
  &= \frac{128\pi}{3}\int_{0}^{\frac{\pi}{3}}\sin\phi\cos^3\phi\diff{\phi} \\
  &= \frac{128\pi}{3}\bigg[-\frac{\cos^4\phi}{4}\bigg]_{0}^{\frac{\pi}{3}} \\
  &= \frac{32\pi}{3}\bigg(-\cos^4\frac{\pi}{3}+\cos^4(0)\bigg) \\
  &= \frac{32\pi}{3}\bigg(\frac{15}{16}\bigg) \\
  &= 10\pi
\end{align*}

\subsubsection*{Exercise 35}
Use cylindrical or spherical coordinates, whichever seems more appropriate, to
find the volume and centroid of the solid \( E \) that lies above the cone
\( z = \sqrt{x^2+y^2} \) and below the sphere \( x^2+y^2+z^2 = 1 \).
\begin{align*}
  V &= \iiint_{E}\diff{V} \\
  &= \int_{0}^{\frac{\pi}{4}}\int_{0}^{2\pi}\int_{0}^{1}\rho^2\sin\phi
    \diff{\rho}\diff{\theta}\diff{\phi} \\
  &= \int_{0}^{\frac{\pi}{4}}\sin\phi\int_{0}^{2\pi}
    \bigg[\frac{\rho^3}{3}\bigg]_{0}^{1}\diff{\theta}\diff{\phi} \\
  &= \frac{1}{3}\int_{0}^{\frac{\pi}{4}}\sin\phi\int_{0}^{2\pi}
    \diff{\theta}\diff{\phi} \\
  &= \frac{2\pi}{3}\int_{0}^{\frac{\pi}{4}}\sin\phi\diff{\phi} \\
  &= \frac{2\pi}{3}\bigg[-\cos\phi\bigg]_{0}^{\frac{\pi}{4}} \\
  &= \frac{2\pi-\sqrt{2}\pi}{3}
\end{align*}
This solid is symmetrical about x and y.
\begin{align*}
  centroid_z &= \frac{1}{V}\iiint_{E}z\diff{V} \\
  &= \frac{3}{2\pi-\sqrt{2}\pi}\int_{0}^{\frac{\pi}{4}}\int_{0}^{2\pi}
    \int_{0}^{1}\rho\cos\phi\rho^2\sin\phi\diff{\rho}\diff{\theta}\diff{\phi} \\
  &= \frac{3}{2\pi-\sqrt{2}\pi}\int_{0}^{\frac{\pi}{4}}\cos\phi\sin\phi
    \int_{0}^{2\pi}\bigg[\frac{\rho^4}{4}\bigg]_{0}^{1}
    \diff{\theta}\diff{\phi} \\
  &= \frac{3}{8(2\pi-\sqrt{2}\pi)}\int_{0}^{\frac{\pi}{4}}\cos\phi\sin\phi
    \int_{0}^{2\pi}\diff{\theta}\diff{\phi} \\
  &= \frac{6}{8(2-\sqrt{2})}\int_{0}^{\frac{\pi}{4}}\cos\phi\sin\phi
    \diff{\phi} \\
  &= \frac{6}{8(2-\sqrt{2})}\int_{0}^{\frac{\pi}{4}}\frac{\sin2\phi}{2}
    \diff{\phi} \\
  &= \frac{3}{16(2-\sqrt{2})}
    \bigg[-\cos2\phi\bigg]_{0}^{\frac{\pi}{4}} \\
  &= \frac{3}{16(2-\sqrt{2})}\bigg(-\cos\frac{\pi}{2}+\cos(0)\bigg) \\
  &= \frac{3}{16(2-\sqrt{2})}(2) \\
  &= \frac{3}{8(2-\sqrt{2})} \\
\end{align*}
Centroid: \( (0,0,\frac{3}{16-8\sqrt{2}}) \)

\begin{center}
  If you have any questions, comments, or concerns, please contact me at
  alvin@omgimanerd.tech
\end{center}

\end{document}
