\documentclass{math}

\title{Multivariable and Vector Calculus}
\author{Alvin Lin}
\date{August 2017 - December 2017}

\begin{document}

\maketitle

\section*{Cylindrical Coordinates}
\[ (x,y,z)\to(r,\theta,z) \]
Cylindrical coordinates describe a point by using polar coordinates and an
elevation.
\begin{align*}
  r &= \sqrt{x^2+y^2} \\
  \theta &= \arctan(\frac{y}{x}) \\
  z &= z
\end{align*}

\subsection*{Theorem}
\[ \iiint_{V}f(x,y,z)\diff{V} =
  \iiint_{C}f(r\cos\theta,r\sin\theta,z)r\diff{z}\diff{\theta}\diff{r} \]

\section*{Spherical Coordinates}
The distance between the origin and a point \( P(x,y,z) \) is \( \rho \). We
can project this point onto the xy-plane to point \( P'(x,y,0) \). We can
compute the polar angle \( \theta \) between \( P' \) and the x-axis. If we
fix the distance \( \rho \) and let all other variables free, we get a sphere.
If we fix \( \theta \) and \( \rho \), the resulting set of points yields a
semicircle. If we take the angle \( \phi \) between \( P \) and the y-axis,
we get a system of coordinates \( (\rho,\theta,\phi) \) described by:
\begin{align*}
  \rho &= \sqrt{x^2+y^2+z^2} \\
  \theta &= \arctan(\frac{y}{x}) \\
  \phi &= \arccos\left(\frac{z}{\sqrt{x^2+y^2+z^2}}\right) =
    \arccos\left(\frac{z}{\rho}\right)
\end{align*}
Converting the other way:
\begin{align*}
  x &= \rho\sin\phi\cos\theta \\
  y &= \rho\sin\phi\sin\theta \\
  z &= \rho\cos\phi
\end{align*}
\[ \ddiff{(x,y,z)}{(\rho,\theta,\phi)} = \rho^2\sin\phi \]

\subsubsection*{Example}
\begin{align*}
  (x,y,z) &\to (\rho,\theta,\phi) \\
  P(2,2,3) &\equiv\left(
    \sqrt{17},\frac{\pi}{4},\arccos\left(\frac{3}{\sqrt{17}}\right)\right)
\end{align*}
\( \rho = 1 \): Sphere \\
\( \phi = \frac{\pi}{4} \): Cone \\
\( \phi = \frac{\pi}{2} \): xy-plane \\
\( \phi = \frac{3\pi}{4} \): Cone \\
\( \phi = \pi \): Negative z-axis \\
\( \phi = 0 \): Positive z-axis

\subsection*{Theorem}
\begin{align*}
  \iiint_{V}f(x,y,z)\diff{V} &= \iiint_{C}f(r\cos\theta,r\sin\theta,z)r
    \diff{r}\diff{\theta}\diff{z} \\
  &= \iiint_{S}f(\rho\sin\phi\cos\theta,\rho\sin\phi\sin\theta,\rho\cos\phi)
    \rho^2\sin\phi\diff{\rho}\diff{\phi}\diff{\theta} \\
\end{align*}

\subsection*{Average Value of a Function}
\[ A_vf = \frac{\iint_{D}f\diff{A}}{\iint_{D}1\diff{A}} =
  \frac{\iiint_{V}f\diff{V}}{\iiint_{V}1\diff{V}} \]

\subsection*{Center of Mass}
\[ \bar{x} = \frac{M_y}{M} =
  \frac{\iint_{D}f\cdot x\diff{A}}{\iint_{D}f\diff{A}} \]

\subsection*{Moment of Inertia}
\[ J_y = \iint_{D}f\cdot x^2\diff{A} \]

\subsubsection*{Example}
Find the center of mass of the solid bounded by the cone \( r = 1, h = 1 \),
\( f(x,y) = \sqrt{x^2+y^2} \).
\begin{align*}
  \bar{x} &= \bar{y} = 0 \\
  \bar{z} &= \frac{M_{xy}}{mass} \\
  &= \frac{\iiint_{V}\sqrt{x^2+y^2}
    z\diff{V}}{\iiint_{V}\sqrt{x^2+y^2}\diff{V}} \\
  &= \frac{
    \int_{0}^{2\pi}\int_{0}^{1}\int_{r}^{1}r^2z\diff{z}\diff{r}\diff{\theta}}
    {\int_{0}^{2\pi}\int_{0}^{1}\int_{r}^{1}r^2\diff{z}\diff{r}\diff{\theta}} \\
  & \text{Note that this can also be represented by spherical coordinates:} \\
  &= \frac{\int_{0}^{2\pi}\int_{0}^{\frac{\pi}{4}}\int_{0}^{\sec\phi}
    \rho^4\sin^3\phi\cos\phi\diff{\rho}\diff{\phi}\diff{\omega}}
    {\int_{0}^{2\pi}\int_{0}^{\frac{\pi}{4}}\int_{0}^{\sec\phi}
    \rho^3\sin^3\phi\diff{\rho}\diff{\phi}\diff{\theta}}
\end{align*}

\subsubsection*{Example}
Find the volume under \( z = xy^2 \) bounded by \( z = 0 \) and the triangle
(1,0), (2,0), (4,0).
\begin{align*}
  V &= \iiint_{V}1\diff{V} \\
  &= \int_{0}^{1}\int_{1+y}^{4-2y}\int_{0}^{x^2y}1\diff{z}\diff{x}\diff{y}
\end{align*}

\begin{center}
  You can find all my notes at \url{http://omgimanerd.tech/notes}. If you have
  any questions, comments, or concerns, please contact me at
  alvin@omgimanerd.tech
\end{center}

\end{document}
