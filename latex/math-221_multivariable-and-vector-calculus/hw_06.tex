\documentclass{math}

\geometry{letterpaper, margin=0.5in}

\title{Multivariable and Vector Calculus: Homework 6}
\author{Alvin Lin}
\date{August 2016 - December 2016}

\begin{document}

\maketitle

\section*{Section 14.4}

\subsubsection*{Exercise 5}
Find an equation of the tangent plane to the given surface at the specified
point.
\[ z = x\sin(x+y) \quad (-1,1,0) \]
\begin{align*}
  z-z_0 &= f_x(x_0,y_0)(x-x_0)+f_y(x_0,y_0)(y-y_0) \\
  \pdiff{z}{x} &= x\cos(x+y)+\sin(x+y) \\
  \pdiff{z}{x}(-1,1) &= -1+0 = 0 \\
  \pdiff{z}{y} &= x\cos(x+y) \\
  \pdiff{z}{y}(-1,1) &= -1 \\
  z &= -1(x+1)-1(y-1) \\
  -x-y &= z \\
  x+y+z &= 0
\end{align*}

\subsubsection*{Exercise 33}
The length and width of a rectangle are measured as 30cm and 24cm, respectively,
with an error in measurement of at most 0.1cm in each. Use differentials to
estimate the maximum error in the calculated area of the rectangle.
\begin{align*}
  l &= 30 \\
  w &= 24 \\
  \diff{l} &= 0.1 \\
  \diff{w} &= 0.1 \\
  A &= lw \\
  \diff{A} &= \pdiff{l}{A}\diff{l}+\pdiff{w}{A}\diff{w} \\
  &= w~\diff{l}+l~\diff{w} \\
  &= 30(0.1)+24(0.1) \\
  &= 5.4cm^2
\end{align*}

\subsubsection*{Exercise 35}
Use differentials to estimate the amount of tin in a closed tin can with
diameter 8cm and height 12cm if the tin is 0.04cm thick.
\begin{align*}
  d &= 8 \\
  r &= 4 \\
  h &= 12 \\
  \diff{d} &= \diff{r} = 0.04 \\
  \diff{h} &= 0.08 \\
  V &= \pi r^2h \\
  \diff{V} &= \pdiff{V}{d}\diff{d}+\pdiff{V}{h}\diff{h} \\
  &= 2\pi rh(0.04)+\pi r^2(0.08) \\
  &= 16.0768cm^2
\end{align*}

\subsubsection*{Exercise 39}
If \( R \) is the total resistance of three resistors, connected in parallel,
with resistances \( R_1,R_2,R_3 \), then:
\[ \frac{1}{R} = \frac{1}{R_1}+\frac{1}{R_2}+\frac{1}{R_3} \]
If the resistances are measured in ohms as \( R_1 = 25\Omega, R_2 = 40\Omega,
R_3 = 50\Omega \), with a possible error of 0.5\% in each case, estimate the
maximum error in the calculated value of \( R \).
\begin{align*}
  R_1 &= 25 \quad \diff{R_1} = 0.125 \\
  R_2 &= 40 \quad \diff{R_2} = 0.2 \\
  R_3 &= 50 \quad \diff{R_3} = 0.25 \\
  \frac{1}{R} &= \frac{1}{25}+\frac{1}{40}+\frac{1}{50} \\
  R &= 11.764 \\
  \frac{1}{R} &= \frac{1}{R_1}+\frac{1}{R_2}+\frac{1}{R_3} \\
  \diff{R} &= \pdiff{R}{R_1}\diff{R_1}+\pdiff{R}{R_2}\diff{R_2}+
    \pdiff{R}{R_3}\diff{R_3} \\
  \pdiff{R}{R_1} &= \frac{R^2}{(R_1)^2} \\
  \pdiff{R}{R_2} &= \frac{R^2}{(R_2)^2} \\
  \pdiff{R}{R_3} &= \frac{R^2}{(R_3)^2} \\
  \diff{R} &= \frac{11.764^2}{25^2}(0.125)+\frac{11.764^2}{40^2}(0.2)+
    \frac{11.764^2}{50^2}(0.25) \\
  &\approx 0.0588\Omega
\end{align*}

\section*{Section 14.6}

\subsubsection*{Exercise 5}
Find the directional derivative of \( f \) at the given point in the direction
indicated by the angle \( \theta \).
\[ f(x,y) = y\cos(xy) \quad (0,1) \quad \theta = \frac{\pi}{4} \]
\begin{align*}
  \gradientd{F} &= \langle F_x,F_y\rangle \\
  &= \left\langle -y^2\sin(xy),-xy\sin(xy)+\cos(xy)\right\rangle \\
  \gradientd{F}(0,1) &= \langle-1\sin(0),0+\cos(0)\rangle \\
  &= \langle0,1\rangle \\
  D_{\vec{u}}f &= \gradientd{F}(0,1)\cdot\langle\cos\theta,\sin\theta\rangle \\
  &= \langle0,1\rangle\cdot
    \langle\frac{\sqrt{2}}{2},\frac{\sqrt{2}}{2}\rangle \\
  &= 0+\frac{\sqrt{2}}{2} = \frac{\sqrt{2}}{2}
\end{align*}

\subsubsection*{Exercise 13}
Find the directional derivative of the function at the given point in the
direction of the vector \( \vec{v} \).
\[ g(s,t) = s\sqrt{t} \quad (2,4) \quad \vec{v} = 2\i-\j \]
\begin{align*}
  \gradientd{g} &= \langle\pdiff{g}{s},\pdiff{g}{t}\rangle \\
  &= \langle\sqrt{t},\frac{s}{2\sqrt{t}}\rangle \\
  \gradientd{g}(2,4) &= \langle\sqrt{4},\frac{2}{2\sqrt{4}}\rangle \\
  &= \langle2,\frac{1}{2}\rangle \\
  D_{\vec{u}}g &= \gradientd{g}\cdot\vec{u} \\
  \vec{u} &= \langle\frac{2}{\sqrt{5}},\frac{-1}{\sqrt{5}}\rangle \\
  &= \langle2,\frac{1}{2}\rangle\cdot
    \langle\frac{2}{\sqrt{5}},\frac{-1}{\sqrt{5}}\rangle \\
  &= \frac{4}{\sqrt{5}}-\frac{1}{2\sqrt{5}} \\
  &= \frac{7}{2\sqrt{5}}
\end{align*}

\subsubsection*{Exercise 21}
Find the maximum rate of change of \( f \) at the given point and the direction
in which it occurs.
\[ f(x,y) = 4y\sqrt{x} \quad (4,1) \]
\begin{align*}
  \gradientd{f} &= \langle\pdiff{f}{x},\pdiff{f}{y}\rangle \\
  &= \langle\frac{4y}{2\sqrt{x}},4\sqrt{x}\rangle \\
  &= \langle\frac{2y}{\sqrt{x}},4\sqrt{x}\rangle \\
  \gradientd{f}(4,1) &= \langle\frac{2(1)}{\sqrt{4}},4\sqrt{4}\rangle \\
  &= \langle1,8\rangle \\
  |\gradientd{f}(4,1)| &= \sqrt{65} \\
  \text{unit vector } &= \langle\frac{1}{\sqrt{65}},\frac{8}{\sqrt{65}}\rangle
\end{align*}

\subsubsection*{Exercise 29}
Find all points at which the direction of fastest change of the function
\( f(x,y) = x^2+y^2-2x-4y \) is \( \i+\j \).
\begin{align*}
  \gradientd{f} &= \langle2x-2,2y-4\rangle \\
  \gradientd{f} &= c(\i+\j) \\
  2x-2 &= c = 2y-4 \\
  x-1 &= y-2 \\
  y &= x+1
\end{align*}
Any point on the line \( y = x+1 \).

\subsubsection*{Exercise 43}
Find equations of the tangent plane and the normal line to the given surface
at the specified point.
\[ xy^2z^3 = 8 \quad (2,2,1) \]
\begin{align*}
  F(x,y,z) &= xy^2z^3-8 = 0 \\
  F_x &= y^2z^3 \\
  F_x(2,2,1) &= 4 \\
  F_y &= 2yxz^3 \\
  F_y(2,2,1) &= 8 \\
  F_z &= 3z^2xy^2 \\
  F_z(2,2,1) &= 24 \\\
  \Pi: 0 &= F_x(x,y,z)(x-x_0)+F_y(x,y,z)(y-y_0)+F_z(x,y,z)(z-z_0) \\
  \Pi: 0 &= 4(x-2)+8(y-2)+24(z-1) \\
  \Pi: 0 &= 4x+8y+24z-48 \\
  \Pi: 0 &= x+2y+6z-12 \\
  l &= \begin{cases}
    x &= 4t+2 \\
    y &= 8t+2 \\
    z &= 24t+1
  \end{cases}
\end{align*}

\begin{center}
  If you have any questions, comments, or concerns, please contact me at
  alvin@omgimanerd.tech
\end{center}

\end{document}
