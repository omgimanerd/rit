\documentclass[letterpaper, 12pt]{math}

\usepackage{amsmath}
\usepackage{amssymb}
\usepackage{geometry}

\geometry{letterpaper, margin=1in}

\title{Multivariable and Vector Calculus: Homework 1}
\author{Alvin Lin}
\date{August 2016 - December 2016}

\begin{document}

\maketitle

\section*{Page 796}

\subsection*{Exercise 3}
Which of the points A(-4,0,-1), B(-3,1,-5), and C(2,4,6) is closest to the
yz-plane? Which point lies in the xz-plane?
\[ A \to 4 \quad B -> 3 \quad C -> 2 \]
Point C is closest to the yz-plane. The y-component of A is 0, so it lies in the
xz-plane.

\subsection*{Exercise 9}
Find the lengths of the sides of the triangle PQR. Is it a right triangle?
Is it an isosceles triangle?
\[ P(3,-2,-3) \quad Q(7,0,1) \quad R(1,2,1) \]
\begin{align*}
  \overline{PQ} &= \sqrt{16+4+16} = \sqrt{36} = 6 \\
  \overline{QR} &= \sqrt{36+4+0} = \sqrt{40} \\
  \overline{PR} &= \sqrt{4+16+16} = \sqrt{36} = 6
\end{align*}
\( \triangle \) PQR is an isosceles triangle.

\subsection*{Exercise 13}
Find the intersection of the sphere with center (-3,2,5) and radius 4.
\[ (x+3)^2+(y-2)^2+(z-5)^2 = 16 \]
This sphere intersects the yz-plane at \( x = 0 \):
\[ (0+3)^2+(y-2)^2+(z-5)^2 = 16 \]
\[ (y-2)^2+(z-5)^2 = 7 \]
The intersection is a circle centered at (0,2,5) with radius \( \sqrt{7} \).

\subsection*{Exercise 17}
Show that the equation represents a sphere, and find its center and radius:
\begin{align*}
  x^2+y^2+z^2-2x-4y+8z &= 15 \\
  x^2-2x+y^2-4y+z^2+8z &= 15 \\
  x^2-2x+1+y^2-4y+4+z^2+8z+16 &= 15+1+4+16 \\
  (x-1)^2+(y-2)^2+(z+4)^2 &= 36
\end{align*}
This sphere is centered at (1,2,-4) and has a radius of 6.

\subsection*{Exercise 33}
Describe in words the region of \( \R^3 \) represented by the equations or
inequality.
\[ x^2+y^2+z^2 = 4 \]
This describes a hollow sphere centered at the origin with a radius 2 (excluding
the inside of the sphere).

\subsection*{Exercise 37}
Describe in words the region of \( \R^3 \) represented by the equations or
inequality.
\[ x^2+z^2\le 9 \]
This describes a solid cylinder centered at the origin with radius 9 extending
infinitely in the y-direction.

\subsection*{Exercise 41}
Write inequalities to describe the region: The region consisting of all points
between (but not on) the spheres of radius \( r \) and \( R \) centered at the
origin, where \( r < R \).
\[ r < x^2+y^2+z^2 < R \]

\section*{Page 805}

\subsection*{Exercise 7}
In the figure, the tip of c and the tail of d are both the midpoint
of QR. Express c and d in terms of a and b.
\[ d = \frac{b-a}{2} \]
\[ b = \frac{a+b}{2} \]

\subsection*{Exercise 8}
If the vectors in the figure satisfy \( |u| = |v|= 1 \) and \( u+v+w = 0 \),
what is \( |w| \)?
\begin{align*}
  |w| &= \sqrt{(w_x)^2+(w_y)^2} \\
  &= \sqrt{(w_x)^2+0} \\
  &= w_x \\
  &= u_x+v_x \\
  &= 1\cos45+1\cos45 \\
  &= \frac{\sqrt{2}}{2}+\frac{\sqrt{2}}{2} \\
  &= \sqrt{2}
\end{align*}

\subsection*{Exercise 27}
What is the angle between the given vector and the positive direction of the
x-axis?
\[ \hat{i}+\sqrt{3}\hat{j} \]
\begin{align*}
  \tan\theta &= \frac{y}{x} \\
  \theta &= \arctan\frac{y}{x} \\
  &= \arctan\frac{\sqrt{3}}{1} \\
  &= 60^{\circ}
\end{align*}

\subsection*{Exercise 33}
Find the magnitude of the resultant force and the angle it makes with the
positive x-axis.
\begin{align*}
  \vec{r} &= \langle200\cos-300,200\sin60+0\rangle \\
  &= \langle-200,\frac{200\sqrt{3}}{2}\rangle \\
  \theta &= \arctan|\frac{y}{x}| \\
  &= \frac{\frac{200\sqrt{3}}{2}}{200} \\
  &= \frac{\sqrt{3}}{2} \\
  &\approx 40.9^{\circ}
\end{align*}
This is the angle with the negative x-axis. The angle with the positive x-axis
is:
\[ 180-\theta = 139.1 \]

\subsection*{Exercise 35}
A woman walks due west on the deck of a ship at 3 mi/h. The ship is moving
north at a speed of 22 mi/h. Find the speed and direction of the woman relative
to the surface of the water.
\begin{align*}
  \vec{v} &= \langle-3,22\rangle \\
  |\vec{v}| &= \sqrt{9+484} = \sqrt{493} \approx 22.2 \\
  \theta &= \arctan\frac{3}{22} \approx 7.765
\end{align*}
She is traveling at approximately 22.2 mi/h at \( 7.765^{\circ} \) west of
north or \( 82.235^{\circ} \) north of west.

\subsection*{Exercise 41}
Find the unit vectors that are parallel to the tangent line to the
parabola \( y = x^2 \) at the point (2,4).
\begin{align*}
  y' &= 2x \\
  y'(2) &= 2(2) = 4 \\
  \vec{u} &= \langle1,4\rangle \\
  \frac{\vec{u}}{|\vec{u}|} &= \frac{\langle1,4\rangle}{\sqrt{1+16}} \\
  &= \left\langle\frac{1}{\sqrt{17}},\frac{4}{\sqrt{17}}\right\rangle
\end{align*}

\subsection*{Exercise 43}

\begin{center}
  If you have any questions, comments, or concerns, please contact me at
  alvin@omgimanerd.tech
\end{center}

\end{document}
