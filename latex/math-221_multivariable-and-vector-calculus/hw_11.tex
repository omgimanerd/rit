\documentclass{math}

\geometry{letterpaper, margin=0.5in}

\title{Multivariable and Vector Calculus: Homework 11}
\author{Alvin Lin}
\date{August 2016 - December 2016}

\begin{document}

\maketitle

\section*{Section 16.3}

\subsubsection*{Exercise 3}
Determine whether or not \( F \) is a conservative vector field. If it is, find
a function \( f \) such that \( F = \triangledown f \).
\[ F(x,y) = (xy+y^2)\i+(x^2+2y)\j \]
\begin{align*}
  F(x,y) &= P\i+Q\j \\
  \pdiff{P}{y} &= x+2y \\
  \pdiff{Q}{x} &= 2x \\
  \pdiff{P}{y} &\ne \pdiff{Q}{x}
\end{align*}

\subsubsection*{Exercise 5}
Determine whether or not \( F \) is a conservative vector field. If it is, find
a function \( f \) such that \( F = \triangledown f \).
\[ F(x,y) = y^2\e^{xy}\i+(1+xy)\e^{xy}\j \]
\begin{align*}
  F(x,y) &= P\i+Q\j \\
  \pdiff{P}{y} &= 2y\e^{xy}+xy^2\e^{xy} \\
  \pdiff{Q}{x} &= y\e^{xy}+(1+xy)y\e^{xy} \\
  &= 2y\e^{xy}+xy^2\e^{xy} \\
  \int y^2\e^{xy} &= y\e^{xy}+h(y) \\
  (1+xy)\e^{xy}+h'(y) &= (1+xy)\e^{xy} \\
  h'(y) &= 0 \\
  h(y) &= c \\
  f(x,y) &= y\e^{xy}+c
\end{align*}

\subsubsection*{Exercise 13}
Find a function \( f \) such that \( F = \triangledown f \) and use it to
evaluate \( \int_{C}F\cdot\diff{r} \) along the given curve \( C \).
\[ F(x,y) = x^2y^3\i+x^3y^2\j \quad
  C:\overrightarrow{r(t)} = \langle t^3-2t,t^3+2t\rangle \quad 0\le t\le1 \]
\begin{align*}
  f(x,y) &= \frac{1}{3}x^3y^3+c \\
  \overrightarrow{r(0)} &= \langle0,0\rangle \\
  \overrightarrow{r(1)} &= \langle-1,3\rangle \\
  \int_{C}F\diff{r} &= f(-1,3)-f(0,0) \\
  &= \frac{1}{3}(-1)^3(3)^3-0 \\
  &= -9
\end{align*}

\subsubsection*{Exercise 17}
Find a function \( f \) such that \( F = \triangledown f \) and use it to
evaluate \( \int_{C}F\cdot\diff{r} \) along the given curve \( C \).
\[ F(x,y,z) = yz\e^{xz}\i+\e^{xz}\j+xy\e^{xz}\k \]
\[ C: \vec{r(t)} = (t^2+1)\i+(t^2-1)\j+(t^2-2t)\k \quad 0\le t\le2 \]
\begin{align*}
  f_x &= yz\e^{xz} \\
  f_y &= \e^{xz} \\
  f_z &= xy\e^{xz} \\
  f &= \int f_x\diff{x} \\
  &= \int yz\e^{xz}\diff{x} \\
  &= yz\frac{\e^{xz}}{z}+h(y,z) \\
  f_y &= yz\e^{xz} = yz\e^{xz}+h_y(y,z) \\
  h_y(y,z) &= 0 \\
  h(y,z) &= g(z) \\
  f_z &= xy\e^{xz}+g'(z) = xy\e^{xz} \\
  g'(z) &= 0 \\
  g(z) &= h(y,z) = c \\
  f(x,y) &= y\e^{xz}+c \\
  \overrightarrow{r(0)} &= \langle1,-1,0\rangle \\
  \overrightarrow{r(2)} &= \langle5,3,0\rangle \\
  \int_{C}F\cdot\diff{r} &= f(5,3,0)-f(1,-1,0) \\
  &= 3\e^{(5)(0)}-(-1)\e^{(1)(0)} \\
  &= 3+1 \\
  &= 4
\end{align*}

\subsubsection*{Exercise 21}
Suppose you're asked to determine the curve that requires the least work for a
force field \( F \) to move a particle from one point to another point. You
decide to check first whether \( F \) is conservative, and indeed turns out that
it is. How would you reply to the request? \par
Because \( F \) is conservative, the path taken to \( F \) does not matter since
\( \int_{C}F\cdot\diff{r} \) is \( f(\overrightarrow{r(b)})-
f(\overrightarrow{r(a)}) \) and only the endpoints \( a \) and \( b \) of the
curve matter.

\subsubsection*{Exercise 23}
Find the work done by the force field \( F \) in moving an object from \( P \)
to \( Q \).
\[ F(x,y) = x^3\i+y^3\j \quad P(1,0),Q(2,2) \]
\begin{align*}
  f_x &= x^3 \\
  f_y &= y^3 \\
  f(x,y) &= \int f_x\diff{x} \\
  &= \int x^3\diff{x} \\
  &= \frac{1}{4}x^4+h(y) \\
  f_y(x,y) &= h'(y) = y^3 \\
  h(y) &= \int y^3\diff{y} \\
  &= \frac{1}{4}y^4+c \\
  f(x,y) &= \frac{1}{4}x^4+(\frac{1}{4}y^4+c) \\
  W &= \int f\cdot\diff{r} \\
  &= f(2,2)-f(1,1) \\
  &= \frac{1}{4}(32)-\frac{1}{4}(1) \\
  &= \frac{31}{4}
\end{align*}

\subsubsection*{Exercise 31}
Determine whether or not the given set is open, connected, and/or simply
connected.
\[ \{(x,y)\mid0<y<3\} \]
This defines all points within the square from (0,0) to (3,3) but excluding
the boundary of the square. This region is open since it does not include those
boundary points. The set is connected since all points in the set can be joined
by a path entirely composed of points into the set.

\subsubsection*{Exercise 33}
Determine whether or not the given set is open, connected, and/or simply
connected.
\[ \{(x,y)\mid1\le x^2+y^2\le 4\quad y\ge0\} \]
This defines the area between the semicircles above the x-axis with radius 1 and
2. This set is not an open set because it contains its boundary points. It is
a connected set since all points can be joined with a path inside the set.

\section*{Section 16.6}

\subsubsection*{Exercise 2}
Determine whether the points \( P \) and \( Q \) lie on the given surface.
\[ \vec{r}(u,v) = \langle1+u-v,u+v^2,u^2-v^2\rangle \quad P(1,2,1),Q(2,3,3) \]
\begin{align*}
  1+u-v &= 1 \\
  u+v^2 &= 2 \\
  u^2-v^2 &= 1 \\
  1+u-v-(u+v^2) = 1-2 \\
  v^2+v-2 &= 0 \\
  (v+2)(v-1) &= 0 \\
  v &= -2 \quad v = 1 \\
  u &= -2 \quad u = 1 \\
  u^2-v^2 &\ne 1
\end{align*}
The point \( P \) does not lie on the surface.
\begin{align*}
  1+u-v &= 2 \\
  u+v^2 &= 3 \\
  u^2-v^2 &= 3 \\
  1+u-v-(u+v^2) &= 2-3 \\
  v^2+v-2 &= 0 \\
  v &= -2 \quad v = 1 \\
  u &= -1 \quad u = 2 \\
  u^2-v^2 &= 1
\end{align*}
The point \( Q \) lies on the surface.

\subsubsection*{Exercise 19}
Find a parametric representation for the plane through the origin that contains
the vectors \( \i-\j \) and \( \j-\k \).
\begin{align*}
  \overrightarrow{r(u,v)} &= \vec{p}+(\i-\j)u+(\j-\k)v \\
  &= u\i-u\j+v\j-v\k \\
  &= u\i+(v-u)\j-v\k \\
  x &= u \quad y = v-u \quad z = -v \\
\end{align*}

\subsubsection*{Exercise 33}
Find an equation of the tangent plane to the given parametric surface at the
specified point.
\[ x = u+v \quad y = 3u^2 \quad z = u-v \quad (2,3,0) \]
\begin{align*}
  u+v &= 2 \\
  3u^2 &= 3 \\
  u-v &= 0 \\
  u &= \pm1 \\
  v &= 1 \\
  u &= v = 1 \\
  \overrightarrow{r(u,v)} &= (u+v)\i+3u^2\j+(u-v)\k \\
  \vec{r_u} &= \i+6u\j+\k \\
  \vec{r_v} &= \i-\k \\
  \vec{n} &= \vec{r_u}\times\vec{r_v} \\
  &= \begin{vmatrix}
    \i & \j & \k \\
    1 & 6u & 1 \\
    1 & 0 & -1
  \end{vmatrix} \\
  &= -6u\i+2\j+-6u\k \\
  0 &= (\langle x,y,z\rangle-P)\cdot\vec{n} \\
  &= (x-2)(-6(1))+(y-3)(2)+(z-0)(-6(1)) \\
  &= -6x+12+2y-6-6z \\
  6x-2y+6z &= 6 \\
  3x-y+3z &= 3
\end{align*}

\subsubsection*{Exercise 39}
Find the area of the part of the plane \( 3x+2y+z = 6 \) that lies in the first
octant.
\begin{align*}
  z &= 6-3x-2y \\
  \pdiff{z}{x} &= -3 \\
  \pdiff{z}{y} &= -2
\end{align*}
\begin{align*}
  A(S) &= \iint_{D}\sqrt{1+(\pdiff{z}{x})^2+(\pdiff{z}{y})^2}\diff{A} \\
  &= \iint_{D}\sqrt{1+9+4}\diff{A} \\
  &= \sqrt{14}\iint_{D}\diff{A} \\
  &= \sqrt{14}\int_{0}^{2}\int_{0}^{3-\frac{3x}{2}}\diff{y}\diff{x} \\
  &= \sqrt{14}\int_{0}^{2}3-\frac{3x}{2}\diff{x} \\
  &= \sqrt{14}\left[3x-\frac{3x^2}{4}\right]_{0}^{2} \\
  &= 3\sqrt{14}
\end{align*}

\subsubsection*{Exercise 45}
Find the area of the part of the surface \( z = xy \) that lies within the
cylinder \( x^2+y^2 = 1 \).
\begin{align*}
  z &= xy \\
  \pdiff{z}{x} &= y \\
  \pdiff{z}{y} &= x \\
  A(S) &= \iint_{D}\sqrt{1+y^2+x^2}\diff{A} \\
  &= \int_{-1}^{1}\int_{-\sqrt{1-x^2}}^{\sqrt{1-x^2}}\sqrt{1+y^2+x^2}
    \diff{y}\diff{x} \\
  &= \int_{0}^{2\pi}\int_{0}^{1}\sqrt{1+r^2}r\diff{r}\diff{\theta} \\
  &= \int_{0}^{2\pi}\left[\frac{1}{3}(1+r^2)^{\frac{3}{2}}\right]_{0}^{1}
    \diff{\theta} \\
  &= \frac{1}{3}\int_{0}^{2\pi}2\sqrt{2}-1\diff{\theta} \\
  &= \frac{2\sqrt{2}-1}{3}\bigg[\theta\bigg]_{0}^{2\pi} \\
  &= \frac{2\pi(2\sqrt{2}-1)}{3}
\end{align*}

\section*{Section 16.7}

\subsubsection*{Exercise 5}
Evaluate the surface integral.
\[ \iint_{S}(x+y+z)\diff{S} \]
\( S \) is the parallelogram with parametric equations \( x = u+v, y = u-v,
z = 1+2u+v, 0\le u\le 2, 0\le v\le 1 \).
\begin{align*}
  \overrightarrow{r(u,v)} &= (u+v)\i+(u-v)\j+(1+2u+v)\k \\
  \vec{r_u} &= \i+\j+2\k \\
  \vec{r_v} &= \i-\j+\k \\
  \iint_{S}(x+y+z)\diff{S} &=
    \iint_{D}f(\overrightarrow{r(u,v)})|\vec{r_u}\times\vec{r_v}|\diff{A} \\
  &= \int_{0}^{2}\int_{0}^{1}(4u+1+v)|\vec{r_u}\times\vec{r_v}|
    \diff{v}\diff{u} \\
  &= \sqrt{14}\int_{0}^{2}\bigg[4uv+v+\frac{v^2}{2}\bigg]_{0}^{1}\diff{u} \\
  &= \sqrt{14}\int_{0}^{2}4u+1+\frac{1}{2}\diff{u} \\
  &= \sqrt{14}\bigg[\frac{4u^2}{2}+u+\frac{u}{2}\bigg]_{0}^{1} \\
  &= \sqrt{14}(8+3-(0+0)) \\
  &= 11\sqrt{14}
\end{align*}

\subsubsection*{Exercise 11}
Evaluate the surface integral.
\[ \iint_{S}x\diff{S} \]
\( S \) is the triangular region with vertices \( (1,0,0),(0,-2,0),(0,0,4) \).
\begin{align*}
  \frac{x}{1}+\frac{y}{-2}+\frac{z}{4} &= 1 \\
  4x-2y+z &= 4 \\
  z &= 4-4x+2y \\
  \pdiff{z}{x} &= -4 \\
  \pdiff{z}{y} &= 2
\end{align*}
\begin{align*}
  \iint_{S}x\diff{S} &=
    \iint_{D}f(x,y,z)\sqrt{1+(\pdiff{z}{x})^2+(\pdiff{z}{y})^2}\diff{A} \\
  &= \iint_{D}\sqrt{1+16+4}x\diff{A} \\
  &= \sqrt{21}\int_{0}^{1}\int_{2x-2}^{0}x\diff{y}\diff{x} \\
  &= \sqrt{21}\int_{0}^{1}\bigg[xy\bigg]_{2x-2}^{0}\diff{x} \\
  &= \sqrt{21}\int_{0}^{1}-2x^2+2x\diff{x} \\
  &= \sqrt{21}\bigg[\frac{2x^2}{2}-\frac{2x^3}{3}\bigg]_{0}^{1} \\
  &= \sqrt{21}\bigg(\frac{2}{2}-\frac{2}{3}\bigg) \\
  &= \frac{\sqrt{21}}{3}
\end{align*}

\subsubsection*{Exercise 15}
Evaluate the surface integral.
\[ \iint_{S}x\diff{S} \]
\( S \) is the surface \( y = x^2+4z,0\le x\le1,0\le z\le 1 \).
\begin{align*}
  \pdiff{y}{x} &= 2x \\
  \pdiff{y}{z} &= 4 \\
  \iint_{S}x\diff{S} &= \iint_{D}x\sqrt{1+16+4x^2}\diff{x}\diff{z} \\
  &= \int_{0}^{1}\int_{0}^{1}x\sqrt{17+4x^2}\diff{x}\diff{z} \\
  &= \int_{0}^{1}\bigg[\frac{1}{12}(4x^2+17)^{\frac{3}{2}}\bigg]_{0}^{1}
    \diff{z} \\
  &= \int_{0}^{1}\frac{1}{12}(21^{\frac{3}{2}}-17^{\frac{3}{2}})\diff{z} \\
  &= \frac{21^{\frac{3}{2}}-17^{\frac{3}{2}}}{12}
\end{align*}

\subsubsection*{Exercise 17}
Evaluate the surface integral.
\[ \iint_{S}(x^2z+y^2z)\diff{S} \]
\( S \) is the hemisphere \( x^2+y^2+z^2 = 4, z\ge0 \).
\begin{align*}
  x^2+y^2+z^2 = 4 \equiv \rho = 2 \\
  x &= 2\sin\phi\cos\theta \\
  y &= 2\sin\phi\sin\theta \\
  z &= 2\cos\phi \\
  \vec{r_{\phi}} &= 2\cos\phi\cos\theta\i+2\cos\phi\cos\theta\j-2\sin\phi\k \\
  \vec{r_{\phi}} &= 2\sin\phi\sin\theta\i+2\sin\phi\cos\theta\j+0\k \\
  |\vec{r_{\phi}}\times\vec{r_{\theta}}| &= \left|\begin{vmatrix}
    \i & \j & \k \\
    2\cos\phi\cos\theta & 2\cos\phi\sin\theta & -2\sin\phi \\
    -2\sin\phi\sin\theta & 2\sin\phi\cos\theta & 0 \\
  \end{vmatrix}\right| \\
  &= 4\sin\phi \\
  \iint_{S}(x^2x+y^2z)\diff{S} &= \int_{0}^{\frac{\pi}{2}}\int_{0}^{2\pi}
    (8\sin^2\phi\cos\phi\cos^2\theta+8\sin^2\phi\cos\phi\sin^2\theta)
    (4\sin\phi)\diff{\theta}\diff{\phi} \\
  &= \int_{0}^{\frac{\pi}{2}}\int_{0}^{2\pi}32\sin^3\phi\cos\theta
    \diff{\phi}\diff{\theta} \\
  &= 32\int_{0}^{2\pi}\diff{\theta}
    \int_{0}^{\frac{\pi}{2}}\sin^3\theta\phi\cos\theta\diff{\theta} \\
  &= 32(2\pi)(\frac{1}{4}) \\
  &= 16\pi
\end{align*}

\subsubsection*{Exercise 19}
Evaluate the surface integral.
\[ \iint_{S}xz\diff{S} \]
\( S \) is the boundary of the region enclosed by the cylinder \( y^2+z^2 = 9 \)
and the planes \( x = 0 \) and \( x+y = 5 \).
\begin{align*}
\end{align*}

\subsubsection*{Exercise 23}
Evaluate the surface integral \( \iint_{S}F\cdot\diff{S} \) for the given vector
field \( F \) and the oriented surface \( S \). In other words, find the flux of
\( F \) across \( S \). For closed surfaces, use the positive (outward)
orientation.
\[ F(x,y,z) = xy\i+yz\j+zx\k \]
\( S \) is the part of paraboloid \( z = 4-x^2-y^2 \) that lies above the square
\( 0\le x\le1,0\le y\le1 \) and has upward orientation.

\begin{center}
  If you have any questions, comments, or concerns, please contact me at
  alvin@omgimanerd.tech
\end{center}

\end{document}
