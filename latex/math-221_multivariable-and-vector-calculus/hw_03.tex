\documentclass{math}

\geometry{letterpaper, margin=0.5in}

\title{Multivariable and Vector Calculus: Homework 3}
\author{Alvin Lin}
\date{August 2016 - December 2016}

\begin{document}

\maketitle

\section*{Section 12.6}

\subsubsection*{Exercise 11}
Use traces to sketch and identify the surface.
\[ x = y^2+4z^2 \]
\[\begin{split}
  x &= 0 \\
  0 &= y^2+4z^2
\end{split}\quad
\begin{split}
  x &= 1 \\
  1 &= y^2+4z^2
\end{split}\quad
\begin{split}
  x &= 5 \\
  5 &= y^2+4z^2
\end{split}\quad
\begin{split}
  y &= 0 \\
  x &= 4z^2
\end{split} \]
\\ \\ \\ \\ \\
The surface is a elliptic paraboloid expanding along the positive x-axis.

\subsubsection*{Exercise 13}
Use traces to sketch and identify the surface.
\[ x^2 = 4y^2+z^2 \]
\[\begin{split}
  x &= 0 \\
  0 &= 4y^2+z^2
\end{split}\quad
\begin{split}
  x &= 1 \\
  1 &= 4y^2+z^2
\end{split}\quad
\begin{split}
  x &= 2 \\
  4 &= 4y^2+z^2
\end{split}\quad
\begin{split}
  y &= 0 \\
  x^2 &= 4z^2 \\
  x &= \pm2z
\end{split} \]
\\ \\ \\ \\ \\
This surface is a cone defined by ellipses along the x-axis.

\subsubsection*{Exercise 15}
Use traces to sketch and identify the surface.
\[ 9y^2+4z^2 = x^2+36 \]
\[\begin{split}
  x &= 0 \\
  9y^2+4z^2 &= 36
\end{split}\quad
\begin{split}
  x &= 1 \\
  9y^2+4z^2 &= 37
\end{split}\quad
\begin{split}
  x &= 2 \\
  9y^2+4z^2 &= 40
\end{split}\quad
\begin{split}
  y &= 0 \\
  4z^2-x^2 &= 36
\end{split}\quad
\begin{split}
  y &= 2 \\
  36+4z^2 &= x^2+36 \\
  4z^2-x^2 &= 0
\end{split} \]
\\ \\ \\ \\ \\
This surface is a hyperboloid of one sheet.

\subsubsection*{Exercise 47}
Find an equation for the surface consisting of all points that are
equidistant from the point (-1,0,0) and the plane \( x = 1 \). Identify the
surface.
\[ \text{Plane} \equiv 1x+0y+0z-1 = 0 \]
\begin{align*}
  d_1 &= d_2 \\
  \sqrt{(x+1)^2+y^2+z^2} &= \frac{|1x-1|}{\sqrt{1^2}} \\
  (x+1)^2+y^2+z^2 &= (x-1)^2 \\
  x^2+2x+1+y^2+z^2 &= x^2-2x+1 \\
  y^2+z^2 &= -4x
\end{align*}
The surface is a series of expanding circles along the negative x-axis. The
cross section of the circles is a parabola. The surface is a elliptic
paraboloid.

\section*{Section 13.1}

\subsubsection*{Exercise 31}
At what points does the curve \( \vec{r(t)} = t\i+(2t-t^2)\k \) intersect the
the paraboloid \( z = x^2+y^2 \)?
\begin{align*}
  \overrightarrow{r(t)} &= \begin{cases}
    x &= t \\
    y &= 2t-t^2 \\
    z &= 0
  \end{cases} \\
  z &= x^2+y^2 \\
  0 &= t^2+(2t-t^2)^2 \\
  0 &= t^2+(4t^2-4t^3+t^4) \\
  t^4-4t^3+5t^2 &= 0 \\
  t^2(t-4)(t-1) &= 0 \\
  t &= 4 \quad t = 1 \quad t = 0 \\
  \overrightarrow{r(0)} &= \langle0,0,0\rangle \\
  \overrightarrow{r(1)} &= \langle1,1,0\rangle \\
  \overrightarrow{r(4)} &= \langle4,-8,0\rangle
\end{align*}

\subsubsection*{Exercise 41}
Show that the curve with parametric equations \( x = t^2, y = 1-3t, z = 1+3t^2
\) passes through the points (1,4,0) and (9,-8,28) but not through the point
(4,7,-6).
\[ \begin{split}
  1 &= t^2 \quad t = \pm1 \\
  4 &= 1-3t \quad t = -1 \\
  0 &= 1+t^3 \quad t = -1
\end{split}\quad\begin{split}
  9 &= t^2 \quad t = \pm3 \\
  -8 &= 1-3t \quad t = 3 \\
  28 &= 1+t^3 \quad t = 3
\end{split}\quad\begin{split}
  4 &= t^2 \quad t = \pm2 \\
  7 &= 1-3t \quad t = -2 \\
  -6 &= 1+t^3 \quad t = \sqrt[3]{7}
\end{split} \]
The equation does not solve to the same \( t \) for the point (4,7,-6).

\subsubsection*{Exercise 43}
Find a vector function that represents the curve of intersection of the cone
\( z = \sqrt{x^2+y^2} \) and the plane \( z = 1+y \).
\begin{align*}
  \sqrt{x^2+y^2} &= 1+y \\
  x^2+y^2 &= (1+y)^2 \\
  x^2+y^2 &= 1+2y+y^2 \\
  x^2-2y &= 1 \\
  x &= \cos(t) \\
  -2y &= \sin^2(t) \\
  z &= 1-\frac{\sin^2(t)}{2} \\
  \overrightarrow{r(t)} &=
    \langle\cos(t),\frac{-\sin^2(t)}{2},1-\frac{\sin^2(t)}{2}\rangle
\end{align*}

\subsubsection*{Exercise 45}
Find a vector function that represents the curve of intersection of the
hyperboloid \( z = x^2-y^2 \) and the cylinder \( x^2+y^2 = 1 \).
\begin{align*}
  x &= \cos(t) \\
  y &= \sin(t) \\
  z &= x^2-y^2 \\
  &= \cos^2(t)-\sin^2(t) \\
  \overrightarrow{r(t)} &= \langle\cos(t),\sin(t),\cos^2(t)-\sin^2(t)\rangle
\end{align*}

\section*{Section 13.2}

\subsubsection*{Exercise 23}
Find the parametric equations for the tangent line to the curve with the
given parametric equations at the specified point.
\[ x = t^2+1 \quad y = 4\sqrt{t} \quad z = \e^{t^2-t}\ at\ (2,4,1) \]
\[ x' = 2t \quad y' = \frac{2}{\sqrt{t}} \quad z' = (2t-1)\e^{t^2-t} \]
\[ x'(2) = 4 \quad y'(4) = 1 \quad z'(1) = \e^0 = 1 \]
\[ l = \begin{cases}
  x &= 2+4t \\
  y &= 4+t \\
  z &= 1+t
\end{cases} \]

\subsubsection*{Exercise 27}
Find a vector equation for the tangent line to the curve of intersection of
the cylinders \( x^2+y^2 = 25 \) and \( y^2+z^2 = 20 \) at the point
(3,4,2).
\begin{align*}
  \frac{x^2}{25} &= \cos^2(t) \\
  x &= 5\cos(t) \\
  \frac{y^2}{25} &= \sin^2(t) \\
  y &= 5\sin(t) \\
  z^2 &= 20-y^2 = 20-25\sin^2(t) \\
  z &= \sqrt{20-25\sin^2(t)} \\
  \overrightarrow{r(t)} &=
    \left\langle5\cos(t),5\sin(t),\sqrt{20-25\sin^2(t)}\right\rangle \\
  \overrightarrow{r(x)} &= \langle3,4,2\rangle \\
  \cos(x) &= \frac{3}{5} \\
  \sin(x) &= \frac{4}{5} \\
  \overrightarrow{r'(t)} &= \left\langle
    -5\sin(t),
    5\cos(t),
    \frac{-25\sin(t)\cos(t)}{\sqrt{20-25\sin^2(t)}}
  \right\rangle \\
  \overrightarrow{r'(x)} &=
    \left\langle-4,3,\frac{-12}{\sqrt{4}}\right\rangle \\
  &= \langle-4,3,-6\rangle \\
  \overrightarrow{l(t)} &= \langle3-4t,4+3t,2-6t\rangle
\end{align*}

\begin{center}
  If you have any questions, comments, or concerns, please contact me at
  alvin@omgimanerd.tech
\end{center}

\end{document}
