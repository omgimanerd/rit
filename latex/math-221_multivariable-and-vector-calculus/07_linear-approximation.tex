\documentclass{math}

\title{Multivariable and Vector Calculus}
\author{Alvin Lin}
\date{August 2017 - December 2017}

\begin{document}

\maketitle

\section*{Linear Approximation}
\[ F(x,y,z) = 0 \]
In a special case:
\[ z-f(x,y) = 0 \]
Then:
\[ -f_x(x_{\circ},y_{\circ})(x-x_{\circ})-f_y(x_{\circ},y_{\circ})(y-y_{\circ})+
  1(z-f(x_{\circ},y_{\circ})) = 0 \]
\begin{align*}
  z-f(x_{\circ},y_{\circ}) &= f_x(x_{\circ},y_{\circ})(x-x_{\circ})+
    f_y(x_{\circ},y_{\circ})(y-y_{\circ}) \\
  f(x,y)-f(x_{\circ},y_{\circ}) &= f_x(x_{\circ},y_{\circ})(x-x_{\circ})+
    f_y(x_{\circ},y_{\circ})(y-y_{\circ}) \\
  f(x,y) &\approx f(x_{\circ},y_{\circ})+f_x(x_{\circ},y_{\circ})(x-x_{\circ})+
    f_y(x_{\circ},y_{\circ})(y-y_{\circ}) \\
\end{align*}
This is known as the linearization of the function. It provides a linear
approximation of the function.

\subsubsection*{Example}
Give the linear approximation of \( f(x,y) = \sqrt{x^2+y^2} \) at (4,3).
\begin{align*}
  \sqrt{x^2+y^2} &\approx 5+(\frac{1(2x)}{2\sqrt{x^2+y^2}})(x-4)+
    \frac{1}{2\sqrt{x^2+y^2}}(y-3) \\
  &\approx 5+\frac{4}{5}(x-4)+\frac{3}{5}(y-3)
\end{align*}

\begin{center}
  You can find all my notes at \url{http://omgimanerd.tech/notes}. If you have
  any questions, comments, or concerns, please contact me at
  alvin@omgimanerd.tech
\end{center}

\end{document}
