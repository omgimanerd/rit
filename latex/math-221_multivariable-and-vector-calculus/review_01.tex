\documentclass[letterpaper, 12pt]{math}

\title{Multivariable and Vector Calculus}
\author{Alvin Lin}
\date{August 2017 - December 2017}

\begin{document}

\maketitle

\section*{Review 1}
\[ \vec{a}\cdot\vec{b} = a_1b_2+a_2b_2+a_3b_3 =
  |\vec{a}||\vec{b}|\cos\theta \]
\[ proj_{\vec{a}}\vec{b} = \frac{\vec{a}\cdot\vec{v}}{|\vec{a}|} \]
\[ \vec{a}\cdot\vec{b} = 0 \equiv \vec{a}\bot\vec{b} \]
\[ \vec{a}\times\vec{b} = \begin{vmatrix}
  \i & \j & \k \\
  a_1 & a_2 & a_3 \\
  b_1 & b_2 & b_3
\end{vmatrix} \]
\[ \vec{a}\times\vec{b}\bot\vec{a},\vec{b} \]
\[ \vec{a}\times\vec{b} = -\vec{b}\times\vec{a} \]
\[ \vec{a}\times\vec{b} = \vec{0} \equiv \vec{a}\parallel\vec{b} \]
\[ \text{Volume of a parellelepiped} = (\vec{a}\times\vec{b})\cdot\vec{c} =
\begin{vmatrix}
  c_1 & c_2 & c_3 \\
  a_1 & a_2 & a_3 \\
  b_1 & b_2 & b_3
\end{vmatrix} \]
\[ \text{Line} = \begin{cases}
  x &= x_{\circ}+ta_1 \\
  y &= y_{\circ}+ta_2 \\
  z &= z_{\circ}+ta_3
\end{cases} \]
\[ \text{Plane} = n_1(x-x_{\circ})+n_2(y-y_{\circ})+n_3(z-z_{\circ}) = 0 \]

\subsubsection*{Example}
Find the angle between the diagonal of a cube and the edge of the base
starting at (0,0,0).
\[ \cos\alpha = \frac{\langle1,1,1\rangle\cdot\langle0,1,0\rangle}
  {|\langle1,1,1\rangle||\langle0,1,0\rangle|} = \frac{1}{\sqrt{3}} \]

\subsubsection*{Example}
Find \( x \) such that \( \langle1,x,2\rangle\bot\langle x,3,4\rangle \).
\begin{align*}
  \langle1,x,2\rangle\cdot\langle x,3,4\rangle &= 0 \\
  x+3x+8 &= 0 \\
  x &= -2
\end{align*}
Is there such an \( x \) that the two vectors are parallel?
\begin{align*}
  c\langle1,x,2\rangle &= \langle x,3,r\rangle \\
  c &= x \\
  cx &= 3 \\
  2c &= 4
\end{align*}
This system of linear equations is inconsistent, therefore the two vectors
are not parallel.

\subsubsection*{Example}
Given the planes \( \Pi_1: x-2y+z = 4 \) and \( \Pi_2: 2x-2x-z = 6 \). Find the
line of intersection between the planes \( l = \Pi_1\cap\Pi_2 \). \\
If we let \( x = 0 \):
\[ \begin{cases}
  -2y+z &= 4 \\
  -2y-z &= 6
\end{cases} \]
\[ y = -\frac{5}{2} \quad z = -1 \quad P_1 = (0,-\frac{5}{2},1) \]
If we let \( y = 0 \):
\[ \begin{cases}
  x+z &= 4 \\
  2x-z &= 6
\end{cases} \]
\[ x = \frac{10}{3} \quad z = \frac{2}{3} \quad P_2 =
  (\frac{10}{3},0,\frac{2}{3}) \]
\[ \overrightarrow{P_1P_2} =
  \langle\frac{10}{3},\frac{5}{2},-\frac{1}{3}\rangle \]
\[ l = \begin{cases}
  x &= \frac{10}{3}+t\frac{10}{3} \\
  x &= t\frac{5}{2} \\
  z &= \frac{2}{3}-t\frac{1}{3}
\end{cases} \]
Find the angle between \( \Pi_1,\Pi_2 \).
\[ \cos\alpha = \frac{\vec{n_1}\cdot\vec{n_2}}{|\vec{n_1}||\vec{n_2}|} =
  \frac{\langle1,-2,1\rangle\cdot\langle2,-2,-1\rangle}{\sqrt{6}\sqrt{9}} =
  \frac{5}{3\sqrt{6}} \]

\subsubsection*{Example}
Given \( \Pi_1: x-2y+z = 4 \) and \( l: x-2 = y+5 = z-4 = t \), find
\( P = l\cap\Pi \).
\begin{align*}
  x-2y+z &= 4 \\
  (t+2)-2(t-5)+(t+4) &= 4 \\
  16 &\ne 4
\end{align*}
Since there is no solution, the line \( l \) does not intersect the plane. \\
Find the distance between \( l \) and \( \Pi \). Pick any point \( P_0 \) on
the plane and any point \( P_1 \) on the line.
\begin{align*}
  P_0 &= (4,0,0) \\
  P_1 &= (2,-5,4) \\
  dist(l,\Pi) &= comp_{\vec{n}}{\overrightarrow{P_0P_1}} \\
  &= \frac{\overrightarrow{P_0P_1}\cdot\vec{n}}{|\vec{n}|} \\
  &= \frac{\langle-2,-5,4\rangle\cdot\langle1,-2,1\rangle}
    {|\langle1,-2,1\rangle|} \\
  &= \frac{14}{\sqrt{6}}
\end{align*}

\begin{center}
  If you have any questions, comments, or concerns, please contact me at
  alvin@omgimanerd.tech
\end{center}

\end{document}
