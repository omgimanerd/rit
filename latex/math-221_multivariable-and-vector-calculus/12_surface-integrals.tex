\documentclass{math}

\usepackage{tikz}

\title{Multivariable and Vector Calculus}
\author{Alvin Lin}
\date{August 2017 - December 2017}

\begin{document}

\maketitle

\section*{Surface Integrals}
If we have a surface \( S \) described by \( r(u,v), (u,v)\in D \), the
area of the surface can be approximated as:
\[ \left|\pdiff{r}{u}\times\pdiff{r}{v}\right|\diff{u}\diff{v} \]
Find the mass of the surface \( S \) if the density of the surface is
\( f(x,y,z), S:\overrightarrow{r(u,v)}, (u,v)\in D \).
\begin{center}
  \begin{tikzpicture}
    \draw[very thick,->] (0,0) -- (5,0) node[right] {x axis};
    \draw[very thick,->] (0,0) -- (0,5) node[above] {y axis};
    \draw (2,3.9) -- (2,-1) node[below] {\( a \)};
    \draw (4,3.9) -- (4,-1) node[below] {\( b \)};
    \draw (4,1.1) -- (-1,1.1) node[left] {\( c \)};
    \draw (4,3.9) -- (-1,3.9) node[left] {\( d \)};
    \node at (3.3,3.6) {\( D \)};
    \draw[red] (2,2) .. controls (2,0.7) and (3,1) .. (4,2);
    \draw[red] (2,2) .. controls (3,5) and (4,4) .. (4,2);
    \draw (3,2) -- (3.2,2) -- (3.2,1.8) -- (3,1.8) -- cycle node[above]
      {\( P_{ij} \)};
  \end{tikzpicture}
\end{center}
\[ D\subset R = [a,b]\times[c,d] \]
We divide \( [a,b],[c,d] \) into \( nm \) equal parts to get \( R_{ij} \).
\[ mass \approx \sum_{i=1}^{n}\sum_{i=1}^{m}S_yf(P_{ij}) =
  \sum_{i=1}^{n}\sum_{j=1}^{m}\left|\pdiff{r}{u}\times\pdiff{r}{v}\right|
    f(P_{ij}) \diff{u}\diff{v} \]
If we take the limit as \( n,m\to\infty \), this becomes the integral:
\[ \iint_{S}f\diff{S} =
  \iint_{D}f\left|\pdiff{r}{u}\times\pdiff{r}{v}\right|\diff{u}\diff{v} \]

\subsubsection*{Example}
Find \( \iint_{S}z^2\diff{S} \) given \( S:z=y^2+x^2, 0\le x\le1 \).
\begin{align*}
  S:r(R,\theta) &= \langle R^2,R\cos\theta, R\sin\theta\rangle \\
  \left|\pdiff{r}{R}\times\pdiff{r}{\theta}\right| &= \left|\begin{vmatrix}
    \i & \j & \k \\
    2R & \cos\theta & \sin\theta \\
    0 & -R\sin\theta & R\cos\theta
  \end{vmatrix}\right| \\
  &= \left|\langle R,-2R^2\cos\theta,-2R^2\sin\theta\rangle\right| \\
  &= \sqrt{R^2+4R^4\cos^2\theta+4R^4\sin^2\theta} \\
  &= R\sqrt{1+4R^2} \\
  \iint_{S}z^2\diff{S} &= \int_{0}^{1}\int_{0}^{2\pi}R^2\sin^2\theta\diff{S} \\
  &= \int_{0}^{1}\int_{0}^{2\pi}R^2\sin^2\theta R\sqrt{1+4R^2}
    \diff{\theta}\diff{r} \\
  &= \int_{0}^{1}R^3\sqrt{1+4R^2}
    \left[\frac{\theta}{2}-\frac{\sin2\theta}{4}\right]_{0}^{2\pi}\diff{R} \\
  &= \pi\int_{0}^{1}R^3\sqrt{1+4R^2}\diff{R} \\
  Let:~u &= R^2 \quad \diff{v} = R\sqrt{1+4R^2}\diff{R} \\
  \diff{u} &= 2R\diff{R} \quad
    v = \frac{2}{3}(1+4R^2)^{\frac{3}{2}}\frac{1}{8} \\
  &= \pi\left[R^2(1+4R^2)^{\frac{3}{2}}\frac{1}{2}-
    \int_{0}^{1}\frac{1}{2}(1+4R^2)^{\frac{3}{2}}2R\diff{R}\right] \\
  &= \pi\left[R^2(1+4R^2)^{\frac{3}{2}}\frac{1}{2}-
    \frac{1}{6}\frac{2}{5}(1+4R^2)^{\frac{5}{2}}\frac{1}{8}\right]
\end{align*}

\subsubsection*{Example}
Find \( \iint_{S}y^2\diff{S} \) given \( S:x^2+y^2+z^2 = 1 \) bounded above
\( z = \sqrt{x^2+y^2} \).
\begin{align*}
  r(\phi,\theta) &=
    \langle\sin\phi\cos\theta,\sin\phi\sin\theta,\cos\phi\rangle \\
  \left|\pdiff{r}{\phi}\times\pdiff{r}{\theta}\right| &= \left|\begin{vmatrix}
    \i & \j & \k \\
    \cos\phi\cos\theta & \cos\phi\sin\theta & -\sin\phi \\
    -\sin\phi\sin\theta & \sin\phi\cos\theta & 0
  \end{vmatrix}\right| \\
  &= \sin\phi \\
  \iint_{S}y^2\diff{S} &= \int_{0}^{2\pi}\int_{0}^{1}\sin^2\phi\sin^2\theta
    \sqrt{\sin^4\theta+\cos^2\phi\sin^2\theta}\diff{\phi}\diff{\theta}
\end{align*}

\subsubsection*{Example}
Find the surface area of sphere of radius \( R \).
\begin{align*}
  r(\phi,\theta) &= \langle
    R\sin\phi\cos\theta,R\sin\phi\cos\theta,R\cos\phi\rangle \\
  \left|\pdiff{r}{\phi}\times\pdiff{r}{\theta}\right| &= \left|\begin{vmatrix}
    \i & \j & \k \\
    R\cos\phi\cos\theta & R\cos\phi\sin\theta & -R\sin\phi \\
    -R\sin\phi\sin\theta & R\sin\phi\cos\theta & 0
  \end{vmatrix}\right| \\
  &= R^2\sin\phi \\
  \iint_{S}1\diff{S} &= \int_{0}^{2\pi}\int_{0}^{1}
    R^2\sin\phi\diff{\phi}\diff{\theta} \\
  &= 4\pi R^2
\end{align*}

\subsubsection*{Example}
Suppose there is a function \( \vec{F} \) on \( S \).
\begin{align*}
  \iint_{S}\vec{F}\diff{S} &= \iint_{S}(\vec{F}\cdot\vec{r})\diff{S} \\
  &= \iint_{S}\vec{F}\cdot\frac{\pdiff{r}{u}\times\pdiff{r}{v}}
    {\left|\pdiff{r}{u}\times\pdiff{r}{v}\right|}\diff{S} \\
  &= \iint_{D}\vec{F}\cdot\frac{\pdiff{r}{u}\times\pdiff{r}{v}}
    {\left|\pdiff{r}{u}\times\pdiff{r}{v}\right|}
    \left|\pdiff{r}{u}\times\pdiff{r}{v}\right|\diff{u}\diff{v} \\
  &= \iint_{D}\vec{F}\cdot\left(\pdiff{r}{u}\times\pdiff{r}{v}\right)
    \diff{u}\diff{v}
\end{align*}

\subsubsection*{Example}
Find \( \iint_{S}\langle x,y,5\rangle\diff{S} \) given \( S:x^2+y^2 = 1, y = 0,
x+y = 2 \), picking an outwards normal vector to the surface.
\begin{align*}
  r(\theta,y) &= \langle\cos\theta,y,\sin\theta\rangle \\
  \pdiff{r}{\theta}\times\pdiff{r}{y} &= \begin{vmatrix}
    \i & \j & \k \\
    -\sin\theta & 0 & \cos\theta \\
    0 & 1 & 0
  \end{vmatrix} \\
  &= \langle-\cos\theta,0,-\sin\theta\rangle
    \text{ reverse signs for outwards normal vector} \\
  \iint_{S}\langle x,y,5\rangle\diff{S} &=
    \int_{0}^{2\pi}\int_{0}^{2-\cos\theta}\langle\cos\theta,y,5\rangle\cdot
    \langle\cos\theta,0,\sin\theta\rangle\diff{y}\diff{\theta} \\
  &= \int_{0}^{2\pi}\int_{0}^{2-\cos\theta}\cos^2\theta+5\sin\theta
    \diff{y}\diff{\theta} \\
\end{align*}

\subsection*{Divergence Theorem}
Let \( S \) be a oriented and closed surface.
\[ \iint_{S}\vec{F}\diff{S} =
  \iiint_{V}\vec{\triangledown}\cdot\vec{F}\diff{V} \]
\[ \vec{\triangledown}\cdot\vec{F} =
  \langle\ddiff{}{x},\ddiff{}{y},\ddiff{}{z}\rangle\cdot
  \langle P,Q,R\rangle = \pdiff{P}{x}+\pdiff{Q}{y}+\pdiff{R}{z} \]

\subsubsection*{Example}
\begin{align*}
  \vec{\triangledown}\cdot\langle x^yz,xy^2z,xyz^2\rangle &= 2xyz+2xyz+2xyz \\
  \iint_{S}\langle x^yz,xy^2z,xyz^2\rangle\diff{S} &=
    \int_{0}^{1}\int_{0}^{1}\int_{0}^{1}6xyz\diff{z}\diff{y}\diff{x} \\
\end{align*}

\begin{center}
  You can find all my notes at \url{http://omgimanerd.tech/notes}. If you have
  any questions, comments, or concerns, please contact me at
  alvin@omgimanerd.tech
\end{center}

\end{document}
