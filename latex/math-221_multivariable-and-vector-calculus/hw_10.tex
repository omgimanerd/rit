\documentclass{math}

\geometry{letterpaper, margin=0.5in}

\title{Multivariable and Vector Calculus: Homework 10}
\author{Alvin Lin}
\date{August 2016 - December 2016}

\begin{document}

\maketitle

\section*{Section 16.2}

\subsubsection*{Exercise 3}
Evaluate the line integral, where \( C \) is the given curve.
\[ \int_{C}xy^4\diff{s} \quad C \text{ is the right half of the circle }
  x^2+y^2 = 16 \]
\begin{align*}
  \diff{s} &= \sqrt{(-4\sin(t))^2+(4\cos(t))^2}\diff{t} \\
  &= 4\diff{t} \\
  \int_{C}xy^4\diff{s} &= \int_{-\frac{\pi}{2}}^{\frac{\pi}{2}}
    (4\cos(t))(4\sin(t))^4(4\diff{t}) \\
  &= 4^6\int_{-\frac{\pi}{2}}^{\frac{\pi}{2}}\cos(t)\sin^4(t)\diff{t} \\
  &= 4^6\bigg[\frac{\sin^5(t)}{5}\bigg]_{-\frac{\pi}{2}}^{\frac{\pi}{2}} \\
  &= \frac{4096(2)}{5} = \frac{8192}{5}
\end{align*}

\subsubsection*{Exercise 11}
Evaluate the line integral, where \( C \) is the given curve.
\[ \int_{C}x\e^{yz}\diff{s} \quad C \text{ is the line segment from }
  (0,0,0) \text{ to } (1,2,3) \]
\begin{align*}
  c(t) &= \langle t,2t,3t\rangle \quad 0\le t\le 1 \\
  c'(t) &= \langle1,2,3\rangle \\
  |c'(t)| &= \sqrt{14} \\
  \diff{s} &= \sqrt{14}\diff{t} \\
  \int_{C}x\e^{yz}\diff{s} &= \int_{0}^{1}t\e^{(2t)(3t)}\sqrt{14}\diff{t} \\
  &= \sqrt{14}\int_{0}^{1}t\e^{6t^2}\diff{t} \\
  &= \sqrt{14}\bigg[\frac{\e^{6x^2}}{12}\bigg]_{0}^{1} \\
  &= \sqrt{14}\frac{\e^6-1}{12}
\end{align*}

\subsubsection*{Exercise 21}
Evaluate the line integral \( \int_{C}F\diff{r} \), where \( C \)
is given by the vector function \( r(t) \).
\[ F(x,y,z) = \sin(x)\i+\cos(y)\j+xz\k \quad r(t) = t^3\i-t^2\j+t\k \quad
  (0\le t\le1) \]
\begin{align*}
  r(t) &= t^3\i-t^2\j+t\k \\
  r'(t) &= 3t^2\i-2t\j+\k \\
  \diff{r} &= (3t^2\i-2t\j+\k)\diff{t} \\
  F(x,y,z) &= \sin(x)\i+\cos(y)\j+xz\k \\
  &= \sin(t^3)\i+\cos(-t^2)\j+t^3(t)\k \\
  \int_{C}F\cdot\diff{r} &= \int_{0}^{1}3t^2\sin(t^3)-2t\cos(t^2)+t^4\diff{t} \\
  &= \bigg[-\cos(t^3)-\sin(t^2)+\frac{t^5}{5}\bigg]_{0}^{1} \\
  &= \frac{6}{5}-\cos(1)-\sin(1)
\end{align*}

\subsubsection*{Exercise 25}
Use a calculator to evaluate the line integral correct to four decimal places.
\[ \int_{C}xy\arctan{z}\diff{s} \quad \text{ where } C \text{ has parametric
equations } x = t^2, y = t^3, z = \sqrt{t}, 1\le t\le 2 \]
\begin{align*}
  r(t) &= \langle t^2,t^3,\sqrt{t}\rangle \\
  r'(t) &= \langle 2t,3t^2,\frac{1}{2\sqrt{t}}\rangle \\
  |r'(t)| &= \sqrt{4t^2+9t^2+\frac{1}{4t}} \\
  \int_{C}xy\arctan{z}\diff{s} &= \int_{1}^{2}(t^2)(t^3)\arctan{\sqrt{t}}
    \sqrt{4t^2+9t^2+\frac{1}{4t}}\diff{t} \\
  &\approx 94.8232
\end{align*}

\subsubsection*{Exercise 32}
Find the work done by the force field \( F(x,y) = x^2\i+xy\j \) on a particle
that moves once around the circle \( x^2+y^2 = 4 \) oriented in the counter
clockwise direction.
\begin{align*}
  r(t) &= 2\cos(t)\i+2\sin(t)\j \quad (0\le t\le 2\pi) \\
  r'(t) &= -2\sin(t)\i+2\cos(t)\j \\
  \int_{C}F\cdot\diff{r} &= \int_{0}^{2\pi}4\cos^2(t)(-2\sin(t))+
    (2\cos(t))(2\sin(t))(2\cos(t))\diff{t} \\
  &= \int_{0}^{2\pi}-8\cos^2(t)\sin(t)+8\cos^2(t)\sin(t)\diff{t} \\
  &= \int_{0}^{2\pi}0\diff{t} \\
  &= 0
\end{align*}

\subsubsection*{Exercise 41}
Find the work done by the force field
\[ F(x,y,z) = \langle x-y^2,y-z^2,z-x^2\rangle \]
on a particle that moves along the line segment from (0,0,1) to (2,1,0).
\begin{align*}
  r(t) &= \langle2t,t,1-t\rangle \quad (0\le t\le1) \\
  r'(t) &= \langle2,1,-1\rangle \\
  F(x,y,z) &= \langle2t-t^2,-t^2+3t-1,1-t-4t^2\rangle \\
  \int_{0}^{1}F(x,y,z)\cdot r'(t)\diff{t} &= \int_{0}^{1}
    2(2t-t^2)+(-t^2-3t-1)-(1-t-4t^2)\diff{t} \\
  &= \int_{0}^{1}4t-2t^2-t^2+3t-1-1+t+4t^2\diff{t} \\
  &= \int_{0}^{1}t^2+8t-2\diff{t} \\
  &= \bigg[\frac{t^3}{3}+\frac{8t^2}{2}-2t\bigg]_{0}^{1} \\
  &= \frac{1}{3}+4-2 \\
  &= \frac{7}{3}
\end{align*}

\begin{center}
  If you have any questions, comments, or concerns, please contact me at
  alvin@omgimanerd.tech
\end{center}

\end{document}
