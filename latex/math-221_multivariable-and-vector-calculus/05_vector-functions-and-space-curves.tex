\documentclass[letterpaper, 12pt]{math}

\title{Multivariable and Vector Calculus}
\author{Alvin Lin}
\date{August 2017 - December 2017}

\begin{document}

\maketitle

\section*{Vector Functions and Space Curves}
\[ v:t\in\R \longrightarrow \vec{v(t)} = \langle x(t),y(t),z(t)\rangle \]
Parametric line equations are similar to this form:
\[ l = \begin{cases}
  x &= x_{\circ}+ta_1 \\
  y &= y_{\circ}+ta_2 \\
  z &= z_{\circ}+ta_3
\end{cases} \]
\[ r(t) = \langle x_{\circ}+ta_1,y_{\circ}+ta_2,z_{\circ}+ta_3\rangle \]

\subsubsection*{Example}
\[ \vec{v(t)} = \langle\cos(t),\sin(t),t^3\rangle \]
We know that \( x = \cos(t) \) and \( y = \sin(t) \) and we have the identity
\( \sin^2\theta+\cos^2\theta = 1 \), therefore \( x^2+y^2 = 1 \). We can
determine this this vector traces a helix on the perimeter of a cylinder along
the z-axis.

\subsubsection*{Example}
Find the point where the helix \( v(t) = \langle\cos(t),\sin(t),t\rangle \)
intersects the sphere with equation \( x^2+y^2+z^2 = 5 \).
\begin{align*}
  x^2+y^2+z^2 &= 5 \\
  \cos^2t+\sin^2t+t^2 &= 5 \\
  1+t^2 &= 5 \\
  t^2 &= 4 \\
  t &= \pm2
\end{align*}
There are two points at which they intersect, when \( t = 2 \) and when
\( t = -2 \), \( P_1(\cos2,\sin(2),2) \) and \( P_2(\cos(-2),\sin(-2),-2) \).

\subsubsection*{Example}
Suppose you have two surfaces \( S_1:x^2+y^2 = 4 \) and \( S_2:z=xy \). Find
the curve of intersection \( C \). We can let \( x = 2\cos(t) \) and
\( y = 2\sin(t) \) in order to fulfill \( x^2+y^2 = 4 \).
\[ \vec{r(t)} = \langle2\cos(t),2\sin(t),4\cos(t)\sin(t) \]

\subsubsection*{Example}
Given the vector equation for the curve of intersection between
\( S_1: z = 4x^2+y^2 \) and \( S_2: y = x^2 \). In terms of x:
\[ \vec{r(x)} = \langle x,x^2,4x^2+y^2\rangle = \langle x,x^2,4x^2+x^4\rangle \]
\[ \vec{r(t)} = \langle t,t^2,4t^2+t^4\rangle \]

\subsubsection*{Example}
Give the vector equation for the curve of intersection between
\( S_1: x^2+z^2 = 1 \) and \( S_2: x^2+y^2+4z^2 = 4, y\ge0 \).
\[ \vec{r(t)} = \langle\cos(t),\sqrt{4-\cos^2(t)-4\sin^2(t)},\sin(t)\rangle \]

\subsubsection*{Example}
Given a surface \( S \) such that the curve \( \vec{r(t)} = \langle t^2,\ln(t),
\frac{1}{2}\rangle \) belongs to \( S \).
\begin{align*}
  t^2-t^2 &= 0 \\
  x-\e^{2\ln(t)} = 0
\end{align*}

\subsubsection*{Example}
Find the length of the arc of the curve \( \overrightarrow{r(t)} =
\langle t^2,9t,4t^\frac{3}{2}\rangle \quad 1\le t\le 4 \):
\begin{align*}
  L &= \int_{1}^{4}|\langle2t,9,4\frac{3}{2}\sqrt{t}\rangle|\diff{t} \\
  &= \int_{1}^{4}\sqrt{4t^2+81+36t}\diff{t} \\
  &= \int_{1}^{4}2t+9\diff{t} \\
  &= t^2+9\bigg]^{4}_{1}
\end{align*}

\subsubsection*{Example}
Find a point on \( \overrightarrow{r(t)} = \langle3\sin(t),4t,3\cos(t)
\rangle \) 5 units from (0,0,3) in the direction of increasing \( t \).
\begin{align*}
  L &= 5 \\
  &= \int_{0}^{\tau}|\langle3\cos(t),4,-3\sin(t)\rangle|\diff{t} \\
  &= \int_{0}^{\tau}\sqrt{9\cos^2(t)+16+9\sin^2(t)}\diff{t} \\
  &= \int_{0}^{\tau}\sqrt{9(\cos^2(t)+\sin^2(t))16}\diff{t} \\
  &= \int_{0}^{\tau}\sqrt{9+16}\diff{t} \\
  &= \int_{0}^{\tau}5\diff{t} \\
  &= 5t\bigg]_{t=0}^{t=\tau} \\
  &= 5\tau \\
  \tau &= 1 \\
  P &= \big(3\sin(1),4,3\cos(1)\big)
\end{align*}

\subsubsection*{Example}
Given:
\begin{align*}
  \overrightarrow{v(0)} &= \langle1,0,-4\rangle \\
  \overrightarrow{v'(0)} &= \langle0,0,-1\rangle \\
  \overrightarrow{v''(t)} &= \langle\sin(t),2\cos(t),6t\rangle \\
\end{align*}
Find \( \overrightarrow{v(t)} \):
\begin{align*}
  \overrightarrow{v'(t)} &= \langle-\cos(t)+c_1,2\sin(t)+c_2,3t^2+c_3
    \rangle = \langle0,0,-1\rangle \\
  \therefore & \quad c_1 = 1 \quad c_2 = 0 \quad c_3 = -1 \\
  \overrightarrow{v'(t)} &= \langle-\cos(t)+1,2\sin(t),3t^2-1\rangle \\
  \overrightarrow{v(t)} &= \langle-\sin(t)+t+c_1,-2\cos(t)+c_2,t^3-t+c_3
    \rangle = \langle1,0,-4\rangle \\
  \therefore & \quad c_1 = 1 \quad c_2 = 2 \quad c_3 = -4 \\
  \overrightarrow{v(t)} &= \langle-\sin(t)+t+1,-2\cos(t)+2,t^3-t-4\rangle
\end{align*}

\subsubsection*{Example}
A bullet fired at an angle of \( 30^{\circ} \) to the horizontal lands 1600ft
away. Find \( v_{\circ} \). \\
Newton's Second Law:
\begin{align*}
  \vec{F} &= m\vec{a} \\
  &\equiv m\vec{v''} = \langle0,-mg\rangle \\
  \text{Assume } m = 1 \\
  &\equiv \overrightarrow{v''(t)} = \langle0,-g\rangle \\
  &= \langle c_1,-gt+c_2\rangle = \langle
    v_{\circ}\cos(30),v_{\circ}\cos(60)\rangle \text{ at } t = 0 \\
  \therefore & \quad c_1 = \frac{v_{\circ}\sqrt{3}}{2} \quad
    c_2 = \frac{v_{\circ}}{2} \\
  \overrightarrow{v''(t)} &= \langle\frac{v_{\circ}\sqrt{3}}{2},-gt+
    \frac{v_{\circ}}{2}\rangle \\
  \overrightarrow{v(t)} &= \langle\frac{v_{\circ}\sqrt{3}}{2}t+c_1,-gt+
    \frac{v_{\circ}t}{2}+c_2\rangle = \langle0,0\rangle \\
  \therefore & \quad c_1 = 0 \quad c_2 = 0 \\
  \overrightarrow{v(t)} &= \langle\frac{v_{\circ}\sqrt{3}}{2}t,
    \frac{-gt^2}{2}+\frac{v_{\circ}}{2}t\rangle = \langle0,0\rangle \\
  \frac{-gt^2}{2}+\frac{v_{\circ}}{2}t &= 0 \\
  \therefore t &= \frac{v_{\circ}}{g} \\
  \frac{v_{\circ}\sqrt{3}}{2}t &= 1600 \\
  \frac{v_{\circ}\sqrt{3}}{2}\frac{v_{\circ}}{g} &= 1600 \\
  v_{\circ} &= \sqrt{\frac{3200g}{\sqrt{3}}}
\end{align*}

\begin{center}
  You can find all my notes at \url{http://omgimanerd.tech/notes}. If you have
  any questions, comments, or concerns, please contact me at
  alvin@omgimanerd.tech
\end{center}

\end{document}
