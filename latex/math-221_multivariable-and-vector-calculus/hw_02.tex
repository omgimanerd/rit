\documentclass[letterpaper, 12pt]{math}

\usepackage{amsmath}
\usepackage{amssymb}
\usepackage{geometry}
\usepackage{tikz}

\geometry{letterpaper, margin=0.5in}

\title{Multivariable and Vector Calculus: Homework 2}
\author{Alvin Lin}
\date{August 2016 - December 2016}

\begin{document}

\maketitle

\section*{Page 821}

\subsubsection*{Problem 5}
Find the cross product \( \vec{a}\times\vec{b} \) and verify that it is
orthogonal to both \( a \) and \( b \).
\begin{align*}
  \vec{a} &= \frac{1}{2}\i+\frac{1}{3}\j+\frac{1}{4}\k \\
  \vec{b} &= \i+2\j-3\k \\
  \vec{a}\times\vec{b} &= \begin{vmatrix}
    \i & \j & \k \\
    \frac{1}{2} & \frac{1}{3} & \frac{1}{4} \\
    1 & 2 & -3
  \end{vmatrix} = \frac{-3}{2}\i+\frac{7}{4}\j+\frac{2}{3}\k \\
  \vec{a}\times\vec{b}\cdot\vec{a} &=
    \frac{-3}{4}+\frac{7}{12}+\frac{2}{12} = 0 \\
  \vec{a}\times\vec{b}\cdot\vec{b} &=
    \frac{-3}{2}+\frac{14}{4}-2 = 0
\end{align*}

\subsubsection*{Problem 7}
Find the cross product \( \vec{a}\times\vec{b} \) and verify that it is
orthogonal to both \( a \) and \( b \).
\begin{align*}
  \vec{a} &= \langle t,1,\frac{1}{t}\rangle \\
  \vec{b} &= \langle t^2,t^2,1\rangle \\
  \vec{a}\times\vec{b} &= \begin{vmatrix}
    \i & \j & \k \\
    t & 1 & \frac{1}{t} \\
    t^2 & t^2 & 1
  \end{vmatrix} = \langle1-t,0,t^3-t^2\rangle \\
  \vec{a}\times\vec{b}\cdot\vec{a} &=
    t(1-t)+0+\frac{t^3-t^2}{t} \\
  &= t-t^2+t^2-t = 0 \\
  \vec{a}\times\vec{b}\cdot\vec{b} &=
    t^2(1-t)+0+t^3-t^2 = 0
\end{align*}

\subsubsection*{Problem 11}
Find the vector, not with determinants, but by using properties of cross
products.
\begin{align*}
  (\j-\k)\times(\k-\i) &= (\j-\k)\times\k+(\j-\k)\times-\i \\
  &= (\j\times\k)+(\j\times-\i)+(-\k\times\k)+(-\k\times-\i) \\
  &= \i+\k+0+\j = \i+\j+\k
\end{align*}

\subsubsection*{Problem 13}
State whether each expression is meaningful:
\begin{enumerate}
  \item \( \vec{a}\cdot(\vec{b}\times\vec{c}) \) \\
  Scalar
  \item \( \vec{a}\times(\vec{b}\cdot\vec{c}) \) \\
  Nonsensical (vector cross scalar)
  \item \( \vec{a}\times(\vec{b}\times\vec{c}) \) \\
  Vector
  \item \( \vec{a}\cdot(\vec{b}\cdot\vec{c}) \) \\
  Nonsensical (vector dot scalar)
  \item \( (\vec{a}\cdot\vec{b})\times(\vec{c}\cdot\vec{d}) \) \\
  Nonsensical (scalar cross scalar)
  \item \( (\vec{a}\times\vec{b})\cdot(\vec{c}\times\vec{d}) \) \\
  Scalar
\end{enumerate}

\subsubsection*{Problem 27}
Find the area of the parallelogram with vertices A(-3,0), B(-1,3), C(5,2), and
D(3,-1).
\begin{center}
  \begin{tikzpicture}
    \draw (-3,0) node[left] {A} -- (-1,3) node[above] {B} --
      (5,2) node[right] {C} -- (3,-1) node[below] {D} -- (-3,0);
  \end{tikzpicture}
\end{center}
\begin{align*}
  \overrightarrow{AB} &= \langle2,3\rangle \\
  \overrightarrow{AD} &= \langle6,-1\rangle \\
  \overrightarrow{AB}\times\overrightarrow{AD} &= \begin{vmatrix}
    \i & \j & \k \\
    2 & 3 & 0 \\
    6 & -1 & 0 \\
  \end{vmatrix} = \langle0,0,-16\rangle \\
  \text{area of ABCD} = |\langle0,0,-16\rangle| = 16
\end{align*}

\subsubsection*{Problem 29}
Find a nonzero vector orthogonal to the plane through the points P,Q, and R and
find the area of \( \triangle \) PQR: P(1,0,1), Q(-2,1,3), R(4,2,5).
\begin{align*}
  \overrightarrow{PQ} &= \langle-3,1,2\rangle \\
  \overrightarrow{PR} &= \langle3,2,4\rangle \\
  \overrightarrow{PQ}\times\overrightarrow{PR} &= \begin{vmatrix}
    \i & \j & \k \\
    -3 & 1 & 2 \\
    3 & 2 & 4
  \end{vmatrix} = \langle0,18,9\rangle
\end{align*}
\( \langle0,2,1\rangle \) is a vector orthogonal to the plane and the area of
\( \triangle \) PQR is:
\[ \frac{1}{2}|\langle0,18,9\rangle| = \frac{\sqrt{405}}{2} \]

\subsubsection*{Problem 33}
Find the volume of the parallelepiped determine by the vectors:
\[ \vec{a} = \langle1,2,3\rangle \quad \vec{b} = \langle-1,1,2\rangle
  \quad \vec{c} = \langle2,1,4\rangle \]
\begin{align*}
  V &= (\vec{a}\times\vec{b})\cdot\vec{c} \\
  &= \begin{vmatrix}
    \i & \j & \k \\
    1 & 2 & 3 \\
    -1 & 1 & 2
  \end{vmatrix}\cdot\langle2,1,4 \\
  &= \langle0,-5,3\rangle\cdot\langle2,1,4\rangle \\
  &= 0-5+12 = 7
\end{align*}

\subsubsection*{Problem 37}
Use the scalar triple product to verify that the vectors \( \vec{u} = \i+5\j-
2\k \), \( \vec{v} = 3\i-\j \), and \( w = 5\i+9\j-4\k \) are coplanar.
\begin{align*}
  V &= (\vec{u}\times\vec{b})\cdot\vec{w} \\
  &= \begin{vmatrix}
    \i & \j & \k \\
    1 & 5 & -2 \\
    3 & -1 & 0
  \end{vmatrix}\cdot\langle5,9,-4\rangle \\
  &= \langle-2,-6,-16\rangle\cdot\langle5,9,-4\rangle \\
  &= -10-54+64 = 0
\end{align*}
Since the scalar triple product is 0, the vectors must be coplanar.

\subsection*{831}

\subsubsection*{5}
Find a vector equation and parametric equations for the line through the
point (1,0,6) and perpendicular to the plane \( x+3y+z = 5 \).
\[ \vec{n} = \langle1,3,1\rangle \]
\begin{align*}
  \vec{r} &= \langle1,0,6\rangle+\langle1,3,1\rangle t \\
  &= \langle1+t,3t,6+t\rangle \\
  &\equiv \begin{cases}
    x &= 1+t \\
    y &= 3t \\
    z &= 6+t
  \end{cases}
\end{align*}

\subsubsection*{9}
Find parametric equations and symmetric equations for the line through the
points (-8,1,4) and (3,-2,4).
\[ \vec{d} = \langle11,-3,0\rangle \]
\begin{align*}
  l &= \begin{cases}
    x &= 3+11t \\
    y &= -2-3t \\
    z &= 4
  \end{cases} \\
  &\equiv \frac{x-3}{11} = \frac{y+2}{-3}
\end{align*}

\subsubsection*{19}
Determine whether the lines \( L_1 \) and \( L_2 \) are parallel, skew, or
intersecting. If they intersect, find the point of intersection.
\begin{align*}
  L_1 &= \begin{cases}
    x &= 3+2t \\
    y &= 4-t \\
    z &= 1+3t
  \end{cases} \\
  L_2 &= \begin{cases}
    x &= 1+4s \\
    y &= 3-2s \\
    z &= 4+5s
  \end{cases}
  \vec{n_1} &= \langle2,-1,3\rangle \\
  \vec{n_2} &= \langle4,-2,5\rangle \\
  3+2t &= 1+4s \\
  t &= 2s-1 \\
  4-t &= 3-2s \\
  4-(2s-1) &= 3-2s \\
  5-2s &= 3-2s
\end{align*}
No solution, therefore the lines are skew.

\subsubsection*{27}
Find an equation of the plane through the point (1,-1,-1) and parallel to the
plane \( 5x-y-z = 6 \).
\begin{align*}
  \vec{n} &= \langle5,-1,-1\rangle \\
  \Pi &\equiv 5(x-1)-1(y+1)-1(z+1) = 0 \\
  &\equiv 5x-5-y-1-z-1 = 0 \\
  &\equiv 5x-y-z = 7
\end{align*}

\subsubsection*{31}
Find an equation of the plane through the points (0,1,1), (1,0,1), and (1,1,0).
\begin{align*}
  \vec{u} &= \langle1,-1,0\rangle \\
  \vec{v} &= \langle1,0,-1\rangle \\
  \vec{n} &= \vec{u}\times\vec{v} \\
  &= \begin{vmatrix}
    \i & \j & \k \\
    1 & -1 & 0 \\
    1 & 0 & -1
  \end{vmatrix} = \langle1,1,1\rangle \\
  \Pi &\equiv 1(x-0)+1(y-1)+1(z-1) = 0 \\
  &\equiv x+y-1+z-1 = 0 \\
  &\equiv x+y+z = 2
\end{align*}

\subsubsection*{37}
Find the equation of the plane that passes through the point (3,1,4) and
contains the line of intersection of the planes \( x+2y+3z = 1 \) and
\( 2x-y+z = -3 \). \\
Let \( z = 0 \):
\begin{align*}
  x+2y &= 1 \\
  2x-y &= -3 \\
  x &= 1-2y \\
  2(1-2y)-y &= -3 \\
  2-4y-y &= -3 \\
  2-5y &= -3 \\
  -5y &= -5 \\
  y &= 1 \\
  x &= -1
\end{align*}
Let \( y = 0 \):
\begin{align*}
  x+3z &= 1 \\
  2x+z &= -3 \\
  x &= 1-3z \\
  2(1-3z)+z &= -3 \\
  2-6z+z &= -3 \\
  -5z &= -5 \\
  z &= 1 \\
  x &= -2
\end{align*}
Points (-1,1,0) and (-2,0,1):
\begin{align*}
  \vec{u} &= (3,1,4)-(-1,1,0) = \langle4,0,4\rangle \\
  \vec{v} &= (3,1,3)-(-2,0,1) = \langle5,1,2\rangle \\
  \vec{n} &= \vec{u}\times\vec{v} \\
  &= \begin{vmatrix}
    \i & \j & \k \\
    4 & 0 & 4 \\
    5 & 1 & 2 \\
  \end{vmatrix} = \langle-4,12,4\rangle \\
  \Pi &\equiv -4(x-(-1))+12(y-1)+4(z-0) = 0 \\
  &\equiv -4x-4+12y-12+4z = 0 \\
  &\equiv -4x+12y+4z = 16 \\
  &\equiv -x+3y+z = 4 \\
  &\equiv x-3y-z = -4
\end{align*}

\subsubsection*{69}
Use the formula to find the distance from the point to the given line.
P = (4,1,-2)
\[ l = \begin{cases}
  x &= 1+t \\
  y &= 3-2t \\
  z &= 4-3t
\end{cases} \]
\begin{align*}
  \vec{u} &= \langle1,-2,-3\rangle \\
  P_0 &= (1,3,4) \\
  \vec{v} &= \overrightarrow{P_0P} = \langle3,-2,-6\rangle \\
  d &= \frac{|\vec{u}\times\vec{v}|}{|\vec{u}|} \\
  &= \left|\begin{vmatrix}
    \i & \j & \k \\
    1 & -2 & -3 \\
    3 & -2 & -6
  \end{vmatrix}\right|\left(\frac{1}{|\langle1,-2,-3\rangle|}\right) \\
  &= \frac{|\langle6,-3,4\rangle|}{\sqrt{1+4+9}} \\
  &= \frac{\sqrt{36+9+16}}{\sqrt{14}} \\
  &= \sqrt{\frac{61}{14}}
\end{align*}

\subsubsection*{71}
Find the distance from the point to the given plane. \( P_1 = (1,-2,4) \)
\[ 3x+2y+6z = 5 \]
\begin{align*}
  D &= \frac{|ax_1+by_1+cz_1+d|}{\sqrt{a^2+b^2+c^2}} \\
  &= \frac{|3(1)+2(-2)+6(4)+(-5)|}{\sqrt{9+4+36}} \\
  &= \frac{|3-4+24-5|}{\sqrt{49}} \\
  &= \frac{18}{7}
\end{align*}

\subsubsection*{73}
Find the distance between the given parallel planes.
\[ 2x-3y+z = 4 \quad 4x-6y+2z = 3 \]
\begin{align*}
  P_0 &= \langle0,0,4\rangle \\
  D &= \frac{|ax_1+by_1+cz_1+d|}{\sqrt{a^2+b^2+c^2}} \\
  &= \frac{|4(0)-6(0)+2(4)-3|}{\sqrt{16+36+4}} \\
  &= \frac{|8-3|}{\sqrt{56}} \\
  &= \frac{5}{4\sqrt{14}}
\end{align*}

\subsubsection*{77}
Show that the lines with symmetric equations \( x = y = z \) and
\( x+1 = \frac{y}{2} = \frac{z}{3} \) are skew, and find the distance
between these lines.
\begin{align*}
  l_1 &= \begin{cases}
    x &= t \\
    y &= t \\
    z &= t
  \end{cases} \\
  l_2 &= \begin{cases}
    x &= -1+s \\
    y &= 2s \\
    z &= 3s
  \end{cases} \\
  t &= s-1 \\
  s-1 &= 2s \\
  s &= -1 \\
  t &\stackrel{?}{=} 3s \\
  s-1 &= 3s \\
  s &= \frac{-1}{2}
\end{align*}
Since the solutions do not match, the lines are skew.
\begin{align*}
  \vec{v_1} &= \langle1,1,1\rangle \\
  \vec{v_2} &= \langle1,2,3\rangle \\
  \vec{n} &= \vec{v_1}\times\vec{v_2} \\
  &= \begin{vmatrix}
    \i & \j & \k \\
    1 & 1 & 1 \\
    1 & 2 & 3
  \end{vmatrix} = \langle1,-2,1\rangle \\
  P_0 &= (0,0,0) \\
  \Pi &= 1x-2y+1z = 0 \\
  P_1 &= (-1,0,0) \\
  D &= \frac{|ax_1+by_1+cz_1+d|}{\sqrt{a^2+b^2+c^2}} \\
  &= \frac{|1(-1)+0+0+0|}{\sqrt{1+4+1}} \\
  &= \frac{1}{\sqrt{6}}
\end{align*}

\begin{center}
  If you have any questions, comments, or concerns, please contact me at
  alvin@omgimanerd.tech
\end{center}

\end{document}
