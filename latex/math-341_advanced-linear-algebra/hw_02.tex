\documentclass{math}

\usepackage{enumerate}

\geometry{letterpaper, margin=0.5in}

\title{Advanced Linear Algebra: Homework 2}
\author{Alvin Lin}
\date{August 2016 - December 2016}

\begin{document}

\maketitle

\subsubsection*{Question 1}
Show that \( \vec{v} \) is an eigenvector of \( A \) and find the corresponding
eigenvalue, \( \lambda \).
\begin{align*}
  A &= \begin{bmatrix}
    1 & 2 \\
    2 & 1
  \end{bmatrix} \quad \vec{v} = \begin{bmatrix}6 \\ -6\end{bmatrix} \\
  A\vec{v} &= \lambda\vec{v} \\
  \begin{bmatrix}
    1 & 2 \\
    2 & 1
  \end{bmatrix}\begin{bmatrix}6 \\ -6\end{bmatrix} &= \lambda\begin{bmatrix}
    6 \\ -6
  \end{bmatrix} \\
  \begin{bmatrix}
    -6 \\ 6
  \end{bmatrix} &= \lambda\begin{bmatrix}
    6 \\ -6
  \end{bmatrix} \\
  \lambda &= -1
\end{align*}

\subsubsection*{Question 2}
Show that \( \vec{v} \) is an eigenvector of \( A \) and find the corresponding
eigenvalue, \( \lambda \).
\begin{align*}
  A &= \begin{bmatrix}
    0 & 1 & -1 \\
    1 & 1 & 3 \\
    1 & 4 & 0
  \end{bmatrix} \quad \vec{v} = \begin{bmatrix}
    -4 \\ 1 \\ 1
  \end{bmatrix} \\
  A\vec{v} &= \lambda\vec{v} \\
  \begin{bmatrix}
    0 & 1 & -1 \\
    1 & 1 & 3 \\
    1 & 4 & 0
  \end{bmatrix}\begin{bmatrix}
    -4 \\ 1 \\ 1
  \end{bmatrix} &= \lambda\begin{bmatrix}
    -4 \\ 1 \\ 1
  \end{bmatrix} \\
  \begin{bmatrix}
    0 \\ 0 \\ 0
  \end{bmatrix} &= \lambda\begin{bmatrix}
    -4 & 1 & 1
  \end{bmatrix} \\
  \lambda &= 0
\end{align*}

\subsubsection*{Question 3}
Show that \( \lambda \) is an eigenvalue of \( A \) and find one eigenvector
\( \vec{v} \) corresponding to this eigenvalue.
\begin{align*}
  A &= \begin{bmatrix}
    4 & 3 \\
    2 & -1
  \end{bmatrix} \quad \lambda = 5 \\
  A\vec{v} &= \lambda\vec{v} \\
  (A-\lambda I)\vec{v} &= 0 \\
  \begin{bmatrix}
    4-5 & 3 \\
    2 & -1-5
  \end{bmatrix}\vec{v} &= 0 \\
  \begin{bmatrix}
    -1 & 3 & 0 \\
    2 & -6 & 0
  \end{bmatrix} &\to \begin{bmatrix}
    1 & -3 & 0 \\
    0 & 0 & 0
  \end{bmatrix} \\
  x_1 &= 3x_3 = 3s \\
  \vec{x} &= s\begin{bmatrix}3 \\ 1\end{bmatrix}
\end{align*}

\subsubsection*{Question 4}
Show that \( \lambda \) is an eigenvalue of \( A \) and find one eigenvector
\( \vec{v} \) corresponding to this eigenvalue.
\begin{align*}
  A &= \begin{bmatrix}
    6 & 1 & -1 \\
    1 & 4 & 1 \\
    4 & 2 & 3
  \end{bmatrix} \quad \lambda = 5 \\
  A\vec{v} &= \lambda\vec{v} \\
  (A-\lambda I)\vec{v} &= 0 \\
  \begin{bmatrix}
    6-5 & 1 & -1 \\
    1 & 4-5 & 1 \\
    4 & 2 & 3-5
  \end{bmatrix}\vec{v} &= 0 \\
  \begin{bmatrix}
    -1 & 1 & -1 & 0 \\
    1 & -1 & 1 & 0 \\
    4 & 2 & -2 & 0
  \end{bmatrix} &\to \begin{bmatrix}
    1 & 0 & 0 & 0 \\
    0 & 1 & -1 & 0 \\
    0 & 0 & 0 & 0
  \end{bmatrix} \\
  x_1 &= 0 \quad x_2-x_3 = 0 \\
  x_2 &= x_3 = s \\
  \vec{v} &= s\begin{bmatrix}0 \\ 1 \\ 1\end{bmatrix}
\end{align*}

\subsubsection*{Question 5}
Use the method of Example 4.5 to find all of the eigenvalues \( \lambda \) of
the matrix \( A \).
\begin{align*}
  A &= \begin{bmatrix}
    2 & 7 \\
    9 & 0
  \end{bmatrix} \\
  \det(A-\lambda I) &= 0 \\
  \begin{vmatrix}
    2-\lambda & 7 \\
    9 & -\lambda
  \end{vmatrix} &= 0 \\
  (2-\lambda)(-\lambda)-63 &= 0 \\
  -2\lambda+\lambda^2-63 &= 0 \\
  \lambda^2-2\lambda-63 &= 0 \\
  (\lambda-9)(\lambda+7) &= 0 \\
  \lambda_1 &= 9 \quad \lambda_2 = -7
\end{align*}
Give bases for each of the corresponding eigenspaces.
\begin{align*}
  (A+7I)\vec{v} &= 0 \\
  \begin{bmatrix}
    9 & 7 & 0 \\
    9 & 7 & 0
  \end{bmatrix} &\to \begin{bmatrix}
    9 & 7 & 0 \\
    0 & 0 & 0
  \end{bmatrix} \\
  9x_1+7x_2 &= 0 \\
  x_2 &= -\frac{9}{7}x_1 \\
  \vec{v} &= s\begin{bmatrix}1 \\ -\frac{9}{7}\end{bmatrix} \\
  (A-9I)\vec{v} &= 0 \\
  \begin{bmatrix}
    -7 & 7 & 0 \\
    9 & -9 & 0
  \end{bmatrix} &\to \begin{bmatrix}
    1 & -1 & 0 \\
    0 & 0 & 0
  \end{bmatrix} \\
  x_1 &= x_2 = s \\
  \vec{v} &= s\begin{bmatrix}1 \\ 1\end{bmatrix}
\end{align*}

\subsubsection*{Question 6}
Find all of the eigenvalues of the matrix \( A \) over the complex numbers
\( \mathbb{C} \). Give bases for each of the corresponding eigenspaces.
\begin{align*}
  A &= \begin{bmatrix}
    2 & -1 \\
    1 & 2
  \end{bmatrix} \\
  \det(A-\lambda I) &= 0 \\
  \begin{vmatrix}
    2-\lambda & -1 \\
    1 & 2-\lambda
  \end{vmatrix} &= 0 \\
  (2-\lambda)(2-\lambda)+1 &= 0 \\
  4-4\lambda+\lambda^2+1 &= 0 \\
  \lambda^2-4\lambda-5 &= 0 \\
  \lambda_1 &= 2+i \quad \lambda_2 = 2-i \\
\end{align*}
\begin{align*}
  (A-\lambda_1 I)\vec{v} &= 0 \\
  \begin{bmatrix}
    2-(2+i) & -1 & 0 \\
    1 & 2-(2+i) & 0
  \end{bmatrix} &\to \begin{bmatrix}
    1 & -i & 0 \\
    0 & 0 & 0
  \end{bmatrix} \\
  x_1-ix_2 &= 0 \\
  x_1 &= ix_2 = is \\
  \vec{v} &= s\begin{bmatrix}i \\ 1\end{bmatrix}
\end{align*}
\begin{align*}
  (A-\lambda_2 I)\vec{v} &= 0 \\
  \begin{bmatrix}
    2-(2-i) & -1 & 0 \\
    1 & 2-(2-i) & 0
  \end{bmatrix} &\to \begin{bmatrix}
    1 & i & 0 \\
    0 & 0 & 0
  \end{bmatrix} \\
  x_1+ix_2 &= 0 \\
  x_1 &= -ix_2 = -is \\
  \vec{v} &= s\begin{bmatrix}-i \\ 1\end{bmatrix}
\end{align*}

\subsubsection*{Question 7}
Consider matrix \( A \).
\[ A = \begin{bmatrix}a & b \\ c & d\end{bmatrix} \]
Give conditions on \( a,b,c,d \) such that \( A \) has the following.
\begin{align*}
  \det(A-\lambda I) &= 0 \\
  \begin{vmatrix}
    a-\lambda & b \\
    c & d-\lambda
  \end{vmatrix} &= 0 \\
  (a-\lambda)(d-\lambda)-bc &= 0 \\
  ad-a\lambda-d\lambda+\lambda^2-bc &= 0 \\
  \lambda^2+(-a-d)\lambda+(ad-bc) &= 0 \\
  D &= B^2-4AC \\
  &= (-a-d)^2-4(1)(ad-bc) \\
  &= a^2+2ad+d^2-4ad+4bc \\
  &= a^2+d^2-2ad+4bc
\end{align*}
\begin{enumerate}[(a)]
  \item two distinct real eigenvalues
  \[ a^2+d^2-2ad+4bc > 0 \]
  \item one real eigenvalue
  \[ a^2+d^2-2ad+4bc = 0 \]
  \item no real eigenvalues
  \[ a^2+d^2-2ad+4bc < 0 \]
\end{enumerate}

\subsubsection*{Question 8}
Compute the determinant using cofactor expansion along the first row and along
the first column.
\begin{align*}
  \begin{vmatrix}
    1 & 0 & 2 \\
    4 & 1 & 1 \\
    0 & 1 & 3
  \end{vmatrix} &= 1\begin{vmatrix}
    1 & 1 \\
    1 & 3
  \end{vmatrix}-0+2\begin{vmatrix}
    4 & 1 \\
    0 & 1
  \end{vmatrix} \\
  &= 1(3-1)+2(4-0) \\
  &= 2+8 = 10 \\
  &= 1\begin{vmatrix}
    1 & 1 \\
    1 & 3
  \end{vmatrix}-4\begin{vmatrix}
    0 & 2 \\
    1 & 3
  \end{vmatrix}+0 \\
  &= 1(3-1)-4(0-2) \\
  &= 2-(-8) = 10
\end{align*}

\subsubsection*{Question 9}
Compute the determinant using cofactor expansion along any row or column that
seems convenient.
\begin{align*}
  \begin{vmatrix}
    \tan\theta & \sin\theta & \cos\theta \\
    \cos\theta & 0 & -\sin\theta \\
    \sin\theta & 0 & \cos\theta
  \end{vmatrix} &= -\sin\theta\begin{vmatrix}
    \cos\theta & -\sin\theta \\
    \sin\theta & \cos\theta
  \end{vmatrix}+0-0 \\
  &= -\sin\theta(\cos^2\theta+\sin^2\theta) \\
  &= -\sin\theta
\end{align*}

\subsubsection*{Question 10}
Compute the determinant using cofactor expansion along any row or column that
seems convenient.
\begin{align*}
  \begin{vmatrix}
    1 & -1 & 0 & 4 \\
    2 & 5 & 2 & 7 \\
    0 & 1 & 0 & 0 \\
    1 & 6 & 2 & 1
  \end{vmatrix} &= 0-1\begin{vmatrix}
    1 & 0 & 4 \\
    2 & 2 & 7 \\
    1 & 2 & 1
  \end{vmatrix}+0-0 \\
  &= -\left(0+2\begin{vmatrix}
    1 & 4 \\
    1 & 1
  \end{vmatrix}-2\begin{vmatrix}
    1 & 4 \\
    2 & 7
  \end{vmatrix}\right) \\
  &= -(2(1-4)-2(7-8)) \\
  &= -(-6+2) \\
  &= 4
\end{align*}

\subsubsection*{Question 11}
Compute the determinant using cofactor expansion along any row or column that
seems convenient.
\begin{align*}
  \begin{vmatrix}
    0 & 0 & 0 & a \\
    0 & 0 & b & c \\
    0 & d & e & f \\
    g & h & i & j
  \end{vmatrix} &= 0-0+0-a\begin{vmatrix}
    0 & 0 & b \\
    0 & d & e \\
    g & h & i
  \end{vmatrix} \\
  &= -a\left(0-0+b\begin{vmatrix}
    0 & d \\
    g & h
  \end{vmatrix}\right) \\
  &= -a(b(0-dg)) \\
  &= abdg
\end{align*}

\subsubsection*{Question 12}
Let \( A = [a_{ij}] \) be a square matrix.
\begin{enumerate}[a.]
  \item If \( A \) has a zero row (column), then \( \det(A) = 0 \).
  \item If \( B \) is obtained by interchanging two rows (columns) of \( A \),
    then \( \det(B) = -\det(A) \).
  \item If \( A \) has two identical rows (columns), then \( \det(A) = 0 \).
  \item If \( B \) is obtained by multiplying a row (column) of \( A \) by
    \( k \), then \( \det(B) = k\det(A) \).
  \item If \( A,B,C \) are identical except that the \( i\text{th} \) row
    (column) of \( C \) is the sum of the \( i\text{th} \) rows (columns) of
    \( A \) and \( B \), then \( \det(C) = \det(A)+\det(B) \).
  \item If \( B \) is obtained by adding a multiple of one row (column) of
    \( A \) to another row (column), then \( \det(B) = \det(A) \).
\end{enumerate}
Evaluate the given determinant using elementary row and/or column operations and
the theorem above to reduce the matrix to row echelon form.
\begin{align*}
  A &= \begin{vmatrix}
    1 & 0 & 3 \\
    5 & 1 & 1 \\
    0 & 1 & 4
  \end{vmatrix} \\
  &= \begin{vmatrix}
    1 & 0 & 3 \\
    0 & 1 & -14 \\
    0 & 1 & 4
  \end{vmatrix} \quad (R_2-5R_1)(f) \\
  &= \begin{vmatrix}
    1 & 0 & 3 \\
    0 & 1 & -14 \\
    0 & 0 & 18
  \end{vmatrix} \quad (R_3-R_2)(f) \\
  \det(A) &= 1(1(18)) = 18
\end{align*}

\subsubsection*{Question 13}
Let \( A = [a_{ij}] \) be a square matrix.
\begin{enumerate}[a.]
  \item If \( A \) has a zero row (column), then \( \det(A) = 0 \).
  \item If \( B \) is obtained by interchanging two rows (columns) of \( A \),
    then \( \det(B) = -\det(A) \).
  \item If \( A \) has two identical rows (columns), then \( \det(A) = 0 \).
  \item If \( B \) is obtained by multiplying a row (column) of \( A \) by
    \( k \), then \( \det(B) = k\det(A) \).
  \item If \( A,B,C \) are identical except that the \( i\text{th} \) row
    (column) of \( C \) is the sum of the \( i\text{th} \) rows (columns) of
    \( A \) and \( B \), then \( \det(C) = \det(A)+\det(B) \).
  \item If \( B \) is obtained by adding a multiple of one row (column) of
    \( A \) to another row (column), then \( \det(B) = \det(A) \).
\end{enumerate}
Evaluate the given determinant using elementary row and/or column operations and
the theorem above to reduce the matrix to row echelon form.
\begin{align*}
  A &= \begin{vmatrix}
    1 & -1 & 0 & 3 \\
    2 & 6 & 2 & 6 \\
    0 & 1 & 0 & 0 \\
    1 & 4 & 2 & 1
  \end{vmatrix} \\
  &= \begin{vmatrix}
    1 & 0 & 0 & 3 \\
    2 & 0 & 2 & 6 \\
    0 & 1 & 0 & 0 \\
    1 & 0 & 2 & 1
  \end{vmatrix} \quad (R_1+R_3,R_2-6R_3,R_4-4R_3)(f) \\
  &= \begin{vmatrix}
    1 & 0 & 0 & 3 \\
    2 & 0 & 2 & 6 \\
    0 & 1 & 0 & 0 \\
    0 & 0 & 2 & -2
  \end{vmatrix} \quad (R_4-R_1)(f) \\
  &= \begin{vmatrix}
    1 & 0 & 0 & 3 \\
    2 & 0 & 0 & 8 \\
    0 & 1 & 0 & 0 \\
    0 & 0 & 2 & -2
  \end{vmatrix} \quad (R_2-R_4)(f) \\
  &= \begin{vmatrix}
    1 & 0 & 0 & 3 \\
    0 & 0 & 0 & 2 \\
    0 & 1 & 0 & 0 \\
    0 & 0 & 2 & -2
  \end{vmatrix} \quad (R_2-2R_1)(f) \\
  &= \begin{vmatrix}
    1 & 0 & 0 & 3 \\
    0 & 1 & 0 & 0 \\
    0 & 0 & 2 & -2 \\
    0 & 0 & 0 & 2
  \end{vmatrix} \quad (R_2\leftrightarrow R_3,R_3\leftrightarrow R_4)(b) \\
  &= 1(1(1(-1(-1(4))))) = 4
\end{align*}

\subsubsection*{Question 14}
A square matrix \( A \) is invertible if and only if \( \det(A)\ne0 \). Use the
theorem to find all values of \( k \) for which \( A \) is invertible.
\begin{align*}
  A &= \begin{bmatrix}
    k & -k & 3 \\
    0 & k+1 & 1 \\
    k & -8 & k-1
  \end{bmatrix} \\
  \det(A) &\ne 0 \\
  &= 0+(k+1)\begin{vmatrix}
    k & 3 \\
    k & k-1
  \end{vmatrix}-1\begin{vmatrix}
    k & -k \\
    k & -8
  \end{vmatrix} \\
  &= (k+1)(k(k-1)-3k)-1(-8k+k^2) \\
  &= (k+1)(k^2-k-3k)+8k-k^2 \\
  &= k^3-4k^2+k^2-4k+8k-k^2 \\
  &= k^3-4k^2+4k \\
  &= k(k^2-4k+4) \\
  &= k(k-2)(k-2) \\
  k &\ne 0,2
\end{align*}

\subsubsection*{Question 15}
Assume that \( A \) and \( B \) are \( n\times n \) matrices with
\( \det(A) = 3 \) and \( \det(B) = -4 \). Find the indicated determinant.
\[ \det(AB) = \det(A)\det(B) = -12 \]

\subsubsection*{Question 16}
Assume that \( A \) and \( B \) are \( n\times n \) matrices with
\( \det(A) = 5 \) and \( \det(B) = -2 \). Find the indicated determinant.
\[ \det(A^2) = \det(AA) = \det(A)\det(A) = 25 \]

\subsubsection*{Question 17}
Assume that \( A \) and \( B \) are \( n\times n \) matrices with
\( \det(A) = 5 \) and \( \det(B) = -4 \). Find the indicated determinant.
\[ \det(B^{-1}A) = \frac{1}{\det(B)}\det(A) = \frac{1}{-4}5 = -\frac{5}{4} \]

\subsubsection*{Question 18}
Assume that \( A \) and \( B \) are \( n\times n \) matrices with
\( \det(A) = 3 \) and \( \det(B) = -2 \). Find the indicated determinant.
\[ \det(5B^T) = 5^n\det(B^T) = 5^n(-2) \]

\subsubsection*{Question 19}
\( A \) and \( B \) are \( n\times n \) matrices. If \( A \) is idempotent
(that is, \( A^2 = A \)), find all possible values of \( \det(A) \).
\begin{align*}
  A^2 &= A \\
  \det(A^2) &= \det(A) \\
  \det(AA) &= \det(A) \\
  \det(A)^2 &= \det(A) \\
  \det(A)^2-\det(A) &= 0 \\
  \det(A)(\det(A)-1) &= 0 \\
  \det(A) &= 0,1
\end{align*}

\subsubsection*{Question 20}
\( A \) and \( B \) are \( n\times n \) matrices. A square matrix \( A \) is
called \textbf{nilpotent} if \( A^m = 0 \) for some \( m>1 \). Find all possible
values of \( \det(A) \) if \( A \) is nilpotent.
\begin{align*}
  A^m &= 0 \\
  \det(A^m) &= 0 \\
  \det(AA^{m-1}) &= 0 \\
  \det(A)\det(A^{m-1}) &= 0 \\
  \det(A) &= 0
\end{align*}
No other possible values can arise.

\subsubsection*{Question 21}
Consider the following.
\[ A = \begin{bmatrix}1 & 5 \\ -2 & 8\end{bmatrix} \]
\begin{enumerate}[(a)]
  \item Compute the characteristic polynomial of \( A \).
  \begin{align*}
    \det(A-\lambda I) &= \begin{vmatrix}
      1-\lambda & 5 \\
      -2 & 8-\lambda
    \end{vmatrix} \\
    &= (1-\lambda)(8-\lambda)-(-10) \\
    &= 8-9\lambda+\lambda^2+10 \\
    &= \lambda^2-9\lambda+18
  \end{align*}
  \item Compute the eigenvalues and bases of the corresponding eigenspaces in
    \( A \).
  \begin{align*}
    \lambda^2-9\lambda+18 &= 0 \\
    (\lambda-6)(\lambda-3) &= 0 \\
    \lambda_1 &= 6 \quad \lambda_2 = 3 \\
    (A-6I)\vec{v_1} &= 0 \\
    \begin{bmatrix}
      -5 & 5 \\
      -2 & 3
    \end{bmatrix}\vec{v_1} &= 0 \\
    \begin{bmatrix}
      -5 & 5 & 0 \\
      -2 & 2 & 0
    \end{bmatrix} &\to \begin{bmatrix}
      1 & -1 & 0 \\
      0 & 0 & 0
    \end{bmatrix} \\
    x_1 &= x_2 = s \\
    \vec{v_1} &= s\begin{bmatrix}1 \\ 1\end{bmatrix} \\
    (A-3I)\vec{v_2} &= 0 \\
    \begin{bmatrix}
      -2 & 5 \\
      -2 & 5
    \end{bmatrix}\vec{v_2} &= 0 \\
    \begin{bmatrix}
      -2 & 5 & 0 \\
      -2 & 5 & 0
    \end{bmatrix} &\to \begin{bmatrix}
      1 & -\frac{5}{2} & 0 \\
      0 & 0 & 0
    \end{bmatrix} \\
    x_1 &= \frac{5}{2}x_2 = \frac{5}{2}s \\
    \vec{v_2} &= s\begin{bmatrix}\frac{5}{2} \\ 1\end{bmatrix}
  \end{align*}
\end{enumerate}

\subsubsection*{Question 22}
Consider the following.
\[ A = \begin{bmatrix}
  1 & 1 & 0 \\
  0 & -4 & 1 \\
  0 & 0 & 5
\end{bmatrix} \]
\begin{enumerate}[(a)]
  \item Compute the characteristic polynomial of \( A \).
  \begin{align*}
    \det(A-\lambda I) &= \begin{vmatrix}
      1-\lambda & 1 & 0 \\
      0 & -4-\lambda & 1 \\
      0 & 0 & 5-\lambda
    \end{vmatrix} \\
    &= (1-\lambda)(-4-\lambda)(5-\lambda)
  \end{align*}
  \item Compute the eigenvalues and bases of the corresponding eigenspaces of
    \( A \).
  \begin{align*}
    \lambda_1 &= 1 \\
    (A-1I)\vec{v_1} &= 0 \\
    \begin{bmatrix}
      0 & 1 & 0 & 0 \\
      0 & -5 & 1 & 0 \\
      0 & 0 & 4 & 0
    \end{bmatrix} &\to \begin{bmatrix}
      0 & 1 & 0 & 0 \\
      0 & 0 & 1 & 0 \\
      0 & 0 & 0 & 0
    \end{bmatrix} \\
    x_1 &= s \quad x_2 = 0 \quad x_3 = 0 \\
    \vec{v_1} &= s\begin{bmatrix}1 \\ 0 \\ 0\end{bmatrix} \\
    \lambda_2 &= -4 \\
    (A+4I)\vec{v_2} &= 0 \\
    \begin{bmatrix}
      5 & 1 & 0 & 0 \\
      0 & 0 & 1 & 0 \\
      0 & 0 & 9 & 0
    \end{bmatrix} &\to \begin{bmatrix}
      1 & \frac{1}{5} & 0 & 0 \\
      0 & 0 & 1 & 0 \\
      0 & 0 & 0 & 0
    \end{bmatrix} \\
    x_1 &= -\frac{1}{5}x_2 = -\frac{1}{5}s \quad x_3 = 0 \\
    \vec{v_2} &= s\begin{bmatrix}-\frac{1}{5} \\ 1 \\ 0\end{bmatrix} \\
    \lambda_3 &= 5 \\
    (A-5I)\vec{v_3} &= 0 \\
    \begin{bmatrix}
      -4 & 1 & 0 & 0 \\
      0 & -9 & 1 & 0 \\
      0 & 0 & 0 & 0
    \end{bmatrix} &\to \begin{bmatrix}
      1 & 0 & -\frac{1}{36} & 0 \\
      0 & 1 & -\frac{1}{9} & 0 \\
      0 & 0 & 0 & 0
    \end{bmatrix} \\
    x_1 &= \frac{1}{36}x_3 \quad x_2 = \frac{1}{9}x_3 \quad x_3 = s \\
    \vec{v_3} &= s\begin{bmatrix}\frac{1}{36} \\ \frac{1}{9} \\ 1\end{bmatrix}
  \end{align*}
\end{enumerate}

\subsubsection*{Question 23}
Consider the following.
\[ A = \begin{bmatrix}
  -3 & 18 & 0 \\
  -1 & 7 & 1 \\
  0 & 1 & 1
\end{bmatrix} \]
\begin{enumerate}[(a)]
  \item Compute the characteristic polynomial of \( A \).
  \begin{align*}
    \det(A-\lambda I) &= \begin{vmatrix}
      -3-\lambda & 18 & 0 \\
      -1 & 7-\lambda & 1 \\
      0 & 1 & 1-\lambda
    \end{vmatrix} \\
    &= 0-1\begin{vmatrix}
      -3-\lambda & 0 \\
      -1 & 1
    \end{vmatrix}+(1-\lambda)\begin{vmatrix}
      -3-\lambda & 18 \\
      -1 & 7-\lambda
    \end{vmatrix} \\
    &= -1(-3-\lambda)+(1-\lambda)((-3-\lambda)(7-\lambda)+18) \\
    &= 3+\lambda+(1-\lambda)(-21-4\lambda+\lambda^2+18) \\
    &= 3+\lambda+(1-\lambda)(\lambda^2-4\lambda-3) \\
    &= 3+\lambda+(\lambda^2-4\lambda-3-\lambda^3+4\lambda^2+3\lambda) \\
    &= -\lambda^3+5\lambda^2 \\
    &= -\lambda^2(\lambda-5)
  \end{align*}
  \item Compute the eigenvalues and bases of the corresponding eigenspaces of
    \( A \).
  \begin{align*}
    \lambda_1 &= \lambda_2 = 0 \\
    (A-0)\vec{v_1} &= 0 \\
    \begin{bmatrix}
      -3 & 18 & 0 & 0 \\
      -1 & 7 & 1 & 0 \\
      0 & 1 & 1 & 0
    \end{bmatrix} &\to \begin{bmatrix}
      1 & 0 & 6 & 0 \\
      0 & 1 & 1 & 0 \\
      0 & 0 & 0 & 0
    \end{bmatrix} \\
    x_1 &= -6x_3 \quad x_2 = -x_3 \quad x_3 = s \\
    \vec{v_1} &= \vec{v_2} = s\begin{bmatrix}-6 \\ -1 \\ 1\end{bmatrix} \\
    (A-5I)\vec{v_3} &= 0 \\
    \begin{bmatrix}
      -8 & 18 & 0 & 0 \\
      -1 & 2 & 1 & 0 \\
      0 & 1 & -4 & 0
    \end{bmatrix} &\to \begin{bmatrix}
      1 & 0 & -9 & 0 \\
      0 & 1 & -4 & 0 \\
      0 & 0 & 0 & 0
    \end{bmatrix} \\
    x_1 &= 9x_3 \quad x_2 = 4x_3 \quad x_3 = s \\
    \vec{v_3} &= s\begin{bmatrix}9 \\ 4 \\ 1\end{bmatrix}
  \end{align*}
\end{enumerate}

\subsubsection*{Question 24}
Let \( p(x) \) be the polynomial
\[ p(x) = x^n+a_{n-1}x^{n-1}+\dots+a_{1}x+a_0 \]
The \textbf{companion matrix} of \( p(x) \) is the \( n\times n \) matrix
\[ C(p) = \begin{bmatrix}
  -a_{n-1} & -a_{n-2} & \dots & -a_1 & -a_0 \\
  1 & 0 & \dots & 0 & 0 \\
  0 & 1 & \dots & \vdots & \vdots \\
  0 & 0 & \dots & 0 & 0 \\
  0 & 0 & \dots & 1 & 0
\end{bmatrix} \]
Find the companion matrix of \( p(x) = x^3+2x^2-4x+10 \) and then find the
characteristic polynomial of \( C(p) \).
\begin{align*}
  C(p) &= \begin{bmatrix}
    -2 & 4 & -10 \\
    1 & 0 & 0 \\
    0 & 1 & 0
  \end{bmatrix} \\
  \det(C(p)-\lambda I) &= \begin{vmatrix}
    -2-\lambda & 4 & -10 \\
    1 & -\lambda & 0 \\
    0 & 1 & -\lambda
  \end{vmatrix} \\
  &= 0-1\begin{vmatrix}
    -2-\lambda & -10 \\
    1 & 0
  \end{vmatrix}-\lambda\begin{vmatrix}
    -2-\lambda & 4 \\
    1 & -\lambda
  \end{vmatrix} \\
  &= -1(0+10)-\lambda(2\lambda+\lambda^2-4) \\
  &= -10-2\lambda^2-\lambda^3+4\lambda
\end{align*}

\subsubsection*{Question 25}
For the matrix \( A \), use the Cayley-Hamilton Theorem to express \( A^3 \) and
\( A^4 \) as a linear combination of \( I,A,A^2 \).
\begin{align*}
  A &= \begin{bmatrix}
    1 & 1 & 0 \\
    1 & 0 & 1 \\
    0 & 1 & 1
  \end{bmatrix} \\
  c_A(\lambda) &= \begin{vmatrix}
    1-\lambda & 1 & 0 \\
    1 & -\lambda & 1 \\
    0 & 1 & 1-\lambda
  \end{vmatrix} \\
  &= (1-\lambda)\begin{vmatrix}
    -\lambda & 1 \\
    1 & 1-\lambda
  \end{vmatrix}-1\begin{vmatrix}
    1 & 1 \\
    0 & 1-\lambda
  \end{vmatrix}+0 \\
  &= (1-\lambda)(-\lambda+\lambda^2-1)-1(1-\lambda) \\
  &= -\lambda+\lambda^2-1+\lambda^2-\lambda^3+\lambda-1+\lambda \\
  &= -\lambda^3+2\lambda^2+\lambda-2 \\
  -A^3+2A^2+A-2I &= 0 \\
  A^3 &= 2A^2+A-2I \\
  A^4 &= AA^3 \\
  &= A(2A^2+A-2I) \\
  &= 2A^3+A^2-2A \\
  &= 2(2A^2+A-2I)+A^2-2A \\
  &= 4A^2+2A-4I+A^2-2A \\
  &= 5A^2-4I
\end{align*}

\subsubsection*{Question 26}
Show that \( A \) and \( B \) are not similar matrices.
\begin{align*}
  A &= \begin{bmatrix}
    1 & 5 & 0 \\
    0 & 1 & -1 \\
    0 & -1 & 1
  \end{bmatrix} \quad B = \begin{bmatrix}
    5 & 1 & 1 \\
    0 & 1 & 0 \\
    5 & 0 & 1
  \end{bmatrix} \\
  c_A(\lambda) &= \begin{vmatrix}
    1-\lambda & 5 & 0 \\
    0 & 1-\lambda & -1 \\
    0 & -1 & 1-\lambda
  \end{vmatrix} \\
  &= (1-\lambda)\begin{vmatrix}
    1-\lambda & -1 \\
    -1 & 1-\lambda
  \end{vmatrix}-0+0 \\
  &= (1-\lambda)((1-\lambda)^2-1) \\
  &= (1-\lambda)(1-2\lambda+\lambda^2-1) \\
  &= \lambda(1-\lambda)(\lambda-2) \\
  c_B(\lambda) &= \begin{vmatrix}
    5-\lambda & 1 & 1 \\
    0 & 1-\lambda & 0 \\
    5 & 0 & 1-\lambda
  \end{vmatrix} \\
  &= (5-\lambda)\begin{vmatrix}
    1-\lambda & 0 \\
    0 & 1-\lambda
  \end{vmatrix}-0+5\begin{vmatrix}
    1 & 1 \\
    1-\lambda & 0
  \end{vmatrix} \\
  &= (5-\lambda)(1-2\lambda+\lambda^2)+5(0-(1-\lambda)) \\
  &= 5-10\lambda+5\lambda^2-\lambda+2\lambda^2-\lambda^3-5+5\lambda \\
  &= -\lambda^3+7\lambda^2-6\lambda \\
  &= -\lambda(\lambda-6)(\lambda-1)
\end{align*}

\subsubsection*{Question 27}
A diagonalization of the matrix \( A \) is given in the form \( P^{-1}AP = D \).
List the eigenvalues of \( A \) and bases for the corresponding eigenspaces.
\begin{align*}
  \begin{bmatrix}
    2 & -1 \\
    -1 & 1
  \end{bmatrix}\begin{bmatrix}
    3 & 1 \\
    -2 & 6
  \end{bmatrix}\begin{bmatrix}
    1 & 1 \\
    1 & 2
  \end{bmatrix} &= \begin{bmatrix}
    4 & 0 \\
    0 & 5
  \end{bmatrix} \\
  \lambda_1 &= 4 \quad \vec{v_1} = \begin{bmatrix}1 \\ 1\end{bmatrix} \\
  \lambda_2 &= 5 \quad \vec{v_1} = \begin{bmatrix}1 \\ 2\end{bmatrix}
\end{align*}

\subsubsection*{Question 28}
A diagonalization of the matrix \( A \) is given in the form \( P^{-1}AP = D \).
List the eigenvalues of \( A \) and bases fo the corresponding eigenspaces.
\begin{align*}
  \begin{bmatrix}
    \frac{1}{14} & \frac{1}{14} & \frac{1}{14} \\
    \frac{1}{2} & -\frac{1}{2} & -\frac{1}{2} \\
    \frac{3}{7} & \frac{3}{7} & -\frac{4}{7}
  \end{bmatrix}\begin{bmatrix}
    3 & 3 & 3 \\
    0 & 0 & 1 \\
    3 & 3 & 2
  \end{bmatrix}\begin{bmatrix}
    7 & 1 & 0 \\
    1 & -1 & 0 \\
    6 & 0 & -1
  \end{bmatrix} &= \begin{bmatrix}
    6 & 0 & 0 \\
    0 & 0 & 0 \\
    0 & 0 & -1
  \end{bmatrix} \\
  \lambda_1 &= 6 \quad \vec{v_1} = \begin{bmatrix}7 \\ 1 \\ 6\end{bmatrix} \\
  \lambda_2 &= 0 \quad \vec{v_2} = \begin{bmatrix}1 \\ -1 \\ 0\end{bmatrix} \\
  \lambda_3 &= -1 \quad \vec{v_3} = \begin{bmatrix}0 \\ 0 \\ 1\end{bmatrix}
\end{align*}

\subsubsection*{Question 29}
Determine whether \( A \) is diagonalizable.
\[ A = \begin{bmatrix}
  0 & 1 & 1 \\
  1 & 1 & 0 \\
  1 & 0 & 1
\end{bmatrix} \]
Find an invertible matrix \( P \) and a diagonal matrix \( D \) such that
\( P^{-1}AP = D \).
\begin{align*}
  c_A(\lambda) &= \begin{vmatrix}
    -\lambda & 1 & 1 \\
    1 & 1-\lambda & 0 \\
    1 & 0 & 1-\lambda
  \end{vmatrix} \\
  &= 1\begin{vmatrix}
    1 & 1 \\
    1-\lambda & 0
  \end{vmatrix}-0+(1-\lambda)\begin{vmatrix}
    -\lambda & 1 \\
    1 & 1-\lambda
  \end{vmatrix} \\
  &= -(1-\lambda)+(1-\lambda)(-\lambda+\lambda^2-1) \\
  &= -1+\lambda-\lambda+\lambda^2-1+\lambda^2-\lambda^3+\lambda \\
  &= -\lambda^3+2\lambda^2+\lambda-2 \\
  &= (\lambda+1)(\lambda-1)(\lambda-2)
\end{align*}
\begin{align*}
  \lambda_1 &= -1 \\
  (A+1I)\vec{v_1} &= 0 \\
  \begin{bmatrix}
    1 & 1 & 1 & 0\\
    1 & 2 & 0 & 0\\
    1 & 0 & 2 & 0
  \end{bmatrix} &\to \begin{bmatrix}
    1 & 0 & 2 & 0 \\
    0 & 1 & -1 & 0 \\
    0 & 0 & 0 & 0
  \end{bmatrix} \\
  x_1 &= -2x_3 \quad x_2 = x_3 \quad x_3 = s \\
  \vec{v_1} &= s\begin{bmatrix}-2 \\ 1 \\ 1\end{bmatrix} \\
  \lambda_2 &= 1 \\
  (A-1I)\vec{v_2} &= 0 \\
  \begin{bmatrix}
    -1 & 1 & 1 & 0 \\
    1 & 0 & 0 & 0 \\
    1 & 0 & 0 & 0
  \end{bmatrix} &\to \begin{bmatrix}
    1 & 0 & 0 & 0 \\
    0 & 1 & 1 & 0 \\
    0 & 0 & 0 & 0
  \end{bmatrix} \\
  x_1 &= 0 \quad x_2 = -x_3 \quad x_3 = s \\
  \vec{v_2} &= s\begin{bmatrix}0 \\ -1 \\ 1\end{bmatrix} \\
  \lambda_3 &= 2 \\
  (A-2I)\vec{v_3} &= 0 \\
  \begin{bmatrix}
    -2 & 1 & 1 & 0 \\
    1 & -1 & 0 & 0 \\
    1 & 0 & -1 & 0
  \end{bmatrix} &\to \begin{bmatrix}
    1 & 0 & -1 & 0 \\
    0 & 1 & -1 & 0 \\
    0 & 0 & 0 & 0
  \end{bmatrix} \\
  x_1 &= x_3 \quad x_2 = x_3 \quad x_3 = s \\
  \vec{v_3} &= s\begin{bmatrix}1 \\ 1 \\ 1\end{bmatrix} \\
  D &= \begin{bmatrix}
    -1 & 0 & 0 \\
    0 & 1 & 0 \\
    0 & 0 & 2
  \end{bmatrix} \\
  P &= \begin{bmatrix}
    -2 & 0 & 1 \\
    1 & -1 & 1 \\
    1 & 1 & 1
  \end{bmatrix}
\end{align*}

\subsubsection*{Question 30}
Determine whether \( A \) is diagonalizable.
\[ A = \begin{bmatrix}
  1 & 0 & 0 \\
  4 & 4 & 1 \\
  5 & 0 & 1
\end{bmatrix} \]
Find an invertible matrix \( P \) and a diagonal matrix \( D \) such that
\( P^{-1}AP = D \).
\begin{align*}
  c_A(\lambda) &= \begin{vmatrix}
    1-\lambda & 0 & 0 \\
    4 & 4-\lambda & 1 \\
    5 & 0 & 1-\lambda
  \end{vmatrix} \\
  &= (1-\lambda)\begin{vmatrix}
    4-\lambda & 1 \\
    0 & 1-\lambda
  \end{vmatrix}-0+0 \\
  &= (1-\lambda)(4-\lambda)(1-\lambda)
\end{align*}
\begin{align*}
  \lambda_1 &= \lambda_2 = 1 \\
  (A-1I)\vec{v_{12}} &= 0 \\
  \begin{bmatrix}
    0 & 0 & 0 & 0 \\
    4 & 3 & 1 & 0 \\
    5 & 0 & 0 & 0
  \end{bmatrix} &\to \begin{bmatrix}
    1 & 0 & 0 & 0 \\
    0 & 1 & \frac{1}{3} & 0 \\\
    0 & 0 & 0 & 0
  \end{bmatrix} \\
  x_1 &= 0 \quad x_2 = \frac{1}{3}x_3 \quad x_3 = 0 \\
  \vec{v_{12}} &= s\begin{bmatrix}0 \\ \frac{1}{3} \\ 1\end{bmatrix}
\end{align*}
The geometric multiplicity of this eigenvalue is less than its algebraic
multiplicity, so this matrix is not diagonalizable.

\subsubsection*{Question 31}
Use the method of Example 4.29 to compute the indicated power of the matrix.
\[ \begin{bmatrix}
  1 & 1 & 1 \\
  0 & -1 & 0 \\
  0 & 0 & -1
\end{bmatrix}^{2017} \]
\begin{align*}
  A &= \begin{bmatrix}
    1 & 1 & 1 \\
    0 & -1 & 0 \\
    0 & 0 & -1
  \end{bmatrix} \\
  c_A(\lambda) &= \begin{vmatrix}
    1-\lambda & 1 & 1 \\
    0 & -1-\lambda & 0 \\
    0 & 0 & -1-\lambda
  \end{vmatrix} \\
  &= (1-\lambda)\begin{vmatrix}
    -1-\lambda & 0 \\
    0 & -1-\lambda
  \end{vmatrix}-0+0 \\
  &= (1-\lambda)(-1-\lambda)(-1-\lambda)
\end{align*}
\begin{align*}
  \lambda_1 &= 1 \\
  (A-1I)\vec{v_1} &= 0 \\
  \begin{bmatrix}
    0 & 1 & 1 & 0 \\
    0 & -2 & 0 & 0 \\
    0 & 0 & -2 & 0
  \end{bmatrix} &\to \begin{bmatrix}
    0 & 1 & 0 & 0 \\
    0 & 0 & 1 & 0 \\
    0 & 0 & 0 & 0
  \end{bmatrix} \\
  \vec{v_1} &= s\begin{bmatrix}1 \\ 0 \\ 0\end{bmatrix} \\
  \lambda_{23} &= -1 \\
  (A+1I)\vec{v_{23}} &= 0 \\
  \begin{bmatrix}
    2 & 1 & 1 \\
    0 & 0 & 0 \\
    0 & 0 & 0
  \end{bmatrix} &\to \begin{bmatrix}
    1 & \frac{1}{2} & \frac{1}{2} & 0 \\
    0 & 0 & 0 \\
    0 & 0 & 0
  \end{bmatrix} \\
  2x_1 &= -x_2-x_3 \quad x_2 = s \quad x_3 = t \\
  \vec{v_{23}} &= s\begin{bmatrix}-\frac{1}{2} \\ 1 \\ 0\end{bmatrix}+
    t\begin{bmatrix}-\frac{1}{2} \\ 0 \\ 1\end{bmatrix} \\
  D &= \begin{bmatrix}
    1 & 0 & 0 \\
    0 & -1 & 0 \\
    0 & 0 & -1
  \end{bmatrix} \\
  P &= \begin{bmatrix}
    1 & -\frac{1}{2} & -\frac{1}{2} \\
    0 & 1 & 0 \\
    0 & 0 & 1
  \end{bmatrix} \\
  P^{-1} &= \begin{bmatrix}
    1 & \frac{1}{2} & \frac{1}{2} \\
    0 & 1 & 0 \\
    0 & 0 & 1
  \end{bmatrix} \\
  P^{-1}AP &= D \\
  A &= PDP^{-1} \\
  A^{2017} &= PD^{2017}P^{-1} \\
  &= \begin{bmatrix}
    1 & -\frac{1}{2} & -\frac{1}{2} \\
    0 & 1 & 0 \\
    0 & 0 & 1
  \end{bmatrix}\begin{bmatrix}
    1 & 0 & 0 \\
    0 & -1 & 0 \\
    0 & 0 & -1
  \end{bmatrix}^{2017}\begin{bmatrix}
    1 & \frac{1}{2} & \frac{1}{2} \\
    0 & 1 & 0 \\
    0 & 0 & 1
  \end{bmatrix} \\
  &= \begin{bmatrix}
    1 & -\frac{1}{2} & -\frac{1}{2} \\
    0 & 1 & 0 \\
    0 & 0 & 1
  \end{bmatrix}\begin{bmatrix}
    1 & 0 & 0 \\
    0 & -1 & 0 \\
    0 & 0 & -1
  \end{bmatrix}\begin{bmatrix}
    1 & \frac{1}{2} & \frac{1}{2} \\
    0 & 1 & 0 \\
    0 & 0 & 1
  \end{bmatrix} \\
  &= \begin{bmatrix}
    1 & -\frac{1}{2} & -\frac{1}{2} \\
    0 & 1 & 0 \\
    0 & 0 & 1
  \end{bmatrix}\begin{bmatrix}
    1 & \frac{1}{2} & \frac{1}{2} \\
    0 & -1 & 0 \\
    0 & 0 & -1
  \end{bmatrix} \\
  &= \begin{bmatrix}
    1 & 1 & 1 \\
    0 & -1 & 0 \\
    0 & 0 & -1
  \end{bmatrix}
\end{align*}

\begin{center}
  If you have any questions, comments, or concerns, please contact me at
  alvin@omgimanerd.tech
\end{center}

\end{document}
