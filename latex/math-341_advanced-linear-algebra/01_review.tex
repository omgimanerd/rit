\documentclass{math}

\usepackage{tikz}

\title{Advanced Linear Algebra}
\author{Alvin Lin}
\date{January 2019 - May 2019}

\begin{document}

\maketitle

\section*{Linear Algebra}
To define a plane, we need a point on the plane and a normal vector, or two
vectors on the plane.
\[ \vec{n}\cdot(\vec{x}-\vec{p}) = 0 \]
This is the point normal form of a plane (where \( \vec{n} \) is the normal
vector, \( \vec{p} \) is the point, and \( \vec{x} \) is a vector on the plane).
If instead we are given a point \( \vec{p} \) and two vectors \( \vec{u} \) and
\( \vec{v} \), we can describe any point in the plane as a linear combination of
\( \vec{u} \) and \( \vec{v} \).
\[ s\vec{u}+t\vec{v} = \vec{x}-\vec{p} \]
This is the vector form of a plane.

\subsubsection*{Example}
Find the equation of the plane that contains \( (6,0,1) \) and has normal vector
\( \vec{n} = \begin{bmatrix}1 \\ 2 \\ 3\end{bmatrix} \).
\begin{align*}
  \langle1,2,3\rangle\cdot(\langle x,y,z\rangle-\langle6,0,1\rangle) &= 0 \\
  \langle1,2,3\rangle\cdot\langle x-6,y-0,z-1\rangle &= 0 \\
  1(x-6)+2y+3(z-1) &= 0 \\
  x-6+2y+3z-3 &= 0 \\
  x+2y+3z &= 9
\end{align*}
Knowing \( \vec{p} = \langle6,0,1\rangle \) is on the plane, find the vector
form. To do this, we need to find two other points \( Q \) and \( R \) in the
plane. We can pick these arbitrarily by plugging them into the equation.
\begin{align*}
  Q &= (9,0,0) \\
  R &= (3,3,0)
\end{align*}
We can then construct the vectors \( \vec{u} \) and \( \vec{v} \) by calculating
the vector between Q and the point represented by \( \vec{p} \), and analogously
for point R.
\begin{align*}
  \vec{u} &= \overrightarrow{QP} = \langle9,0,0\rangle-\langle6,0,1\rangle =
    \langle3,0,-1\rangle \\
  \vec{v} &= \overrightarrow{RP} = \langle3,3,0\rangle-\langle6,0,1\rangle =
    \langle-3,3,-1\rangle
\end{align*}
These two vectors can then construct the vector form of the plane.
\begin{align*}
  \begin{bmatrix}x \\ y \\ z\end{bmatrix} &=
    \begin{bmatrix} 6 \\ 0 \\ 1\end{bmatrix}+
    s\begin{bmatrix}3 \\ 0 \\ -1\end{bmatrix}+
    t\begin{bmatrix}-3 \\ 3 \\ -1\end{bmatrix} \\
  x &= 6+3s-3t \\
  y &= 3t \\
  z &= 1-s-t
\end{align*}

\subsection*{Systems of Linear Equations}
For two lines in \( \R^2 \), two lines can either intersect at a single point,
infinitely many points if they lie on top of each other, or no points if they
are parallel.
\begin{align*}
  x-y &= 1 \\
  x+y &= 3
\end{align*}
\begin{center}
  \begin{tikzpicture}[scale=0.7]
    \draw[thick,<->] (-5,0) -- (5,0);
    \draw[thick,<->] (0,-5) -- (0,5);
    \draw[red] (-2,-3) -- (3,2);
    \draw[red] (-2,5) -- (3,0);
    \draw[fill=black] (2,1) circle (0.1cm);
  \end{tikzpicture}
\end{center}
These are two lines that intersect with a unique solution at \( (3,1) \). As
counterexamples:
\begin{align*}
  x-y &= 2 \\
  2x-2y &= 4
\end{align*}
These two lines lie on top of each other, resulting in infinitely many
solutions.
\begin{align*}
  x-y &= 1 \\
  x-y &= 3
\end{align*}
These two lines are parallel, and have no solutions.

\subsection*{Gaussian Elimination}
We can use Gaussian Elimination to solve systems of linear equations.
\begin{align*}
  x-y-z &= 2 \\
  3x-3y+2x &= 16 \\
  2x-y+z &= 9
\end{align*}
We can compose these equations into an augmented matrix.
\[ \begin{bmatrix}
  1 & -1 & -1 & 2 \\
  3 & -3 & 2 & 16 \\
  2 & -1 & 1 & 9
\end{bmatrix} \]
The goal is acquire an upper triangular matrix. First perform \( R_2-3R_1 \)
and replace \( R_2 \) with the result.
\[ \begin{bmatrix}
  1 & -1 & -1 & 2 \\
  0 & 0 & 5 & 10 \\
  2 & -1 & 1 & 9
\end{bmatrix} \]
Then we perform \( R_3-2R_1 \) and replace \( R_3 \) with the result.
\[ \begin{bmatrix}
  1 & -1 & -1 & 2 \\
  0 & 0 & 5 & 10 \\
  0 & 1 & 3 & 5
\end{bmatrix} \]
We can now swap \( R_2 \) and \( R_3 \).
\[ \begin{bmatrix}
  1 & -1 & -1 & 2 \\
  0 & 1 & 3 & 5 \\
  0 & 0 & 5 & 10
\end{bmatrix} \]
This matrix is now in row echelon form, where the leading entry in any row has
zeroes below it. From this point, we can perform back substitution to get back
a system of equations:
\begin{align*}
  5z &= 10 \quad z = 2 \\
  y+3z &= 5 \quad y+6 = 5 \quad y = -1 \\
  x-y-z &= 2 \quad x = 2+(-1)+(2) = 3 \\
  (x,y,z) &= (3,-1,2)
\end{align*}

\begin{center}
  You can find all my notes at \url{http://omgimanerd.tech/notes}. If you have
  any questions, comments, or concerns, please contact me at
  alvin@omgimanerd.tech
\end{center}

\end{document}
