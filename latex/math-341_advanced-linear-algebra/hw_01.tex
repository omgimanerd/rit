\documentclass{math}

\geometry{letterpaper, margin=0.5in}

\title{Advanced Linear Algebra: Homework 1}
\author{Alvin Lin}
\date{August 2016 - December 2016}

\begin{document}

\maketitle

\subsubsection*{Question 1}
Refer to the vectors below.
\[ \vec{b} = [4,2,1] \quad \vec{c} = [1,-3,1] \quad \vec{d} = [-1,-1,-2] \]
Compute the indicated vector.
\begin{align*}
  3\vec{b}-2\vec{c}+\vec{d} &= [12,6,3]-[2,-6,2]+[-1,-1,-2] \\
  &= [9,11,-1]
\end{align*}

\subsubsection*{Question 2}
\begin{align*}
  \vec{u} &= \bigg[\cos(\frac{\pi}{3}),\sin(\frac{\pi}{3})\bigg] \\
  \vec{v} &= \bigg[-\cos(\frac{\pi}{6}),-\sin(\frac{\pi}{6})\bigg] \\
  \vec{u}+\vec{v} &= \bigg[\cos(\frac{\pi}{3})-\cos(\frac{\pi}{6}),
    \sin(\frac{\pi}{3})-\sin(\frac{\pi}{6})\bigg] \\
  \vec{u}-\vec{v} &= \bigg[\cos(\frac{\pi}{3})+\cos(\frac{\pi}{6}),
    \sin(\frac{\pi}{3})+\sin(\frac{\pi}{6})\bigg]
\end{align*}

\subsubsection*{Question 3}
Find the projection of \( \vec{v} \) onto \( \vec{u} \).
\[ \vec{u} = \begin{bmatrix}\frac{1}{2} \\ -\frac{1}{4} \\
  -\frac{1}{2}\end{bmatrix} \quad\
  \vec{v} = \begin{bmatrix}4 \\ 4 \\ -4\end{bmatrix} \]
\begin{align*}
  proj_{\vec{u}}(\vec{v}) &=
    \frac{\vec{u}\cdot\vec{v}}{\vec{u}\cdot\vec{u}}\vec{u} \\
  &= \frac{2-1+2}{\frac{1}{4}+\frac{1}{16}+\frac{1}{4}}\begin{bmatrix}
    \frac{1}{2} \\ -\frac{1}{4} \\ -\frac{1}{2}
  \end{bmatrix} \\
  &= \frac{3}{\frac{9}{16}}\begin{bmatrix}
    \frac{1}{2} \\ -\frac{1}{4} \\ -\frac{1}{2}
  \end{bmatrix} \\
  &= \begin{bmatrix}
    \frac{8}{3} \\
    -\frac{4}{3} \\
    -\frac{8}{3} \\
  \end{bmatrix}
\end{align*}

\subsubsection*{Question 4}
Find \( \vec{u}\cdot\vec{v} \).
\begin{align*}
  \vec{u} &= \begin{bmatrix}3 \\ 1 \\ 2\end{bmatrix} \\
  \vec{v} &= \begin{bmatrix}3 \\ 2 \\ 1\end{bmatrix} \\
  \vec{u}\cdot\vec{v} &= (3)(3)+(1)(2)+(2)(1) = 13
\end{align*}

\subsubsection*{Question 5}
Find \( \|\vec{u}\| \) for the given vector.
\begin{align*}
  \vec{u} &= \begin{bmatrix}3 \\ 4 \\ 1\end{bmatrix} \\
  \|\vec{u}\| &= \sqrt{(3)(3)+(4)(4)+(1)(1)} = \sqrt{26}
\end{align*}
Give a unit vector in the direction of \( \vec{u} \).
\[ \frac{\vec{u}}{\|\vec{u}\|} = \begin{bmatrix}
  \frac{3}{\sqrt{26}} \\ \frac{4}{\sqrt{26}} \\
  \frac{1}{\sqrt{26}}\end{bmatrix} \]

\subsubsection*{Question 6}
Find the angle between \( \vec{u} \) and \( \vec{v} \).
\begin{align*}
  \vec{u} &= [4,3,-1] \quad \vec{v} = [1,-1,1] \\
  \cos\theta &= \frac{\vec{u}\cdot\vec{v}}{\|\vec{u}\|\|\vec{v}\|} \\
  \theta &= \arccos(\frac{4-3-1}{(16+9+1)(1+1+1)}) \\
  &= \arccos(0) = \frac{\pi}{2}
\end{align*}

\subsubsection*{Question 7}
Find the angle between \( \vec{u} \) and \( \vec{v} \).
\begin{align*}
  \vec{u} &= [0.9,1.9,1.2] \quad \vec{v} = [-4.5,2.4,-0.8] \\
  \cos\theta &= \frac{\vec{u}\cdot\vec{v}}{\|\vec{u}\|\|\vec{v}\|} \\
  \theta &= \arccos(\frac{-4.05+4.56-0.96}
    {\sqrt{0.81+3.61+1.44}\sqrt{20.25+5.76+0.64}}) \\
  &= \arccos(\frac{-0.45}{(2.42)(5.16)}) \approx 92.06^{\circ}
\end{align*}

\subsubsection*{Question 8}
Find all values of the scalar \( k \) for which the two vectors are orthogonal.
\begin{align*}
  \vec{u} &= \begin{bmatrix}1 \\ -1 \\ 3\end{bmatrix} \\
  \vec{v} &= \begin{bmatrix}k^2 \\ k \\ -4\end{bmatrix} \\
  \vec{u}\cdot\vec{v} &= 0 \\
  k^2-k-12 &= 0 \\
  (k-4)(k+3) &= 0 \\
  k &= 4,-3
\end{align*}

\subsubsection*{Question 9}
Write the equation of the line passing through \( P \) with direction vector
\( \vec{d} \) in vector form and parametric form.
\[ P = (5,0,-3) \quad \vec{d} = \begin{bmatrix}3 \\ 2 \\ 0\end{bmatrix} \]
\[ \begin{bmatrix}x \\ y \\ z\end{bmatrix} =
  \begin{bmatrix}5 \\ 0 \\ -3\end{bmatrix}+
  t\begin{bmatrix}3 \\ 2 \\ 0\end{bmatrix} \]
\begin{align*}
  x &= 5+3t \\
  y &= 2t \\
  z &= -3
\end{align*}

\subsubsection*{Question 10}
Write the equation of the plane passing through \( P \) with normal vector
\( \vec{n} \) in normal form and general form.
\[ P = (0,1,0) \quad \vec{n} = \begin{bmatrix}3 \\ 4 \\ 1\end{bmatrix} \]
\[ \begin{bmatrix}3 \\ 4 \\ 1\end{bmatrix}\cdot\left(
  \begin{bmatrix}x \\ y \\ z\end{bmatrix}-
  \begin{bmatrix}0 \\ 1 \\ 0\end{bmatrix}
\right) = 0 \]
\begin{align*}
  3x+4(y-1)+z &= 0 \\
  3x+4y+z &= -4
\end{align*}

\subsubsection*{Question 11}
Write the equation of the plane passing through \( P \) with direction vectors
\( \vec{u} \) and \( \vec{v} \) in vector form and parametric form.
\[ P = (0,0,0) \quad \vec{u} = \begin{bmatrix}3 \\ 1 \\ 3\end{bmatrix} \quad
  \vec{v} = \begin{bmatrix}-4 \\ 3 \\ 1\end{bmatrix} \]
\[ \begin{bmatrix}x \\ y \\ z\end{bmatrix} =
  \begin{bmatrix}0 \\ 0 \\ 0\end{bmatrix}+
  s\begin{bmatrix}3 \\ 1 \\ 3\end{bmatrix}+
  t\begin{bmatrix}-4 \\ 3 \\ 1\end{bmatrix} \]
\begin{align*}
  x &= 3s-4t \\
  y &= s+3t \\
  z &= 3s+t
\end{align*}

\subsubsection*{Question 12}
Given the vector equation of the line passing through \( P \) and \( Q \).
\[ P = (0,1,-1) \quad (-4,1,6) \]
\begin{align*}
  P-Q &= \begin{bmatrix}4 \\ 0 \\ -7\end{bmatrix} \\
  \vec{x} &= \begin{bmatrix}0 \\ 1 \\ -1\end{bmatrix}+
    t\begin{bmatrix}4 \\ 0 \\ -7\end{bmatrix} \\
\end{align*}

\subsubsection*{Question 13}
Find the normal form of the equation of the plane that passes through
\( P = (0,-2,5) \) and is parallel to the plane with general equation
\( 4x-y+5z = 3 \).
\begin{align*}
  \vec{n} &= \begin{bmatrix}4 \\ -1 \\ 5\end{bmatrix} \\
  \vec{n}\cdot\begin{bmatrix}0 \\ -2 \\ 5\end{bmatrix} &= 27 \\
  \begin{bmatrix}4 \\ -1 \\ 5\end{bmatrix}\cdot
    \begin{bmatrix}x \\ y \\ z\end{bmatrix} &= 27
\end{align*}

\subsubsection*{Question 14}
Solve the given system by back substitution.
\begin{align*}
  x-3y+z &= 5 \\
  y-2z &= -1 \\
  y &= -1+2z \\
  x-3(-1+2z)+z &= 5 \\
  x+3-6z+z &= 5 \\
  x &= 2+5z \\
  z &= s \\
  x &= 2+5s \\
  y &= -1+2s
\end{align*}

\subsubsection*{Question 15}
Find a system of linear equations that has the given matrix as its augmented
matrix.
\[ \begin{bmatrix}
  0 & 1 & 1 & 1 \\
  1 & -1 & 0 & 1 \\
  4 & -1 & 1 & 1
\end{bmatrix} \]
\begin{align*}
  y+z &= 1 \\
  x-y &= 1 \\
  4x-y+z &= 1
\end{align*}

\subsubsection*{Question 16}
Solve the linear system.
\begin{align*}
  -2x_1+3x_2-x_3 &= 1 \\
  x_1+x_3 &= 0 \\
  -x_1+2x_2-2x_3 &= 0
\end{align*}
\begin{align*}
  \begin{bmatrix}
    -2 & 3 & -1 & 1 \\
    1 & 0 & 1 & 0 \\
    -1 & 2 & -2 & 0
  \end{bmatrix} &\to \begin{bmatrix}
    -2 & 3 & -1 & 1 \\
    1 & 0 & 1 & 0 \\
    0 & 2 & -1 & 0
  \end{bmatrix} \\
  &\to \begin{bmatrix}
    0 & 3 & 1 & 1 \\
    1 & 0 & 1 & 0 \\
    0 & 2 & -1 & 0
  \end{bmatrix} \\
  &\to \begin{bmatrix}
    1 & 0 & 1 & 0 \\
    0 & 3 & 1 & 1 \\
    0 & 1 & -\frac{1}{2} & 0
  \end{bmatrix} \\
  &\to \begin{bmatrix}
    1 & 0 & 1 & 0 \\
    0 & 0 & \frac{5}{2} & 1 \\
    0 & 1 & -\frac{1}{2} & 0
  \end{bmatrix} \\
  &\to \begin{bmatrix}
    1 & 0 & 1 & 0 \\
    0 & 1 & -\frac{1}{2} & 0 \\
    0 & 0 & 1 & \frac{2}{5}
  \end{bmatrix} \\
  &\to \begin{bmatrix}
    1 & 0 & 0 & -\frac{2}{5} \\
    0 & 1 & 0 & \frac{1}{5} \\
    0 & 0 & 1 & \frac{2}{5}
  \end{bmatrix} \\
  \begin{bmatrix}x \\ y \\ z\end{bmatrix} &= \begin{bmatrix}
    -\frac{2}{5} \\ \frac{1}{5} \\ \frac{2}{5}\end{bmatrix}
\end{align*}

\subsubsection*{Question 17}
Solve the linear system.
\begin{align*}
  a-2b+d &= 3 \\
  -a+b-c-4d &= 1
\end{align*}
\begin{align*}
  \begin{bmatrix}
    1 & -2 & 0 & 1 & 3 \\
    -1 & 1 & -1 & -4 & 1
  \end{bmatrix} &\to \begin{bmatrix}
    1 & -2 & 0 & 1 & 3 \\
    0 & -1 & -1 & -3 & 4
  \end{bmatrix} \\
  &\to \begin{bmatrix}
    1 & 0 & 2 & 7 & -5 \\
    0 & -1 & -1 & -3 & 4
  \end{bmatrix} \\
  &\to \begin{bmatrix}
    1 & 0 & 2 & 7 & -5 \\
    0 & 1 & 1 & 3 & -4
  \end{bmatrix} \\
  c &= s \quad d = t \\
  a+2c+7d &= -5 \\
  a &= -5-2c-7d \\
  b+c+3d &= -4 \\
  b &= -4-c-3d
\end{align*}

\subsubsection*{Question 18}
Determine whether the given matrix is in row echelon form.
\[ \begin{bmatrix}
  1 & 0 & 1 \\
  0 & 0 & 4 \\
  0 & 1 & 0
\end{bmatrix} \]
The matrix is not in row echelon form.

\subsubsection*{Question 19}
Determine whether the given matrix is in row echelon form.
\[ \begin{bmatrix}
  0 & 1 & 4 & 9 \\
  0 & 0 & 0 & 1
\end{bmatrix} \]
The matrix is in row echelon form and reduced row echelon form.

\subsubsection*{Question 20}
Use elementary row operations to reduce the given matrix to row echelon form
and reduced row echelon form.
\[ \begin{bmatrix}
  0 & 0 & 1 \\
  0 & 1 & 1 \\
  1 & 1 & 1
\end{bmatrix} \]
\[ REF = \begin{bmatrix}
  1 & 1 & 1 \\
  0 & 1 & 1 \\
  0 & 0 & 1
\end{bmatrix} \]
\[ RREF = \begin{bmatrix}
  1 & 0 & 0 \\
  0 & 1 & 0 \\
  0 & 0 & 1
\end{bmatrix} \]

\subsubsection*{Question 21}
Use elementary row operations to reduce the given matrix to row echelon form
and reduced row echelon form.
\[ \begin{bmatrix}
  -2 & -4 & 9 \\
  -4 & -8 & 17 \\
  1 & 2 & -4
\end{bmatrix} \]
\begin{align*}
  \begin{bmatrix}
    -2 & -4 & 9 \\
    -4 & -8 & 17 \\
    1 & 2 & -4
  \end{bmatrix} &\to \begin{bmatrix}
    0 & 0 & 1 \\
    -4 & -8 & 17 \\
    1 & 2 & -4
  \end{bmatrix} \\
  &\to \begin{bmatrix}
    0 & 0 & 1 \\
    0 & 0 & 1 \\
    1 & 2 & -4
  \end{bmatrix} \\
  &\to \begin{bmatrix}
    1 & 2 & -4 \\
    0 & 0 & 1 \\
    0 & 0 & 0
  \end{bmatrix} (REF) \\
  &\to \begin{bmatrix}
    1 & 2 & 0 \\
    0 & 0 & 1 \\
    0 & 0 & 0
  \end{bmatrix} (RREF) \\
\end{align*}

\subsubsection*{Question 22}
What is the rank of each of the matrices below?
\begin{align*}
  \begin{bmatrix}
    1 & 0 & 1 \\
    0 & 0 & 4 \\
    0 & 1 & 0
  \end{bmatrix} & 3 \\
  \begin{bmatrix}
    8 & 0 & 1 & 0 \\
    0 & 1 & -1 & 5 \\
    0 & 0 & 0 & 0
  \end{bmatrix} & 2 \\
  \begin{bmatrix}
    0 & 1 & 4 & 0 \\
    0 & 0 & 0 & 1
  \end{bmatrix} & 2 \\
  \begin{bmatrix}
    0 & 0 & 0 \\
    0 & 0 & 0 \\
    0 & 0 & 0
  \end{bmatrix} & 0 \\
  \begin{bmatrix}
    1 & 0 & 4 & -5 & 0 \\
    0 & 0 & 0 & 0 & 0 \\
    0 & 1 & 6 & 0 & 1
  \end{bmatrix} & 2 \\
  \begin{bmatrix}
    0 & 0 & 1 \\
    0 & 1 & 0 \\
    1 & 0 & 0
  \end{bmatrix} & 3 \\
  \begin{bmatrix}
    1 & 2 & 4 \\
    1 & 0 & 0 \\
    0 & 1 & 1 \\
    0 & 0 & 1
  \end{bmatrix} & 3 \\
  \begin{bmatrix}
    2 & 1 & 4 & 6 \\
    0 & 0 & 1 & -1 \\
    0 & 0 & 0 & 4 \\
    0 & 0 & 0 & 0
  \end{bmatrix} & 3
\end{align*}

\subsubsection*{Question 23}
Solve the given system of equations using either Gaussian or Gauss-Jordan
elimination.
\begin{align*}
  x_1+2x_2-3x_3 &= 13 \\
  3x_1-x_2+x_3 &= 0 \\
  4x_1-x_2+x_3 &= 2
\end{align*}
\begin{align*}
  \begin{bmatrix}
    1 & 2 & -3 & 13 \\
    3 & -1 & 1 & 0 \\
    4 & -1 & 1 & 2
  \end{bmatrix} &\to \begin{bmatrix}
    1 & 2 & -3 & 13 \\
    0 & -7 & 10 & -39 \\
    4 & -1 & 1 & 2
  \end{bmatrix} \\
  &\to \begin{bmatrix}
    1 & 2 & -3 & 13 \\
    0 & -7 & 10 & -39 \\
    0 & -9 & 13 & -50
  \end{bmatrix} \\
  &\to \begin{bmatrix}
    1 & 2 & -3 & 13 \\
    0 & 1 & \frac{-10}{7} & \frac{39}{7} \\
    0 & -9 & 13 & -50
  \end{bmatrix} \\
  &\to \begin{bmatrix}
    1 & 2 & -3 & 13 \\
    0 & 1 & \frac{-10}{7} & \frac{39}{7} \\
    0 & 0 & \frac{1}{7} & \frac{1}{7} \\
  \end{bmatrix} \\
  &\to \begin{bmatrix}
    1 & 2 & -3 & 13 \\
    0 & 1 & \frac{-10}{7} & \frac{39}{7} \\
    0 & 0 & 1 & 1 \\
  \end{bmatrix} \\
  &\to \begin{bmatrix}
    1 & 2 & -3 & 13 \\
    0 & 1 & 0 & \frac{49}{7} \\
    0 & 0 & 1 & 1 \\
  \end{bmatrix} \\
  &\to \begin{bmatrix}
    1 & 2 & -3 & 13 \\
    0 & 1 & 0 & 7 \\
    0 & 0 & 1 & 1 \\
  \end{bmatrix} \\
  &\to \begin{bmatrix}
    1 & 0 & -3 & -1 \\
    0 & 1 & 0 & 7 \\
    0 & 0 & 1 & 1 \\
  \end{bmatrix} \\
  &\to \begin{bmatrix}
    1 & 0 & 0 & 2 \\
    0 & 1 & 0 & 7 \\
    0 & 0 & 1 & 1 \\
  \end{bmatrix} \\
\end{align*}

\subsubsection*{Question 24}
Solve the given system of equations using either Gaussian or Gauss-Jordan
elimination.
\begin{align*}
  -x_1+4x_2-2x_3+4x_4 &= 0 \\
  2x_1-8x_2+x_3-2x_4 &= -3 \\
  x_1-4x_2+4x_3-8x_4 &= 2
\end{align*}
\[ \begin{bmatrix}
  -1 & 4 & -2 & 4 & 0 \\
  2 & -8 & 1 & -2 & -3 \\
  1 & -4 & 4 & -8 & 2
\end{bmatrix} \to \begin{bmatrix}
  1 & -4 & 0 & 0 & -2 \\
  0 & 0 & 1 & -2 & 1 \\
  0 & 0 & 0 & 0 & 0
\end{bmatrix} \]
\begin{align*}
  x_1-4x_2 &= -2 \\
  x_3-2x_4 &= 1 \\
  x_2 &= s \quad x_4 = t \\
  x_1 &= -2+4s \\
  x_3 &= 1+2t
\end{align*}

\subsubsection*{Question 25}
Solve the given system of equations using either Gaussian or Gauss-Jordan
elimination.
\begin{align*}
  \frac{1}{2}x_1+x_2-x_3-6x_4 &= 2 \\
  \frac{1}{6}x_1+\frac{1}{2}x_2-3x_4+x_5 &= -1 \\
  \frac{1}{3}x_1-2x_3-4x_5 &= 8
\end{align*}
\[ \begin{bmatrix}
  \frac{1}{2} & 1 & -1 & -6 & 0 & 2 \\
  \frac{1}{6} & \frac{1}{2} & 0 & -3 & 1 & -1 \\
  \frac{1}{3} & 0 & -2 & 0 & -4 & 8
\end{bmatrix} \to \begin{bmatrix}
  1 & 0 & -6 & 0 & -12 & 24 \\
  0 & 1 & 2 & -6 & 6 & -10 \\
  0 & 0 & 0 & 0 & 0 & 0
\end{bmatrix} \]
\begin{align*}
  x_1-6x_3-12x_5 &= 24 \\
  x_2+2x_3-6x_4+6x_5 &= -10 \\
  x_3 &= r \quad x_4 = s \quad x_5 = t \\
  x_1 &= 24+6r+12t \\
  x_2 &= -10-2r+6s-6t
\end{align*}

\subsubsection*{Question 26}
Solve the given system of equations using either Gaussian or Gauss-Jordan
elimination.
\begin{align*}
  w+x+2y+z &= 1 \\
  w-x-y+z &= 0 \\
  x+y &= -1 \\
  w+x+z &= 4
\end{align*}
\[ \begin{bmatrix}
  1 & 1 & 2 & 1 & 1 \\
  1 & -1 & -1 & 1 & 0 \\
  0 & 1 & 1 & 0 & -1 \\
  1 & 1 & 0 & 1 & 4
\end{bmatrix} \]
No solution.

\subsubsection*{Question 27}
Use elementary row operations to reduce the given matrix to row echelon form
and reduced row echelon form.
\begin{align*}
  \begin{bmatrix}
    -2 & 6 & -7 \\
    3 & -9 & 10 \\
    -6 & 18 & -19
  \end{bmatrix} &\to \begin{bmatrix}
    -2 & 6 & -7 \\
    3 & -9 & 10 \\
    0 & 0 & 1
  \end{bmatrix} \\
  &\to \begin{bmatrix}
    -1 & 3 & -\frac{7}{2} \\
    3 & -9 & 10 \\
    0 & 0 & 1
  \end{bmatrix} \\
  &\to \begin{bmatrix}
    -1 & 3 & -\frac{7}{2} \\
    0 & 0 & -\frac{1}{2} \\
    0 & 0 & 1
  \end{bmatrix} \\
  &\to \begin{bmatrix}
    1 & -3 & \frac{7}{2} \\
    0 & 0 & 1 \\
    0 & 0 & 0
  \end{bmatrix} (REF) \\
  &\to \begin{bmatrix}
    1 & -3 & 0 \\
    0 & 0 & 1 \\
    0 & 0 & 0
  \end{bmatrix} (RREF)
\end{align*}

\subsubsection*{Question 28}
Determine if the vector \( \vec{v} \) is a linear combination of the
remaining vectors.
\[ \vec{v} = \begin{bmatrix}4 \\ 3 \\ 4\end{bmatrix} \quad
  \vec{u_1} = \begin{bmatrix}1 \\ 1 \\ 1\end{bmatrix} \quad
  \vec{u_2} = \begin{bmatrix}0 \\ 1 \\ 1\end{bmatrix} \quad
  \vec{u_3} = \begin{bmatrix}1 \\ 0 \\ 1\end{bmatrix} \]
\begin{align*}
  x\vec{u_1}+y\vec{u_2}+z\vec{u_3} &= \vec{v} \\
  x+z &= 4 \\
  x+y &= 3 \\
  x+y+z &= 4 \\
  \begin{bmatrix}
    1 & 0 & 1 & 4 \\
    1 & 1 & 0 & 3 \\
    1 & 1 & 1 & 4
  \end{bmatrix} &\to \begin{bmatrix}
    1 & 0 & 0 & 3 \\
    0 & 1 & 0 & 0 \\
    0 & 0 & 1 & 1
  \end{bmatrix} \\
  \therefore 3\vec{u_1}+\vec{u_3} &= \vec{v}
\end{align*}

\subsubsection*{Question 29}
Determine if the vector \( \vec{b} \) is in the span of the columns of the
matrix \( A \).
\begin{align*}
  A &= \begin{bmatrix}
    1 & 2 & 3 \\
    4 & 5 & 6 \\
    6 & 7 & 8
  \end{bmatrix} \quad \vec{b} = \begin{bmatrix}
    4 \\ 7 \\ 9
  \end{bmatrix} \\
  x\begin{bmatrix}1 \\ 4 \\ 6\end{bmatrix}+
    y\begin{bmatrix}2 \\ 5 \\ 7\end{bmatrix}+
    z\begin{bmatrix}3 \\ 6 \\ 8\end{bmatrix} &=
    \begin{bmatrix}4 \\ 7 \\ 9\end{bmatrix} \\
  x+2y+3z &= 4 \\
  4x+5y+6z &= 7 \\
  6x+7y+8z &= 9 \\
  \begin{bmatrix}
    1 & 2 & 3 & 4 \\
    4 & 5 & 6 & 7 \\
    6 & 7 & 8 & 9
  \end{bmatrix} &\to \begin{bmatrix}
    1 & 0 & -1 & -2 \\
    0 & 1 & 2 & 3 \\
    0 & 0 & 0 & 0
  \end{bmatrix}
\end{align*}
This system has a solution so \( \vec{b} \) can be represented as a linear
combination of the columns of \( A \), therefore it is in the span.

\subsubsection*{Question 30}
Describe the span of the given vectors geometrically and algebraically.
\[ \vec{u} = \begin{bmatrix}1 \\ 3 \\ 0\end{bmatrix} \quad
  \vec{v} = \begin{bmatrix}5 \\ 3 \\ -1\end{bmatrix} \]
The set of all linear combinations of \( \vec{u} \) and \( \vec{v} \) is a
plane containing both \( \vec{u} \) and \( \vec{v} \) as direction vectors
and contains all points described by \( s\vec{u}+t\vec{v} \) where \( s,t \)
are arbitrary constants. The vector equation of this plane is
\[ \begin{bmatrix}x \\ y \\ z\end{bmatrix} =
  s\begin{bmatrix}1 \\ 3 \\ 0\end{bmatrix}+
  t\begin{bmatrix}5 \\ 3 \\ -1\end{bmatrix} \]

\subsubsection*{Question 31}
Let \( \vec{v_1},\vec{v_2},\dots,\vec{v_m} \) be (column) vectors in \( \R^n \)
and let \( A \) be the \( n\times m \) matrix
\( [\vec{v_1},\vec{v_2},\dots,\vec{v_m}] \) with these vectors as its columns.
Then \( \vec{v_1},\vec{v_2},\dots,\vec{v_m} \) are linearly dependent if and
only if the homogeneous linear system with augmented matrix \( [A][0] \) has a
nontrivial solution. Use the theorem above to determine if the set of vectors is
linearly dependent.
\[ \begin{bmatrix}2 \\ 2 \\ 1\end{bmatrix},
  \begin{bmatrix}3 \\ 1 \\ 2\end{bmatrix},
  \begin{bmatrix}1 \\ -5 \\ 2\end{bmatrix} \]
\begin{align*}
  \begin{bmatrix}
    2 & 3 & 1 & 0 \\
    2 & 1 & -5 & 0 \\
    1 & 2 & 2 & 0
  \end{bmatrix} &\to \begin{bmatrix}
    1 & 0 & -4 & 0 \\
    0 & 1 & 3 & 0 \\
    0 & 0 & 0 & 0
  \end{bmatrix}
\end{align*}
The set of vectors is linearly dependent.

\subsubsection*{Question 32}
If \( A = \begin{bmatrix}a & b \\ c & d\end{bmatrix} \), then \( A \) is
invertible if \( ad-bc \ne 0 \), in which case
\[ A^{-1} = \frac{1}{ad-bc}\begin{bmatrix}d & -b \\ -c & a\end{bmatrix} \]
If \( ad-bc = 0 \), then \( A \) is not invertible. Find the inverse of the
given matrix (if it exists) using the theorem above.
\[ \begin{bmatrix}
  4 & 19 \\
  1 & 5
\end{bmatrix} \]
\begin{align*}
  ad-bc &= 20-19 = 1 \\
  \begin{bmatrix}
    4 & 19 & 1 & 0 \\
    1 & 5 & 0 & 1
  \end{bmatrix} &\to \begin{bmatrix}
    0 & -1 & 1 & -4 \\
    1 & 5 & 0 & 1
  \end{bmatrix} \\
  &\to \begin{bmatrix}
   0 & -1 & 1 & -4 \\
   1 & 0 & 5 & -19
  \end{bmatrix} \\
  &\to \begin{bmatrix}
    1 & 0 & 5 & -19 \\
    0 & 1 & -1 & 4
  \end{bmatrix}
\end{align*}

\subsubsection*{Question 33}
Use the Gauss-Jordan method to find the inverse of the given matrix (if it
exists).
\[ \begin{bmatrix}
  3 & 2 & 0 \\
  -2 & -1 & 0 \\
  2 & 1 & -1
\end{bmatrix} \]
\begin{align*}
  \begin{bmatrix}
    3 & 2 & 0 & 1 & 0 & 0 \\
    -2 & -1 & 0 & 0 & 1 & 0 \\
    2 & 1 & -1 & 0 & 0 & 1
  \end{bmatrix} &= \begin{bmatrix}
    3 & 2 & 0 & 1 & 0 & 0 \\
    0 & 0 & -1 & 0 & 1 & 1 \\
    2 & 1 & -1 & 0 & 0 & 1
  \end{bmatrix} \\
  &= \begin{bmatrix}
    6 & 4 & 0 & 2 & 0 & 0 \\
    0 & 0 & 1 & 0 & -1 & -1 \\
    6 & 3 & -3 & 0 & 0 & 3
  \end{bmatrix} \\
  &= \begin{bmatrix}
    6 & 4 & 0 & 2 & 0 & 0 \\
    0 & 0 & 1 & 0 & -1 & -1 \\
    0 & 1 & 3 & 2 & 0 & -3
  \end{bmatrix} \\
  &= \begin{bmatrix}
    6 & 4 & 0 & 2 & 0 & 0 \\
    0 & 0 & 1 & 0 & -1 & -1 \\
    0 & 1 & 0 & 2 & 3 & 0
  \end{bmatrix} \\
  &= \begin{bmatrix}
    6 & 0 & 0 & -6 & -12 & 0 \\
    0 & 0 & 1 & 0 & -1 & -1 \\
    0 & 1 & 0 & 2 & 3 & 0
  \end{bmatrix} \\
  &= \begin{bmatrix}
    1 & 0 & 0 & -1 & -2 & 0 \\
    0 & 1 & 0 & 2 & 3 & 0 \\
    0 & 0 & 1 & 0 & -1 & -1
  \end{bmatrix}
\end{align*}

\subsubsection*{Question 34}
Determine whether \( \vec{b} \) is in \( col(A) \). Determine whether
\( \vec{w} \) is in \( row(A) \).
\[ A = \begin{bmatrix}
  1 & 0 & -1 \\
  1 & 1 & 1
\end{bmatrix}, \vec{b} = \begin{bmatrix}
  4 \\ 2
\end{bmatrix}, \vec{w} = \begin{bmatrix}-1 & 1 & 1\end{bmatrix} \]
\[ \begin{bmatrix}
  1 & 0 & -1 & 4 \\
  1 & 1 & 1 & 2
\end{bmatrix} \to \begin{bmatrix}
  1 & 0 & -1 & 4 \\
  0 & 1 & 2 & -2
\end{bmatrix} \]
This augmented matrix has a solution, so \( \vec{b} \) is a linear combination
of the columns of \( A \), therefore it is in the column space of \( A \).
\[ \begin{bmatrix}
  -1 & 1 & 1 \\
  0 & 1 & 1 \\
  1 & 1 & -1
\end{bmatrix} \]
This system does not have a solution, so \( \vec{w} \) is not a linear
combination of the rows of \( A \), therefore it is not in the row space of
\( A \).

\subsubsection*{Question 35}
Determine whether \( \vec{b} \) is in \( col(A) \). Determine whether
\( \vec{w} \) is in \( row(A) \).
\[ A = \begin{bmatrix}
  1 & 1 & -5 \\
  0 & 2 & 1 \\
  1 & -1 & 6
\end{bmatrix}, \vec{b} = \begin{bmatrix}
  1 \\ 1 \\ 0
\end{bmatrix}, \vec{w} = \begin{bmatrix}2 & 4 & -9\end{bmatrix} \]
\[ \begin{bmatrix}
  1 & 1 & -5 & 1 \\
  0 & 2 & 1 & 1 \\
  1 & -1 & 6 & 0
\end{bmatrix} \to \begin{bmatrix}
  1 & 0 & 0 & \frac{1}{2} \\
  0 & 1 & 0 & \frac{1}{2} \\
  0 & 0 & 1 & 0
\end{bmatrix} \]
This augmented matrix has a solution, so \( \vec{b} \) is a linear combination
of the columns of \( A \), therefore it is in the column space of \( A \).
\[ \begin{bmatrix}
  -5 & 1 & 6 & -9 \\
  1 & 2 & -1 & 4 \\
  1 & 0 & 1 & 2
\end{bmatrix} \to \begin{bmatrix}
  1 & 0 & 0 & 2 \\
  0 & 1 & 0 & 1 \\
  0 & 0 & 1 & 0
\end{bmatrix} \]
This augmented matrix has a solution, so \( \vec{w} \) is a linear combination
of the rows of \( A \), therefore it is in the row space of \( A \).

\subsubsection*{Question 36}
If \( A = \begin{bmatrix}1 & 0 & -1 \\ 1 & 1 & 1\end{bmatrix} \), is
\( \vec{v} = \begin{bmatrix}-1 \\ 4 \\ -1\end{bmatrix} \) in \( null(A) \)?
\begin{align*}
  A\vec{v} &= \begin{bmatrix}
    (1)(-1)+(0)(4)+(-1)(-1) \\
    (1)(-1)+(1)(4)+(1)(-1) \\
  \end{bmatrix} \\
  &= \begin{bmatrix}
    0 \\ 2
  \end{bmatrix}
\end{align*}
\( \vec{v} \) is not in the null space of \( A \).

\subsubsection*{Question 37}
Give bases for \( row(A), col(A), null(A) \).
\[ A = \begin{bmatrix}
  2 & -4 & 2 & 1 & 0 \\
  -1 & 2 & 1 & 3 & 2 \\
  1 & -2 & 3 & 4 & 2
\end{bmatrix} \]
\begin{align*}
  RREF(A) &= \begin{bmatrix}
    1 & -2 & 0 & -\frac{5}{4} & -1 \\
    0 & 0 & 1 & \frac{7}{4} & 1 \\
    0 & 0 & 0 & 0 & 0
  \end{bmatrix} \\
  col(A) &= span\left(\begin{bmatrix}
    2 \\ -1 \\ 1
  \end{bmatrix},\begin{bmatrix}
    2 \\ 1 \\ 3
  \end{bmatrix}\right) \\
  x_1-2x_2-\frac{5}{4}x_4-x_5 &= 0 \\
  x_3+\frac{7}{4}x_4+x_5 &= 0 \\
  x_2 &= r \quad x_4 = s \quad x_5 = t \\
  x_1 &= 2r+\frac{5}{4}s+t \\
  x_3 &= -\frac{7}{4}s-t \\
  \begin{bmatrix} x_1 \\ x_2 \\ x_3 \\ x_4 \\ x_5\end{bmatrix} &=
    r\begin{bmatrix}2 \\ 1 \\ 0 \\ 0 \\ 0\end{bmatrix}+
    s\begin{bmatrix}\frac{5}{4} \\ 0 \\ -\frac{7}{4} \\ 1 \\ 0\end{bmatrix}+
    t\begin{bmatrix}1 \\ 0 \\ -1 \\ 0 \\ 1\end{bmatrix} \\
  null(A) &= span\left(
    \begin{bmatrix}2 \\ 1 \\ 0 \\ 0 \\ 0\end{bmatrix},
    \begin{bmatrix}\frac{5}{4} \\ 0 \\ -\frac{7}{4} \\ 1 \\ 0\end{bmatrix},
    \begin{bmatrix}1 \\ 0 \\ -1 \\ 0 \\ 1\end{bmatrix}
  \right)
\end{align*}

\subsubsection*{Question 38}
Find a basis \( \mathbb{B} \) for the span of the given vectors.
\[ \begin{bmatrix}4 \\ -5 \\ 1\end{bmatrix},
  \begin{bmatrix}1 \\ -1 \\ 0\end{bmatrix},
  \begin{bmatrix}2 \\ -2 \\ 1\end{bmatrix} \]
\begin{align*}
  RREF\left(\begin{bmatrix}
    4 & -5 & 1 \\
    1 & -1 & 0 \\
    2 & -2 & 1
  \end{bmatrix}\right) &= \begin{bmatrix}
    1 & 0 & -1 \\
    0 & 1 & -1 \\
    0 & 0 & 1
  \end{bmatrix} \\
  \mathbb{B} &= span\left(
    \begin{bmatrix}1 \\ 0 \\ -1\end{bmatrix},
    \begin{bmatrix}0 \\ 1 \\ -1\end{bmatrix},
    \begin{bmatrix}0 \\ 0 \\ 1\end{bmatrix}
  \right)
\end{align*}

\subsubsection*{Question 39}
Give the rank and nullity of the matrix.
\begin{align*}
  A &= \begin{bmatrix}
    1 & 1 & -3 \\
    0 & 2 & 1 \\
    1 & -1 & -4
  \end{bmatrix} \\
  RREF(A) &= \begin{bmatrix}
    1 & 0 & -\frac{7}{2} \\
    0 & 2 & \frac{1}{2} \\
    0 & 0 & 0
  \end{bmatrix} \\
  rank(A) &= 2 \\
  nullity(A) = 1
\end{align*}

\subsubsection*{Question 40}
Give the rank and nullity of the matrix below.
\begin{align*}
  A &= \begin{bmatrix}
    2 & -5 & 0 & 2 & 1 \\
    -1 & 2 & 1 & 2 & 4 \\
    1 & -3 & 1 & 4 & 5
  \end{bmatrix} \\
  RREF(A) &= \begin{bmatrix}
    1 & 0 & -5 & -14 & -22 \\
    0 & 1 & -2 & -6 & -9 \\
    0 & 0 & 0 & 0 & 0
  \end{bmatrix} \\
  rank(A) &= 2 \\
  x_1-5x_3-14x_4-22x_5 &= 0 \\
  x_2-2x_3-6x_4-9x_5 &= 0 \\
  nullity(A) &= 3 \quad (free~variables: x_3,x_4,x_5)
\end{align*}

\begin{center}
  If you have any questions, comments, or concerns, please contact me at
  alvin@omgimanerd.tech
\end{center}

\end{document}
