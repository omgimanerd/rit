\documentclass{math}

\title{Advanced Linear Algebra}
\author{Alvin Lin}
\date{January 2019 - May 2019}

\begin{document}

\maketitle

\section*{Subspaces Review}
Suppose we have the following:
\[ A\vec{x} = \vec{b} \]
We can represent the following in matrix form as:
\begin{align*}
  \begin{bmatrix}
    \vec{a_1} & \vec{a_2} & \dots & \vec{a_n}
  \end{bmatrix}\begin{bmatrix}
    x_1 \\ x_2 \\ \vdots x_n
  \end{bmatrix} &= \begin{bmatrix}
    b_1 \\ b_2 \\ \vdots b_n
  \end{bmatrix} \\
  x_1\vec{a_1}+x_2\vec{a_2}+\dots+x_n\vec{a_n} &= \vec{b}
\end{align*}
The span of the columns of \( A \) (all linear combinations of the columns)
is known as the column space of \( A \), while the span of the rows of \( A \)
(all linear combinations of the rows) is known as the row space. The system is
consistent if \( \vec{b} \) is in the column space of \( A \).

\subsubsection*{Example}
Find the row space, column space, and null space of \( A \).
\[ A = \begin{bmatrix}
  1 & 1 & 3 & 1 & 6 \\
  2 & -1 & 0 & 1 & -1 \\
  -3 & 2 & 1 & -2 & 1 \\
  4 & 1 & 6 & 1 & 3
\end{bmatrix} \]
The first step is to put the matrix into reduced row echelon form. For
simplicity, we will skip the row operation steps.
\begin{align*}
  rref(A) &= \begin{bmatrix}
    1 & 0 & 1 & 0 & -1 \\
    0 & 1 & 2 & 0 & 3 \\
    0 & 0 & 0 & 1 & 4 \\
    0 & 0 & 0 & 0 & 0
  \end{bmatrix} \\
  row(A) &= span\left\{
    \begin{bmatrix}1 \\ 0 \\ 1 \\ 0 \\ -1\end{bmatrix},
    \begin{bmatrix}0 \\ 1 \\ 2 \\ 0 \\ 3\end{bmatrix},
    \begin{bmatrix}0 \\ 0 \\ 0 \\ 1 \\ 4\end{bmatrix}
  \right\} \\
  col(A) &= span\left\{
    \begin{bmatrix}1 \\ 2 \\ -3 \\ 4\end{bmatrix},
    \begin{bmatrix}1 \\ -1 \\ 2 \\ 1\end{bmatrix},
    \begin{bmatrix}1 \\ 1 \\ -2 \\ 1\end{bmatrix}
  \right\}
\end{align*}
The row space of \( A \) is determined by the span of the nonzero rows. The
column space of \( A \) is determined by the span of the original columns which
correspond to columns with leading ones in reduced row echelon form. This
involves the first, second, and fourth columns. We can justify this by showing
that the third and fifth columns are linear combinations of the others. We need
the null space of \( A \) in order to determine this.
\begin{align*}
  null(A) &: A\vec{x} = \vec{0} \\
  &: x_1\vec{a_1}+x_2\vec{a_2}+\dots+x_n\vec{a_n} = \vec{0} \\
\end{align*}
We can compute the reduced row echelon form of the augmented matrix (taking
\( \vec{b} = \vec{0} \)).
\[ \begin{bmatrix}
  1 & 1 & 3 & 1 & 6 & 0 \\
  2 & -1 & 0 & 1 & -1 & 0 \\
  -3 & 2 & 1 & -2 & 1 & 0 \\
  4 & 1 & 6 & 1 & 3 & 0
\end{bmatrix} \to \begin{bmatrix}
  1 & 0 & 1 & 0 & -1 & 0 \\
  0 & 1 & 2 & 0 & 3 & 0 \\
  0 & 0 & 0 & 1 & 4 & 0 \\
  0 & 0 & 0 & 0 & 0 & 0
\end{bmatrix} \]
We can rewrite our null space as a system of equations.
\begin{align*}
  x_1+x_3-x_5 &= 0 \\
  x_2+2x_3+3x_5 &= 0 \\
  x_4+4x_5 &= 0
\end{align*}
If we let \( x_3 = s, x_5 = t \), we can write our null space as a span of
two vectors.
\begin{align*}
  \begin{bmatrix}x_1 \\ x_2 \\ x_3 \\ x_4 \\ x_5\end{bmatrix} &=
    s\begin{bmatrix}-1 \\ -2 \\ 1 \\ 0 \\ 0\end{bmatrix}+
    t\begin{bmatrix}1 \\ -3 \\ 0 \\ -4 \\ 1\end{bmatrix},
      \quad \infty<s,t<\infty \\
  null(A) &= span\left\{
    \begin{bmatrix}-1 \\ -2 \\ 1 \\ 0 \\ 0\end{bmatrix},
    \begin{bmatrix}1 \\ -3 \\ 0 \\ -4 \\ 1\end{bmatrix}
  \right\}
\end{align*}
The entries in the null space represent the column dependencies of \( A \) and
the reduced row echelon form of \( A \).
\[ A\vec{x} = \vec{0} \quad\forall\quad \vec{x}\in null(A) \]
The entries in the null space describe how we can represent the third and fifth
column as linear combinations of the others.
\begin{align*}
  \begin{bmatrix}-1 \\ -2 \\ 1 \\ 0 \\ 0\end{bmatrix} &\to
    -\vec{x_1}-2\vec{x_2}+\vec{x_3} = \vec{0} \\
  & -\vec{x_1}-2\vec{x_1} = \vec{x_3} \\
  \begin{bmatrix}1 \\ -3 \\ 0 \\ -4 \\ 1\end{bmatrix} &\to
    \vec{x_1}-3\vec{x_2}-4\vec{x_4}+\vec{x_5} = \vec{0} \\
  & \vec{x_1}-3\vec{x_2}-4\vec{x_4} = -\vec{x_5}
\end{align*}
The dimensionality of the null space is known as the nullity of \( A \). In this
case, \( dim(null(A)) = 2 \). Another surprising fact is that the the row space
is orthogonal to the null space.
\begin{align*}
  \begin{bmatrix}\vec{r_1} \\ \vec{r_2} \\ \vdots \\ \vec{r_m}\end{bmatrix}
    \begin{bmatrix}x_1 \\ x_2 \\ \vdots \\ x_n\end{bmatrix} &=
    \begin{bmatrix}0 \\ 0 \\ \vdots \\ 0\end{bmatrix} \\
  \begin{bmatrix}
    \vec{r_1}\cdot\vec{x} \\
    \vec{r_2}\cdot\vec{x} \\
    \vdots \\
    \vec{r_m}\cdot\vec{x}
  \end{bmatrix} &= \begin{bmatrix}0 \\ 0 \\ \vdots \\ 0\end{bmatrix}
\end{align*}
If \( \vec{x} \) is in the null space, then it must be perpendicular to the
rows, and therefore it must be perpendicular to the row space of \( A \).
\begin{align*}
  row(A) &= span\left\{\vec{r_1},\vec{r_2},\dots,\vec{r_n}\right\} \\
  null(A) &= \left\{\vec{x}\mid A\vec{x} = \vec{0}\right\} \\
  row(A) &\bot null(A)
\end{align*}

\begin{center}
  You can find all my notes at \url{http://omgimanerd.tech/notes}. If you have
  any questions, comments, or concerns, please contact me at
  alvin@omgimanerd.tech
\end{center}

\end{document}
