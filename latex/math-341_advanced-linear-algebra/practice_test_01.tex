\documentclass{math}

\usepackage{enumerate}

\title{Advanced Linear Algebra}
\author{Alvin Lin}
\date{January 2019 - May 2019}

\begin{document}

\maketitle

\section*{Practice Test 1}

\subsubsection*{Question 1}
Find the equation of the plane passing through \( (0, 1, 0) \) with normal
vector \( \vec{n} = \langle3,2,1\rangle \).
\begin{align*}
  \vec{n}\cdot\vec{x} &= \vec{n}\cdot\vec{p} \\
  \langle3,2,1\rangle\cdot\vec{x} &=
    \langle3,2,1\rangle\cdot\langle0,1,0\rangle \\
  3x+2y+z &= 0+2+0 \\
  3x+2y+z &= 2
\end{align*}

\subsubsection*{Question 2}
Consider the plane \( x+y-z = 1 \).
\begin{enumerate}[(a)]
  \item What is the normal vector \( \vec{n} \) to the plane?
  \[ \vec{n} = \langle1,1,-1\rangle \]
  \item Consider the point on the plane \( A(1,0,0) \) and the point off of the
    plane \( B(1,0,2) \). Find the projection of the vector
    \( \overrightarrow{AB} \) onto \( \vec{n} \).
  \begin{align*}
    \overrightarrow{AB} &= \langle1-1,0-0,2-0\rangle = \langle0,0,2\rangle \\
    proj_{\vec{n}}\overrightarrow{AB} &= \frac{\overrightarrow{AB}\cdot\vec{n}}
      {\vec{n}\cdot\vec{n}}\vec{n} \\
    &= \frac{0+0-2}{1+1+1}\langle1,1,-1\rangle \\
    &= \langle-\frac{2}{3},-\frac{2}{3},\frac{2}{3}\rangle
  \end{align*}
  \item What is the distance from point \( B \) to the plane?
  \begin{align*}
    \vec{p} &= \vec{v}-c\vec{n} \\
    c &= \frac{\vec{n}\cdot\vec{v}}{\vec{n}\cdot\vec{n}} \\
    &= -\frac{2}{3}
  \end{align*}
\end{enumerate}

\subsubsection*{Question 3}
Consider the matrix
\[ A = \begin{bmatrix}
  1 & 2 & 0 & 1 \\
  2 & 4 & 1 & 4 \\
  3 & 6 & 3 & 9
\end{bmatrix} \]
\begin{enumerate}
  \item Find the reduced row echelon form \( R = rref(A) \). What is the rank
    of A?
  \begin{align*}
    A &= \begin{bmatrix}
      1 & 2 & 0 & 1 \\
      2 & 4 & 1 & 4 \\
      3 & 6 & 3 & 9
    \end{bmatrix} \\
    &= \begin{bmatrix}
      0 & 0 & -1 & -2 \\
      0 & 0 & -1 & -2 \\
      1 & 2 & 1 & 3
    \end{bmatrix} (R_1-R_3 \to R_1, R_2-2R_3 \to R_2) \\
    &= \begin{bmatrix}
      1 & 2 & 0 & 1 \\
      0 & 0 & 1 & 2 \\
      0 & 0 & 0 & 0
    \end{bmatrix} (R_1 \leftrightarrow R_2, -R_2 \to R_2) \\
    rank(A) &= 2
  \end{align*}
  \item Find a basis for the null space of \( A \).
  \begin{align*}
    A\vec{x} &= 0 \\
    \begin{bmatrix}
      1 & 2 & 0 & 1 & 0 \\
      0 & 0 & 1 & 2 & 0 \\
      0 & 0 & 0 & 0 & 0
    \end{bmatrix} &\quad x_1+2x_2+x_4 = 0 \quad x_3+x_4 = 0 \\
    x_2 &= s \quad x_4 = t \\
    x_1 &= -2s-t \\
    x_3 &= -t \\
    \vec{x} &= s\begin{bmatrix}-2 \\ 1 \\ 0 \\ 0\end{bmatrix}+
      t\begin{bmatrix}-1 \\ 0 \\ -1 \\ 1\end{bmatrix} \\
    null(A) &= span\left(\begin{bmatrix}-2 \\ 1 \\ 0 \\ 0\end{bmatrix}+
      t\begin{bmatrix}-1 \\ 0 \\ -1 \\ 1\end{bmatrix}\right)
  \end{align*}
  \item If the vector \( \vec{b} \) is the sum of the four columns of \( A \),
  write down the complete solution to \( A\vec{x} = \vec{b} \).
  \begin{align*}
    A\vec{x} &= \begin{bmatrix}4 \\ 11 \\ 21\end{bmatrix} \\
    \vec{x} &= \begin{bmatrix}1 \\ 1 \\ 1 \\ 1\end{bmatrix}
  \end{align*}
\end{enumerate}

\subsubsection*{Question 4}
Suppose row operations change \( A\vec{x} = \vec{b} \) to a row reduced
\( R\vec{x} = \vec{d} \) and the complete solution is
\[ \vec{x} = \begin{bmatrix}4 \\ 0 \\ 0\end{bmatrix}+
  c_1\begin{bmatrix}2 \\ 1 \\ 0\end{bmatrix}+
  c_1\begin{bmatrix}5 \\ 0 \\ 1\end{bmatrix} \]
then what is the 3 by 3 reduced row echelon matrix \( R \) and what is
\( \vec{d} \)?
\begin{align*}
  x &= 4+2c_1+5x_2 \\
  y &= c_1 \\
  z &= c_2 \\
  \begin{bmatrix}x \\ y \\ z\end{bmatrix} &= \begin{bmatrix}
    4+2x+5t \\ s \\ t
  \end{bmatrix} \\
  x-2w-5y &= 4 \\
  R &= \begin{bmatrix}
    1 & -2 & 5 \\
    0 & 0 & 0 \\
    0 & 0 & 0
  \end{bmatrix} \\
  \vec{d} &= \begin{bmatrix}4 \\ 0 \\ 0\end{bmatrix}
\end{align*}

\subsubsection*{Question 5}
Suppose \( A \) is the matrix
\[ \begin{bmatrix}
  2 & 1 \\
  6 & 5 \\
  2 & 4
\end{bmatrix} \]
Is the vector \( \vec{b} = \begin{bmatrix}8 \\ 28 \\ 14\end{bmatrix} \) in
the column space of \( A \)?
\begin{align*}
  rref\left(\begin{bmatrix}
    2 & 1 & 8 \\
    6 & 5 & 28 \\
    2 & 4 & 14
  \end{bmatrix}\right) &= \begin{bmatrix}
    1 & 0 & 0 \\
    0 & 1 & 0 \\
    0 & 0 & 1
  \end{bmatrix}
\end{align*}
This system is inconsistent and would imply \( 0 = 1 \), so \( \vec{b} \) is
not in the column space of \( A \).

\subsubsection*{Question 6}
Use row operations to find the inverse of
\[ A = \begin{bmatrix}
  1 & 2 & -1 \\
  2 & 2 & 4 \\
  1 & 3 & -3
\end{bmatrix} \]
\begin{align*}
  \left[\begin{array}{ccc|ccc}
    1 & 2 & -1 & 1 & 0 & 0 \\
    2 & 2 & 4 & 0 & 1 & 0 \\
    1 & 3 & -3 & 0 & 0 & 1
  \end{array}\right] &= \left[\begin{array}{ccc|ccc}
    1 & 2 & -1 & 1 & 0 & 0 \\
    0 & -2 & 6 & -2 & 1 & 0 \\
    0 & 1 & -2 & -1 & 0 & 1
  \end{array}\right] (R_2-2R_1 \to R_2, R_3-R_1 \to R_3) \\
  &= \left[\begin{array}{ccc|ccc}
    1 & 0 & 3 & 3 & 0 & -2 \\
    0 & 0 & 2 & -1 & 1 & 2 \\
    0 & 1 & -2 & -1 & 0 & 1
  \end{array}\right] (R_1-2R_3 \to R_1, R_2+2R_3 \to R_3) \\
  &= \left[\begin{array}{ccc|ccc}
    1 & 0 & 3 & 3 & 0 & -2 \\
    0 & 0 & 1 & -2 & \frac{1}{2} & 1 \\
    0 & 1 & -2 & -1 & 0 & 1
  \end{array}\right] (\frac{1}{2}R_2 \to R_2) \\
  &= \left[\begin{array}{ccc|ccc}
    1 & 0 & 0 & 9 & -\frac{3}{2} & -5 \\
    0 & 1 & 0 & -5 & 1 & 3 \\
    0 & 0 & 1 & -2 & \frac{1}{2} & 1
  \end{array}\right] (R_3+2R_2 \to R_3, R_1-3R_2\to R_1,
    R_2\leftrightarrow R_3)
\end{align*}

\subsubsection*{Question 7}
Let \( A = \begin{bmatrix}
  1 & 1 & 1 & 1 \\
  1 & 1 & 1 & 1 \\
  1 & 1 & 1 & 0 \\
  1 & 1 & 0 & 0
\end{bmatrix} \) and \( B = \begin{bmatrix}
  0 & 1 & 1 & 1 \\
  1 & 1 & 1 & 1 \\
  1 & 1 & 1 & 0 \\
  1 & 1 & 0 & 0
\end{bmatrix} \)
\begin{enumerate}
  \item Find \( \det(AB) \).
  \begin{align*}
    \det(AB) &= \det(A)\det(B) \\
    \det(A) &= -\begin{vmatrix}
      1 & 1 & 1 \\
      1 & 1 & 1 \\
      1 & 1 & 0
    \end{vmatrix}+\begin{vmatrix}
      1 & 1 & 1 \\
      1 & 1 & 1 \\
      1 & 1 & 0
    \end{vmatrix}-0+0 = 0 \\
    \det(AB) &= 0 \\
  \end{align*}
  \item Find \( \det(B) \).
  \begin{align*}
    \det(B) &= -\begin{vmatrix}
      1 & 1 & 1 \\
      1 & 1 & 1 \\
      1 & 1 & 0
    \end{vmatrix}+\begin{vmatrix}
      0 & 1 & 1 \\
      1 & 1 & 1 \\
      1 & 1 & 0
    \end{vmatrix}-0+0 \\
    &= 0+\begin{vmatrix}
      1 & 1 \\ 1 & 1
    \end{vmatrix}-\begin{vmatrix}
      0 & 1 \\ 1 & 1
    \end{vmatrix} \\
    &= 1
  \end{align*}
\end{enumerate}

\subsubsection*{Question 8}
Let \( A = \begin{bmatrix}0 & -1 \\ 4 & 0\end{bmatrix} \).
\begin{enumerate}
  \item Find the eigenvalues of \( A \).
  \begin{align*}
    \det(A-\lambda I) &= \begin{vmatrix}
      -\lambda & -1 \\
      4 & -\lambda
    \end{vmatrix} \\
    &= \lambda^2+4 = 0 \\
    \lambda &= \pm2i
  \end{align*}
  \item Find the eigenvectors of \( A \).
  \begin{align*}
    (A-(2i)I)\vec{x} &= 0 \\
    \left[\begin{array}{cc|c}
      -2i & -1 & 0 \\
      4 & -2i & 0
    \end{array}\right] &\to \left[\begin{array}{cc|c}
      2 & -i & 0 \\
      0 & 0 & 0
    \end{array}\right] \\
    \vec{x} &= s\begin{bmatrix}\frac{1}{2}i \\ 1\end{bmatrix} \\
    (A+(2i)I)\vec{x} &= 0 \\
    \left[\begin{array}{cc|c}
      2i & -1 & 0 \\
      4 & 2i & 0
    \end{array}\right] &\to \left[\begin{array}{cc|c}
      2 & i & 0 \\
      0 & 0 & 0
    \end{array}\right] \\
    \vec{x} &= s\begin{bmatrix}-\frac{1}{2}i \\ 1\end{bmatrix}
  \end{align*}
\end{enumerate}

\subsubsection*{Question 9}
Solve the linear system of coupled ODEs.
\begin{align*}
  \ddiff{x}{t} &= -y \quad \ddiff{y}{t} = 4x \\
  \ddiff{}{t}\begin{bmatrix}x \\ y\end{bmatrix} &= \begin{bmatrix}
    0 & -1 \\ 4 & 0
  \end{bmatrix}\begin{bmatrix}x \\ y\end{bmatrix} \\
  D &= \begin{bmatrix}
    2i & 0 \\ 0 & -2i
  \end{bmatrix} \\
  P &= \begin{bmatrix}
    -\frac{1}{2}i & -\frac{1}{2}i \\ 1 & 1
  \end{bmatrix} \\
  \ddiff{}{t}(P^{-1}\vec{x}) &= D(P^{-1}\vec{x}) \quad
    \vec{z} = P^{-1}\vec{x} \\
  \ddiff{\vec{z}}{t} &= D\vec{z} \\
  \ddiff{}{t}\begin{bmatrix}z_1 \\ z_2\end{bmatrix} &=
    \begin{bmatrix}2i & 0 \\ 0 & -2i\end{bmatrix}
    \begin{bmatrix}z_1 \\ z_2\end{bmatrix} \\
  \ddiff{z_1}{t} &= 2iz_1 \quad z_1 = c_1\e^{2it} \\
  \ddiff{z_2}{t} &= -2iz_2 \quad z_2 = c_2\e^{-2it} \\
  \vec{x} &= P\vec{z} \\
  &= \begin{bmatrix}
    -\frac{1}{2}i & -\frac{1}{2}i \\ 1 & 1
  \end{bmatrix}\begin{bmatrix}
    c_1\e^{2it} \\ c_2\e^{-2it}
  \end{bmatrix} \\
  &= \begin{bmatrix}
    -\frac{1}{2}ic_1\e^{2it}-\frac{1}{2}c_2\e^{-2it} \\
    c_1\e^{2it}+c_2\e^{-2it}
  \end{bmatrix}
\end{align*}

\subsubsection*{Question 10}
Solve the system of coupled ODEs.
\begin{align*}
  \ddiff{x}{t} &= 3x-4y \quad x(0) = 1 \\
  \ddiff{y}{t} &= 2x-3y \quad y(0) = 0 \\
  \ddiff{}{t}\begin{bmatrix}x \\ y\end{bmatrix} &=
    \begin{bmatrix}3 & -4 \\ 2 & -3\end{bmatrix}
    \begin{bmatrix}x \\ y\end{bmatrix} \\
  &= (PDP^{-1})\begin{bmatrix}x \\ y\end{bmatrix} \\
  A &= \begin{bmatrix}3 & -4 \\ 2 & -3\end{bmatrix} = PDP^{-1} \\
  P &= \begin{bmatrix}-2 & 1 \\ 1 & 1\end{bmatrix} \\
  D &= \begin{bmatrix}1 & 0 \\ 0 & -1\end{bmatrix} \\
  \ddiff{(P^{-1}\vec{x})}{t} &= D(P^{-1}\vec{x}) \quad
    \vec{z} = P^{-1}\vec{x} \\
  \ddiff{}{t}\begin{bmatrix}z_1 \\ z_2\end{bmatrix} &=
    \begin{bmatrix}1 & 0 \\ 0 & -1\end{bmatrix}
    \begin{bmatrix}z_1 \\ z_2\end{bmatrix} \\
  \ddiff{z_1}{t} &= z_1 \quad z_1 = c_1\e^{t} \\
  \ddiff{z_2}{t} &= -x_2 \quad x_2 = c_2\e^{-t} \\
  \vec{x} &= P\vec{z} \\
  &= \begin{bmatrix}-2 & 1 \\ 1 & 1\end{bmatrix}\begin{bmatrix}
    c_1\e^t \\ c_2\e^{-t}
  \end{bmatrix} \\
  &= \begin{bmatrix}
    -2c_1\e^t+c_2\e^{-t} \\ c_1\e^t+c_2\e^{-t}
  \end{bmatrix} \\
  x(0) &= 1 = -2c_1+c_2 \\
  y(0) &= 0 = c_1+c_2 \\
  c_1 &= -\frac{1}{3} \quad c_2 = \frac{1}{3} \\
  \vec{x} &= \begin{bmatrix}
    \frac{2}{3}\e^t+\frac{1}{3}\e^{-t} \\
    -\frac{1}{3}\e^t+\frac{1}{3}\e^{-t}
  \end{bmatrix}
\end{align*}

\subsubsection*{Question 11}
Consider
\[ A = \begin{bmatrix}
  0 & 0 & -2 \\
  1 & 2 & 1 \\
  1 & 0 & 3
\end{bmatrix} \]
Find matrices \( P \) and \( D \) such that \( D = P^{-1}AP \).
\begin{align*}
  \det(A-\lambda I) &= \begin{vmatrix}
    -\lambda & 0 & -2 \\
    1 & 2-\lambda & 1 \\
    1 & 0 & 3-\lambda
  \end{vmatrix} \\
  &= -\lambda\begin{vmatrix}2-\lambda & 1 \\ 0 & 3-\lambda\end{vmatrix}-
    2\begin{vmatrix}1 & 2-\lambda \\ 1 & 0\end{vmatrix} \\
  &= -\lambda^3+5\lambda^2-8\lambda+4 \\
  &= (\lambda-1)(\lambda-2)(\lambda-2) = 0 \\
  \lambda &= 1 \\
  (A-1I)\vec{v_1} &= \vec{0} \\
  \left[\begin{array}{ccc|c}
    -1 & 0 & -2 & 0 \\
    1 & 1 & 1 & 0 \\
    1 & 0 & 2 & 0
  \end{array}\right] &\to \left[\begin{array}{ccc|c}
    1 & 0 & 2 & 0 \\
    0 & 1 & -1 & 0 \\
    0 & 0 & 0 & 0
  \end{array}\right] \\
  \vec{v_1} &= s\begin{bmatrix}-2 \\ 1 \\ 1\end{bmatrix} \\
  \lambda &= 2 \\
  (A-2I)\vec{v_2} &= \vec{0} \\
  \left[\begin{array}{ccc|c}
    -2 & 0 & -2 & 0 \\
    1 & 0 & 1 & 0 \\
    1 & 0 & 1 & 0
  \end{array}\right] &\to \left[\begin{array}{ccc|c}
    1 & 0 & 1 & 0 \\
    0 & 0 & 0 & 0 \\
    0 & 0 & 0 & 0
  \end{array}\right] \\
  \vec{v_2} &= s\begin{bmatrix}-1 \\ 0 \\ 1\end{bmatrix}+
    t\begin{bmatrix}0 \\ 1 \\ 0\end{bmatrix} \\
  D &= \begin{bmatrix}
    1 & 0 & 0 \\
    0 & 2 & 0 \\
    0 & 0 & 2
  \end{bmatrix} \quad P = \begin{bmatrix}
    -2 & -1 & 0 \\
    1 & 0 & 1 \\
    1 & 1 & 0
  \end{bmatrix}
\end{align*}

\begin{center}
  You can find all my notes at \url{http://omgimanerd.tech/notes}. If you have
  any questions, comments, or concerns, please contact me at
  alvin@omgimanerd.tech
\end{center}

\end{document}
