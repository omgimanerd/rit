\documentclass{math}

\usepackage{enumerate}

\title{Advanced Linear Algebra}
\author{Alvin Lin}
\date{January 2019 - May 2019}

\begin{document}

\maketitle

\section*{Practice Test 1}

\subsubsection*{Question 1}
Find the equation of the plane passing through \( (0, 1, 0) \) with normal
vector \( \vec{n} = \langle3,2,1\rangle \).
\begin{align*}
  \vec{n}\cdot\vec{x} &= \vec{n}\cdot\langle0,1,0\rangle \\
  3x+2y+z &= 0+2+0 \\
  3x+2y+z &= 2
\end{align*}

\subsubsection*{Question 2}
Consider the plane \( x+y-z = 1 \).
\begin{enumerate}[(a)]
  \item What is the normal vector \( \vec{n} \) to the plane?
  \[ \vec{n} = \langle1,1,-1\rangle \]
  \item Consider the point on the plane \( A(1,0,0) \) and the point off of the
    plane \( B(1,0,2) \). Find the projection of the vector
    \( \overrightarrow{AB} \) onto \( \vec{n} \).
  \begin{align*}
    \overrightarrow{AB} &= \langle1-1,0-0,2-0\rangle = \langle0,0,2\rangle \\
    proj_{\vec{n}}\overrightarrow{AB} &= \frac{\overrightarrow{AB}\cdot\vec{n}}
      {\vec{n}\cdot\vec{n}}\vec{n} \\
    &= \frac{0+0-2}{1+1+1}\langle0,0,2\rangle \\
    &= \langle0,0,-\frac{2}{3}\rangle
  \end{align*}
  \item What is the distance from point \( B \) to the plane?
\end{enumerate}

\subsubsection*{Question 3}
Consider the matrix
\[ A = \begin{bmatrix}
  1 & 2 & 0 & 1 \\
  2 & 4 & 1 & 4 \\
  3 & 6 & 3 & 9
\end{bmatrix} \]
\begin{enumerate}
  \item Find the reduced row echelon form \( R = rref(A) \). What is the rank
    of A?
  \begin{align*}
    A &= \begin{bmatrix}
      1 & 2 & 0 & 1 \\
      2 & 4 & 1 & 4 \\
      3 & 6 & 3 & 9
    \end{bmatrix} \\
    &= \begin{bmatrix}
      1 & 2 & 0 & 1 \\
      2 & 4 & 1 & 4 \\
      1 & 2 & 1 & 3
    \end{bmatrix} (\frac{1}{3}R_3 \to R_3) \\
    &= \begin{bmatrix}
      0 & 0 & -1 & -2 \\
      0 & 0 & -2 & -2 \\
      1 & 2 & 1 & 3
    \end{bmatrix} (R_1-R_3 \to R_1, R_2-R_3 \to R_2) \\
    &= \begin{bmatrix}
      1 & 2 & 1 & 3 \\
      0 & 0 & 1 & 2 \\
      0 & 0 & 1 & 1 \\
    \end{bmatrix} (-R_1 \to R_1,-R_2 \to R_2, R_1 \leftrightarrow R_3) \\
    &= \begin{bmatrix}
      1 & 2 & 0 & 2 \\
      0 & 0 & 0 & 1 \\
      0 & 0 & 1 & 1 \\
    \end{bmatrix} (R_2-R_3 \to R_2, R_1-R_3 \to R_1) \\
    &= \begin{bmatrix}
      1 & 2 & 0 & 0 \\
      0 & 0 & 1 & 0 \\
      0 & 0 & 0 & 1 \\
    \end{bmatrix} (R_2\leftrightarrow R_3, R_1-2R_3 \to R_1) \\
    rank(A) = 3
  \end{align*}
  \item Find a basis for the null space of \( A \).
  \begin{align*}
    A\vec{x} &= 0 \\
    \begin{bmatrix}
      1 & 2 & 0 & 0 & 0 \\
      0 & 0 & 1 & 0 & 0 \\
      0 & 0 & 0 & 1 & 0 \\
    \end{bmatrix} &\to (x_1+2x_2 = 0, x_3 = 0, x_4 = 0) \\
    x_1 &= -2x_2 \quad x_2 = s \\
    \vec{x} &= s\begin{bmatrix}
      -2 \\ 1 \\ 0 \\ 0
    \end{bmatrix} \\
    null(A) &= span\left(\begin{bmatrix}
      -2 \\ 1 \\ 0 \\ 0
    \end{bmatrix}\right)
  \end{align*}
  \item If the vector \( \vec{b} \) is the sum of the four columns of \( A \),
  write down the complete solution to \( A\vec{x} = \vec{b} \).
  \begin{align*}
    A\vec{x} &= \begin{bmatrix}6 \\ 12 \\ 4 \\ 14\end{bmatrix} \\
    A\begin{bmatrix}1 \\ 1 \\ 1 \\ 1\end{bmatrix} &=
      \begin{bmatrix}6 \\ 12 \\ 4 \\ 14\end{bmatrix}
  \end{align*}
\end{enumerate}

\subsubsection*{Question 4}
Suppose row operations change \( A\vec{x} = \vec{b} \) to a row reduced
\( R\vec{x} = \vec{d} \) and the complete solution is
\[ \vec{x} = \begin{bmatrix}4 \\ 0 \\ 0\end{bmatrix}+
  c_1\begin{bmatrix}2 \\ 1 \\ 0\end{bmatrix}+
  c_1\begin{bmatrix}5 \\ 0 \\ 1\end{bmatrix} \]
then what is the 3 by 3 reduced row echelon matrix \( R \) and what is
\( \vec{d} \)?

\subsubsection*{Question 5}
Suppose \( A \) is the matrix
\[ \begin{bmatrix}
  2 & 1 \\
  6 & 5 \\
  2 & 4
\end{bmatrix} \]
Is the vector \( \vec{b} = \begin{bmatrix}8 \\ 28 \\ 14\end{bmatrix} \) in
the column space of \( A \)?
\begin{align*}
  rref(A) &= \begin{bmatrix}
    1 & 0 \\
    0 & 1 \\
    0 & 0
  \end{bmatrix} \\
  col(A) &= span\left(\begin{bmatrix}2 \\ 6 \\ 2\end{bmatrix},
    \begin{bmatrix}1 \\ 5 \\ 4\end{bmatrix}\right) \\
  \begin{bmatrix}8 \\ 28 \\ 14\end{bmatrix} &=
    c_1\begin{bmatrix}2 \\ 6 \\ 2\end{bmatrix}+
    c_2\begin{bmatrix}1 \\ 5 \\ 4\end{bmatrix} \\
  8 &= 2c_1+c_1 \\
  28 &= 6c_1+5c_2 \\
  14 &= 2c_1+4c_2
\end{align*}
This system of equations is inconsistent, so \( \vec{b} \) is not in the
column space of \( A \).

\begin{center}
  You can find all my notes at \url{http://omgimanerd.tech/notes}. If you have
  any questions, comments, or concerns, please contact me at
  alvin@omgimanerd.tech
\end{center}

\end{document}
