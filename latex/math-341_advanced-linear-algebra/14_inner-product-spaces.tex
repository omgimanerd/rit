\documentclass{math}

\title{Advanced Linear Algebra}
\author{Alvin Lin}
\date{January 2019 - May 2019}

\begin{document}

\maketitle

\section*{Inner Product Spaces}
An inner product is an operation that assigns to every pair \( u,v \) in \( V \)
a real number \( \langle u,v\rangle \) such that:
\begin{enumerate}
  \item \( \langle u,v\rangle = \langle v,u\rangle \)
  \item \( \langle u,v+w\rangle = \langle u,v\rangle+\langle u,w\rangle \)
  \item \( \langle cu,v\rangle = c\langle u,v\rangle \)
  \item \( \langle u,v\rangle \ge 0 \) and \( \langle u,u\rangle = 0 \) if
    \( u = 0 \)
\end{enumerate}
\( \langle \vec{u},\vec{v}\rangle = \vec{u}\cdot\vec{v} \) is an inner product.
Let \( \vec{u} = \begin{bmatrix}u_1 \\ u_2\end{bmatrix} \) and \( \vec{v} =
\begin{bmatrix}v_1 \\ v_2\end{bmatrix} \). We can show that
\( \langle u,v\rangle = 2u_1v_1+3u_2v_2 \) defines an inner product.
\begin{enumerate}
  \item This satisfies the first property by inspection.
  \item By the following, this definition satisfies the second property:
  \begin{align*}
    \langle u,v+w\rangle &= 2u_1(v_1+w_1)+3u_2(v_2+w_3) \\
    &= 2u_1v_1+3u_2v_2+2u_1w_1+3u_2w_2 \\
    &= \langle u,v\rangle+\langle u,w\rangle
  \end{align*}
  \item This satisfies the third property by inspection.
  \item This satisfies the fourth property by inspection.
\end{enumerate}

\subsection*{Weighted Dot Product}
The weighted dot product \( \langle u,v\rangle \) is defined as:
\begin{align*}
  \langle \vec{u},\vec{v}\rangle &= w_1u_1v_1+\dots+w_nu_nv_n \\
  &= \vec{u}^TW\vec{v}
\end{align*}
where \( W \) is a matrix with the weights along the diagonal and everything
else is 0.

\subsubsection*{Example}
Let \( f,g \) in \( C[a,b] \).
\[ \langle f,g\rangle = \int_{a}^{b}f(x)g(x)\diff{x} \]
\begin{enumerate}
  \item This satisfies the first property by inspection.
  \item \( \langle f,g+h\rangle = \int_{a}^{b}f(g+h)\diff{x} =
    \int_{a}^{b}fg\diff{x}+\int_{a}^{b}fh\diff{x} = \langle f,g\rangle+
    \langle f,h\rangle \)
  \item This satisfies the third property by inspection.
  \item This satisfies the fourth property by inspection.
\end{enumerate}

\subsection*{Definitions}
\begin{enumerate}
  \item The length (or norm) is \( \|\vec{v}\| = \sqrt{\langle v,v\rangle} \)
  \item The distance is \( d(\vec{u},\vec{v}) = \|\vec{u}-\vec{v}\| \)
  \item \( \vec{u} \) and \( \vec{v} \) are orthogonal if \( \langle u,v\rangle
    = 0 \)
\end{enumerate}
Suppose in \( C[0,1] \), we define \( f(x) = x, g(x) = 3x-2 \).
\begin{enumerate}
  \item \( \|f\|^2 = \langle f,f\rangle = \int_{0}^{1}f^2\diff{x} =
    \int_{0}^{1}x^2\diff{x} = \frac{1}{3} \)
  \item \( d(f,g) = \|f-g\| = \sqrt{\langle f-g,f-g\rangle} =
    \sqrt{\int_{0}^{1}(f-g)^2\diff{x}} = \sqrt{\int_{0}^{1}(-2x+2)^2\diff{x}} =
    \frac{2}{\sqrt{3}} \)
  \item \( \langle f,g\rangle = \int_{0}^{1}fg\diff{x} =
    \int_{0}^{1}x(3x-2)\diff{x} = \int_{0}^{1}(3x^2-2x)\diff{x} =
    \left[x^2-x^2\right]_{0}^{1} = 0 \) \\
    (\( f \) and \( g \) are orthogonal)
\end{enumerate}

\subsection*{Pythagoras' Theorem}
\begin{align*}
  \|u+v\|^2 &= \langle u+v,u+v\rangle \\
  &= \langle u,u\rangle+2\langle u,v\rangle+\langle v,v\rangle \\
  &= \|u\|^2+2\langle u,v\rangle+\|v\|^2
\end{align*}

\subsubsection*{Example}
Construct an orthogonal basis for \( P_2 \) with respect to the inner product
\[ \langle f,g\rangle = \int_{-1}^{1}f(x)g(x)\diff{x} \]
by applying Gram-Schmidt to \( \{1,x,x^2\} \).
\begin{align*}
  v_1 &= x_1 = 1 \quad W_1 = span(v_1) \\
  v_2 &= x_2-\text{proj}_{W_1}x_2 \\
  &= x_2-\left(\frac{\langle x_1,x_2\rangle}{\langle v_1,v_1\rangle}\right) \\
  &= x_2-\left(
    \frac{\int_{-1}^{1}(1)(x)\diff{x}}{\int_{-1}^{1}(1)^2\diff{x}}\right) \\
  &= x_2-\left(\frac{0}{2}\right)(1) \\
  &= x \quad W_2 = span(1,x) \\
  v_3 &= x_3-\text{proj}_{W_2}x_3 \\
  &= x_3-\frac{\langle v_1,x_3\rangle}{\langle v_1,v_1\rangle} v_1-
    \frac{\langle v_2,x_3\rangle}{\langle v_2,v_3\rangle}v_2 \\
  &= x^2-\left(\frac{\frac{2}{3}}{2}\right)1-
    \left(\frac{0}{\frac{2}{3}}\right)x \\
  &= x^2-\frac{1}{3} \\
  P_2 &= span\left\{1,x,x^2-\frac{1}{3}\right\}
\end{align*}
Historical note: these three elements for the basis of \( P_2 \) are the first
three Legendre polynomials.

\subsection*{Cauchy-Schwarz Inequality and Triangle Inequality}
\[ |\langle u,v\rangle| = \|u\|\|v\| \]
In abstract vector spaces, the dot product \( \R^n \) is
\[ \vec{u}\cdot\vec{v} = \|u\|\|v\|\cos\theta \]
The triangle inequality generalizes to
\[ \|u+v\| \le \|u\|+\|v\| \]
In summary, inner products allow us to generalize many properties we used in
\( \R^2 \) to abstract vector spaces.

\begin{center}
  You can find all my notes at \url{http://omgimanerd.tech/notes}. If you have
  any questions, comments, or concerns, please contact me at
  alvin@omgimanerd.tech
\end{center}

\end{document}
