\documentclass{math}

\title{Advanced Linear Algebra}
\author{Alvin Lin}
\date{January 2019 - May 2019}

\begin{document}

\maketitle

\section*{Orthogonal Bases for Subspaces}
Recall:
\[ \text{proj}_{\vec{u}}\vec{v} =
  \left(\frac{\vec{u}\cdot\vec{v}}{\vec{u}\cdot\vec{u}}\right)\vec{u} \]
Let \( W \) be a subspace of \( \R^n \) and let
\( \{\vec{u_1},dots,\vec{u_k}\} \) be an orthogonal basis for \( W \). For any
vector \( \vec{v} \) in \( \R^n \), the orthogonal projection of \( \vec{v} \)
onto \( W \) is
\[ \text{proj}_{W}\vec{v} =
  \left(\frac{\vec{u_1}\cdot\vec{v}}{\vec{u_1}\cdot\vec{u_1}}\right)\vec{u_1}+
  \dots+
  \left(\frac{\vec{u_k}\cdot\vec{v}}{\vec{u_k}\cdot\vec{u_k}}\right)\vec{u_k} \]
The component of \( \vec{v} \) orthogonal to \( W \) is
\[ \text{perp}_{W}\vec{v} = \vec{v}-\text{proj}_{W}\vec{v} \]

\subsubsection*{Example}
Let \( W \) be the plane \( x-y+2z = 0 \) and \( \vec{v} = \langle3,-1,2\rangle
\). Find \( \text{proj}_{W}\vec{v} \) and \( \text{perp}_{W}\vec{v} \) if the
basis for W is
\[ \vec{u_1} = \begin{bmatrix}1 \\ 1 \\ 0\end{bmatrix} \quad
  \vec{u_2} = \begin{bmatrix}-1 \\ 1 \\ 1\end{bmatrix} \]
\begin{align*}
  \text{perp}_{W}\vec{v} &= \vec{v}-\text{proj}_{W}\vec{v} = \begin{bmatrix}
    \frac{4}{3} \\ -\frac{4}{3} \\ \frac{8}{3}\end{bmatrix} \\
  \text{proj}_{W}\vec{v} &=
    \left(\frac{\vec{u_1}\cdot\vec{v}}{\vec{u_1}\cdot\vec{u_1}}\right)\vec{u_1}+
    \left(\frac{\vec{u_2}\cdot\vec{v}}{\vec{u_2}\cdot\vec{u_2}}\right)\vec{u_2}
    \\
  &= \begin{bmatrix}
    \frac{5}{3} \\ \frac{1}{3} \\ -\frac{2}{3}
  \end{bmatrix}
\end{align*}

\subsection*{Orthogonal Decomposition Theorem}
Let \( W \) be a subspace of \( \R^n \) and \( \vec{v} \) be a vector in
\( \R^n \). Then there are unique vectors \( \vec{w} \) in \( W \) and
\( \vec{w}^{\bot} \) in \( W^{\bot} \) such that \( \vec{v} =
\vec{w}+\vec{w}^{\bot} \) and \( dim(W)+dim(W^T) = n \).

\subsection*{The Gram-Schmidt Process}
Given a basis \( \{\vec{x_1},\dots,\vec{x_k}\} \) for a subspace \( W \), can we
construct from it an orthogonal basis? \par
Let \( W = span(\vec{x_1},\vec{x_2}) \) where \( \vec{x_1} = \begin{bmatrix}
1 \\ 1 \\ 0\end{bmatrix} \) and \( \vec{x_2} = \begin{bmatrix}-2 \\ 0 \\
1\end{bmatrix} \). Construct an orthogonal basis for \( W \). \par
We want to turn \( \{\vec{x_1},\vec{x_2}\} \) into \( \{\vec{v_1},\vec{v_2}\} \)
where \( \{\vec{x_1},\vec{x_2}\} \) are not orthogonal and
\( \{\vec{v_1},\vec{v_2}\} \) are orthogonal.
\begin{align*}
  \vec{v_1} &= \vec{x_2} \\
  \vec{v_2} &= \text{perp}_{\vec{v_1}}\vec{x_2} \\
  &= \vec{x_2}-\text{proj}_{\vec{v_2}}\vec{x_2} \\
  &= \vec{x_2}-\left(\frac{\vec{v_1}\cdot\vec{x_2}}{\vec{v_1}\cdot\vec{v_1}}
    \right)\vec{v_1} \\
  &= \begin{bmatrix}-2 \\ 0 \\ 1\end{bmatrix}-\frac{(-2)}{2}\begin{bmatrix}
    1 \\ 1 \\ 0\end{bmatrix} \\
  &= \begin{bmatrix}-1 \\ 1 \\ 1\end{bmatrix}
\end{align*}
In general, let \( \{\vec{x_1},\dots,\vec{v_k} \) be a basis for a subspace
\( W \) of \( \R^n \).
\begin{align*}
  \vec{v_1} &= \vec{x_1} \quad W_1 = span(\vec{x_1}) \\
  \vec{v_2} &= \vec{x_2}-\text{proj}_{W_1}\vec{x_2} \quad
    W_2 = span(\vec{x_1},\vec{x_2}) \\
  &= \vec{x_2}-
    \left(\frac{\vec{v_1}\cdot\vec{x_2}}{\vec{v_1}\cdot\vec{v_1}}\right)
    \vec{v_1} \\
  \vec{v_3} &= \vec{x_3}-\text{proj}_{W_2}\vec{x_3} \quad
    W_3 = span(\vec{x_1},\vec{x_2},\vec{x_3}) \\
  &= \vec{x_2}-
    \left(\frac{\vec{v_1}\cdot\vec{x_3}}{\vec{v_1}\cdot\vec{v_1}}\right)
    \vec{v_1}-
    \left(\frac{\vec{v_2}\cdot\vec{x_3}}{\vec{v_2}\cdot\vec{v_2}}\right)
    \vec{v_2} \\
  &\vdots \\
  \vec{v_k} &= \vec{x_k}-\text{proj}_{W_{k-1}}\vec{x_k} \quad
    W_k = span(\vec{x_1},\dots,\vec{x_k}) \\
  &= \vec{x_k}-
    \left(\frac{\vec{v_1}\cdot\vec{x_k}}{\vec{v_1}\cdot\vec{v_1}}\right)
    \vec{v_1}-\dots-
    \left(\frac{\vec{v_{k-1}}\cdot\vec{x_k}}{\vec{v_{k-1}}\cdot\vec{v_{k-1}}}
    \right)\vec{v_{k-1}} \\
\end{align*}

\subsubsection*{Example}
Apply the Gram-Schmidt process to construct an orthonormal basis for the
subspace.
\begin{align*}
  W &= span(\vec{x_1},\vec{x_2},\vec{x_3}) \text{ of } \R^4 \text{ where } \\
  \vec{x_1} &= \begin{bmatrix}1 \\ -1 \\ -1 \\ 1\end{bmatrix} \quad
    \vec{x_2} = \begin{bmatrix}2 \\ 1 \\ 0 \\ 1\end{bmatrix} \quad
    \vec{x_3} = \begin{bmatrix}2 \\ 2 \\ 1 \\ 2\end{bmatrix} \\
  \text{Let: } \vec{v_1} &= \vec{x_1} \quad W_1 = span(\vec{v_1}) \\
  \vec{v_2} &= \text{perp}_{W_1}\vec{x_2} \\
  &= \vec{x_2}-\text{proj}_{W_1}\vec{x_2} \\
  &= \vec{x_2}-\left(\frac{\vec{v_1}\cdot\vec{x_2}}
    {\vec{v_1}\cdot\vec{v_1}}\right)\vec{v_1} \\
  &= \begin{bmatrix}
    \frac{3}{2} \\ \frac{3}{2} \\ \frac{1}{2} \\ \frac{1}{2}
  \end{bmatrix} \\
  &= \begin{bmatrix}3 \\ 3 \\ 1 \\ 1\end{bmatrix} \quad
    W_2 = span(\vec{v_1},\vec{v_2}) \\
  \vec{v_3} &= \text{perp}_{W_2}\vec{x_3} \\
  &= \vec{x_3}-\text{proj}_{W_3}\vec{x_3} \\
  &= \vec{x_3}-\left(\frac{\vec{v_1}\cdot\vec{x_3}}
    {\vec{v_1}\cdot\vec{v_1}}\right)\vec{v_1}-
    \left(\frac{\vec{v_2}\cdot\vec{x_3}}
    {\vec{v_2}\cdot\vec{v_2}}\right)\vec{v_2} \\
  &= \begin{bmatrix}-\frac{1}{2} \\ 0 \\ \frac{1}{2} \\ 1\end{bmatrix} \\
  &= \begin{bmatrix}-1 \\ 0 \\ 1 \\ 2\end{bmatrix} \\
  W &= span(\vec{v_1},\vec{v_2},\vec{v_3}) = span\left(
    \begin{bmatrix}1 \\ -1 \\ -1 \\ 1\end{bmatrix},
    \begin{bmatrix}3 \\ 3 \\ 1 \\ 1\end{bmatrix},
    \begin{bmatrix}-1 \\ 0 \\ 1 \\ 2\end{bmatrix}
  \right)
\end{align*}

\subsection*{QR Factorization: A = QR}
Let \( A \) be an \( m\times n \) matrix with linearly independent columns.
\[ A = \begin{bmatrix}\vec{a_1} & \vec{a_2} & \dots & \vec{a_n}\end{bmatrix} \]
Apply Gram-Schmidt to the column vectors to find an orthonormal basis
\( \hat{q_1},\dots,\hat{q_n} \).
\begin{align*}
  \vec{a_1} &= r_{11}\hat{q_1} \\
  \vec{a_2} &= r_{12}\hat{q_1}+r_{22}\hat{q_2} \\
  & \vdots \\
  \vec{a_n} &= r_{1n}\hat{q_1}+r_{2n}\hat{q_2}+\dots+r_{nn}\hat{q_n} \\
  \begin{bmatrix}\vec{a_1} & \vec{a_2} & \dots & \vec{a_n}\end{bmatrix} &=
    \begin{bmatrix}\hat{q_1} & \hat{q_2} & \dots & \hat{q_3}\end{bmatrix}
    \begin{bmatrix}
      r_{11} & r_{12} & \dots & r_{1n} \\
      0 & r_{22} & \dots & \dots \\
      \vdots & \vdots & \vdots & \vdots \\
      0 & 0 & \dots & r_{nn}
    \end{bmatrix}
\end{align*}

\subsubsection*{Example}
Find the QR factorization of
\[ A = \begin{bmatrix}
  1 & 2 & 2 \\
  -1 & 1 & 2 \\
  -1 & 0 & 1 \\
  1 & 1 & 2
\end{bmatrix} \]
Apply Gram-Schmidt to the column vectors.
\begin{align*}
  Q &= \begin{bmatrix}
    \frac{1}{2} & \frac{3}{2\sqrt{5}} & -\frac{1}{\sqrt{6}} \\
    -\frac{1}{2} & \frac{3}{2\sqrt{5}} & 0 \\
    -\frac{1}{2} & \frac{1}{2\sqrt{5}} & \frac{1}{\sqrt{6}} \\
    \frac{1}{2} & \frac{1}{2\sqrt{5}} & \frac{2}{\sqrt{6}}
  \end{bmatrix} \\
  A &= QR \\
  Q^TA &= Q^TQR \\
  R &= Q^TA \\
  &= \begin{bmatrix}
    2 & 1 & \frac{1}{2} \\
    0 & \sqrt{5} & \frac{3\sqrt{5}}{2} \\
    0 & 0 & \frac{\sqrt{6}}{2}
  \end{bmatrix}
\end{align*}

\begin{center}
  You can find all my notes at \url{http://omgimanerd.tech/notes}. If you have
  any questions, comments, or concerns, please contact me at
  alvin@omgimanerd.tech
\end{center}

\end{document}
