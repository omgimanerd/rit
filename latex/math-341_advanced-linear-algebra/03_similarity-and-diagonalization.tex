\documentclass{math}

\usepackage{tikz}

\title{Advanced Linear Algebra}
\author{Alvin Lin}
\date{January 2019 - May 2019}

\begin{document}

\maketitle

\section*{Similarity and Diagonalization}
Let \( A,B \) be \( n\times n \) matrices. \( A \) is similar to \( B \)
(\( A\sim B\)) if there is an invertible matrix \( P \) such that
\( B = P^{-1}AP \). If \( A\sim B \), then:
\begin{enumerate}
  \item \( \det(A) = \det(B) \)
  \item \( A \) and \( B \) have the same characteristic polynomial.
\end{enumerate}
We can prove the first statement by taking the determinant the similarity
definition.
\begin{align*}
  B &= P^{-1}AP \\
  \det(B) &= \det(P^{-1}AP) \\
  &= \det(P^{-1})\det(A)\det(P) \\
  &= \frac{1}{\det(P)}\det(A)\det(P) \\
  &= \det(A)
\end{align*}
We can prove that similar matrices have the same characteristic polynomial in
the same way.
\begin{align*}
  \det(B-\lambda I) &= \det(P^{-1}AP-\lambda I) \\
  &= \det(P^{-1}AP-\lambda P^{-1}IP) \\
  &= \det\left[P^{-1}(A-\lambda I)P\right] \\
  &= \det(A-\lambda I)
\end{align*}

\subsection*{Diagonalization}
An \( n\times n \) matrix \( A \) is diagonalizable if there is a diagonal
matrix \( D \) such that \( A\sim D \). An \( n\times n \) matrix is
diagonalizable if and only if \( A \) has \( n \) linearly independent
eigenvectors.
\begin{align*}
  P^{-1}AP &= D \\
  AP &= PD \\
  A\begin{bmatrix}\vec{p_1} & \dots & \vec{p_n}\end{bmatrix} &=
    \begin{bmatrix}\vec{p_1} & \dots & \vec{p_n}\end{bmatrix}
    \begin{bmatrix}
      \lambda_1 & \dots & 0 \\
      \vdots & \vdots & \vdots \\
      0 & \dots & \lambda_n
    \end{bmatrix} \\
  \begin{bmatrix}
    A\vec{p_1} & \dots & A\vec{p_n}
  \end{bmatrix} &= \begin{bmatrix}
    \lambda_1\vec{p_1} & \dots & \lambda_n\vec{p_n}
  \end{bmatrix}
\end{align*}

\subsubsection*{Example}
If possible, find a matrix \( P \) that diagonalizes
\[ A = \begin{bmatrix}
  -1 & 0 & 1 \\
  3 & 0 & -3 \\
  1 & 0 & -1
\end{bmatrix} \]
First we need to find the eigenvalues and eigenvectors of \( A \). For brevity,
we will skip the steps necessary to do this.
\begin{align*}
  \lambda_1 &= \lambda_2 = 0 \quad
  \vec{p} = s\begin{bmatrix}0 \\ 1 \\ 0\end{bmatrix}+
    t\begin{bmatrix}1 \\ 0 \\ 1\end{bmatrix} \\
  \vec{p_1} &= \begin{bmatrix}0 \\ 1 \\ 1\end{bmatrix} \quad
  \vec{p_2} = \begin{bmatrix}1 \\ 0 \\ 1\end{bmatrix} \\
  \lambda_3 &= -2 \quad
  \vec{p} = s\begin{bmatrix}-1 \\ 3 \\ 1\end{bmatrix} \\
  \vec{p_3} &= \begin{bmatrix}-1 \\ 3 \\ 1\end{bmatrix} \\
  P &= \begin{bmatrix}
    0 & 1 & -1 \\
    1 & 0 & 3 \\
    0 & 1 & 1
  \end{bmatrix} \\
  D &= P^{-1}AP \\
  \begin{bmatrix}
    0 & 0 & 0 \\
    0 & 0 & 0 \\
    0 & 0 & -2
  \end{bmatrix} &= \begin{bmatrix}
    0 & 1 & -1 \\
    1 & 0 & 3 \\
    0 & 1 & 1
  \end{bmatrix}^{-1}\begin{bmatrix}
    -1 & 0 & 1 \\
    3 & 0 & -3 \\
    1 & 0 & -1
  \end{bmatrix}\begin{bmatrix}
    0 & 1 & -1 \\
    1 & 0 & 3 \\
    0 & 1 & 1
  \end{bmatrix}
\end{align*}

\subsubsection*{Example}
Let \( A = \begin{bmatrix}0 & 1 \\ 2 & 1\end{bmatrix} \). Compute \( A^{10} \).
\begin{align*}
  D &= P^{-1}AP \\
  D^2 &= (P^{-1}AP)(P^{-1}AP) \\
  &= P^{-1}A^2P \\
  D^{10} &= P^{-1}A^{10}P \\
  A^{10} &= PD^{10}P^{-1} \\
\end{align*}
To use this argument, we need to compute the eigenvalues and eigenvectors of
\( A \).
\begin{align*}
  \det(A-\lambda I) &= 0 \\
  \begin{vmatrix}-\lambda & 1 \\ 2 & 1-\lambda\end{vmatrix} &= 0 \\
  -\lambda(1-\lambda)-2 &= 0 \\
  \lambda^2-\lambda-2 &= 0 \\
  (\lambda+1)(\lambda-2) &= 0 \\
  \lambda_1 &= -1 \quad \lambda_2 = 2 \\
  \vec{p_1} &= \begin{bmatrix}1 \\ -1\end{bmatrix} \\
  \vec{p_2} &= \begin{bmatrix}1 \\ 2\end{bmatrix} \\
  P &= \begin{bmatrix}
    1 & 1 \\
    -1 & 2
  \end{bmatrix} \\
  D &= \begin{bmatrix}
    -1 & 0 \\
    0 & 2
  \end{bmatrix} \\
  A^{10} &= \begin{bmatrix}1 & 1 \\ -1 & 2\end{bmatrix}
    \begin{bmatrix}-1 & 0 \\ 0 & 2\end{bmatrix}^{10}
    \begin{bmatrix}1 & 1 \\ -1 & 2\end{bmatrix}^{-1} \\
  &= \begin{bmatrix}1 & 1 \\ -1 & 2\end{bmatrix}
    \begin{bmatrix}(-1)^{10} & 0 & 0 & 2^{10}\end{bmatrix}
    \begin{bmatrix}
      \frac{2}{3} & -\frac{1}{3} \\ \frac{1}{3} & \frac{1}{3}
    \end{bmatrix} \\
  &= \begin{bmatrix}
    342 & 341 \\
    682 & 683
  \end{bmatrix}
\end{align*}

\begin{center}
  You can find all my notes at \url{http://omgimanerd.tech/notes}. If you have
  any questions, comments, or concerns, please contact me at
  alvin@omgimanerd.tech
\end{center}

\end{document}
