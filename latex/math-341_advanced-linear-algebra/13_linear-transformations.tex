\documentclass{math}

\usepackage{tikz}

\title{Advanced Linear Algebra}
\author{Alvin Lin}
\date{January 2019 - May 2019}

\begin{document}

\maketitle

\section*{Linear Transformations}
A linear transformation \( T:V\to W \) is a mapping such that
\begin{enumerate}
  \item \( T(u+v) = T(u)+T(v) \)
  \item \( T(vu) = cT(u) \)
\end{enumerate}
In a more compressed form
\[ T(c_1u_1+\dots+c_ku_k) = c_1T(u_1)+\dots+c_kT(u_k) \]
For example, consider the transformation \( T:M_{mn}\to M_{mn} \)
where \( T(A) = A^T \).
\begin{enumerate}
  \item \( T(A+B) = (A+B)^T = A^T+B^T = T(A)+T(B) \)
  \item \( T(cA) = (cA)^T = cA^T = cT(A) \)
\end{enumerate}
This also applies for transformations in the space of functions. Suppose we have
\( D:D\to F, D(f) = f' \).
\begin{enumerate}
  \item \( D(f+g) = (f+g)' = f'+g' = D(f)+D(g) \)
  \item \( D(cf) = (cf)' = cf' = cD(f) \)
\end{enumerate}
As another example, consider the transformation \( S:C[a,b]\to\R \) where
\( S(f) = \int_{a}^{b}f\diff{x} \).
\begin{enumerate}
  \item \( S(f+g) = \int_{a}^{b}(f+g)\diff{x} =
    \int_{a}^{b}f\diff{x}+\int_{a}^{b}g\diff{x} \)
  \item \( S(cf) = \int_{a}^{b}cf\diff{x} = cS(f) \)
\end{enumerate}
Now consider the transformation \( T:M_{22}\to\R, T(A) = \det(A) \).
\begin{align*}
  T(A+B) &= \det(A+B) \\
  &\ne \det(A)+\det(B) \\
  &\ne T(A)+T(B)
\end{align*}
Thus, this transformation is not a linear transformation.

\subsection*{Zero and Identity Transformation}
The zero and identity transformations are special transformations:
\begin{itemize}
  \item Zero transformation: \( T_0(A) = 0 \)
  \item Identity transformation: \( T_I(A) = A \)
\end{itemize}

\subsubsection*{Example}
Given the linear transformation \( T:\R^2\to P_2 \) and
\begin{align*}
  T\begin{bmatrix}1 \\ 1\end{bmatrix} &= 2-3x+x^2 \\
  T\begin{bmatrix}2 \\ 3\end{bmatrix} &= 1-x^2
\end{align*}
Find \( T\begin{bmatrix}-1 \\ 2\end{bmatrix} \). \par
Note that \( B = \left\{\begin{bmatrix}1 \\ 1\end{bmatrix}, \begin{bmatrix}2 \\
3 \end{bmatrix}\right\} \) is a basis for \( \R^2 \).
\begin{align*}
  \begin{bmatrix}-1 \\ 2\end{bmatrix} &=
    -7\begin{bmatrix}1 \\ 1\end{bmatrix}+
    3\begin{bmatrix}2 \\ 3\end{bmatrix} \\
  T\begin{bmatrix}-1 \\ 2\end{bmatrix} &=
    -7T\begin{bmatrix}1 \\ 1\end{bmatrix}+
    3T\begin{bmatrix}2 \\ 3\end{bmatrix} \\
  &= -7(2-3x+x^2)+3(1-x^2) \\
  &= -11+21x-10x^2
\end{align*}

\subsection*{Composition}
Suppose we have the transformations \( T:U\to V \) and \( S:V\to W \). The
composition of the transformations \( (S\circ T)u = S(T(u)) \). For example:
\begin{align*}
  T:\R^2\to P_1, T\begin{bmatrix}a \\ b\end{bmatrix} &= a+(a+b)x \\
  S:P_1\to P_2, S(p) &= xp \\
  (S\circ P)\begin{bmatrix}3 \\ -2\end{bmatrix} &=
    S\left(T\begin{bmatrix}3 \\ -2\end{bmatrix}\right) \\
  &= S(3+x) \\
  &= x(3+x) \\
  &= 3x+x^2
\end{align*}

\subsection*{Inverse Transformations}
The definition of an inverse transformation \( T' \) is that
\( T'\circ T = Iv \). For example:
\begin{align*}
  T:\R^2\to P_1, T\begin{bmatrix}a \\ b\end{bmatrix} &= a+(a+b)x \\
  T':P_1\to\R_2, T'(c+dx) &= \begin{bmatrix}c \\ d-c\end{bmatrix} \\
  (T'\circ T)\begin{bmatrix}a \\ b\end{bmatrix} &\stackrel{?}{=}
    \begin{bmatrix}a \\ b\end{bmatrix} \\
  T'\left(T\begin{bmatrix}a \\ b\end{bmatrix}\right) &=
    T'(a+(a+b)x) \\
  &= \begin{bmatrix}a \\ a+b-a\end{bmatrix} \\
  &= \begin{bmatrix}a \\ b\end{bmatrix}
\end{align*}

\subsection*{Kernel and Range of \( T:V\to W \)}
The kernel and range of a transformation \( T \) are defined as follows:
\begin{align*}
  ker(T) &= \{v\in V: T(v) = 0\} \\
  range(T) &= \{T(v):v\in V\} \\
  &= \{w\in W: w = T(v)\text{ for some }v\in V\}
\end{align*}
For example, given the transformation \( T:\R^n\to\R^m, T(v) = Av \):
\begin{align*}
  range(T) &= \{Av: v\in\R^n\} = col(A) \\
  ker(T) &= \{v\in\R^n: Av = 0\} = null(A)
\end{align*}
The kernel and the range are generalizations of the null space and the column
space.
\begin{align*}
  rank(T) &= dim(range(T)) \\
  nullity(T) &= dim(ker(T)) \\
  rank(T)+nullity(T) &= dim(V)
\end{align*}

\subsubsection*{Example}
Suppose have the transformation \( D:P_3\to P_2, Dp = p' \). Find the kernel
of the linear operator \( D \).
\begin{align*}
  ker(D) &= \{p\in P_3: Dp = 0 \} \\
  &= \{a:a\in\R\}
\end{align*}
Since the derivative of any constant is 0, the kernel of the linear operator is
the set of real numbers. \\
Find the range of \( D \).
\[ range(D) = \{Dp:p\in P_3\} = P_2 \]
Find the rank and nullity of \( D \).
\begin{align*}
  rank(D) &= dim(P_2) = 3 \\
  nullity(D) &= dim(ker(D)) = 1
\end{align*}

\subsubsection*{Example}
Given the transformation \( S:P_1\to\R, S(p(x)) = \int_{0}^{1}p(x)\diff{x} \),
find \( ker(S) \).
\begin{align*}
  Let: p &= a+bx \text{ in } P_1 \\
  Sp &= \int_{0}^{1}(a+bx)\diff{x} \\
  &= \left[ax+\frac{b}{2}x^2\right]_{0}^{1} \\
  &= a+\frac{b}{2} \\
  ker(s) &= \{p\in P_1: Sp = 0 \} \\
  &= \left\{a+bx\in P_1: a+\frac{b}{2} = 0\right\} \\
  &= \left\{-\frac{b}{2}+bx\right\}
\end{align*}
Find the rank and nullity of \( S \).
\begin{align*}
  basis(ker(S)) &= span\left(-\frac{1}{2}+x\right) \\
  nullity(S) &= dim(ker(S)) = 1 \\
  rank(S) &= dim(range(S)) = dim(\R) = 1
\end{align*}

\subsection*{One to One Mappings}
\( T:V\to W \) is called ``one-to-one'' if it maps distinct vectors in \( V \)
to distinct vectors in \( W \). If \( range(T) = W \), then \( T \) is
``onto''. \\
\textbf{Theorem:} A linear transformation \( T:V\to W \) is 1-1 if and only if
\( ker(T) = \{0\} \). To prove this, assume \( T \) is 1-1 and that
\( v\in ker(T) \).
\begin{align*}
  T(v) &= 0 \quad T(0) = 0 \\
  \therefore T(v) &= T(0)
\end{align*}
Since \( T \) is 1-1, then \( v = 0 \). Conversely, assume \( ker(T) = \{0\} \).
Let \( u,v \) in \( V \) with \( T(u) = T(v) \).
\[ T(u-v) = T(u)-T(v) = 0 \]
If this statement is true, then \( u-v \) is in \( ker(T) \), and since
\( ker(T) = \{0\} \):
\[ u-v = 0 \quad u = v \]
Therefore, \( T \) is 1-1. \\
\textbf{Theorem:} \( T:V\to W \). Let \( dim(V) = dim(W) = n \), then \( T \) is
1-1 if and only if it is onto. \\
\textbf{Theorem:} Let \( T:V\to W \) be 1-1. If \( S = \{v_1,\dots,v_n\} \) is
linearly independent in \( V \) then \( T(s) = \{T(v_1),\dots,T(v_n)\} \) is
linearly independent in \( W \). \\
\textbf{Theorem:} \( T:V\to W \) is invertible if and only if it is 1-1 and
onto.

\subsubsection*{Example}
Given the transformation \( T:\R^2\to\R^3, T\begin{bmatrix}x \\ y\end{bmatrix}
= \begin{bmatrix}2x \\ x-y \\ 0\end{bmatrix} \), is \( T \) 1-1? Does
\( T(u) = T(v) \) imply \( u = v \)?
\begin{align*}
  T\begin{bmatrix}x_1 \\ y_1\end{bmatrix} =
    T\begin{bmatrix}x_2 \\ y_2\end{bmatrix} &\stackrel{?}{\Rightarrow}
    \begin{bmatrix}x_1 \\ y_1\end{bmatrix} =
    \begin{bmatrix}x_2 \\ y_2\end{bmatrix} \\
  \begin{bmatrix}2x_1 \\ x_1-y_1 \\ 0\end{bmatrix} &=
    \begin{bmatrix}2x_2 \\ x_2-y_2 \\ 0\end{bmatrix} \\
  2x_1 = 2x_2 &\to x_1 = x_2 \\
  x_1-y_1 = x_2-y_2 &\to y_1 = y_2 \\
  \begin{bmatrix}x_1 \\ y_1\end{bmatrix} &=
    \begin{bmatrix}x_2 \\ y_2\end{bmatrix}
\end{align*}
\( T \) is one to one, but not onto because
\[ T\begin{bmatrix}x \\ y\end{bmatrix} \ne \begin{bmatrix}0 \\ 0 \\ 1
\end{bmatrix} \text{ for any } \begin{bmatrix}x \\ y\end{bmatrix} \]

\subsubsection*{Example}
\( T:\R^2\to P_1, T\begin{bmatrix}a \\ b\end{bmatrix} = a+(a+b)x \). Show that
\( T \) is 1-1 and onto. \\
Let \( \begin{bmatrix}a \\ b\end{bmatrix} \) be in \( ker(T) \).
\begin{align*}
  T\begin{bmatrix}a \\ b\end{bmatrix} &= a+(a+b)x = 0 \\
  a &= 0 \quad a+b = 0 \\
  b &= -a \quad b = 0 \\
  \begin{bmatrix}a \\ b\end{bmatrix} &= \begin{bmatrix}0 \\ 0\end{bmatrix} =
    ker(T)
\end{align*}
Since \( ker(T) = \left\{\begin{bmatrix}0 \\ 0\end{bmatrix}\right\} \), it is
1-1. We can prove \( T \) is onto using the rank theorem.
\begin{align*}
  rank(T)+nullity(T) &= dim(V) \\
  rank(T)+0 &= 2 \\
  dim(range(T)) &= 2 \\
  rank(T) &= dim(range(T))
\end{align*}
Therefore, \( T \) is onto.

\subsection*{Isomorphism}
\( T:V\to W \) is called an ``isomorphism'' if it is 1-1 and onto. If \( V \)
and \( W \) are vector spaces such that there is an isomorphism from \( V \) to
\( W \), then we say \( V \) and \( W \) are ``isomorphic'' (\( V\cong W \)).

\subsubsection*{Example}
Suppose \( P_{n-1} \) and \( \R^n \) are isomorphic.
\[ T(a_0+a_1x+\dots+a_{n-1}x^{n-1}) = \begin{bmatrix}a_0 \\ a_1 \\ \vdots \\
  a_{n-1}\end{bmatrix} \]
Note that we know \( T \) is 1-1 since \( ker(T) = \{0\} \). Therefore,
\( P_{n-1} \cong \R^n \).

\subsection*{Matrix of Linear Transformation}
\begin{center}
  \begin{tikzpicture}
    \draw (0,8) rectangle ++(2,2)
      node[above left] {\( B = \{v_1,\dots,v_n\} \)};
    \node[above] at (1,9) {\( V \)};

    \draw[->] (2.5,9) -- (7.5,9) node[pos=0.5,above] {\( T \)};

    \draw (8,8) rectangle ++(2,2)
      node[above] {\( C = \{T(v_1),\dots,T(v_n)\} \)};
    \node[above] at (9,9) {\( T(v) \)};

    \draw[->] (0.5,3) -- (0.5,7) node[pos=0.5,left] {\( R^{-1} \)};
    \draw[<-] (1.5,3) -- (1.5,7) node[pos=0.5,right] {\( R \)};

    \draw (0,0) rectangle ++(2,2)
      node[above] {Standard Basis: \( \{\hat{e_1},\dots,\hat{e_n}\} \)};
    \node[above] at (1,1) {\( [v]_B \)};

    \draw[->] (2.5,1) -- (7.5, 1) node[pos=0.5,above] {\( T_A \)};

    \draw (8,0) rectangle ++(2,2)
      node[above] {Standard Basis: \( \{\hat{e_1},\dots,\hat{e_m}\} \)};
    \node[above] at (9,1) {\( [T(v)]_C \)};

    \draw[<-] (9,3) -- (9,7) node[pos=0.5,right] {\( S \)};
  \end{tikzpicture}
\end{center}
\begin{align*}
  [T(v)]_C &= A[v]_B \\
  &= (S\circ T\circ R^{-1})[v]_B \\
  A &= \bigg[[T(v_1)]_C, T(v_2)_C, \dots, [T(v_n)]_C\bigg] \\
  &= [T]_{C\leftarrow B}
\end{align*}
Every linear transformation can be represented as a matrix multiplication
under some given bases.

\subsubsection*{Example}
Suppose we have the transformation \( T:\R^3\to\R^2, T\begin{bmatrix}x \\ y \\
z\end{bmatrix} = \begin{bmatrix}x-2y \\ x+y-3z\end{bmatrix} \). Let
\( B = \{\hat{e_1},\hat{e_2},\hat{e_3}\} \) and \( C = \{\hat{e_2},\hat{e_1}\}
\).
\begin{align*}
  T(\hat{e_1}) &= T\begin{bmatrix}1 \\ 0 \\ 0\end{bmatrix} =
    \begin{bmatrix}1 \\ 1\end{bmatrix} \quad
    [T(\hat{e_1})]_C = \begin{bmatrix}1 \\ 1\end{bmatrix}_C \\
  T(\hat{e_2}) &= T\begin{bmatrix}0 \\ 1 \\ 0\end{bmatrix} =
    \begin{bmatrix}-2 \\ 1\end{bmatrix} \quad
    [T(\hat{e_2})]_C = \begin{bmatrix}1 \\ -2\end{bmatrix}_C \\
  T(\hat{e_3}) &= T\begin{bmatrix}0 \\ 0 \\ 1\end{bmatrix} =
    \begin{bmatrix}0 \\ -3\end{bmatrix} \quad
    [T(\hat{e_3})]_C = \begin{bmatrix}-3 \\ 0\end{bmatrix}_C \\
  A &= \begin{bmatrix}
    1 & 1 & -3 \\
    1 & -2 & 0
  \end{bmatrix}
\end{align*}
We can verify that this works for \( \vec{v} = \begin{bmatrix}1 \\ 3 \\ -2
\end{bmatrix} \).
\begin{align*}
  T\begin{bmatrix}1 \\ 3 \\ -2\end{bmatrix} &=
    \begin{bmatrix}-5 \\ 10\end{bmatrix} \\
  [v]_B &= \begin{bmatrix}1 \\ 3 \\ -2\end{bmatrix} \\
  A[v]_B &= [T(v)]_C \\
  \begin{bmatrix}
    1 & 1 & -3 \\
    1 & -2 & 0
  \end{bmatrix}\begin{bmatrix}1 \\ 3 \\ -2\end{bmatrix} &\stackrel{?}=
    \begin{bmatrix}10 \\ -5\end{bmatrix} \\
  \begin{bmatrix}10 \\ -5\end{bmatrix} &= \begin{bmatrix}10 \\ -5\end{bmatrix}
\end{align*}

\subsubsection*{Example}
Suppose we have the transformation \( D:P_3\to P_2, D(p(x)) = p'(x) \). Let
\( B = \{1,x,x^2,x^3\} \) and \( C = \{1,x,x^2\} \).
\begin{align*}
  D(1) &= 0 \quad [D(1)]_C = \begin{bmatrix}0 \\ 0 \\ 0\end{bmatrix} \\
  D(x) &= 1 \quad [D(x)]_C = \begin{bmatrix}1 \\ 0 \\ 0\end{bmatrix} \\
  D(x^2) &= 2x \quad [D(x)]_C = \begin{bmatrix}0 \\ 2 \\ 0\end{bmatrix} \\
  D(x^3) &= 3x^2 \quad [D(x^3)]_C = \begin{bmatrix}0 \\ 0 \\ 3\end{bmatrix} \\
  A &= \begin{bmatrix}
    0 & 1 & 0 & 0 \\
    0 & 0 & 2 & 0 \\
    0 & 0 & 0 & 3
  \end{bmatrix}
\end{align*}
Compute \( D(5-x+2x^3) \).
\begin{align*}
  D(5-x+2x^3) &= -1+6x^2 \\
  &= \begin{bmatrix}-1 \\ 0 \\ 6\end{bmatrix}_C
\end{align*}
We can check that we get the same answer using the matrix \( A[v]_B = [D(v)]_C
\).
\[ \begin{bmatrix}
  0 & 1 & 0 & 0 \\
  0 & 0 & 2 & 0 \\
  0 & 0 & 0 & 3
\end{bmatrix}\begin{bmatrix}5 \\ -1 \\ 0 \\ 2\end{bmatrix} = \begin{bmatrix}
  -1 \\ 0 \\ 6\end{bmatrix} \]

\begin{center}
  You can find all my notes at \url{http://omgimanerd.tech/notes}. If you have
  any questions, comments, or concerns, please contact me at
  alvin@omgimanerd.tech
\end{center}

\end{document}
