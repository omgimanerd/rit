\documentclass{math}

\title{Advanced Linear Algebra}
\author{Alvin Lin}
\date{January 2019 - May 2019}

\begin{document}

\maketitle

\section*{Orthogonality}
Recall that the orthogonal projection of \( \vec{v} \) onto \( \hat{u} \) is
defined as:
\begin{align*}
  \vec{v}\cdot\hat{u} &= \|\vec{v}\|\|\hat{u}\|\cos\theta \\
  proj_{\hat{u}}\vec{v} &= (\|\vec{v}\|\cos\theta)\hat{u} \\
  &= \|\vec{v}\|\frac{\vec{v}\cdot\hat{u}}{\|\vec{v}\|}\hat{u} \\
  &= (\langle v_1,v_2\rangle\cdot\langle u_1,u_2\rangle)\begin{bmatrix}
    u_1 \\ u_2
  \end{bmatrix} \\
  &= (v_1u_1+v_2u_2)\begin{bmatrix}
    u_1 \\ u_2
  \end{bmatrix} \\
  &= \begin{bmatrix}
    v_1u_1^2+v_2u_1u_2 \\
    v_1u_1u_2+v_2u_2^2
  \end{bmatrix} \\
  &= \begin{bmatrix}
    u_1^2 & u_1u_2 \\
    u_1u_2 & u_2^2
  \end{bmatrix}\begin{bmatrix}
    v_1 \\ v_2
  \end{bmatrix} \\
  &= P\vec{v}
\end{align*}
If we look at the projection matrix, we observe the following:
\begin{align*}
  P &= \begin{bmatrix}
    u_1^2 & u_1u_2 \\
    u_1u_2 & u_2^2
  \end{bmatrix} \\
  &= \begin{bmatrix}u_1 \\ u_2\end{bmatrix}
    \begin{bmatrix}u_1 & u_2\end{bmatrix} \\
  &= \hat{u}\hat{u}^T
\end{align*}
Projection is a linear transformation that can be expressed as a matrix
operation.

\subsubsection*{Example}
Given \( \hat{u} = \begin{bmatrix}\frac{1}{\sqrt{2}} \\
-\frac{1}{\sqrt{2}}\end{bmatrix} \) and \( \vec{v} = \begin{bmatrix}3 \\
4\end{bmatrix} \), find \( proj_{\hat{u}}\vec{v} \).
\begin{align*}
  P &= \hat{u}\hat{u}^T \\
  &= \begin{bmatrix}
    \frac{1}{\sqrt{2}} \\ -\frac{1}{\sqrt{2}}
  \end{bmatrix}\begin{bmatrix}
    \frac{1}{\sqrt{2}} & -\frac{1}{\sqrt{2}}
  \end{bmatrix} \\
  &= \begin{bmatrix}
    \frac{1}{2} & -\frac{1}{2} \\
    -\frac{1}{2} & \frac{1}{2}
  \end{bmatrix} \\
  proj_{\hat{u}}\vec{v} &= \begin{bmatrix}
    \frac{1}{2} & -\frac{1}{2} \\
    -\frac{1}{2} & \frac{1}{2}
  \end{bmatrix}\begin{bmatrix}
    3 \\ -4
  \end{bmatrix} \\
  &= \begin{bmatrix}
    \frac{7}{2} \\ -\frac{7}{2}
  \end{bmatrix}
\end{align*}

\subsubsection*{Example}
Show that \( P = \hat{u}\hat{u}^T \) is symmetric.
\begin{align*}
  P^T &= (\hat{u}\hat{u}^T)^T \\
  &= (\hat{u}^T)^T(\hat{u})^T \\
  &= \hat{u}\hat{u}^T
\end{align*}
Show that \( P^2 = P \) (P is idempotent).
\begin{align*}
  P^2 &= (\hat{u}\hat{u}^T)(\hat{u}\hat{u}^T) \\
  &= \hat{u}(\hat{u}^T\hat{u})\hat{u}^T \\
  \hat{u}^T\hat{u} &= 1 \\
  &= \hat{u}\hat{u}^T
\end{align*}

\subsection*{Orthogonal Projection of \( \vec{v} \) onto a Plane through Origin}
Given a plane, its normal vector \( \vec{n} \), a vector \( \vec{v} \), the
distance from the tip of \( \vec{v} \) to the plane is some length that can be
represented by a vector \( -c\vec{n} \). We can represent the projection of
\( \vec{v} \) onto the plane as \( \vec{p} = \vec{v}-c\vec{n} \).
\begin{align*}
  \vec{p} &= \vec{v}-c\vec{n} \\
  \vec{n}\cdot\vec{p} &= \vec{n}\cdot(\vec{v}-c\vec{n}) \\
  0 &= \vec{n}\cdot\vec{v}-c(\vec{n}\cdot\vec{n}) \\
  c &= \frac{\vec{n}\cdot\vec{v}}{\vec{n}\cdot\vec{n}}
\end{align*}
Alternatively, if instead of being given a unit vector \( \vec{n} \), suppose we
are given two orthogonal vectors \( \hat{u_1} \) and \( \hat{u_2} \) that span
the plane. Let \( \vec{p_1} = proj_{\hat{u_1}}\vec{v} \) and \( \vec{p_2} =
proj_{\hat{u_2}}\vec{v} \). The projection of \( \vec{v} \) onto the plane
spanned by \( \hat{u_1} \) and \( \hat{u_2} \) is equal to \( \vec{p} =
\vec{p_1}+\vec{p_2} \).
\begin{align*}
  \vec{p_1} &= proj_{\hat{u_1}}\vec{v} = \hat{u_1}\hat{u_1}^T\vec{v} \\
  \vec{p_2} &= proj_{\hat{u_2}}\vec{v} = \hat{u_2}\hat{u_2}^T\vec{v} \\
  \vec{p} &= \vec{p_1}+\vec{p_2} \\
  &= (\hat{u_1}\hat{u_1}^T+\hat{u_2}\hat{u_2}^T)\vec{v} = P\vec{v}
\end{align*}

\subsubsection*{Example}
Project \( \vec{v} = \langle1,0,-2\rangle \) onto the plane \( x+y+z = 0 \).
\begin{align*}
  \langle1,1,1\rangle\cdot\langle x,y,z\rangle &= 0 \\
  \vec{n} &= \langle1,1,1\rangle \\
  c &= \frac{\vec{n}\cdot\vec{v}}{\vec{n}\cdot\vec{n}} \\
  &= \frac{1+0-2}{1+1+1} \\
  &= -\frac{1}{2} \\
  \vec{p} &= \vec{v}+\frac{1}{3}\vec{n} \\
  &= \langle\frac{2}{3},-\frac{1}{3},-\frac{7}{3}\rangle
\end{align*}

\subsubsection*{Example}
Project \( \vec{v} = \langle1,0,-2\rangle \) onto \( x+y+z = 0 \) with
\( \hat{u_1} = \frac{1}{\sqrt{6}}\begin{bmatrix}-2 \\ 1 \\ 1\end{bmatrix} \)
and
\( \hat{u_2} = \frac{1}{\sqrt{2}}\begin{bmatrix}0 \\ 1 \\ -1\end{bmatrix} \).
\begin{align*}
  P &= \begin{bmatrix}
    -\frac{2}{\sqrt{6}} \\ \frac{1}{\sqrt{6}} \\ \frac{1}{\sqrt{6}}
  \end{bmatrix}\begin{bmatrix}
    -\frac{2}{\sqrt{6}} & \frac{1}{\sqrt{6}} & \frac{1}{\sqrt{6}}
  \end{bmatrix}+\begin{bmatrix}
    0 \\ \frac{1}{\sqrt{2}} \\ -\frac{1}{\sqrt{2}}
  \end{bmatrix}\begin{bmatrix}
    0 & \frac{1}{\sqrt{2}} & \frac{1}{\sqrt{2}}
  \end{bmatrix} \\
  &= \begin{bmatrix}
    \frac{4}{6} & -\frac{2}{6} & -\frac{2}{6} \\
    -\frac{2}{6} & \frac{1}{6} & \frac{1}{6} \\
    -\frac{2}{6} & \frac{1}{6} & \frac{1}{6}
  \end{bmatrix}+\begin{bmatrix}
    0 & 0 & 0 \\
    0 & \frac{1}{2} & -\frac{1}{2} \\
    0 & -\frac{1}{2} & \frac{1}{2}
  \end{bmatrix} \\
  &= \frac{1}{6}\begin{bmatrix}
    4 & -2 & -2 \\
    -2 & 4 & -2 \\
    -2 & -2 & 4
  \end{bmatrix} \\
  P\vec{v} &= \frac{1}{6}\begin{bmatrix}
    4 & -2 & -2 \\
    -2 & 4 & -2 \\
    -2 & -2 & 4
  \end{bmatrix}\begin{bmatrix}1 \\ 0 \\ -2\end{bmatrix} \\
  &= \begin{bmatrix}
    \frac{8}{6} \\ \frac{2}{6} \\ -\frac{10}{6}
  \end{bmatrix} \\
  &= \begin{bmatrix}
    \frac{4}{3} \\ \frac{1}{3} \\ -\frac{5}{3}
  \end{bmatrix}
\end{align*}

\subsubsection*{Concept}
What is the rank of \( P = \hat{u_1}\hat{u_1}^T+\hat{u_2}\hat{u_2}^T \)? \par
Every vector in \( \R^3 \) gets mapped by \( P\vec{v} \) onto a two dimensional
plane, so that means that the column space of P must be two dimensional.

\subsection*{Orthogonality in \( \R^n \)}
A set of vectors is \textbf{orthogonal} if \( \vec{v_i}\cdot\vec{v_j} = 0,
i\ne j \). For example:
\[ \vec{v_1} = \begin{bmatrix}2 \\ 1 \\ -1\end{bmatrix} \quad
  \vec{v_2} = \begin{bmatrix}0 \\ 1 \\ 1\end{bmatrix} \quad
  \vec{v_3} = \begin{bmatrix}1 \\ -1 \\ 1\end{bmatrix} \]
An \textbf{orthogonal basis} is a basis that is an orthogonal set.

\subsubsection*{Example}
Find an orthogonal basis for the subspace \( W \) of \( \R^3 \) given by
\[ W = \left\{\begin{bmatrix}x \\ y \\ z\end{bmatrix}: x-y+2z = 0\right\} \]
\begin{align*}
  \begin{bmatrix}
    y-2z \\ y \\ z
  \end{bmatrix} &= y\begin{bmatrix}
    1 \\ 1 \\ 0
  \end{bmatrix}+z\begin{bmatrix}
    -2 \\ 0 \\ 1
  \end{bmatrix} \\
  \vec{u} &= \begin{bmatrix}1 \\ 1 \\ 0\end{bmatrix} \\
  \vec{v} &= \begin{bmatrix}-2 \\ 0 \\ 1\end{bmatrix}
\end{align*}
These two vectors form a basis for \( W \) but are not orthogonal. Let
\( \vec{w} = \begin{bmatrix}x \\ y \\z\end{bmatrix} \) be orthogonal to
\( \vec{u} \) in the plane.
\begin{align*}
  x-y+2z &= 0 \\
  \vec{w}\cdot\vec{u} &= 0 \\
  x+y &= 0 \quad y = -x \\
  2x+2z &= 0 \\
  \vec{w} &= \begin{bmatrix}-z \\ z \\ z\end{bmatrix} = z\begin{bmatrix}
    -1 \\ 1 \\ 1
  \end{bmatrix}
\end{align*}
This vector \( \vec{w} \) satisfies the equation and the orthogonality
condition.
\[ \vec{w} = \begin{bmatrix}-1 \\ 1 \\ 1\end{bmatrix} \quad
  \vec{u} = \begin{bmatrix}1 \\ 1 \\ 0\end{bmatrix} \]

\subsubsection*{Fact}
Let \( [\vec{v_1},\vec{v_2},\dots,\vec{v_k}] \) be an orthogonal basis for a
subspace \( W \). Let \( \vec{w} \) be any vector in \( W \).
\begin{align*}
  \vec{w} &= c_1\vec{v_1}+c_2\vec{v_2}+\dots+c_k\vec{v_k} \\
  c_1 &= \frac{\vec{v_1}\cdot\vec{w}}{\vec{v_1}\cdot\vec{v_1}} \\
  c_2 &= \frac{\vec{v_2}\cdot\vec{w}}{\vec{v_2}\cdot\vec{v_2}} \\
  &\dots \\
  c_i &= \frac{\vec{v_i}\cdot\vec{w}}{\vec{v_i}\cdot\vec{v_i}}
\end{align*}

\subsubsection*{Example}
Let \( \vec{w} = \begin{bmatrix}1 \\ 2 \\ 3\end{bmatrix} \). Find the
coordinates of \( \vec{w} \) with respect to
\[ \vec{v_1} = \begin{bmatrix}2 \\ 1 \\ -1\end{bmatrix}\quad
  \vec{v_2} = \begin{bmatrix}0 \\ 1 \\ 1\end{bmatrix}\quad
  \vec{v_3} = \begin{bmatrix}1 \\ -1 \\ 1\end{bmatrix} \]
\begin{align*}
  \vec{w} &= c_1\vec{v_1}+c_2\vec{v_2}+c_3\vec{v_3} \\
  c_1 &= \frac{2+2-3}{4+1+1} = \frac{1}{6} \\
  c_2 &= \frac{0+2+3}{0+1+1} = \frac{5}{2} \\
  c_3 &= \frac{1-2+3}{1+1+1} = \frac{2}{3}
\end{align*}

\subsection*{Orthonormality}
A set of vectors is \textbf{orthonormal} if it is an orthogonal set of unit
vectors. An orthonormal basis is a basis that is an orthonormal set. Let
\( [\hat{q_1},\hat{q_2},\dots,\hat{q_k}] \) be an orthonormal basis for a
subspace \( W \). Then \( \vec{w} = c_1\hat{q_1}+c_2\hat{q_2}+\dots+
c_k\hat{q_k} \) where \( c_i = \hat{q_i}\cdot\vec{w} \).

\begin{center}
  You can find all my notes at \url{http://omgimanerd.tech/notes}. If you have
  any questions, comments, or concerns, please contact me at
  alvin@omgimanerd.tech
\end{center}

\end{document}
