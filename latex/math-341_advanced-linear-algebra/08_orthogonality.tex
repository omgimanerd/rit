\documentclass{math}

\title{Advanced Linear Algebra}
\author{Alvin Lin}
\date{January 2019 - May 2019}

\begin{document}

\maketitle

\section*{Orthogonality}
Recall that the orthogonal projection of \( \vec{v} \) onto \( \hat{u} \) is
defined as:
\begin{align*}
  \vec{v}\cdot\hat{u} &= \|\vec{v}\|\|\hat{u}\|\cos\theta \\
  proj_{\hat{u}}\vec{v} &= (\|\vec{v}\|\cos\theta)\hat{u} \\
  &= \|\vec{v}\|\frac{\vec{v}\cdot\hat{u}}{\|\vec{v}\|}\hat{u} \\
  &= (\langle v_1,v_2\rangle\cdot\langle u_1,u_2\rangle)\begin{bmatrix}
    u_1 \\ u_2
  \end{bmatrix} \\
  &= (v_1u_1+v_2u_2)\begin{bmatrix}
    u_1 \\ u_2
  \end{bmatrix} \\
  &= \begin{bmatrix}
    v_1u_1^2+v_2u_1u_2 \\
    v_1u_1u_2+v_2u_2^2
  \end{bmatrix} \\
  &= \begin{bmatrix}
    u_1^2 & u_1u_2 \\
    u_1u_2 & u_2^2
  \end{bmatrix}\begin{bmatrix}
    v_1 \\ v_2
  \end{bmatrix} \\
  &= P\vec{v}
\end{align*}
If we look at the projection matrix, we observe the following:
\begin{align*}
  P &= \begin{bmatrix}
    u_1^2 & u_1u_2 \\
    u_1u_2 & u_2^2
  \end{bmatrix} \\
  &= \begin{bmatrix}u_1 \\ u_2\end{bmatrix}
    \begin{bmatrix}u_1 & u_2\end{bmatrix} \\
  &= \hat{u}\hat{u}^T
\end{align*}
Projection is a linear transformation that can be expressed as a matrix
operation.

\subsubsection*{Example}
Given \( \hat{u} = \begin{bmatrix}\frac{1}{\sqrt{2}} \\
-\frac{1}{\sqrt{2}}\end{bmatrix} \) and \( \vec{v} = \begin{bmatrix}3 \\
4\end{bmatrix} \), find \( proj_{\hat{u}}\vec{v} \).
\begin{align*}
  P &= \hat{u}\hat{u}^T \\
  &= \begin{bmatrix}
    \frac{1}{\sqrt{2}} \\ -\frac{1}{\sqrt{2}}
  \end{bmatrix}\begin{bmatrix}
    \frac{1}{\sqrt{2}} & -\frac{1}{\sqrt{2}}
  \end{bmatrix} \\
  &= \begin{bmatrix}
    \frac{1}{2} & -\frac{1}{2} \\
    -\frac{1}{2} & \frac{1}{2}
  \end{bmatrix} \\
  proj_{\hat{u}}\vec{v} &= \begin{bmatrix}
    \frac{1}{2} & -\frac{1}{2} \\
    -\frac{1}{2} & \frac{1}{2}
  \end{bmatrix}\begin{bmatrix}
    3 \\ -4
  \end{bmatrix} \\
  &= \begin{bmatrix}
    \frac{7}{2} \\ -\frac{7}{2}
  \end{bmatrix}
\end{align*}

\subsubsection*{Example}
Show that \( P = \hat{u}\hat{u}^T \) is symmetric.
\begin{align*}
  P^T &= (\hat{u}\hat{u}^T)^T \\
  &= (\hat{u}^T)^T(\hat{u})^T \\
  &= \hat{u}\hat{u}^T
\end{align*}
Show that \( P^2 = P \) (P is idempotent).
\begin{align*}
  P^2 &= (\hat{u}\hat{u}^T)(\hat{u}\hat{u}^T) \\
  &= \hat{u}(\hat{u}^T\hat{u})\hat{u}^T \\
  \hat{u}^T\hat{u} &= 1 \\
  &= \hat{u}\hat{u}^T
\end{align*}

\subsection*{Orthogonal Projection of \( \vec{v} \) onto a Plane through Origin}
Given a plane, its normal vector \( \vec{n} \), a vector \( \vec{v} \), the
distance from the tip of \( \vec{v} \) to the plane is some length that can be
represented by a vector \( -c\vec{n} \). We can represent the projection of
\( \vec{v} \) onto the plane as \( \vec{p} = \vec{v}-c\vec{n} \).
\begin{align*}
  \vec{p} &= \vec{v}-c\vec{n} \\
  \vec{n}\cdot\vec{p} &= \vec{n}\cdot(\vec{v}-c\vec{n}) \\
  0 &= \vec{n}\cdot\vec{v}-c(\vec{n}\cdot\vec{n}) \\
  c &= \frac{\vec{n}\cdot\vec{v}}{\vec{n}\cdot\vec{n}}
\end{align*}
Alternatively, if instead of being given a unit vector \( \vec{n} \), suppose we
are given two orthogonal vectors \( \hat{u_1} \) and \( \hat{u_2} \) that span
the plane. Let \( \vec{p_1} = proj_{\hat{u_1}}\vec{v} \) and \( \vec{p_2} =
proj_{\hat{u_2}}\vec{v} \). The projection of \( \vec{v} \) onto the plane
spanned by \( \hat{u_1} \) and \( \hat{u_2} \) is equal to \( \vec{p} =
\vec{p_1}+\vec{p_2} \).
\begin{align*}
  \vec{p_1} &= proj_{\hat{u_1}}\vec{v} = \hat{u_1}\hat{u_1}^T\vec{v} \\
  \vec{p_2} &= proj_{\hat{u_2}}\vec{v} = \hat{u_2}\hat{u_2}^T\vec{v} \\
  \vec{p} &= \vec{p_1}+\vec{p_2} \\
  &= (\hat{u_1}\hat{u_1}^T+\hat{u_2}\hat{u_2}^T)\vec{v} = P\vec{v}
\end{align*}

\subsubsection*{Example}
Project \( \vec{v} = \langle1,0,-2\rangle \) onto the plane \( x+y+z = 0 \).
\begin{align*}
  \langle1,1,1\rangle\cdot\langle x,y,z\rangle &= 0 \\
  \vec{n} &= \langle1,1,1\rangle \\
  c &= \frac{\vec{n}\cdot\vec{v}}{\vec{n}\cdot\vec{n}} \\
  &= \frac{1+0-2}{1+1+1} \\
  &= -\frac{1}{2} \\
  \vec{p} &= \vec{v}+\frac{1}{3}\vec{n} \\
  &= \langle\frac{2}{3},-\frac{1}{3},-\frac{7}{3}\rangle
\end{align*}

\begin{center}
  You can find all my notes at \url{http://omgimanerd.tech/notes}. If you have
  any questions, comments, or concerns, please contact me at
  alvin@omgimanerd.tech
\end{center}

\end{document}
