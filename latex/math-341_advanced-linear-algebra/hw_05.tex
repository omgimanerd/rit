\documentclass{math}

\usepackage{enumerate}

\geometry{letterpaper, margin=0.5in}

\title{Advanced Linear Algebra: Homework 4}
\author{Alvin Lin}
\date{August 2016 - December 2016}

\begin{document}

\maketitle

\subsubsection*{Question 1}
Orthogonally diagonalize the matrix by finding an orthogonal matrix \( Q \)
and a diagonal matrix \( D \) such that \( Q^TAQ = D \).
\begin{align*}
  A &= \begin{bmatrix}3 & 1 \\ 1 & 3\end{bmatrix} \\
  D &= P^{-1}AP \\
  \begin{bmatrix}2 & 0 \\ 0 & 4\end{bmatrix} &=
    \begin{bmatrix}-1 & 1 \\ 1 & 1\end{bmatrix}^{-1}
    \begin{bmatrix}3 & 1 \\ 1 & 3\end{bmatrix}
    \begin{bmatrix}-1 & 1 \\ 1 & 1\end{bmatrix} \\
  \vec{v_1} &= \begin{bmatrix}-1 \\ 1\end{bmatrix} \quad
    \vec{v_2} = \begin{bmatrix}1 \\ 1\end{bmatrix} \\
  \vec{u_1} &= \begin{bmatrix}
    -\frac{1}{\sqrt{2}} \\ \frac{1}{\sqrt{2}}
  \end{bmatrix} \quad \vec{u_2} = \begin{bmatrix}
    \frac{1}{\sqrt{2}} \\ \frac{1}{\sqrt{2}}
  \end{bmatrix} \\
  Q &= \begin{bmatrix}
    -\frac{1}{\sqrt{2}} & \frac{1}{\sqrt{2}} \\
    \frac{1}{\sqrt{2}} & \frac{1}{\sqrt{2}}
  \end{bmatrix}
\end{align*}

\subsubsection*{Question 2}
Orthogonally diagonalize the matrix by finding an orthogonal matrix \( Q \)
and a diagonal matrix \( D \) such that \( Q^TAQ = D \).
\begin{align*}
  A &= \begin{bmatrix}
    5 & 0 & 0 \\
    0 & 1 & 3 \\
    0 & 3 & 1
  \end{bmatrix} \\
  D &= P^{-1}AP \\
  \begin{bmatrix}
    -2 & 0 & 0 \\
    0 & 4 & 0 \\
    0 & 0 & 5
  \end{bmatrix} &= \begin{bmatrix}
    0 & 0 & 1 \\
    -1 & 1 & 0 \\
    1 & 1 & 0
  \end{bmatrix}^{-1}\begin{bmatrix}
    5 & 0 & 0 \\
    0 & 1 & 3 \\
    0 & 3 & 1
  \end{bmatrix}\begin{bmatrix}
    0 & 0 & 1 \\
    -1 & 1 & 0 \\
    1 & 1 & 0
  \end{bmatrix} \\
  \vec{v_1} &= \begin{bmatrix}0 \\ -1 \\ 1\end{bmatrix} \quad
    \vec{v_2} = \begin{bmatrix}0 \\ 1 \\ 1\end{bmatrix} \quad
    \vec{v_3} = \begin{bmatrix}1 \\ 0 \\ 0\end{bmatrix} \\
  \vec{u_1} &= \begin{bmatrix}
    0 \\ -\frac{1}{\sqrt{2}} \\ \frac{1}{\sqrt{2}}
  \end{bmatrix} \quad \vec{u_2} = \begin{bmatrix}
    0 \\ \frac{1}{\sqrt{2}} \\ \frac{1}{\sqrt{2}}
  \end{bmatrix} \quad \vec{u_3} = \vec{v_3} \\
  Q &= \begin{bmatrix}
    0 & 0 & 1 \\
    -\frac{1}{\sqrt{2}} & \frac{1}{\sqrt{2}} & 0 \\
    \frac{1}{\sqrt{2}} & \frac{1}{\sqrt{2}} & 0
  \end{bmatrix}
\end{align*}

\subsubsection*{Question 3}
Find a spectral decomposition of the matrix.
\begin{align*}
  A &= \begin{bmatrix}3 & 1 \\ 1 & 3\end{bmatrix} \\
  \lambda_1 &= 2 \quad \lambda_2 = 4 \\
  \vec{v_1} &= \begin{bmatrix}-1 \\ 1\end{bmatrix} \quad
    \vec{v_2} = \begin{bmatrix}1 \\ 1\end{bmatrix} \\
  \vec{q_1} &= \begin{bmatrix}
    -\frac{1}{\sqrt{2}} \\ \frac{1}{\sqrt{2}}
  \end{bmatrix} \quad \vec{q_2} = \begin{bmatrix}
    \frac{1}{\sqrt{2}} \\ \frac{1}{\sqrt{2}}
  \end{bmatrix} \\
  \lambda_1\vec{q_1}\vec{q_1}^T &=
    \begin{bmatrix}1 & -1 \\ -1 & 1\end{bmatrix} \\
  \lambda_2\vec{q_2}\vec{q_2}^T &=
    \begin{bmatrix}2 & 2 \\ 2 & 2\end{bmatrix}
\end{align*}

\subsubsection*{Question 4}
Find a spectral decomposition of the matrix.
\begin{align*}
  A &= \begin{bmatrix}
    6 & 0 & 0 \\
    0 & 1 & 3 \\
    0 & 3 & 1
  \end{bmatrix} \\
  \lambda_1 &= -2 \quad \lambda_2 = 4 \quad \lambda_3 = 6 \\
  \vec{v_1} &= \begin{bmatrix}0 \\ -1 \\ 1\end{bmatrix} \quad
    \vec{v_2} = \begin{bmatrix}0 \\ 1 \\ 1\end{bmatrix} \quad
    \vec{v_3} = \begin{bmatrix}1 \\ 0 \\ 0\end{bmatrix} \\
  \vec{q_1} &= \begin{bmatrix}
    0 \\ -\frac{1}{\sqrt{2}} \\ \frac{1}{\sqrt{2}}
  \end{bmatrix} \quad \vec{q_2} = \begin{bmatrix}
    0 \\ \frac{1}{\sqrt{2}} \\ \frac{1}{\sqrt{2}}
  \end{bmatrix} \quad \vec{q_3} = \vec{v_3} \\
  \lambda_1\vec{q_1}\vec{q_1}^T &= \begin{bmatrix}
    0 & 0 & 0 \\
    0 & -1 & 1 \\
    0 & 1 & -1
  \end{bmatrix} \\
  \lambda_2\vec{q_2}\vec{q_2}^T &= \begin{bmatrix}
    0 & 0 & 0 \\
    0 & 2 & 2 \\
    0 & 2 & 2
  \end{bmatrix} \\
  \lambda_2\vec{q_3}\vec{q_3}^T &= \begin{bmatrix}
    6 & 0 & 0 \\
    0 & 0 & 0 \\
    0 & 0 & 0
  \end{bmatrix}
\end{align*}

\subsubsection*{Question 5}
Find a symmetric \( 2\times2 \) matrix with eigenvalues \( \lambda_1 \) and
\( \lambda_2 \) and corresponding orthogonal eigenvectors \( \vec{v_1} \) and
\( \vec{v_2} \).
\begin{align*}
  \lambda_1 &= -1 \quad \lambda_2 = 4 \\
  \vec{v_1} &= \begin{bmatrix}1 \\ 1\end{bmatrix} \quad
    \vec{v_2} = \begin{bmatrix}1 \\ -1\end{bmatrix} \\
  \vec{q_1} &= \begin{bmatrix}
    \frac{1}{\sqrt{2}} \\ \frac{1}{\sqrt{2}}
  \end{bmatrix} \quad \vec{q_2} = \begin{bmatrix}
    \frac{1}{\sqrt{2}} \\ -\frac{1}{\sqrt{2}}
  \end{bmatrix} \\
  \lambda_1\vec{q_1}\vec{q_1}^T &= \begin{bmatrix}
    -\frac{1}{2} & -\frac{1}{2} \\ -\frac{1}{2} & -\frac{1}{2}
  \end{bmatrix} \\
  \lambda_2\vec{q_2}\vec{q_2}^T &= \begin{bmatrix}
    2 & -2 \\ -2 & 2
  \end{bmatrix} \\
  A &= \begin{bmatrix}
    \frac{3}{2} & -\frac{5}{2} \\ -\frac{5}{2} & \frac{3}{2}
  \end{bmatrix}
\end{align*}

\subsubsection*{Question 6}
Evaluate the quadratic form \( f(\vec{x}) = \vec{x}^TA\vec{x} \) for the given
\( A \) and \( \vec{x} \).
\begin{align*}
  A &= \begin{bmatrix}3 & 1 \\ 1 & 4\end{bmatrix} \quad
    \vec{x} = \begin{bmatrix}x \\ y\end{bmatrix} \\
  f(\vec{x}) &= 3x^2+xy+xy+4y^2 \\
  &= 3x^2+2xy+4y^2
\end{align*}

\subsubsection*{Question 7}
Evaluate the quadratic form \( f(\vec{x}) = \vec{x}^TA\vec{x} \) for the given
\( A \) and \( \vec{x} \).
\begin{align*}
  A &= \begin{bmatrix}3 & -2 \\ -2 & 4\end{bmatrix} \quad
    \vec{x} = \begin{bmatrix}1 \\ 5\end{bmatrix}
  A &= 3x^2-4xy+4y^2 \\
  &= 3(1)^2-4(1)(5)+4(5)^2 \\
  &= 83
\end{align*}

\subsubsection*{Question 8}
Evaluate the quadratic form \( f(\vec{x}) = \vec{x}^TA\vec{x} \) for the given
\( A \) and \( \vec{x} \).
\begin{align*}
  A &= \begin{bmatrix}
    1 & 0 & -3 \\
    0 & 2 & 1 \\
    -3 & 1 & 3
  \end{bmatrix} \quad \vec{x} = \begin{bmatrix}x \\ y \\ z\end{bmatrix} \\
  f(\vec{x}) &= \begin{bmatrix}x & y & z\end{bmatrix}\begin{bmatrix}
    x-3z \\ 2y+z \\ -3x+y+3z
  \end{bmatrix} \\
  &= x^2-3xz+2y^2+yz-3xz+yz+3z^2 \\
  &= x^2+2y^2+3z^2+2yz-6xz
\end{align*}

\subsubsection*{Question 9}
Find the symmetric matrix \( A \) associated with the given quadratic form.
\[ x_1^2+4x_2^2+6x_1x_2 \]
\[ \begin{bmatrix}1 & 3 \\ 3 & 4\end{bmatrix} \]

\subsubsection*{Question 10}
Diagonalize the quadratic form by finding an orthogonal matrix \( Q \) such that
the change of variable \( \vec{x} = Q\vec{y} \) transforms the given form into
one with no cross-product terms. Given \( Q \) and the new quadratic form
\( f(\vec{y}) \).
\begin{align*}
  f(x_1,x_2) &= 4x_1^2+7x_2^2-4x_1x_2 \\
  A &= \begin{bmatrix}4 & -2 \\ - 2 & 7\end{bmatrix} \\
  \lambda_1 &= 3 \quad \lambda_2 = 8 \\
  \vec{v_1} &= \begin{bmatrix}2 \\ 1\end{bmatrix} \quad
    \vec{v_2} = \begin{bmatrix}-\frac{1}{2} \\ 1\end{bmatrix} =
      \begin{bmatrix}-1 \\ 2\end{bmatrix} \\
  \vec{q_1} &= \begin{bmatrix}
    \frac{2}{\sqrt{5}} \\ \frac{1}{\sqrt{5}}
  \end{bmatrix} \quad \vec{q_2} = \begin{bmatrix}
    -\frac{1}{\sqrt{5}} \\ \frac{2}{\sqrt{5}}
  \end{bmatrix} \\
  Q &= \begin{bmatrix}
    \frac{2}{\sqrt{5}} & -\frac{1}{\sqrt{5}} \\
    \frac{1}{\sqrt{5}} & \frac{2}{\sqrt{5}}
  \end{bmatrix} \\
  D &= \begin{bmatrix}3 & 0 \\ 0 & 8\end{bmatrix} \\
  f(y_1,y_2) &= 3y_1^2+8y_2^2
\end{align*}

\subsubsection*{Question 11}
Use a rotation of axes to put the conic in standard position.
\begin{align*}
  x^2+xy+y^2 &= 9 \\
  A &= \begin{bmatrix}
    1 & \frac{1}{2} \\
    \frac{1}{2} & 1
  \end{bmatrix} \\
  \lambda_1 &= \frac{1}{2} \quad \lambda_2 = \frac{3}{2} \\
  D &= \begin{bmatrix}\frac{1}{2} & 0 \\ 0 & \frac{3}{2}\end{bmatrix} \\
  \frac{1}{2}x'+\frac{3}{2}y' &= 9
\end{align*}

\subsubsection*{Question 12}
Determine whether the given set, together with specified operations of addition
and scalar multiplication, is a vector space. If it is not, select all of the
axioms that fail to hold. \par
The set of all vectors in \( \R^2 \) of the form \( \begin{bmatrix}-3x \\ -3x
\end{bmatrix} \), with the usual vector addition and scalar multiplication. \par
All of the axioms hold, so the given set is a vector space.

\subsubsection*{Question 13}
Determine whether the given set, together with specified operations of addition
and scalar multiplication, is a vector space. If it is not, select all of the
axioms that fail to hold. \par
The set of all vectors in \( \begin{bmatrix}x \\ y\end{bmatrix} \) in \( \R^2 \)
with \( x\ge0,y\ge0 \), with the usual vector addition and scalar
multiplication. \par
\begin{enumerate}
  \item For each \( \vec{u} \) in \( V \), there is an element \( -\vec{u} \)
    in \( V \) such that \( \vec{u}+(-\vec{u}) = \vec{0} \).
  \item \( c\vec{u} \) is in \( V \).
\end{enumerate}

\subsubsection*{Question 14}
Determine whether the given set, together with specified operations of addition
and scalar multiplication, is a vector space. If it is not, select all of the
axioms that fail to hold. \par
\( \R^2 \), with the usual addition but scalar multiplication defined by
\[ c\begin{bmatrix}x \\ y\end{bmatrix} = \begin{bmatrix}x \\ cy\end{bmatrix} \]
?

\subsubsection*{Question 15}
Let \( V \) be a vector space and let \( W \) be a nonempty subset of \( V \).
Then \( W \) is a subspace of \( V \) if and only if the following conditions
hold.
\begin{enumerate}[(a)]
  \item If \( \vec{u} \) and \( \vec{v} \) are in \( W \), then \( \vec{u}+
    \vec{v} \) is in \( W \).
  \item If \( \vec{u} \) is in \( W \) and \( c \) is a scalar, then
    \( c\vec{u} \) is in \( W \).
\end{enumerate}
Use the theorem above to determine whether \( W \) is a subspace of \( V \).
\[ V = \R^3, W = \left\{\begin{bmatrix}a \\ b \\ |a|\end{bmatrix}\right\} \]
\( W \) is not a subspace of \( V \).

\subsubsection*{Question 16}
Let \( V \) be a vector space and let \( W \) be a nonempty subset of \( V \).
Then \( W \) is a subspace of \( V \) if and only if the following conditions
hold.
\begin{enumerate}[(a)]
  \item If \( \vec{u} \) and \( \vec{v} \) are in \( W \), then \( \vec{u}+
    \vec{v} \) is in \( W \).
  \item If \( \vec{u} \) is in \( W \) and \( c \) is a scalar, then
    \( c\vec{u} \) is in \( W \).
\end{enumerate}
Use the theorem above to determine whether \( W \) is a subspace of \( V \).
\[ V = M_{22}, W = \left\{\begin{bmatrix}a & b \\ c & d\end{bmatrix}: ad\ge bc
  \right\} \]
\( W \) is not a subspace of \( V \).

\subsubsection*{Question 17}
Let \( V \) be a vector space and let \( W \) be a nonempty subset of \( V \).
Then \( W \) is a subspace of \( V \) if and only if the following conditions
hold.
\begin{enumerate}[(a)]
  \item If \( \vec{u} \) and \( \vec{v} \) are in \( W \), then \( \vec{u}+
    \vec{v} \) is in \( W \).
  \item If \( \vec{u} \) is in \( W \) and \( c \) is a scalar, then
    \( c\vec{u} \) is in \( W \).
\end{enumerate}
Use the theorem above to determine whether \( W \) is a subspace of \( V \).
\[ V = M_{nn}, W = \{A\text{ in }M_{nn}: \det(A) = 1\} \]
\( W \) is not a subspace of \( V \).

\subsubsection*{Question 18}
Let \( V \) be a vector space and let \( W \) be a nonempty subset of \( V \).
Then \( W \) is a subspace of \( V \) if and only if the following conditions
hold.
\begin{enumerate}[(a)]
  \item If \( \vec{u} \) and \( \vec{v} \) are in \( W \), then \( \vec{u}+
    \vec{v} \) is in \( W \).
  \item If \( \vec{u} \) is in \( W \) and \( c \) is a scalar, then
    \( c\vec{u} \) is in \( W \).
\end{enumerate}
Use the theorem above to determine whether \( W \) is a subspace of \( V \).
\[ V = P_2, W = \{bx+cx^2\} \]
\( W \) is a subspace of \( V \).

\subsubsection*{Question 19}
Let \( V \) be a vector space and let \( W \) be a nonempty subset of \( V \).
Then \( W \) is a subspace of \( V \) if and only if the following conditions
hold.
\begin{enumerate}[(a)]
  \item If \( \vec{u} \) and \( \vec{v} \) are in \( W \), then \( \vec{u}+
    \vec{v} \) is in \( W \).
  \item If \( \vec{u} \) is in \( W \) and \( c \) is a scalar, then
    \( c\vec{u} \) is in \( W \).
\end{enumerate}
Use the theorem above to determine whether \( W \) is a subspace of \( V \).
\[ V = P_2, W = \{a+bx+cx^2: abc = 0\} \]
\( W \) is not a subspace of \( V \).

\subsubsection*{Question 20}
Let \( A = \begin{bmatrix}1 & 1 \\ -1 & 1\end{bmatrix} \) and \( B =
\begin{bmatrix}1 & -1 \\ 1 & 0\end{bmatrix} \). Determine whether \( C \) is
in \( span(A,B) \).
\[ C = \begin{bmatrix}1 & 3 \\ 3 & 4\end{bmatrix} \]
\( C \) is not in \( span(A,B) \).

\subsubsection*{Question 21}
Let \( p(x) = 1-2x, q(x) = x-x^2, r(x) = -2+3x+x^2 \). Determine whether
\( s(x) \) is in \( span(p(x),q(x),r(x)) \).
\begin{align*}
  s(x) &= 4+x+x^2 \\
  ap(x)+bq(x)+cr(x) &= s(x) \\
  a-2ax+bx-bx^2-2c+3cx+cx^2 &= 4+x+x^2 \\
  (a-2c)+(-2a+b+3c)x+(-b+c)x^2 &= 4+x+x^2 \\
  \left[\begin{array}{ccc|c}
    1 & 0 & -2 & 4 \\
    -2 & 1 & 3 & 1 \\
    0 & -1 & 1 & 1
  \end{array}\right] &\to \text{inconsistent}
\end{align*}
Since the system is inconsistent, \( s(x) \) is not in the span.

\subsubsection*{Question 22}
Test the set of matrices for linear independence in \( M_{22} \). If it is
linearly dependent, express one of the matrices as a linear combination of the
others.
\[ \left\{A = \begin{bmatrix}1 & 1 \\ 0 & -1\end{bmatrix},
  B = \begin{bmatrix}1 & -1 \\ 1 & 0\end{bmatrix},
  C = \begin{bmatrix}1 & 0 \\ 5 & 4\end{bmatrix}\right\} \]
\begin{align*}
  c_1A+c_2B+c_3C &= 0 \\
  \left[\begin{array}{ccc|c}
    1 & 1 & 1 & 0 \\
    1 & -1 & 0 & 0 \\
    0 & 1 & 5 & 0 \\
    -1 & 0 & 4 & 0
  \end{array}\right] &\to \text{trivial}
\end{align*}
This system has only the trivial solution, so the matrices are linearly
independent.

\subsubsection*{Question 23}
Test the set of matrices for linear dependence in \( M_{22} \). If it is
linearly dependent, express one of the matrices as a linear combination of the
others.
\[ \left\{A = \begin{bmatrix}-1 & 1 \\ -2 & 2\end{bmatrix},
  B = \begin{bmatrix}3 & 0 \\ 1 & 1\end{bmatrix},
  C = \begin{bmatrix}0 & 2 \\ -3 & 1\end{bmatrix},
  D = \begin{bmatrix}-1 & 0 \\ -1 & 7\end{bmatrix}\right\} \]
\begin{align*}
  c_1A+c_2B+c_3C+c_3D &= 0 \\
  \left[\begin{array}{cccc|c}
    -1 & 3 & 0 & -1 & 0 \\
    1 & 0 & 2 & 0 & 0 \\
    -2 & 1 & -3 & -1 & 0 \\
    2 & 1 & 1 & 7 & 0
  \end{array}\right] &\to \left[\begin{array}{cccc|c}
    1 & 0 & 0 & 4 & 0 \\
    0 & 1 & 0 & 1 & 0 \\
    0 & 0 & 1 & -2 & 0 \\
    0 & 0 & 0 & 0 & 0
  \end{array}\right] \\
  c_1 &= -4c_4 \quad c_2 = -c_4 \quad c_3 = 2c_4 \\
\end{align*}

\subsubsection*{Question 24}
Test the set of polynomials for linear independence. If it is linearly
dependent, express one of the polynomials as a linear combination of the others.
\[ \{f(x) = 3+x, g(x) = 3+x^2, h(x) = 3-x+x^2\} \]
\begin{align*}
  af(x)+bg(x)+ch(x) &= 0 \\
  3a+ax+3b+bx^2+3c-cx+cx^2 &= 0 \\
  (3a+3b+3c)+(a-c)x+(b+c)x^2 &= 0 \\
  \left[\begin{array}{ccc|c}
    3 & 3 & 3 & 0 \\
    1 & 0 & -1 & 0 \\
    0 & 1 & 1 & 0
  \end{array}\right] &\to \text{trivial}
\end{align*}
This system has only the trivial solution, so the matrices are linearly
independent.

\subsubsection*{Question 25}
Test the set of polynomials for linear independence. If it is linearly
dependent, express one of the polynomials as a linear combination of the
others.
\[ \{f(x) = x, g(x) = 2x-x^2, h(x) = 2x+3x^2\} \]
\begin{align*}
  af(x)+bg(x)+ch(x) &= 0 \\
  ax+2bx-bx^2+2cx+3cx^2 &= 0 \\
  (a+2b+2c)x+(-b+3c)x^2 &= 0 \\
  \left[\begin{array}{ccc|c}
    1 & 2 & 2 & 0 \\
    0 & -1 & 3 & 0
  \end{array}\right] &\to \left[\begin{array}{ccc|c}
    1 & 0 & 8 & 0 \\
    0 & 1 & -3 & 0
  \end{array}\right] \\
  a &= -8c \quad b = 3c
\end{align*}

\subsubsection*{Question 26}
Determine whether the set \( \mathbb{B} \) is a basis for the vector space
\( V \).
\begin{align*}
  V &= M_{22}, \mathbb{B} = \left\{
    \begin{bmatrix}1 & 0 \\ 0 & 1\end{bmatrix},
    \begin{bmatrix}1 & 1 \\ 0 & 1\end{bmatrix},
    \begin{bmatrix}-1 & 1 \\ 1 & -1\end{bmatrix}
  \right\} \\
  \left[\begin{array}{ccc|c}
    1 & 1 & -1 & 0 \\
    0 & 1 & 1 & 0 \\
    0 & 0 & 1 & 0 \\
    1 & 1 & -1 & 0
  \end{array}\right] &\to \text{trivial}
\end{align*}
The set \( \mathbb{B} \) is linearly independent but does not span the vector
space, so it is not a basis.

\subsubsection*{Question 27}
Determine whether the set \( \mathbb{B} \) is a basis for the vector space
\( V \).
\begin{align*}
  V &= M_{22}, \mathbb{B} = \left\{
    \begin{bmatrix}1 & 0 \\ 0 & 1\end{bmatrix},
    \begin{bmatrix}0 & 1 \\ -1 & 0\end{bmatrix},
    \begin{bmatrix}1 & -1 \\ -1 & 1\end{bmatrix},
    \begin{bmatrix}-1 & 1 \\ 1 & 1\end{bmatrix}
  \right\} \\
  \left[\begin{array}{cccc|c}
    1 & 0 & 1 & -1 & 0 \\
    0 & 1 & -1 & 1 & 0 \\
    0 & -1 & -1 & 1 & 0 \\
    1 & 0 & 1 & 1 & 0
  \end{array}\right] &\to \text{trivial}
\end{align*}
The set \( \mathbb{B} \) is linearly independent and spans the vector space
(a solution exists), so it is a basis.

\subsubsection*{Question 28}
Determine whether the set \( \mathbb{B} \) is a basis for the vector space
\( V \).
\begin{align*}
  V &= P_2, \mathbb{B} = \{x,4+x,x-x^2\} \\
  (ax)+(4b+bx)+(cx-cx^2) &= 0 \\
  (4b)+(a+b+c)x+(-c)x^2 &= 0 \\
  \left[\begin{array}{ccc|c}
    0 & 4 & 0 & 0 \\
    1 & 1 & 1 & 0 \\
    0 & 0 & -1 & 0
  \end{array}\right]
\end{align*}
The set \( \mathbb{B} \) is linearly independent and spans the vector space
(a solution exists), so it is a basis.

\subsubsection*{Question 29}
Find the coordinate vector of \( A = \begin{bmatrix}3 & 4 \\ 5 & 6\end{bmatrix}
\) with respect to the basis \( \mathbb{B} = \{E_{22},E_{21},E_{12},E_{11}\} \)
of \( M_{22} \).
\[ A = 6E_{22}+5E_{21}+4E_{12}+3E_{11} \]

\subsubsection*{Question 30}
Find the coordinate vector of \( p(x)=6-x+5x^2 \) with respect to the basis
\( \mathbb{B} = \{1,1+x,-1+x^2\} \) of \( P_2 \).
\begin{align*}
  (a)+(b+bx)+(-c+cx^2) &= 6-x+5x^2 \\
  (a+b-c)+(b)x+(c)x^2 &= 6-x+5x^2 \\
  \left[\begin{array}{ccc|c}
    1 & 1 & -1 & 6 \\
    0 & 1 & 0 & -1 \\
    0 & 0 & 1 & 5
  \end{array}\right] &\to \left[\begin{array}{ccc|c}
    1 & 0 & 0 & 12 \\
    0 & 1 & 0 & -1 \\
    0 & 0 & 1 & 5
  \end{array}\right] \\
  [p]_{\mathbb{B}} &= \begin{bmatrix}12 \\ -1 \\ 5\end{bmatrix}
\end{align*}

\subsubsection*{Question 31}
Find a formula for the dimension of the vector space \( V \) of symmetric
\( n\times n \) matrices.
\[ dim(V) = \frac{n(n+1)}{2} \]
The basis of an \( n\times n \) matrix

\begin{center}
  If you have any questions, comments, or concerns, please contact me at
  alvin@omgimanerd.tech
\end{center}

\end{document}
