\documentclass{math}

\title{Advanced Linear Algebra}
\author{Alvin Lin}
\date{January 2019 - May 2019}

\begin{document}

\maketitle

\section*{Abstract Vector Spaces}
Recall the definition of linear dependence:
\begin{align*}
  c_1v_1+\dots+c_nv_n &= 0 \\
  c_1 = c_2 = \dots &= 0 \\
  v_1,\dots,v_n &\text{ are linearly dependent}
\end{align*}

\subsubsection*{Example}
\[ \{\sin^2(x),\cos^2(x),\cos(2x)\} \]
Since \( \cos(2x) = \cos^2(x)-\sin^2(x) \), this set is linearly dependent.

\subsubsection*{Example}
\[ \{1+x,x+x^2,1+x^2\} \]
If the equation
\[ c_1(1+x)+c_2(x+x^2)+c_3(1+x^2) = 0 \]
has a nontrivial solution for \( c_1,c_2,c_3 \), then one element can be
written as a linear combination of the other elements, and thus the set is
linearly dependent.
\begin{align*}
  (c_1+c-3)+(c_1+c_3)x+(c_2+c_3)x^2 &= 0 \\
  c_1+c_3 &= 0 \\
  c_1+c_2 &= 0 \\
  c_2+c_3 &= 0
\end{align*}
We must examine this system of linear equations to determine if it has a
nontrivial solution. We can so this using an augmented matrix.
\[ \left[\begin{array}{ccc|c}
  1 & 0 & 1 & 0 \\
  1 & 1 & 0 & 0 \\
  0 & 1 & 1 & 0
\end{array}\right] \to \left[\begin{array}{ccc|c}
  1 & 0 & 1 & 0 \\
  0 & 1 & -1 & 0 \\
  0 & 0 & 2 & 0
\end{array}\right] \]
Since this augmented matrix has only the trivial solution \( c_1 = c_2 = c_3 =
0 \), the set is linearly independent. Recall that \( \mathbb{B} \) is a basis
for \( V \) if \( \mathbb{B} \) spans \( V \) and \( \mathbb{B} \) is linearly
independent. We know that \( \mathbb{B} \) spans \( V \) if
\[ a+bx+cx^2 = c_1v_1+c_2v_2+c_3v_3 \]
We can use the augmented matrix from before to construct the following matrix:
\[ \left[\begin{array}{ccc|c}
  1 & 0 & 1 & a \\
  1 & 1 & 0 & b \\
  0 & 1 & 1 & c
\end{array}\right] \]
Since the rank of the matrix is 3, the matrix is invertible, which means there
exist a unique \( c_1,c_3,c_3 \) for any \( a,b,c \). Therefore, the set
\( \{1+x,x+x^2,1+x^2\} \) is a basis for the polynomial space \( P_2 \).

\subsubsection*{Example}
Find a basis for \( W_2 = \{a+bx-bx^2+ax^3\} \) where \( a \) and \( b \) are
arbitrary constants.
\begin{align*}
  a+bx-bx^2+ax^3 &= a(1+x^3)+b(x-x^2) \\
  &= au+bv \\
  u &= 1+x^3 \quad v = x-x^2 \\
  W_2 &= span\{1+x^3,x-x^2\}
\end{align*}

\subsubsection*{Example}
Find a basis for \( W_3 = \begin{bmatrix}a & b \\ -b & a\end{bmatrix} \).
\begin{align*}
  \begin{bmatrix}a & b \\ -b & a\end{bmatrix} &=
    a\begin{bmatrix}1 & 0 \\ 0 & 1\end{bmatrix}+
    b\begin{bmatrix}0 & 1 \\ -1 & 0\end{bmatrix} \\
  W_2 &= span\left\{\begin{bmatrix}1 & 0 \\ 0 & 1\end{bmatrix},
    \begin{bmatrix}0 & 1 \\ -1 & 0\end{bmatrix}\right\}
\end{align*}

\subsection*{Standard Basis}
\begin{align*}
  \R^n &= span\{\hat{e_1},\hat{e_2},\dots,\hat{e_n}\} \\
  P^n &= span\{1,x,x^2,\dots,x^n\} \\
  M_{mn} &=
    span\{E_{11},\dots,E_{1n},E_{21},\dots,E_{2n},\dots,E_{m1},\dots,E_{mn}\}
\end{align*}

\subsection*{Coordinates}
Let \( \mathbb{B} = \{v_1,\dots,v_n\} \) and let \( v = c_1v_2+\dots+c_nv_n \).
Then the coordinate representation \( [v]_{\mathbb{B}} = \begin{bmatrix}
c_1 \\ c_2 \\ \vdots \\ c_n\end{bmatrix} \). For example, if we have the
space \( P_2 \) whose basis is \( \mathbb{B} = \{1,x,x^2\} \) and a polynomial
\( p(x) = 2-3x+5x^2 \), then \( [p]_{\mathbb{B}} = \begin{bmatrix}2 \\ -3 \\ 5
\end{bmatrix} \).

\subsubsection*{Example}
Suppose we have the space \( P_2 \) with basis \( \mathbb{B} =
\{1+x,x+x^2,1+x^2\} \) and \( p = 1+2x-x^2 \). Find \( [p]_{\mathbb{B}} \).
\begin{align*}
  c_1(1+x)+c_2(x+x^2)+c_3(1+x^2) &= 1+2x-x^2 \\
  (c_1+c_3)+(c_1+c_2)x+(c_2+c_3)x^2 &= 1+2x-x^2 \\
  \left[\begin{array}{ccc|c}
    1 & 0 & 1 & 1 \\
    1 & 1 & 0 & 2 \\
    0 & 1 & 1 & -1
  \end{array}\right] &\to \left[\begin{array}{ccc|c}
    1 & 0 & 0 & 2 \\
    0 & 1 & 0 & 0 \\
    0 & 0 & 1 & -1
  \end{array}\right] \\
  [p]_{\mathbb{B}} &= \begin{bmatrix}c_1 \\ c_2 \\ c_3\end{bmatrix} =
    \begin{bmatrix}2 \\ 0 \\ -1\end{bmatrix}
\end{align*}

\subsubsection*{Example}
Find the dimension of the vector space of symmetric \( 2\times2 \) matrices.
\begin{align*}
  \begin{bmatrix}a & b \\ b & c\end{bmatrix} &=
    a\begin{bmatrix}1 & 0 \\ 0 & 0\end{bmatrix}+
    b\begin{bmatrix}0 & 1 \\ 1 & 0\end{bmatrix}+
    c\begin{bmatrix}0 & 0 \\ 0 & 1\end{bmatrix} \\
  W &= span\left\{
    \begin{bmatrix}1 & 0 \\ 0 & 0\end{bmatrix},
    \begin{bmatrix}0 & 1 \\ 1 & 0\end{bmatrix},
    \begin{bmatrix}0 & 0 \\ 0 & 1\end{bmatrix}
  \right\} \\
  \dim(W) &= 3
\end{align*}

\begin{center}
  You can find all my notes at \url{http://omgimanerd.tech/notes}. If you have
  any questions, comments, or concerns, please contact me at
  alvin@omgimanerd.tech
\end{center}

\end{document}
