\documentclass{math}

\title{Advanced Linear Algebra}
\author{Alvin Lin}
\date{January 2019 - May 2019}

\begin{document}

\maketitle

\section*{Systems of Linear Ordinary Differential Equations}
A solution to an ordinary differential equation is a function that satisfies
the differential equation. Suppose we have the following differential equation:
\begin{align*}
  \ddiff{x_1}{t} &= x_1+2x_2 \\
  \ddiff{x_2}{t} &= 3x_1+2x_2
\end{align*}
We can convert it into a matrix equation:
\begin{align*}
  \ddiff{}{t}\begin{bmatrix}x_1 \\ x_2\end{bmatrix} &= \begin{bmatrix}
    1 & 2 \\
    3 & 2
  \end{bmatrix}\begin{bmatrix}x_1 \\ x_2\end{bmatrix} \\
  \diff{\vec{x}}{t} &= A\vec{x}
\end{align*}
If we diagonalize the matrix \( A \), we can obtain a matrix with its
eigenvalues:
\begin{align*}
  \ddiff{\vec{x}}{t} &= (PDP^{-1})\vec{x} \\
  P &= \begin{bmatrix}
    2 & -1 \\
    3 & 1
  \end{bmatrix} \\
  D &= \begin{bmatrix}
    4 & 0 \\
    0 & -1
  \end{bmatrix}
\end{align*}
If we multiply both sides of the equation by \( P^{-1} \), we can group together
the term with \( \vec{x} \):
\begin{align*}
  (P^{-1})\ddiff{\vec{x}}{t} &= (P^{-1})(PDP^{-1})\vec{x} \\
  \ddiff{P^{-1}\vec{x}}{t} &= D(P^{-1}\vec{x})
\end{align*}
Let \( \vec{y} = P^{-1}\vec{x} \), we can now decouple the system of equations
and solve it since it is multiplied by a diagonal matrix of eigenvalues.
\begin{align*}
  \ddiff{\vec{y}}{t} &= D\vec{y} \\
  \ddiff{}{t}\begin{bmatrix}y_1 \\ y_2\end{bmatrix} &= \begin{bmatrix}
    4 & 0 \\
    0 & -1
  \end{bmatrix}\begin{bmatrix}y_1 \\ y_2\end{bmatrix} \\
  \ddiff{y_1}{t} &= 4y_1 \quad y_1 = c_2\e^{4t} \\
  \ddiff{y_2}{t} &= -y_2 \quad y_2 = c_2\e^{-t}
\end{align*}
Using this solution for \( \vec{y} \), we can solve for \( \vec{x} \):
\begin{align*}
  \vec{y} &= \begin{bmatrix}
    c_1\e^{4t} \\ c_2\e^{-t}
  \end{bmatrix} \\
  \vec{x} &= P\vec{y} \\
  \begin{bmatrix}x_1 \\ x_2\end{bmatrix} &= \begin{bmatrix}
    2 & -1 \\
    3 & 1
  \end{bmatrix}\begin{bmatrix}c_1\e^{4t} \\ c_2\e^{-t}\end{bmatrix} \\
  \vec{x} &= c_1\e^{4t}\begin{bmatrix}2 \\ 3\end{bmatrix}+
    c_2\e^{-t}\begin{bmatrix}-1 \\ 1\end{bmatrix} \\
  x_1 &= 2c_1\e^{4t}-c_2\e^{-t} \\
  x_2 &= 3c_1\e^{4t}-c_2\e^{-t}
\end{align*}
In general, with coupled systems of linear ordinary differential equations,
solutions will be of the form:
\[ \vec{x} = c_1\e^{\lambda_1t}\vec{v_1}+c_2\e^{\lambda_2t}\vec{v_2}+\dots+
  c_n\e^{\lambda_nt}\vec{v_n} \]

\subsubsection*{Example}
Solve the following system of linear ordinary differential equations.
\[ \ddiff{x_1}{t} = x_2 \quad x_1(0) = -5 \quad \ddiff{x_2}{t} = -x_1 \quad
  x_2(0) = 8 \]
\begin{align*}
  \ddiff{}{t}\begin{bmatrix}x_1 \\ x_2\end{bmatrix} &= \begin{bmatrix}
    0 & 1 \\
    -1 & 0
  \end{bmatrix}\begin{bmatrix}x_1 \\ x_2\end{bmatrix} \\
  \ddiff{\vec{x}}{t} &= A\vec{x} \\
  \det(A-\lambda I) &= 0 \\
  \begin{vmatrix}
    -\lambda & 1 \\
    -1 & -\lambda
  \end{vmatrix} &= 0 \\
  \lambda^2+1 &= 0 \\
  \lambda &= \pm i \\
  \vec{v_1} &= \begin{bmatrix}1 \\ i\end{bmatrix} \quad
    \vec{v_2} = \begin{bmatrix}1 \\ -i\end{bmatrix} \\
  \vec{x} &= c_1\e^{it}\begin{bmatrix}1 \\ i\end{bmatrix}
    +c_2\e^{-it}\begin{bmatrix}1 \\ -i\end{bmatrix} \\
  \e^{it} &= \cos(t)+i\sin(t) \\
  x_1 &= c_2\e^{it}+c_2\e^{-it} \\
  &= c_1(\cos(t)+i\sin(t))+c_2(\cos(t)-i\sin(t)) \\
  &= (c_1+c_2)\cos(t)+(ic_1-ic_2)\sin(t) \\
  x_2 &= c_1i\e^{it}-ic_2\e^{-it} \\
  &= ic_1(\cos(t)+i\sin(t))-ic_2(\cos(t)-i\sin(t)) \\
  &=  (ic_1-ic_2)\cos(t)-(c_1+c_2)\sin(t)
\end{align*}
Since \( c_1 \) and \( c_2 \) are arbitrary constants, we can substitute them
with other arbitrary constants.
\begin{align*}
  d_1 &= c_1+c_2 \\
  d_2 &= ic_1-ic_2 \\
  x_1 &= d_1\cos(t)+d_2\sin(t) \\
  x_2 &= d_2\cos(t)-d_1\sin(t)
\end{align*}
We can use our initial conditions to solve for \( d_1 \) and \( d_2 \).
\begin{align*}
  x_1(0) &= d_1\cos(0)+d_2\sin(0) = -4 \\
  d_1 &= -4 \\
  x_2(0) &= d_2\cos(0)+d_1\sin(0) = 8 \\
  d_2 &= 0 \\
  x_1 &= -4\cos(t)+8\sin(t) \\
  x_2 &= 8\cos(t)+4\sin(t)
\end{align*}

\begin{center}
  You can find all my notes at \url{http://omgimanerd.tech/notes}. If you have
  any questions, comments, or concerns, please contact me at
  alvin@omgimanerd.tech
\end{center}

\end{document}
