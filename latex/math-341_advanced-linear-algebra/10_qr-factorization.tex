\documentclass{math}

\title{Advanced Linear Algebra}
\author{Alvin Lin}
\date{January 2019 - May 2019}

\begin{document}

\maketitle

\section*{QR Factorization: A = QR}
Let \( A \) be an \( m\times n \) matrix with linearly independent columns.
\[ A = \begin{bmatrix}\vec{a_1} & \vec{a_2} & \dots & \vec{a_n}\end{bmatrix} \]
Apply Gram-Schmidt to the column vectors to find an orthonormal basis
\( \hat{q_1},\dots,\hat{q_n} \).
\begin{align*}
  \vec{a_1} &= r_{11}\hat{q_1} \\
  \vec{a_2} &= r_{12}\hat{q_1}+r_{22}\hat{q_2} \\
  & \vdots \\
  \vec{a_n} &= r_{1n}\hat{q_1}+r_{2n}\hat{q_2}+\dots+r_{nn}\hat{q_n} \\
  \begin{bmatrix}\vec{a_1} & \vec{a_2} & \dots & \vec{a_n}\end{bmatrix} &=
    \begin{bmatrix}\hat{q_1} & \hat{q_2} & \dots & \hat{q_3}\end{bmatrix}
    \begin{bmatrix}
      r_{11} & r_{12} & \dots & r_{1n} \\
      0 & r_{22} & \dots & \dots \\
      \vdots & \vdots & \vdots & \vdots \\
      0 & 0 & \dots & r_{nn}
    \end{bmatrix}
\end{align*}

\subsubsection*{Example}
Find the QR factorization of
\[ A = \begin{bmatrix}
  1 & 2 & 2 \\
  -1 & 1 & 2 \\
  -1 & 0 & 1 \\
  1 & 1 & 2
\end{bmatrix} \]
Apply Gram-Schmidt to the column vectors.
\begin{align*}
  Q &= \begin{bmatrix}
    \frac{1}{2} & \frac{3}{2\sqrt{5}} & -\frac{1}{\sqrt{6}} \\
    -\frac{1}{2} & \frac{3}{2\sqrt{5}} & 0 \\
    -\frac{1}{2} & \frac{1}{2\sqrt{5}} & \frac{1}{\sqrt{6}} \\
    \frac{1}{2} & \frac{1}{2\sqrt{5}} & \frac{2}{\sqrt{6}}
  \end{bmatrix} \\
  A &= QR \\
  Q^TA &= Q^TQR \\
  R &= Q^TA \\
  &= \begin{bmatrix}
    2 & 1 & \frac{1}{2} \\
    0 & \sqrt{5} & \frac{3\sqrt{5}}{2} \\
    0 & 0 & \frac{\sqrt{6}}{2}
  \end{bmatrix}
\end{align*}

\subsection*{QR Algorithm for Approximating Eigenvalues}
\begin{align*}
  \text{Let } A &= QR \\
  A_1 &= RQ = Q_1R_1 \\
  A_2 &= R_1Q_1 = Q_2R_2 \\
  A_3 &= R_2Q_2 = Q_3R_3 \\
  &\vdots \\
  A_k &= R_{k-1}Q_{k-1}
\end{align*}
Notice that this iterative process yields similar matrices. Therefore, the
eigenvalues remain the same. As \( k\to\infty \), \( A_k \) converges to an
upper triangular matrix. Since \( A_k \) is similar to \( A \), the eigenvalues
of \( A \) are the values on the diagonal of \( A_k \).

\subsection*{Orthogonal Diagonalization of Symmetric Matrices}
\begin{align*}
  A &= \begin{bmatrix}1 & 2 \\ 2 & -2\end{bmatrix} \\
  D &= P^{-1}AP \\
  D &= \begin{bmatrix}-3 & 0 \\ 0 & 2\end{bmatrix} \\
  \begin{bmatrix}-3 & 0 \\ 0 & 2\end{bmatrix} &=
    \begin{bmatrix}1 & 2 \\ -2 & 1\end{bmatrix}^{-1}
    \begin{bmatrix}1 & 2 \\ 2 & -2\end{bmatrix}
    \begin{bmatrix}1 & 2 \\ -2 & 1\end{bmatrix} \\
  \vec{v_1} &= \begin{bmatrix}1 \\ -2\end{bmatrix} \quad
    \vec{v_2} = \begin{bmatrix}2 \\ 1\end{bmatrix} \quad
    \vec{v_1}\bot\vec{v_2} \\
  \hat{u_1} &=
    \begin{bmatrix}\frac{1}{\sqrt{5}} \\ -\frac{2}{\sqrt{5}}\end{bmatrix} \quad
    \hat{u_2} =
    \begin{bmatrix}\frac{2}{\sqrt{5}} \\ \frac{1}{\sqrt{5}}\end{bmatrix} \\
  Q &= \begin{bmatrix}
    \frac{1}{\sqrt{5}} & \frac{2}{\sqrt{5}} \\
    -\frac{2}{\sqrt{5}} & \frac{1}{\sqrt{5}}
  \end{bmatrix} \\
  D &= Q^{-1}AQ = Q^TAQ
\end{align*}
In general, a square matrix is orthogonally orthogonally diagonalizable if
\( Q^TAQ = D \) where \( Q \) is an orthogonal matrix and \( D \) is a diagonal
matrix.

\subsubsection*{Theorem}
If a matrix \( A \) is orthogonally diagonalizable, then \( A \) is symmetric.
Proof:
\begin{align*}
  Q^TQA &= D \\
  \therefore A &= QDQ^T \\
  A^T &= (QDQ^T)^T = (Q^T)^TD^TQ^T = QDQ^T \\
  &= A^T
\end{align*}
The converse is also true, if a matrix \( A \) is symmetric, it is also
orthogonally diagonalizable.

\subsubsection*{Theorem}
Additionally, if \( A \) is a real symmetric matrix, then the eigenvalues of
\( A \) are real. Proof:
\begin{align*}
  \overline{zw} &= \bar{z}\bar{w} \\
  Av &= \lambda v \\
  \overline{Av} &= \bar{\lambda v} \\
  A\bar{v} &= \bar{A}\bar{v} = \overline{Av} = \overline{\lambda v} =
    \bar{\lambda}\bar{v} \\
  (A\bar{v})^T &= (\bar{\lambda}\bar{v})^T \\
  \bar{v}^TA^T &= \bar{\lambda}\bar{v}^T \\
  \bar{v}^TA &= \bar{\lambda}\bar{v}^T \\
  \lambda\bar{v}^Tv &= \bar{v}^T(\lambda v) = \bar{v}^T(Av) = (\bar{v}^TA)v =
    \bar{\lambda}\bar{v}^Tv \\
  (\lambda-\bar{\lambda})\bar{v}^Tv &= 0 \\
  \bar{v}^Tv &\ne 0 \\
  \lambda &= \bar{\lambda}
\end{align*}

\subsubsection*{Theorem}
If \( A \) is a symmetric matrix, then any two eigenvectors corresponding to
distinct eigenvalues are orthogonal.
\begin{align*}
  A\vec{v_1} &= \lambda_1\vec{v_1} \quad A\vec{v_2} = \lambda_2\vec{v_2} \\
  \lambda_1(\vec{v_1}\cdot\vec{v_2}) &= (\lambda_1\vec{v_1})\cdot\vec{v_2} =
    (A\vec{v_1})\cdot\vec{v_1} = (A\vec{v_1})^T\vec{v_2} \\
  &= \vec{v_1}^TA^T\vec{v_2} = \vec{v_1}^T(A\vec{v_2}) =
    \vec{v_1}^T\lambda_2\vec{v_2} = \lambda_2(\vec{v_1}\cdot\vec{v_2}) \\
  (\lambda_1-\lambda_2)(\vec{v_1}\cdot\vec{v_2}) &= 0 \\
  (\lambda_1-\lambda_2) &\ne 0 \\
  \vec{v_1}\cdot\vec{v_2} &= 0
\end{align*}

\subsection*{The Spectral Theorem}
Let \( A \) be an \( n\times n \) real matrix. Then \( A \) is symmetric if and
only if it is orthogonally diagonalizable \( Q^TAQ = D \). Example: \\
Orthogonally diagonalize \( A \):
\begin{align*}
  A &= \begin{bmatrix}
    2 & 1 & 1 \\
    1 & 2 & 1 \\
    1 & 1 & 2
  \end{bmatrix} \\
  \lambda_1 &= 4 \quad \lambda_2 = \lambda_3 = 1 \\
  E_4 &= span\left\{\begin{bmatrix}1 \\ 1 \\ 1\end{bmatrix}\right\}
    (\vec{x_1}) \\
  E_1 &= span\left\{
    \begin{bmatrix}-1 \\ 0 \\ 1\end{bmatrix},
    \begin{bmatrix}-1 \\ 1 \\ 0\end{bmatrix}
  \right\} (\vec{x_2},\vec{x_3}) \\
  \vec{x_1}\cdot\vec{x_2} &= 0 \quad \vec{x_1}\cdot\vec{x_3} = 0 \\
  \vec{x_2}\cdot\vec{x_3} &= 1 \ne 0 \\
  \text{Let } \vec{v_1} &= \vec{x_1} \\
  \vec{v_2} &= \vec{x_2} \\
  \vec{v_3} &= \text{perp}_{\vec{x_2}}\vec{x_3} =
    x_3-\text{proj}_{\vec{x_2}}\vec{x_3} =
    \vec{x_3}-\left(\frac{\vec{x_3}\cdot\vec{x_2}}
      {\vec{x_2}\cdot\vec{x_2}}\right) =
    \begin{bmatrix}-\frac{1}{2} \\ 1 \\ -\frac{1}{2}\end{bmatrix} \\
  \|\vec{v_3}\| &= \frac{\sqrt{6}}{2} \\
  \hat{q_1} &= \begin{bmatrix}
    \frac{1}{\sqrt{3}} \\ \frac{1}{\sqrt{3}} \\ \frac{1}{\sqrt{3}}
  \end{bmatrix}, \hat{q_2} = \begin{bmatrix}
    -\frac{1}{\sqrt{2}} \\ 0 \\ \frac{1}{\sqrt{2}}
  \end{bmatrix}, \hat{q_3} = \begin{bmatrix}
    -\frac{1}{\sqrt{6}} \\ \frac{2}{\sqrt{6}} \\ -\frac{1}{\sqrt{6}}
  \end{bmatrix} \\
  Q &= \begin{bmatrix}\hat{q_1} & \hat{q_2} & \hat{q_3}\end{bmatrix} \\
  D &= \begin{bmatrix}
    4 & 0 & 0 \\
    0 & 1 & 0 \\
    0 & 0 & 1
  \end{bmatrix}
\end{align*}

\subsection*{Spectral Decomposition of \( A = QDQ^T \)}
\begin{align*}
  A &= \begin{bmatrix}\hat{q_1} & \dots & \hat{q_n}\end{bmatrix}
    \begin{bmatrix}\lambda_1 & \vdots \\ \vdots & \lambda_n\end{bmatrix}
    \begin{bmatrix}\hat{q_1}^T \\ \vdots \\ \hat{q_n}^T\end{bmatrix} \\
  &= \begin{bmatrix}\lambda_1\hat{q_1} & \dots & \lambda_n\hat{q_n}\end{bmatrix}
    \begin{bmatrix}\hat{q_1}^T \\ \vdots \\ \hat{q_n}^T\end{bmatrix} \\
  &= \lambda_1\hat{q_1}\hat{q_1}^T+\dot+\lambda_n\hat{q_n}\hat{q_n}^T
\end{align*}
\( \lambda_1\hat{q_1}\hat{q_1}^T \) is a projection onto the subspace spanned
by \( \hat{q_1} \), and \( \lambda_n\hat{q_n}\hat{q_n}^T \) is a projection
onto the subspace spanned by \( \hat{q_n} \). This form is known as the
projection form of the Spectral Theorem.

\subsubsection*{Example}
Find a \( 2\times2 \) matrix with eigenvalues \( \lambda_1 = 3,\lambda_2 = -2 \)
and eigenvectors
\begin{align*}
  \vec{v_1} &= \begin{bmatrix}3 \\ 4\end{bmatrix} \quad
    \vec{v_2} = \begin{bmatrix}-4 \\ 3\end{bmatrix} \\
  \hat{q_1} &= \begin{bmatrix}\frac{3}{5} \\ \frac{4}{5}\end{bmatrix} \quad
    \hat{q_2} = \begin{bmatrix}-\frac{4}{5} \\ \frac{3}{5}\end{bmatrix} \\
  A &= \lambda\hat{q_1}\hat{q_1}^T+\lambda_2\hat{q_2}\hat{q_2}^T \\
  &= 3\begin{bmatrix}\frac{3}{5} \\ \frac{4}{5}\end{bmatrix}
    \begin{bmatrix}\frac{3}{5} & \frac{4}{5}\end{bmatrix}+
    (-2)\begin{bmatrix}-\frac{4}{5} \\ \frac{3}{5}\end{bmatrix}
    \begin{bmatrix}-\frac{4}{5} & \frac{3}{5}\end{bmatrix} \\
  &= \begin{bmatrix}
    -\frac{1}{5} & \frac{12}{5} \\
    \frac{12}{5} & \frac{6}{5}
  \end{bmatrix}
\end{align*}

\begin{center}
  You can find all my notes at \url{http://omgimanerd.tech/notes}. If you have
  any questions, comments, or concerns, please contact me at
  alvin@omgimanerd.tech
\end{center}

\end{document}
