\documentclass{math}

\title{Advanced Linear Algebra}
\author{Alvin Lin}
\date{January 2019 - May 2019}

\begin{document}

\maketitle

\section*{Changes of Basis}
Suppose we have two different bases:
\begin{align*}
  \vec{u_1} &= \begin{bmatrix}-1 \\ 2\end{bmatrix} \quad
    \vec{u_2} = \begin{bmatrix}2 \\ -1\end{bmatrix} \\
  B &= \{\vec{u_1},\vec{u_2}\} \\
  \vec{v_1} &= \begin{bmatrix}1 \\ 0\end{bmatrix} \quad
    \vec{v_2} = \begin{bmatrix}1 \\ 1\end{bmatrix} \\
  C &= \{\vec{v_1},\vec{v_2}\}
\end{align*}
Suppose we have a vector \( \vec{x} \) represented in one basis:
\[ [\vec{x}]_{B} = \begin{bmatrix}1 \\ 3\end{bmatrix} =
  1\vec{u_1}+3\vec{u_2} = \begin{bmatrix}5 \\ -1\end{bmatrix} \]
The same vector represented with respect to the basis \( C \) is:
\[ [\vec{x}]_{C} = \begin{bmatrix}6 \\ -1\end{bmatrix} =
  6\vec{v_1}-\vec{v_2} = \begin{bmatrix}5 \\ -1\end{bmatrix} \]
In the standard basis, this vector is \( 5\i-\j \). In general, given a vector
\( \vec{x} \) and a set of bases, how do we convert the vector between the
different bases? We just need to know how to represent the basis vectors in
terms of each other in order to convert between them.
\begin{align*}
  \vec{u_1} &= -3\vec{v_1}+2\vec{v_2} = \begin{bmatrix}-3 \\ 2\end{bmatrix}_C \\
  \vec{u_2} &= 3\vec{v_1}-\vec{v_2} = \begin{bmatrix}3 \\ -1\end{bmatrix}_C \\
  \vec{x} &= 1\vec{u_1}+3\vec{u_2} \\
  &= 1(-3\vec{v_1}+2\vec{v_2})+3(3\vec{v_1}-\vec{v_2}) \\
  &= 6\vec{v_1}-\vec{v_2}
\end{align*}
We can compactly represent this as:
\begin{align*}
  \begin{bmatrix}6 \\ -1\end{bmatrix}_C &= \begin{bmatrix}
    -3 & 3 \\
    2 & -1
  \end{bmatrix}\begin{bmatrix}1 \\ 3\end{bmatrix}_B \\
  \vec{x}_C &= \begin{bmatrix}\vec{u_1}_C & \vec{u_2}_C\end{bmatrix}
    \begin{bmatrix}1 \\ 3\end{bmatrix}_B \\
  [\vec{x}]_C &= P_{C\leftarrow B}[\vec{x}]_B
\end{align*}

\subsubsection*{Theorem}
Let \( B = \{\vec{u_1},\dots,\vec{u_n}\} \) and \( C =
\{\vec{v_1},\dots,\vec{v_n}\} \) be the bases for some vector space. Then
\begin{align*}
  P_{C\leftarrow B} &= \begin{bmatrix}
    [\vec{u_1}]_C & \dots & [\vec{u_n}]_C
  \end{bmatrix} \\
  [\vec{x}]_C &= P_{C\leftarrow B}[\vec{x}]_B \\
  [\vec{x}]_B &= (P_{C\leftarrow B})^{-1}[\vec{x}]_C \\
  P_{B\leftarrow C} &= (P_{C\leftarrow B})^{-1}
\end{align*}
The change of basis matrix \( P_{C\leftarrow B} \) is unique and can be used
to convert between the bases.

\subsubsection*{Example}
\begin{align*}
  B &= \{1,x,x^2\} \\
  C &= \{1+x,x+x^2,1+x^2\} \\
  p &= 1+2x-x^2 = \begin{bmatrix}1 \\ 2 \\ -1\end{bmatrix}_B
\end{align*}
Represent \( p \) in the basis \( C \).
\begin{align*}
  [1+x]_B &= \begin{bmatrix}1 \\ 1 \\ 0\end{bmatrix}_B \\
  [x+x^2]_B &= \begin{bmatrix}0 \\ 1 \\ 1\end{bmatrix}_B \\
  [1+x^2]_B &= \begin{bmatrix}1 \\ 0 \\ 1\end{bmatrix}_B \\
  P_{B\leftarrow C} &= \begin{bmatrix}
    1 & 0 & 1 \\
    1 & 1 & 0 \\
    0 & 1 & 1
  \end{bmatrix} \\
  [p]_B &= (P_{B\leftarrow C})[p]_C \\
  [p]_C &= (P_{B\leftarrow C})^{-1}[p]_B \\
  &= \begin{bmatrix}
    \frac{1}{2} & \frac{1}{2} & -\frac{1}{2} \\
    -\frac{1}{2} & \frac{1}{2} & \frac{1}{2} \\
    \frac{1}{2} & -\frac{1}{2} & \frac{1}{2} \\
  \end{bmatrix}\begin{bmatrix}1 \\ 2 \\ -1\end{bmatrix}_B \\
  &= \begin{bmatrix}2 \\ 0 \\ -1\end{bmatrix}_C
\end{align*}

\subsubsection*{Example}
\begin{align*}
  B &= \{E_{11},E_{21},E_{12},E_{22}\} \\
  &= \left\{
    \begin{bmatrix}1 & 0 \\ 0 & 0\end{bmatrix},
    \begin{bmatrix}0 & 0 \\ 1 & 0\end{bmatrix},
    \begin{bmatrix}0 & 1 \\ 0 & 0\end{bmatrix},
    \begin{bmatrix}0 & 0 \\ 0 & 1\end{bmatrix}
  \right\} \\
  C &= \{A,B,D,E\} \\
  &= \left\{
    \begin{bmatrix}1 & 0 \\ 0 & 0\end{bmatrix},
    \begin{bmatrix}1 & 1 \\ 0 & 0\end{bmatrix},
    \begin{bmatrix}1 & 1 \\ 1 & 0\end{bmatrix},
    \begin{bmatrix}1 & 1 \\ 1 & 1\end{bmatrix}
  \right\}
\end{align*}
Find the change of basis matrix.
\begin{align*}
  E_{11} &= 1A \quad
    [E_{11}]_C = \begin{bmatrix}1 \\ 0 \\ 0 \\ 0\end{bmatrix} \\
  E_{21} &= -B+D \quad
    [E_{21}]_C = \begin{bmatrix}0 \\ -1 \\ 1 \\ 0\end{bmatrix} \\
  E_{12} &= -A+B \quad
    [E_{12}]_C = \begin{bmatrix}-1 \\ 1 \\ 0 \\ 0\end{bmatrix} \\
  E_{22} &= E-D \quad
    [E_{22}]_C = \begin{bmatrix}0 \\ 0 \\ -1 \\ 1\end{bmatrix} \\
  P_{C\leftarrow B} &= \begin{bmatrix}
    1 & 0 & -1 & 0 \\
    0 & -1 & 1 & 0 \\
    0 & 1 & 0 & -1 \\
    0 & 0 & 0 & 1
  \end{bmatrix}
\end{align*}
Suppose we have the matrix
\[ \vec{x} = \begin{bmatrix}1 & 2 \\ 3 & 4\end{bmatrix} =
  1E_{11}+3E_{21}+2E_{12}+4E_{22} =
  \begin{bmatrix}1 \\ 3 \\ 2 \\ 4\end{bmatrix}_B \]
Find the coordinates of \( \vec{x} \) in terms of the basis \( C \).
\begin{align*}
  [\vec{x}]_C &= P_{C\leftarrow B}\begin{bmatrix}
    1 \\ 3 \\ 2 \\ 4
  \end{bmatrix}_B \\
  &= \begin{bmatrix}-1 \\ -1 \\ -1 \\ 4\end{bmatrix}_C
\end{align*}

\begin{center}
  You can find all my notes at \url{http://omgimanerd.tech/notes}. If you have
  any questions, comments, or concerns, please contact me at
  alvin@omgimanerd.tech
\end{center}

\end{document}
