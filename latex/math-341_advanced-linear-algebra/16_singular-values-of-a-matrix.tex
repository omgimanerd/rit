\documentclass{math}

\usepackage{enumerate}

\title{Advanced Linear Algebra}
\author{Alvin Lin}
\date{January 2019 - May 2019}

\begin{document}

\maketitle

\section*{Singular Values of a Matrix}
Suppose we have an \( m\times n \) matrix \( A \), and we construct an
\( n\times n \) symmetric matrix \( A^TA \). Let \( \lambda \) be an
eigenvalue of \( A^TA \) with unit eigenvector \( \hat{v} \).
\[ 0 \le \|A\hat{v}\|^2 = (A\hat{v})\cdot(A\hat{v}) = (A\hat{v})^T(A\hat{v}) =
  \hat{v}^TA^TA\hat{v} = \hat{v}^T\lambda\hat{v} = \lambda\|\hat{v}\|^2 =
  \lambda \]
If \( A \) is an \( m\times n \) matrix, the ``singular values'' of \( A \) are
the square roots of the eigenvalues of \( A^TA \) and are denoted
\( \sigma_1 \ge \sigma_2 \ge \dots \ge \sigma_n \).
Let \( \{\hat{v_1},\dots,\hat{v_n}\} \) be orthonormal eigenvectors of
\( A^TA \) such that \( \lambda_1 \ge \lambda_2 \ge \dots \ge \lambda_n \).
We know \( \lambda_i = \|A\hat{v_i}\|^2 \) and \( \sigma_i = \|A\hat{v_i}\| =
\sqrt{\lambda_i} \).

\subsubsection*{Example}
Suppose we have the matrix:
\begin{align*}
  A &= \begin{bmatrix}
    1 & 1 \\
    1 & 0 \\
    0 & 1
  \end{bmatrix} \\
  A^TA &= \begin{bmatrix}
    1 & 1 & 0 \\
    1 & 0 & 1
  \end{bmatrix}\begin{bmatrix}
    1 & 1 \\
    1 & 0 \\
    0 & 1
  \end{bmatrix} = \begin{bmatrix}
    2 & 1 \\
    1 & 2
  \end{bmatrix}
\end{align*}
The eigenvalues of \( A^TA \) are \( \lambda_1 = 3 \) and \( \lambda_2 = 1 \)
and thus the singular values of \( A^TA \) are \( \sigma_1 = \sqrt{3} \) and
\( \sigma_2 = \sqrt{1} = 1 \). The eigenvectors of \( A^TA \) are:
\begin{align*}
  \lambda_1 &= 3 \quad \hat{v_1} = \begin{bmatrix}
    \frac{1}{\sqrt{2}} \\ \frac{1}{\sqrt{2}}
  \end{bmatrix} \\
  \lambda_2 &= 1 \quad \hat{v_2} = \begin{bmatrix}
    -\frac{1}{\sqrt{2}} \\ \frac{1}{\sqrt{2}}
  \end{bmatrix}
\end{align*}
Consider a vector in \( \R^2 \)
\[ \hat{x} = \begin{bmatrix}x_1 \\ x_2\end{bmatrix} \]
such that \( \|\hat{x}\| = 1 \).
\begin{align*}
  \|A\hat{x}\|^2 &= (A\hat{x})\cdot(A\hat{x}) \\
  &= (A\hat{x})^T(A\hat{x}) \\
  &= \hat{x}^TA^TA\hat{x} \\
  &= \begin{bmatrix}x_1 & x_2\end{bmatrix}
    \begin{bmatrix}2 & 1 \\ 1 & 2\end{bmatrix}
    \begin{bmatrix}x_1 \\ x_2\end{bmatrix} \\
  &= 2x_1^2+2x_1x_2+2x_2^2
\end{align*}
We are interested in what the largest or smallest that the quadratic
polynomial can become, which represents the largest or smallest amount that the
matrix \( A \) can expand or shrink the vector.

\subsection*{Facts}
If \( \|\vec{x}\| = 1 \) and \( f(\vec{x}) = \vec{x}^TB\vec{x} \) where \( B \)
is a symmetric matrix:
\begin{enumerate}[1)]
  \item The maximum \( f(\vec{x}) \) is \( \lambda_1 \) and occurs at
    \( \vec{x} = \hat{v_1} \).
  \item The minimum \( f(\vec{x}) \) is \( \lambda_n \) and occurs at
    \( \vec{x} = \hat{v_n} \).
\end{enumerate}
Continuing the previous example, the maximum of \( \|A\hat{x}\|^2 = \lambda_1 =
3 \). The max \( \|A\hat{x}\| = \sqrt{3} = \sigma_1 \), which occurs at
\( \hat{v_1} \), and the min \( \|A\hat{x}\| = \sqrt{1} = \sigma_2 \), which
occurs at \( \hat{v_2} \). If we consider \( A \) as a mapping from one
space to another in \( \R^3 \):
\[ \begin{bmatrix}x \\ y \\ z\end{bmatrix} = \begin{bmatrix}
  1 & 1 \\
  1 & 0 \\
  0 & 1
\end{bmatrix}\begin{bmatrix}x_1 \\ x_2\end{bmatrix} = \begin{bmatrix}
  x_1+x_2 \\
  x_1 \\
  x_2
\end{bmatrix} \]
\( A \) maps all of \( \R^2 \) to the plane \( x-y-z = 0 \). In particular, the
vectors on the unit circle get mapped to an ellipse on this plane. The maximum
amount of stretching that \( A \) can do with respect to the previous example is
\( \sqrt{3} \), occurring at the vector \( \hat{v_1} = \begin{bmatrix}
\frac{1}{\sqrt{2}} \\ \frac{1}{\sqrt{2}}\end{bmatrix} \) on the unit circle.
The minimum amount of stretching occurs at \( \hat{v_2} = \begin{bmatrix}
-\frac{1}{\sqrt{2}} \\ \frac{1}{\sqrt{2}}\end{bmatrix} \).

\begin{center}
  You can find all my notes at \url{http://omgimanerd.tech/notes}. If you have
  any questions, comments, or concerns, please contact me at
  alvin@omgimanerd.tech
\end{center}

\end{document}
