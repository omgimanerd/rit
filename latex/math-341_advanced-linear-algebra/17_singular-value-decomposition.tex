\documentclass{math}

\usepackage{enumerate}

\title{Advanced Linear Algebra}
\author{Alvin Lin}
\date{January 2019 - May 2019}

\begin{document}

\maketitle

\section*{Singular Value Decomposition}
Recall that the diagonalizaton of \( A \) can be written as:
\[ A = \begin{bmatrix}\vec{v_1} & \dots & \vec{v_n}\end{bmatrix}
  \begin{bmatrix}
    \lambda_1 & \vdots \\
    \dots & \lambda_n
  \end{bmatrix}\begin{bmatrix}\vec{v_1} & \dots & \vec{v_n}\end{bmatrix}^{-1} \]
If \( A \) is symmetric, we can orthogonally diagonalize it as \( A = PDP^T \).
\[ A = \begin{bmatrix}\hat{v_1} & \dots & \hat{v_n}\end{bmatrix}
  \begin{bmatrix}
    \lambda_1 & \vdots \\
    \dots & \lambda_n
  \end{bmatrix}\begin{bmatrix}
    \dots & \vec{v_1}^T & \dots \\
    \dots & \vec{v_n}^T & \dots
  \end{bmatrix}^{-1} \]
Recall from the definition of singular values of A:
\[ (A^TA)\hat{v_i} = \lambda_i\hat{v_i} \quad
  (\lambda_1\ge\lambda_2\ge\dots\ge\lambda_n) \]
We define the singular values \( \sigma_i = \sqrt{\lambda_i} =
\|A\hat{v_i}\| \).
\begin{align*}
  \|A\hat{v_i}\|^2 &= (A\hat{v_i})^T(A\hat{v_i}) \\
  &= \hat{v_i}^TA^TA\hat{v_i} \\
  &= \lambda(\hat{v_i}^T\hat{v_i}) = \lambda
\end{align*}

\subsection*{Singular Value Decomposition}
\begin{align*}
  A &= U\Sigma V^T \\
  [A] &= \begin{bmatrix}\hat{u_1} & \dots & \hat{u_n}\end{bmatrix}
    \begin{bmatrix}
      \sigma_1 & \vdots \\
      \dots & \sigma_n
    \end{bmatrix}\begin{bmatrix}
      \dots & \hat{v_1}^T & \dots \\
      \dots & \hat{v_n}^T & \dots
    \end{bmatrix} \\
\end{align*}
\begin{enumerate}
  \item Find the eigenvalues \( \lambda_i \) and eigenvectors \( \hat{v_i} \)
    of \( A^TA \).
  \item Let \( \sigma_i = \sqrt{\lambda_i} \) and normalize \( \hat{v_i} \).
  \item Let \( \hat{u_i} = \frac{1}{\sigma_i}A\hat{v_i} \)
\end{enumerate}
Why does this work?
\begin{align*}
  AV &= U\Sigma \\
  A\begin{bmatrix}\hat{v_1} & \dots & \hat{v_n}\end{bmatrix} &\stackrel{?}{=}
    \begin{bmatrix}\hat{u_1} & \dots & \hat{u_n}\end{bmatrix}\Sigma \\
  \begin{bmatrix}A\hat{v_1} & \dots & A\hat{v_n}\end{bmatrix} &\stackrel{?}{=}
    \begin{bmatrix}\sigma_1\hat{u_1} & \dots & \sigma_n\hat{u_n}\end{bmatrix} \\
  \begin{bmatrix}
    \sigma_1\hat{u_1} & \dots & \sigma_n\hat{u_n}
  \end{bmatrix} &= \begin{bmatrix}
    \sigma_1\hat{u_1} & \dots & \sigma_n\hat{u_n}
  \end{bmatrix}
\end{align*}
Observations:
\begin{itemize}
  \item \( V \) is orthogonal because the columns are eigenvectors of the
    symmetric matrix \( A^TA \).
  \item \( U \) is orthogonal because the columns \( \hat{u_i} \) and
    \( \hat{u_j} \) are orthogonal.
  \begin{align*}
    \hat{u_i}\cdot\hat{u_j} &= \left(\frac{1}{\sigma_i}A\hat{v_i}\right)\cdot
      \left(\frac{1}{\sigma_j}A\hat{v_j}\right) \\
    &= \frac{1}{\sigma_i\sigma_j}(A\hat{v_j})^T(A\hat{v_i}) \\
    &= \frac{1}{\sigma_i\sigma_j}\hat{v_j}^T(A^TA\hat{v_i}) \\
    &= \frac{\lambda_i}{\sigma_i\sigma_j}\hat{v_j}^T\hat{v_i} \\
    &= 0
  \end{align*}
\end{itemize}
In general, \( A = U\Sigma V^T \) where
\begin{itemize}
  \item \( A \) is an \( m\times n \) matrix
  \item \( U \) is an \( m\times m \) orthogonal matrix
  \item \( \Sigma \) is an \( m\times n \) matrix
  \item \( V^T \) is an \( n\times n \) orthogonal matrix
\end{itemize}

\subsubsection*{Example}
Find the singular value decomposition of \( A = \begin{bmatrix}1 & 1 & 0 \\
0 & 0 & 1\end{bmatrix} \).
\begin{align*}
  A &= U\Sigma V^T \\
  A^TA &= \begin{bmatrix}
    1 & 1 & 0 \\
    1 & 1 & 0 \\
    0 & 0 & 1
  \end{bmatrix} \\
  \lambda_1 &= 2 \quad \sigma_1 = \sqrt{2} \quad \hat{v_1} = \begin{bmatrix}
    \frac{1}{\sqrt{2}} \\ \frac{1}{\sqrt{2}} \\ 0
  \end{bmatrix} \\
  \lambda_2 &= 1 \quad \sigma_2 = \sqrt{1} \quad \hat{v_2} = \begin{bmatrix}
    0 \\ 0 \\ 1
  \end{bmatrix} \\
  \lambda_3 &= 0 \quad \sigma_3 = 0 \quad \hat{v_3} = \begin{bmatrix}
    -\frac{1}{\sqrt{2}} \\ \frac{1}{\sqrt{2}} \\ 0
  \end{bmatrix} \\
  V &= \begin{bmatrix}
    \frac{1}{\sqrt{2}} & 0 & -\frac{1}{\sqrt{2}} \\
    \frac{1}{\sqrt{2}} & 0 & \frac{1}{\sqrt{2}} \\
    0 & 1 & 1
  \end{bmatrix} \\
  \Sigma &= \begin{bmatrix}
    \sqrt{2} & 0 & 0 \\
    0 & 1 & 0
  \end{bmatrix} \\
  \hat{u_1} &= \frac{1}{\sigma_1}A\hat{v_1} = \frac{1}{\sqrt{2}}\begin{bmatrix}
    1 & 1 & 0 \\
    0 & 0 & 1
  \end{bmatrix}\begin{bmatrix}
    \frac{1}{\sqrt{2}} \\ \frac{1}{\sqrt{2}} \\ 0
  \end{bmatrix} = \begin{bmatrix}1 \\ 0\end{bmatrix} \\
  \hat{u_2} &= \frac{1}{\sigma_2}A\hat{v_2} = \frac{1}{1}\begin{bmatrix}
    1 & 1 & 0 \\
    0 & 0 & 1
  \end{bmatrix}\begin{bmatrix}0 \\ 0 \\ 1\end{bmatrix} = \begin{bmatrix}
    0 \\ 1
  \end{bmatrix} \\
  U &= \begin{bmatrix}
    1 & 0 \\
    0 & 1
  \end{bmatrix} \\
  A &= \begin{bmatrix}
    1 & 1 & 0 \\
    0 & 0 & 1
  \end{bmatrix} = \begin{bmatrix}
    1 & 0 \\
    0 & 1
  \end{bmatrix}\begin{bmatrix}
    \sqrt{2} & 0 & 0 \\
    0 & 1 & 1
  \end{bmatrix}\begin{bmatrix}
    \frac{1}{\sqrt{2}} & \frac{1}{\sqrt{2}} & 0 \\
    0 & 0 & 1 \\
    -\frac{1}{\sqrt{2}} & \frac{1}{\sqrt{2}} & 0
  \end{bmatrix}
\end{align*}

\begin{center}
  You can find all my notes at \url{http://omgimanerd.tech/notes}. If you have
  any questions, comments, or concerns, please contact me at
  alvin@omgimanerd.tech
\end{center}

\end{document}
