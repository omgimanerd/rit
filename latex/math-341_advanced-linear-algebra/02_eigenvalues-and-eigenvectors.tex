\documentclass{math}

\usepackage{tikz}

\title{Advanced Linear Algebra}
\author{Alvin Lin}
\date{January 2019 - May 2019}

\begin{document}

\maketitle

\section*{Eigenvalues and Eigenvectors}
Let \( A \) be an \( n\times n \) matrix. \( \lambda \) is an eigenvalue of
\( A \) if and only if there is a nonzero vector \( \vec{x} \) such that
\( A\vec{x} = \lambda\vec{x} \). Such a vector \( \vec{x} \) is called an
eigenvector.

\subsubsection*{Example}
\begin{align*}
  A &= \begin{bmatrix}
    3 & 1 \\
    1 & 3
  \end{bmatrix} \quad \vec{v} = \begin{bmatrix}1 \\ 1\end{bmatrix} \\
  A\vec{x} &= \begin{bmatrix}
    3 & 1 \\
    1 & 3
  \end{bmatrix}\begin{bmatrix}
    1 \\ 1
  \end{bmatrix} \\
  &= \begin{bmatrix}
    4 \\ 4
  \end{bmatrix} \\
  \lambda\vec{x} &= 4\begin{bmatrix}1 \\ 1\end{bmatrix}
\end{align*}
Thus, \( \lambda = 4 \) is an eigenvalue.

\subsubsection*{Example}
Show that \( \lambda = 5 \) is an eigenvalue of \( A = \begin{bmatrix}1 & 2 \\
4 & 3\end{bmatrix} \).
\begin{align*}
  A\vec{x} = 5\vec{x} \\
  A\vec{x}-5\vec{x} &= \vec{0} \\
  A\vec{x}-5I\vec{x} &= \vec{0} \\
  (A-5I)\vec{x} &= \vec{0} \\
  \left(\begin{bmatrix}
    1 & 2 \\
    4 & 3
  \end{bmatrix}-\begin{bmatrix}
    5 & 0 \\
    0 & 5
  \end{bmatrix}\right)\vec{x} &= 0 \\
  \begin{bmatrix}
    -4 & 2 \\
    4 & -2
  \end{bmatrix}\begin{bmatrix}
    x_1 \\ x_2
  \end{bmatrix} &= \begin{bmatrix}
    0 \\ 0
  \end{bmatrix}
\end{align*}
If we solve this using an augmented matrix and Gauss-Jordan elimination, we
get the following:
\begin{align*}
  \begin{bmatrix}
    -4 & 2 & 0 \\
    0 & 0 & 0
  \end{bmatrix} &\to \begin{bmatrix}
    1 & -\frac{1}{2} & 0 \\
    0 & 0 & 0
  \end{bmatrix} \\
  x_1-\frac{1}{2}x_2 &= 0 \\
  x_2 &= s \\
  \begin{bmatrix}
    x_1 \\ x_2
  \end{bmatrix} &= \begin{bmatrix}
    \frac{1}{2}s \\ s
  \end{bmatrix} = s\begin{bmatrix}\frac{1}{2} \\ 1\end{bmatrix}
\end{align*}
For \( \lambda = 5 \), \( \vec{x} = \begin{bmatrix}\frac{1}{2} \\
1\end{bmatrix} \), but an eigenvector can be found for any \( s \).

\subsubsection*{Example}
To find eigenvalues, \( A\vec{x} = \lambda\vec{x} \) needs to have nontrivial
solutions.
\begin{align*}
  A\vec{x}-\lambda I\vec{x} &= \vec{0} \\
  (A-\lambda I)\vec{x} &= \vec{0}
\end{align*}
In order for this to be true, the determinant of \( A-\lambda I \) must be 0.
For example:
\begin{align*}
  A &= \begin{bmatrix}
    3 & 1 \\
    1 & 3
  \end{bmatrix} \\
  det(A-\lambda I) &= 0 \\
  det\left(\begin{bmatrix}
    3 & 1 \\
    1 & 3
  \end{bmatrix}-\begin{bmatrix}
    \lambda & 0 \\
    0 & \lambda
  \end{bmatrix}\right) &= 0 \\
  det\left(\begin{bmatrix}
    3-\lambda & 1 \\
    1 & 3-\lambda
  \end{bmatrix}\right) &= 0 \\
  (3-\lambda)(3-\lambda)-1 &= 0 \\
  9-6\lambda+\lambda^2-1 &= 0 \\
  \lambda^2-6\lambda+8 &= 0 \\
  (\lambda-4)(\lambda-2) &= 0 \\
  \lambda_1 &= 4 \quad \lambda_2 = 2
\end{align*}
To find eigenvectors for \( \lambda_1 = 4 \):
\begin{align*}
  A\vec{x} &= 4\vec{x} \\
  (A-4I)\vec{x} &= 0 \\
  \begin{bmatrix}
    -1 & 1 & 0 \\
    1 & -1 & 0
  \end{bmatrix} &\to \begin{bmatrix}
    -1 & 1 & 0 \\
    0 & 0 & 0
  \end{bmatrix} \\
  x_1 &= x_2 = s \\
  \vec{x} &= \begin{bmatrix}s \\ s\end{bmatrix} =
    s\begin{bmatrix}1 \\ 1\end{bmatrix}
\end{align*}
\( \lambda_1 = 4 \) has an infinite number of eigenvectors along \( \vec{x} =
\begin{bmatrix}1 \\ 1\end{bmatrix} \). To find eigenvectors for \( \lambda_2 =
2 \):
\begin{align*}
  A\vec{x} &= 2\vec{x} \\
  (A-2I)\vec{x} &= 0 \\
  \begin{bmatrix}
    1 & 1 & 0 \\
    1 & 1 & 0
  \end{bmatrix} &\to \begin{bmatrix}
    1 & 1 & 0 \\
    0 & 0 & 0
  \end{bmatrix} \\
  x_1 &= -x_2 = -s \\
  \vec{x} &= \begin{bmatrix}-s \\ s\end{bmatrix} =
    s\begin{bmatrix}-1 \\ 1\end{bmatrix}
\end{align*}
\( \lambda_2 = 2 \) has an infinite number of eigenvectors along \( \vec{x} =
\begin{bmatrix}-1 \\ 1\end{bmatrix} \).

\begin{center}
  You can find all my notes at \url{http://omgimanerd.tech/notes}. If you have
  any questions, comments, or concerns, please contact me at
  alvin@omgimanerd.tech
\end{center}

\end{document}
