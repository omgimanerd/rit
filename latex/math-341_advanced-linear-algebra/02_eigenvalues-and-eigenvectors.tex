\documentclass{math}

\usepackage{tikz}

\title{Advanced Linear Algebra}
\author{Alvin Lin}
\date{January 2019 - May 2019}

\begin{document}

\maketitle

\section*{Eigenvalues and Eigenvectors}
Let \( A \) be an \( n\times n \) matrix. \( \lambda \) is an eigenvalue of
\( A \) if and only if there is a nonzero vector \( \vec{x} \) such that
\( A\vec{x} = \lambda\vec{x} \). Such a vector \( \vec{x} \) is called an
eigenvector.

\subsubsection*{Example}
\begin{align*}
  A &= \begin{bmatrix}
    3 & 1 \\
    1 & 3
  \end{bmatrix} \quad \vec{v} = \begin{bmatrix}1 \\ 1\end{bmatrix} \\
  A\vec{x} &= \begin{bmatrix}
    3 & 1 \\
    1 & 3
  \end{bmatrix}\begin{bmatrix}
    1 \\ 1
  \end{bmatrix} \\
  &= \begin{bmatrix}
    4 \\ 4
  \end{bmatrix} \\
  \lambda\vec{x} &= 4\begin{bmatrix}1 \\ 1\end{bmatrix}
\end{align*}
Thus, \( \lambda = 4 \) is an eigenvalue.

\subsubsection*{Example}
Show that \( \lambda = 5 \) is an eigenvalue of \( A = \begin{bmatrix}1 & 2 \\
4 & 3\end{bmatrix} \).
\begin{align*}
  A\vec{x} = 5\vec{x} \\
  A\vec{x}-5\vec{x} &= \vec{0} \\
  A\vec{x}-5I\vec{x} &= \vec{0} \\
  (A-5I)\vec{x} &= \vec{0} \\
  \left(\begin{bmatrix}
    1 & 2 \\
    4 & 3
  \end{bmatrix}-\begin{bmatrix}
    5 & 0 \\
    0 & 5
  \end{bmatrix}\right)\vec{x} &= 0 \\
  \begin{bmatrix}
    -4 & 2 \\
    4 & -2
  \end{bmatrix}\begin{bmatrix}
    x_1 \\ x_2
  \end{bmatrix} &= \begin{bmatrix}
    0 \\ 0
  \end{bmatrix}
\end{align*}
If we solve this using an augmented matrix and Gauss-Jordan elimination, we
get the following:
\begin{align*}
  \begin{bmatrix}
    -4 & 2 & 0 \\
    0 & 0 & 0
  \end{bmatrix} &\to \begin{bmatrix}
    1 & -\frac{1}{2} & 0 \\
    0 & 0 & 0
  \end{bmatrix} \\
  x_1-\frac{1}{2}x_2 &= 0 \\
  x_2 &= s \\
  \begin{bmatrix}
    x_1 \\ x_2
  \end{bmatrix} &= \begin{bmatrix}
    \frac{1}{2}s \\ s
  \end{bmatrix} = s\begin{bmatrix}\frac{1}{2} \\ 1\end{bmatrix}
\end{align*}
For \( \lambda = 5 \), \( \vec{x} = \begin{bmatrix}\frac{1}{2} \\
1\end{bmatrix} \), but an eigenvector can be found for any \( s \).

\subsubsection*{Example}
To find eigenvalues, \( A\vec{x} = \lambda\vec{x} \) needs to have nontrivial
solutions.
\begin{align*}
  A\vec{x}-\lambda I\vec{x} &= \vec{0} \\
  (A-\lambda I)\vec{x} &= \vec{0}
\end{align*}
In order for this to be true, the determinant of \( A-\lambda I \) must be 0.
For example:
\begin{align*}
  A &= \begin{bmatrix}
    3 & 1 \\
    1 & 3
  \end{bmatrix} \\
  det(A-\lambda I) &= 0 \\
  det\left(\begin{bmatrix}
    3 & 1 \\
    1 & 3
  \end{bmatrix}-\begin{bmatrix}
    \lambda & 0 \\
    0 & \lambda
  \end{bmatrix}\right) &= 0 \\
  det\left(\begin{bmatrix}
    3-\lambda & 1 \\
    1 & 3-\lambda
  \end{bmatrix}\right) &= 0 \\
  (3-\lambda)(3-\lambda)-1 &= 0 \\
  9-6\lambda+\lambda^2-1 &= 0 \\
  \lambda^2-6\lambda+8 &= 0 \\
  (\lambda-4)(\lambda-2) &= 0 \\
  \lambda_1 &= 4 \quad \lambda_2 = 2
\end{align*}
To find eigenvectors for \( \lambda_1 = 4 \):
\begin{align*}
  A\vec{x} &= 4\vec{x} \\
  (A-4I)\vec{x} &= 0 \\
  \begin{bmatrix}
    -1 & 1 & 0 \\
    1 & -1 & 0
  \end{bmatrix} &\to \begin{bmatrix}
    -1 & 1 & 0 \\
    0 & 0 & 0
  \end{bmatrix} \\
  x_1 &= x_2 = s \\
  \vec{x} &= \begin{bmatrix}s \\ s\end{bmatrix} =
    s\begin{bmatrix}1 \\ 1\end{bmatrix}
\end{align*}
\( \lambda_1 = 4 \) has an infinite number of eigenvectors along \( \vec{x} =
\begin{bmatrix}1 \\ 1\end{bmatrix} \). To find eigenvectors for \( \lambda_2 =
2 \):
\begin{align*}
  A\vec{x} &= 2\vec{x} \\
  (A-2I)\vec{x} &= 0 \\
  \begin{bmatrix}
    1 & 1 & 0 \\
    1 & 1 & 0
  \end{bmatrix} &\to \begin{bmatrix}
    1 & 1 & 0 \\
    0 & 0 & 0
  \end{bmatrix} \\
  x_1 &= -x_2 = -s \\
  \vec{x} &= \begin{bmatrix}-s \\ s\end{bmatrix} =
    s\begin{bmatrix}-1 \\ 1\end{bmatrix}
\end{align*}
\( \lambda_2 = 2 \) has an infinite number of eigenvectors along \( \vec{x} =
\begin{bmatrix}-1 \\ 1\end{bmatrix} \).

\subsubsection*{Example}
Find the eigenvalues and eigenvectors of the matrix.
\begin{align*}
  A &= \begin{bmatrix}
    0 & 1 & 0 \\
    0 & 0 & 1 \\
    2 & -5 & 4
  \end{bmatrix} \\
  \det(A-\lambda I) &= 0 \\
  \det\left(\begin{bmatrix}
    -\lambda & 1 & 0 \\
    0 & -\lambda & 1 \\
    2 & -5 & 4-\lambda
  \end{bmatrix}\right) &= 0 \\
  -\lambda\begin{vmatrix}
    -\lambda & 1 \\
    -5 & 4-\lambda
  \end{vmatrix}-1\begin{vmatrix}
    0 & 1 & \\
    2 & 4-\lambda
  \end{vmatrix}+0 &= 0 \\
  -\lambda\left[(-\lambda)(4-\lambda)-(-5)\right]-\left[0-2\right] &= 0 \\
  \lambda^2(4-\lambda)-5\lambda+2 &= 0 \\
  -\lambda^3+4\lambda^2-5\lambda+2 &= 0 \\
  \lambda^3-4\lambda^2+5\lambda-2 &= 0 \\
  (\lambda-1)^2(\lambda-2) &= 0 \\
  \lambda_1 &= 1 \quad \lambda_2 = 2 \\
\end{align*}
Each eigenvalue will have a corresponding set of eigenvectors. For
\( \lambda = 1 \):
\begin{align*}
  (A-1I)\vec{x} &= \vec{0} \\
  \begin{bmatrix}
    -1 & 1 & 0 & 0 \\
    0 & -1 & 1 & 0 \\
    2 & -5 & 3 & 0
  \end{bmatrix} &\to \begin{bmatrix}
    1 & 0 & -1 & 0 \\
    0 & 1 & -1 & 0 \\
    0 & 0 & 0 & 0
  \end{bmatrix} \\
  x_1-x_3 &= 0 \\
  x_2-x_3 &= 0 \\
  x_1 &= x_3 = x_2 = s \\
  \vec{x} &= \begin{bmatrix}
    x_1 \\ x_2 \\ x_3
  \end{bmatrix} = \begin{bmatrix}
    s \\ s \\ s
  \end{bmatrix} = s\begin{bmatrix}
    1 \\ 1 \\ 1
  \end{bmatrix}
\end{align*}
For \( \lambda = 2 \):
\begin{align*}
  (A-2I)\vec{x} &= \vec{0} \\
  \begin{bmatrix}
    -2 & 1 & 0 & 0 \\
    0 & -2 & 1 & 0 \\
    2 & -5 & 2 & 0
  \end{bmatrix} &\to \begin{bmatrix}
    1 & 0 & -\frac{1}{4} & 0 \\
    0 & 1 & -\frac{1}{2} & 0 \\
    0 & 0 & 0 & 0
  \end{bmatrix} \\
  x_1-\frac{1}{4}x_3 &= 0 \\
  x_2-\frac{1}{2}x_3 &= 0 \\
  x_3 &= s \\
  \vec{x} &= \begin{bmatrix}
    x_1 \\ x_2 \\ x_3
  \end{bmatrix} = \begin{bmatrix}
    \frac{1}{4}s \\ \frac{1}{2}s \\ s
  \end{bmatrix} = s\begin{bmatrix}
    \frac{1}{4} \\ \frac{1}{2} \\ s
  \end{bmatrix}s
\end{align*}

\subsubsection*{Example}
Find the eigenvalues and eigenvectors of the matrix.
\begin{align*}
  A &= \begin{bmatrix}
    -1 & 0 & 1 \\
    3 & 0 & -3 \\
    1 & 0 & -1
  \end{bmatrix} \\
  \det(A-\lambda I) &= 0 \\
  \begin{vmatrix}
    -1-\lambda & 0 & 1 \\
    3 & -\lambda & -3 \\
    1 & 0 & -1-\lambda
  \end{vmatrix} &= 0 \\
  \lambda^3+2\lambda^2 &= 0 \\
  \lambda^2(\lambda+2) &= 0 \\
  \lambda_1 &= \lambda_2 = 0 \\
  \lambda_3 &= -2
\end{align*}
For \( \lambda = 0 \):
\begin{align*}
  (A-0I)\vec{x} &= \vec{0} \\
  \begin{bmatrix}
    -1 & 0 & 1 & 0 \\
    3 & 0 & -3 & 0 \\
    1 & 0 & -1 & 0
  \end{bmatrix} &\to \begin{bmatrix}
    1 & 0 & -1 & 0 \\
    0 & 0 & 0 & 0 \\
    0 & 0 & 0 & 0
  \end{bmatrix} \\
  x_1 &= x_3 \\
  \vec{x} &= \begin{bmatrix}
    x_1 \\ x_2 \\ x_3
  \end{bmatrix} = \begin{bmatrix}
    t \\ s \\ t
  \end{bmatrix} = s\begin{bmatrix}
    0 \\ 1 \\ 0
  \end{bmatrix}+t\begin{bmatrix}
    1 \\ 0 \\ 1
  \end{bmatrix}
\end{align*}
This is a two dimensional eigenspace (the eigenvalue has a geometric
multiplicity of 2) spanned by \( \begin{bmatrix}0 \\ 1 \\ 0\end{bmatrix},
\begin{bmatrix}1 \\ 0 \\ 1\end{bmatrix} \). Since the eigenvalue
\( \lambda = 0 \) also appears twice, the eigenvalue has a algebraic
multiplcity of 2. For \( \lambda = -2 \):
\begin{align*}
  (A+2I)\vec{x} &= \vec{0} \\
  \begin{bmatrix}
    1 & 0 & 1 & 0 \\
    3 & 2 & -3 & 0 \\
    1 & 0 & 1 & 0
  \end{bmatrix} &\to \begin{bmatrix}
    1 & 0 & 1 & 0 \\
    0 & 1 & -3 & 0 \\
    0 & 0 & 0 & 0
  \end{bmatrix} \\
  x_1+x_3 &= 0 \\
  x_2-3x_3 &= 0 \\
  \vec{x} &= \begin{bmatrix}
    x_1 \\ x_2 \\ x_3
  \end{bmatrix} = \begin{bmatrix}
    -s \\ 3s \\ s
  \end{bmatrix} = s\begin{bmatrix}
    -1 \\ 3 \\ 1
  \end{bmatrix}
\end{align*}
In summary, the algebraic multiplicity of an eigenvalue is the number of times
it appears as a solution to the characteristic polynomial, and the geometric
multiplicity of the eigenvalue is the dimensionality of its corresponding
eigenspace.


\subsection*{Cayley-Hamilton Theorem}
If \( \lambda^n+c_{n-1}\lambda^{n-1}+\dots+c_1\lambda+c_0 = 0 \) is the
characteristic equation of \( A \), then \( A^n+c_{n-1}A^{n-1}+\dots+
c_1A+c_0I = 0 \). Essentially, the matrix satisfies its own polynomial. This can
be used to prove and verify the following facts.
\begin{itemize}
  \item \( \lambda_1+\lambda_2+\dots+\lambda_n = trace(A) \) \\
    This quantity is also the sum of the diagonal elements.
  \item \( \lambda_1\lambda_2\lambda_3\dots\lambda_n = \det(A) \)
  \item \( \lambda^n \) is an eigenvalue of \( A^n \).
  \item \( \frac{1}{\lambda} \) is an eigenvalue of \( A^{-1} \).
\end{itemize}

\subsection*{Theorem}
Suppose an \( n\times n \) matrix \( A \) has eigenvectors \( \vec{v_1},\dots,
\vec{v_m} \) with corresponding eigenvalues \( \lambda_1,\dots,\lambda_m \).
If \( \vec{x} = c_1\vec{v_1}+\dots+c_m\vec{v_m} \) then \( A^k\vec{x} = A^k
(c_1\vec{v_1}+\dots+c_m\vec{v_m}) = c_1\lambda_1^k\vec{v_1}+
\dots+c_m\lambda_m^k\vec{v_m} \).

\begin{center}
  You can find all my notes at \url{http://omgimanerd.tech/notes}. If you have
  any questions, comments, or concerns, please contact me at
  alvin@omgimanerd.tech
\end{center}

\end{document}
