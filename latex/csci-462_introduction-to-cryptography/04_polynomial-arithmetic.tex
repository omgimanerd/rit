\documentclass{math}

\usepackage{graphicx}

\title{Introduction to Cryptography}
\author{Alvin Lin}
\date{January 2018 - May 2018}

\begin{document}

\maketitle

\section*{Polynomial Arithmetic}

\subsection*{Finite Fields of the Form \( GF(p^m) \)}
\textbf{Galois' Theorem:} An order-\textit{n} finite field exists if and only if
\( n = p^m \) for some prime \( p \) and some positive integer \( m \).
\begin{itemize}
  \item \( p \) is called the characteristic of this finite field.
  \item The order of a finite field is its number of elements.
  \item We use \( GF(p^m) \) or \( \mathbb{F}_{p^m} \) to represent the finite
  field of order \( p^m \).
  \item An order-\textit{n} finite field is unique (up to isomorphism).
  \item Addition and multiplication module a prime number \( p \) form a finite
  field \( \Z_p = GF(p) \).
  \item If \( m = 1 \), then \( \Z_p = GF(p) \)
  \item One way to construct a finite field with \( m > 1 \) is using the
  polynomial basis. The field is constructed as a set of \( p^m \) polynomials
  along with two polynomial operations.
\end{itemize}

\subsection*{Polynomial Arithmetic}
A polynomial \( f(x) \) is a mathematical expression of the form:
\[ a_nx^n+a_{n-1}x^{n-1}+\dots+a_0 \]
The highest exponent of \( x \) is the degree of the polynomial. \( a_n,
a_{n-1}, \dots, a_0 \) are called coefficients. We can add, subtract, multiply,
and divide polynomials. AES is a byte oriented cipher where 1 byte can be
represented as a 7th degree polynomial.
\par All arithmetic can be done in the Galois field \( GF(2^8) \). If a
polynomial is divisible only by itself and constants, then we call this
polynomial an irreducible polynomial. \( gcd(a(x), b(x)) \) is the polynomial
of maximum degree that divides both \( a(x) \) and \( b(x) \). Similar to
integers, you can do modular arithmetic with polynomials over a field. Now the
operands and modulus are polynomials.

\subsubsection*{Polynomial Arithmetic Modulo (\( x^3+x+1 \))}
Over addition: \\[0.5cm]
\scalebox{0.7}{
  \begin{tabular}{|c||c|c|c|c|c|c|c|c|}
    \hline
    + & 0 & 1 & \( x \) & \( x+1 \) & \( x^2 \) & \( x^2+1 \) & \( x^2+x \) &
    \( x^2+x+1 \) \\ \hline\hline
    0 & 0 & 1 & \( x \) & \( x+1 \) & \( x^2 \) & \( x^2+1 \) & \( x^2+x \) &
    \( x^2+x+1 \) \\ \hline
    1 & 1 & 0 & \( x + 1 \) & \( x \) & \( x^2+1 \) & \( x^2 \) &
    \( x^2+x+1 \) & \( x^2+x \) \\ \hline
    \( x \) & \( x \) & \( x+1 \) & 0 & 1 & \( x^2+x \) & \( x^2+x+1 \) &
    \( x^2 \) & \( x^2+1 \) \\ \hline
    \( x+1 \) & \( x+1 \) & \( x \) & 1 & 0 & \( x^2+x+1 \) & \( x^2+x \) &
    \( x^2+1 \) & \( x^2 \) \\ \hline
    \( x^2 \) & \( x^2 \) & \( x^2+1 \) & \( x^2+x \) & \( x^2+x+1 \) &
    0 & 1 & \( x \) & \( x+1 \) \\ \hline
    \( x^2+1 \) & \( x^2+1 \) & \( x^2 \) & \( x^2+x+1 \) & \( x^2+x \) &
    1 & 0 & \( x+1 \) & \( x \) \\ \hline
    \( x^2+x \) & \( x^2+x \) & \( x^2+x+1 \) & \( x^2 \) & \( x^2+1 \) &
    \( x \) & \( x+1 \) & 0 & 1 \\ \hline
    \( x^2+x+1 \) & \( x^2+x+1 \) & \( x^2+x \) & \( x^2+1 \) & \( x^2 \) &
    \( x+1 \) & \( x \) & 1 & 0 \\ \hline
  \end{tabular}
} \\[0.5cm]
Over multiplication: \\[0.5cm]
\scalebox{0.7}{
  \begin{tabular}{|c||>{\centering\arraybackslash}p{2cm}|c|c|c|c|c|c|c|}
    \hline
    \( \times \) & 0 & 1 & \( x \) & \( x+1 \) & \( x^2 \) & \( x^2+1 \) &
      \( x^2+x \) & \( x^2+x+1 \) \\ \hline\hline
    0 & 0 & 0 & 0 & 0 & 0 & 0 & 0 & 0 \\ \hline
    1 & 0 & 1 & \( x \) & \( x+1 \) & \( x^2 \) & \( x^2+1 \) & \( x^2+x \) &
      \( x^2+x+1 \) \\ \hline
    \( x \) & 0 & \( x \) & \( x^2 \) & \( x^2+x \) & \( x+1 \) & 1 &
      \( x^2+x+1 \) & \( x^2+1 \) \\ \hline
    \( x+1 \) & 0 & \( x+1 \) & \( x^2+x \) & \( x^2+1 \) & \( x^2+x+1 \) &
      \( x^2 \) & 1 & \( x \) \\ \hline
    \( x^2 \) & 0 & \( x^2 \) & \( x+1 \) & \( x^2+x+1 \) & \( x^2+x \) &
      \( x \) & \( x^2+1 \) & 1 \\ \hline
    \( x^2+1 \) & 0 & \( x^2+1 \) & 1 & \( x^2 \) & \( x \) & \( x^2+x+1 \) &
      \( x+1 \) & \( x^2+x \) \\ \hline
    \( x^2+x \) & 0 & \( x^2+x \) & \( x^2+x+1 \) & 1 & \( x^2+1 \) &
      \( x+1 \) & \( x \) & \( x^2 \) \\ \hline
    \( x^2+x+1 \) & 0 & \( x^2+x+1 \) & \( x^2+1 \) & \( x \) & 1 &
      \( x^2+x \) & \( x^2 \) & \( x+1 \) \\ \hline
  \end{tabular}
} \\[0.5cm]

\begin{center}
  You can find all my notes at \url{http://omgimanerd.tech/notes}. If you have
  any questions, comments, or concerns, please contact me at
  alvin@omgimanerd.tech
\end{center}

\end{document}
