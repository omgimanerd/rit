\documentclass{math}

\title{Introduction to Cryptography}
\author{Alvin Lin}
\date{January 2018 - May 2018}

\begin{document}

\maketitle

\section*{Polynomial Arithmetic}

\subsection*{Finite Fields of the Form \( GF(p^m) \)}
\textbf{Galois' Theorem:} An order-\textit{n} finite field exists if and only if
\( n = p^m \) for some prime \( p \) and some positive integer \( m \).
\begin{itemize}
  \item \( p \) is called the characteristic of this finite field.
  \item The order of a finite field is its number of elements.
  \item We use \( GF(p^m) \) or \( \mathbb{F}_{p^m} \) to represent the finite
  field of order \( p^m \).
  \item An order-\textit{n} finite field is unique (up to isomorphism).
  \item Addition and multiplication module a prime number \( p \) form a finite
  field \( \Z_p = GF(p) \).
  \item If \( m = 1 \), then \( \Z_p = GF(p) \)
  \item One way to construct a finite field with \( m > 1 \) is using the
  polynomial basis. The field is constructed as a set of \( p^m \) polynomials
  along with two polynomial operations.
\end{itemize}

\subsection*{Polynomial Arithmetic}
A polynomial \( f(x) \) is a mathematical expression of the form:
\[ a_nx^n+a_{n-1}x^{n-1}+\dots+a_0 \]
The highest exponent of \( x \) is the degree of the polynomial. \( a_n,
a_{n-1}, \dots, a_0 \) are called coefficients. We can add, subtract, multiply,
and divide polynomials.

\begin{center}
  You can find all my notes at \url{http://omgimanerd.tech/notes}. If you have
  any questions, comments, or concerns, please contact me at
  alvin@omgimanerd.tech
\end{center}

\end{document}
