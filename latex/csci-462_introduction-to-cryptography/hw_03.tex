\documentclass{math}

\usepackage{graphicx}
\usepackage{listings}

\geometry{letterpaper, margin=0.5in}

\title{Introduction to Cryptography: Homework 3}
\author{Alvin Lin}
\date{January 2018 - May 2018}

\begin{document}

\maketitle

\subsection*{Exercise 1}
Find the value of the Euler totient function \( \phi(n) \) for \( n = 937, 938,
939, 940, 941, 942 \). Show the details of computations.
\begin{itemize}
  \item \( \phi(937) = 936 \)
  \begin{align*}
    937 &= 937^1 \\
    \phi(937) &= (937^1-936^0) = (937-1) = 936
  \end{align*}
  \item \( \phi(938) = 396 \)
  \begin{align*}
    938 &= 2^1\times7^1\times67^1 \\
    \phi(938) &= (2^1-2^0)\times(7^1-7^0)\times(67^1-67^0) \\
    &= 1\times6\times66 = 396
  \end{align*}
  \item \( \phi(939) = 624 \)
  \begin{align*}
    939 &= 3\times313 \\
    \phi(939) &= (3^1-3^0)\times(313^1-313^0) = 2\times312 = 624
  \end{align*}
  \item \( \phi(940) = 368 \)
  \begin{align*}
    940 &= 2^2\times5\times47 \\
    \phi(940) &= (2^2-2^1)\times(5^1-5^0)\times(47^1-47^0) \\
    &= 2\times4\times46 = 368
  \end{align*}
  \item \( \phi(941) = 940 \)
  \begin{align*}
    941 &= 941^1 \\
    \phi(941) &= (941^1-941^0) = 940
  \end{align*}
  \item \( \phi(942) = \)
  \begin{align*}
    942 &= 2\times3\times157 \\
    \phi(942) &= (2^1-2^0)\times(3^1-3^0)\times(157^1-157^0) \\
    &= 1\times2\times156 = 312
  \end{align*}
\end{itemize}

\subsection*{Exercise 2}
Compute \( 41^{41}\mod937 \), using the modular square and multiply
exponentiation algorithm. Show the details of the computation.
\[ 41 = 101001_2 \]
\begin{center}
  \begin{tabular}{|c|c|c|}
    \hline
    Step & Accumulated Result & Binary Exponent \\
    \hline
    1 & \( r = 1^2\times41 \equiv 41\mod937 \) & 1 \\
    \hline
    2 & \( r = 41^2 \equiv 744\mod937 \) & 10 \\
    \hline
    3 & \( r = 744^2\times41 \equiv 836\mod937 \) & 101 \\
    \hline
    4 & \( r = 836^2 \equiv 831\mod937 \) & 1010 \\
    \hline
    5 & \( r = 831^2 \equiv 929\mod937 \) & 10100 \\
    \hline
    6 & \( r = 929^2\times41 \equiv 750\mod937 \) & 101001 \\
    \hline
  \end{tabular}
\end{center}
\[ 41^{41} \equiv 750\mod937 \]

\subsection*{Exercise 3}
Use the extended Euclidean algorithm to find the multiplicative inverse of
27 module \( n \), if it exists, for \( n = 1033, 1034, 1035 \). Show the
details of the computations.
\begin{itemize}
  \item \( 27^{-1}\mod1033 = 880 \)
  \begin{align*}
    1033 &= 38\times27+7 \\
    27 &= 3\times7+6 \\
    7 &= 1\times6+1 \\
    6 &= 6\times1+0 \\
    gcd(1033,27) &= 1 \\
    1 &= 7-1\times6 \\
    1 &= 7-(27-3\times7) \\
    &= -27+4\times7 \\
    &= -27+4\times(1033-38\times27) \\
    &= -153\times27+4\times1033 \\
    &= -153\times27\mod1033 \\
    27^{-1} &= 880\mod1033
  \end{align*}
  \item \( 27^{-1}\mod1034 = 383 \)
  \begin{align*}
    1034 &= 38\times27+8 \\
    27 &= 3\times8+3 \\
    8 &= 2\times3+2 \\
    3 &= 1\times2+1 \\
    gcd(1034,27) &= 1 \\
    1 &= 3-1\times2 \\
    &= 3-(8-2\times3) \\
    &= -8+3\times3 \\
    &= -8+3\times(27-3\times8) \\
    &= 3\times27-10\times8 \\
    &= 3\times27-10\times(1034-38\times27) \\
    &= 383\times27-10\times1034 \\
    &= 383\times27\mod1034 \\
    27^{-1} &= 383\mod1034
  \end{align*}
  \item \( 27^{-1}\mod1035 = \)
  \begin{align*}
    1035 &= 38\times27+9 \\
    27 &= 3\times9+0
  \end{align*}
  No modular inverse exist for 27 modulo 1035.
\end{itemize}

\subsection*{Exercise 4}
For each of the following compute the value of or argue that it is not defined.
For at least two of the six cases below, do the computations without using any
program, and describe briefly how you did it.
\begin{itemize}
  \item discrete logarithm of 2 base 3 mod 11
  \begin{align*}
    3^x\mod11 &\equiv 2 \\
    3^1\mod11 &\equiv 3 \\
    3^2\mod11 &\equiv 9\mod11 \equiv 9 \\
    3^3\mod11 &\equiv 9\times3\mod11 \equiv 5 \\
    3^4\mod11 &\equiv 5\times3\mod11 \equiv 4 \\
    3^5\mod11 &\equiv 4\times3\mod11 \equiv 1 \\
    3^6\mod11 &\equiv 1\times3\mod11 \equiv 3^1\mod11 \equiv 3
  \end{align*}
  We can calculate the discrete logarithm by brute force starting from an
  exponent of 1. This discrete logarithm is not defined since the modulo cycles
  and is not evenly distributed among all the numbers from 1 to 11.
  \item discrete logarithm of 3 base 2 mod 19
  \begin{align*}
    2^x\mod19 &\equiv 3 \\
    2^1\mod19 &\equiv 2\mod19 \equiv 2 \\
    2^2\mod19 &\equiv 2\times2\mod19 \equiv 4 \\
    2^3\mod19 &\equiv 4\times2\mod19 \equiv 8 \\
    2^4\mod19 &\equiv 8\times2\mod19 \equiv 16 \\
    2^5\mod19 &\equiv 16\times2\mod19 \equiv 13 \\
    2^6\mod19 &\equiv 13\times2\mod19 \equiv 7 \\
    2^7\mod19 &\equiv 7\times2\mod19 \equiv 14 \\
    2^8\mod19 &\equiv 14\times2\mod19 \equiv 9 \\
    2^9\mod19 &\equiv 9\times2\mod19 \equiv 18 \\
    2^{10}\mod19 &\equiv 18\times2\mod19 \equiv 17 \\
    2^{11}\mod19 &\equiv 17\times2\mod19 \equiv 15 \\
    2^{12}\mod19 &\equiv 15\times2\mod19 \equiv 11 \\
    2^{13}\mod19 &\equiv 11\times2\mod19 \equiv 3 \\
    x &= 13
  \end{align*}
  The discrete logarithm of 3 base 2 mod 19 is 13. \( 2^{13}\mod19 \equiv 3 \).
  \item discrete logarithm of 3 base 3 mod 97
  \[ 3^x\mod97 \equiv 3 \quad x = 1 \]
  \item discrete logarithm of 3 base 4 mod 97
  \[ 4^x\mod97 \equiv 3 \]
  No discrete logarithm exists. The resulting modulus will enter a loop and
  will never be equal to 3.
  \item discrete logarithm of 4 base 3 mod 97
  \[ 3^x\mod97 \equiv 4 \quad x = 38 \]
  See attached Python program for calculation script.
  \item discrete logarithm of 43 base 3 mod 97
  \[ 3^x\mod97 \equiv 43 \quad x = 22 \]
  See attached Python program for calculation script.
\end{itemize}

\subsection*{Exercise 5}
Solve problem 6.10 on page 171. Show the details of the computations. Compute
the inverse \( a^{-1}\mod n \) with Fermat's Theorem (if applicable) or
Euler's Theorem:
\begin{itemize}
  \item \( a = 4, n = 7 \)
  \begin{align*}
    a^{-1} &= a^{p-2}\mod p \quad\text{if } p \text{ is prime} \\
    4^{-1} &= 4^{7-2}\mod7 \\
    &= 4^5\mod7 = 2
  \end{align*}
  \item \( a = 5, n = 12 \)
  \begin{align*}
    12 &= 2\times5+2 \\
    5 &= 2\times2+1 \\
    gcd(12,5) &= 1 \\
    1 &= 5-(2\times2) \\
    &= 5-2\times(12-2\times5) \\
    &= 5\times5-2\times12 \\
    &= 5\times5\mod12 \\
    5^{-1} &= 5\mod12
  \end{align*}
  \item \( a = 6, n = 13 \)
  \begin{align*}
    a^{-1} &= a^{p-2}\mod p \quad\text{if } p \text{ is prime} \\
    6^{-1} &= 6^{13-2}\mod13 \\
    &= 6^{11}\mod13 = 11
  \end{align*}
\end{itemize}

\subsection*{Exercise 6}
Solve problem 7.1 on page 200. Show the details of the computations. Let the two
primes \( p = 41 \) and \( q = 17 \) be given as set-up parameters for RSA.
\begin{enumerate}
  \item Which of the parameters \( e_1 = 32, e_2 = 49 \) is a valid RSA
  exponent? Justify your choice.
  \begin{align*}
    n &= pq = 41\times17 = 697 \\
    \phi(n) &= (p-1)(q-1) = 640 \\
    gcd(\phi(n),e_1) &= gcd(640,32) = 32 \\
    gcd(\phi(n),e_2) &= gcd(640,49) = 1
  \end{align*}
  \( e_2 = 49 \) is a valid choice because it is coprime to \( \phi(n) \).
  \item Compute the corresponding private key \( K_{pr} = (p,q,d) \). Use the
  extended Euclidean algorithm for the inversion and point out every
  calculation step.
  \begin{align*}
    \phi(n) &= 640 \quad e = 49 \\
    de &\equiv 1\mod\phi(n) \\
    640 &= 13\times49+3 \\
    49 &= 16\times3+1 \\
    gcd(640,49) &= 1 \\
    1 &= 49-16\times3 \\
    &= 49-16\times(640-13\times49) \\
    &= 209\times49-16\times640 \\
    &= 209\times49\mod640 \\
    d &= e^{-1} = 209
  \end{align*}
\end{enumerate}

\subsection*{Exercise 7}
Solve problem 7.2 on page 200. Show the details of the computations. Computing
modular exponentiation efficiently is inevitable for the practicability of RSA.
Compute the following exponentiations \( x^e\mod m \) applying the square and
multiply algorithm:
\begin{enumerate}
  \item \( x = 2, e = 79, m = 101 \)
  \[ 79 = 1001111_2 \]
  \begin{center}
    \begin{tabular}{|c|c|c|c|}
      \hline
      Step & Accumulated Result & Binary Exponent \\
      \hline
      1 & \( r = 1^2\times2 \equiv 2\mod101 \) & 1 \\
      \hline
      2 & \( r = 2^2 \equiv 4\mod101 \) & 10 \\
      \hline
      3 & \( r = 4^2 \equiv 16\mod101 \) & 100 \\
      \hline
      4 & \( r = 16^2\times2 \equiv 7\mod101 \) & 1001 \\
      \hline
      5 & \( r = 7^2\times2 \equiv 98\mod101 \) & 10011  \\
      \hline
      6 & \( r = 98^2\times2 \equiv 18\mod101 \) & 100111 \\
      \hline
      7 & \( r = 18^2\times2 \equiv 42\mod101 \) & 1001111 \\
      \hline
    \end{tabular}
  \end{center}
  \[ 2^{79}\mod101 = 42 \]
  \item \( x = 3, e = 197, m = 101 \)
  \[ 197 = 11000101_2 \]
  \begin{center}
    \begin{tabular}{|c|c|c|c|}
      \hline
      Step & Accumulated Result & Binary Exponent \\
      \hline
      1 & \( r = 1^2\times2 \equiv 2\mod101 \) & 1 \\
      \hline
      2 & \( r = 2^2\times2 \equiv 8\mod101 \) & 11 \\
      \hline
      3 & \( r = 8^2 \equiv 64\mod101 \) & 110 \\
      \hline
      4 & \( r = 64^2 \equiv 56\mod101 \) & 1100 \\
      \hline
      5 & \( r = 56^2 \equiv 5\mod101 \) & 11000  \\
      \hline
      6 & \( r = 5^2\times2 \equiv 50\mod101 \) & 110001 \\
      \hline
      7 & \( r = 50^2 \equiv 76\mod101 \) & 1100010 \\
      \hline
      8 & \( r = 76^2\times2 \equiv 38\mod101 \) & 11000101 \\
      \hline
    \end{tabular}
  \end{center}
  \[ 2^{197}\mod101 = 38 \]
\end{enumerate}

\subsection*{Exercise 8}
Solve problem 7.3 on page 200. Show the details of the computations. Encrypt
and decrypt by means of the RSA algorithm with the following system parameters:
\begin{itemize}
  \item \( p = 3, q = 11, d = 7, x = 5 \)
  \begin{align*}
    n &= pq = 33 \\
    \phi(n) &= \phi(33) = (11-1)(3-1) = 20 \\
    de &\equiv 1\mod\phi(n) \\
    e &= d^{-1}\mod\phi(n) = 3\mod20 \\
    y &= x^e\mod n = 5^3\mod 33 = 26
  \end{align*}
  \item \( p = 5, q = 11, e = 3, x = 9 \)
  \begin{align*}
    n &= pq = 55 \\
    \phi(n) &= \phi(55) = (11-1)(5-1) = 40 \\
    de &\equiv 1\mod\phi(n) \\
    d &= \e^{-1}\mod\phi(n) = 27\mod40 \\
    y &= x^d\mod n = 9^{27}\mod55 = 4
  \end{align*}
\end{itemize}

\begin{center}
  If you have any questions, comments, or concerns, please contact me at
  alvin@omgimanerd.tech
\end{center}

\end{document}
