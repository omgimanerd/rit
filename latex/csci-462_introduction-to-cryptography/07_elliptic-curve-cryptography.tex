\documentclass{math}

\title{Introduction to Cryptography}
\author{Alvin Lin}
\date{January 2018 - May 2018}

\begin{document}

\maketitle

\section*{Elliptic Curve Cryptography}
Asymmetric schemes like RSA and El Gamal require exponentiations in integer
rings and fields with parameters of more than 1000 bits. This requires a lot
of computational effort and are difficult to store on small or embedded devices.
Smaller field sizes providing equivalent security are desirable. Elliptic Curve
Cryptography uses a group of points (instead of integers) for cryptographic
schemes with coefficient sizes of 160-256 bits, reducing the computational
effort significantly.

\subsection*{Abelian Group}
An Abelian group is a set of elements with a binary operation,
denoted by \( \bullet \), that associates to each ordered pair \( (a,b) \) of
elements in \( G \) an element \( (a\bullet b)\in G \) such that the following
axioms are obeyed.
\begin{itemize}
  \item Closure: If \( a \) and \( b \) belong to \( G \), then \( a\bullet b \)
  is also in \( G \).
  \item Associative: \( a\bullet(b\bullet c) = (a\bullet b)\bullet c \) for all
  \( a,b,c \) in \( G \).
  \item Identity Element: There is an element \( e \) in \( G \) such that
  \( a\bullet e = e\bullet a = a \) for all \( a \) in \( G \).
  \item Inverse Element: For each \( a \) in \( G \), there is an element
  \( a^{-1} \) in \( G \) such that \( a\bullet a^{-1} = a^{-1}\bullet a = e \).
  \item Commutative: \( a\bullet b = b\bullet a \) for all \( a,b \) in \( G \).
\end{itemize}

\subsection*{Elliptic Curves over the Reals}
Elliptic curves are not ellipses. They are named so because they are described
by cubic equations, similar to those used for calculating the circumference of
an ellipse. In general, cubic equations for elliptic curves take the following
form, known as a \textbf{Weierstrass equation}:
\[ y^2+axy+by = x^3+cs^2+dx+e \]
where \( a,b,c,d,e \) are real numbers and \( x \) and \( y \) take on values
in the real number system. For our purpose it is sufficient to limit ourselves
to equations of the form:
\[ y^2 = x^3+ax+b \]
Such equations are said to be cubic, or of degree 3, because the highest
exponent they contain is a 3. Also included in the definition of an elliptic
curve is a single element denoted \( O \) and called the \textbf{point at
infinity} or the \textbf{zero point}. To plot such a curve, we need to compute
\[ y = \sqrt{x^3+ax+b} \]
For given values of \( a \) and \( b \), the plot consists of positive and
negative evalues of \( y \) for each value of \( x \). Thus, each curve is
symmetric about \( y = 0 \). The identity point \( O \) is added to the group
definition. Elliptic curves are symmetric along the \( x-axis \). Up to two
solutions \( y \) and \( -y \) exist for each quadratic residue \( x \) of the
elliptic curve. For each point \( P = (x,y) \), the inverse or negative point
is defined as \( -P = (x,-y) \).
\par Group elements are defined as all points \( (x,y) \) on some elliptic curve
\( y^2 = x^3+ax+b \) and \( O \), the identity point. This is an infinite group
and the group operation is customarily called ``point addition'', symbolized
using the \( + \) sign.

\subsection*{Computations on Elliptic Curves}
Generating a group of points on elliptic curves is based on the point addition
operation \( P+Q = R \): \( (x_p,y_p)+(x_q,y_q) = (x_r,y_r) \).

\subsubsection*{Elliptic Curve Point Addition and Doubling}
\begin{align*}
  x_3 &= s^2-x_1-x_2\mod p \\
  y_3 &= s(x_1-x_3)-y_1\mod p \\
  s &= \begin{cases}
    \frac{y_2-y_1}{x_2-x_1}\mod p &\quad \text{ if } P\ne Q \\
    \frac{3x_1^2+a}{2y_1}\mod p &\quad \text{ if } P=Q
  \end{cases}
\end{align*}
Elliptic curves can not just be defined over the real numbers \( \R \) but also
over many other types of finite fields.

\subsection*{Application in Cryptography}
Elliptic curve cryptography uses curves whose variables and coefficients are
finite. There are two families of elliptic curves used in cryptography
applications:
\begin{itemize}
  \item Binary curves of \( GF(2^m) \): All variables and coefficients take on
  values in \( GF(2^m) \) and calculations are performed over \( GF(2^m) \).
  This is best for hardware applications.
  \item Prime curves over \( \Z_p \): These use cubic equations in which the
  variables and coefficients all take on values in the set of integers from
  \( 0 \) to \( p-1 \) and in which calculations are perfoemd module \( p \).
  This is best for software applications.
\end{itemize}

\subsection*{Elliptic Curves over Prime Fields}
The elliptic curve \( E_p(a,b) \) over \( \Z_p, p>3 \) is the set of all pairs
\( (x,y)\in\Z_p \) which fulfill
\[ y^2 = x^3+ax+b\mod p \]
together with an imaginary point of infinity \( O \), where \( a,b\in\Z_p \)
and the condition \( 4a^3+27b^2\ne 0\mod p \).
\begin{itemize}
  \item These points form a finite group. Note that \( \Z_p = \{0,1,\dots,p-1\}
  \) is a set of integers with modulo \( p \) arithmetic. There is no obvious
  geometric interpretation of elliptic curve arithmetic over finite fields.
  \item The algebraic interpretation used for elliptic curve arithmetic carries
  over since point addition is the same as before, except all arithmetic is
  done modulo \( p \).
  \item If \( P = (x,y) \), the inverse \( -P = (x,-y) \)
  \item \( P+(-P) = O \)
\end{itemize}

\subsubsection*{Example}
Given \( E: y^2 = x^3+2x+2\mod 17 \) and point \( P = (5,1) \), compute
\( 2P = P+P = (5,1)+(5,1) = (x_3,y_3) \).
\begin{align*}
  s &= \frac{3x_1^2+a}{2y_1} \\
  &= (2\cdot1)^{-1}(3\cdot5^2+2) \\
  &= 2^{-1}\cdot9 \equiv 9\cdot9 \equiv 13\mod17 \\
  x_3 &= s^2-x-1-x_2 = 13^2-5-5 = 159 \equiv 6\mod17 \\
  y_3 &= x(x_1-x_3)-y_1 = 13(5-6)-1 = -14 \equiv 3\mod17 \\
  2P &= (5,1)+(5,1) = (6,3)
\end{align*}

\subsubsection*{Example}
Consider an elliptic curve \( y^2 = x^3+5x+7\mod19 \). To find all the group
elements, we need to find the numbers that are squares modulo 19. Then for each
value of \( x \), we ask if \( z = x^3+5x+7 \) is a square modulo 19. If \( z \)
is a square, then points \( (x,\sqrt{z}) \) are group elements. The point at
infinity \( O \) is always a group element. The points on an elliptic curve and
the point at infinity \( O \) form \textbf{cyclic subgroups}.

\subsection*{Number of Points on an Elliptic Curve}
How many points can be on an arbitrary elliptic curve? Consider the previous
example of \( E: y^2 = x^3+2x+2\mod19 \). This curve has 19 points. However,
determining the point count on elliptic curves in general is hard. Hasse's
Theorem bounds the number of points to a restricted interval. Given an
elliptic curve modulo \( p \), the number of points on the curve is denoted by
\( \#E \) and is bounded by:
\[ p+1-2\sqrt{p} \le \#E \le p+1+2\sqrt{p} \]
The number of points is close to the prime \( p \).

\subsection*{Elliptic Curve Discrete Logarithm Problem}
Given a primitive element \( P \) and another element \( Q \) on an elliptic
curve \( E \), the elliptic curve discrete logarithm problem is finding the
integer \( d \) where \( 1\le d\le \#E \) such that:
\[ P+P+\dots+P ~(d\text{ times}) = dP = Q \]
Cryptosystems are based on the idea that \( d \) is large and difficult for
attackers to compute. Consider the group \( E_23(9,17) \). This is the group
defined by the equation \( y_2\mod23 = (x^3+9x+17)\mod23 \). What is the
discrete logarithm \( d \) of \( Q = (4,5) \) to the base \( P = (16,5) \)?
The brute force method is to compute multiples of \( P \) until \( Q \) is
found.
\begin{align*}
  P &= (16,5) \\
  2P &= (20,20) \\
  3P &= (14,14) \\
  4P &= (19,20) \\
  5P &= (13,10) \\
  6P &= (7,3) \\
  7P &= (8,7) \\
  8P &= (12,17) \\
  9P &= (4,5)
\end{align*}
Because \( 9P = (4,5) = Q \), the discrete logarithm \( Q = (4,5) \) to the
base \( P = (16,5) \) is \( d = 9 \). In a real application, \( d \) would be so
large as to make the brute force approach infeasible. If \( d \) is known, an
efficient method to compute the point multiplication \( dP \) is required to
create a reasonable cryptosystem. The square and multiply algorithm can be
adapted to elliptic curves, but here it is known as the double and add
algorithm.

\subsection*{Double and Add Algorithm}
If \( P \) is a point (elliptic curve group element) and \( n \) is an integer,
then \( n\bullet P \) is \( n \) copies of \( P \) added together using point
addition. In the elliptic curve \( y^2 = x^3+5x+7\mod19 \), \( 19\bullet(3,12)
\) is computed by repeatedly doubling using point addition:
\begin{align*}
  1\bullet(3,12) &= (3,12) \\
  2\bullet(3,12) &= (3,12)+(3,12) = (0,11) \\
  4\bullet(3,12) &= (0,11)+(0,11) = (7,9) \\
  8\bullet(3,12) &= (7,9)+(7,9) = (5,10) \\
  16\bullet(3,12) &= (5,10)+(5,10) = (6,5)
\end{align*}
Then the point multiplication is computed by adding the proper multiples of
\( (3,12) \):
\begin{align*}
  19\bullet(3,12) &= (16+2+1)\bullet(3,12) \\
  &= 16\bullet(3,12)+2\bullet(3,12)+1\bullet(3,12) \\
  &= (6,5)+(0,11)+(3,12) \\
  &= (14,3)+(3,12) \\
  &= (0,8)
\end{align*}

\subsection*{Point Addition Algorithm}
The point addition algorithm is an algorithm that takes an prime \( p > 3 \),
\( a,b \) for an elliptic curve \( E: y^2 = x^3+ax+b\mod p \), and points
\( P_0 = (x_0,y_0), P_1 = (x_1,y_1) \) on \( E \) and outputs a point
\( P_2 := P_0+P_1 \).
\begin{enumerate}
  \item If \( P_0 = O \), output \( P_2\leftarrow P_1 \).
  \item If \( P_1 = O \), output \( P_2\leftarrow P_0 \).
  \item If \( x_0\ne x_1 \), then set
  \[ s\leftarrow \frac{y_0-y_1}{x_0-x_1}\mod p \]
  and go to step 7.
  \item If \( y_0\ne y_1 \), then output \( P_2\leftarrow O \).
  \item If \( y_1 = 0 \), then output \( P_2\leftarrow O \).
  \item Set
  \[ s\leftarrow \frac{3x_1^2+a}{2y_1}\mod p \]
  \item Set
  \[ x_2\leftarrow s^2-x_0-x_1\mod p \]
  \item Set
  \[ y_2\leftarrow s(x_1-x_2)-y_1\mod p \]
  \item Output \( P_2\leftarrow (x_2,y_2) \)
\end{enumerate}

\subsection*{Elliptic Curve Parameter Generation}
Elliptic curve cryptographic algorithms require the following parameters:
\begin{itemize}
  \item A prime modulus \( p \)
  \item Elliptic curve coefficients \( a \) and \( b \)
  \item A generator (a point on the elliptic curve) \( G = (x_g,y_g) \) on
  which point multiplication will be done. The order of \( G \) is the value
  \( n \) such that \( n\bullet G = O \) where \( n \) should be a large prime
  number.
\end{itemize}
The security of the depends on the size of \( n \). Generating suitable elliptic
curve parameters is complicated, but NIST has recommended several sets of
elliptic curve parameters. \texttt{Secp256k1} refers to the parameters of the
ECDSA curve used in Bitcoin. The ellipic curve equation used is \( y^2 = x^3+7
\), with \( p = 2^{256}-2^{32}-2^9-2^8-2^7-2^6-2^4-1 \).

\subsection*{Security Aspects}
Why are parameters significantly smaller for elliptic curves (160-256 bits) than
for RSA (1024-3076 bits)?
\begin{itemize}
  \item Attacks on groups of elliptic curves are weaker than available factoring
  algorithms or integer discrete logarithm attacks.
  \item The best known attacks on elliptic curves (chosen according to
  cryptography criteria) are the Baby-Step Giant-Step and Pollard-Rho method.
  \item The complexity of these methods on average requires roughly \( \sqrt{p}
  \) steps before the elliptic curve discrete logarithm problem can be
  successfully solved. An elliptic curve using a prime \( p \) with 160 bits
  and roughly \( 2^{160} \) points provides a security of \( 2^{80} \) steps
  that are required by an attacker on average. An elliptic curve using a prime
  \( p \) with 256 bits (roughly \( 2^{256} \) points) provides a security of
  \( 2^{128} \) steps on average.
\end{itemize}

\subsection*{Implementation}
Elliptic curve computations are usually regarded as consisting of four layers:
\begin{enumerate}
  \item Basic modular arithmetic operations, which are computationally the most
  expensive.
  \item Group operations implementing point doubling and point addition.
  \item Point multiplication operations implemented using the double and add
  algorithm.
  \item Upper layer protocols like Elliptic Curve Diffie-Hellman and the
  Elliptic Curve Digital Signature Algorithm.
\end{enumerate}

\begin{center}
  You can find all my notes at \url{http://omgimanerd.tech/notes}. If you have
  any questions, comments, or concerns, please contact me at
  alvin@omgimanerd.tech
\end{center}

\end{document}
