\documentclass{math}

\usepackage{graphicx}
\usepackage{listings}

\geometry{letterpaper, margin=0.5in}

\title{Introduction to Cryptography: Homework 2}
\author{Alvin Lin}
\date{January 2018 - May 2018}

\begin{document}

\maketitle

\subsection*{Exercise 1}
Assume the initialization vector and the key of Trivium each consist of 80
all-zero bits. Compute the first 70 bits \( s_1,\dots,s_{70} \) during the
warm-up phase of Trivium. Note that these are only internal bits which are
not used for encryption since the warm-up phase lasts for 1152 clock cycles. \\
\( s_1, \dots, s_{70} \) =
{\tt 0110000000000000000000000000000000000000000000000000000000000000000110} \\
Code used:
\begin{lstlisting}
def trivium_cycle(a, b, c):
  assert(len(a) == 93)
  assert(len(b) == 84)
  assert(len(c) == 111)
  assert(type(a) is deque)
  assert(type(b) is deque)
  assert(type(c) is deque)

  a_out = a[65] ^ a[92] ^ (a[90] & a[91])
  b_out = b[68] ^ b[83] ^ (b[81] & b[82])
  c_out = c[65] ^ c[110] ^ (c[108] & c[109])

  a_in = a[68] ^ c_out
  b_in = b[77] ^ a_out
  c_in = c[86] ^ b_out

  a.rotate(1)
  b.rotate(1)
  c.rotate(1)

  a[0] = a_in
  b[0] = b_in
  c[0] = c_in

  return a_out ^ b_out ^ c_out
\end{lstlisting}

\subsection*{Exercise 2}
A linear feedback shift register (LFSR) is used to generate a keystream for a
shift cipher. The LSFR has five bits (\( s_4 s_3 s_2 s_1 s_0 \)), the feedback
bit is given by the formula \( s_3+s_0 (\mod 2) \), and the sequence of
\( s_0 \) values forms the keystream. The LFSR is initialized with the
\( (s_4 s_3 s_2 s_1 s_0) = (1 1 0 1 1) \). How many keystream bits will be
generated before the keystream starts repeating? What is the sequence of
keystream bits?
\begin{center}
  \begin{tabular}{|c|c|c|c|c|c|}
    \hline
    Clock & \( s_4 \) & \( s_3 \) & \( s_2 \) & \( s_1 \) & \( s_0 = k \) \\
    \hline
    0 & 1 & 1 & 0 & 1 & 1 \\
    1 & 0 & 1 & 1 & 0 & 1 \\
    2 & 0 & 0 & 1 & 1 & 0 \\
    3 & 0 & 0 & 0 & 1 & 1 \\
    4 & 1 & 0 & 0 & 0 & 1 \\
    5 & 1 & 1 & 0 & 0 & 0 \\
    6 & 1 & 1 & 1 & 0 & 0 \\
    7 & 1 & 1 & 1 & 1 & 0 \\
    8 & 1 & 1 & 1 & 1 & 1 \\
    9 & 0 & 1 & 1 & 1 & 1 \\
    10 & 0 & 0 & 1 & 1 & 1 \\ \hline
    \textbf{11} & 1 & 0 & 0 & 1 & 1 \\
    12 & 1 & 1 & 0 & 0 & 1 \\
    13 & 0 & 1 & 1 & 0 & 0 \\
    14 & 1 & 0 & 1 & 1 & 0 \\
    15 & 0 & 1 & 0 & 1 & 1 \\
    16 & 0 & 0 & 1 & 0 & 1 \\ \hline
    \textbf{17} & 1 & 0 & 0 & 1 & 1 \\
    \hline
  \end{tabular}
\end{center}
10 bits will be generated before the keystream begins to enter a repeating
sequence. This LSFR has a keystream sequence of 110011.

\subsection*{Exercise 3}
Alex and Blake are encrypting messages using RC4. You, Harry the Hacker, are
eavesdropping on their communications. Each plaintext message is a sequence of
characters; each character is represented as an 8-bit binary number using the
ASCII character encoding. Alex and Blake are using the same key to encrypt
every message. Because RC4 does not define how to incorporate a nonce into the
keystream generator algorithm, Alex and Blake are using this (insecure!)
scheme: Generate the keystream using the (fixed) key, then add (mod 256) the
nonce to each byte of the keystream. You happen to know that when Alex sent the
plaintext \texttt{BARACKOBAMA} with a nonce of 1, the ciphertext was: \\
\texttt{01000011 00011011 00010010 00110000 11111000 10100111 10001110 11101001
00010100 00011101 01100100} \\
You now observe Blake send the following ciphertext with a nonce of 2:
\texttt{01000110 00010100 00001111 00110011 11110000 10101001 10010110 11111110
00000011 00011100 01110110} \\
What is the plaintext of Blake's message? Explain how you found the plaintext.
Your answer must be a narrative description, not code or pseudocode. \\[1cm]
The first thing we need to do is figure out the keystream. We can convert the
known plaintext into its ASCII representation in bits. Because of the
properties of XOR, if \( a\oplus b = c \), then it is also true that
\( a\oplus c = b \). Thus, can XOR the bits of the known plaintext with the bits
of the known ciphertext output to get the keystream nonce combination that was
used to encrypt it.
\begin{center}
  \scalebox{0.7}{
    \begin{tabular}{|c|c|c|c|c|c|c|c|c|c|c|c|}
      \hline
      Plaintext ASCII & B & A & R & A & C & K & O & B & A & M & A \\
      \hline
      Plaintext bits & 01000010 & 01000001 & 01010010 & 01000001 & 01000011 &
        01001011 & 01001111 & 01000010 & 01000001 & 01001101 & 01000001 \\
      \hline
      Ciphertext bits & 01000011 & 00011011 & 00010010 & 00110000 & 11111000 &
        10100111 & 10001110 & 11101001 & 00010100 & 00011101 & 01100100 \\
      \hline
      Keystream + Nonce & 00000001 & 01011010 & 01000000 & 01110001 & 10111011 &
        11101100 & 11000001 & 10101011 & 01010101 & 01010000 & 00100101 \\
      \hline
    \end{tabular}
  }
\end{center}
Since we know that a nonce of 1 was added to each byte for this specific
message, we can subtract that from each byte of the keystream to get the fixed
key. We don't need to do this since we know the second message was encrypted
using a nonce of two. The keystream used to encrypt the second message can be
obtained merely by adding 1 (mod 256) to the first keystream.
\begin{center}
  \scalebox{0.7}{
    \begin{tabular}{|c|c|c|c|c|c|c|c|c|c|c|c|}
      \hline
      Fixed Key + 1 & 00000001 & 01011010 & 01000000 & 01110001 & 10111011 &
        11101100 & 11000001 & 10101011 & 01010101 & 01010000 & 00100101 \\
      \hline
      Fixed Key + 2 & 00000010 & 01011011 & 01000001 & 01110010 & 10111100 &
        11101101 & 11000010 & 10101100 & 01010110 & 01010001 & 00100110 \\
      \hline
    \end{tabular}
  }
\end{center}
We can now decrypt the second ciphertext using this new keystream. We can XOR
each bit of the new keystream with each bit of the second ciphertext to get each
bit of the plaintext.
\begin{center}
  \scalebox{0.7}{
    \begin{tabular}{|c|c|c|c|c|c|c|c|c|c|c|c|}
      \hline
      Keystream & 00000010 & 01011011 & 01000001 & 01110010 & 10111100 &
        11101101 & 11000010 & 10101100 & 01010110 & 01010001 & 00100110 \\
      \hline
      Ciphertext bits & 01000110 & 00010100 & 00001111 & 00110011 & 11110000 &
        10101001 & 10010110 & 11111110 & 00000011 & 00011100 & 01110110 \\
      \hline
      Plaintext bits & 01000100 & 01001111 & 01001110 & 01000001 & 01001100 &
        01000100 & 01010100 & 01010010 & 01010101 & 01001101 & 01010000 \\
      \hline
    \end{tabular}
  }
\end{center}
Once we have the bits of the plaintext, we can simply convert each byte back
into ASCII to get the readable plaintext.
\begin{center}
  \scalebox{0.7}{
    \begin{tabular}{|c|c|c|c|c|c|c|c|c|c|c|c|}
      \hline
      Plaintext bits & 01000100 & 01001111 & 01001110 & 01000001 & 01001100 &
        01000100 & 01010100 & 01010010 & 01010101 & 01001101 & 01010000 \\
      \hline
      Plaintext ASCII & D & O & N & A & L & D & T & R & U & M & P \\
      \hline
    \end{tabular}
  }
\end{center}
Blake's message was \texttt{DONALDTRUMP}.

\begin{center}
  If you have any questions, comments, or concerns, please contact me at
  alvin@omgimanerd.tech
\end{center}

\end{document}
