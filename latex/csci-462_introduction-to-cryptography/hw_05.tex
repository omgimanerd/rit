\documentclass{math}

\usepackage{listings}
\geometry{letterpaper, margin=0.5in}

\title{Introduction to Cryptography: Homework 5}
\author{Alvin Lin}
\date{January 2018 - May 2018}

\begin{document}

\maketitle

\subsection*{Hash Algorithm}
This hash function uses a pseudorandom number generator to generate a 16-bit
hash and uses the input to the hash function as a seed to the random number
generator.

\subsubsection*{Pseudocode}
\begin{lstlisting}
  def string_to_seed(s):
    seed = ""
    for character in s:
      seed += ascii_code(character)
    return int(seed)

  def hash(s):
    seed = string_to_seed(s)
    random.seed(s)
    bits = []
    for i in 1..16:
      bit[i] = random.randint(1)
    return bits
\end{lstlisting}
This hash function defers the job of generating relatively unique sequences to
the pseudorandom number generator, which makes this hash function work
differently depending on the underlying \texttt{random} library implementation.
\\

\subsubsection*{Analysis}
The Python \texttt{random} library uses the Mersenne Twister algorithm to
generate pseudorandom numbers. Because the bit length of the hash is only 16
bits, collisions were found almost instanteously.
\begin{lstlisting}
  hash("356") == hash("24") == 0x6378
  hash("440") == hash("313") == 0x4384
  hash("841") == hash("197") == 0xc18
  hash("1013") == hash("679") == 0x26a7
  hash("1014") == hash("186") == 0x87ce
  hash("1060") == hash("624") == 0xafa2
  hash("1097") == hash("952") == 0x2882
  hash("1219") == hash("760") == 0x929a
  hash("1348") == hash("391") == 0xba6
  hash("1439") == hash("244") == 0x92f0
  hash("1487") == hash("1248") == 0xb645
  hash("1511") == hash("678") == 0xf1cc
  hash("1624") == hash("1350") == 0xd2d7
  hash("1683") == hash("376") == 0x7b59
  hash("1700") == hash("1459") == 0xc10c
  hash("1789") == hash("1039") == 0x9626
  hash("1812") == hash("92") == 0xb39d
  hash("1897") == hash("206") == 0xff3
  hash("1933") == hash("1391") == 0x85a9
  hash("1994") == hash("1181") == 0xa5bc
\end{lstlisting}
This hash function's properties will depend on the underlying PRNG. If a linear
shift feedback register is used, then the input (seed) is used as the start
state in the LSFR. Many start states are congruent in that the will have the
same period and yield the same sequence of numbers, leading to collision in the
hash function above.

\subsection*{Exercise 11.5}
We consider three different hash functions which produce outputs of lengths
64, 128, and 160 bit. After how many random inputs do we have a probability of
\( \epsilon = 0.5 \) for a collision? After how many random inputs do we have a
probability of \( \epsilon = 0.1 \) for a collision. \\
\[ t \approx 2^{\frac{n+1}{2}}\sqrt{\ln(\frac{1}{1-\epsilon})} \]
Output length 64 bit:
\begin{align*}
  t_{0.5} &\approx 2^{\frac{65}{2}}\sqrt{\ln\frac{1}{1-0.5}} \\
  &= 2^{32.5}2^{\log_2(0.832)} \\
  &= 2^{32.5-0.264} \\
  &= 2^{32.23} \text{ random inputs} \\
  t_{0.1} &\approx 2^{\frac{65}{2}}\sqrt{\ln\frac{1}{1-0.1}} \\
  &= 2^{32.5}2^{\log_2(0.324)} \\
  &= 2^{32.5-1.623} \\
  &= 2^{30.87} \text{ random inputs}
\end{align*}
Output length 128 bit:
\begin{align*}
  t_{0.5} &\approx 2^{\frac{129}{2}}\sqrt{\ln\frac{1}{1-0.5}} \\
  &= 2^{64.5}2^{\log_2(0.832)} \\
  &= 2^{64.5-0.264} \\
  &= 2^{64.236} \text{ random inputs} \\
  t_{0.1} &\approx 2^{\frac{129}{2}}\sqrt{\ln\frac{1}{1-0.1}} \\
  &= 2^{64.5-1.623} \\
  &= 2^{62.877} \text{ random inputs}
\end{align*}
Output length 160 bit:
\begin{align*}
  t_{0.5} &\approx 2^{\frac{161}{2}-0.264} \\
  &= 2^{80.236} \text{ random inputs} \\
  t_{0.1} &\approx 2^{\frac{161}{2}-1.623} \\
  &= 2^{78.877} \text{ random inputs}
\end{align*}

\begin{center}
  If you have any questions, comments, or concerns, please contact me at
  alvin@omgimanerd.tech
\end{center}

\end{document}
