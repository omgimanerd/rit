\documentclass{math}

\title{Introduction to Cryptography}
\author{Alvin Lin}
\date{January 2018 - May 2018}

\begin{document}

\maketitle

\section*{Ciphers and Cryptosystems}

\subsection*{Hill Cipher}
The Hill Cipher was developed by Lester Hill in 1929. It's strength was that it
completely hid single-letter frequencies. It was strong against a
ciphertext-only attack but easily broken by a known plaintext attack. The Hill
cipher is done by taking a matrix as the key:
\[ K = \begin{bmatrix}2 & 1 \\ 1 & 1\end{bmatrix} \]
Given the plaintext HELP, we split it into \( 1\times2 \) column vectors so
that they can be multiplied by the key vector:
\[ \begin{bmatrix}H \\ E\end{bmatrix},
  \begin{bmatrix}L \\ P\end{bmatrix}\to
  \begin{bmatrix}7 \\ 4\end{bmatrix},
  \begin{bmatrix}11 \\ 15\end{bmatrix} \]
The ciphertext is produced by multiplying the key matrix with the column
vectors.
\begin{align*}
  \begin{bmatrix}2 & 1 \\ 1 & 1\end{bmatrix}
    \begin{bmatrix}7 \\ 4\end{bmatrix} &\equiv
    \begin{bmatrix}18 \\ 11\end{bmatrix}\mod(26) \\
  \begin{bmatrix}2 & 1 \\ 1 & 1\end{bmatrix}
    \begin{bmatrix}11 \\ 15\end{bmatrix} &\equiv
    \begin{bmatrix}11 \\ 0\end{bmatrix}\mod(26)
\end{align*}
This yields the cipher text 18-11-11-0 or RKKA. Decryption is done by computing
the inverse of the key matrix \( K^{-1} \) and repeating the same process.

\subsection*{Stream Ciphers}
Stream ciphers were invented in 1917 by Gilbert Vernam. It involves a plaintext
\( x \) and a key stream \( k \) to yield a ciphertext \( y \). Encryption
and decryption operations are simple additions modulo 2 (aka XOR).
\begin{itemize}
  \item Encryption: \( y_i = x_i+k_i\mod2 \)
  \item Decryption: \( x_i = y_i+k_i\mod2 \)
\end{itemize}
Stream ciphers encrypt bits individually. They're usually small and fast, and
are common in embedded devices. Block ciphers on the other hand, encrypt whole
blocks of information and are common for Internet applications. The security of
stream ciphers depend entirely on the key stream. The key stream \( k \) should
be random, but reproducible by the sender and the receiver. \textbf{Synchronous}
stream ciphers use a key stream that only depends on the key, while
\textbf{asynchronous} stream ciphers use a key stream that also depends on the
ciphertext.

\subsection*{Random Number Generators}
Random number generators are needed in cryptography, particularly for stream
ciphers. There are three types: true random number generators, pseudorandom
number generators, and cryptographically secure random number generators. The
basic requirements for randomness:
\begin{itemize}
  \item Uniform Distribution (The frequency of occurrence of ones and zeroes
  should be approximately equal)
  \item Independence (Each bit should be uncorrelated with all previous bits)
\end{itemize}
For cryptography, the compromise of one output must not compromise future or
previous outputs.

\subsubsection*{True Random Number Generators}
True random number generators are based on some random physical processes, such
as coin flipping, dice rolling, semiconductor noise, radioactive decay, mouse
movement, or clock jitter of digital circuits. The output stream should have
good statistical properties and should not be able to be predicted or
reproduced. These are typically used for the generation of keys and nonces.

\subsubsection*{Pseudorandom Number Generators}
Pseudorandom number generators are algorithms used to produce an open-ended
sequence of bits. They generate sequences from an initial seed value. The output
stream has good statistical properties and can typically be reproduced and
predicted.
\par The basic requirement when a pseudorandom number generator or pseudorandom
function is used for a cryptographic application is that an adversary who does
not know the seed is unable to determine the pseudorandom string. The bit stream
should appear random even though it is deterministic, and should have no
correlation with the seed.

\subsubsection*{Linear Congruential Generator}
The linear congruential generator is an algorithm first proposed by Lehmer
parameterized with four numbers \( a,b,m,seed \). The sequence of random
numbers \( x_n \) is obtained via the following iterative equation:
\[ x_{n+1} = (ax_n+b)\mod(m) \]
This has bad cryptographic properties due to its linearity.

\subsection*{One-Time Pad}
The one-time pad was an improvement to the Vernam cipher propsed by Army
Signal Corp officer Joseph Mauborgne. It uses and random key that is as long
as the message so that the key does not need to be repeated. The key is used
to encrypt and decrypt a single message and then is discarded. Each new message
requires a new key of the same length as the new message.
\par This scheme is unbreakable because it produces random output with no
statistical relationship to the plaintext. The one-time pad offers complete
security but has fundamental problems, namely key distribution and key
generation. Thus, the one-time pad is useful primarily only in low-bandwidth
channels requiring very high security.

\subsection*{Linear Feedback Shift Registers}
Linear feedback shift registers are cascades of flip flops, sharing the same
clock, whose input bit is a linear function of its previous state. The feedback
portion computes fresh input bits by calculating the XOR of certain state bits.
Their degree is given by the number of storage elements \( m \), with the
maximum output length being \( 2^m-1 \). Linear feedback shift registers are
typically described by polynomials:
\[ P(x) = x^m+p_{m-1}x^{m-1}+\dots+p_2x^2+p_1x+x_0 \]
Single linear feedback shift registers generate highly predictable output. If
\( 2m \) output bits of a linear shift feedback register of degree \( m \) are
known, the feedback coefficients \( p_i \) can be found by solving a system of
linear equations. Because of this, most stream ciphers use combinations of
stream ciphers.

\begin{center}
  You can find all my notes at \url{http://omgimanerd.tech/notes}. If you have
  any questions, comments, or concerns, please contact me at
  alvin@omgimanerd.tech
\end{center}

\end{document}
