\documentclass{math}

\geometry{letterpaper, margin=0.5in}

\title{Introduction to Cryptography: Homework 1}
\author{Alvin Lin}
\date{January 2018 - May 2018}

\begin{document}

\maketitle

\subsection*{Problem 1.5}
As we learned in this chapter, modular arithmetic is the basis of many
cryptosystems. As a consequence, we will address this topic with several
problems in this and upcoming chapters.
\begin{enumerate}
  \item \( 15\cdot29\mod13 = 6 \)
  \item \( 2\cdot29\mod13 = 6 \)
  \item \( 2\cdot3\mod13 = 6 \)
  \item \( -11\cdot3\mod13 = 6 \)
\end{enumerate}
The modulo operation can be applied before the multiplication. The first factor
modulo 13 comes out to 2 for all the problems and the second factor comes out to
3 for all the problems, leaving an answer of 6.

\subsection*{Problem 1.6}
Compute without a calculator:
\begin{enumerate}
  \item \( 1/5\mod13 \equiv 1\cdot8\mod13 \equiv 8 \)
  \item \( 1/5\mod7 \equiv 1\cdot3\mod7 \equiv 3 \)
  \item \( 3\cdot2/5\mod7 = 3\cdot2\cdot3\mod7 \equiv 4 \)
\end{enumerate}

\subsection*{Problem 1.7}
We consider the ring \( \Z_4 \). Construct table which describes the addition of
all the elements in the ring with each other.
\begin{center}
  \begin{tabular}{c|cccc}
    + & 0 & 1 & 2 & 3 \\ \hline
    0 & 0 & 1 & 2 & 3 \\
    1 & 1 & 2 & 3 & 0 \\
    2 & 2 & 3 & 0 & 1 \\
    3 & 3 & 0 & 1 & 2
  \end{tabular}
\end{center}
Construct the multiplication table for \( \Z_4 \):
\begin{center}
  \begin{tabular}{c|cccc}
    \( \times \) & 0 & 1 & 2 & 3 \\ \hline
    0 & 0 & 0 & 0 & 0 \\
    1 & 0 & 1 & 2 & 3 \\
    2 & 0 & 2 & 0 & 2 \\
    3 & 0 & 3 & 2 & 1 \\
  \end{tabular}
\end{center}
Construct the addition and multiplication tables for \( \Z_6 \): \\[1cm]
\begin{minipage}[c]{8cm}
  \begin{center}
    \begin{tabular}{c|cccccc}
      + & 0 & 1 & 2 & 3 & 4 & 5 \\ \hline
      0 & 0 & 1 & 2 & 3 & 4 & 5 \\
      1 & 1 & 2 & 3 & 4 & 5 & 0 \\
      2 & 2 & 3 & 4 & 5 & 0 & 1 \\
      3 & 3 & 4 & 5 & 0 & 1 & 2 \\
      4 & 4 & 5 & 0 & 1 & 2 & 3 \\
      5 & 5 & 0 & 1 & 2 & 3 & 4 \\
    \end{tabular}
  \end{center}
\end{minipage}
\begin{minipage}[c]{8cm}
  \begin{tabular}{c|cccccc}
    \( \times \) & 0 & 1 & 2 & 3 & 4 & 5 \\ \hline
    0 & 0 & 0 & 0 & 0 & 0 & 0 \\
    1 & 0 & 1 & 2 & 3 & 4 & 5 \\
    2 & 0 & 2 & 4 & 0 & 2 & 4 \\
    3 & 0 & 3 & 0 & 3 & 0 & 3 \\
    4 & 0 & 4 & 2 & 0 & 4 & 2 \\
    5 & 0 & 5 & 4 & 3 & 2 & 1 \\
  \end{tabular}
\end{minipage}
There are elements in \( \Z_4 \) and \( \Z_6 \) without a multiplicative
inverse. Which elements are these? Why does a multiplicative inverse exist for
nonzero elements in \( \Z_5 \)? \\
\( 2 \) does not have a multiplicative inverse in \( \Z_4 \). \( 2,3,4 \) do not
have multiplicative inverses in \( \Z_6 \). A multiplicative element exists for
all nonzero elements in \( \Z_5 \) because all nonzero elements are coprime to
5 (Because 5 is a prime number).

\subsection*{Problem 1.8}
What is the multiplicative inverse of 5 in \( \Z_{11}, \Z_{12}, \Z_{13} \)?
\begin{align*}
  5^{-1}\mod11 &= 9 \\
  5^{-1}\mod12 &= 5 \\
  5^{-1}\mod13 &= 8
\end{align*}

\subsection*{Problem 1.9}
Compute \( x \) as far as possible without a calculator. Where appropriate,
make use of a smart decomposition of the exponent.
\begin{enumerate}
  \item \( x = 3^2\mod13 = 9 \)
  \item \( x = 7^2\mod13 = 10 \)
  \item \( x = 3^{10}\mod13 = 3^4\cdot3^4\cdot3^2\mod13 = 3\cdot3\cdot9\mod13 =
    3 \)
  \item \( x = 7^{100}\mod13 = (7^2)^{50}\mod13 = 10^{50}\mod13 =
    100^{25}\mod13 = 9^25\mod13 = (9^2)^{12}\cdot9\mod13 = 3^{12}\cdot9\mod13 =
    (3^4)^3\cdot9\mod13 = 3^3\cdot9\mod13 = 9 \)
  \item \( 7^x = 11\mod13 \)
  \begin{align*}
    7^2\mod13 &= 10 \\
    7^3\mod13 &= 10\cdot7\mod13 = 5 \\
    7^4\mod13 &= 5\cdot7\mod13 = 9 \\
    7^5\mod13 &= 9\cdot7\mod13 = 11
  \end{align*}
\end{enumerate}

\clearpage
\subsection*{Problem 1.11}
This problem deals with the affine cipher with the key parameters \( a = 7, b =
22 \). Decrypt the text below: \\
\texttt{falszztysyjzyjkywjrztyjztyynaryjkyswarztyegyyj} \\
Who write the line? \\
\texttt{firstthesentenceandthentheevidencesaidthequeen} \\
From \textit{Alice In Wonderland} by Lewis Carroll

\begin{center}
  If you have any questions, comments, or concerns, please contact me at
  alvin@omgimanerd.tech
\end{center}

\end{document}
