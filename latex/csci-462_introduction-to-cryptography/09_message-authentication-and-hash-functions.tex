\documentclass{math}

\title{Introduction to Cryptography}
\author{Alvin Lin}
\date{January 2018 - May 2018}

\begin{document}

\maketitle

\section*{Message Authentication and Hash Functions}
\textbf{Message authentication} is a procedure to verify that received messages
from the alleged source have not been altered. Message authentication may also
verify sequencing and timeliness. A \textbf{digital signature} is an
authentication technique that also includes measures to counter repudiation by
the source.

\subsection*{Message Authentication Functions}
\begin{itemize}
  \item Message encryption: the ciphertext of the entire message serves as its
  authenticator.
  \item Hash function: a function that maps a message of any length into a
  fixed-length hash value which serves as the authenticator.
  \item Message authentication code (MAC): a function of the message and a
  secret key that produces a fixed-length value that serves as the
  authenticator.
\end{itemize}

\subsubsection*{Hash Functions}
A hash function \( H \) accepts a variable-length block of data \( M \) as input
and produces a fixed-size hash value. The principal objective of a hash function
is data integrity. Cryptographic hash functions are algorithms for which it is
computationally infeasible to find a data object that maps to a pre-specified
hash result (one-way property), or two data objects that map to the same hash
result (the collision-free property). Hash functions are also commonly used for:
\begin{itemize}
  \item Creating one-way password files
  \item Intrusion detection and virus detection
  \item Constructing a pseudorandom function or pseudorandom number generator
\end{itemize}

\subsubsection*{Message Authentication Code (MAC) Functions}
A message authentication code is also known as a keyed has function. It is
typically used between two parties that share a secret key to authenticate
information exchanges. A MAC function takes a secret key and a data block as
input and produces a hash value referred to as the MAC, which is then associated
with the protected message. An attacker who alters the message will be unable to
alter the associated MAC value without knowledge of the secret key.

\subsubsection*{Digital Signatures}
Digital signatures operate similar to that of a MAC. The hash value of a message
is encrypted with a user's private key. Anyone who knows the user's public key
can verify the integrity of the message. An attack who wishes to alter the
message would need to know the user's private key.

\subsubsection*{Digital Signatures vs MAC}
The combination of hashing and encryption results in an overall function that is
conceptually a MAC. Its goal is to be a function \( E(K,H(M)) \) for a variable
length message \( M \) and a secret key \( K \) to produce a fixed length output
that is secure against an opponent who does not know the secret key. In
practice, specific MAC algorithms are designed to be more efficient that an
encryption algorithm.

\subsection*{Attacks on Hash Functions}
\textbf{Brute Force Attacks} do not depend on the specific algorithm, they only
depend on the bit length. In the case of a hash function, the attack depends
only on the bit length of the hash value. This method involves picking values
and random and hashing each one until a collision occurs. \par
\textbf{Cryptanalysis} attacks are based on weaknesses in a particular
cryptographic algorithm. They seek to exploit some property of the algorithm to
perform some attack other than an exhaustive search. \par
If we have a hash with an output digest length \( n \), we can find a preimage
using a brute force search in \( 2^n \) operations. We can find a second
preimage also with \( 2^n \) operations. Thus we can find a collision with
\( \sqrt{2^n} = 2^{\frac{n}{2}} \) operations using a brute force search
(birthday attack). Collision resistance causes most of the security problems
with hash functions. It is related to the birthday problem.

\subsubsection*{Birthday Paradox}
How many people are needed such that two of them have the same birthday with a
probability of 0.5? Conventionally, one would think \( \frac{365}{2} = 183 \),
but in fact only 23 people are required because the search takes \( \approx
\sqrt{2^n} \) steps. To deal with this paradox, hash functions need an output
size of at least 160 bits.

\subsubsection*{Using Collision Attacks}
For attacks based on hash collisions, an adversary wishes to find two messages
or data blocks that yield the same hash function. The effort required is
explained above by the birthday paradox. The strategy to use is as follows:
\begin{enumerate}
  \item Suppose Alice is prepared to sign a correct message \( X \) by appending
  the appropriate \( n \)-bit hash code and encrypting that hash code with the
  private key.
  \item An opponent generates \( 2^{\frac{n}{2}} \) varations \( X' \) of
  \( X \), all with essentially the same meaning, and stores the messages and
  their hash values.
  \item The opponent prepares a deceptive message \( Y \) for which Alice's
  signature is desired.
  \item The opponent generates minor variations \( Y' \) of \( Y \), all of
  which convey essentially the same meaning. For each \( Y' \), the opponent
  computes \( H(Y') \) and checks for matches with any of the \( H(X') \) until
  a match is found. The process continues until a \( Y' \) is generated with a
  hash value equal to the hash value of one of the \( X' \) values.
  \item The opponent offers the valid message variation to Alice for signing.
  The signature can then be attached to to the fraudulent variation for
  transmission to the intended recipient.
\end{enumerate}
Because the two varations have the same hash code, they will produce the same
signature and the opponent is assured of success even though the encryption key
is not know. The generation of many variations that convey the same meaning is
not difficult. An opponent can insert extra spaces in between words or simply
reword the message while retaining the meaning.

\subsubsection*{Examples of Cryptographic Hash Functions}
\begin{center}
  \begin{tabular}{|c|c|c|c|}
    \hline
    Hash Algorithm & Date & Digest Length (bits) & Security \\
    \hline
    Message Digest 4 (MD4) & 1990 & 128 & Broken \\
    \hline
    Message Digest 5 (MD5) & 1991 & 129 & Broken \\
    \hline
    Secure Hash Algorithm 0 (SHA-0) & 1993 & 160 & Broken \\
    \hline
    Secure Hash Algorithm 1 (SHA-1) & 1995 & 160 & Broken \\
    \hline
    Secure Hash Algorithm 2 (SHA-2) & 2002 & 224,256,384,512 & Questionable \\
    \hline
    Secure Hash Algorithm 3 (SHA-3) & 2015 & 224,256,384,512 & Secure \\
    \hline
  \end{tabular}
\end{center}

\subsection*{Hash Functions Based On Cipher Block Chaining}
A number of proposals have been made for hash functions based on using a
cipher block chaining technique, but without using a secret key. One of the
first proposals was that of Rabin, where we divide a message \( M \) into fixed
sized blocks \( M_1,M_2,\dots,M_n \) and use a symmetric encryption system such
as DES to compute the hash code \( h \) as:
\begin{align*}
  H_0 &= \text{initial value} \\
  H_i &= E(M_i,H_{i-1}) \\
  h &= H_n
\end{align*}
This is similar to the cipher block chaining technique but uses no secret key.
As with any hash code, this scheme is subject to the birthday attack. If the
encryption algorithm is DES and only a 64-bit hash code is produced, the
system is susceptible.

\subsection*{The Merkle-Damg{\aa}rd Construction}
The Merkle-Damg{\aa}rd Construction was invented by Ralph Merkle and Ivan
Damg{\aa}rd working independently. The message to hash is extended with padding
bits (1000,000+) plus the message length. The padded message is fed through an
iterated compression function. In 1989, Merkle and Damg{\aa} proved that if
the compression function is collision-resistance, then the whole hash funcion
will be collision resistance. Because of this proof, most hash functions
proposed since then use the Merkle-Damg{\aa}rd construction or a variation.
Possible attacks:
\begin{itemize}
  \item Distinguishing attack
  \item Length extension attack
  \item Partial-message collision attack
  \item Fixed-point attack
  \item Multicollision attack
  \item Expandable message attack
  \item Herding attack
\end{itemize}

\begin{center}
  You can find all my notes at \url{http://omgimanerd.tech/notes}. If you have
  any questions, comments, or concerns, please contact me at
  alvin@omgimanerd.tech
\end{center}

\end{document}
