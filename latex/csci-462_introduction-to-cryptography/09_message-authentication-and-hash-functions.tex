\documentclass{math}

\title{Introduction to Cryptography}
\author{Alvin Lin}
\date{January 2018 - May 2018}

\begin{document}

\maketitle

\section*{Message Authentication and Hash Functions}
\textbf{Message authentication} is a procedure to verify that received messages
from the alleged source have not been altered. Message authentication may also
verify sequencing and timeliness. A \textbf{digital signature} is an
authentication technique that also includes measures to counter repudiation by
the source.

\subsection*{Message Authentication Functions}
\begin{itemize}
  \item Message encryption: the ciphertext of the entire message serves as its
  authenticator.
  \item Hash function: a function that maps a message of any length into a
  fixed-length hash value which serves as the authenticator.
  \item Message authentication code (MAC): a function of the message and a
  secret key that produces a fixed-length value that serves as the
  authenticator.
\end{itemize}

\subsubsection*{Hash Functions}
A hash function \( H \) accepts a variable-length block of data \( M \) as input
and produces a fixed-size hash value. The principal objective of a hash function
is data integrity. Cryptographic hash functions are algorithms for which it is
computationally infeasible to find a data object that maps to a pre-specified
hash result (one-way property), or two data objects that map to the same hash
result (the collision-free property). Hash functions are also commonly used for:
\begin{itemize}
  \item Creating one-way password files
  \item Intrusion detection and virus detection
  \item Constructing a pseudorandom function or pseudorandom number generator
\end{itemize}

\subsubsection*{Message Authentication Code (MAC) Functions}
A message authentication code is also known as a keyed has function. It is
typically used between two parties that share a secret key to authenticate
information exchanges. A MAC function takes a secret key and a data block as
input and produces a hash value referred to as the MAC, which is then associated
with the protected message. An attacker who alters the message will be unable to
alter the associated MAC value without knowledge of the secret key.

\subsubsection*{Digital Signatures}
Digital signatures operate similar to that of a MAC. The hash value of a message
is encrypted with a user's private key. Anyone who knows the user's public key
can verify the integrity of the message. An attack who wishes to alter the
message would need to know the user's private key.

\subsubsection*{Digital Signatures vs MAC}
The combination of hashing and encryption results in an overall function that is
conceptually a MAC. Its goal is to be a function \( E(K,H(M)) \) for a variable
length message \( M \) and a secret key \( K \) to produce a fixed length output
that is secure against an opponent who does not know the secret key. In
practice, specific MAC algorithms are designed to be more efficient that an
encryption algorithm.

\begin{center}
  You can find all my notes at \url{http://omgimanerd.tech/notes}. If you have
  any questions, comments, or concerns, please contact me at
  alvin@omgimanerd.tech
\end{center}

\end{document}
