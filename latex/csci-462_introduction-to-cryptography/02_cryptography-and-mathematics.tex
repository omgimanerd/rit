\documentclass{math}

\title{Introduction to Cryptography}
\author{Alvin Lin}
\date{January 2018 - May 2018}

\begin{document}

\maketitle

\section*{Cryptography and Mathematics}

\subsection*{Greatest Common Divisor}
The greatest common divisor is the largest number that divides evenly into two
numbers \( a \) and \( b \). In general, we will denote the greatest common
divisor of \( a \) and \( b \) as \( gcd(a,b) \). Formally:
\[ gcd(a,b) = max[k, \text{such that } k|a \text{ and } k|b] \]
Properties:
\begin{itemize}
  \item \( gcd(0,0) = 0 \)
  \item Because we require that the greatest common divisor be positive:
  \[ gcd(a,b) = gcd(a,-b) = gcd(-a,b) = gcd(-a,-b) \]
  \item In general, \( gcd(a,b) = gcd(|a|,|b|) \)
  \item Because all non-zero numbers divide 0, we have \( gcd(a,0) = |a| \).
  \item Two integers \( a \) and \( b \) are \textbf{relatively prime} if their
  only common positive integer factor is 1. This is equivalent to saying that
  \( a \) and \( b \) are relatively prime if \( gcd(a,b) = 1 \).
\end{itemize}

\subsection*{Modular Arithmetic}
Modular arithmetic is extremely important for aymmetric cryptography (RSA,
elliptic curves, etc). Some historical ciphers can be elegantly described with
modular arithmetic. Generally speaking, most cryptosystems are based on sets of
numbers that are discrete and finite.
\par Let \( a,r,m \) be integers and \( m>0 \). We write
\[ a \equiv r\mod(m) \]
if \( (r-a) \) is divisible by \( m \). \( m \) is called the modulus and
\( r \) is called the remainder.
\begin{itemize}
  \item \( 12 \equiv 3\mod(9) \)
  \item \( 34 \equiv 7\mod(9) \)
  \item \( -7 \equiv 2\mid(0) \)
\end{itemize}
The remainder is not unique. It is somewhat surprising that for every given
modulus \( m \) and a number \( a \), there are infinitely many valid
remainders. By convention, we usually agree on the smallest positive integer
\( r \) as the remainder. This integer can be computed as:
\[ a \equiv qm+r \]
where \( 0\le r\le m=1 \). This is just a convention. Algorithmically, we are
free to choose any other valid remainder to compute our crypto functions.
Properties:
\begin{itemize}
  \item \( (a+b)\mod(m) \equiv a\mod(m)+b\mod(m) \)
  \item \( (a\cdot b)\mod(m) \equiv a\mod(m)\cdot b\mod(m) \)
\end{itemize}

\subsubsection*{Modular Division}
Rather than performing a division, we prefer to multiply by the inverse:
\[ \frac{b}{a}\equiv b\times a^{-1}\mod(m) \]
The inverse \( a^{-1} \) is defined such that:
\[ a\times a^{-1} \equiv 1\mod(m) \]
The inverse of a number \( a\mod(m) \) only exists if and only if:
\[ gcd(a,m) = 1 \]
Example:
\[ 5/7\mod(9) \]
the inverse of \( 7\mod(9) \) is 4, since \( 7\times4 \equiv 28 \equiv
1\mod(9) \), hence:
\[ 5/7 \equiv 5\times4 = 20 \equiv 2\mod(9) \]

\subsubsection*{Modular Reduction}
Modular reduction can be performed at any point during a calculation.
\[ 3^8 = 3^4\times3^4 = 81\times81 \equiv 4\times4\mod(7) = 16\mod(7) =
  2\mod(7) \]

\subsection*{An Algebraic View on Modulo Arithmetic: The Ring \( \Z_m \)}
The integer ring \( \Z_m \) has the following properties:
\begin{itemize}
  \item \textbf{Closure:} We can add and multiply any two numbers and the result
  is always in the ring.
  \item \textbf{Addition and multiplication are associative:} For all \( a,b,c
  \in \Z_m \):
  \[ a+(b+c) = (a+b)+c \quad a\times(b\times c) = (a\times b)\times c \]
  and addition is \textbf{commutative}: \( a+b = b+a \)
  \item \textbf{The distributive law holds:} \( a\times(b+c) = (a\times
  b)+(a\times c) \) for all \( a,b,c\in\Z_m \)
  \item There is the \textbf{neutral element 0} with respect to addition. For
  all \( a\in\Z_m, a+0\equiv a\mod(m) \).
  \item For all \( a\in\Z_m \), there is always an \textbf{additive inverse}
  element \( a \) such that \( a+(-a) \equiv 0\mod(m) \).
  \item There is the \textbf{neutral element 1} with respect to multiplication.
  For all \( a\in\Z_m, a\times1 \equiv a\mod(m) \)
  \item The \textbf{multiplicative inverse \( a^{-1} \)} \( a\times a^{-1}
  \equiv 1\mod(m) \) exists only for some, but not for all, elements in
  \( \Z_m \).
\end{itemize}
Roughly speaking, a ring is a structure in which we can always add, subtract,
and multiply, but we can only divide by certain elements (namely by those for
which a multiplicative inverse exists). We consider the ring \( \Z_9 = \{0,1,2,
3,4,5,6,7,8\} \). The elements 0, 3, and 6, do not have inverses since they are
not coprime to 9.

\subsection*{Shift (or Caesar) Cipher}
The shift cipher is an ancient cipher allegedly used by Julius Caesar. To use
it, each plaintext letter is replaced by another one. The replacement rule is
very simple: take the letter that follows after \( k \) positions in the
alphabet. Note that letters wrap around the end of the alphabet, which can be
elegantly expressed as reduction modulo 26.
\begin{itemize}
  \item Encryption: \( y = e_k(x) \equiv x+k\mod(26) \)
  \item Decryption: \( x = d_k(y) \equiv y-k\mod(26) \)
\end{itemize}
The shift cipher is not secure since it is vulnerable to an exhaustive key
search since the key space is only 26. It is also vulnerable to letter
frequency analysis, similar to any attack against substitution ciphers.

\subsection*{Affine Cipher}
The Affine cipher is an extension of the shift cipher, rather than just adding
the key to the plaintext, we also multiply by the key. We use a key consisting
of two parts: \( k = (a,b) \).
\begin{itemize}
  \item Encryption: \( y = e_k(x) \equiv ax+k\mod(26) \)
  \item Decryption: \( x = d_k(y) \equiv a^{-1}(y-k)\mod(26) \)
\end{itemize}
Since the inverse of \( a \) is needed for decryption, we can only use values
for which \( gcd(a,26) = 1 \).

\subsection*{Vigen\'{e}re Cipher}
The Vigen\'{e}re Cipher is one of the simplest and best known polyalphabetic
substituion ciphers. In this scheme, each plaintext letter is Caesar-shifted
by a different amount according to the key, which is usually a repeating
keyword.

\subsection*{The Euclidean Algorithm}
Computing the greatest common divisor of two numbers \( a \) and \( b \) is
easy for small numbers. We can factor both numbers and look for the highest one
that they share in common. Factoring is complicated and often infeasible for
large numbers. The Euclidean algorithm follows from the idea that \( gcd(a,b) =
gcd(a-b,b) \). The core idea is to reduce the problem of finding the gcd of
two numbers to that of the gcd of two smaller numbers. We repeat this process
recursively until we reach \( gcd(b,0) = b \).

\subsubsection*{Example}
\[ gcd(710,310) \]
\begin{align*}
  710 &= 2\times310+90 \\
  310 &= 3\times90+40 \\
  90 &= 2\times40+10 \\
  40 &= 4\times10
\end{align*}
\[ gcd(710,310) = 10 \]

\begin{center}
  You can find all my notes at \url{http://omgimanerd.tech/notes}. If you have
  any questions, comments, or concerns, please contact me at
  alvin@omgimanerd.tech
\end{center}

\end{document}
