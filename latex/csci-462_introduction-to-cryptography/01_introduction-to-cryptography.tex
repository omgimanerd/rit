\documentclass{math}

\title{Introduction to Cryptography}
\author{Alvin Lin}
\date{January 2018 - May 2018}

\begin{document}

\maketitle

\section*{Cryptography}
Desired security properties in the digital world:
\begin{itemize}
  \item confidentiality, secrecy
  \item data integrity
  \item authentication, of data origin and entity
  \item non-repudiation
\end{itemize}
Cryptography is an important, but only a relatively small part of security:
\begin{itemize}
  \item the right choice of tools is hard
  \item implementation errors are common
  \item a variety of side-channel attacks can bypass the best cryptography
  \item social engineering
\end{itemize}

\subsection*{Unkeyed, Symmetric-Key, and Public-Key}
Primitives, algorithms, and protocols can be \textbf{unkeyed},
\textbf{symmetric-key}, or \textbf{public-key}.

\subsubsection*{Unkeyed}
This includes hashing and the SHA-family of algorithms. It can be used for
signing and the generation of random sequences.

\subsubsection*{Symmetric Keys}
\begin{itemize}
  \item Block ciphers since the 1970s: IBM's Lucifer, DES (Data Encryption
  Standard), IDEA (International Data Encryption Standard), AES (Advanced
  Encryption Standard).
  \item Stream ciphers: RC4, also can come from counter mode of block ciphers
  or hash functions.
  \item MAC, HMAC: message authentication codes.
  \item PRNG: pseudorandom number generators
\end{itemize}

\subsubsection*{Public Key}
\begin{itemize}
  \item Public-key cryptosystems: RSA (Rivest, Shamir, Adleman), ElGamal,
  McEliece cryptosystems, ECC (elliptic curve cryptosystems).
  \item Signatures: DSS/DSA (Digital Signature Standard/Algorithm), ECDSA
  (Elliptic Curve Digital Signature Algorithm).
  \item PKI: public key infrastructure, DH (Diffie Hellman key agreement),
  key management, and distribution.
  \item Homomorphic cryptography: Paillier, Gentry
\end{itemize}

\subsection*{Main Public-Key Systems in Use}
RSA by Rivest-Shamir-Adleman (1977) has an edge of ECC because:
\begin{itemize}
  \item it is simple and well understood
  \item links nicely to basic number theory
  \item deployed earlier on many systems
\end{itemize}
ECC by Koblitz-Miller (1985) has an edge over RSA because:
\begin{itemize}
  \item it uses short keys (163+ bits vs 1024+ bits for RSA)
  \item delivers much better performance
  \item uses great theory of elliptic curves on top of classical number theory
  used by RSA
\end{itemize}

\subsection*{Other composite and special functionalities}
\begin{itemize}
  \item Zero-knowledge protocols
  \item Authenticated encryption CAESAR competition
  \item Electronic cash: untraceable, no double-spending, bank-shop-customer
  roles, central banks
  \item Cryptocurrencies: Bitcoin, Zerocoin, Litecoin, Darkcoin, etc
  \item Electronic voting: no individual vote auditing
  \item Oblivious transfer: two millionaires problem
  \item Quantum and post-quantum cryptography, quantum key distribution,
  quantum computing
\end{itemize}

\begin{center}
  You can find all my notes at \url{http://omgimanerd.tech/notes}. If you have
  any questions, comments, or concerns, please contact me at
  alvin@omgimanerd.tech
\end{center}

\end{document}
