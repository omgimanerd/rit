\documentclass{math}

\usepackage{enumerate}
\usepackage{tikz}

\geometry{letterpaper, margin=0.5in}

\title{Boundary Value Problems: Homework 5}
\author{Alvin Lin}
\date{August 2018 - December 2018}

\begin{document}

\maketitle

\subsection*{Problem 1}
Exercises 3.2: problems 1, 2, 4, 5, 13

\subsubsection*{Exercise 1}
Sketch the graph of \( f \), determine the Fourier series corresponding to
\( f \), and indicate the convergence at the given points. It is assumed that
the functions are periodic and one period is given.
\[ f(x) = \begin{cases}
  -2 &, -2<x<0 \\
  2 &, 0<x<2
\end{cases} \]
\begin{center}
  \begin{tikzpicture}
    \draw[thick,<->] (-3,0) -- (3,0) node[below] {\( x \)};
    \draw[thick,<->] (0,-3) -- (0,3) node[left] {\( y \)};
    \draw[very thick,red] (-2,-2) -- (0,-2);
    \draw[very thick,red] (0,2) -- (2,2);
  \end{tikzpicture}
\end{center}
\begin{align*}
  a_0 &= \frac{1}{2}\int_{-2}^{2}f(x)\diff{x} \\
  &= \frac{1}{2}\bigg[\int_{-2}^{0}-2\diff{x}+\int_{0}^{2}2\diff{x}\bigg] \\
  &= 0 \\
  a_n &= \frac{1}{2}\int_{-2}^{2}f(x)\cos(\frac{n\pi x}{2})\diff{x} \\
  &= 0
\end{align*}
\( f(x) \) is characteristically odd, and since \( \cos \) is an even function,
the resulting integral is odd.
\begin{align*}
  b_n &= \frac{1}{2}\int_{-2}^{2}f(x)\sin(\frac{n\pi x}{2})\diff{x} \\
  &= \frac{1}{2}\bigg[\int_{-2}^{0}-2\sin(\frac{n\pi x}{2})\diff{x}+
    \int_{0}^{2}2\sin(\frac{n\pi x}{2})\diff{x}\bigg] \\
  &= \frac{1}{2}\left(
    \bigg[\frac{2}{n\pi}2\cos(\frac{n\pi x}{2})\bigg]_{-2}^{0}-
    \bigg[\frac{2}{n\pi}2\cos(\frac{n\pi x}{2})\bigg]_{0}^{2}\right) \\
  &= \frac{2}{n\pi}\cos(-n\pi)-\frac{2}{n\pi}\cos(0)-
    \frac{2}{n\pi}\cos(n\pi)+\frac{2}{n\pi}\cos(0) \\
  &= 0 \\
  f(x) &\sim 0
\end{align*}
At the point \( x = 0 \),
\[ \frac{f(0^-)+f(0^+)}{2} = \frac{-2+2}{2} = 0 \]

\subsubsection*{Exercise 2}
Sketch the graph of \( f \), determine the Fourier series corresponding to
\( f \), and indicate the convergence at the given points. It is assumed that
the functions are periodic and one period is given.
\[ f(x) = \begin{cases}
  0 &, -2<x<-1 \\
  2 &, -1<x<1 \\
  0 &, 1<x<2
\end{cases} \]
\begin{center}
  \begin{tikzpicture}
    \draw[thick,<->] (-3,0) -- (3,0) node[below] {\( x \)};
    \draw[thick,<->] (0,-3) -- (0,3) node[left] {\( y \)};
    \draw[very thick,red] (-2,0) -- (-1,0);
    \draw[very thick,red] (-1,2) -- (1,2);
    \draw[very thick,red] (1,0) -- (2,0);
  \end{tikzpicture}
\end{center}
\begin{align*}
  a_0 &= \frac{1}{2}\int_{-2}^{2}f(x)\diff{x} \\
  &= \frac{1}{2}\left(\int_{-2}^{-1}0\diff{x}+\int_{-1}^{1}2\diff{x}+
    \int_{1}^{2}0\diff{x}\right) \\
  &= \frac{1}{2}\int_{-1}^{1}2\diff{x} \\
  &= \frac{4}{2} \\
  &= 2 \\
  a_n &= \frac{1}{2}\int_{-2}^{2}f(x)\cos(\frac{n\pi x}{2})\diff{x} \\
  &= \frac{1}{2}\int_{-1}^{1}2\cos(\frac{n\pi x}{2})\diff{x} \\
  &= \frac{1}{2}\bigg[\frac{2}{n\pi}(-2\sin(\frac{n\pi x}{2}))\bigg]_{-1}^{1} \\
  &= \frac{1}{2}\bigg[\frac{-4}{n\pi}\sin(\frac{n\pi}{2})+
    \frac{4}{n\pi}\sin(\frac{-n\pi}{2})\bigg] \\
  &= \frac{1}{2}\bigg[\frac{-4}{n\pi}\sin(\frac{n\pi}{2})-
    \frac{4}{n\pi}\sin(\frac{n\pi}{2})] \\
  &= -\frac{4}{n\pi}\sin(\frac{n\pi}{2}) \\
  b_n &= \frac{1}{2}\int_{-2}^{2}f(x)\sin(\frac{n\pi x}{2})\diff{x} \\
  &= \frac{1}{2}\int_{-1}^{1}2\sin(\frac{n\pi x}{2})\diff{x} \\
  &= 0 \\
  f(x) &\sim \frac{2}{2}+\sum_{n=1}^{\infty}
    -\frac{4}{n\pi}\sin(\frac{n\pi}{2})\cos(\frac{n\pi x}{2}) \\
  &\sim 1+\sum_{n=1}^{\infty}
    -\frac{4}{n\pi}\sin(\frac{n\pi}{2})\cos(\frac{n\pi x}{2})
\end{align*}
At the point \( x = -1 \),
\[ \frac{f(-1^-)+f(-1^+)}{2} = \frac{0+2}{2} = 1 \]

\subsubsection*{Exercise 3}
Sketch the graph of \( f \), determine the Fourier series corresponding to
\( f \), and indicate the convergence at the given points. It is assumed that
the functions are periodic and one period is given.
\[ f(x) = x, \quad -1<x<1 \]
\begin{center}
  \begin{tikzpicture}
    \draw[thick,<->] (-3,0) -- (3,0) node[below] {\( x \)};
    \draw[thick,<->] (0,-3) -- (0,3) node[left] {\( y \)};
    \draw[very thick,red] (-1,-1) -- (1,1);
  \end{tikzpicture}
\end{center}
\begin{align*}
  a_0 &= \frac{1}{1}\int_{-1}^{1}x\diff{x} \\
  &= 0 \\
  a_n &= \int_{-1}^{1}x\cos(\frac{n\pi x}{1})\diff{x} \\
  u = x \quad \diff{u} &= 1 \quad \diff{v} = \cos(n\pi x) \quad
    v = \frac{1}{n\pi}\sin(n\pi x) \\
  &= \bigg[\frac{1}{n\pi}x\sin(n\pi x)\bigg]_{-1}^{1}-
    \int_{-1}^{1}\frac{1}{n\pi}\sin(n\pi x)\diff{x} \\
  &= \bigg[\frac{1}{n\pi}\sin(n\pi)+\frac{1}{n\pi}\sin(-n\pi)\bigg]+
    \frac{1}{n\pi}\bigg[\frac{1}{n\pi}\cos(n\pi x)\bigg]_{-1}^{1} \\
  &= 0-\frac{1}{n^2\pi^2}\bigg[\cos(n\pi)-\cos(-n\pi)\bigg] \\
  &= 0 \\
  b_n &= \int_{-1}^{1}x\sin(\frac{n\pi x}{1})\diff{x} \\
  u = x \quad \diff{u} &= 1 \quad \diff{v} = \sin(n\pi x) \quad
    v = -\frac{1}{n\pi}\cos(n\pi x) \\
  &= \bigg[-\frac{x}{n\pi}\cos(n\pi x)\bigg]_{-1}^{1}-
    \int_{-1}^{1}\frac{1}{n\pi}\cos(n\pi x)\diff{x} \\
  &= \bigg[-\frac{1}{n\pi}\cos(n\pi)-\frac{1}{n\pi}\cos(-n\pi)\bigg]-
    \frac{1}{n\pi}\bigg[\frac{1}{n\pi}\sin(n\pi x)\bigg]_{-1}^{1} \\
  &= -\frac{2\cos(n\pi)}{n\pi}-
    \frac{1}{n^2\pi^2}\bigg[\sin(n\pi)-\sin(-n\pi)\bigg] \\
  &= -\frac{2\cos(n\pi)}{n\pi}-\frac{2\sin(n\pi)}{n^2\pi^2} \\
  f(x) &\sim 0+\sum_{n=1}^{\infty}(-\frac{2\cos(n\pi)}{n\pi}-
    \frac{2\sin(n\pi)}{n^2\pi^2})\sin(n\pi x) \\
  &\sim \sum_{n=1}^{\infty}-\frac{2\cos(n\pi)\sin(n\pi x)}{n\pi}-
    \frac{2\sin(n\pi)\sin(n\pi x)}{n^2\pi^2}
\end{align*}
At the point \( x = 1 \),
\[ \frac{f(1^-)+f(1^+)}{2} = \frac{1+(-1)}{2} = 0 \]

\subsubsection*{Exercise 4}
Sketch the graph of \( f \), determine the Fourier series corresponding to
\( f \), and indicate the convergence at the given points. It is assumed that
the functions are periodic and one period is given.
\[ f(x) = \begin{cases}
  0 &, -2<x<-1 \\
  2+x &, -1<x<0 \\
  2-x &, 0<x<1 \\
  0 &, 1<x<2
\end{cases} \]
\begin{center}
  \begin{tikzpicture}
    \draw[thick,<->] (-3,0) -- (3,0) node[below] {\( x \)};
    \draw[thick,<->] (0,-3) -- (0,3) node[left] {\( y \)};
    \draw[very thick,red] (-2,0) -- (-1,0);
    \draw[very thick,red] (-1,1) -- (0,2) -- (1,1);
    \draw[very thick,red] (1,0) -- (2,0);
  \end{tikzpicture}
\end{center}
\begin{align*}
  a_0 &= \frac{1}{2}\int_{-2}^{2}f(x)\diff{x} \\
  &= \frac{1}{2}\bigg[\int_{-2}^{1}0\diff{x}+\int_{-1}^{0}2+x\diff{x}+
    \int_{0}^{1}2-x\diff{x}+\int_{1}^{2}0\diff{x}\bigg] \\
  &= \frac{1}{2}\bigg[0+\frac{3}{2}+\frac{3}{2}+0\bigg] \\
  &= \frac{3}{2} \\
  a_n &= \frac{1}{2}\int_{-2}^{2}f(x)\cos(\frac{n\pi x}{2})\diff{x} \\
  &= \frac{1}{2}\bigg[\int_{-1}^{0}(2+x)\cos(\frac{n\pi x}{2})+
    \int_{0}^{1}(2-x)\cos(\frac{n\pi x}{2})\bigg] \\
  &= \frac{1}{2}\bigg[
    \frac{2(2+x)}{n\pi}\sin(\frac{n\pi x}{2})\bigg|_{-1}^0+
    \int_{-1}^{0}\frac{2}{n\pi}\sin(\frac{n\pi x}{2})\diff{x}+ \\
  & \quad \quad \frac{2(2-x)}{n\pi}\sin(\frac{n\pi x}{2})\bigg|_{0}^{1}-
    \int_{0}^{1}\frac{2}{n\pi}\sin(\frac{n\pi x}{2})\diff{x}
  \bigg] \\
  &= \frac{1}{2}\bigg[\frac{-2}{n\pi}\sin(\frac{-n\pi}{2})+1+
    \frac{2}{n\pi}\sin(\frac{n\pi}{2})-0\bigg] \\
  &= \frac{2}{n\pi}\sin(\frac{n\pi}{2})+1
\end{align*}
\begin{align*}
  b_n &= \frac{1}{2}\int_{-2}^{2}f(x)\sin(\frac{n\pi x}{2})\diff{x} \\
  &= 0
\end{align*}
\( f(x) \) is characteristically even and \( \sin \) is odd, so over a
symmetric interval, the integral of their product is 0.
\[ f(x) \sim \frac{2}{3}+\sum_{n=1}^{\infty}
  (\frac{2}{n\pi}\sin(\frac{n\pi}{2})+1)\cos(\frac{n\pi x}{2}) \]
At the point \( x = 1 \),
\[ \frac{f(1^-)+f(1^+)}{2} = \frac{0+1}{2} = \frac{1}{2} \]
At the point \( x = 2 \),
\[ \frac{f(2^-)+f(2^+)}{2} = \frac{0+0}{2} = 0 \]

\subsubsection*{Exercise 5}
Determine the Fourier series corresponding to \( f \), and indicate the
convergence at the given points. It is assumed that the functions are periodic
and one period is given.
\[ f(x) = \e^{-x} \quad -1<x<1 \]
\begin{align*}
  a_0 &= \frac{1}{1}\int_{-1}^{1}f(x)\diff{x} \\
  &= \int_{-1}^{1}\e^{-x}\diff{x} \\
  &= \bigg[-\e^{-x}\bigg]_{-1}^{1} \\
  &= -\e^{-1}+\e^{1} \\
  &= \e-\frac{1}{\e} \\
  a_n &= \int_{-1}^{1}\e^{-x}\cos(\frac{n\pi x}{1})\diff{x} \\
  \bigg[\int\e^{bx}\cos(ax)\diff{x} &=
    \frac{1}{a^2+b^2}\e^{bx}(a\sin(ax)+b\cos(ax))\bigg] \\
  &= \bigg[\frac{1}{1+n^2\pi^2}\e^{-x}(n\pi\sin(n\pi x)-
    \cos(n\pi x))\bigg]_{-1}^{1} \\
  &= \frac{\e^{-1}(n\pi\sin(n\pi)-\cos(-n\pi))}{1+n^2\pi^2}-
    \frac{\e^{1}(n\pi\sin(-n\pi)-\cos(-n\pi))}{1+n^2\pi^2} \\
  &= \frac{-\e^{-1}\cos(-n\pi)+\e\cos(-n\pi)}{1+n^2\pi^2} \\
  &= \frac{\cos(n\pi)(\e-\frac{1}{\e})}{1+n^2\pi^2}
\end{align*}
\begin{align*}
  b_n &= \int_{-1}^{1}\e^{-x}\sin(\frac{n\pi x}{1})\diff{x} \\
  \bigg[\int\e^{bx}\sin(ax)\diff{x} &=
    \frac{1}{a^2+b^2}\e^{bx}(b\sin(ax)-a\cos(ax))\bigg] \\
  &= \bigg[\frac{1}{1+n^2\pi^2}\e^{-x}(-\sin(n\pi x)-
    n\pi\cos(n\pi x))\bigg]_{-1}^{1} \\
  &= \frac{\e^{-1}(-\sin(n\pi)-n\pi\cos(n\pi))}{1+n^2\pi^2}-
    \frac{\e(-\sin(-n\pi)-n\pi\cos(-n\pi))}{1+n^2\pi^2} \\
  &= \frac{-e^{-1}n\pi\cos(n\pi)+\e n\pi\cos(n\pi)}{1+n^2\pi^2} \\
  &= \frac{n\pi\cos(n\pi)(\e-\frac{1}{\e})}{1+n^2\pi^2} \\
  f(x) &\sim \frac{1}{\e-\frac{1}{\e}}+\sum_{n=1}^{\infty}
    \frac{(\e-\frac{1}{\e})\cos(n\pi)\cos(n\pi x)}{1+n^2\pi^2}+
    \frac{(\e-\frac{1}{\e})n\pi\cos(n\pi)\sin(n\pi x)}{1+n^2\pi^2}
\end{align*}
Find the convergence at \( x = 1 \) and \( x = -1 \).
\[ \frac{f(1^-)+f(1^+)}{2} = \frac{\e+\e^{-1}}{2} \]
\[ \frac{f(-1^-)+f(-1^+)}{2} = \frac{\e+\e^{-1}}{2} \]

\subsubsection*{Exercise 13}
If \( f(x) = x \) for \( -\pi<x<\pi \), find the Fourier series for the
function. If the series represents \( f(x) \) on the given interval show
graphically the function represented by the series for all \( x \).
\begin{align*}
  a_0 &= \frac{1}{\pi}\int_{-\pi}^{\pi}x\diff{x} \\
  &= 0 \\
  a_n &= \frac{1}{\pi}\int_{-\pi}^{\pi}x\cos(\frac{n\pi x}{\pi})\diff{x} \\
  &= 0 \\
  b_n &= \frac{1}{\pi}\int_{-\pi}^{\pi}x\sin(\frac{n\pi x}{\pi})\diff{x} \\
  u = x \quad \diff{u} &= \diff{x} \quad \diff{v} = \sin(nx) \quad
    v = \frac{1}{n}(-\cos(nx)) \\
  &= \bigg[\frac{-x\cos(nx)}{n}\bigg]_{-\pi}^{\pi}-
    \int_{-\pi}^{\pi}\frac{-\cos(nx)}{n} \\
  &= \bigg[-\frac{\pi\cos(n\pi)}{n}-\frac{\pi\cos(n\pi)}{n}\bigg]-
    \bigg[\frac{-\sin(nx)}{n^2}\bigg]_{-\pi}^{\pi} \\
  &= \frac{-2\pi\cos(n\pi)}{n}-\bigg[0-0\bigg] \\
  &= \frac{-2\pi\cos(n\pi)}{n}
\end{align*}
\begin{center}
  \begin{tikzpicture}
    \draw[thick,<->] (-10,0) -- (10,0) node[below] {\( x \)};
    \draw[thick,<->] (0,-3) -- (0,3) node[left] {\( y \)};
    \draw[very thick,red] (-3.14,-3.14) -- (3.14,3.14);
    \draw[very thick,red] (-9.42,-3.14) -- (-3.14,3.14);
    \draw[very thick,red] (3.14,-3.14) -- (9.42,3.14);
  \end{tikzpicture}
\end{center}

\begin{center}
  If you have any questions, comments, or concerns, please contact me at
  alvin@omgimanerd.tech
\end{center}

\end{document}
