\documentclass{math}

\usepackage{tikz}

\title{Boundary Value Problems}
\author{Alvin Lin}
\date{August 2018 - December 2018}

\begin{document}

\maketitle

\section*{Laplace's Equation}
In one dimension, the heat equation looks like:
\[ u_t = a^2u_{xx} \]
We will see that the heat equation in two dimensions looks like:
\[ u_t = a^2(u_{xx}+u_{yy}) \]
We can reach a steady state by letting \( t\to\infty \) and \( u_t\to0 \).
The steady state temperature distribution is given by Laplace's equations:
\[ u_{xx}+u_{xx} = 0 \]
This is an elliptic partial differential equation, which can be written with
the shorthand \( \triangle u = 0 \). In polar coordinates, this can be written
as:
\[ \triangle u = \triangledown^2 u =
  u_{rr}+\frac{1}{r}u_r+\frac{1}{r^2}u_{\theta\theta} \]

\subsection*{Boundary Conditions}
Different boundary conditions are possible if \( u(x,y) \) is a steady state
temperature distribution on a rectangular plate. We can have homogeneous
Dirichlet boundary conditions on the vertical and horizontal edges.
\begin{center}
  \begin{tikzpicture}
    \draw[thick,->] (-1,0) -- (6,0) node[above]{\( x \)};
    \draw[thick,->] (0,-1) -- (0,6) node[above]{\( y \)};
    \draw (4,0) --
      node[right]{\( u(a,y) = 0 \)} (4,4) --
      node[above]{\( u(x,b) = 0 \)} (0,4) --
      node[left]{\( u(0,y) = 0 \)} (0,0) --
      node[below]{\( u(x,0) = f(x) \)} (4,0);
    \node[below] at (4,0) {\( a \)};
    \node[left] at (0,4) {\( b \)};
  \end{tikzpicture}
\end{center}
If we insulate the left and right edges of the plate, we can also have
homogeneous Neumann boundary conditions.
\begin{center}
  \begin{tikzpicture}
    \draw[thick,->] (-1,0) -- (6,0) node[above]{\( x \)};
    \draw[thick,->] (0,-1) -- (0,6) node[above]{\( y \)};
    \draw (4,0) --
      node[right]{\( \pdiff{u}{x}(a,y) = 0 \)} (4,4) --
      node[above]{\( u(x,b) = 0 \)} (0,4) --
      node[left]{\( \pdiff{u}{x}(0,y) = 0 \)} (0,0) --
      node[below]{\( u(x,0) = f(x) \)} (4,0);
    \node[below] at (4,0) {\( a \)};
    \node[left] at (0,4) {\( b \)};
  \end{tikzpicture}
\end{center}

\subsection*{Solution}
The approach to solving this is still generally the same. Suppose we have the
boundary value problem:
\begin{align*}
  u_{xx}+u_{yy} &= 0 \\
  u_x(0,y) &= u_x(a,y) = 0 \\
  u(x,b) &= 0 \\
  u(x,0) &= f(x)
\end{align*}
We start with the separation of variables by substituting \( u(x,y) =
X(x)Y(y) \) to obtain two ordinary differential equations.
\begin{align*}
  u_{xx}+u_{yy} &= 0 \\
  X''Y+XY'' &= 0 \\
  \frac{X''}{X} &= \frac{-Y''}{Y} = -\lambda = -\alpha^2 \\
  X''+\lambda X &= 0 \\
  Y''-\lambda Y &= 0
\end{align*}
We can then solve the eigenvalue problem by going through all three cases of
\( \lambda \). Since we have done this before, we know the following:
\begin{align*}
  \lambda_n &= \left(\frac{n\pi}{a}\right)^2 \quad n = 0,1,2,\dots \\
  X_n(x) &= c_n\cos\left(\frac{n\pi x}{a}\right)
\end{align*}
With \( \lambda \), we can solve the other ordinary differential equation.
\begin{align*}
  Y''-\lambda Y &= 0 \\
  \lambda_0 &= 0 \\
  X_0 &= c_0 \\
  Y_0(y) &= A_0+B_0y \\
  \lambda_n &= \left(\frac{n\pi}{a}\right)^2 \\
  Y_n(y) &= a_n\e^{\sqrt{\lambda}ny}+b_n\e^{-\sqrt{\lambda}ny} \\
  &= a_n\cosh(\frac{n\pi y}{a})+b_n\sinh(\frac{n\pi y}{a}) \\
  u(x,b) &= X(x)Y(b) = 0 \\
  Y(b) &= 0 \\
  a_n &= -b_n\tanh(\frac{n\pi b}{a}) \\
  Y_n(y) &= -b_n\tanh(\frac{n\pi b}{a})\cosh(\frac{n\pi y}{a})+
    b_n\sinh(\frac{n\pi y}{a}) \\
  &= d_n\sinh(\frac{n\pi}{a}(y-b))
\end{align*}
Now we can substitute our solutions for both ordinary differential equations
back to get \( u(x,y) \).
\begin{align*}
  u_n(x,y) &= e_n\cos(\frac{n\pi x}{a})\sinh(\frac{n\pi}{a}(y-b)) \\
  u(x,y) &= X_0Y_0+\sum X_nY_n \\
  &= e_0(y-b)+\sum_{n=1}^{\infty}e_n\cos(\frac{n\pi x}{a})
    \sinh(\frac{n\pi}{a}(y-b)) \\
  e_0 &= -\frac{1}{ab}\int_{0}^{a}f(x)\diff{x} \\
  e_n &= \frac{2}{a\sinh(\frac{-n\pi b}{a})}
    \int_{0}^{a}f(x)\cos(\frac{n\pi x}{a})\diff{x}
\end{align*}

\subsection*{Polar Coordinates}
In polar coordinates, the Laplacian of \( u(r,\theta) \) is
\[ \triangle u =
  \pdiff{^2u}{r^2}+\frac{1}{2}\pdiff{u}{r}+\frac{1}{r^2}\pdiff{^2u}{\theta^2} \]
Suppose we have the following Dirichlet boundary value problem on a 2D disk
shaped domain.
\begin{align*}
  u_{rr}+\frac{u_r}{r}+\frac{u_{\theta\theta}}{r^2} &= 0 \quad 0\le r<a \quad
    -\pi\le\theta\le\pi \\
  u(r,-\pi) &= u(r,\pi) \quad 0\le r<a \\
  u_{\theta}(r,-\pi) &= u_{\theta}(r,\pi) \quad 0\le r<a \\
  u(a,\theta) &= f(\theta) \quad -\pi<\theta<\pi
\end{align*}
We have to impose the additional constraint that \( u(r,\theta) \) is bounded
on the domain. Like before, we begin with separation of variables.
\[ u(r,\theta) = R(r)T(\theta) \]
This yields the following ordinary differential equations and boundary
conditions.
\begin{align*}
  T''(\theta)+\lambda T'(\theta) &= 0 \\
  T(\pi) &= T(\pi) \\
  T'(-\pi) &= T'(\pi) \\
  r^2R''(r)+rR'(r)-\lambda R(r) &= 0
\end{align*}
The solution to the eigenvalue problem presented here is
\begin{align*}
  \lambda_n &= n^2 \quad n = 0,1,2,\dots \\
  T_0(\theta) &= B_0 \\
  T_n(\theta) &= A_n\cos(n\theta)+B_n\sin(n\theta) \quad n = 1,2,\dots
\end{align*}
The other ordinary differential equation (\( r^2R''(r)+rR'(r)-\lambda R(r) =
0 \)) is known as a Cauchy-Euler equation. We cannot use auxiliary equations to
solve this because of the variable coefficients.

\subsubsection*{Case \( n = 0 \):}
Suppose we take the case of
\( \lambda_0 \):
\begin{align*}
  \lambda_0 &= 0 \\
  r^2R''+rR' &= 0 \\
  Let: y &= \ddiff{R}{r} = R' \\
  r^2y'+ry &= 0 \\
  ry'+y &= 0 \\
  \ddiff{y}{r} &= -\frac{y}{r} \\
  \frac{\diff{y}}{y} &= -\frac{\diff{r}}{r} \\
  y &= R' = \e^{-\ln(r)+c} \\
  &= C\e^{-\ln(r)} = \frac{C}{r} \\
  \ddiff{R}{r} &= \frac{C}{r} \\
  \int\diff{R} &= \int\frac{C}{r}\diff{r} \\
  R_0(r) &= C\ln(r)+d
\end{align*}
Since we want a reasonable bounded solution, we can apply the constraint
\( |u(r,\theta)| < \infty \) for \( 0\le r\le a \) and \( -\pi\le\theta<\pi \).
Since \( \ln(r) \) approaches \( -\infty \) at 0, we set \( D_0 \) to 0 since
it doesn't make sense for our circular disk to have infinite temperature at
its center.
\[ R_0(r) = C_0 \]

\subsubsection*{Case \( n > 0 \):}
Suppose we take \( n = 1,2,3,\dots (n\in\N) \). We can use a test solution
\( R(r) = r^m \) to try and obtain an \( m \) that satifies the differential
equation.
\begin{align*}
  r^2R''+rR'-n^2R &= 0 \\
  R(r) &= r^m \\
  R'(r) &= mr^{m-1} \\
  R''(R) &= m(m-1)r^{m-2} \\
  r^2(m(m-1)r^{m-2})+r(mr^{m-1})-n^2r^m &= 0 \\
  (m(m-1)+m-n^2)r^m &= 0 \\
  m^2-m+m-n^2 &= 0 \\
  m^2 &= n^2 \\
  m &= \pm n
\end{align*}
The general solution can be put in the form:
\[ R(r) = C_nr^n+D_nr^{-n} \]
If \( R(r) \) is bounded, then we must set \( D_n = 0 \), otherwise \( r^{-n} \)
goes to infinity as \( r \) goes to 0, which also does not make sense in the
context of the problem.
\[ R(r) = C_nr^n \]
With this, we can assemble the modes and construct our solution for
\( u_n(r,\theta) \). For \( n = 0 \):
\begin{align*}
  u_0(r,\theta) &= R_0(r)T_0(\theta) \\
  &= C_0B_0 \\
  &= \frac{a_0}{2} \quad \text{renamed for convenience}
\end{align*}
For all other \( n\in\N \):
\begin{align*}
  u_n(r,\theta) &= R_nT_n \\
  &= C_nr^n(A_n\cos(n\theta)+B_n\sin(n\theta)
\end{align*}
We rename variables to put this in the form of a Fourier Series solution:
\begin{align*}
  C_nA_n &= \frac{a_n}{a^n} \\
  C_nB_n &= \frac{b_n}{a^n} \\
  u(r,\theta) &= \frac{a_0}{2}+\sum_{n=1}^{\infty}(\frac{r}{a})^n
    (a_n\cos(n\theta)+b_n\sin(n\theta))
\end{align*}
If we apply the final boundary condition:
\begin{align*}
  u(a,\theta) &= f(\theta) =
    \frac{a_0}{2}+\sum_{n=1}^{\infty}(\frac{a}{a})^n
      a_n\cos(n\theta)+b_n\sin(n\theta) \\
  &= \frac{a_0}{2}+\sum_{n=1}^{\infty}a_n\cos(n\theta)+b_n\sin(n\theta)
\end{align*}
This matches the form of a standard Fourier series solution.

\subsection*{Heat Equation in 2D}
\begin{center}
  \begin{tikzpicture}
    \draw[thick,->] (-1,0) -- (6,0) node[above]{\( x \)};
    \draw[thick,->] (0,-1) -- (0,6) node[above]{\( y \)};
    \draw (4,0) --  (4,4) -- (0,4) -- (0,0) -- (4,0);
    \node[below] at (4,0) {\( L \)};
    \node[left] at (0,4) {\( W \)};
    \node[below] at (2,2) {\( u(x,y,t) \)};
  \end{tikzpicture}
\end{center}
If \( u(x,y,t) \) is 0 on all four boundaries and defined by the initial
temperature \( u(x,y,0) = f(0) \), then we have the following boundary
value problem.
\begin{align*}
  u_t &= \beta\triangle u = \beta(u_{xx}+u_{yy}) \\
  u(0,y,t) &= 0 \\
  u(L,y,t) &= 0 \\
  u(x,0,t) &= 0 \\
  u(x,W,t) &= 0 \\
  u(x,y,0) &= f(x)
\end{align*}
Again, we begin with separation of variables obtain ordinary differential
equations. We will have to employ it twice here to obtain three ordinary
differential equations.
\begin{align*}
  u(x,y,t) &= X(x)Y(y)T(t) \\
  u_t &= \beta(u_{xx}+u_{yy}) \\
  XYT' &- \beta(X''YT+XY''T) \\
  \frac{T'}{\beta T} &= \frac{X''}{X}+\frac{Y''}{Y} = -\lambda \\
  T'+\beta\lambda T &= 0 \\
  \frac{X''}{X}+\frac{Y''}{Y} &= -\lambda \\
  \frac{X''}{X} &= -\mu \quad X''+\mu X = 0 \\
  -\frac{Y''}{Y}-\lambda &= -\mu \quad Y''+(\lambda-\mu)Y = 0
\end{align*}
If we substitute and solve for the boundary conditions for our ordinary
differential equations:
\begin{align*}
  X(0) &= X(L) = 0 \\
  Y(0) &= Y(W) = 0
\end{align*}
We've solved these eigenproblems before:
\begin{align*}
  \mu_m &= (\frac{m\pi}{L})^2 \quad m\in\N \\
  X_m(x) &= c_m\sin(\frac{m\pi x}{L}) \\
  Let: \epsilon &= \lambda-\mu \\
  \epsilon_n &= (\frac{n\pi}{W})^2 \quad n\in\N \\
  Y_n(y) &= a_n\sin(\frac{n\pi y}{W}) \\
  \lambda_{mn} &= (\frac{n\pi}{W})^2+(\frac{m\pi}{L})^2 \\
  T' &= -\beta\lambda T \\
  T(t) &= b\e^{-\beta\lambda t} \\
  T_{mn}(t) &= b_{mn}\e^{-\beta(\frac{m^2}{L^2}+\frac{n^2}{W^2})\pi^2t}
\end{align*}
We can add the modes into a superposition as the solution and apply the
initial condition.
\begin{align*}
  u_{mn}(x,y,t) &= X_m(x)Y_n(y)T_{mn}(t) \\
  u(x,y,t) &= \sum_{m=1}^{\infty}\sum_{n=1}^{\infty}u_{mn}(x,y,t) \\
  u(x,y,0) &= f(x,y) \\
  &= \sum_{m=1}^{\infty}\sum_{n=1}^{\infty}a_{mn}
    \sin(\frac{m\pi x}{L})\sin(\frac{n\pi y}{W})
\end{align*}
This takes the form of a Double Fourier Series solution. If we solve for
\( a_{mn} \) using the Double Fourier Series formulas:
\[ a_{mn} = \frac{4}{LW}\int_{0}^{L}\int_{0}^{W}f(x,y)\sin(\frac{m\pi x}{L})
  \sin(\frac{n\pi y}{W})\diff{x}\diff{y} \]

\begin{center}
  You can find all my notes at \url{http://omgimanerd.tech/notes}. If you have
  any questions, comments, or concerns, please contact me at
  alvin@omgimanerd.tech
\end{center}

\end{document}
