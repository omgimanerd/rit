\documentclass{math}

\usepackage{enumerate}

\geometry{letterpaper, margin=0.5in}

\title{Boundary Value Problems: Homework 3}
\author{Alvin Lin}
\date{August 2018 - December 2018}

\begin{document}

\maketitle

\subsection*{Problem 1}
Exercises 2.1: problems 4-7

\subsubsection*{Exercise 4}
Show that the set of functions \( \{\sin(n\pi x), -1<x<1, n\in\N \) is
orthogonal.
\begin{align*}
  n&\in\N,m\in\N,n\ne m \\
  (f_n,f_m) &= \int_{-1}^{1}\sin(m\pi x)\sin(n\pi x)\diff{x} \\
  &= \int_{-1}^{1}\frac{1}{2}\left(\cos(m\pi x-n\pi x)-
    \cos(m\pi x+n\pi x)\right)\diff{x} \\
  &= \frac{1}{2}\left[\frac{1}{m\pi-n\pi}\sin(m\pi x-n\pi x)+
    \frac{1}{m\pi+n\pi}\sin(m\pi x+n\pi x)\right]_{-1}^{1} \\
  &= \frac{1}{2}\left[
    \frac{\sin(m\pi-n\pi)}{m\pi-n\pi}+\frac{\sin(m\pi+n\pi)}{m\pi+n\pi}-
    \frac{\sin(n\pi-m\pi)}{m\pi-n\pi}-\frac{\sin(-m\pi-n\pi)}{m\pi+n\pi}
    \right] \\
  &= \frac{1}{2}(0+0-0-0) \\
  &= 0
\end{align*}
Because \( m\in\N \) and \( n\in\N \), all the sines evaluate to some
\( \sin(c\pi) \) where \( c\in\N \). The sine of any multiple of \( \pi \) is
0.
\begin{align*}
  n&\in\N,m\in\N,n=m \\
  (f_n,f_m) &= \int_{-1}^{1}\frac{1}{2}\left(\cos(m\pi x-n\pi x)-
    \cos(m\pi x+n\pi x)\right)\diff{x} \\
  &= -\frac{1}{2}\int_{-1}^{1}\cos(2n\pi x)\diff{x} \\
  &= -\frac{1}{2}\left[\frac{\cos(2n\pi x)}{2n\pi}\right]_{-1}^{1} \\
  &\ne 1
\end{align*}
This set is not orthonormal.

\subsubsection*{Exercise 5}
Find \( \alpha \) so that \( \{1,x,1+\alpha x^2\} \) on \( (-1,1) \) is
orthogonal.
\begin{align*}
  0 &= (1,1+\alpha x^2) \\
  &= \int_{-1}^{1}1+\alpha x^2\diff{x} \\
  &= 2+\alpha\left[\frac{x^3}{3}\right]_{-1}^{1} \\
  &= 2+\alpha\frac{2}{3} \\
  -2 &= \frac{2\alpha}{3} \\
  \alpha &= -3
\end{align*}
Normalize the set.
\begin{align*}
  \|1\| &= \sqrt{\int_{-1}^{1}1^2\diff{x}} \\
  &= \sqrt{2} \\
  \|x\| &= \sqrt{\int_{-1}^{1}x^2\diff{x}} \\
  &= \sqrt{\frac{2}{3}} \\
  \|1-3x^2\| &= \sqrt{\int_{-1}^{1}(1-3x^2)^2} \\
  &= \sqrt{\int_{-1}^{1}1-6x^2+9x^4\diff{x}} \\
  &= \sqrt{\left[x-2x^3+\frac{9}{5}x^5\right]_{-1}^{1}} \\
  &= \sqrt{1-2+\frac{9}{5}+1-2+\frac{9}{5}} \\
  &= \sqrt{\frac{18}{5}-2} \\
  &= \sqrt{\frac{8}{5}}
\end{align*}
The normalized set is \( \{\frac{1}{\sqrt{2}},\frac{x\sqrt{3}}{\sqrt{2}},
\frac{(1-3x^2)\sqrt{5}}{2\sqrt{2}}\} \).

\subsubsection*{Exercise 6}
Show that the set \( \{1,\cos(\frac{n\pi x}{L}), \sin(\frac{m\pi x}{L})\},
n,m\in\N, -L<x<L \) is orthogonal but not orthonormal.

\begin{center}
  If you have any questions, comments, or concerns, please contact me at
  alvin@omgimanerd.tech
\end{center}

\end{document}
