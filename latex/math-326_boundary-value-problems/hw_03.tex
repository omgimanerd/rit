\documentclass{math}

\usepackage{enumerate}

\geometry{letterpaper, margin=0.5in}

\title{Boundary Value Problems: Homework 3}
\author{Alvin Lin}
\date{August 2018 - December 2018}

\begin{document}

\maketitle

\subsection*{Problem 1}
Exercises 2.1: problems 4-7

\subsubsection*{Exercise 4}
Show that the set of functions \( \{\sin(n\pi x), -1<x<1, n\in\N \) is
orthogonal.
\begin{align*}
  n&\in\N,m\in\N,n\ne m \\
  (f_n,f_m) &= \int_{-1}^{1}\sin(m\pi x)\sin(n\pi x)\diff{x} \\
  &= \int_{-1}^{1}\frac{1}{2}\left(\cos(m\pi x-n\pi x)-
    \cos(m\pi x+n\pi x)\right)\diff{x} \\
  &= \frac{1}{2}\left[\frac{1}{m\pi-n\pi}\sin(m\pi x-n\pi x)+
    \frac{1}{m\pi+n\pi}\sin(m\pi x+n\pi x)\right]_{-1}^{1} \\
  &= \frac{1}{2}\left[
    \frac{\sin(m\pi-n\pi)}{m\pi-n\pi}+\frac{\sin(m\pi+n\pi)}{m\pi+n\pi}-
    \frac{\sin(n\pi-m\pi)}{m\pi-n\pi}-\frac{\sin(-m\pi-n\pi)}{m\pi+n\pi}
    \right] \\
  &= \frac{1}{2}(0+0-0-0) \\
  &= 0
\end{align*}
Because \( m\in\N \) and \( n\in\N \), all the sines evaluate to some
\( \sin(c\pi) \) where \( c\in\N \). The sine of any multiple of \( \pi \) is
0.
\begin{align*}
  n&\in\N,m\in\N,n=m \\
  (f_n,f_m) &= \int_{-1}^{1}\frac{1}{2}\left(\cos(m\pi x-n\pi x)-
    \cos(m\pi x+n\pi x)\right)\diff{x} \\
  &= -\frac{1}{2}\int_{-1}^{1}\cos(2n\pi x)\diff{x} \\
  &= -\frac{1}{2}\left[\frac{\cos(2n\pi x)}{2n\pi}\right]_{-1}^{1} \\
  &\ne 1
\end{align*}
This set is not orthonormal.

\subsubsection*{Exercise 5}
Find \( \alpha \) so that \( \{1,x,1+\alpha x^2\} \) on \( (-1,1) \) is
orthogonal.
\begin{align*}
  0 &= (1,1+\alpha x^2) \\
  &= \int_{-1}^{1}1+\alpha x^2\diff{x} \\
  &= 2+\alpha\left[\frac{x^3}{3}\right]_{-1}^{1} \\
  &= 2+\alpha\frac{2}{3} \\
  -2 &= \frac{2\alpha}{3} \\
  \alpha &= -3
\end{align*}
Normalize the set.
\begin{align*}
  \|1\| &= \sqrt{\int_{-1}^{1}1^2\diff{x}} \\
  &= \sqrt{2} \\
  \|x\| &= \sqrt{\int_{-1}^{1}x^2\diff{x}} \\
  &= \sqrt{\frac{2}{3}} \\
  \|1-3x^2\| &= \sqrt{\int_{-1}^{1}(1-3x^2)^2} \\
  &= \sqrt{\int_{-1}^{1}1-6x^2+9x^4\diff{x}} \\
  &= \sqrt{\left[x-2x^3+\frac{9}{5}x^5\right]_{-1}^{1}} \\
  &= \sqrt{1-2+\frac{9}{5}+1-2+\frac{9}{5}} \\
  &= \sqrt{\frac{18}{5}-2} \\
  &= \sqrt{\frac{8}{5}}
\end{align*}
The normalized set is \( \{\frac{1}{\sqrt{2}},\frac{x\sqrt{3}}{\sqrt{2}},
\frac{(1-3x^2)\sqrt{5}}{2\sqrt{2}}\} \).

\subsubsection*{Exercise 6}
Show that the set \( \{1,\cos(\frac{n\pi x}{L}), \sin(\frac{m\pi x}{L})\},
n,m\in\N, -L<x<L \) is orthogonal but not orthonormal.
\begin{align*}
  (\cos(\frac{n\pi x}{L}),\sin(\frac{m\pi x}{L})) &=
    \int_{-L}^{L}\cos(\frac{n\pi x}{L})\sin(\frac{m\pi x}{L})\diff{x} \\
  &= \int_{-L}^{L}\frac{1}{2}(\sin(\frac{n\pi x+m\pi x}{L})-
    \sin(\frac{n\pi x-m\pi x}{L}))\diff{x} \\
  &= 0 \\
  (1,\sin(\frac{n\pi x}{L})) &= \int_{-L}^{L}\sin(\frac{n\pi x}{L})\diff{x} \\
  &= 0
\end{align*}
Because \( \sin \) is an odd function over the symmetric integral \( [-L,L] \).
\begin{align*}
  (1,\cos(\frac{n\pi x}{L})) &= \int_{-L}^{L}\cos(\frac{n\pi x}{L})\diff{x} \\
  &= \left[\frac{L}{n\pi}\sin(\frac{n\pi x}{L})\right]_{-L}^{L} \\
  &= \frac{L}{n\pi}(\sin(n\pi)-\sin(-n\pi)) \\
  &= 0
\end{align*}
\( (f_m,f_n) = 0, m\ne n \), therefore the set is orthogonal.
\begin{align*}
  (\cos(\frac{n\pi x}{L}),\cos(\frac{n\pi x}{L})) &=
    \sqrt{\int_{-L}^{L}\cos^2(\frac{n\pi x}{L})\diff{x}} \\
  &= \sqrt{\int_{-L}^{L}\frac{1+\cos(\frac{2n\pi x}{L})}{2}} \\
  &= \sqrt{L+\frac{1}{2}
    \left[\frac{L}{2n\pi}\sin(\frac{2n\pi x}{L})\right]_{-L}^{L}} \\
  &\ne 1
\end{align*}
Therefore, the set is not orthonormal.

\subsubsection*{Exercise 7}
Is the set \( \{\cos(\frac{n\pi x}{2})\}, n\in\N_0\} \), \( 0<x<2 \),
orthonormal? If it fails to be orthonormal, write the corresponding
orthonormal set.
\begin{align*}
  n &\ne m \\
  (\cos(\frac{n\pi x}{2}),\cos(\frac{m\pi x}{2})) &=
    \int_0^2\cos(\frac{n\pi x}{2})\cos(\frac{m\pi x}{2})\diff{x} \\
  &= \frac{1}{2}\int_0^2\cos(\frac{n\pi x-m\pi x}{2})+
    \cos(\frac{n\pi x+m\pi x}{2}) \\
  &= \bigg[\sin(n\pi x-m\pi x)+\sin(n\pi x+m\pi x)\bigg]_0^2 \\
  &= \sin(2n\pi-2m\pi)+\sin(2n\pi+2m\pi)-\sin(0)-\sin(0) \\
  &= 0
\end{align*}
The set is orthogonal.
\begin{align*}
  n &= m \\
  (\cos(\frac{n\pi x}{2}),\cos(\frac{n\pi x}{2})) &=
    \sqrt{\int_0^2\cos^2(\frac{n\pi x}{2})\diff{x}} \\
  &= \sqrt{\frac{1}{2}\int_0^2 1+\cos(n\pi x)\diff{x}} \\
  &= \sqrt{\frac{1}{2}\bigg[x+\frac{\sin(n\pi x)}{n\pi}\bigg]_0^2} \\
  &= \sqrt{\frac{1}{2}\bigg[2+0-0-0\bigg]} \\
  &= \sqrt{1} \\
  &= 1
\end{align*}
The set is also orthonormal. \\

\noindent Exercises 3.2: problems 6-8

\subsubsection*{Exercise 6}
Prove that the sum of two odd functions is odd.
\begin{align*}
  f(-x) &= -f(x) \\
  \int_{-L}^{L}f(x)\diff{x} &= 0 \\
  g(-x) &= -g(x) \\
  \int_{-L}^{L}g(x)\diff{x} &= 0 \\
  \int_{-L}^{L}f(x)+g(x)\diff{x} &= \int_{-L}^{L}f(x)\diff{x}+
    \int_{-L}^{L}g(x)\diff{x} \\
  &= 0+0 \\
  &= 0
\end{align*}

\subsubsection*{Exercise 7}
Show that if \( f \) is odd, then \( |f| \) and \( f^2 \) are even functions.
\begin{align*}
  f(-x) &= -f(x) \\
  |f(-x)| &= |-f(x)| \\
  f(-x) &= f(x)
\end{align*}
An odd function is symmetric about the origin. Taking the absolute value
reflects the negative portion of the function across the x axis and because
of the original symmetry, the function is now symmetric about the y axis.
\begin{align*}
  f(-x) &= -f(x) \\
  f(-x)^2 &= (-f(x))^2 \\
  f(-x)^2 &= f(x)^2
\end{align*}
Squaring an odd function takes the negative portion of the function and reflects
it across the x axis due to the change in sign while increasing the whole
function by a factor of itself. Because of the original symmetry, this function
is now symmetric about the y axis.

\subsection*{Problem 2}
Classify the following functions as even, odd, or neither. Show steps to justify
your answers.
\begin{enumerate}[(a)]
  \item \( f(x) = (1-x^2)^{-\frac{1}{2}} \)
  \begin{align*}
    f(a) &= f(-a) \\
    (1-a^2)^{-\frac{1}{2}} &= (1-a^2)^{-\frac{1}{2}}
  \end{align*}
  Even.
  \item \( f(x) = \e^{-x}\cos(3x) \)
  \begin{align*}
    f(a) &= f(-a) \\
    \e^{-a}\cos(3a) &= \e^{a}\cos(-3a) \\
    \e^{-a}\cos(3a) &\ne \e^{a}\cos(3a) \\
  \end{align*}
  Neither.
  \item \( f(x) = \sinh(x) \)
  \begin{align*}
    f(a) &= f(-a) \\
    \sinh(a) &= \sinh(-a) \\
    \sinh(a) &\ne -\sinh(a)
  \end{align*}
  Odd.
\end{enumerate}

\begin{center}
  If you have any questions, comments, or concerns, please contact me at
  alvin@omgimanerd.tech
\end{center}

\end{document}
