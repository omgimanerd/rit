\documentclass{math}

\usepackage{enumerate}

\geometry{letterpaper, margin=0.5in}

\title{Boundary Value Problems: Homework 2}
\author{Alvin Lin}
\date{August 2018 - December 2018}

\begin{document}

\maketitle

\subsubsection*{Problem 1}
Complete Exercises 1.2 problems 1, 2, 3, 5 \\
\textbf{Exercise 1} \\
Determine a general solution for the equation \( y''+5y'+6y = 0 \).
\begin{align*}
  y''+5y'+6y &= 0 \\
  y &= \e^{rt} \\
  r^2+5r+6 &= (r+2)(r+3) = 0 \\
  r_1 &= -2 \quad r_2 = -3 \\
  y &= c_1\e^{-2t}+c_2\e^{-3t}
\end{align*}
\textbf{Exercise 2} \\
Find a general solution for the equation \( y''-4y'+4y = 0 \).
\begin{align*}
  y''-4y'+4y &= 0 \\
  r^2-4r+4 &= 0 \\
  r_{1,2} &= -2 \\
  y &= \e^{-2t}+t\e^{-2t}
\end{align*}
\textbf{Exercise 3} \\
Solve the differential equation \( y''+2y'+2y = 0 \).
\begin{align*}
  y''+2y'+2y &= 0 \\
  r^2+2r+2 &= 0 \\
  r &= \frac{-2+\sqrt{4-(4)(1)(2)}}{2} \\
  &= -1\pm i \quad \alpha = -1 \quad \beta = 1 \\
  y &= \e^{-t}\left[c_1\cos(t)+c_2\sin(t)\right]
\end{align*}
\textbf{Exercise 5} \\
Solve the boundary value problem \( y''-y = 0, y(0) = 0, y'(\pi) = 1 \).
\begin{align*}
  y''-y &= 0 \\
  r^2-1 &= (r+1)(r-1) = 0 \\
  r_1 &= -1 \quad r_2 = 1 \\
  y &= c_1\e^{-t}+c_2\e^t \\
  y(0) &= 0 = c_1+c_2 \\
  y &= c_1\e^{-t}-c_1\e^t \\
  y' &= -c_1\e^{-t}-c_1\e^t \\
  y'(\pi) &= 1 = -c_1\e^{-\pi}-c_1\e^{\pi} \\
  &= c_1(-\e^{-\pi}-\e^{\pi}) \\
  c_1 &= \frac{1}{-\e^{-\pi}-\e^{\pi}} \\
  y &= \frac{\e^{-t}}{-\e^{-\pi}-\e^{\pi}}-\frac{\e^t}{-\e^{-\pi}-\e^{\pi}}
\end{align*}

\subsubsection*{Problem 2}
Determine all solutions to
\[ y''+y = 0, y(0) = 0, y(2\pi) = 1 \]
You may assume the domain is \( 0<x<2\pi \).
\begin{align*}
  y''+y &= 0 \\
  r^2+1 &= 0 \\
  r &= 0\pm i \quad \alpha = 0 \quad \beta = 1 \\
  y &= \e^{0}(c_1\cos(t)+c_2\sin(t)) \\
  &= c_1\cos(t)+c_2\sin(t) \\
  y(0) &= 0 = c_1\cos(0)+c_2\sin(0) \\
  c_1 &= 0 \\
  y(2\pi) &= 1 = c_2\sin(2\pi) \\
  1 &= 0
\end{align*}
No solution.

\subsubsection*{Problem 3}
Determine all solutions to
\[ y''+4y = 0, y'(0) = 0, y'(\frac{\pi}{2}) = 0 \]
You may assume the domain is \( 0<x<\frac{\pi}{2} \).
\begin{align*}
  y''+4y &= 0 \\
  r^2+4 &= 0 \\
  r &= 0\pm2i \quad \alpha = 0 \quad \beta = 2 \\
  y &= \e^{0t}(c_1\cos(2t)+c_2\sin(2t)) \\
  y' &= -2c_1\sin(2t)+2c_2\cos(2t) \\
  y'(0) &= 0 = -2c_1\sin(0)+2c_2\cos(0) \\
  c_2 &= 0 \\
  y'(\frac{\pi}{2}) &= 0 = -2c_1\sin(\frac{\pi}{2}) \\
  c_1 &= 0 \\
  y &= 0
\end{align*}

\subsubsection*{Problem 4}
Classify the following PDEs as hyperbolic, parabolic, or elliptic.
\begin{enumerate}[(a)]
  \item \[\pdiff{u}{t}+\pdiff{^2u}{t^2} =
    -2\pdiff{^2u}{x^2}+0.5\pdiff{^u}{x\partial t} \]
  \begin{align*}
    B &= 2 \quad A = 1 \quad C = -0.5 \\
    B^2-4AC &= 4-4(1)(-0.5) \\
    &= 6
  \end{align*}
  hyperbolic
  \item \[ 3u_{xx}+2u_{xy}+u_y = 7u_{yy} \]
  \begin{align*}
    B &= 2 \quad A = 3 \quad C = -7 \\
    B^2-4AC &= 4-4(3)(-7) \\
    &= 88
  \end{align*}
  hyperbolic
  \item \[ u_{yy} = u_x+u \]
  \begin{align*}
    B &= 0 \quad A = 0 \quad C = 1 \\
    B^2-4AC &= 0-4(0)(1) \\
    &= 0
  \end{align*}
  parabolic
  \item \[ -3u_{yy}+2u_{xy} = -u_{xx} \]
  \begin{align*}
    B &= 2 \quad A = -3 \quad C = 1 \\
    B^2-4AC &= 4-4(-3)(1) \\
    &= 16
  \end{align*}
  hyperbolic
\end{enumerate}

\subsubsection*{Problem 5}
Find all values of \( c\in\R \) such that the vectors \( \vec{a} = \langle
c,0.5,c\rangle \) and \( \vec{b} = \langle-3,4,c\rangle \) are orthogonal.
\begin{align*}
  \vec{a}\cdot\vec{b} &= 0 \\
  -3c+2+c^2 &= 0 \\
  (c-2)(c-1) &= 0 \\
  c_1 &= 2 \quad c_2 = 1
\end{align*}

\subsubsection*{Problem 6}
Use the dot product to find the angle (in radians) between the vectors
\( \vec{a} = \langle2,4,0\rangle \) and \( \vec{b} = \langle-1,-1,4\rangle \).
\begin{align*}
  \vec{a}\cdot\vec{b} &= \|\vec{a}\|\|\vec{b}\|\cos\theta \\
  -2+(-4)+0 &= \sqrt{4+16}\sqrt{18}\cos\theta \\
  -6 &= (2\sqrt{5})(3\sqrt{2})\cos\theta \\
  \cos\theta &= -\frac{1}{\sqrt{10}} \\
  &= -\frac{\sqrt{10}}{10} \\
  \theta &\approx 1.892
\end{align*}

\subsubsection*{Problem 7}
Let \( S = \{\vec{x_1},\vec{x_2},\vec{x_3}\} \), where \( \vec{x_1} =
\langle1,-1,0\rangle, \vec{x_2} = \langle1,1,0\rangle, \vec{x_3} = \langle
0,0,1\rangle \).
\begin{enumerate}[(a)]
  \item Is \( S \) an orthogonal set? You will need to compute several dot
    products to answer this question.
    \begin{align*}
      \vec{x_1}\cdot\vec{x_2} &= 1+(-1)+0 = 0 \\
      \vec{x_1}\cdot\vec{x_3} &= 0+0+0 = 0 \\
      \vec{x_2}\cdot\vec{x_3} &= 0+0+0 = 0
    \end{align*}
    All the vectors are mutually orthogonal so the set is an orthogonal set.
  \item Is \( S \) an orthonormal set? Remember to show calculation(s) that
    support your answer.
    \[ \|\vec{x_1}\| = \sqrt{1+1} = \sqrt{2} \]
    Since not all the vectors are unit vectors, \( S \) is not orthonormal.
\end{enumerate}

\begin{center}
  If you have any questions, comments, or concerns, please contact me at
  alvin@omgimanerd.tech
\end{center}

\end{document}
