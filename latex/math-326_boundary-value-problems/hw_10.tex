\documentclass{math}

\usepackage{enumerate}

\geometry{letterpaper, margin=0.5in}

\title{Boundary Value Problems: Homework 10}
\author{Alvin Lin}
\date{August 2018 - December 2018}

\begin{document}

\maketitle

\subsection*{Problem 1}
Classify the following boundary conditions as Dirichlet, Neumann, separated, or
periodic. If the boundary conditions are Dirichlet, Neumann, or separated, also
state whether the conditions are homogeneous or nonhomogeneous.
\begin{enumerate}[(a)]
  \item \( y'(1) = 0, y'(2) = 1 \) \\
  Nonhomogeneous Neumann
  \item \( u(0,t) = u(9,t), u_x(0,t) = u_x(9,t) \) \\
  Periodic
  \item \( u(\pi,y) = 0, u(\pi,y) = 0 \) \\
  Homogeneous Dirichlet
  \item \( 3y(0)-2y'(0) = 6, y(5)+4y'(5) = 8 \) \\
  Nonhomogeneous separated
\end{enumerate}

\subsection*{Problem 2}
The ends of a rod are at temperatures \( 10^{\circ} \)C and \( 50^{\circ} \)C.
the rod is 4 units long and has an initial temperature distribution of
\( 30^{\circ} \)C. The diffusivity constant \( a^2 \) is 4. The corresponding
boundary value problem is then
\begin{align*}
  u_t &= 4u_{xx} \quad 0<x<4,t>0 \\
  u(0,t) &= 10 \quad u(4,t) = 50, t\ge 0 \\
  u(x,0) &= 30 \quad 0<x<4
\end{align*}
Find the temperature \( u(x,t) \).
\begin{align*}
  u(x,t) &= \psi(x)+u(x,t) \\
  \psi_t(x)+v_t(x,t) &= 4(\psi_{xx}(x)+v_{xx}(x,t)) \\
  v_t(x,t) &= 4(\psi_{xx}(x)+v_{xx}(x,t)) \\
  u(0,t) &= \psi(0)+v(0,t) = 10 \\
  u(4,t) &= \psi(4)+v(4,t) = 50 \\
  t&\to\infty \\
  0 &= 4\psi_{xx}(x) \\
  \psi(0) &= 10 \\
  \psi(4) &= 50 \\
\end{align*}
\begin{align*}
  \psi(x) &= Ax+B \\
  \psi(0) &= 10 = B \\
  \psi(4) &= A(4)+10 = 50 \\
  A &= 10 \\
  \psi(x) &= 10x+10 \\
  u(0,t) &= \psi(0)+v(0,t) = 10 \\
  u(4,t) &= \psi(4)+v(4,t) = 50 \\
  v(0,t) &= v(4,t) = 0 \\
  u(x,0) &= \psi(x)+v(x,0) = 30 \\
  v(x,0) &= 30-\psi(x) \\
  &= 30-(10x+10) \\
  &= 20-10x \\
  v(x,t) &= \sum_{n=1}^{\infty}
    b_n\sin(\frac{n\pi x}{4})\e^{-4(\frac{n\pi}{4})^2t} \\
  b_n &= \frac{2}{4}\int_{0}^{4}(20-10x)\sin(\frac{n\pi x}{4})\diff{x} \\
  &= \frac{40+40\cos(n\pi)}{n\pi} \\
  u(x,t) &= \psi(x)+v(x,t) \\
  &= 10x+10+\sum_{n=1}^{\infty}
    \frac{40+40\cos(n\pi)}{n\pi}
    \sin(\frac{n\pi x}{4})\e^{-4(\frac{n\pi}{4})^2t} \\
\end{align*}

\subsection*{Problem 3}
Solve the boundary value problem
\begin{align*}
  u_t &= u_{xx} \quad 0<x<\pi,t>0 \\
  u(0,t) &= 0 \quad u(\pi,t) = T_0 \quad t\ge 0 \\
  u(x,0) &= 0 \quad 0<x<\pi
\end{align*}
\begin{align*}
  u(x,t) &= \psi(x)+v(x,t) \\
  \psi_t+v_t &= \psi_{xx}+v_{xx} \\
  u(0,t) &= \psi(0)+v(0,t) = 0 \\
  u(\pi,t) &= \psi(\pi)+v(\pi,t) = T_0 \\
  t &\to \infty \\
  0 &= \psi_{xx} \\
  \psi(0) &= 0 \\
  \psi(\pi) &= T_0 \\
  \psi(x) &= Ax+B \\
  \psi(0) &= B = 0 \\
  \psi(\pi) &= A(\pi) = T_0 \\
  \psi(x) &= \frac{T_0x}{\pi}
\end{align*}
\begin{align*}
  v_t &= \psi_{xx}+v_{xx} \\
  u(0,t) &= \psi(0)+v(0,t) = 0 \\
  u(\pi,t) &= \psi(\pi)+v(\pi,t) = T_0 \\
  v(0,t) &= v(\pi,t) = 0 \\
  u(x,0) &= \psi(x)+v(x,0) = 0 \\
  v(x,0) &= -\psi(x) = -\frac{T_0x}{\pi} \\
  v(x,t) &= \sum_{n=1}^{\infty}
    b_n\sin(\frac{n\pi x}{\pi})\e^{-(\frac{n\pi}{\pi})^2t} \\
  b_n &= \frac{2}{\pi}\int_{0}^{\pi}(-\frac{T_0x}{\pi})
    \sin(\frac{n\pi x}{\pi})\diff{x} \\
  &= \frac{2T_0\cos(n\pi)}{n\pi} \\
  u(x,t) &= \frac{T_0x}{\pi}+\sum_{n=1}^{\infty}
    \frac{2T_0\cos(n\pi)}{n\pi}\sin(nx)\e^{-n^2t}
\end{align*}

\subsection*{Problem 4}
Given the boundary value problem
\begin{align*}
  u_t &= u_{xx}+k\sin(x) \quad 0<x<\pi,t>0 \\
  u(0,t) &= u(\pi,t) = 0 \quad t\ge0 \\
  u(x,0) &= \sin(x) \quad 0\le x\le\pi
\end{align*}
Determine \( u(x,t) \).
\begin{align*}
  u(x,t) &= \psi(x)+v(x,t) \\
  \psi_t+v_t &= \psi_{xx}+v_{xx}+k\sin(x) \\
  v_t &= \psi_{xx}+v_{xx}+k\sin(x) \\
  u(0,t) &= \psi(0)+v(0,t) = 0 \\
  u(\pi,t) &= \psi(\pi)+v(\pi,t) = 0 \\
  t &\to \infty \\
  0 &= \psi_{xx}+k\sin(x) \\
  \psi(0) &= \psi(\pi) = 0 \\
  \psi_{xx}(x) &= -k\sin(x) \\
  \psi_{x}(x) &= k\cos(x)+A \\
  \psi(x) &= k\sin(x)+Ax+B \\
  \psi(0) &= k\sin(0)+0+B = 0 \\
  \psi(\pi) &= k\sin(\pi)+A\pi+0 = 0 \\
  A &= 0 \quad B = 0 \\
  \psi(x) &= k\sin(x) \\
  u(0,t) &= \psi(0)+v(0,t) = 0 \\
  u(\pi,t) &= \psi(\pi)+v(\pi,t) = 0 \\
  v(0,t) &= v(\pi,t) = 0 \\
  u(x,0) &= \psi(x)+v(x,0) = \sin(x) \\
  v(x,0) &= \sin(x)-k\sin(x) = (1-k)\sin(x) \\
  v(x,t) &= \sum_{n=1}^{\infty}
    b_n\sin(\frac{n\pi x}{\pi})\e^{-(\frac{n\pi x}{\pi})^2t} \\
  b_n &= \frac{2}{\pi}\int_{0}^{\pi}
    (1-k)\sin(x)\sin(\frac{n\pi x}{\pi})\diff{x} = 0 \\
  v(x,t) &= 0 \\
  u(x,t) &= k\sin(x)
\end{align*}

\begin{center}
  If you have any questions, comments, or concerns, please contact me at
  alvin@omgimanerd.tech
\end{center}

\end{document}
