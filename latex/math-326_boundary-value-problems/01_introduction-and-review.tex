\documentclass{math}

\title{Boundary Value Problems}
\author{Alvin Lin}
\date{August 2018 - December 2018}

\begin{document}

\maketitle

\section*{Introduction and Review}
Recall that ordinary differential equations (ODEs) were often focused on
initial value problems. For example:
\[ y''+3y'+2y = 0 \quad y(x_0) = 3 \quad y'(x_0) = -1 \]
Boundary value problems consist of a differential equation and boundary
conditions given at two or more values of the independent variable. For example:
\[ y''+16y = 0 \quad y(x_0) = y_0 \quad y(x_1) = y_1 \]
It is implied here that \( x_0 \ne x_1 \). Boundary conditions can vary greatly:
\begin{align*}
  y'(x_0) &= y_0 \quad y(x_1) = y_1 \\
  y(x_0) &= y_0 \quad y'(x_1) = y_1 \\
  y'(x_0) &= y_0 \quad y'(x_1) = y_1
\end{align*}
Notationally, we may write \( \ddiff{y}{x} \) as \( D_y \) where \( D \) is an
operator applied to the variable \( y \). Similarly, we may also abbreviate
\( \ddiff{^2y}{x^2} \) as \( DDy \) or \( D^2y \). For example, the following
two differential equations are equivalent:
\begin{align*}
  y''+3y'+2y &= 0 \\
  D^2y+3Dy+2y &= 0
\end{align*}
Suppose we define the operator \( L = D^2+3D+2 \) (short for ``linear''
operator), then \( Ly = 0 \). Linear operations on \( y \) include
\( y, y', y'', 3y, 3y', 3xy, 5xy' \), etc. Non-linear operations on \( y \)
include \( y^2, \sin(y), yy' \), etc. Note that \( L \) is a linear operator
if and only if \( L(c_1y_1+c_2y_2) = c_1Ly_1+c_2Ly_2 \). For example:
\begin{itemize}
  \item \( y'+2xy = 0 \) is a linear differential equation.
  \item \( y'+3x\sin(y) = 0 \) is a non-linear differential equation.
\end{itemize}
For an ordinary differential equation, the order of the equation is the highest
order of derivative that appears in the equation. Partial differential equations
may contain partial derivatives, and their order is also defined by the highest
partial derivative it contains. \par
The generic form of a linear differential operator is as follows:
\[ L = \alpha_n(x)D^n+\alpha_{n-1}D^{n-1}+\dots+\alpha_1(x)D+\alpha_0(x) \]
where \( \alpha \) can either be a constant or coefficient in terms of \( x \).
If \( \alpha_0,\alpha_1,\dots,\alpha_n \) are all constants, then the
differential equation is a constant coefficient equation. If any \( \alpha \)
depends explicitly on \( x \), then the differential equation is a variable
coefficient equation. \par
If an equation has the form \( Ly = 0 \), then the differential equation is
linear and homogeneous. If an equation has the form \( L(y) = f(x) \ne 0 \),
then we call the equation non-homogeneous. These properties all influence the
type of solution we will use as well as the ease of solving the differential
equation.
\begin{center}
  \begin{tabular}{|c|c|c|c|c|}
    \hline
    linear & low order & homogeneous & const coefficient & ODE \\
    \hline
    non-linear & high order & non-homogeneous & variable coefficient & PDE \\
    \hline
  \end{tabular}
\end{center}
\begin{itemize}
  \item \( 2\sin(x)y''+y+6 = 0 \) \\
    ODE \\
    linear \\
    second order \\
    variable coefficient \\
    non-homogeneous
  \item \( uu_{xxx}+xu_t = 0 \) \\
    PDE \\
    non-linear \\
    third order \\
    variable coefficient \\
    homogeneous
\end{itemize}

\begin{center}
  You can find all my notes at \url{http://omgimanerd.tech/notes}. If you have
  any questions, comments, or concerns, please contact me at
  alvin@omgimanerd.tech
\end{center}

\end{document}
