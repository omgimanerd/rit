\documentclass{math}

\usepackage{enumerate}

\geometry{letterpaper, margin=0.5in}

\title{Boundary Value Problems: Homework 1}
\author{Alvin Lin}
\date{August 2018 - December 2018}

\begin{document}

\maketitle

\subsection*{Problem 1}
Use the definition of a linear operator (given in Equation 1.1 in your textbook)
to show that the operator \( L \) as defined \( Ly = 3Dy-2y \) is a linear
operator. Here \( D \) is the first-derivative operator.
\begin{align*}
  L(c_1y_1+c_2y_2) &= c_1Ly_1+c_2Ly_2 \\
  3D(c_1y_1+c_2y_2)-2(c_1y_1+c_2y_2) &= c_1(3Dy_1-2y_1)+c_2(3Dy_2-2y_2) \\
  3c_1Dy_1+3c_2Dy_2-2c_1y_1-2c_2y_2 &= 3c_1Dy_1-2c_1y_1+3c_2Dy_2-2c_2y_2 \\
\end{align*}

\subsection*{Problem 2}
Use the definition of a lienar operator (given in Equation 1.1 of your textbook)
to show that the squaring operator, \( Q \), defined as \( Qy = y^2 \) is not
linear.
\begin{align*}
  L(y_1+y_2) &= Ly_1+Ly_2 \\
  Q(y_1+y_2) &= Qy_1+Qy_2 \\
  (y_1)^2+2y_1y_2+(y_2)^2 &\ne (y_1)^2+(y_2)^2 \\
\end{align*}

\subsection*{Problem 3}
For each of the following equations in parts (a)-(e),
\begin{itemize}
  \item classify the equation as an ODE or a PDE
  \item state the order of the equation
  \item list the dependent and independent variables
  \item state whether the equation is linear or nonlinear
  \item if the equation is linear, state whether the equation has variable or
    constant coefficients and whether the equation is homogeneous or
    non-homogeneous
\end{itemize}
\begin{enumerate}[(a)]
  \item \[ 5\pdiff{^2u}{x^2}-\pdiff{u}{t}+u = 0 \]
    PDE, second order, dependent variables (\( u \)), independent variables
    (\( x, t \)), linear, constant coefficients, homogeneous
  \item \[ \left(\left(\ddiff{x}{t}\right)^2+1\right)x = 10 \]
    ODE, first order, dependent variables (\( x \)), independent variables
    (\( t \)), non-linear
  \item \[ \sin(x)\ddiff{^3y}{x^3}+\ddiff{y}{x} = -y \]
    ODE, third order, dependent variables (\( y \)), independent variables
    (\( x \)), linear, variable coefficients, homogeneous
  \item \[ 10y = (2-y^3)\ddiff{y}{t}-\ddiff{^2y}{t^2} \]
    ODE, second order, dependent variables (\( y \)), independent variables
    (\( t \)), non-linear
  \item \[ 2\theta+t^2-\theta''' = 0 \]
    ODE, third order, dependent variables (\( \theta \)), independent variables
    (\( t \)), linear, constant coefficients, non-homogeneous
\end{enumerate}

\subsection*{Problem 4}
Classify the following PDEs as linear or nonlinear, and as homogeneous or
non-homogeneous.
\begin{enumerate}[(a)]
  \item \[ u_{rr}-(\frac{\theta}{r})u_r = -\frac{u_{\theta}}{r^2} \]
  linear and homogeneous
  \item \[ 10+5\pdiff{^2u}{x^2}+25\frac{\partial^2u}{\partial x\partial y}+
    5\pdiff{^2u}{y^2} = 0 \]
    linear and non-homogeneous
  \item \[ u_t = 4u_xx+3uu_x \]
    non-linear and non-homogeneous
\end{enumerate}

\subsection*{Problem 5}
Classify each of the following problems as an IVP, a BVP, or neither.
\begin{enumerate}[(a)]
  \item \[ x^2y''+7xy'+3y = 5, \quad y(0) = 2, \quad y'(-1) = 0 \]
    boundary value problem
  \item \[ x^2y''+7xy'+3y = 5, \quad y(-1) = 2, \quad y'(-1) = 0 \]
    initial value problem
  \item \[ x^2y''+7xy'+3y = 5, \quad y(0) = 2, \quad y(-1) = 0 \]
    boundary value problem
\end{enumerate}

\begin{center}
  If you have any questions, comments, or concerns, please contact me at
  alvin@omgimanerd.tech
\end{center}

\end{document}
