\documentclass{math}

\usepackage{tikz}

\title{Boundary Value Problems}
\author{Alvin Lin}
\date{August 2018 - December 2018}

\begin{document}

\maketitle

\section*{Fourier Series}
Terminology:
\begin{itemize}
  \item RH: right hand
  \item LH: left hand
  \item RH limit: \( \lim_{x\to x_0^+}f(x) = f(x_0^+) \)
  \item LH limit: \( \lim_{x\to x_0^+}f(x) = f(x_0^-) \)
  \item RH derivative: \( \lim_{x\to x_0^+}f'(x) = f_+'(x_0) \)
  \item LH derivative: \( \lim_{x\to x_0^-}f'(x) = f_-'(x_0) \)
  \item Continuous function: \( f(x_0^-) = f(x_0^+) = f(x_0) \)
  \item \( f(x) \) is smooth at \( x_0 \) when \( f'(x) \) is continuous at
    \( x_0 \)
\end{itemize}
Many types of piecewise functions we care about are piecewise continouous and
piecewise smooth. A function \( f(x) \) is piecewise continuous if it has at
most a finite number of jump or removable discontinuities. A function \( f \) is
piecewise smooth if \( f'(x) \) is piecewise continuous.

\subsubsection*{Example}
\[ f(x) = \begin{cases}
  x^2 &, x\ge0 \\
  -1 &, x<0
\end{cases} \]
This is piecewise smooth because
\[ f'(x) = \begin{cases}
  2x &, x\ge0 \\
  0 &, x<0
\end{cases} \]
and this is piecewise continuous. However, the function
\[ f(x) = \begin{cases}
  \frac{1}{x-1} &, x>\frac{1}{2} \\
  2x &, x<\frac{1}{2}
\end{cases} \]
is neither piecewise smooth nor piecewise continuous due to the infinite
discontinuity at \( x = 1 \).

\subsection*{Fourier Series}
A Fourier series of \( f(x) \) on \( -L<x<L \) is associated with:
\[ f(x) ~ \frac{a_0}{2}+\sum_{n=1}^{\infty}(a_n\cos(\frac{n\pi x}{L}+
b_n\sin(\frac{n\pi x}{L}))) \]
Euler formulas for coefficients:
\begin{align*}
  a_n &= \frac{1}{L}\int_{-L}^{L}f(x)\cos(\frac{n\pi x}{L})\diff{x} \\
  &= \frac{1}{L}(f,\cos(\frac{n\pi x}{L})) \quad n = 0,1,2,\dots \\
  b_n &= \frac{1}{L}\int_{-L}^{L}f(x)\sin(\frac{n\pi x}{L})\diff{x} \\
  &= \frac{1}{L}(f,\sin(\frac{n\pi x}{L})) \quad n = 1,2,3,\dots
\end{align*}
\( a_n,b_n \) are components of \( f(x) \) along \( \sin \) and \( \cos \).
As long as \( f(x) \) is piecewise smooth on \( [-L,L] \), the Fourier series
of \( f(x) \) will converge to \( \frac{1}{2}[f(x_0^++f(x_0^-))] \) at a point
\( x_0\in[-L,L] \). At a point \( x_0 \), the Fourier series will converge to
\( \frac{f(x_0^-)+f(x_0^+)}{2} \), so for
\[ f(x) = \begin{cases}
  0 &, 0<x\le L \\
  1 &, -L\le x<0
\end{cases} \]
the Fourier series of \( f \) converges to \( \frac{f(0^-)-f(0^+)}{2} =
\frac{1+0}{2} = \frac{1}{2} \) at \( x = 0 \). In general, if \( f \) has a
discontinuity (jump or removable) at \( x_0 \), the Fourier series converges to
\[ \frac{1}{2}[f(x_0^+)+f(x_0^-)] \]
If \( f \) is continuous at \( x_0 \), the Fourier series converges to
\[ \frac{1}{2}[f(x_0^+)+f(x_0^-)] = f(x_0) \]

\subsubsection*{Example}
Find the convergence of the Fourier series of
\[ f(x) = \begin{cases}
  2 &, 0<x<1 \\
  0 &, -1<x<0
\end{cases} \]
at the following points: \( x = 0, x = 0.5, x = -\frac{3}{4} \)
\begin{align*}
  \frac{f(0^-)+f(0^+)}{2} &= \frac{0+2}{2} = 1 \\
  \frac{f(0.5^-)+f(0.5^+)}{2} &= \frac{2+2}{2} = 2 \\
  \frac{f(\frac{-3}{4}^-)+f(\frac{-3}{4}^+)}{2} &= \frac{0+0}{2} = 0
\end{align*}

\subsection*{Deriving the Euler formulas}
Euler formulas:
\begin{enumerate}
  \item \( a_0 = \frac{1}{L}\int_{-L}^{L}f(x)\diff{x} \)
  \item \( a_n = \frac{1}{L}\int_{-L}^{L}f(x)\cos(\frac{n\pi x}{L})\diff{x} \)
  \item \( b_n = \frac{1}{L}\int_{-L}^{L}f(x)\sin(\frac{n\pi x}{L})\diff{x} \)
\end{enumerate}
These formulas arise from orthogonality. Recall that \( \{1,
\cos(\frac{n\pi x}{L}),\sin(\frac{n\pi x}{L}) \} \) is an orthogonal set on
\( [-L,L] \). Suppose that
\[ f(x) = \frac{a_0}{2}+\sum_{n=1}^{\infty}a_n\cos(\frac{n\pi x}{L})+
  b_n\sin(\frac{n\pi x}{L}) \]
If we integrate both sides:
\begin{align*}
  \int_{-L}^{L}f(x)\diff{x} &= \int_{-L}^{L}\frac{a_0}{2}\diff{x}+
    \sum_{n=1}^{\infty}\{a_n\cos(\frac{n\pi x}{L})+
    b_n\sin(\frac{n\pi x}{L})\} \\
  &= \int_{-L}^{L}\frac{a_0}{2}\diff{x}+0 \\
  &= a_0L
\end{align*}
If we then divide by \( L \):
\[ \frac{1}{L}\int_{-L}^{L}f(x)\diff{x} = a_0 \]
This extends and generalizes for \( a_n \) and \( b_n \) as well.

\subsubsection*{Note}
Sometimes, one of the terms evaluates down to a cosine. For example:
\[ a_n = \frac{2}{n^2\pi^2}(cos(n\pi)-1) \]
This term is 0 when \( n \) is even and nonzero when \( n \) is odd since
\( cos(n\pi) \) oscillates between -1 and 1. Because of this, we only care about
\( a_{2n-1} \) and we can rewrite the Fourier series with this.
\[ f(x) \sim \frac{a_0}{2}+\sum_{n=1}^{\infty}a_{2n-1}\cos(\frac{n\pi x}{L})+
  b_n\sin(\frac{n\pi x}{L}) \]

\subsubsection*{Example}
Find \( a_0 \) and \( a_n, n\in\N \) for \( f(x) \).
\[ f(x) = \begin{cases}
  1 &, -2<x<0 \\
  x &, 0<x<2
\end{cases} \]
\begin{align*}
  a_0 &= \frac{1}{L}\int_{-L}^{L}f(x)\diff{x} \\
  &= \frac{1}{2}\left\{\int_{-2}^{0}1\diff{x}+\int_{0}^{2}x\diff{x}\right\} \\
  &= \frac{1}{2}\left(\bigg[x\bigg]_{-2}^{0}+
    \bigg[\frac{x^2}{2}\bigg]_0^2\right) \\
  &= \frac{1}{2}(2+2) \\
  &= 2 \\
  a_n &= \frac{1}{L}\int_{-L}^{L}f(x)\cos(\frac{n\pi x}{L})\diff{x} \\
  &= \frac{1}{2}\left\{\int_{-2}^{0}1\cos(\frac{n\pi x}{L})\diff{x}+
    \int_{0}^{2}x\cos(\frac{n\pi x}{L})\diff{x}\right\} \\
  &= \frac{1}{2}\left\{\bigg[\frac{2}{n\pi}\sin(\frac{n\pi x}{2})\bigg]_{-2}^0+
    \bigg[x\frac{2}{n\pi}\sin(\frac{n\pi x}{2})\bigg]_0^2-
    \int_0^2\frac{2}{n\pi}\sin(\frac{n\pi x}{2})\diff{x}\right\} \\
  &= \frac{1}{2}\left\{0+0+\frac{4}{n^2\pi^2}(\cos(n\pi)-1)\right\} \\
  &= \frac{2}{n^2\pi^2}(cos(n\pi)-1) \\
  &= \frac{2}{n^2\pi^2}(-1^n-1)
\end{align*}

\subsection*{Fourier Sine and Cosine Functions}
Suppose \( f(x) \) is piecewise smooth on \( 0<x<L \).
\begin{center}
  \begin{tikzpicture}
    \draw[thick,->] (0,0) -- (4,0) node[below]{x};
    \draw[thick,->] (0,0) -- (0,2) node[left]{y};
    \draw[red] (0,0) -- (1,0.5) node[above]{\( f(x) \)};
    \draw (1,-0.1) -- (1,0.1) node[below,yshift=-0.2cm]{\( L \)};
  \end{tikzpicture}
\end{center}
Because the period of \( \cos(\frac{n\pi x}{L}) \) and
\( \sin(\frac{n\pi x}{L}) \) is \( \frac{2L}{n} \), the Fourier series is 2L
periodic. We can come up with special cases of the Fourier series by
considering odd or even extensions of a function. We can define \( f_o(x) \) as
an odd 2L periodic extension of \( f(x) \),
\begin{center}
  \begin{tikzpicture}
    \draw[thick,<->] (-4,0) -- (4,0) node[below]{x};
    \draw[thick,<->] (0,-2) -- (0,2) node[left]{y};
    \draw[red] (-3,-0.5) -- (-1,0.5);
    \draw[red] (-1,-0.5) -- (1,0.5);
    \draw[red] (1,-0.5) -- (3,0.5);
    \draw (1,-0.1) -- (1,0.1) node[below,yshift=-0.2cm]{\( L \)};
    \draw (-1,-0.1) -- (-1,0.1) node[below,yshift=-0.2cm]{\( -L \)};
  \end{tikzpicture}
\end{center}
and \( f_e(x) \) as an even 2L periodic extension of \( f(x) \).
\begin{center}
  \begin{tikzpicture}
    \draw[thick,<->] (-4,0) -- (4,0) node[below]{x};
    \draw[thick,<->] (0,-2) -- (0,2) node[left]{y};
    \draw[red] (-3,0.5) -- (-2,0) -- (-1,0.5);
    \draw[red] (-1,0.5) -- (0,0) -- (1,0.5);
    \draw[red] (1,0.5) -- (2,0) -- (3,0.5);
    \draw (1,-0.1) -- (1,0.1) node[below,yshift=-0.2cm]{\( L \)};
    \draw (-1,-0.1) -- (-1,0.1) node[below,yshift=-0.2cm]{\( -L \)};
  \end{tikzpicture}
\end{center}
We define the Fourier Cosine Series (FCS) of \( f(x) \) on \( 0<x<L \) as
\begin{align*}
  f(x) &\sim \frac{a_0}{2}+\sum_{n=1}^{\infty}a_n\cos(\frac{n\pi x}{L}) \\
  a_0 &= \frac{2}{L}\int_{0}^{L}f(x)\diff{x} \\
  a_n &= \frac{2}{L}\int_{0}^{L}f(x)\cos(\frac{n\pi x}{L})\diff{x}
\end{align*}
If \( f \) is piecewise smooth on \( (0,L) \), the Fourier Cosine Series
converges to. The Fourier Sine Series (FSS) of \( f(x) \) on \( 0<x<L \) is
defined as
\[ f(x) \sim \sum_{n=1}^{\infty}b_n\sin(\frac{n\pi x}{L}) \]

\begin{center}
  You can find all my notes at \url{http://omgimanerd.tech/notes}. If you have
  any questions, comments, or concerns, please contact me at
  alvin@omgimanerd.tech
\end{center}

\end{document}
