\documentclass{math}

\usepackage{tikz}

\title{Boundary Value Problems}
\author{Alvin Lin}
\date{August 2018 - December 2018}

\begin{document}

\maketitle

\section*{Wave Equation in 1D}
Suppose have a string stretched between two endpoints \( x = 0 \) and
\( x = L \). \( y(x,t) \) is the transverse displacement from \( y = 0 \),
the resting or equilibrium position. At rest, \( y = 0 \) and \( y_t \), the
velocity of the string, is also 0. If we take the following assumptions:
\begin{itemize}
  \item no gravity
  \item string is perfectly flexible
  \item string has uniform properties (e.g. density)
  \item \( |y| << L \) and derivatives of \( y \) remain relatively small
  \item a uniform tension \( T \) is exerted on the string
  \item no other external forces
\end{itemize}
We can develop a partial differential equation using force balance for a
segment of the string. If we apply Newton's second law, letting \( \delta x\to0
\), we obtain the following equation.
\[ y_{tt} = a^2y_{xx} \]
where \( a^2 = \frac{|T|}{\rho} > 0 \) and \( \rho \) is the string density.
\( a \) is analogous to acceleration in the formula \( a = \frac{F}{m} \).
This is the wave equation in 1 dimension, but we will need two boundary
conditions and two initial conditions (unlike the heat equation which needed
one). Once we have those, it can be used to model longitudinal waves and water
waves in two dimensions or higher.

\subsection*{Boundary Conditions}
Suppose at one end of the string \( x = 0 \), the string is fixed to a pole.
This gives us the boundary condition \( y(0,t) = 0 \), a homogeneous Dirichlet
boundary condition. \par
Suppose instead that the string is attached to a ring that freely slides up and
down the pole, this would instead give us the boundary condition
\( \pdiff{y}{x}(0,t) = 0 \) or \( y_x(0,t) = 0 \), which is a homogeneous
Neumann boundary condition. Thus we have the partial differential equation:
\begin{align*}
  y_{tt} &= a^2y_{xx} \\
  y(0,t) &= y(L,t) = 0 \\
  y(x,0) &= f(x) \quad \text{(some initial displacement)} \\
  y_t(x,0) &= g(x) \quad \text{(some initial velocity)}
\end{align*}

\subsection*{Process}
Solving this follows the same process:
\begin{enumerate}
  \item Use separation of variables by substituting \( y = X(x)T(t) \).
  \item Solve the \( X \) based eigenvalue/eigenfunction problem to get
    some eigenvalue \( \lambda_n \) and its corresponding eigenfunction
    \( X_n(x) \).
  \item Solve the \( T \) based ordinary differential equation using
    \( \lambda_n \).
  \item Substitute back to get an equation for the modes.
  \[ y_n(x,t) = X_n(x)T_n(t) \]
  Use superposition to produce a formal solution.
  \[ y(x,t) = \sum_{n}y_n(x,t) \]
  where \( y_n \) contains some arbitrary constants dependent on \( n \).
  \item Apply the initial conditions. Use Fourier series methods to solve for
    the arbitrary constants and plug it back into \( y(x,t) \).
\end{enumerate}


\begin{center}
  You can find all my notes at \url{http://omgimanerd.tech/notes}. If you have
  any questions, comments, or concerns, please contact me at
  alvin@omgimanerd.tech
\end{center}

\end{document}
