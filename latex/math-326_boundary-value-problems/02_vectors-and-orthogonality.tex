\documentclass{math}

\title{Boundary Value Problems}
\author{Alvin Lin}
\date{August 2018 - December 2018}

\begin{document}

\maketitle

\section*{Vectors and Orthogonality}
\[ \vec{v} = \begin{bmatrix}2 \\ 4\end{bmatrix} \]
or
\[ \vec{v} = \begin{bmatrix}2 & 4\end{bmatrix} \]
Individual components of a vector are known as scalars, and are members of a
field such as \( \R \) or \( \mathbb{C} \).

\subsection*{Adding vectors}
\begin{align*}
  \vec{a} &= \langle1,4\rangle \\
  \vec{b} &= \langle3,2\rangle \\
  \vec{a}+\vec{b} &= \langle4,6\rangle
\end{align*}

\subsection*{Scalar Multiplication}
\begin{align*}
  3\vec{a} &= 3\langle1,4\rangle \\
  &= \langle3,12\rangle
\end{align*}

\subsection*{Magnitude}
Given \( \vec{a} = \langle a_1,a_2\rangle \):
\[ \|\vec{a}\| = \sqrt{(a_1)^2+(a_2)^2} \]
This is also known as the length or norm of a vector. If \( \vec{b}\in\R^n \):
\[ \|\vec{b}\| = \sqrt{\sum_1^n{(b_i)^2}} \]
A \textbf{unit vector} is a vector of magnitude 1.

\subsection*{Linear Combinations}
A \textbf{linear combination} of vectors \( \vec{a} \) and \( \vec{b} \) is
\[ \vec{v} = c_1\vec{a}+c_2\vec{b} \]
where each vector is multiplied by a constant \( c_i \) and added together.

\subsection*{Dot Product}
The dot product is a special case of an inner product where:
\begin{align*}
  \vec{a} &= \langle a_1,a_2,\dots,a_n \rangle \\
  \vec{b} &= \langle b_1,b_2,\dots,b_n \rangle \\
  \vec{a},\vec{b} &\in \R \\
  \vec{a}\cdot\vec{b} &= (\vec{a},\vec{b}) \\
  &= a_1b_1+a_2b_2+\dots+a_nb_n \\
  &= \sum_1^n{a_ib_i}
\end{align*}
The dot product is used to test for orthogonality. The dot product can also
be calculated using the following property:
\[ \vec{a}\cdot\vec{b} = \|\vec{a}\|\|\vec{b}\|\cos\theta \]
where \( \theta \) is the smaller of the two angles between \( \vec{a} \) and
\( \vec{b} \).

\subsection*{Orthogonality}
Any two vectors \( \vec{a} \) and \( \vec{b} \) are orthogonal if and only if
their dot product is zero. If a set of nonzero vectors
\( \{\vec{x_1},\dots,\vec{x_n}\} \) are all mutually orthogonal, that is
\[ \vec{x_i}\cdot\vec{x_j} = 0 \text{ for all } i\ne j \]
then the set of vectors is an \textbf{orthogonal set}. An \textbf{orthonormal
set} is an orthogonal set where all \( \vec{x_i} \) are unit vectors.
\[ \{\i,\j,\j\} \text{ is an orthonormal set} \]
\[ \{3\i,\j,\k\} \text{ is an orthogonal set but not orthonormal} \]

\subsection*{Orthogonal Functions}
We can treat functions like vectors. For \( f(x) = x^2 \), as
\( x = 0,1,2,\dots \), \( f(x) = 0,1,4,\dots \). This is an uncountably
infinite list of numbers which we cannot write down. We can take linear
combinations of functions (analogous to vectors). For example:
\[ g(x) = 3\sin(x)-2\cos(5x) \]
The function \textbf{norm} of \( f:\R\to\R \) for \( f \) defined on
\( x\in[a,b] \) is defined as:
\[ \|f(x)\| = \sqrt{\int_a^b(f(x))^2\diff{x}} \]
The \textbf{inner product} of two functions \( f \) and \( g \) is
defined as:
\[ (f,g) = (f(x),g(x)) = \int_a^bf(x)g(x)\diff{x} \]
If \( (f,g) = 0 \), then \( f \) and \( g \) are orthogonal on the interval
\( [a,b] \) and vice versa. Note that \( (f,f) = \|f\|^2 \). \par
The set of functions \( \{f_0(x),f_1(x),f_2(x),\dots\} \) is an orthogonal set
of functions if \( (f_m,f_n) = 0 \) whenever \( m\ne n \). If \( \{f_n\} \) is
an orthogonal set where \( \|f_n(x)\| = 1 \) for \( n = 0,1,2,\dots \) then
\( \{f_n\} \) is an orthonormal set.

\subsubsection*{Example}
Show that \( \{\cos(\frac{n\pi x}{L})\}_{n=0}^{\infty} = \{1,
\cos\frac{\pi x}{L},\cos\frac{2\pi x}{L},\dots\} \) (where \( L \) is a given
length greater than 0) is an orthogonal set on \( [-L,L] \). First we need to
check if
\[ (f_n,f_m) = \int_{-L}^{L}\cos\frac{m\pi x}{L}\cos\frac{n\pi x}{L}\diff{x} =
  0 \text{ wherever } m\ne n \]
We can exploit the even and odd properties of functions to solve this. The first
thing we should do is rewrite this function into a more familiar form using a
trigonometric identity.
\[ \cos(A)\cos(B) = \frac{1}{2}\bigg[\cos(A-B)+\cos(A+B)\bigg] \]
In the case where \( m\ne n \):
\begin{align*}
  (f_n,f_m) &= \int_{-L}^{L}\cos(\frac{m\pi x}{L})
    \cos(\frac{n\pi x}{L})\diff{x} \\
  &= \frac{1}{2}\left[\cos\left(\frac{(m-n)\pi x}{L}\right)+
    \cos\left(\frac{(m+n)\pi x}{L}\right)\right]\diff{x} \\
  &= \frac{1}{2}\bigg[\frac{L}{(m-n)\pi}
    \sin\left(\frac{(m-n)\pi x}{L}\right)\bigg]_{-L}^{L}+
    \frac{1}{2}\bigg[\frac{L}{(m+n)\pi}
    \sin\left(\frac{(m+n)\pi x}{L}\right)\bigg]_{-L}^{L} \\
  &= (\dots)(\sin((m-n)\pi)-\sin(-(m-n)\pi)) \\
  &= (\dots)(\sin(n\pi)) \text{ where } n\in\Z \\
  &= 0 \therefore \\
  (f_n,f_m) &= 0 \text{ for } m\ne n
\end{align*}
Therefore, \( \{\cos(\frac{n\pi x}{L})\}_{n=0}^{\infty} \) is orthogonal.
\[ (f_n,f_m) = \begin{cases}
  0, & m\ne n \\
  L, & m = n\ne 0 \\
  2L, & m = n = 0
\end{cases} \]
Since the inner product is 2L when \( m = n \), the set is not orthonormal
because there exists no \( L \) where \( L = 2L = 1 \). We can normalize the set
by choosing choefficients \( c_nf_n \) such that \( \|c_nf_n\| = 1 \). Since we
have already found the function norm, we know the following:
\[ \|f_n(x)\| = \sqrt{(f_n,f_n)} = \begin{cases}
  \sqrt{L}, & n>0 \\
  \sqrt{2L}, & n = 0
\end{cases} \]
Choosing \( c_n = \frac{1}{\|f_n(x)\|} \) yields us:
\[ \left\{\frac{1}{\sqrt{2L}},\frac{1}{\sqrt{L}}\cos(\frac{\pi x}{L}),
  \frac{1}{\sqrt{L}}\cos(\frac{2\pi x}{L}),\dots\right\} \]
which is an orthonormal set on \( [-L,L] \).

\subsubsection*{Shortcuts}
Recall that:
\begin{itemize}
  \item odd function \( \times \) odd function = even function
  \item even function \( \times \) even function = even function
  \item even function \( \times \) odd function = odd function
\end{itemize}
We can use this to quickly determine if two functions are orthogonal.
\[ (\sin(x),\cos(x)) = \int_{-L}^{L}\sin(x)\cos(x)\diff{x} \]
Because \( \sin(x) \) is odd and \( \cos(x) \) is even, their product is an odd
function. Thus their integral is 0 over any symmetric interval. We can use this
to quickly determine that they are orthogonal over \( [-L,L] \).

\subsection*{Fourier Series}
We will use the inner product and orthogonality to find the Fourier series
corresponding to \( f(x) \).
\[ f(x) = \frac{a_0}{2}+\sum_{n=1}^{\infty}(a_n\cos(\frac{n\pi x}{L})+
  b_n\sin(\frac{n\pi x}{L})) \]

\begin{center}
  You can find all my notes at \url{http://omgimanerd.tech/notes}. If you have
  any questions, comments, or concerns, please contact me at
  alvin@omgimanerd.tech
\end{center}

\end{document}
