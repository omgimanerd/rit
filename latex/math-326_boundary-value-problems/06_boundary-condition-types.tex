\documentclass{math}

\usepackage{tikz}

\title{Boundary Value Problems}
\author{Alvin Lin}
\date{August 2018 - December 2018}

\begin{document}

\maketitle

\section*{Boundary Condition Types}
\begin{center}
  \begin{tabular}{|c|p{4cm}|p{4cm}|}
    \hline
    & In terms of \( X(x) \) & In terms of \( u(x,t) \) \\
    \hline
    Dirichlet & \begin{align*}
      X(a) = c_1 \\
      X(b) = c_2
    \end{align*} & \begin{align*}
      u(a,t) = c_1 \\
      u(b,t) = c_2
    \end{align*} \\
    \hline
    Neumann & \begin{align*}
      X'(a) = c_1 \\
      X'(b) = c_2
    \end{align*} & \begin{align*}
      u_x(a,t) = c_1 \\
      u_x(b,t) = c_2
    \end{align*} \\
    \hline
    Separated & \begin{align*}
      a_1X(a)+a_2X'(a) = c_1 \\
      b_1X(b)+b_2X'(b) = c_2
    \end{align*} & \begin{align*}
      a_1u(a,t)+a_2u_x(a,t) = c_1 \\
      b_1u(b,t)+b_2u_x(b,t) = c_2
    \end{align*} \\
    \hline
    Periodic & \begin{align*}
      X(a) = X(b) \\
      X'(a) = X'(b)
    \end{align*} & \begin{align*}
      u(a,t) = u(b,t) \\
      u_x(a,t) = u_x(b,t)
    \end{align*} \\
    \hline
  \end{tabular}
\end{center}
Dirichlet, Neumann, and Robin boundary conditions are homogeneous if
\( c_1 = c_2 = 0 \) and nonhomogeneous if \( c_1 \ne 0 \) or \( c_2 \ne 0 \).
A combination of these can result however, for example:
\begin{align*}
  u(0,t) &= 0 \quad \text{homogeneous Dirichlet} \\
  u_x(L,t) &= 1 \quad \text{nonhomogeneous Neumann}
\end{align*}
Boundary conditions are usually considered separately for homogeneity.

\subsubsection*{Example}
In a previous section, we solved
\begin{align*}
  u_t &= \beta u_xx \\
  u(0,t) &= u(L,t) = 0 \\
  u(x,0) &= f(x)
\end{align*}
We arrived at the solution:
\begin{align*}
  u(x,t) &= \sum_{n=1}^{\infty}b_n
    \sin(\frac{n\pi x}{L})\e^{-\beta(\frac{n\pi}{L})^2t} \\
  b_n &= \frac{2}{L}\int_{0}^{L}f(x)\sin(\frac{n\pi x}{L})\diff{x}
\end{align*}
The individual terms when \( n \) is chosen are referred to as modes:
\begin{align*}
  u_n(x,t) &= \sin(\frac{n\pi x}{L})\e^{-\beta(\frac{n\pi}{L}^2t)} \\
  u_1(x,t) &= \sin(\frac{\pi x}{L})\e^{-\beta\frac{\pi}{L}t} \\
  u_2(x,t) &= \sin(\frac{2\pi x}{L})\e^{-\beta\frac{2\pi}{L}t}
\end{align*}
Due to the exponential term, the modes \( u_n \) approach 0 as \( t \)
approaches infinity.

\begin{center}
  You can find all my notes at \url{http://omgimanerd.tech/notes}. If you have
  any questions, comments, or concerns, please contact me at
  alvin@omgimanerd.tech
\end{center}

\end{document}
