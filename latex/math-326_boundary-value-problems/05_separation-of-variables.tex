\documentclass{math}

\usepackage{tikz}

\title{Boundary Value Problems}
\author{Alvin Lin}
\date{August 2018 - December 2018}

\begin{document}

\maketitle

\section*{Separation of Variables}
We need to know when a partial differential equation is separable. Suppose we
have the function \( u_t = 8u_{xx} \). To test for separability:
\begin{enumerate}
  \item Suppose \( u(x,t) = X(x)T(t) \).
  \item We need to substitute this into the partial differential equation.
  \item If you can isolate all instances of the \( x \) independent variable
    on one side and the \( t \) independent variable on the other side of the
    equation, then the partial differential equation is separable.
\end{enumerate}
\begin{align*}
  u_(x,t) &= X(x)T(t) \\
  \pdiff{}{t}(X(x)T(t)) &= 8\pdiff{^2}{x^2}(X(x)T(t)) \\
  X(x)\pdiff{}{t}T(t) &= 8\pdiff{^2}{x^2}(X(x))T(t) \\
  X(x)T'(t) &= 8X''(x)T(t) \\
  \frac{X(x)}{X''(x)} &= \frac{8T'(t)}{T(t)}
\end{align*}
In their separated form, we can set both sides equal to some separation constant
to solve two ordinary differential equations. This separation constant is
usually denoted \( \lambda, -\lambda, \alpha^2, \text{ or } -\alpha^2 \).
\[ \frac{X(x)}{X''(x)} = \frac{8T'(t)}{T(t)} = -\lambda \]
Separation is useful because we can break a partial differential equation into
multiple ordinary differential equations.

\subsubsection*{Example}
Generic form of the heat equation:
\[ \pdiff{u}{t} = \beta\pdiff{^2u}{x^2} \]
We want to find a solution that is nontrivial and bounded. It helps to narrow
things down if we have boundary conditions and/or initial conditions.
\[ u(0,t) = 0 \quad u(L,t) = 0 \quad u(x,0) = f(x) \]
We want to find \( u(x,t) \) that satisfies the partial differential equation,
its boundary conditions, and its initial condition. Since the equation is
separable, we can substitute \( u(x,t) = X(x)T(t) \) into the equation and show
that
\[ \frac{X''(x)}{X(x)} = \frac{T'(t)}{\beta T(t)} \]
If we choose the separation constant \( -\lambda \):
\begin{align*}
  \frac{X''(x)}{X(x)} &= -\lambda = \frac{T'(t)}{\beta T(t)} \\
  X''(x) &= -\lambda X(x) \\
  T'(t) &= -\lambda\beta T(t)
\end{align*}
With the boundary condition \( u(0,t) = 0 \), we know that \( X(0)T(t) = 0 \).
With the boundary condition \( u(L,t) = 0 \), we know that \( X(L)T(t) = 0 \).
\begin{align*}
  u(x,0) &= X(x)T(0) = f(x) \\
  X(x) &= \frac{f(x)}{T(0)}
\end{align*}
We can start with either ordinary differential equation:
\begin{align*}
  X''(x) &= -\lambda X(x) \quad X(0) = 0 \quad X(L) = 0 \\
  X''(x)+\lambda X(x) &= 0 \\
  r^2+\lambda &= 0
\end{align*}
Using this characteristic equation, we have three cases:
\begin{enumerate}
  \item \( \lambda < 0 \)
  \begin{align*}
    r^2 &= -\lambda \\
    r_{1,2} &= \pm\sqrt{-\lambda} \\
    X(x) &= c_1\e^{\sqrt{-\lambda}x}+c_2\e^{-\sqrt{\lambda}x} \\
    X(0) &= 0 = c_1\e^{0}+c_2\e^{0} \\
    c_1 &= -c_2 \\
    X(L) &= 0 = c_1(\e^{\sqrt{-\lambda}L}-\e^{-\sqrt{\lambda}L}) \\
    0 &= c_1(\e^{2\sqrt{-\lambda}L}-1) \\
    c_1 &= 0 = c_2 \\
    X(x) &= 0 \quad \text{(trivial solution)}
  \end{align*}
  \item \( \lambda = 0 \)
  \begin{align*}
    X''(x) &= 0 \\
    r^2 &= 0 \\
    r &= 0 \\
    X(x) &= c_1\e^{0t}+c_2x\e^{0t} \\
    &= c_1+xc_2 \\
    X(0) &= 0 = c_1 \\
    c_1 &= 0 \\
    X(L) &= xc_2 \\
    c_2 &= 0 \\
    X(x) &= 0 \quad \text{(trivial solution)}
  \end{align*}
  \item \( \lambda > 0 \)
  \begin{align*}
    r^2+\lambda &= 0 \\
    r_{1,2} &= \pm\sqrt{-\lambda} \\
    \alpha^2 &= -\lambda \\
    r_{1,2} &= \pm i\alpha = \pm i\sqrt{\lambda} \\
    X(x) &= c_1\cos(\alpha x)+c_2\sin(\alpha x) \\
    X(0) &= 0 = c_1\cos(0)+c_2\sin(0) \\
    c_1 &= 0 \\
    X(L) &= 0 = c_2\sin(\alpha L)
  \end{align*}
  If \( c_2 \) is 0, then the solution is again trivial. If it is not zero,
  then the solution depends on the separation constant.
  \begin{align*}
    \sin(\alpha L) &= 0 \\
    \alpha L &= n\pi, \quad n\in\Z \\
    \alpha &= \frac{n\pi}{L} \\
    X(x) &= c_2\sin(\frac{n\pi}{L}x)
  \end{align*}
\end{enumerate}
To produce a non-trivial solution for \( X(x) \), we needed to use
\( \sqrt{\lambda} = \frac{n\pi}{L} \). We can use this to solve the second
ordinary differential equation.
\begin{align*}
  T'(t) &= -\lambda\beta T(t) \\
  T'(t) &= -\frac{n^2\pi^2}{L^2}\beta T(t) \\
  T(t) &= c_2\e^{ct} \\
  T'(t) &= \ddiff{}{t}(T(t)) \\
  &= c_2c\e^{ct} \\
  &= cT(t) \\
  T(t) &= c_3\e^{-\frac{n^2\pi^2}{L^2}\beta t}
\end{align*}
We can substitute this back into \( u(x,t) = X(x)T(t) \) to find our solution.
\begin{align*}
  u(x,t) &= X(x)T(t) \\
  &= c_2\sin(\frac{n\pi x}{L})c_2\e^{-\frac{n^2\pi^2}{L^2}\beta t} \\
  u_n(x,t) &= c_n\sin(\frac{n\pi x}{L})\e^{-(\frac{n\pi x}{L})^2\beta t}
\end{align*}
This has infinitely many solutions because haven't applied our initial condition
yet.

\subsection*{Eigenvalues and eigenfunctions}
We found special \( X(x) \) and \( \lambda \) values that made the boundary
value problem solutions non-trivial. These are known as eigenfunctions:
\[ X_n(x) = c_n\sin(\frac{n\pi x}{L}) \]
\[ \lambda_n = (\frac{n\pi}{L})^2 = \text{eigenvalues of the BVP} \]
We can associate eigenvalues and eigenfunctions with operators.
\begin{align*}
  X''+\lambda X &= 0 \\
  X'' &= -\lambda X \\
  LX &= -\lambda X
\end{align*}
Problems that require you to find the eigenvalues and eigenfunctions of
boundary value problems require you to find all non-trivial \( y_n \)
solutions.

\begin{center}
  You can find all my notes at \url{http://omgimanerd.tech/notes}. If you have
  any questions, comments, or concerns, please contact me at
  alvin@omgimanerd.tech
\end{center}

\end{document}
