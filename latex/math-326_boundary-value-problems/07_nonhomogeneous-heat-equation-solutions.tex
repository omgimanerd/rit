\documentclass{math}

\usepackage{tikz}

\title{Boundary Value Problems}
\author{Alvin Lin}
\date{August 2018 - December 2018}

\begin{document}

\maketitle

\section*{Nonhomogeneous Heat Equation Solutions}
An example of a homogeneous heat equation is
\begin{align*}
  u_t &= a^2u_{xx} \\
  u(0,t) &= u(L,t) = 0 \\
  u(x,0) &= f(x)
\end{align*}
To make the equation nonhomogeneous:
\begin{align*}
  u_t &= a^2u_{xx}+P(x) \\
  u(0,t) &= u_1 \\
  u(L,t) &= u_2 \\
  u(x,0) &= f(x)
\end{align*}
where \( P(x) \) is a heat source or sink. If \( u_1 \ne 0 \) or
\( u_2 \ne 0 \) or \( P(x)\ne0 \) then the boundary value problem is
nonhomogeneous. The simpler case of this involves \( u_1 = u_2 \ne 0 \),
in which case we will define \( v(x,t) = u(x,t)-u_1 \), which can substitute
back into the boundary value problem. The harder case involves \( u_1 \ne u_2 \)
or \( P(x) \ne 0 \).

\subsection*{Methods for Solving}
\begin{itemize}
  \item Substitute \( u(x,t) = \psi(x)+v(x,t) \) into the partial differential
    equation, boundary conditions, and initial condition.
  \item Solve the ordinary differential equation for \( \psi(x) \).
  \item Solve the homogeneous partial differential equation for \( v(x,t) \).
  \item Substitute back into \( u(x,t) = \psi(x)+v(x,t) \).
\end{itemize}

\subsubsection*{Interpretation}
\[ u(x,t) = \psi(x)+v(x,t) \]
\( \psi(x) \) represents a steady-state solution that satisfies nonhomogeneous
terms while \( v(x,t) \) is a transient solution that dies off as \( t \)
approaches infinity.

\subsubsection*{Example}
\begin{align*}
  u_t &= u_{xx} \quad 0<x<\pi \\
  u(0,t) &= 0 \\
  u(\pi,t) &= 3\pi \\
  u(x,0) &= f(x) = 0
\end{align*}
We want to find a solution of the form \( u(x,t) = \psi(x)+v(x,t) \) so we will
substitute this into the partial differential equation.
\begin{align*}
  \pdiff{}{t}(u) &= \pdiff{}{t}(\psi(x)+v(x,t)) =
    a^2\pdiff{^2}{x^2}(\psi(x)+v(x,t)) \\
  \pdiff{\psi(x)}{t}+v_t(x,t) &= a^2(\psi''(x)+v_{xx}(x,t)) \\
  v_t(x,t) &= a^2(\psi''(x)+v_{xx}(x,t)) \\
  u(0,t) &= \psi(0)+v(0,t) = u_1 \\
  u(L,t) &= \psi(L)+v(L,t) = u_2 \\
  u(x,0) &= \psi(x)+v(x,0) = f(x)
\end{align*}
Let \( t\to\infty \) to obtain the steady state boundary value problem:
\begin{align*}
  v_t &= a^2(\psi''+v_xx) \\
  0 &= a^2\psi'' \\
  \psi(0)+v(0,t) &= u_1 \\
  \psi(0) &= u_1 \\
  \psi(L)+v(L,t) &= u_2 \\
  \psi(L) &= u_2
\end{align*}
We now have a steady state boundary value problem which we can solve:
\begin{align*}
  a^2\psi''(x) &= 0 \quad (a^2>0) \\
  \psi(0) &= u_1 \\
  \psi(L) &= u_2 \\
  \psi(x) &= Ax+B \\
  \psi(0) &= u_1 = A(0)+B = B \\
  u_1 &= B \\
  \psi(L) &= u_2 = AL+B = AL+u_1 \\
  AL &= u_2-u_1 \\
  A &= \frac{u_2-u_1}{L} \\
  \psi(x) &= (\frac{u_2-u_1}{L})x+u_1
\end{align*}
We can plug \( \psi(x) \) back into the boundary value problem to solve for
\( v(x,t) \) for arbitrary \( t>0 \).
\begin{align*}
  v_t &= a^2(\psi''+v_{xx}) \\
  \psi(0)+v(0,t) &= u_1 \\
  u_1+v(0,t) &= u_1 \\
  v(0,t) &= 0 \\
  \psi(L)+v(L,t) &= u_2 \\
  u_2+v(L,t) &= u_2 \\
  v(L,t) &= 0 \\
  u(x,0) &= v(x,0)+\psi(x) = f(x) \\
  v(x,0) &= f(x)-\psi(x)
\end{align*}
We can now solve the homogeneous boundary value problem for \( v(x,t) \).
\begin{align*}
  v_t &= a^2v_{xx} \\
  v(0,t) &= v(L,t) = 0 \\
  v(x,0) &= f(x)-\psi(x) \\
  &= f(x)-(\frac{u_2-u_1}{L})x+u_1
\end{align*}
Note that this homogeneous boundary value problem is in the same form as the
Dirichlet boundary value problem we have already solved.
\begin{align*}
  v(x,t) &= \sum_{n=1}^{\infty}
    b_n\sin(\frac{n\pi x}{L})\e^{-a^2(\frac{n\pi}{L})^2t} \\
  b_n &= \frac{2}{L}
    \int_{0}^{L}(f(x)-(\frac{u_2-u_1}{L})-u_1)\sin(\frac{n\pi x}{L})\diff{x} \\
  u(x,t) &= \psi(x)+v(x,t) \\
  &= (\frac{u_2-u_1}{L})x+u_1+v(x,t)
\end{align*}

\begin{center}
  You can find all my notes at \url{http://omgimanerd.tech/notes}. If you have
  any questions, comments, or concerns, please contact me at
  alvin@omgimanerd.tech
\end{center}

\end{document}
