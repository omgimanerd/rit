\documentclass{math}

\geometry{letterpaper, margin=0.5in}

\title{Introduction to Intelligent Systems: Homework 1}
\author{Alvin Lin}
\date{August 2017 - December 2017}

\begin{document}

\maketitle

\subsection*{Problem 1}
Describe and explain the difference between:
\begin{enumerate}
  \item Strong and Weak AI: Strong AI is the school of thought that a
  sufficiently well programmed computer can be intelligent and conscious in the
  way that a human is. Weak AI is concerned with the application of
  intelligent methods to efficiently solve problems with computers.
  \item Strong and Weak Methods: Strong methods rely on practical world
  knowledge to solve problems, while weak methods use logical analysis instead.
  (Also very analogous to functional vs imperative programming).
  \item Neats and Scruffies: Neats think that AI study should be rigorously
  and mathematically backed by proofs and logic, while scruffies approach AI
  with a trial and error perspective that focuses on AI that works best in the
  real world. These attitudes are similar and analogous to the functional and
  imperative programming paradigms, since a functional approach to programming
  is very similar to a neat's approach to AI.
\end{enumerate}

\subsection*{Problem 2}
Many of the computational models of cognitive activities that have been
proposed involve quite complex mathematicaly operations, such as convolving an
image with a Gaussian or finding a minimum of the entropy function. Most human
(and certainly all animals) never learn this kind of mathematics at all, almost
no one learns it before college, and almost no one can compute the convolution
of a function with a Gaussian in their head. What sense does it make to say that
the ``vision system'' doing this kind of mathematics, whereas the actual person
has no idea how to do it?
\par Computer vision algorithms operate on vast amounts of linear algebra and
perform a very large number of complex calculations to perform a task such as
identifying chairs in a scene. In contrast, humans can do it ``effortlessly''
without thinking or knowing about any math, since the complexity of their
action is abstracted away in the activity of their neurons. A person identifying
chairs in a scene is not aware of the complexity of the neural activity
required to perform the action.

\subsection*{Problem 3}
``Surely computers cannot be intelligent - they can only do what their
programmers tell them.'' Is the latter statement true, does it imply the former?
\par An interesting question arises if we consider the situation where a
programmer tells a computer to be intelligent. The latter statement is true
since computers must be programmed, but whether or not it implies the former
depends on our definition of intelligent. Is a computer intelligent because it
can do something their programmer did not tell them to do? This is possible
since computers can be programmed with models that learn from and affect the
environment to try and accomplish a task, which can lead to the computer doing
something the programmer did not \textit{explicitly} tell it to do.

\subsection*{Problem 4}
``Surely animals cannot be intelligent - they can only do what their genes tell
them.'' Is the latter statement true, and does it imply the former?
\par The latter statement is true since an animal's nature and actions (to some
degree) are determined by their genetics. Animals also act, respond, and
adapt to stimuli in their environment in order to survive. One can argue that
this adaptive behavior is a result of genetic ``programming'', but this does
not imply that their behavior \textit{not} intelligent. Animals can be
intelligent in this capacity because they are programmed by their genes to be
intelligent.

\subsection*{Problem 5}
List and describe at least 5 techniques that can be used to cope with
intractability (they do not have to be AI techniques per se) and for each
technique, explain the advantages and disadvantages of the approach.
\begin{itemize}
  \item Restricting the problem domain: this technique narrows the problem
  domain so that only a small subset of inputs need to be considered. This
  technique works well for localized problems where the input is known to be
  in a small range since a naive algorithm can be used to solve the problem,
  but it cannot scale or be generalized to solve similar problems of higher
  complexity.
  \item Brute force: this technique tries to solve a problem by testing every
  single possible solution. This technique is guaranteed to find a solution
  where the solution space is known, but it will generally take an unreasonable
  amount of time to do so. Brute forcing also requires massive amounts of
  computing resources, and may not work for problems where the solution space
  is not finite.
  \item Approximation: using an approximation algorithm to solve a problem will
  yield near-optimal solutions in a reasonable amount of time. The advantages
  of this technique are that for problems where small errors are not important,
  reasonable solutions can be found in reasonable amounts of time. However,
  approximation algorithms do not give exact answers and may be very inaccurate
  for certain domains of inputs.
  \item Heuristics: using a heuristic allows for the brute force technique to be
  applied in a reasonable amount of time. Heuristics allow for the solution
  space to be reduced based on a set of rules. This allows for a solution to be
  found much faster.
  \item Picking randomly: using randomness to optimize a search pattern is fast
  and inexpensive, but the solution time becomes unknown and the algorithm
  may take much longer to converge towards a solution.
\end{itemize}

\begin{center}
  If you have any questions, comments, or concerns, please contact me at
  alvin@omgimanerd.tech
\end{center}

\end{document}
