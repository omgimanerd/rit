\documentclass{math}

\usepackage{enumerate}
\usepackage{forest}
\usepackage{listings}

\geometry{letterpaper, margin=0.5in}

\title{Introduction to Intelligent Systems: Homework 2}
\author{Alvin Lin}
\date{August 2017 - December 2017}

\begin{document}

\maketitle

\subsection*{Problem 1}
For each of the following, gives a PEAS description of the task and given solver
of the tasks. There may be several reasonable answers, but the key is that all
four parts of your answer go together.
\begin{enumerate}[(a)]
  \item Robots play soccer. The task environment is (usually) fully observable
  via sensors on the robot (though from personal experience they're usually
  notoriously unreliable). By the nature of the game, the environment is multi
  agent, stochastic, sequential, dynamic, and continuous. The rules of the game
  are set, so the robots operate in a known environment.
  \begin{center}
    \begin{tabular}{|p{1cm}|p{4cm}|p{4cm}|p{3cm}|p{4cm}|}
      \hline
      Agent Type & Performance Measure & Environment & Actuators & Sensors \\
      \hline
      Soccer robot &
        Goals scored, ball possession time, acquisition speed &
        Soccer field, other robots, soccer ball &
        Motors for steering, movement, and kicking &
        Infrared (usually for the ball), motor encoders, direction sensors,
        ultrasonic distance sensors \\
      \hline
    \end{tabular}
  \end{center}
  \item Netflix/Amazon on-line recommendation system. The task environment is
  partially observable, so it is necessary to keep state information about a
  user's previous choices. The recommendation system operates as a single agent.
  Because it interacts with a human as a feedback mechanism, the environment is
  uncertain, episodic, static, and discrete. Human preferences can vary
  according to other factors, so the environment is unknown.
  \begin{center}
    \begin{tabular}{|p{3cm}|p{4cm}|p{3cm}|p{3cm}|p{3cm}|}
      \hline
      Agent Type & Performance Measure & Environment & Actuators & Sensors \\
      \hline
      Recommendation System &
        User engagement, retention, and click-through &
        The collection of possible shows a user can watch &
        Interface where suggestions are displayed &
        User clicks, mouse movements, and engagement \\
      \hline
    \end{tabular}
  \end{center}
  \item Expert system for medical diagnosis. The task environment is partially
  observable since it is not feasible to have all the data on a person at any
  given time. The system is single agent and interacts with a stochastic,
  sequential, dynamic, continuous, and unknown environment.
  \begin{center}
    \begin{tabular}{|p{3cm}|p{4cm}|p{3cm}|p{3cm}|p{3cm}|}
      \hline
      Agent Type & Performance Measure & Environment & Actuators & Sensors \\
      \hline
      Recommendation System &
        Diagnosis accuracy and patient resolution accuracy &
        The symptoms, medical history, and predispositions of a patient &
        Interface where diagnoses and suggestions are displayed &
        Database of patient data, x-rays machines, MRI scans and other tests
        performed on the patient \\
      \hline
    \end{tabular}
  \end{center}
\end{enumerate}

\subsection*{Problem 2}
Show the execution of the following search techniques on the tree shown below.
A is the root node and M is the goal state. For BFS, and DFS, keep track of the
open and closed arrays at each step of the search. If a node has more than one
child, add them to the open list in the left-to-right order as they as shown
in the tree (not alphabetical search). For IDS, give a list of the states that
are examined for each limit (starting with limit 0), in the order that the
states are examined.
\begin{enumerate}[(a)]
  \item Breadth-First Search
  \begin{center}
    \begin{tabular}{|c|c|}
      \hline
      open & closed \\ \hline
      A & \\ \hline
      B,C,D & A \\ \hline
      C,D,E,F & B,A \\ \hline
      D,E,F,G & C,B,A \\ \hline
      E,F,G,H,I & D,C,B,A \\ \hline
      F,G,H,I,J,K,L & E,D,C,B,A \\ \hline
      G,H,I,J,K,L & F,E,D,C,B,A \\ \hline
      M is a child of G & G,F,E,D,C,B,A \\ \hline
    \end{tabular}
  \end{center}
  \item Depth-First Search
  \begin{center}
    \begin{tabular}{|c|c|}
      \hline
      open & closed \\ \hline
      A & \\ \hline
      B,C,D & A \\ \hline
      E,F,C,D & B,A \\ \hline
      J,K,L,F,C,D & E,B,A \\ \hline
      K,L,F,C,D & J,E,B,A \\ \hline
      L,F,C,D & K,J,E,B,A \\ \hline
      F,C,D & L,K,J,E,B,A \\ \hline
      C,D & F,L,K,J,E,B,A \\ \hline
      G,D & C,F,L,K,J,E,B,A \\ \hline
      M is a child of G & G,C,F,L,K,J,E,B,A \\ \hline
    \end{tabular}
  \end{center}
  \item Iterative Deepening
  \begin{center}
    \begin{tabular}{|c|c|}
      \hline
      limit & searched \\ \hline
      0 & A \\ \hline
      1 & A,B,C,D \\ \hline
      2 & A,B,E,F,C,G,D,H,I \\ \hline
      3 & A,B,E,J,K,L,F,C,G,M \\ \hline
    \end{tabular}
  \end{center}
\end{enumerate}

\subsection*{Problem 3}
There is a famous problem similar to the missionaries and cannibals from the
notes: A farmer has to get a fox, a chicken, and a sack of corn across a river.
The farmer has a rowboat that can only carry one thing at a time (in addition to
the farmer). If the fox and the chicken are left together, the fox will eat the
chicken. If the chicken and the corn are left together, the chicken will eat the
corn. How does the farmer get everyone across the river?
\begin{enumerate}[(a)]
  \item Give a complete problem formulation for this problem. Choose a
  formulation that is precise enough to be implemented. Formulate a scheme for
  the problem similar to the one from the notes, including an initial state,
  goal state, and solution sequence.
  \par To represent the problem, I will use a version more familiar to me
  involving a wolf instead of a fox, sheep instead of a chicken, and cabbage
  instead of corn (this allows me to abbreviate the wolf as W, the sheep as S,
  the cabbage as C, and the farmer as F, to avoid naming conflicts). The two
  sides of the river will be separated with a horizontal dash (-). The start
  state is represented with WSCF-, and the end state is represented with -WSCF.
  Since the wolf should not be left alone with the sheep, and the sheep should
  not be left alone with the cabbage, the following states are examples of
  forbidden states: WS-CF, SC-WF, CF-WS. Solution:
  \begin{center}
    \begin{tabular}{|c|c|c|}
      \hline
      step & state & instruction \\ \hline
      0 & \texttt{WSCF-} & start state \\ \hline
      1 & \texttt{WC-SF} & farmer brings sheep \\ \hline
      2 & \texttt{WCF-S} & farmer returns alone \\ \hline
      3 & \texttt{W-SCF} & farmer brings cabbage \\ \hline
      4 & \texttt{WSF-C} & farmer returns with sheep \\ \hline
      5 & \texttt{S-WCF} & farmer brings wolf \\ \hline
      6 & \texttt{SF-WC} & farmer returns alone \\ \hline
      7 & \texttt{-WSCF} & farmer brings sheep \\ \hline
    \end{tabular}
  \end{center}
  \item Show the entire search tree for the farmer problem. Remove illegal
  states, and duplicate states.
  \begin{center}
    \begin{forest}
      [WSCF-
        [WC-SF
          [WCF-S
            [W-SCF
              [WSF-C
                [S-WCF
                  [SF-WC
                    [-WSCF]
                  ]
                ]
              ]
            ]
            [C-WSF
              [SCF-W
                [\( \Longleftarrow \)
                ]
              ]
            ]
          ]
        ]
      ]
    \end{forest}
  \end{center}
\end{enumerate}

\subsection*{Problem 4}
For each of the following assertions, say whether it is true or false and
support your answer with examples or counterexamples where appropriate.
\begin{enumerate}[(a)]
  \item An agent that senses only partial information about the state cannot be
  perfectly rational. False, rational agents can operate in a partially
  observable environment. Autonomous cars can be rational agents by only acting
  on the information they can obtain from their sensors.
  \item There exist task environments in which no pure reflex agent can behave
  rationally. True, reflex agents act based on rules and the current state of
  the world. They would not be able to perform rationally for environments where
  state memory is required, such as Netflix recommendation.
  \item There exists a task environment in which every agent is rational. True,
  in the simplest possible state where only one action (or inaction) is
  possible, every agent would be rational.
  \item The input to an agent program is the same as the input to the agent
  function. Sometimes false, it depends whether or not the input function needs
  to perform preprocessing on the input or combine it with state information.
  \item Every agent function is implementable by some program/machine
  combination. False, it is possible for some environment to require infinite
  memory if each state requires memory and the rational agent is run forever.
  This cannot be overcome by any heuristic.
  \item Every agent is rational in an unobservable environment.
  False, in an observable environment, it is impossible for some agents to
  maximize performance without any data from which to make a decision.
  \item A perfectly rational poker-playing agent never loses. False, someone has
  to lose if a perfectly rational agent plays another perfectly rational agent.
  There are situations in poker where a loss will result no matter what decision
  is taken.
\end{enumerate}

\subsection*{Problem 5}
Write pseudocode agent programs for the following agents:
\begin{enumerate}[(a)]
  \item goal-based agent
  \begin{lstlisting}
  def action(agent, environment):
    let state = agent.sense(environment)
    for action in agent.availableActions:
      let resultingState = agent.do(action, state)
      if resultingState == agent.goal:
        return action
  \end{lstlisting}
  \item utility-based agent
  \begin{lstlisting}
  def action(agent, environment):
    let state = agent.sense(environment)
    let maxUtility = -Infinity
    let maxUtilityState = None
    for action in agent.availableActions:
      let resultingState = agent.do(action, state)
      let resultingStateUtility = agent.getUtility(resultingState)
      if resultingStateUtility > maxUtility:
        maxUtility = resultingStateUtility
        maxUtilityState = resultingState
    return maxUtilityState
  \end{lstlisting}
\end{enumerate}

\subsection*{Problem 6}
Give a complete problem formation for each of the following. Choose a
formulation that is precise enough to be implemented (this includes: initial
state, goal test, actions, path cost, and potentially a solution if a problem
is specific enough to be solved).
\begin{enumerate}
  \item Using only four colors, you have to color a map in such a way that no
  two adjacent regions have the same color.
  \par The initial state can be created by picking a random color for a random
  section.  It does not matter what color is selected for the initial state
  because if a solution exists, then the starting section can be of any color
  and a solution with exist for it. After that, a search tree can be created
  where the branches are possible colors for the adjacent sections. Each branch
  of the tree would represent a possible action from that point. In our
  search tree, we can prune out all invalid branches where adjacent colors are
  the same. In a map with \( n \) sections, this search tree with have a height
  of \( n \) with a maximum branching factor on any level equal to the maximum
  number of adjacent sections to any given section on the map. We can use a
  depth first search to search through this tree (though it will take some
  time and is not very efficient). A goal test would check if all regions are
  colored such that no two adjacent regions are the same colors.
  \item A 3-foot-tall monkey is in a room where some bananas are suspended from
  the 8-foot feiling. The monkey would like to get the bananas. The room
  contains two stackable, movable, climbable 3-foot-high-crates.
  \par The initial state would be the problem as described. Actions that the
  monkey can do are: move itself, move the crates, stack the crates, climb the
  creates, and acquire the bananas (given the prerequisite state that the
  monkey is at a height where it can reach the bananas). A goal test would be
  checking if the monkey is currently at a height where it can reach the
  bananas, while the current height of the monkey relative to the ground could
  be used as a metric of how close it is to the goal. The number of actions
  taken by the monkey could be used as a path cost. If the monkey stacks the
  two crates and climbs on top of them, then it has reached the goal state.
  \item You have two jugs, measuring 5 gallons and 3 gallons, and a water
  faucet. You can fill the jugs up or empty them out (completely) from one to
  another or onto the ground. You need to end up with exactly 4 gallons in the
  larger jug.
  \par The initial state would have both jugs empty. Possible actions are: fill
  the large jug from the faucet, fill the small jug from the faucet, empty the
  large jug onto the ground, empty the small jug onto the ground, pour the
  large jug into the small jug, or pour the small jug into the large jug. A goal
  test would check if the large jug is currently holding 4 gallons, while path
  cost could be determined by the number of actions taken. The problem could be
  solved as follows:
  \begin{center}
    \begin{tabular}{|c|c|c|}
      \hline
      Small Jug & Large Jug & Action \\ \hline
      0/3 & 0/5 & Initial State \\ \hline
      0/3 & 5/5 & Fill the large jug from the faucet \\ \hline
      3/3 & 2/5 & Pour the large jug into the small jug \\ \hline
      0/3 & 2/5 & Empty the small jug \\ \hline
      2/3 & 0/5 & Pour the large jug into the small jug \\ \hline
      2/3 & 5/5 & Fill the large jug from the faucet \\ \hline
      3/3 & 4/5 & Pour the large jug into the small jug \\ \hline
    \end{tabular}
  \end{center}
  This solution can also be determined by using a search tree.
  \item You are able to pick from coins in three denominations: 3 cent, 7 cent,
  and 12 cent. What is the largest amount that you \textit{cannot} represent
  with these coins?
  \par (Frobenius coin problem - NP hard). An initial state for this problem
  would be having no coins. Each step, you can pick one of the three
  denominations to add to your current amount as an action. For a goal test,
  one would check if the current amount is equal to the desired amount. Path
  cost would be determined by the number of steps taken where a coin is chosen
  during each step. This problem must be brute forced, so a search tree could
  be created from the initial state to simulate every possible combination of
  1 coin, 2 coins, 3 coins, and so forth, with the \( n^{th} \) level of the
  tree representing any combination of \( n \) coins. This tree would be
  searched downward for the largest integer not included in any of its branches
  that was unable to be produced by a further branch.
\end{enumerate}

\begin{center}
  If you have any questions, comments, or concerns, please contact me at
  alvin@omgimanerd.tech
\end{center}

\end{document}
