\documentclass{math}

\usepackage{enumerate}
\usepackage{multirow}

\geometry{letterpaper, margin=0.5in}

\title{Introduction to Intelligent Systems: Homework 4}
\author{Alvin Lin - Section 1}
\date{August 2017 - December 2017}

\begin{document}

\maketitle

\subsection*{Problem 1}
It's February 1943, the Island of New Guinea during World War II. The Allies
control the southern coast of the island, the Japanese control the northern
coast. The Japanese are bringing in reinforcements - lost of them - from China
and Japan. They've already reached the city of Rabaul. Rabaul is on the eastern
tip of an island called New Britain. The troops massed in Rabaul tended for the
city of Lae in New Guinea. They'll be convoyed by ship, and everybody knows it.
What's not clear is what route they'll take. New Britain runs east-west, so the
trip will be either along the north coast of New Britain or the southern coast.
Either way, it's a three-day trip. The significant difference is the expected
weather. To the south, the weather is clear; to the north, the weather is
stormy. The Japanese fleet is fine with either kind of weather.
\par It's the Allies who care about the weather. They're going to send recon
planes to find the fleet, then bombers to bomb it. In clear weather, both of
these things can be done on the same day. In bad weather, the bombers go out a
day after the enemy is sighted. That cuts down on bombing days, and of course,
if you don't find the fleet, you can't bomb it at all. The recon planes can't
search both routes on the same day. All of this information is known to both
sides. Which route (north or south) should the Japanese commanders sail, and
which route (north or south) should the Allied Forces commanders search?
\begin{center}
  \begin{tabular}{|p{4cm}|p{6cm}|p{6cm}|}
    \hline
    & Allies search north (stormy) & Allies search south (clear) \\
    \hline
    Japanese sail north (stormy) &
      Japanese fleet found on first day and bombed on second and third days &
      Japanese fleet not found on the first day, found on second day, and
        bombed on the third day \\
    \hline
    Japanese sail south (clear) &
      Japanese fleet found on second day and bombed on the second and third
        days &
      Japanese fleet found on first day and bombed all three days \\
    \hline
  \end{tabular}
\end{center}
In terms of days of safety/bombing:
\begin{center}
  \begin{tabular}{|p{4cm}|p{6cm}|p{6cm}|}
    \hline
    & Allies search north (stormy) & Allies search south (clear) \\
    \hline
    Japanese sail north (stormy) &
      Japanese = 1, Allies = 2 &
      Japanese = 2, Allies = 1 \\
    \hline
    Japanese sail south (clear) &
      Japanese = 1, Allies = 2 &
      Japanese = 0, Allies = 3 \\
    \hline
  \end{tabular}
\end{center}
The Japanese should sail to the north, and the Allied commanders should search
the northern route as well.

\subsection*{Problem 2}
A dominant strategy is a Nash equilibrium. Is a Nash equilibrium necessarily a
dominant strategy? (If your answer is ``yes'', prove this. If your answer is
``no'' provide a counter-example.)
\par Yes. It is irrational for a dominated strategy to be played and thus they
cannot be part of Nash equilibrium. A Nash equilibrium will always consist of
both player's dominant strategy (strong or weak).

\subsection*{Problem 3}
Two vendors are selling breakfast items. The price depends on demand. The fixed
cost for making breakfast sandwiches is \$2; the fixed cost for a danish is \$1.
Payoffs are repesented as profits for danish and sandwich sellers respectively.
\begin{center}
  \begin{tabular}{|l||c|c|c|c|c}
    \hline
    & Sandwich: \$3 & Sandwich: \$4 & Sandwich: \$5 & Sandwich: \$6 \\ \hline
    \hline
    Danish: \$3 & $D=40,S=12$ & $D=44,S=22$ & $D=48, S=28$ & $D=39,S=18$ \\
    \hline
    Danish: \$4 & $D=36,S=14$ & $D=42,S=24$ & $D=46,S=32$ & $D=38,S=24$ \\
    \hline
    Danish: \$5 & $D=16,S=17$ & $D=24,S=26$ & $D=32,S=36$ & $D=40,S=16$ \\
    \hline
    Danish: \$6 & $D=32, S=18$ & $D=48,S=25$ & $D=50, S=29$ & $D=41, S=32$ \\
    \hline
    Danish: \$7 & $D=42, S=19$ & $D=36,S=18$ & $D=49, S=17$ & $D=42, S=28$ \\
    \hline
  \end{tabular}
\end{center}
\begin{enumerate}[(a)]
  \item Find the Nash Equilibrium using Iterated Elimination of Dominated
  Strategies (IEDS). Show your work with each iteration by specifying which
  row/column is dominated and eliminated. \\
  Sandwich: \$3 and Sandwich: \$4 are eliminated first because they are
  dominated by either Sandwich: \$5 or Sandwich: \$6.
  \begin{center}
    \begin{tabular}{|l||c|c|}
      \hline
      & Sandwich: \$5 & Sandwich: \$6 \\ \hline
      \hline
      Danish: \$3 & $D=48, S=28$ & $D=39,S=18$ \\
      \hline
      Danish: \$4 & $D=46,S=32$ & $D=38,S=24$ \\
      \hline
      Danish: \$5 & $D=32,S=36$ & $D=40,S=16$ \\
      \hline
      Danish: \$6 & $D=50, S=29$ & $D=41, S=32$ \\
      \hline
      Danish: \$7 & $D=49, S=17$ & $D=42, S=28$ \\
      \hline
    \end{tabular}
  \end{center}
  Given this choice, Danish: \$3, \$5, and \$5 are eliminated because they are
  dominated by Danish: \$6 and Danish: \$7.
  \begin{center}
    \begin{tabular}{|l||c|c|}
      \hline
      & Sandwich: \$5 & Sandwich: \$6 \\ \hline
      \hline
      Danish: \$6 & $D=50, S=29$ & $D=41, S=32$ \\
      \hline
      Danish: \$7 & $D=49, S=17$ & $D=42, S=28$ \\
      \hline
    \end{tabular}
  \end{center}
  Sandwich \$6 is a strongly dominant strategy at this point.
  \begin{center}
    \begin{tabular}{|l||c|}
      \hline
      & Sandwich: \$6 \\ \hline
      \hline
      Danish: \$6 & $D=41, S=32$ \\
      \hline
      Danish: \$7 & $D=42, S=28$ \\
      \hline
    \end{tabular}
  \end{center}
  The best choice for the Danish seller is Danish: \$7, making (Sandwich: \$6,
  Danish: \$7) the Nash equilibrium value.
  \begin{center}
    \begin{tabular}{|l||c|}
      \hline
      & Sandwich: \$6 \\ \hline
      \hline
      Danish: \$7 & $D=42, S=28$ \\
      \hline
    \end{tabular}
  \end{center}
  \item Show any/all cells that strongly Pareto Dominate the Nash Equilibrium
  value.
  \par (Sandwich: \$5, Danish: \$4) and (Sandwich: \$5, Danish: ) strongly
  dominate the Nash Equilibrium value. (Sandwich: \$5, Danish: \$3) will result
  in a higher payoff for Danish and an equal payoff for Sandwich.
\end{enumerate}

\subsection*{Problem 4}
The game ``Rock-Paper-Scissors-Lizard-Spock'' is an extension of the popular
``Rock-Paper-Scissors''. In addition to the usual rules of Rock smashes
Scissors, Scissors cuts Paper, and Paper covers Rock, we also have:
\begin{itemize}
  \item Rock crushes Lizard
  \item Lizard poisons Spock
  \item Spock smashes Scissors
  \item Scissors decapitates Lizard
  \item Lizard eats Paper
  \item Paper disproves Spock
  \item Spock vaporizes Rock
\end{itemize}
\begin{enumerate}[(a)]
  \item Fill in the payoff matrix for this game using values of 0 for a tie,
  -1 for a loss, and +1 for a win.
  \begin{center}
    \begin{tabular}{|c|c|c|c|c|c|c|}
      \hline
      \multicolumn{7}{|c|}{Player A} \\ \hline
      \multirow{6}{*}{Player B}
      & & Rock & Paper & Scissors & Lizard & Spock \\ \cline{2-7}
      & Rock & 0,0 & -1,1 & 1,-1 & 1,-1 & -1,1 \\ \cline{2-7}
      & Paper & 1,-1 & 0,0 & -1,1 & -1,1 & 1,-1 \\ \cline{2-7}
      & Scissors & -1,1 & 1,-1 & 0,0 & 1,-1 & -1,1 \\ \cline{2-7}
      & Lizard & -1,1 & 1,-1 & -1,1 & 0,0 & 1,-1 \\ \cline{2-7}
      & Spock & 1,-1 & -1,1 & 1,-1 & -1,1 & 0,0 \\ \hline
    \end{tabular}
  \end{center}
  \item If Player A is utilizing the following mixed strategy: \\
  Lizard = 22\% \\
  Spock = 25\% \\
  Rock = 20\% \\
  Paper = 15\% \\
  Scissors = 18\% \\
  What are the expected Payoff values for each option for Player B?
  \begin{center}
    \begin{tabular}{|c|c|c|c|c|c|c|}
      \hline
      \multicolumn{7}{|c|}{Player A} \\ \hline
      \multirow{6}{*}{Player B}
      & & Rock (0.20) & Paper (0.15) & Scissors (0.18) & Lizard (0.22) &
        Spock (0.25) \\ \cline{2-7}
      & Rock & 0,0 & -1,1 & 1,-1 & 1,-1 & -1,1 \\ \cline{2-7}
      & Paper & 1,-1 & 0,0 & -1,1 & -1,1 & 1,-1 \\ \cline{2-7}
      & Scissors & -1,1 & 1,-1 & 0,0 & 1,-1 & -1,1 \\ \cline{2-7}
      & Lizard & -1,1 & 1,-1 & -1,1 & 0,0 & 1,-1 \\ \cline{2-7}
      & Spock & 1,-1 & -1,1 & 1,-1 & -1,1 & 0,0 \\ \hline
    \end{tabular} \\[0.5cm]
    \begin{tabular}{|c|c|c|}
      \hline
      \multirow{5}{*}{Player B}
      & Rock & 0-0.15+0.18+0.22-0.25 = 0 \\ \cline{2-3}
      & Paper & 0.2+0-0.18-0.22+0.25 = 0.05 \\ \cline{2-3}
      & Scissors & -0.2+0.15+0+0.22-0.25 = -0.08 \\ \cline{2-3}
      & Lizard & -0.2+0.15-0.18+0+0.25 = 0.02 \\ \cline{2-3}
      & Spock & 0.2-0.15+0.18-0.22+0 = 0.01 \\ \hline
    \end{tabular}
  \end{center}
  \item Based on the expected Payoff values for Player B, what is the best
  mixed strategy for Player B?
  \par Given these values, Player B should expect to lose more when playing
  scissors and win slightly more when playing paper. The best mixed strategy for
  Player B would be to favor playing Paper, Lizard, Spock while playing Scissors
  less.
\end{enumerate}

\begin{center}
  If you have any questions, comments, or concerns, please contact me at
  alvin@omgimanerd.tech
\end{center}

\end{document}
