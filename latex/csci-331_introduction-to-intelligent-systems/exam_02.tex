\documentclass{math}

\usepackage{enumerate}
\usepackage{tikz}

\geometry{letterpaper, margin=0.5in}

\title{Introduction to Intelligent Systems: Exam 2}
\author{Alvin Lin}
\date{August 2017 - December 2017}

\begin{document}

\maketitle

\subsection*{Problem 1}
Consider differences in policy between politicians and the Federal Reserve
Board. Politicians can expand or contract fiscal policy, while the feds can
expand or contract monetary policy (and of course, each side can choose to do
nothing). Each side also has preferences for who should do what - neither side
wants to look like the ``bad guy''. The payoff matrix for the two sides is shown
below (P = politicians, F = federal reserve).
\begin{center}
  \begin{tabular}[h!]{|l||c|c|c|}
    \hline
    & Fed: contract  & Fed: do nothing & Fed: expand \\ \hline \hline
    Pol: contract & \( P=1, F=7 \) & \( P=4, F=9 \) & \( P=6, F=6 \) \\ \hline
    Pol: do nothing & \( P=2, F=8 \) & \( P=5, F=5 \) & \( P=9, F=4 \) \\ \hline
    Pol: expand & \( P=3, F=3 \) & \( P=7, F=2 \) & \( P=8, F=1 \) \\ \hline
  \end{tabular}
\end{center}
\begin{enumerate}
  \item Find the pure strategy Nash equilibrium. Assuming that both sides play
  their Nash equilibrium, what will each side decide to do?
  \par Fed: expand is eliminated first because it is a dominated strategy for
  the feds.
  \begin{center}
    \begin{tabular}[h!]{|l||c|c|}
      \hline
      & Fed: contract  & Fed: do nothing \\ \hline \hline
      Pol: contract & \( P=1, F=7 \) & \( P=4, F=9 \) \\ \hline
      Pol: do nothing & \( P=2, F=8 \) & \( P=5, F=5 \) \\ \hline
      Pol: expand & \( P=3, F=3 \) & \( P=7, F=2 \) \\ \hline
    \end{tabular}
  \end{center}
  Pol: contract and Pol: do nothing are eliminated next because they are
  dominated by Pol: expand.
  \begin{center}
    \begin{tabular}[h!]{|l||c|c|}
      \hline
      & Fed: contract  & Fed: do nothing \\ \hline \hline
      Pol: expand & \( P=3, F=3 \) & \( P=7, F=2 \) \\ \hline
    \end{tabular}
  \end{center}
  Fed: do nothing is eliminated since it is dominated by Fed: contract. Thus
  the pure strategy Nash equilibrium is Fed: contract, Pol: expand.
  \begin{center}
    \begin{tabular}[h!]{|l||c|c|}
      \hline
      & Fed: contract \\ \hline \hline
      Pol: expand & \( P=3, F=3 \) \\ \hline
    \end{tabular}
  \end{center}
  \item Is this a Pareto optimal solution? If yes, explain why. If no, explain
  why not.
  \par This is not a Pareto optimal solution because there exist situations
  that give both of them a higher payoffs. (Fed: do nothing, Pol: contract),
  (Fed: expand, Pol: contract), (Fed: do nothing, Pol: do nothing), and
  (Fed: expand, Pol: do nothing) all have higher payoffs for both players.
\end{enumerate}

\subsection*{Problem 2}
Construct a formal \textbf{direct} proof of validity for the following. For each
line in your proof you must include the ruel you are using AND the line(s) it
refers to:
\par Steve took the bus or the train. If he took the bus or drove his own car,
then he arrived late and missed the first session. He did not arrive late.
Therefore, he took the train.
\begin{align}
  B\vee T &\quad \text{Assumption: (Steve took the bus or the train)} \\
  (B\vee C)\to (L\wedge M) &\quad \text{Assumption: (If he took the bus or drove
  his own car,} \\
  &\quad \text{then he arrived late and missed the first session)} \\
  \neg L &\quad \text{Assumption: (He did not arrive late)} \\
  \neg L\vee\neg M &\quad \text{Or-Introduction on line 4} \\
  \neg(L\wedge M) &\quad \text{De Morgan's Law on line 5} \\
  \neg(L\wedge M)\to\neg(B\vee C) &\quad \text{Contraposition on line 2} \\
  \neg(B\vee C) &\quad \text{Modus Ponens on line 7 using line 6} \\
  \neg B\wedge\neg C &\quad \text{De Morgan's Law on line 8} \\
  \neg B &\quad \text{And-Elimination on line 9} \\
  T &\quad \text{Disjunctive Syllogism using lines 10 and 1} \\
  &\quad \text{Conclusion: T (Therefore, he took the train)}
\end{align}

\subsection*{Problem 3}
Give the \textit{most general unifier} for each pair of atomic sentences, if it
exists. If it does not exist, explain why. You may assume that an uppercase
letter represents a variable and a lowercase letter represents a constant.
\begin{enumerate}
  \item \( p(a,b,b) \) and \( p(X,Y,Z) \)
  \[ \{X/a, Y/b, Z/b\} \]
  \item \( q(Y,g(a,b)) \) and \( q(Z,g(X,X)) \) \\
  Does not exist, this would require the variable \( Z \) to be bound to another
  variable and the variable \( X \) to be bound to two different constants.
  \item Convert the following to PNF:
  \[ (\forall X)(a(X)\vee\neg b(X))\to(\exists Y)(a(Y)) \]
  \begin{align}
    \neg(\exists Y)(a(Y))\vee(\forall X)(a(X)\vee\neg b(X))
      &\quad \text{Implication Eliminination} \\
    \neg(\exists Y)a(Y)\vee(\forall X)a(X)\vee\neg b(X)
      &\quad \text{Logical Equivalence} \\
    (\forall Y)\neg a(Y)\vee(\forall X)a(X)\vee\neg b(X)
      &\quad \text{Negation of quantifiers} \\
    (\forall Y)(\forall X)(a(Y)\vee a(X)\vee\neg b(X))
      &\quad \text{Logical Equivalence}
  \end{align}
\end{enumerate}

\begin{center}
  If you have any questions, comments, or concerns, please contact me at
  alvin@omgimanerd.tech
\end{center}

\end{document}
