\documentclass{math}

\usepackage{enumerate}

\geometry{letterpaper, margin=0.5in}

\title{Introduction to Intelligent Systems: Homework 5}
\author{Alvin Lin - Section 1}
\date{August 2017 - December 2017}

\begin{document}

\maketitle

\subsection*{Problem 1}
Given the Wumpus world example from class, suppose the agent has progressed to
the point shown in Figure 7.4(a) on page 239, having perceived nothing at
[1,1], a breeze in [2,1], and a stench in [1,2], and is now concerned with the
contents of [1,3], [2,2], [3,1]. Each of these can contain a pit, and at most
one can contain a wumpus. Following the example of Figure 7.5, construct the
set of possible worlds. (Hint: there are 32 of them). Mark the worlds in which
KB is true and those in which each of the following sentences is true: \\
\( \alpha2 \) = ``There is not a pit in [2,2]'' \\
\( \alpha3 \) = ``There is a wumpus in [1,3]'' \\
Hence show that \( KB\models\alpha2 \) and \( KB\models\alpha3 \).

\subsection*{Problem 2}
Use a truth table to show that
\[ \{p\to q,(m\to p\vee q),m\}\models q \]
\begin{center}
  \begin{tabular}{|c|c|c|c|c|}
    \hline
    \( p \) & \( q \) & \( m \) & \( p\to q \) & \( m\to p\vee q \) \\
    \hline
    T & \textbf{T} & T & T & T \\
    T & T & F & T & T \\
    T & F & T & F & T \\
    T & F & F & F & T \\
    F & \textbf{T} & T & T & T \\
    F & T & F & T & T \\
    F & F & T & T & F \\
    F & F & F & T & T \\
    \hline
  \end{tabular}
\end{center}
\( q \) is true in all worlds where \( p\to q \), \( m\to p\vee q \), and
\( m \) are true.

\subsection*{Problem 3}
Use a direct proof (not a proof by contradiction) to show the following.
\begin{gather*}
  p\to q \\
  q\to r \\
  \vdash p\to r
\end{gather*}
For each step of the proof, indicate the premise and the logic rule used.
Use only the rules from the notes.
\begin{align*}
  p\to q \quad q\to r &\quad \text{Assumptions} \\
  p\to q\wedge q\to r &\quad \text{And-Introduction} \\
  p\to r &\quad \text{Hypothetical Syllogism}
\end{align*}

\subsection*{Problem 4}
Which of the following are correct? If they are incorrect, show the truth
assignments that show it. (Hint: Look at page 249 in R\&N).
\begin{enumerate}
  \item \( False\models True \)
  \item \( True\models False \)
  \item \( (A\wedge B)\models (A\Leftrightarrow B) \)
  \item \( (A\Leftrightarrow B)\models A\vee B \)
  \item \( (A\wedge B)\to C\models (A\to C)\vee(B\to C) \)
\end{enumerate}

\subsection*{Problem 5}
Given the following, prove the deduction by (a) a direct proof and (b) a
Reductio Ad Absurdum (proof by contradiction). For each step of the proof,
indicate the premise and the logic rule used.
\begin{gather*}
  H\to I\wedge J\to K \\
  (I\vee K)\to L \\
  \neg L \\
  \vdash\neg(H\vee J)
\end{gather*}
\begin{align*}
  -
\end{align*}

\subsection*{Problem 6}
Convert the following to CNF notation: \\
\textit{Hint:} implication has a higher precedence than AND or OR.
\begin{enumerate}
  \item \( C\wedge F\to\neg B \)
  \begin{gather*}
    C\wedge F\to\neg B \\
    \neg(C\wedge F)\vee\neg B \\
    \neg C\vee\neg F\vee\neg B
  \end{gather*}
  \item \( \neg B\to(C\wedge D\wedge E) \)
  \begin{gather*}
    \neg B\to(C\wedge D\wedge E) \\
    \neg(\neg B)\vee(C\wedge D\wedge E) \\
    B\vee(C\wedge D\wedge E) \\
    (B\vee C)\wedge(B\vee D)\wedge(B\vee E)
  \end{gather*}
  \item \( (A\vee B)\Leftrightarrow(C\wedge D) \)
  \begin{gather*}
    (A\vee B)\Leftrightarrow(C\wedge D) \\
    ((A\vee B)\to(C\wedge D))\wedge((C\wedge D)\to(A\vee B)) \\
    (\neg(A\vee B)\vee(C\wedge D))\wedge(\neg(C\wedge D)\vee(A\vee B)) \\
    ((\neg A\wedge\neg B)\vee(C\wedge D))\wedge
      ((\neg C\vee\neg D)\vee(A\vee B)) \\
    (\neg A\vee C)\wedge(\neg B\vee C)\wedge(\neg A\vee D)\wedge(\neg B\vee D)
      \wedge(A\vee B\vee\neg C\vee\neg D)
  \end{gather*}
\end{enumerate}

\begin{center}
  If you have any questions, comments, or concerns, please contact me at
  alvin@omgimanerd.tech
\end{center}

\end{document}
