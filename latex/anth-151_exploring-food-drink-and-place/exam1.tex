\documentclass[letterpaper, 12pt]{article}

\title{Exploring Food, Drink, and Place: Exam 1}
\author{Alvin Lin}
\date{August 2016 - December 2016}

\begin{document}

\maketitle

\subsubsection*{It is popularly believed that ``you are what you eat". Based
upon the chapters that we have read for the class, would you agree or disagree
with this assertion? Please be sure to explain your point of view.}

The assertion ``you are what you eat" can be argued from a cultural and
physiological point of view. In a literal sense, your health and what you are
composed of depends on the food you consume. However, from a cultural point of
view, a person can be defined by the food they consume. Their background
and identity can be derived from their foods, methods of preparation, and
acquisition thereof. Thus, ``you are what you eat" applies culturally as
well. \par
One primary example is the Japanese obent\={o}, as discussed by Anne Allison in
\textit{Japanese Mothers and Obent\={o}s: The Lunch-Box as Ideological State
Apparatus}. In this article, Allison discusses how the obent\={o} is a key
method of cementing a woman's identity as a mother and a child's identity as
a student of a school. It provides a social structure by identifying a
``producer as a woman and mother, and the consumer as a child of a mother''. In
addition to nutritious value, the appearance of the food prepared plays a major
role as well. The Japanese have strict standards of exactness and perfection
for the obent\={o} boxes, such as small, individual portions, complementing
colors and shapes, and even the containers that the food is served in. All of
the aforementioned play a role in defining the mother and reflecting on her
sense of responsibility and commitment as a parent as well as defining the
child as a member of society. The value of social conformity is reflected by
whether or not the child consumes the entirety of the obent\={o} box. In fact,
this is so important that nursery school teachers will enforce the consumption
of the entire obent\={o}. Allison here notes that it represents the child's
ability to ``follow directions, obey rules, and accept the authority of the
school system''. Thus, ``you are what you eat'' is reflected in the Japanese
obent\={o} since they play a major role in defining the identity of a mother
and child. \par
Another noteworthy example is the case of punks in Seattle, as discussed by
Dylan Clark in \textit{The Raw and the Rotten: Punk Cuisine}. Clark discusses
the preparation and acquisition of food here by punks as ritualistic and
indicative of their ideologies. Many punks believe that the industrial
production of food represents ``White, male, corporate supremacy'', and ``fills
a person's body with the norms, rationales, and moral pollution of corporate
capitalism and imperialism''. A punk's identity can be defined by their food
choice and how they acquire the food. As noted in the article, punks will grow
their own food to make it less commodified, or purchase food in ``brandless
bulk or directly from farmers''. They will also steal and salvage food from
garbage dumpsters, a process which they believe purifies corporate or
mainstream food. This example alone reflects the adage ``you are what you eat''
the most, cementing a punk's identity based on his culinary choices and
practices. \par
In short, ``you are what you eat'' applies culturally to many different groups
of people since they can be defined based on their food choices. A person's
identity is reflected by how they acquire and prepare food, since it reflects
their ideologies on what the food represents when prepared and consumed.

\subsubsection*{What is “food activism”? Pick two of the chapters from our text
and show how these chapters illustrate how you have defined food activism.}
Food activism is an awareness of where our food comes from and how it affects
us when we purchase and consume it. Food activists work to reduce environment
damage caused by large scale industrial food production and its use of
chemical fertilizers and monocultures, and improve access to healthy food,
fruits, and vegetables to people of all incomes. Society is improved by
approaching food consumption and acquisition from a economic, agricultural, and
environmental point of view. \par
Carole Counihan and Valeria Siniscalchi define food activism in \textit{
Ethnography of Food Activism} as a people's effort to ``make the food system
or parts of it more democratic, sustainable, healthy, ethical, culturally
appropriate, and better in quality''. It is a conscious effort by people of
different backgrounds and occupations to improve society by taking control of
food production, distribution, and choice. Counihan describes food activism as
``a fruitful term for examining ... the diverse forms of dissent and resistance
practiced by political activists, farmers, ... producers, and consumers'' and
summarizes how the efforts of activists around the world work to close the gap
between farmers and consumers through local food markets. Food activism is
defined here as the efforts of many different groups of people working under
different motives in order to improve access to healthier local food. \par
In \textit{Food Activism in Western Oregon}, Joan E. Gross discusses her
experiences interacting with food activists in Oregon. She defines food
activism by first emphasizing its importance since Oregon had one of the
highest hunger rates in the nation, highlighting an obvious problem in the
system of food acquisition and movement. One important example that
demonstrates food activism is noted in the experiences of one activist who made
a choice to eat local meat instead of going vegetarian despite the humanitarian
appeal of eating lower on the food chain. In her words, ``better to eat a local
chicken from down the street than tofu that caused a rainforest to be cut down
in Brazil''. This shows an awareness of the origins of the food one consumes as
well as its impact on the environment and demonstrates how food activism
involves choices that take into account all the effects of our food purchases
and choices. \par
Both the above chapters clearly define food activism as an effort to move away
from the industrialized capitalist nature of modern food production in search
of a more ethical, healthy, and local way to produce food. They discuss the
importance of being aware of the societal, environmental, and economic impact
of local food. By being aware of where one's food is sourced, one can make
better choices about food purchases.

\end{document}
