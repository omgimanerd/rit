\documentclass[letterpaper, 12pt]{article}

\title{Exploring Food, Drink, and Place: Short Paper}
\author{Alvin Lin}
\date{August 2016 - December 2016}

\begin{document}

\maketitle

\subsubsection*{What if any connections can be made between Mead's chapter on
obesity and Bordo's chapter on anorexia and bulimia. Do you agree with their conclusions? If so, why?}

Modern society imposes a burden on many people due to its stress and high-speed
pace. Meals and the consumption of food in today's modernized, industrialized,
stainless-steel sterilized society place many constraints on people due to
social, nutritional, and economic concerns. In both Mead and Bordo's arguments,
the changing role of food in our society has contributed to many problems
with regard to body image, leading to obesity, anorexia, and bulimia. \par
One primary example of why this is the case is discussed by Mead, who explains
that food is used as a coping mechanism, and is tied to ``a system of punishment
and reward''. The obese are seen in our society as unhealthy and morally wrong,
tied to ``a lack of character'' and ``a blatant lack of self-discipline'', even
if the aforementioned obesity is a product of society. Children are taught to
eat what is given to them, and eat three meals a day or face punishment. They
are then subsequently rewarded with more food in the form of ``dessert, Sunday
dinner, [and] holiday fare''. Mead notes that having three meals a day is an
obsolete concept that is no longer relevant in today's society where food is
now more readily accessible. People who still live by it face obesity and shame
from society, and should eat according to their own individual needs. \par
Bordo's chapter discusses how body image in advertising and media has affected
the public's view of food across many different groups of people. She argues
that the pervasiveness of eating problems is no longer limited to ``white,
heterosexual, North American, and economically secure'' girls and women. The
concept of body image has become so ingrained in our society that children
talk about it casually. According to Bordo's statistics, 57\% of girls have
fasted, ``one third of all girls in grades nine to 12 think they are overweight,
and only 56 percent of seventh graders say they like the way they look''. This
toxic viewpoint, backed by cultural imagery in the form of thin models and
skinny bodies in media, is now a part of our culture that affects a large
portion of the population. \par
Both Bordo and Mead's assessments of today's society and its relationship to
food paint a grim image. Both are accurate in scary ways and show that society
needs to change its values. Food's relationship with today's society has
brought about obesity, anorexia, and bulimia due to mainstream advertising and
archaic values. Bordo and Mead agree that by encouraging people to pursue their
own eating patterns in ways that suit their own needs, body type, and
self-esteem, these problems could steadily be solved.

\end{document}
