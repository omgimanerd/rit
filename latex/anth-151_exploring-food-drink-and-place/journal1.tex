\documentclass{article}

\title{Honors First Year Seminar: Exploring Food, Drink, and Place}
\author{Alvin Lin}
\date{August 2016 - December 2016}

\begin{document}

\maketitle

\section*{Why Do We Overeat?}
\subsection*{Margaret Mead}
In this piece, Margaret Mead writes about the social pressures and societal impact food abundance has had on our society. Mead discusses the dual pressure of food advertising and weight loss advertisement. We treat food as an art and have a desire to consume good food in large quantities, but the present style of eating has culminated into soaring rates of obesity. \par
Mead discusses the American intolerance towards people who are obese, despite the varied origins of that obesity. We put all obese people in the same category even if the obesity is not controllable by the person in question. This societal exclusion of obese people is evident in our advertising, which exclusively uses underweight and ``thin'' models and does not create fashion or furniture for obese people. \par
The origins of this sentiment likely stem from our belief that it is wrong to be obese, and that it shows ``lack of character, a disregard for health, and a blatant lack of self-discipline''. Mead discusses the impact that this now has on our youth, whom many of which are now severely malnourished in pursuit of a body image imposed by society. \par
Our attitude towards eating, which stems from times when food was scarcer, is now obsolete and has negative repercussions for society and the next generation of young adults. \par
Mead suggests that instead of punishing the obese, we should focus on the intention of sharing it with others rather than using it as a system of punishment and reward as we have often done when it was scarce. She also suggests of taking ``three square meals a day'' with a grain of salt since the abundance of food has now made that rule obsolete. We should eat according to our own individual needs tso that we can better control our own bodies. Once we stop condemning people for their consumption of food, society will improve.

\end{document}
