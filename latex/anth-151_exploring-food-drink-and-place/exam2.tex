\documentclass{article}

\title{Exploring Food, Drink, and Place: Exam 2}
\author{Alvin Lin}
\date{August 2016 - December 2016}

\begin{document}

\maketitle

\subsubsection*{Many of the authors argue that obesity and hunger are related
to various inequalities in our society. What would the democratization of the
food system entail (see Levkoe chpt. 40)? What steps would need to be taken to
democratize our food system? Please explain.}
The lack of nutritious food is often the cause of obesity and hunger, but the
issue runs a lot deeper than that of food itself. Many socioeconomic conditions
and inequalities in our society force many people to make poor food choices.
The obese are trapped by society because the very existence of obesity
legitimizes a ``norm'' and creates a common enemy for society, while the poor
and hungry are trapped by a socioeconomic loop that prevents them from
accessing better food and making better choices. The democratization of food
would solve both of these problems by bringing people close to the source of
their consumption and granting them better access to and understanding of
food.\par
Alice Julier addresses obesity in \textit{The Political Economy of Obesity}
by discussing the role of the obese and their political, economic, and social
function in society. Julier argues that the obese create an entire industry
focused on them, since obesity has been medicalized and turned into an
``epidemic''. ``Labeling obesity an epidemic creates jobs for a number of
occupations and professeions that serve or `service' the diet, exercise, and
health industries''. The obese as a group ``can be identified and punished ...
in order to uphold the legitimacy of conventional norms''. Julier summarizes
this is an profound comparison that ``by making public health ... entirely about
individual behavior, we limit people's autonomy regarding the vast number of
reasons they choose to eat, ... we lose a great deal by valuing food for little
but its nutrients ... and by valuing our citizens for little but their
appearance''. Society imposes a great burden on the obese, and as long
as this burden is imposed on them, society will always enforce this divide and
isolate them as a group. \par
As a solution, Charles Z. Levkoe argues that ``by reframing hunger as an issue
of poverty, it ... will enable marginalized people to make choices around the
foods they purchase''. Levkoe argues that democratizing food could entail
participating in a community garden, which would serve to educate people
on democratic principles. He argues that this democratizes food by allowing
participants to ``take responsiblity for a number of tasks and follow them
through while recognizing their rights within a larger system''. In addition to
``breaking systems of charity ... by producing their own food, ... the garden
serves as a model for the community of what can be collectively accomplished''.
On top of all this, the democratization process brings consumers much closer
to the producers and lowers dependence on mass production based food
corporations, which ``establishes a greater sense of control and power''
over the lives of the participants. \par
Levkoe's proposed solution is not only sustainable, but good for the community
and the environment. This would serve to address both the problems of obesity
and poverty induced by society by empowering people to make smarter food choices
and improving ways in which communities can help themselves.

\subsubsection*{This course has focused on a critique of the industrial food
system. Would you agree that organic foods and the slow food movement provide
an acceptable alternative? Please explain your point of view by making concrete
references to our text.}
The industrial food system has been criticized for its inhumanity, disregard
for nature, and disregard for human safety. The meatpacking industry itself is
notorious for its lack of safety measures and protocols, leading to accidents
injuries. It is safe to say that moving towards organic food would benefit not
only society, but also the environment. \par
In the words of Charles Z. Levkoe, industrial foods systems are driven by
nothing but profit motive and puts a ``focus on people, not as citizens, but as
consumers''. He states that ``the perspective of consumer implies an identity
defined by a direct relationship with the market, one in which profit becomes
the most important factor''. Eric Schlosser describes the industrial
meatpacking industry in \textit{The Chain Never Stops} as deceitful, noting
that they ``[have] a well-documented history of discouraging injury reports,
falsifying injury data, and putting injured workers back on the job to minimize
lost workdays''. He describes the experiences and injuries of former worker
Kenny Dobbins, who is now forced to rely on public assistance due to his
injuries. \par
It is clearly obvious that organic food and the slow food movement would be
more than acceptable as an alternative. Levkoe notes that ``by reclaiming
public space and growing organic vegetables, [people] are breaking dependencies
on ... the market economy by producing their own food''. Aside from the
aforementioned plethora of benefits to consumers as described in the essay
above, the slow food movement pays homage to the diversity of cultures
and cuisines by actively preserving local traditions. As described by Alison
Leitch in \textit{Slow Food and the Politics of ``Virtuous Globalization''},
the slow food movement emphasized ``the importance of food as a cultural
artifact linked the preservation of a distinctive European cultural heritage''.
This so-called ``endangered foods campaign'' served to protect regional tastes
due to effects of the industrial food system such as farming monocultures. \par
In addition to addressing issues such as industrial exploitation, the slow food
movement empowers consumers and improves access to better and healthier food.
The push for more organic food would address the national issue of obesity in
the United States. It is clear that organic food and the slow food movement
would be a positive alternative to the predatory industrial food system.

\end{document}
