\documentclass[letterpaper, 12pt]{article}

\title{Exploring Food, Drink, and Place: Exam 1}
\author{Alvin Lin}
\date{August 2016 - December 2016}

\begin{document}

\maketitle

\subsubsection*{Many of the authors argue that obesity and hunger are related
to various inequalities in our society. What would the democratization of the
food system entail (see Levkoe chpt. 40)? What steps would need to be taken to
democratize our food system? Please explain.}
The lack of nutritious food is often the cause of obesity and hunger, but the
issue runs a lot deeper than that of food itself. Many socioeconomic conditions
and inequalities in our society force many people to make poor food choices.
The obese are trapped by society because the very existence of obesity
legitimizes a ``norm'' and creates a common enemy for society, while the poor
and hungry are trapped by a socioeconomic loop that prevents them from
accessing better food and making better choices. \par
Alice Julier addresses the former in \textit{The Political Economy of Obesity}
by discussing the role of the obese and their political, economic, and social
function in society. Julier argues that the obese create an entire industry
focused on them, since obesity has been medicalized and turned into an
``epidemic''. ``Labeling obesity an epidemic creates jobs for a number of
occupations and professeions that serve or `service' the diet, exercise, and
health industries''. The obese as a group ``can be identified and punished ...
in order to uphold the legitimacy of conventional norms''. Julier summarizes
this is an profound comparison that ``by making public health ... entirely about
individual behavior, we limit people's autonomy regarding the vast number of
reasons they choose to eat, ... we lose a great deal by valuing food for little
but its nutrients ... and by valuing our citizens for little but their
appearance''. Society imposes a great burden on the obese, and as long
as this burden is imposed on them, society will always enforce this divide and
isolate them as a group. \par
Charles Z. Levkoe argues that ``by reframing hunger as an issue of poverty, it
... will enable marginalized people to make choices around the foods they
purchase''. Though this leans toward a more socialist point of view, Levkoe
argues that participating in a community garden serves to educate people on
democratic principles. He argues that this democratizes food by allowing
participants to ``take responsiblity for a number of tasks and follow them
through while recognizing their rights within a larger system''. In addition to
``breaking systems of charity ... by producing their own food, ... the garden
serves as a model for the community of what can be collectively accomplished''.
On top of all this, the democratization process brings consumers much closer
to the producers, which ``establishes a greater sense of control and power''
over the lives of the participants. It is not only sustainable, but good for
the community and the environment. This would serve to address both the
problems of obesity and poverty induced by society by allowing people to make
smarter food choices and improving ways in which communities can help
themselves.

\subsubsection*{This course has focused on a critique of the industrial food
system. Would you agree that organic foods and the slow food movement provide
an acceptable alternative? Please explain your point of view by making concrete
references to our text.}

\end{document}
