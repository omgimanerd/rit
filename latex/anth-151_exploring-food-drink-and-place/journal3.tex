\documentclass{article}

\title{Honors First Year Seminar: Exploring Food, Drink, and Place}
\author{Alvin Lin}
\date{August 2016 - December 2016}

\begin{document}

\maketitle

\section*{Distinction: A Social Critique of the Judgement of Taste}
\subsection*{Pierre Bourdieu}
In this piece, Bourdieu categorizes social classes by their consumption of food, culture, and presentation. He notes patterns in the purchasing of food between teachers, professionals, and industrial and commercial employers, noting that the more one earns, the more one will tend to spend on food. Industrial and commercial employers spent a much larger portion of their wages on food in comparison to teachers. He also observes the consumption of heavy and fat foods in comparison to slim and delicate foods among both the rich and the poor. \par
Bordieu notes food differs between the sexes also because of societal role. He notes that dishes like pot-au-feu symbolize women because of the time-demanding nature of the dish. Working class individuals would prioritize food products that are cheap and nutritious in comparison to ``professions'' which prefer products that are tasty and light. \par
``For example, in the working classes, fish tends to be regarded as an unsuitable food for men, not only because it is a light food, insufficiently `filling', which would only be cooked for health reaons, i.e., for invalids and children, but also because, like fruit (except bananas) it is one of the `fiddly' things which a man's hands cannot cope with and which make him childlike...'' \par
Bordieu makes the comparison that since men are ``brutish'' and ``enormous'', they eat more, as well as eating and drinking ``stronger'' things. He notes that women are incompatible with the tastes of men's food, and reinforces this argument by stressing the differences in bearing, guesture, posture, and behavior of men and women. Tastes in food must be dependent on one's relationship to the world and the requirements of one's own body. \par
The working class meal is described as being `abundant', `elastic', and adapted to its function because its purpose is to be labor saving and filling. There is no need for stringency, measurement, or changing plates. In comparison, the bourgeoisie are concerned with form, rhythm, and restraint. This lifestyle is extended to other parts of the daily routine as well, where order and restraint are emphasized. Food as a primary need and pleasure is a relation to the social world for the bourgeois focused on form and aesthetic rather than function. According to Bordieu, eating is a social ceremony with no focus on the pleasure or necessity of food or the ``basely material vulgarity of those who indulge in the immediate satisfactions of food and drink''.

\end{document}
