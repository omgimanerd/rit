\documentclass{article}

\title{Honors First Year Seminar: Exploring Food, Drink, and Place}
\author{Alvin Lin}
\date{August 2016 - December 2016}

\begin{document}

\maketitle

\section*{Toward a Psychosociology of Contemporary Food Consumption}
\subsection*{Roland Barthes}
In this piece, Roland Barthes discusses the significance of the food we consume in our society. Americans consume almost twice the amount of sugar as the French, owing to the fact that we put large quantities of sugar in our sweets, pastries, ice creams, etc, and even foods that are not supposed to be sweet. The religious usage of sugar in our society has turned it into a set of experiences with an associated set of images, tastes, and choices. \par
Further extended this idea, Barthes discusses our food choices and how we make arbitrary distinctions between different consumer goods based on our exposure to advertising. Even products that are practically identical are seen as different based on the consumer's notion of the manufacturer's brand name, a phenomenon often noticed in the purchasing of cooking oils. In addition to being a nutrition/nourishment product, food has become ``a system of communication, a body of images, a protocol of usages, situations, and behavior'' that has direct impacts on our economy, advertising, and societal attitudes. \par
Like clothing, food has evolved from being a basic need to one with extremely complex levels of structure. The ingredients we use, the various techniques of preparation, and the aesthetic focus on food is evident of this change. Food as a system goes through changes that correspond to changes in its meaning and significance. \par
Barthes explores how analogous food is to our language, with the various subtleties in taste, texture, and \textit{flavor} corresponding to the subtleties of intonation and connotation in our language. Food has its own grammar and syntax that allows us to compare them in quantitative and semantic ways. \par
According to Barthes, three themes are evident in our food. One of which is historical significance in the preparation thereof and the techniques used. Another is the apparent attachment of unrelated ideas such as masculinity and femininity to certain foods due to advertising. This has led to an association of inferiority to certain foods and changes our perception of our day-to-day food items. The third, is the ambiguous nature of the ``health'' benefits that the food can provide. Though food serves its original purpose of provideding nourishment, advertising associates it with other ideas. For example, coffee with alertness, sugar with energy, etc. \par
Food is now associated today with more and more areas and incorporated into more situations. It represents an improvement in the quality of life and adds significance to circumstances in our lives. Barthes discusses how coffee, a stimulant, is now ironically associated to images of breaks, rest, and relaxation due to modern advertising. In additional to serving its physiological function, the meaning and circumstance of food has now extended to encompass specific activities and lifestyles.

\end{document}
