\documentclass{math}

\title{Introduction to Computer Vision}
\author{Alvin Lin}
\date{August 2018 - December 2018}

\begin{document}

\maketitle

\section*{Segmentation}
Image processing involves an image as input, with the output being another
image. Image analysis outputs measurements from an input image, and image
understanding outputs a high level description from an input image. Humans
tend to identify regions of images in groups, so the goal of segmentation is to
identify groups of pixels or regions that are related and go together
perceptually. \par
Segmentation is done to obtain primitives for other tasks, perceptual
organization/recognition, or image manipulation for graphics related tasks.
Segmentation can be used to group together similar pixels for efficiency of
further processing (superpixels). It can also separate images into coherent
objects. Oversegmentation and undersegmentation is always an issue here.

\subsection*{High Level Approaches}
Bottom-up segmentation involves grouping tokens with similar features, while
top-down segmentation involves grouping tokens that likely belong to the same
object. \par
Segmentation can be performed through clustering, graph partitioning, and
labeling. With clustering, similar points are grouped together and represented
with a single token. The K-means clustering algorithm can be used to cluster
together similar regions. This will usually find cluster centers that
minimize conditional variance and is simple to implement, however it is prone
to local minima and it is often difficult to choose \( k \).

\subsection*{K-medoids}
Instead of representing a cluster with the mean of its members, we select a
member of the cluster to represent it and minimize cluster dissimilarity. This
is most applicable when the mean is not meaningful. This is much less sensitive
to outliers than k-means.

\subsection*{Kernel Density Estimation}
Also known as the Parzen-Rosenblatt window method, this is a non-parametric way
to estimate the probability density function of a random variable. Inferences
about a population are made based only on a finite data sample.
\[ \hat{f_h}(x) = \frac{1}{nh}\sum_{i=1}^{n}K\left(\frac{x-x_i}{h}\right) \]

\begin{center}
  You can find all my notes at \url{http://omgimanerd.tech/notes}. If you have
  any questions, comments, or concerns, please contact me at
  alvin@omgimanerd.tech
\end{center}

\end{document}
