\documentclass{math}

\title{Introduction to Computer Vision}
\author{Alvin Lin}
\date{August 2018 - December 2018}

\begin{document}

\maketitle

\section*{Image Formation}
Physical parameters of image formation:
\begin{itemize}
  \item Optical: focal length, field of view, aperture
  \item Geometric: types of projection
  \item Photometric: type, direction, intensity of light reaching the sensor
\end{itemize}

\subsection*{What is an image?}
Until now, an image was defined as a 2D pattern of intensity values. Now, we
interpret it as a 2D projection of 3D points. A camera is a device that allows
for the projection of points from 3D to 2D. In the process of taking a picture,
we preserve things like straight lines and points of incidence, but lose
information such as angles and length. \par
The brightness of an image pixel is determined by the light source, the surface
shape, orientation, and reflectance properties, the optics, exposure, and nature
of the sensor lens.

\subsection*{Types of 3D Projections}
A 3D projection is any method of mapping three-dimensional points to a
two-dimensional plane.
\begin{itemize}
  \item Perspective projections: objects in the distance appear smaller than
    those close by, and parallel lines converge at a image point in infinity, on
    the horizon.
  \item Weak perspective projections: perspective effects, but not over the
    scale of individual objects.
  \item Orthographic projections: objects in the distance appear the same as
    those close by, parallel lengths at all points are of the same scale
    regardless of the distance of the camera.
\end{itemize}
In order to perform transformations, we convert to homogeneous coordinates:
\[ (x,y) \Rightarrow \begin{bmatrix}x \\ y \\ 1\end{bmatrix} \]
\[ (x,y,z) \Rightarrow \begin{bmatrix}x \\ y \\ z \\ 1\end{bmatrix} \]
Projection is a matrix multiplication using these coordinates:
\[ \begin{bmatrix}
  f & 0 & 0 & 0 \\
  0 & f & 0 & 0 \\
  0 & 0 & 1 & 0 \\
\end{bmatrix}\begin{bmatrix}
  x \\ y \\ z \\ 1
\end{bmatrix} = \begin{bmatrix}
  fx \\ fy \\ z
\end{bmatrix} \Rightarrow (f\frac{x}{z}, f\frac{y}{z}) \]

\subsection*{Vanishing Points}
Each set of parallel lines meets at a different point. Sets of parallel lines on
the same plane lead to collinear vanishing points. This is a good way to spot
faked images since scale and perspective don't work and vanishing points behave
badly.

\subsection*{Photometry}
The brightness of a pixel is a function of the brightness of the surface in the
scene that projects to the pixel. It is dependent on how much light is incident
on the surface and the fraction of incident light that gets reflected. \par
When light hits a point on an object, some of the light gets absorbed, some is
transmitted through the object, and some gets reflected. For simplicity, we will
assume that surfaces don't fluoresce or emit light, and that all the light
leaving a point is due to light arriving at that point.

\subsubsection*{Reflection}
Modes of reflection:
\begin{enumerate}
  \item Specular reflection: pure mirror, the incoming ray, outgoing ray, and
    the normal are coplanar, and the angle of incidence is equal to the angle
    of reflection.
  \item Diffuse reflection: light leaves in equal amounts in each direction so
    the surface looks equally bright from each viewing direction. Diffuse
    reflection is described by a parameter called \textbf{albedo}, which
    describes the percentage of light arriving that leaves.
\end{enumerate}

\subsubsection*{Lambert's Law}
\[ B = \rho(N\cdot S) = \rho\|S\|\cos\theta \]
where \( B \) is the radiosity (total power leaving the surface per unit area),
\( \rho \) is the albedo (fraction of incident irradiance reflected by the
surface), \( N \) is the unit normal, and \( S \) is the source vector where
the magnitude is proportional to the intensity of the source.

\subsubsection*{Shadows}
Most shadows are dark because shadow points get light from other surfaces and
not just the light source. Area sources are large bright areas such as the sky
which yield smooth blurry shadows.
\begin{itemize}
  \item Points that can see the whole source are brighter.
  \item Points that can only see part of the source are darker (penumbra).
  \item Points that can see no part of the source are darkest (umbra).
\end{itemize}
Other surfaces can behave light area sources.

\begin{center}
  You can find all my notes at \url{http://omgimanerd.tech/notes}. If you have
  any questions, comments, or concerns, please contact me at
  alvin@omgimanerd.tech
\end{center}

\end{document}
