\documentclass{math}

\title{Introduction to Computer Vision}
\author{Alvin Lin}
\date{August 2018 - December 2018}

\begin{document}

\maketitle

\section*{Texture}
Texture is defined by patterns of structure from changes in surface albedo,
surface shape, or many small surface patches. It is hard to concretely define,
but texture tells us what a surface is, and may sometimes tell us what the
object is or what shape the surface is. Often they are very complex to
recognize. Synthesizing textures from examples is also a complex problem in
computer graphics. \par
The core idea is that textures consist of a set of elements repeated in some
way. Ideally, we should be able to identify the elements and summarize the
repetition.

\subsection*{Filter Based Texture Representations}
\begin{enumerate}
  \item Choose a set of filters, each representing a pattern element.
  \item Apply all the filters to an image at a variety of scales.
  \item Rectify the filtered images. We should avoid averaging contrast
    reversals.
  \item Compute summaries of rectified filtered images.
  \item Describe each pixel by a vector of summaries.
\end{enumerate}
Alternatively, we can compute the responses of blobs and edges at various
orientations and scales. Since textures are made of repeated local patterns,
we need to find the patterns and describe their statistics within each local
window.

\subsection*{Choosing the Filters}
When choosing the filters, we can measure responses to over-complete filter
banks and describe textures using clusters of responses. Alternatively, we can
build a vocabulary of pattern elements from pictures and describe textures using
this vocabulary. There are two steps to building a pooled texture representation
for texture in an image domain.
\begin{enumerate}
  \item Build a dictionary representing the range of possible pattern elements,
    using a large number of texture patches.
  \item Vectorize the patches in images using the clusters learned.
\end{enumerate}
To build the dictionary:
\begin{enumerate}
  \item Collect many training example textures.
  \item Construct the vectors \( \vec{x} \) for the relevant pixels. Either
    reshape a patch around a pixel or compute a vector of filter outputs
    at the pixel.
  \item Obtain \( k \) cluster centers \( c \) for these examples.
  \item For each relevant pixel \( i \) in the image, compute the vector
    representation \( \vec{x_i} \) for that pixel. Obtain \( j \), the index
    of the cluster center \( c_j \) closest to that pixel and insert \( j \)
    into a histogram for that domain.
\end{enumerate}


\begin{center}
  You can find all my notes at \url{http://omgimanerd.tech/notes}. If you have
  any questions, comments, or concerns, please contact me at
  alvin@omgimanerd.tech
\end{center}

\end{document}
