\documentclass{math}

\usepackage{float}
\usepackage{graphicx}
\usepackage{subcaption}

\geometry{letterpaper, margin=0.5in}

\title{Intro to Computer Vision: Quiz 3}
\author{Alvin Lin}
\date{August 2017 - December 2017}

\begin{document}

\maketitle

\subsection*{Question 1}
We are interested in finding key points in this image in a scale and rotation
invariant way. We can do this by taking the Laplacian of Gaussian on the image
in order to find regions of high gradient. These regions identify significant
edges and corners on the image to use as keypoints. We can obtain the
orientation of the keypoint by calculating the orientations of all the pixels
around the keypoint and selecting the most frequently occurring orientation
(using a histogram to bin all the orientations). To uniquely identify this
keypoint, we take a 16x16 window of pixels around the keypoint, broken up into
a 4x4 grid. For each 4x4 window in the 4x4 grid, we calculate gradient
magnitudes and orientations, which give us 128 features which uniquely identify
this keypoint. To make this rotation invariant, we subtract the keypoint
orientation obtained earlier. By doing this process across two images, we
obtain a series of rotation invariant fingerprints for keypoints, which we can
compare for similarity. If clusters of keypoints match across both images, we
can conclude those regions of the image are of the same scene.

\begin{center}
  If you have any questions, comments, or concerns, please contact me at
  alvin@omgimanerd.tech
\end{center}

\end{document}
