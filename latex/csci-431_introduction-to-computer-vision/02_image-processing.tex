\documentclass{math}

\title{Introduction to Computer Vision}
\author{Alvin Lin}
\date{August 2018 - December 2018}

\begin{document}

\maketitle

\section*{Image Processing}
Pixels are derived from the phrase ``picture element'' and are indexed as
(x,y) or (col,row). Color images are 3D tensors and videos are n-dimensional.
Arithmetic operations on image pixels light addition and subtraction translate
to generic brightness adjustments and image blending, while multiplying and
dividing corresponds to contrast adjustment though it is not very common in
image processing.

\subsection*{Image Transforms}
\begin{itemize}
  \item Logarithmic transforms increase the dynamic range of the dark region but
    decreases the range of the bright regions.
  \item Exponential transforms are the inverse of the log transform.
  \item Power-law (gamma) transforms can either enhance contrast in high value
    regions at the expense of low value ones when \( \gamma > 1 \) or do the
    reverse when \( \gamma < 1 \).
\end{itemize}

\subsection*{Histograms}
If we count all pixels with similar intensity values, we get the histogram
image. This gives us a good visual representation of the image distribution.

\subsection*{Adaptive Thresholding}
Adaptive thresholding was designed to overcome the limitations of global
thresholding by using a different threshold value at each pixel. The threshold
is determined by values in the neighborhood.

\begin{center}
  You can find all my notes at \url{http://omgimanerd.tech/notes}. If you have
  any questions, comments, or concerns, please contact me at
  alvin@omgimanerd.tech
\end{center}

\end{document}
