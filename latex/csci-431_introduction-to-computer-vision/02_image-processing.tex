\documentclass{math}

\title{Introduction to Computer Vision}
\author{Alvin Lin}
\date{August 2018 - December 2018}

\begin{document}

\maketitle

\section*{Image Processing}
Pixels are derived from the phrase ``picture element'' and are indexed as
(x,y) or (col,row). Color images are 3D tensors and videos are n-dimensional.
Arithmetic operations on image pixels light addition and subtraction translate
to generic brightness adjustments and image blending, while multiplying and
dividing corresponds to contrast adjustment though it is not very common in
image processing.

\subsection*{Image Transforms}
\begin{itemize}
  \item Logarithmic transforms increase the dynamic range of the dark region but
    decreases the range of the bright regions.
  \item Exponential transforms are the inverse of the log transform.
  \item Power-law (gamma) transforms can either enhance contrast in high value
    regions at the expense of low value ones when \( \gamma > 1 \) or do the
    reverse when \( \gamma < 1 \).
\end{itemize}

\subsection*{Histograms}
If we count all pixels with similar intensity values, we get the histogram
image. This gives us a good visual representation of the image distribution.

\subsection*{Adaptive Thresholding}
Adaptive thresholding was designed to overcome the limitations of global
thresholding by using a different threshold value at each pixel. The threshold
is determined by values in the neighborhood.

\subsection*{Color}
Color is a psychological property of a visual experience when we look at
objects and lights. It is the result of an interaction between the physical
light in an environment with our physical system. Color is the property
possesed by an object to produce different sensations on the eye as a result
of the way the object reflects and emits light.

\subsubsection*{The Human Eye}
The human eye is a camera with many parts:
\begin{itemize}
  \item Lens: changes shape using ciliary muscles to focus on objects at
    different distances.
  \item Pupil: the hole (aperture) in your eye whose size is controlled by the
    iris.
  \item Iris: the colored annulus with radial muscles
  \item Retina: photoreceptor cells (cones and rods) in the back of your eye
\end{itemize}
The cones in your retina are responsible for color vision and are less
sensitive. They operate in high light conditions, unlike the rods, which operate
in low light conditions and are highly sensitive in order to provide gray-scale
vision. Rods and cones are non-uniformly distributed on the retina.

\subsubsection*{Computing Color Matches}
Computing color matches for any given color are useful because we want the
colors in the world, on a monitor, and in print format to all look the same.
We want to match the skin color of a person in a photograph printed on an ink
jet printer to their true skin color. \par
Since it is hard to reproduce color exactly, it is important to know whether a
color difference would be noticeable to a human viewer.

\subsubsection*{Linear Color Spaces}
LAB color space expresses color as three numerical values:
\begin{itemize}
  \item L stands for lightness
  \item A represents green-red color components
  \item B represents blue-yellow color components
\end{itemize}

\subsubsection*{Nonlinear Color Spaces}
HSV has perceptually meaningful dimensions:
\begin{itemize}
  \item H stands for hue
  \item S stands for saturation
  \item V stands for value (intensity)
\end{itemize}

\subsubsection*{Color Based Image Retrieval}
Given a collection of images, we can extract and store a color histogram per
image. We can then compare these to a histogram for a new query image and
compute the intersection to determine the most similar image. Color can also
be used for simpler and faster model tracking.

\begin{center}
  You can find all my notes at \url{http://omgimanerd.tech/notes}. If you have
  any questions, comments, or concerns, please contact me at
  alvin@omgimanerd.tech
\end{center}

\end{document}
