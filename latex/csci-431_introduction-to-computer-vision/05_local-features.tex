\documentclass{math}

\title{Introduction to Computer Vision}
\author{Alvin Lin}
\date{August 2018 - December 2018}

\begin{document}

\maketitle

\section*{Local Features}
We want to find patches that are ``worth representing'' to match from image to
image. These will represent either textures or objects. Ideally, they should be
covariant to translation, rotation, and scale. If the image is translated,
rotated, or scaled, the neighborhoods still preserve their unique properties.
They should also be localizable in translation, rotate, and scale, meaning we
can estimate the position, orientation and size of the patch.

\subsection*{Good Features}
\begin{itemize}
  \item Repeatability: the same feature can be found in several images despite
    geometric and photometric transformations.
  \item Saliency: each feature has a distinctive description.
  \item Compactness and efficiency: many fewer features than image pixels.
  \item Locality: a feature occupies a relatively small area of the image and is
    robust to clutter and occlusion.
\end{itemize}
Feature points are often used for image alignment, 3D reconstruction, motion
tracking, robot navigation, indexing and database retrieval, and object
recognition.

\subsection*{Finding Corners}
A key property of corners is that in the region around a corner, the image
gradient has two or more dominant directions. A strategy to find corners is to
find centers and estimate the scale from the center. Once we have that, we can
estimate the orientation. In a small enough window, a corner has a large
gradient change in all directions. Consider shifting a small window \( W \) by
\( (u,v) \):
\[ E(u,v) = \sum_{(x,y)\in W}[I(x+u,y+v)-I(x,y)]^2 \]
This yields an error comparing each pixel before and after. If we assume a
small amount of motion we can use a first order Taylor Series expansion of
\( I \):
\begin{align*}
  I(x+u,y+v) &\approx I(x,y)+\pdiff{I}{x}u+\pdiff{I}{y}v \\
  &\approx I(x,y)+[I_x\ I_y]\begin{bmatrix}u \\ b\end{bmatrix} \\
  E(u,v) &\approx \sum_{(x,y)\in W}\left[I(x,y)+[I_x\ I_y]
    \begin{bmatrix}u \\ b\end{bmatrix}-I(x,y)\right]^2 \\
  &\approx \sum_{(x,y)\in W}\left[[I_x\ I_y]\begin{bmatrix}u \\ v\end{bmatrix}\right]^2 \\
  &= \sum_{(x,y)\in W}[u\ v]\begin{bmatrix}
    (I_x)^2 & I_xI_y \\
    I_yI_x & (I_y)^2 \\
  \end{bmatrix}\begin{bmatrix}u \\ v\end{bmatrix} \\
  H &= \begin{bmatrix}
    (I_x)^2 & I_xI_y \\
    I_yI_x & (I_y)^2 \\
  \end{bmatrix}
\end{align*}
We can define shifts with the smallest and largest change in \( E(u,v) \),
represented as eigenvalues and eigenvectors of \( H \).
\begin{itemize}
  \item \( x_+ \): direction of largest increase in \( E(u,v) \)
  \item \( \lambda_+ \): amount of increase in direction \( x_+ \)
  \item \( x_- \): direction of smallest increase in \( E(u,v) \)
  \item \( \lambda_- \): amount of increase in direction \( x_- \)
\end{itemize}
\begin{align*}
  Hx_+ &= \lambda_+x_+ \\
  Hx_- &= \lambda_-x_-
\end{align*}
We want \( E(u,v) \) to be large for for small shifts in all directions. The
minimum of \( E(u,v) \) should be large over all unit vectors \( [u\ v] \).
In general, we can do feature detection by:
\begin{enumerate}
  \item Computing the gradient at each point in the image.
  \item Create the \( H \) matrix from the entries in the gradient.
  \item Compute the eigenvalues.
  \item Find points with the largest response \( \lambda > threshold \).
  \item Choose those points where \( \lambda \) is a local maximum as features.
\end{enumerate}

\subsection*{Invariance and Covariance}
We want features to be invariant to photometric transformations and covariant
to geometric transformations. An image is invariant if it is transformed and
features do not change. It is covariant if features are detected in the same
corresponding locations between two transformed versions of the same image.

\begin{center}
  You can find all my notes at \url{http://omgimanerd.tech/notes}. If you have
  any questions, comments, or concerns, please contact me at
  alvin@omgimanerd.tech
\end{center}

\end{document}
