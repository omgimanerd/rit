\documentclass{math}

\title{Principles of Computer Security}
\author{Alvin Lin}
\date{January 2018 - May 2018}

\begin{document}

\maketitle

\section*{The Role of Cryptography in Security}
There are always ways to get around cryptography barriers and these methods have
nothing to do with breaking codes. While systems may use cryptography to make
sure that data is transmitted with perfect security, who's to ensure the
integrity of the person who programs the computer?

\subsection*{Kerckhoff and Shannon}
``A cryptographic system should be designed to be secure, even if all of the
details, except for the key, are publicly known.'' - Auguste Kerckhoff, 1883 \\
``One ought to design systems under the assumption that the enemy will
immediately gain full familiarity with them.'' - Claude Shannon, 1949 \\
These are also related to the Saltzer and Schroeder principles.

\subsection*{Shannon's Characteristics of Good Ciphers}
\begin{enumerate}
  \item The amount of secrecy needed should determine the amount of labor
  appropriate for encryption and decryption.
  \item The set of keys and enciphering algorithm should be free from
  complexity.
  \item The process implementation should be as simple as possible.
  \item Errors in ciphering should not propagate and cause further corruption
  of message information.
  \item The size of enciphered text should not be larger than the original
  message text.
\end{enumerate}

\subsection*{Cryptographic Primitives}
\begin{itemize}
  \item Substitution: one set of bits is exchanged for another
  \item Transposition: rearranging ciphertext order to break repeating patterns
  in plaintext
  \item Confusion: an algorithm providing good confusion has a complex
  functional relationship between the plaintext and ciphertext, so that changing
  one character causes unpredictable changes to ciphertext
  \item Diffusion: distributes information from single plaintext characters over
  the entire ciphertext output
\end{itemize}

\subsection*{Properties of a Trustworthy Cryptosystem}
\begin{itemize}
  \item Must be based on sound mathematics
  \item Must have been analyzed by competent experts and found to be sound
  \item Mus thave stood the test of time
  \item Producing good cryptographic algorithms is a difficult task and
  should be left to those who know how to create such algorithms
\end{itemize}

\subsection*{Symmetric Encryption}
Symmetric encryption is a standard technique for ensuring confidentiality for
data at rest or in transit. It has two requirements: a strong encryption
algorithm, and a secure secret key that is shared by the sender and receiver.
Secure symmetric encryption has one point of failure. Once the key is
compromised, the integrity of the message is compromised.
\begin{itemize}
  \item Cryptanalytic Attacks: these attacks make use of some knowledge of
  general characteristics of the plaintext and use the nature fo the algorithm
  to figure out the specific plaintext or the key used. If this is successful,
  all future and past messages encrypted with that key are compromised.
  \item Brute-Force Attacks: these attacks try all possible keys on some
  ciphertext until an intelligible translation into plaintext is obtained. On
  average, half of all possible keys must be tried for success.
\end{itemize}

\subsection*{DES, 3DES, and AES}
\begin{itemize}
  \item Data Encryption Standard (DES): the DES algorithm uses 64-bit plaintext
  blocks and a 56-bit key to produce a 64-bit ciphertext block. It is no longer
  considered secure due to the inadequacy of 56-bit keys.
  \item Triple DES (3DES): the 3DES algorithm repeats DES three times using two
  or three unique keys. 168-bits help prevent brute-force attacks but it is a
  sluggish algorithm.
  \item Advanced Encryption Standard (AES): AES was developed as a replacement
  for 3DES that significantly improved efficiency without sacrificing security
  strength.
\end{itemize}
Typical symmetric encryption is applied to a unit of data larger than a single
64-bit or 128-bit block.

\subsubsection*{Average Time Required for Exhaustive Key Search}
\begin{center}
  \begin{tabular}{|p{1cm}|c|c|p{3cm}|p{3cm}|}
    \hline
    Key size (bits) & Cipher & Number of Alternative Keys &
      Time Required at \( 10^9 \) decryptions/s &
      Time Required at \( 10^13 \) decryptions/s \\
    \hline
    56 & DES & \( 2^{56}\approx7.2\times10^{16} \) & 1.125 years & 1 hour \\
    \hline
    128 & AES & \( 2^{128}\approx3.4\times10^{38} \) & \( 5.3\times10^{21} \)
      years & \( 5.3\times10^{17} \) years \\
    \hline
    168 & 3DES & \( 2^{168}\approx3.7\times10^{50} \) & \( 5.8\times10^{33} \)
      years & \( 5.8\times10^{29} \) years \\
    \hline
    192 & AES & \( 2^{192}\approx6.3\times10^{57} \) & \( 9.8\times10^{40} \)
      years & \( 9.8\times10^{36} \) years \\
    \hline
    256 & AES & \( 2^{256}\approx1.2\times10^{77} \) & \( 1.8\times10^{60} \)
      years & \( 1.8\times10^{56} \) years \\
    \hline
  \end{tabular}
\end{center}

\subsection*{Block and Stream Ciphers}
Block Cipher:
\begin{itemize}
  \item processes input, one block of elements at a time
  \item produces an output block for each input block
  \item can reuse keys
  \item more common
\end{itemize}
Stream Cipher:
\begin{itemize}
  \item processes input elements continuously
  \item produces output one element at a time
  \item usually always faster, with less code
  \item encrypts plaintext one byte at a time
  \item pseudorandom stream is one that is unpredictable without knowledge of
  the input key
\end{itemize}

\subsection*{Message Authentication}
Message authentication protects against active attacks and verifies that a
received message is authentic. A message is authentic if its contents have not
been altered, it is from an authentic source, and it is timely and in the
correct sequence. For message authentication, hash functions with the following
properties are used:
\begin{itemize}
  \item applicable to a block of data of any size
  \item produces a fixed-length output
  \item one-way or pre-image resistant, computationally infeasible to find
  \( x \) such that \( H(x) = h \)
  \item computationally infeasible to find \( y\ne x \) such that
  \( H(y) = H(x) \)
  \item collision resistant or strong collision resistance, computationally
  infeasible to find any pair \( (x,y) \) such that \( H(x) = H(y) \)
\end{itemize}
Secure hash functions can be attacked by cryptanalysis, which exploits
logical weaknesses in the algorithm, or by brute force, which solely depends
on the hash code length produced.

\subsection*{Public-Key Encryption}
Public-key encryption is based and mathematicaly functions and was publicly
proposed by Diffie and Hellman in 1976. It is asymmetric and uses two separate
keys, a public and private key. Ideally, it should be computationally easy to
create and distribute key pairs, but it should be computationally infeasible
to determine the private key from the public key or otherwise recover the
original message.

\subsubsection*{Asymmetric Encryption Algorithms}
\begin{itemize}
  \item \textbf{RSA (Rivest, Shamir, Adleman), 1977} is the most widely
  accepted and implemented PKE approach. It involves a block cipher in which
  the plaintext and ciphertext are integers are between 0 and \( n-1 \) for
  \( n \).
  \item The \textbf{Diffie-Hellman key exchange algorithm} enables two users to
  securely reach agreement about a shared secret for use as a secret key for
  subsequent symmetric encryption.
  \item The \textbf{Digital Signature Standard (DSS)} provides a digital
  signature function with SHA-1 and cannot be used for encryption or key
  exchange.
  \item \textbf{Elliptic Curve Cryptography (ECC)} has security like RSA, but
  with much smaller keys.
\end{itemize}

\subsection*{Random Numbers}
Random numbers are used to generate keys for public-key algorithms, keys for
symmetric stream ciphers, handshaking, or session keys. In order for a number
sequence to be considered random, it must have uniform distribution,
independence, and unpredictability. The frequency of each number occurrence
should be approximately the same and no value in the sequence should be
inferred from another. Each number should be statistically independent from
other numbers in the sequence and attacks should not be able to predict future
elements of the sequence based on earlier sequences.
\par Cryptographic applications typically use algorithms for random number
generation. Since they are deterministic, they typically produce sequences
that are not statistically random. Thus, they are likely to be predictable.
True random number generators use some nondeterministic source to produce
randomness, such as radiation, gas discharge, or leaky capacitors.

\begin{center}
  You can find all my notes at \url{http://omgimanerd.tech/notes}. If you have
  any questions, comments, or concerns, please contact me at
  alvin@omgimanerd.tech
\end{center}

\end{document}
