\documentclass{math}

\title{Principles of Computer Security}
\author{Alvin Lin}
\date{January 2018 - May 2018}

\begin{document}

\maketitle

\section*{The Role of Cryptography in Security}
There are always ways to get around cryptography barriers and these methods have
nothing to do with breaking codes. While systems may use cryptography to make
sure that data is transmitted with perfect security, who's to ensure the
integrity of the person who programs the computer?

\subsection*{Kerckhoff and Shannon}
``A cryptographic system should be designed to be secure, even if all of the
details, except for the key, are publicly known.'' - Auguste Kerckhoff, 1883 \\
``One ought to design systems under the assumption that the enemy will
immediately gain full familiarity with them.'' - Claude Shannon, 1949 \\
These are also related to the Saltzer and Schroeder principles.

\subsection*{Shannon's Characteristics of Good Ciphers}
\begin{enumerate}
  \item The amount of secrecy needed should determine the amount of labor
  appropriate for encryption and decryption.
  \item The set of keys and enciphering algorithm should be free from
  complexity.
  \item The process implementation should be as simple as possible.
  \item Errors in ciphering should not propagate and cause further corruption
  of message information.
  \item The size of enciphered text should not be larger than the original
  message text.
\end{enumerate}

\subsection*{Cryptographic Primitives}
\begin{itemize}
  \item Substitution: one set of bits is exchanged for another
  \item Transposition: rearranging ciphertext order to break repeating patterns
  in plaintext
  \item Confusion: an algorithm providing good confusion has a complex
  functional relationship between the plaintext and ciphertext, so that changing
  one character causes unpredictable changes to ciphertext
  \item Diffusion: distributes information from single plaintext characters over
  the entire ciphertext output
\end{itemize}

\subsection*{Properties of a Trustworthy Cryptosystem}
\begin{itemize}
  \item Must be based on sound mathematics
  \item Must have been analyzed by competent experts and found to be sound
  \item Mus thave stood the test of time
  \item Producing good cryptographic algorithms is a difficult task and
  should be left to those who know how to create such algorithms
\end{itemize}

\subsection*{Symmetric Encryption}
Symmetric encryption is a standard technique for ensuring confidentiality for
data at rest or in transit. It has two requirements: a strong encryption
algorithm, and a secure secret key that is shared by the sender and receiver.
Secure symmetric encryption has one point of failure. Once the key is
compromised, the integrity of the message is compromised.

\begin{center}
  You can find all my notes at \url{http://omgimanerd.tech/notes}. If you have
  any questions, comments, or concerns, please contact me at
  alvin@omgimanerd.tech
\end{center}

\end{document}
