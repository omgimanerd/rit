\documentclass{math}

\title{Principles of Computer Security}
\author{Alvin Lin}
\date{January 2018 - May 2018}

\begin{document}

\maketitle

\section*{Privacy}
Privacy is a concept that overlaps with security, especially confidentiality.
There has been a dramatic increase in the scale of information that is collected
and stored for law enforcement, national security, and economic incentives.
Individuals have become increasingly aware of access and use of personal
information and private details about their lives. Concerns about the extend
of privacy compromise have led to a variety of legal and technical approaches
to reinforcing privacy rights.

\subsection*{European Union Directive on Data Protection}
This directive was adopted in 1998 and states that members must protect
fundamental privacy rights when processing personal information. It was
organized around the following principles:
\begin{itemize}
  \item Notice
  \item Consent
  \item Consistency
  \item Access
  \item Security
  \item Onward transfer
  \item Enforcement
\end{itemize}

\subsection*{United States Privacy Initiatives}
The Privacy Act of 1974 dealt with personal information collected and used by
federal agencies. It permitted individuals to determine the records kept on
them, forbid records being used for other purposes, and obtain access to
records, and correct and amend records appropriately. This act ensured agencies
would properly collect, maintain, and use personal information. It created a
private right of action for individuals.

\subsection*{Code of Practice for Information Security Management (ISO 27002)}
\begin{itemize}
  \item Organizations must develop and implement a data policy for the privacy
  and protection of personally identifiable information.
  \item They must communicate the policy to all persons who process personally
  identifiable information.
  \item They must develop appropriate management structure and control to comply
  with this policy and relevant legislation/regulations.
  \item They must appoint a responsible person (a privacy officer) to guide
  managers, users, and service providers on their individual responsibilities
  and specific procedures.
  \item Their responsibility for handling personally identifiable information
  and awareness of privacy principles must following relevant laws and
  regulations.
  \item They must implement appropriate technical and organizational measures to
  protect personally identificable information.
\end{itemize}

\subsection*{Properties of Privacy}
\begin{itemize}
  \item Unobservability: A user may use the resource or service without others
  being aware of its use. Germany views unobservability essential to
  constitutional rights.
  \item Anonymity: A user may use the resource or service without disclosing
  their identity. The service should protect the user identity. This makes
  it harder to establish trust between the system and users.
  \item Unlinkability: A user may use the resource or service multiple times
  without others being able to link the uses together.
  \item Pseudonymity: A user may use the resource or service without disclosing
  their identity, but can still be accountable for that use.
\end{itemize}
None of these properties can be implemented totally. Any interaction between a
user and a security system leaks information about users. The main threat is
that attacks can collect long-term user data and use statistical methods to
deanonymize users.

\subsection*{Privacy and Data Surveillance}
The demands of homeland security and counterterrorism have imposed new threats
to personal privacy. Police and intelligence agencies are now aggressive in
using data surveillance to fulfill their missions. Private organizations are
also exploiting their abilities to build details profiles of individuals
through the Internet, increases in electronic payments, cell phones, and other
mobile devices. Both policy and technical approaches are needed to protect
privacy when both government and nongovernment organizations seek to learn as
much as possible about individuals.

\subsection*{K-Anonymity}
Can private data about people be shared with researchers with guarantees that
individuals cannot be re-identified while the data is practically useful? The
k-anonymity protection model allows for data to be released only when any
one person cannot be distinguished from at least k-1 individuals whose data is
also in the release.

\subsection*{Differential Privacy}
Differential privacy approaches only allow queries where the result is claimed
to have strong, provable privacy guarantees.

\begin{center}
  You can find all my notes at \url{http://omgimanerd.tech/notes}. If you have
  any questions, comments, or concerns, please contact me at
  alvin@omgimanerd.tech
\end{center}

\end{document}
