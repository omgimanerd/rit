\documentclass{math}

\title{Principles of Computer Security}
\author{Alvin Lin}
\date{January 2018 - May 2018}

\begin{document}

\maketitle

\section*{Data and Database Security}
Databases are structured collections of data for use by one or more
applications. They contain relationships between data items and groups of data
items. They can contain sensitive data to be secured and use a query language
to provide a uniform interface to the database. A database management system
(DBMS) is a suite of programs for constructing and maintaining the database.

\subsection*{Relational Databases}
Relational databases consist of rows and columns that hold a particular type of
data. Each row has one column where all values are unique, forming a key for
that row. This enables the creation of multiple tables linked together by a
unique identifier that is present in all tables. A relational query language
allows users to request data in this database to fit a given criteria.

\subsubsection*{Relational Database Elements}
Terminology:
\begin{itemize}
  \item A database is composed of relations/tables/files, which holds
  tuples/rows/records that have attributes/columns/fields.
  \item A primary key uniquely identifies a row and consists of one or more
  column names.
  \item A foreign key links one table to attributes in another.
  \item A view or virtual table shows the result of a query that returns
  selected rows and columns from one or more tables.
\end{itemize}

\subsubsection*{Structured Query Language (SQL)}
SQL is a language for data manipulation. At its core, it supports a standard
known as CRUD:
\begin{itemize}
  \item Create data
  \item Read data
  \item Update data
  \item Delete data
\end{itemize}
It also defines the database structure as a data definition language, allowing
for a database structure with tables, views, indexes, unique keys. Database
structures can be created, modified, and deleted. The database schema generates
tables stored in a data dictionary which contains metadata about how the table
should store data.

\subsubsection*{Security}
Security with data is not merely securing the database. The data, database,
DBMS must be secure along with applications, operation systems, web servers,
and network environments that can access the data. Continuous patching of DBMS
is needed in order to fix new vulnerabilities that come to light. The
interaction of the DBMS with the OS must be secure, with secure administrative
accounts, policies, and permissions. The interaction of the DBMS with the OS
must be secure. Connections between the clients and the server must be secure,
though this may have the side effect of limiting possible connections. The
network that the DBMS is on must enforce authentication, integrity, and
encryption. Generally, the database server lies behind a firewall and is
separated from the web server.

\subsubsection*{Secure Application Development}
Applications that access DBMS services can be subject to SQL injections. If
sensitive data is stored in the database, it can be leaked if the user input
is not sanitized, allowing for arbitrary queries to be executed on the DBMS.

\subsubsection*{Data Privacy}
Data privacy is a field that overlaps with data security, especially
confidentiality. In today's world, there has been a dramatic increase in the
amount of data aggregated and stored, often motivated by law enforcement,
national security, and economic incentives. People have become increasingly
aware of the access and use of their personal information. Concerns about the
extent of privacy compromise have led to a variety of legal and technical
approaches to protect privacy.

\subsubsection*{Security Specifications with SQL Databases}
\begin{itemize}
  \item The grant statement is used to confer authorization. It does not grant
  privilege to access any underlying views, and the grantor must already have
  permission to access the specified item before granting permission.
  \item Permission can be granted to select, insert, update, or delete data.
  \item Roles are often assigned to users to specify common classes of
  privileges.
  \item The revoke statement revokes permissions on a user or group. Revoking
  a user privilege may cascade to others, and this can cause interesting
  security issues with both timing and doubly granted permissions.
  \item SQL does not support tuple level authorization. Authorization to access
  individual data must be done by the web application on top of the database.
  This is often difficult because authorization loopholes can occur in multiple
  parts of the web application.
\end{itemize}

\subsubsection*{SQL Injection Attacks}
SQL injection is the most common network based security threat to SQL databases.
It exploits the basic web application design by sending malicious SQL queries to
the database server. The most common use of this is bulk data extraction, but
this can be used to modify data, delete data, execute arbitrary OS commands, or
even launch DDOS attacks. This is often done by prematurely terminating a text
string and appending a command to the end of it. Attack avenues:
\begin{itemize}
  \item User input attack: inject SQL commands by crafting suitable user inputs
  to send
  \item Server variables attack: forge HTTP values and network headers to
  exploit vulnerabilities by placing data directly into the headers
\end{itemize}

\begin{center}
  You can find all my notes at \url{http://omgimanerd.tech/notes}. If you have
  any questions, comments, or concerns, please contact me at
  alvin@omgimanerd.tech
\end{center}

\end{document}
