\documentclass{math}

\title{Principles of Computer Security}
\author{Alvin Lin}
\date{January 2018 - May 2018}

\begin{document}

\maketitle

\section*{Exam 1 Key}
\begin{enumerate}
  \item \begin{enumerate}
    \item CI no-A: Each student can look up their own data and modify their
    registration for a semester (they can do both C and I), but SIS is down (no
    A) to the student.
    \item IA no-C: Each student's data is public, but is only modifiable by
    an authorized faculty member.
  \end{enumerate}
  \item The two steps of authentication are identification (where the subject
  presents an identifier) and verification (where the system corroborates the
  binding between the subject and presented identifier).
  \item Capability lists are the rows of an access control matrix. Each row is
  for a subject (or role), which denotes the access privileges of that subject
  to different objects. For example, a subject Jo's capability list would show
  what Jo can do to each object.
  \item \begin{enumerate}
    \item MAC Lattice
    \item Clearance is a label associated with each subject while
    classification is a label associated with each object. Each label has two
    parts: a level of access and the applicable type of objects.
    \item In Bell-LaPadula, a node x above another node y in the MAC lattice
    dominates y for \textbf{confidentiality (reading)}. When a subject Jo with
    clearance (Secret, {Maps}) wants to \textbf{read} RITCampusMaps with
    classification (Public, {Maps}), the BLP reference monitor checks the
    lattice, and grants access if Jo's label dominates the map's label. It
    rejects access otherwise to maintain confidentiality.
    \item In Biba, a node x above another node y in the MAC lattice dominates
    y for \textbf{integrity (writing)}. When a subject Jo with clearance
    (Secret, {Maps}) wants to \textbf{update} RITCampusMaps with classification
    (Public, {Maps}), the reference monitor checks the lattice and grants Jo
    permission to modify it if Jo's label dominates the map's label.
  \end{enumerate}
  \item Because BLP restricts too much acces, it really needs the notion of a
  trusted subject for it to work correctly and efficiently. Having trusted
  subjects means that BLP is pushing a great deal of power to people who work
  outside the software system.
  \item RBAC's drawbacks are that it is coarse-grained and requires a complex
  initial setup ahead of time.
  \item ABAC's strengths are that it is flexible and fine-grained.
  \item ABAC's drawbacks are that its run-time access decisions can't be
  analyzed ahead of time, policies are complex so making changes to permissions
  is hard, and it is difficult to review permissions.
  \item ``Encraption'' does not meet the needs of a trusted cryptosystem as it
  may violate any or all of the following:
  \begin{enumerate}
    \item sound mathematics
    \item anayzed by competent experts and found to be sound
    \item stood the test of time
  \end{enumerate}
  \item Homomorphic encryption permits operations on ciphertext, reducing
  opportunities for information leakage.
  \item ElGamal supports homomorphic addition between two ciphertexts, and thus
  scalar multiplication. Therefore, the scalar has to be in plaintext and so the
  threshold value used in the secure comparision protocol is also in the clear.
  This limits existing solutions to be in two-party settings only.
  \item \begin{enumerate}
    \item Economy of mechanism: security mechanisms should be as simple as
    possible.
    \item Open design: security of a mechanism should not depend on the secrecy
    its design or implementation.
    \item Separation of privilege: must not grant permission only on a single
    condition.
    \item Complete mediation: check each object access to ensure they are
    allowed.
  \end{enumerate}
  \item \begin{enumerate}
    \item Physical access to a computer renderse all software security
    meaningless. Servers and protected data should also be physically isolated,
    locked, and protected.
    \item Human error and bad practice renders security systems meaningless.
    A human writing passcodes on paper negates the purpose of a password
    authentication system.
    \item Attacks should not be able to compromise all systems irreversibly.
    Sensitive data should be encrypted and backed up.
  \end{enumerate}
  \item \begin{enumerate}
    \item Active risks should be monitored constantly, with prevention systems
    ready to address and mitigate the disaster. Prevention systems should be
    prioritized by the threat level.
    \item Prevention systems should be used to address and mitigate
    disaster. The scope and effect should be minimized as much as possible.
    \item If the prevention systems were not effective, they should be modified
    and adjusted according to the threat. Damage assessment and correct should
    also occur after the disaster.
  \end{enumerate}
\end{enumerate}

\begin{center}
  You can find all my notes at \url{http://omgimanerd.tech/notes}. If you have
  any questions, comments, or concerns, please contact me at
  alvin@omgimanerd.tech
\end{center}

\end{document}
