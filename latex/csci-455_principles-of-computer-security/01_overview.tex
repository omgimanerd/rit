\documentclass{math}

\title{Principles of Computer Security}
\author{Alvin Lin}
\date{January 2018 - May 2018}

\begin{document}

\maketitle

\section*{Overview}
This course is unique in that it is the only course with the presence of an
adversary. We have to defend against all sorts of attacks, crude or not.

\subsubsection*{Military Humor}
Secure a building:
\begin{itemize}
  \item Army: Put guards around the place
  \item Navy: Turn out the lights and lock the doors
  \item Air Force: Take out a 5-year lease with an option to buy
  \item Marines: Kill everyone inside and make it a command post
\end{itemize}

\subsubsection*{What does it mean to secure a computer system?}
\begin{itemize}
  \item Kill every user who is using it?
  \item Turn off the computer?
  \item Prevent unauthorized access
\end{itemize}

\subsubsection*{Important Reminders}
\begin{itemize}
  \item The final project is graded as a group but each individual may receive
  a different grade depending on effort.
  \item Email is not the best medium to contact the professor, use MyCourses.
  Each student will have a private channel to the professor.
  \item Let the professor know in advance if you will be missing class if
  possible.
  \item Do not send assignments by email.
\end{itemize}

\subsubsection*{Discussion Question}
Consider this claim:
\begin{itemize}
  \item Security is only possible in a world in which everyone trusts everyone.
  \item Security is only possible in a world where no one trusts anyone.
\end{itemize}

\begin{center}
  You can find all my notes at \url{http://omgimanerd.tech/notes}. If you have
  any questions, comments, or concerns, please contact me at
  alvin@omgimanerd.tech
\end{center}

\end{document}
