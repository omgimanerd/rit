\documentclass{math}

\title{Principles of Computer Security}
\author{Alvin Lin}
\date{January 2018 - May 2018}

\begin{document}

\maketitle

\section*{Overview}
This course is unique in that it is the only course with the presence of an
adversary. We have to defend against all sorts of attacks, crude or not.

\subsubsection*{Military Humor}
Secure a building:
\begin{itemize}
  \item Army: Put guards around the place
  \item Navy: Turn out the lights and lock the doors
  \item Air Force: Take out a 5-year lease with an option to buy
  \item Marines: Kill everyone inside and make it a command post
\end{itemize}

\subsubsection*{What does it mean to secure a computer system?}
\begin{itemize}
  \item Kill every user who is using it?
  \item Turn off the computer?
  \item Prevent unauthorized access
\end{itemize}

\subsubsection*{Important Reminders}
\begin{itemize}
  \item The final project is graded as a group but each individual may receive
  a different grade depending on effort.
  \item Email is not the best medium to contact the professor, use MyCourses.
  Each student will have a private channel to the professor.
  \item Let the professor know in advance if you will be missing class if
  possible.
  \item Do not send assignments by email.
\end{itemize}

\subsubsection*{Discussion Question}
Consider this claim:
\begin{itemize}
  \item Security is only possible in a world in which everyone trusts everyone.
  \item Security is only possible in a world where no one trusts anyone.
\end{itemize}

\subsection*{What is Computer Security?}
``The protection afforded to an automated information system in order to attain
the applicable objectives of preserving the integrity, availability, and
confidentiality of information system resources (includes hardware, software,
firmware, information/data, and telecommunications).'' -NIST Computer Security
Handbook.
\par A computer scientist is someone who can apply computer science theory and
software development fundamentals to produce computing-based solutions.
Security should be involved in all factors of being a computer scientist.

\clearpage

\subsection*{Security: Knowledge Areas}
\begin{center}
  \begin{tabular}{|c|p{10cm}|}
    \hline
    Data Security & Protection of data at rest, during processing, and in
      transit \\
    \hline
    Software Security & Development and use of software that reliably preserves
      the security properties of the protected information and systems \\
    \hline
    Component Security & The security aspects of the design, procurement,
      testing, analysis, and maintenance of components integrated into larger
      systems \\
    \hline
    Connection Security & Security of the connections between components, both
      physical and logical \\
    \hline
    System Security & Security aspects of systems that are composed of
      components and connections, and use software \\
    \hline
    Human Security & The study of human behavior in the context of data
      protection, privacy, and threat mitigation \\
    \hline
    Organizational Security & Protecting organizations from cybersecurity
      threats and managing risk to support the successful accomplishment of the
      organizations' missions \\
    \hline
    Societal Security & Aspects of cybersecurity that broadly impact society as
      a whole \\
    \hline
  \end{tabular}
\end{center}

\subsection*{C-I-A Pillar: Three Objectives}
\textbf{Confidentiality:}
\begin{itemize}
  \item Data confidentiality: private or confidential information is not made
  available or disclosed to unauthorized individuals.
  \item Privacy: assures that individuals can control or influence what
  information related to them may be collected/stored, and by whom, and to whom
  it may be disclosed.
\end{itemize}
\textbf{Integrity:}
\begin{itemize}
  \item Data integrity: assures that information and programs are changed only
  in a specified and authorized manner.
  \item System integrity: assures that a system performs its intended function
  in an unimpaired manner, free from deliberate or inadvertent unauthorized
  manipulation of the system.
\end{itemize}
\textbf{Availability:}
\begin{itemize}
  \item Assures that systems work promptly and service is never denied to
  authorized users.
\end{itemize}

\subsection*{From CIA to CIANA}
\textbf{Non-repudiation:}
\begin{itemize}
  \item Assures that the system has the ablity to correlate, with high
  certainty, a recorded action with its originating individual or entity.
\end{itemize}
\textbf{Authentication:}
\begin{itemize}
  \item Assures that the system has the ability to verify the identity of an
  individual or entity.
\end{itemize}

\subsection*{Security Principles}
\begin{itemize}
  \item Least privilege: a subject should be given only those priveleges that it
  needs to complete its task.
  \item Fail-safe defaults: unless a subject is given explicit access to an
  object, it should be denied access to it.
  \item Economy of mechanism: security mechanisms should be as simple as
  possible.
  \item Complete mediation: all accesses to objects be checked to ensure that
  they are allowed.
  \item Open design: security of a mechanism must not depend on
  design/implementation secrecy.
  \item Separation of privilege: system must not grant permission based on a
  single condition.
  \item Least common mechanism: mechanisms for resource access must not be
  shared.
  \item Psychological acceptability: security mechanisms must not make resources
  more difficult to access than if they were not present.
\end{itemize}

\subsection*{Security Breach Impact}
FIPS 199: Standards for Security Categorization of Federal Information and
Information Systems. \\
\textbf{Low}:
\begin{itemize}
  \item Loss expected to have limited adverse effect on operations, assets, or
  individuals.
  \item Some degradation, minor asset damange, some financial loss, and minor
  individual harm.
\end{itemize}
\textbf{Moderate}:
\begin{itemize}
  \item Serious adverse effect on operations, organizational assets, or
  individuals.
  \item Significant degradation, damage to assets, financial loss, or harm to
  individuals that does not involve loss of life or serious, life-threatening
  injuries.
\end{itemize}
\textbf{High}:
\begin{itemize}
  \item Severe or catastrophic effect on operations, organization assets, or
  individuals.
  \item Organization is not able to perform one or more of its primary
  functions, or major damage to organizational assets, or major financial loss,
  or severe harm to individuals involving loss of life or serious
  life-threatening injuries.
\end{itemize}

\begin{center}
  You can find all my notes at \url{http://omgimanerd.tech/notes}. If you have
  any questions, comments, or concerns, please contact me at
  alvin@omgimanerd.tech
\end{center}

\end{document}
