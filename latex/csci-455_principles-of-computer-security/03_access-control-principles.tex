\documentclass{math}

\title{Principles of Computer Security}
\author{Alvin Lin}
\date{January 2018 - May 2018}

\begin{document}

\maketitle

\section*{Access Control Principles}
Subjects make requests to active entities such as users or processes. Subjects
can be active users or processes, and objects are usually passive entities
manipulated by a subject such as records, relations, or files. Access control
helps protect objects from unauthorized disclosure and unauthorized
modification. It is an approach to regulate access requests by subjects to
objects to perform certain operations through a set of access policies.

\subsection*{Access Control Policies}
\begin{itemize}
  \item Discretionary Access Control (DAC): based on requestor identity and
  access rules stating what requestors are allowed or not allowed to do.
  \item Mandatory Access Control (MAC): based on comparing security labels
  with security clearances.
  \item Role-based Access Control (RBAC): based on user roles within the system
  and on rules stating what accesses are allowed to users in given roles.
  \item Attribute-based Access Control (ABAC): based on user attributes, the
  resource to be accessed, and current environmental conditions.
  \item Content-based, Context-based, History-based, etc.
\end{itemize}

\subsection*{Access Control Matrix}
\begin{center}
  \begin{tabular}{|c|c|c|c|c|c|}
    \hline
    & Object 1 & Object 2 & Object 3 & Object 4 & Object 5 \\
    \hline
    S1 & r & r & rw & & \\
    \hline
    S2 & rw & & r & wx & \\
    \hline
    S3 & r & rw & rwx & rwx & w \\
    \hline
    S4 & & & & & \\
    \hline
  \end{tabular}
\end{center}
Each column in this matrix is an access control list that determines each
object's list of access rights to subjects. Each row in the matrix is a
capabilities list that determines each subject's list of capabilities for each
object.

\subsection*{DAC Overview}
With discretionary access control, entities are able to access resources based
on some access control matrix. Each entry in the access matrix controls a
subject's permission to access some object. The UNIX file access control uses
discretionary access control by allowing file users, various groups, or the
public to read, write, or execute the file based on the permission bits set for
the file.

\subsection*{MAC Overview}
The mandatory acccess control model is sometimes known as the non-discretionary
model. It works with multi-level security and is suited for handling data with
multiple sensitivity levels. It permits simultaneous data access by users with
different clearance levels and need-to-know, while prevent users from obtaining
access to information for which they lack authorization.
\begin{center}
  \begin{tabular}{|c|c|c|}
    \hline
    Clearance Groups & Read Access & Write Access \\ \hline
    Top Secret & \checkmark & \checkmark \\ \hline
    Secret & \checkmark & \checkmark \\ \hline
    Confidential & \checkmark & \\ \hline
    Public & \checkmark & \\
    \hline
  \end{tabular}
\end{center}
A lattice is often used to categorize different objects into categories under a
clearance level. If a subject's access label does not encompass the object's
label and category, then the subject is denied access.

\subsubsection*{Multi-level Security Models}
Bell-LaPadula:
\begin{itemize}
  \item Designed to protect confidentiality
  \item Top secret, Secret, Confidential, Public
  \item No Read Up, No Write Down
  \item Trusted subjects are allowed to violate insert, update, and delete MACs
\end{itemize}
Biba:
\begin{itemize}
  \item Designed to protect integrity
  \item Top secret, Secret, Confidential, Public
  \item No Read Down, No Write Up
  \item Trusted subjects are allowed to violate insert, update, and delete MACs
\end{itemize}

\subsection*{RBAC Overview}
\begin{itemize}
  \item A subject has access to an object based on an assigned role.
  \item Roles are typically defined based on job functions.
  \item Permissions are defined based on job authority and responsibilities
  within a job function.
  \item Operations on an object invoked based on permissions.
  \item Object access depends on a subject's role, not the subject.
\end{itemize}
This system is much more pragmatic, but has a lack of granularity and
flexibility. Improper actions are not always the same as unauthorized actions.
Certain users may be able to perform actions that are not proper for their role
because it would be logistically difficult to enforce that level of granularity
with RBAC systems. There is no connection between who the person is and who gets
permission.

\subsection*{ABAC Overview}
\begin{itemize}
  \item Can define authorizations that express conditions on properties of both
  the resource and the subject.
  \item Strength is its flexibility and expressive power.
  \item The main obstacle to its adoption in real systems is its performance
  impact when evaluating predicates on both resource and user properties for
  each access.
  \item Web services have been pioneering technologies through the introduction
  of the eXtensible Access Control Markup Language (XAMCL).
  \item There is a considerable interest in applying this model to cloud
  services.
  \item Subject attributes in this model define the identity and
  characterestics of the subject.
  \item Objects have attributes that can be leveraged to make access control
  decisions.
  \item Environment attributes describe operational, technical, and situational
  environment or context in which information access occurs. It is largely
  ignored in most other access control policies.
\end{itemize}
ABAC systems are capable of enforcing DAC, RBAC, and MAC concepts. They allow
an unlimited number of attributes to be combined to satisfy any access control
rule.

\subsection*{RBAC vs ABAC}
\begin{center}
  \begin{tabular}{|c|c|}
    \hline
    \textbf{RBAC} & \textbf{ABAC} \\
    \hline
    Static & Dynamic \\
    Coarse-grained & Fine-grained \\
    Access decisions made in advance & Access decisions made at run-time \\
    Simply policy & Complex policy \\
    Complex setup & Simple setup \\
    \hline
  \end{tabular}
\end{center}

\begin{center}
  You can find all my notes at \url{http://omgimanerd.tech/notes}. If you have
  any questions, comments, or concerns, please contact me at
  alvin@omgimanerd.tech
\end{center}

\end{document}
