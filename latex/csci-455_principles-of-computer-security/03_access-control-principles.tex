\documentclass{math}

\title{Principles of Computer Security}
\author{Alvin Lin}
\date{January 2018 - May 2018}

\begin{document}

\maketitle

\section*{Access Control Principles}
Subjects make requests to active entities such as users or processes. Subjects
can be active users or processes, and objects are usually passive entities
manipulated by a subject such as records, relations, or files. Access control
helps protect objects from unauthorized disclosure and unauthorized
modification. It is an approach to regulate access requests by subjects to
objects to perform certain operations through a set of access policies.

\subsection*{Access Control Policies}
\begin{itemize}
  \item Discretionary Access Control (DAC): based on requstor identity and
  access rules stating what requestors are allowed or not allowed to do.
  \item Mandatory Access Control (MAC): based on comparing security labels
  with security clearances.
  \item Role-based Access Control (RBAC): based on user roles within the system
  and on rules stating what accesses are allowed to users in given roles.
  \item Attribute-based Access Control (ABAC): based on user attributes, the
  resource to be accessed, and current environmental conditions.
  \item Content-based, Context-based, History-based, etc.
\end{itemize}

\subsection*{Access Control Matrix}
\begin{center}
  \begin{tabular}{|c|c|c|c|c|c|}
    \hline
    & Object 1 & Object 2 & Object 3 & Object 4 & Object 5 \\
    \hline
    S1 & r & r & rw & & \\
    \hline
    S2 & rw & & r & wx & \\
    \hline
    S3 & r & rw & rwx & rwx & w \\
    \hline
    S4 & & & & & \\
    \hline
  \end{tabular}
\end{center}
Each column in this matrix is an access control list that determines each
object's list of access rights to subjects. Each row in the matrix is a
capabilities list that determines each subject's list of capabilities for each
object.

\subsection*{RBAC Overview}
\begin{itemize}
  \item A subject has access to an object based on an assigned role.
  \item Roles are typically defined based on job functions.
  \item Permissions are defined based on job authority and responsibilities
  within a job function.
  \item Operations on an object invoked based on permissions.
  \item Object access depends on a subject's role, not the subject.
\end{itemize}
This system is much more pragmatic, but has a lack of granularity and
flexibility. Improper actions are not always the same as unauthorized actions.
Certain users may be able to perform actions that are not proper for their role
because it would be logistically difficult to enforce that level of granularity
with RBAC systems. There is no connection between who the person is and who gets
permission.

\begin{center}
  You can find all my notes at \url{http://omgimanerd.tech/notes}. If you have
  any questions, comments, or concerns, please contact me at
  alvin@omgimanerd.tech
\end{center}

\end{document}
