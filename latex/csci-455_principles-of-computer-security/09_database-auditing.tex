\documentclass{math}

\title{Principles of Computer Security}
\author{Alvin Lin}
\date{January 2018 - May 2018}

\begin{document}

\maketitle

\section*{Database Auditing}
A security audit is an independent review or examination of system records and
activities for the purpose of:
\begin{itemize}
  \item Determining the adequacy of system controls
  \item Ensuring compliance with established security policy and procedures
  \item Detecting breaches in security services
  \item Recommending any changes needed for countermeasures
\end{itemize}
The basic object of an audit is to establish accountability for system entities
that initiate or participate in security-relevant events and actions. Means are
needed to generate and record a security audit trail and to review and analyze
the audit trail to discover and investigate attacks. Usually this is done by
having a chronological record of system activities enabling the reconstruction
and examination of a sequence of environments and activities surrounding
or leading to an operation, procedure, or event in a security relevant
transaction.

\subsection*{Events Subject to Audits}
In order to gain useful insights, we need to define the set of events that are
subject to audit:
\begin{itemize}
  \item Introduction of objects
  \item Deletion of objects
  \item Distribution or revocation of access rights or capabilities
  \item Changes to subject or object security attributes
  \item Policy checks performed by the security software
  \item Use of access rights to bypass a policy check
  \item Use of identification and authentication functions
  \item Security related actions taken by an operator or user
  \item Import or export of data from/to removable media
\end{itemize}

\subsection*{Event Detection}
Appropriate hooks must be available in the application and system software to
enable event detection. Monitoring software needs to be added to the system and
to appropriate places to capture relevant activity. An event recording function
is needed, which includes the need to provide for a secure storage resistant to
tampering or deletion. Event and audit trail analysis software, tools, and
interfaces may be used to analyze collected data as well as for investigating
data trends and anomalies. Any auditing system should have a minimal effect on
functionality.

\subsection*{Implementation Guidelines}
\begin{itemize}
  \item Audit requirements should be agreed upon with appropriate management.
  \item The scope of technical audit tests should be agreed upon and controlled.
  \item Audio tests should be limited to read-only access to software and data.
  \item Access other than read-only should only be allowed for isolated copies
  of system files.
  \item Any requirements for special or additional processing should be
  identified and agreed upon.
  \item Audit tests that could affect system availability should be run outside
  business hours.
  \item All access should be monitored and logged to produce a reference trail.
\end{itemize}

\subsection*{Database Audit Trail}
A mechanism should be provided for the complete reconstruction of every action
taken on the database:
\begin{itemize}
  \item Who: Identify the person viewing/modifying the data
  \item What: Full listing of data viewed or modified
  \item When: Reliable datetime-stamp
  \item Where: Specific application used for access
  \item Why: Context information on data disclosure
\end{itemize}
We are particularly interested in who made and approved the transaction,
what they did, and what the result was.

\subsection*{Practical Auditing Requirements}
In reality, auditing should be robust, comprehensive, efficient, and
customizable. Database activity should be audited by statement, use of system
privilege, object, and the user. Table creation, database connections, and
successful or unsuccessful connections should be monitored for auditing.
Auditing should also be implemented efficiently, where statements are parsed
once for both execution and auditing, not separately. It should be implemented
within the server itself, but stored separately in case of failure. The audit
log should be able to filter objects of interest.
\par It is possible to use triggers to further customize auditing conditions.
This allows administrators to define specific audit policies to detect the
misues of legitimate data access. This can further help reconstruct audited
events to determine whether access rights have been violated and prevents users
from bypassing auditing. Categories:
\begin{itemize}
  \item Accesses, login, sources of usage
  \item Errors of any kind, which may allow you to determine the intent of the
  user
  \item Usage outside of normal hours
  \item Changes to sensitive data
  \item Changes to stored procedures and triggers
  \item Changes to privilieges, user/login definitions, and security attributes
  \item Creation, changes to, and usage of database links
  \item Changes to the definition of what to audit
\end{itemize}

\subsection*{Non-Repudiation}
Non-repudiation is the assurance that a party to a contract cannot deny
authenticity of their signature on a document. This holds principals
accountable for their actions and provides irrefutable evidence about their
events or actions. This is increasingly needed due to increased distributed
and mobile computing. Typical evidence for non-repudiation involves proof of
message creation and proof of the receipt of a message, which also needs proper
logging and auditing.

\subsection*{Baselining and DDL Audits}
A baseline usage pattern should be developed to monitor unusual changes from
the baseline. Changes to the schema as well as changes to the data should be
monitored. This can be done through system triggers or other mechanisms.

\subsection*{Auditing Architectures}
Auditing architectures should not create a false sense of security. They should
provide an independent audit trail. The audit system should be archived, secure,
and also itself be auditable. Ideally, it should be sustainably automatable. The
data stored from auditing can be massive, and this can present other logistical
issues since it needs to be stored to generate reports and alerts in addition
to being capable of archival and restoration. This usually implies the need for
data warehouse capable of handling dynamic changes as demanded by auditors.

\begin{center}
  You can find all my notes at \url{http://omgimanerd.tech/notes}. If you have
  any questions, comments, or concerns, please contact me at
  alvin@omgimanerd.tech
\end{center}

\end{document}
