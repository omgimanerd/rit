\documentclass{math}

\title{Principles of Computer Security}
\author{Alvin Lin}
\date{January 2018 - May 2018}

\begin{document}

\maketitle

\section*{Exam 2 Key}
\begin{enumerate}
  \item The main issue is with the \texttt{gets()} function, as there is no
  check and overflows the variable buf. Some ways to fix the issue: use
  \texttt{fgets()}, or read a fixed number of characters, not an unchecked read.
  \item The canary is generated using random numbers to make it harder for
  attackers to guess the canary value.
  \item No preemption can lead to deadlock. Here, the system is preventing
  ``no preemption'' to prevent deadlocks. Since the system will allow
  preemption of resources, a programmer's abilities to write secure code is
  impacted because preemption will occur, and will need expert handling.
  Exception handling will need to become the default.
  \item Exponential back-off means retrying when a deadlock is reported after
  waiting for \( 2n \) seconds before retrying until some value \( n \) such as
  10 is reached, and then aborting the program. If a DDOS attack is occurring
  via deadlocks, then this technique is ineffective and actually gets in the way
  by delaying mitigation.
  \item A pseudo-random number generator allows an attacker to make 100\%
  accurate guesses, thus allowing the attacker to generate receipt URLs with
  100\% success. Don't use pseudo-random number generator, or use a different
  technique to store receipts that is backed by authentication.
  \item This is a classic TOCTOU attack discussed in class. The file checked
  on lines 1-2 may not be the one used in line 8 because the race condition
  caused by the script could have changed the file. The check for file
  permission should be combined with the file open in an atomic operation.
  \item \texttt{grant select(name) on Student to `abc1234'} \\
  SQL does not provide row level security. To do this, we would need to create
  a view for that student: \\
  \texttt{create view AbcView as select * from Student where name=`abc1234'}
  \item As roles in the database cannot be used in the middleware, it means that
  the system must implement its own RBAC in the middleware. The use of RBAC in
  the database cannot be relied upon, and the system will need to maintain two
  sets of RBAC.
  \item
  \begin{tabular}{|m{3cm}|p{10cm}|}
    \hline
    Unobservability & User may use resource or service without others being
      aware of its use. \\
    \hline
    Anonymity & Use resource or service without disclosing their identity. \\
    \hline
    Unlinkability & Use multiple resources or services without others being
      able to link these uses together. \\
    \hline
    Pseudonymity & Use a resource or service without disclosing their
    identity, but can still be accountable for that use. \\
    \hline
  \end{tabular}
  \item Both have same or similar functionality. Data perturbation needs a copy
  of the entire dataset so space performance is poor. It has good runtime
  performance as data has already been perturbed so time performance is good.
  Output perturbation does not need a copy of the entire dataset so space
  performance is good. Output perturbation has poor runtime performance however,
  since data has to be perturbed on the fly.
  \item A user can enter ``123 or true --'' to make the WHERE clause always
  evaluate to true, which would cause the database to return every account.
  Users can inject arbitrary strings into the query and nullify the rest of the
  query using the SQL comment. The best way to defend from this would be to
  use input sanitization and parameterized replacement when generating the
  query string.
  \item SQL does not provide row level security.
\end{enumerate}

\begin{center}
  You can find all my notes at \url{http://omgimanerd.tech/notes}. If you have
  any questions, comments, or concerns, please contact me at
  alvin@omgimanerd.tech
\end{center}

\end{document}
