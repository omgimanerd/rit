\documentclass{math}

\usepackage{listings}
\lstset{basicstyle=\ttfamily\footnotesize,breaklines=true}

\title{Principles of Computer Security}
\author{Alvin Lin}
\date{January 2018 - May 2018}

\begin{document}

\maketitle

\section*{Secure Coding}
Code that is weak in quality is likely to introduce security flaws and
vulnerabilities that would otherwise not be an issue. Secure code is very much
related to writing high quality code. Secure code is usually a three-way
partnership between the code itself, the compiler, and the operating system it
will run on. The programmer can utilize all three to ensure the behavior of a
certain program.

\subsection*{Terminology}
\begin{itemize}
  \item Security Policy: the set of rules/practices that specify what security
  an organization provides.
  \item Security Mechanism: code that conforms to security policy.
  \item Assurance: proof that the security mechanism conforms to the security
  policy.
  \item Security Flaw: software defect posing a potential security risk.
  \item Vulnerability: conditions that allow attackers to violate security
  policy.
  \item Exploit: technique that takes advantage of security vulnerabilities to
  violate a security policy.
  \item Mitigation: methods, techniques, processes, tools, or runtime libraries
  that cna prevent or limit exploits against vulnerabilities.
\end{itemize}

\subsection*{Then vs Now}
In the past, systems were smaller and operated on less code. Computers operated
behind closed doors with limited network access. Computer users were mostly
experts, backed by defense mechanisms such as ``guns, gates, and guards''.
Today, everyone has a computer and they are becoming increasingly networked.
Often systems are very large, and many have legacy code. For convenience, code
is reused and sacrifices performance.

\subsection*{Trusted Computer Base}
A trusted computing base is the part of a system that must work correctly to
ensure proper function, for example, the OS kernel and hardware. We need to
rely on languages, compilers, and the OS to help enforce protection and security
mechanisms. Dijkstra, in addition to being known for his famous search
algorithm, is known for his Technische Hogeschool Eindhoven (THE) operating
system which conformed to this standard by creating a layered system in which
higher levels ONLY depended on the lower layers.

\subsection*{Process Memory Layout}
\begin{itemize}
  \item Text: Read-only segment with instructions for CPU execution. No
  uncontrolled instruction modification, changing this segment results in a
  segmentation fault.
  \item Data: Global initialized static variables.
  \item Block Started by Symbol (BSS): Global uninitialized static variables.
  This section is zeroed out by the operating system before program execution.
  \item Heap: Dynamic memory allocated at runtime.
  \item Stack: Memory for function variables allocated automatically on call
  and deallocated on return. Process execution control information.
\end{itemize}

\subsection*{Strings}
Strings are a major mechanism for information exchange between users and
software and between software as well. They are used in text fields, command
line arguments, environment variables, and other methods of information storage
and propagation. Even though they are so basic, they are not a built-in type
in C or C++. The standard C library supports strings of type \texttt{char} and
wide strings of type \texttt{wchar\_t}. Software vulnerabilities arise due to
weaknesses in string management, representation, and manipulation. They
are represented with 8 bit ASCII or 16 bit Unicode and must be null terminated.
Multiple string types (like \texttt{std::basic\_string}, \texttt{std::string},
\texttt{std:wstring}) cause issues with secure coding in C++. Common errors
arise from:
\begin{itemize}
  \item Improperly bounded string copies when copying an unbounded string to a
  fixed length \texttt{char} array.
  \item Copying and concatenating strings, since many library calls such as
  \texttt{strcopy()}, \texttt{strcat()}, and \texttt{sprintf()} perform
  unbounded copy operations.
  \item Off-by-one errors
  \item Null termination errors
  \item Incorrect string truncation
\end{itemize}

\subsubsection*{Exploits}
\begin{itemize}
  \item Unsanitized data entered from a user outside of the program control can
  cause security issues.
  \item Buffer overflows from data written outside memory boundaries allow for
  users to leak or modify program data.
  \item Stack smashing allows users to overwrite data in the execution stack and
  execute arbitrary code.
  \item Code injection allows an overwritten stack return address to point to
  a malicious function.
  \item Arc injection allows for the transfer of control from the existing code
  to some in-process memory.
\end{itemize}

\subsection*{Concurrency}
Concurrency is essential to modern computing. Multiple separate execution flows
running simultaneously allow for speed and optimization. Uncontrolled
concurrency can lead to non-determinism however.

\subsubsection*{Race Conditions}
An unanticipated execution ordering in concurrent code flow can lead to a
defect known as a race condition. Three conditions are necessary for a race
condition to occur:
\begin{enumerate}
  \item Concurrency: there must be at least two concurrent control flows
  \item Shared Object: a shared race object must be accessed by both concurrent
  flows
  \item State Change: at least one control flow must alter the state of the race
  object
\end{enumerate}
Attackers can exploit a race condition to manipulate data in unpredictable ways.
To prevent it, access to critical sections must be atomic and code accessing it
must execute alone. Synchronization primitives such as locks, counting
semaphores, binary semaphores, pipes, monitors, and condition variables help
implement mutual exclusion.

\subsubsection*{Deadlock}
Deadlock occurs whenever a set of tasks are blocked waiting for one another to
finish. This often leads to a denial of service vulnerability. Four necessary
conditions are necessary for a deadlock to occur:
\begin{enumerate}
  \item Circular wait cycle
  \item Mutual exclusion
  \item No preemption
  \item Hold and wait
\end{enumerate}
Deadlocks can be exploited by altering processor speeds, changes in task
scheduling algorithms, different memory constraints on execution, or any
asynchronous event capable of interrupting program execution. Attacks often
automate the creation of race conditions to cause load in executing a race
window.

\subsubsection*{Time of Check/Time of Use (TOCTOU)}
TOCTOU race conditions are caused by check and access conditions happening on
a shared object at separate times. An untrusted script running in parallel can
modify an object state after the check but before the access, influencing the
execution of the following code. Consider the following C code:
\begin{lstlisting}
#include <stdio.h>
#include <unistd.h>
int main(int argc, char* argv[]) {
  FILE* fd;
  if (access("/oldfile", W_OK) == 0) {
    printf("access granted.\n");
    fd = fopen("/oldfile", "wb+");
    /* write data to the file */
    fclose(fd);
  }
  return 0;
}
\end{lstlisting}
If a malicious entity removed the file \texttt{oldfile} and symlinked it to
another file, they could gain access to the data written. These attacks are
usually mitigated by combining the file check and open in an atomic operation.

\begin{center}
  You can find all my notes at \url{http://omgimanerd.tech/notes}. If you have
  any questions, comments, or concerns, please contact me at
  alvin@omgimanerd.tech
\end{center}

\end{document}
