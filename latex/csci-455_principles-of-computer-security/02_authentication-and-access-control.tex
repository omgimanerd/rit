\documentclass{math}

\title{Principles of Computer Security}
\author{Alvin Lin}
\date{January 2018 - May 2018}

\begin{document}

\maketitle

\section*{Authentication and Access Control}
Authentication Process:
\begin{itemize}
  \item Basis for access control and user accountability
  \item ``The process of verifying an identity claimed by or for a system
  entity'' (RFC 4949)
  \item Involves an indentification step (presenting an identifier to a
  security system) and a verification step (presenting or generating
  identification information to corroborate binding between the entity and
  identifier)
\end{itemize}

\subsection*{Means of Authentication User Identity}
\begin{tabular}{|p{5cm}|p{8cm}|}
  \hline
  Aspect & Example \\
  \hline
  Something an individual \textbf{knows} & Password, PIN, answers to prior
    questions \\
  \hline
  Something an individual \textbf{possesses} (token) & Smart card, electronic
    keycard, physical key \\
  \hline
  Something an individual \textbf{is} (static biometrics) & Fingerprint, retina,
    face \\
  \hline
  Something an individual \textbf{does} (dynamic biometrics) & Voice pattern,
    typing rhythm, handwriting \\
  \hline
\end{tabular}

\subsection*{Password Authentication}
Password authentication is a widely used line of defense against intruders where
the user provides a name/login and password. The system compares the password
with the one stored for that specified login. The user ID checks if the user is
authorized to access the system, determines the privileges, and is used in
discretionary access control. A widely used password security technique involves
using hashed passwords and a salt value.

\subsection*{Password Vulnerabilities}
\begin{itemize}
  \item Offline dictionary attack
  \item Specific account attack
  \item Popular password attack
  \item Password guessing against a single user
  \item Workstation hijacking
  \item Exploiting user mistakes
  \item Exploiting multiple password use
  \item Electronic monitoring
\end{itemize}

\subsection*{Password Cracking}
Dictionary attacks:
\begin{itemize}
  \item Keep a set of possible passwords and try each against password file
  \item Each password must be hashed using each salt value and then compared to
  stored hash values
\end{itemize}
Rainbow table attacks:
\begin{itemize}
  \item Pre-compute tables of hash values for all salts
  \item A mammoth table of hash values
  \item Countered by a sufficiently large salt value and a hash length
\end{itemize}
Password crackers exploit ``people choose guessable passwords'':
\begin{itemize}
  \item Shorter password lengths are also easier to crack
\end{itemize}
John the Ripper:
\begin{itemize}
  \item Open-source password cracker first developed in 1996
  \item Uses a combination of brute-force and dictionary techniques
\end{itemize}

\subsection*{Modern Approaches}
Complex password policies force users to pick stronger passwords. But
password-cracking has also improved. We have increased processing capacity
for password cracking, sophisticated algorithms to generate likely passwords,
and we can study examples and structures of actual passwords. GPUs now allow
password-cracking programs to work thousands to times faster than just a decade
ago.

\subsection*{Password File Access Control}
We can block offline guessing attacks by denying access to encrypted passwords.
They should be made available only to privileged users and shadow password file.
Vulnerabilities:
\begin{itemize}
  \item Weakness in OS that allows access to the file
  \item Accident with permissions making it readable
  \item Users with same password on other systems
  \item Access from backup media
  \item Sniff passwords in network traffic
\end{itemize}

\subsection*{Password Selection Strategies}
Users can be told or compelled to choose strong passwords. Users have trouble
remembering computer generated passwords. Systems can also periodically run
their own password crackers to find guessable passwords. Complex password
policies allow users to select their own passwords, but the system checks if
the password is acceptable and will reject it otherwise.

\begin{center}
  You can find all my notes at \url{http://omgimanerd.tech/notes}. If you have
  any questions, comments, or concerns, please contact me at
  alvin@omgimanerd.tech
\end{center}

\end{document}
