\documentclass{math}

\usepackage{tikz}

\title{University Physics 2}
\author{Alvin Lin}
\date{January 2018 - May 2018}

\begin{document}

\maketitle

\section*{Introduction to Electric Charge}
Initial Ideas:
\begin{itemize}
  \item Charges are positive, negative, or neutral, and measured in units of
  Coulombs. Charge is quantized in units of electrons, which has a charge of
  \( e = -1.6\times10^{-19} \) coulombs.
  \item They are related and analogous to electrons, protons, and neutrons in
  atoms.
  \item The movement of charges is the driving forces behind electricity and
  magnetism.
  \item Like charges repel and dislike charges attract.
  \item The strength of the charge is distance dependent.
  \item Conductors allows for the movement of charge through it, while
  insulators do not.
  \item Ground is the term for a large neutral conductor which excess charge
  can go into.
\end{itemize}

\subsection*{Induced Charge}
Induced charging creates unbalanced charges using only interactions, not
physical touching.
\begin{center}
  \begin{tikzpicture}
    \draw (-4,1) -- (0,1) -- (0,0) -- (-4,0) -- cycle;
    \node[blue] at (-1,0.5) {+};
    \node[blue] at (-0.5,0.5) {+};
    \draw (2,0.5) circle (1cm);
    \node[blue] at (2.6,1) {+};
    \node[blue] at (2.6,0.5) {+};
    \node[blue] at (2.6,0) {+};
    \node[red] at (1.4,1) {-};
    \node[red] at (1.4,0.5) {-};
    \node[red] at (1.4,0) {-};
  \end{tikzpicture}
\end{center}
\begin{center}
  \begin{tikzpicture}
    \draw (-4,1) -- (0,1) -- (0,0) -- (-4,0) -- cycle;
    \node[blue] at (-1,0.5) {+};
    \node[blue] at (-0.5,0.5) {+};
    \draw (2,0.5) circle (1cm);
    \draw (3,0.5) -- (5,0.5) node[right] {ground};
    \node[red] at (2,1) {-};
    \node[red] at (2,0.5) {-};
    \node[red] at (2,0) {-};
  \end{tikzpicture}
\end{center}


\begin{center}
  You can find all my notes at \url{http://omgimanerd.tech/notes}. If you have
  any questions, comments, or concerns, please contact me at
  alvin@omgimanerd.tech
\end{center}

\end{document}
