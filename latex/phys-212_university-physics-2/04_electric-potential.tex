\documentclass{math}

\usepackage{tikz}

\title{University Physics 2}
\author{Alvin Lin}
\date{January 2018 - May 2018}

\begin{document}

\maketitle

\section*{Electric Potential}
Conservation of Energy:
\[ \Delta K+\Delta U = W_{ext} \]
The change in potential energy plus the change in kinetic energy is equal to
the external work by non-conservative forces. If there is no outside work:
\begin{align*}
  \Delta K+\Delta U &= 0 \\
  E_i &= E_f \\
  K_i+U_i &= K_f+U_f
\end{align*}
\( K \) is usually \( \frac{1}{2}mv^2 \) while \( U \) depends on the specifics
of the problem. Also recall the formula for work:
\[ W = \vec{F}\cdot\Delta\vec{r} \]
With many different forces:
\[ W = \sum\vec{F}\cdot\Delta\vec{r_i} \]
\( \vec{F} \) will generally not be constant, which requires us to integrate:
\[ W = \int_{i}^{f}\vec{F}\cdot\diff{\vec{r}} \]
If a potential energy is doing the work:
\[ W = -\Delta U_{internal} \]
For electrial fields, we will start with:
\[ W = \int\vec{F}\cdot\diff{r} \]
and plug in forces from electric fields.
\begin{align*}
  W &= \int\vec{F}\cdot\diff{\vec{r}} \\
  &= \int(q\vec{E})\cdot\diff{\vec{r}} \\
  &= qE\Delta x \\
  \Delta U &= -W = -qE\Delta x \\
  \text{Electric potential} &= \Delta V = \frac{\Delta U}{q} = -E\Delta x
\end{align*}
Electric potential is measured in volts, where one volt is equal to one joule
per Coulomb.

\begin{center}
  You can find all my notes at \url{http://omgimanerd.tech/notes}. If you have
  any questions, comments, or concerns, please contact me at
  alvin@omgimanerd.tech
\end{center}

\end{document}
