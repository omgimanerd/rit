\documentclass{math}

\usepackage{circuitikz}
\usepackage{tikz}

\title{University Physics 2}
\author{Alvin Lin}
\date{January 2018 - May 2018}

\begin{document}

\maketitle

\section*{Capacitors}
Capacitors are represented usually as two parallel plates.
\begin{center}
  \begin{tikzpicture}
    \draw (0,0) -- (0,4);
    \draw[->,green!40!black,very thick] (0.5,1) -- (2.5,1);
    \draw[->,green!40!black,very thick] (0.5,2) -- (2.5,2)
      node[pos=0.5,above] {\( \vec{E} \)};
    \draw[->,green!40!black,very thick] (0.5,3) -- (2.5,3);
    \draw (3,4) -- (3,0) node[pos=1,below] {\( A \)};
    \draw[|<->|] (0,4.5) -- (3,4.5) node[pos=0.5,above] {\( d \)};
    \foreach \y in {0,0.5,1,1.5,2,2.5,3,3.5,4} {
      \node[red] at (-0.5,\y) {+};
      \node[blue] at (3.5,\y) {-};
    }
    \draw[green!40!black] (-1.5,2) node {\( \vec{E} = 0 \)};
    \draw[green!40!black] (4.5,2) node {\( \vec{E} = 0 \)};
  \end{tikzpicture}
\end{center}
We already know the following from electric fields and potential:
\[ |\overrightarrow{E_{capacitor}}| = \frac{\sigma}{\epsilon_{\circ}} \]
\[ \sigma = \frac{Q}{A} \]
\[ \Delta V = -\int_{0}^{d}\vec{E}\cdot\diff{\vec{x}} =
  -\int_{0}^{d}\frac{\sigma}{\epsilon_{\circ}}\diff{x} =
  -\frac{\sigma d}{\epsilon_{\circ}} =
  -\frac{Qd}{A\epsilon_{\circ}} \]
Capacitance of any object is defined as:
\[ C \equiv \bigg|\frac{Q}{\Delta V}\bigg| \]
Capacitance is measured in units of Farads, with one Farad equivalent to one
Coulomb over 1 volt. For parallel plates:
\[ C_{parallel~plates} = \frac{Q}{\Delta V} =
  \frac{Q}{\frac{Qd}{A\epsilon_{\circ}}} =
  \frac{A\epsilon_{\circ}}{d} \]
The assumes you have a vacuum between plates, generally this is not the case as
there is some insulating material between the plates. Different materials
determine the properties of the capacitor. Capacitance for this is:
\[ C = C_{\circ}\kappa = \kappa\epsilon_{\circ}\frac{A}{d} =
  \epsilon\frac{A}{d} \]
where \( \kappa \) is the dielectric constant. \( \kappa\epsilon_{\circ} \) is
the permitivity of whatever material is used, and both constants are properties
of the material. \( \kappa \) is always greater than 1, with \( \kappa = 1 \)
being the dielectric constant for a vacuum.

\subsection*{Capacitors in Series}
\begin{center}
  \begin{circuitikz}
    \draw (0,0) to[battery, label=\( \Delta V_{total} \)] (0,2) -- (2,2)
      to[capacitor, label=\( C_1 \)] (2,1)
      to[capacitor, label=\( C_2 \)] (2,0) -- (0,0);
  \end{circuitikz}
\end{center}
We can simplify this circuit to:
\begin{center}
  \begin{circuitikz}
    \draw (0,0) to[battery, label=\( \Delta V_{total} \)] (0,2) -- (2,2)
      to[capacitor, label=\( C_{eq} \)] (2,0) -- (0,0);
  \end{circuitikz}
\end{center}
Both \( C_1 \) and \( C_2 \) have the same charge \( Q \) due to being in series
with the battery.
\[ Q_1 = Q_2 = Q_{total} \quad C_1 = \frac{Q_{total}}{\Delta V_1} \quad
  C_2 = \frac{Q_{total}}{\Delta V_2} \]
By conservation of energy, we know that:
\[ \Delta V_{total} = \Delta V_1+\Delta V_2 \]
Thus, for capacitors in series, the equivalent capacitance can be calculated as:
\begin{align*}
  C_{eq} &= \frac{Q_{total}}{\Delta V_{total}} \\
  Q_{total} &= C_{eq}(\Delta V_1+\Delta V_2) \\
  Q_{total} &= C_{eq}\bigg(\frac{Q_{total}}{C_1}+\frac{Q_{total}}{C_2}\bigg) \\
  \frac{1}{C_{eq}} &= \frac{1}{C_1}+\frac{1}{C_2} \\
  C_{eq} &= \frac{C_1C_2}{C_1+C_2}
\end{align*}

\subsection*{Capacitors in Parallel}
\begin{center}
  \begin{circuitikz}
    \draw (0,0) to[battery, label=\( \Delta V_{total} \)] (0,2) -- (2,2)
      to[capacitor, label=\( C_1 \)] (2,0) -- (0,0);
    \draw (2,2) -- (4,2)
      to[capacitor, label=\( C_2 \)] (4,0) -- (2,0);
  \end{circuitikz}
\end{center}
Again, we can simplify this circuit to:
\begin{center}
  \begin{circuitikz}
    \draw (0,0) to[battery, label=\( \Delta V_{total} \)] (0,2) -- (2,2)
      to[capacitor, label=\( C_{eq} \)] (2,0) -- (0,0);
  \end{circuitikz}
\end{center}
Because the capacitors are in parallel:
\[ C_1 = \frac{Q_1}{\Delta V_1} \quad C_2 = \frac{Q_2}{\Delta V_2} \]
By conservation of charge and conservation of energy:
\[ Q_{total} = Q_1+Q_2 \quad \Delta V_1 = \Delta V_{total} \quad
  \Delta V_2 = \Delta V_{total} \]
Thus, for capacitors in parallel, the equivalent resistance can be calculated
as:
\begin{align*}
  C_{eq} &= \frac{Q_{total}}{\Delta V} = \frac{Q_1+Q_2}{\Delta V} \\
  &= \frac{Q_1}{\Delta V}+\frac{Q_2}{\Delta V} \\
  &= C_1+C_2
\end{align*}

\subsection*{Energy in a Capacitor}
To find the energy in a capacitor, find the work needed to charge it. Recall:
\[ C = \frac{Q}{V} \quad V = \frac{Q}{C} \]
Suppose it is partially charged:
\[ V = \frac{q}{C} \quad \diff{V} = \frac{\diff{q}}{C} \]
where \( V \) is the intermediate voltage and \( q \) is the intermediate
charge.
\begin{align*}
  W &= \int\vec{F}\cdot\diff{\vec{r}} \quad
    \diff{W} = \vec{F}\cdot\diff{\vec{r}} \\
  V &= -\int\vec{E}\cdot\diff{\vec{r}} \quad
    \diff{V} = |\vec{E}|\diff{r} \\
  \diff{W} &= F\diff{r} = (q|\vec{E}|)\diff{r} = q\diff{V} \\
  W &= \int q\diff{V} = \int_{0}^{Q}q(\frac{\diff{q}}{c}) = \frac{Q^2}{2C}
\end{align*}
As a relation to the equations for voltage:
\[ U = \frac{Q^2}{2C} = \frac{1}{2}CV^2 = \frac{1}{2}QV \]
If we want energy density as energy over volume:
\[ u = \frac{U}{volume} =
  \frac{\frac{1}{2}CV^2}{Ad} =
  \frac{\frac{1}{2}(\frac{\epsilon_{\circ}A}{d})(Ed)^2}{Ad} =
  \frac{1}{2}\epsilon_{\circ}E^2 \]

\begin{center}
  You can find all my notes at \url{http://omgimanerd.tech/notes}. If you have
  any questions, comments, or concerns, please contact me at
  alvin@omgimanerd.tech
\end{center}

\end{document}
