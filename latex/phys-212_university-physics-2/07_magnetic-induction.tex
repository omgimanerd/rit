\documentclass{math}

\usepackage{tikz}

\title{University Physics 2}
\author{Alvin Lin}
\date{January 2018 - May 2018}

\begin{document}

\maketitle

\section*{Magnetic Induction}
Lenz's Law states that if there is a change in magnetic flux through a
conductor, there will be an induced magnetic field to oppose that change. This
is done by the creation of an induced current through the conductor. The
direction is given by the right hand rule.

\subsection*{Faraday's Law}
Recall the formula for magnetic flux:
\[ \Phi_B = \oint\vec{B}\cdot\diff{\vec{A}} = 0 \]
Magnetic flux through a closed loop is zero because there are no magnetic
monopoles. Faraday's Law is the following:
\[ \epsilon = -\ddiff{\Phi_B}{t} \]
We also know the following about EMF:
\[ \epsilon = \oint\vec{E}\cdot\diff{\vec{L}} \]
Therefore:
\[ \oint\vec{E}\cdot\diff{L} = -\ddiff{\Phi_B}{t} \]
Note that this implies a changing magnetic fields creates an electric field.
If we substitute the formulas for magnetic flux:
\[ \epsilon = -\ddiff{}{t}\bigg(\oint\vec{B}\cdot\diff{\vec{A}}\bigg) \]
We can get EMF three ways:
\[ \epsilon = -\ddiff{}{t}\bigg(BA\cos\theta\bigg) \]
If we have \( N \) wires:
\[ \epsilon = -N\ddiff{\Phi_B}{t} \]

\begin{center}
  You can find all my notes at \url{http://omgimanerd.tech/notes}. If you have
  any questions, comments, or concerns, please contact me at
  alvin@omgimanerd.tech
\end{center}

\end{document}
