\documentclass{math}

\usepackage{circuitikz}

\title{University Physics 2}
\author{Alvin Lin}
\date{January 2018 - May 2018}

\begin{document}

\maketitle

\section*{Magnetic Induction}
Lenz's Law states that if there is a change in magnetic flux through a
conductor, there will be an induced magnetic field to oppose that change. This
is done by the creation of an induced current through the conductor. The
direction is given by the right hand rule.

\subsection*{Faraday's Law}
Recall the formula for magnetic flux:
\[ \Phi_B = \oint\vec{B}\cdot\diff{\vec{A}} = 0 \]
Magnetic flux through a closed loop is zero because there are no magnetic
monopoles. Faraday's Law is the following:
\[ \epsilon = -\ddiff{\Phi_B}{t} \]
We also know the following about EMF:
\[ \epsilon = \oint\vec{E}\cdot\diff{\vec{L}} \]
Therefore:
\[ \oint\vec{E}\cdot\diff{L} = -\ddiff{\Phi_B}{t} \]
Note that this implies a changing magnetic fields creates an electric field.
If we substitute the formulas for magnetic flux:
\[ \epsilon = -\ddiff{}{t}\bigg(\oint\vec{B}\cdot\diff{\vec{A}}\bigg) \]
We can get EMF three ways:
\[ \epsilon = -\ddiff{}{t}\bigg(BA\cos\theta\bigg) \]
If we have \( N \) wires:
\begin{align*}
  \epsilon &= -N\ddiff{\Phi_B}{t} \\
  &= -N\ddiff{}{t}(BA\cos\theta)
\end{align*}

\subsection*{Transformers}
Transformers use Faraday's Law with coils of wire wrapped around a magnetic
core to step up or step down voltage. The voltage in and out of a transformer
can be described by:
\[ \frac{V_1}{V_2} = \frac{N_1}{N_2} \]
If \( N_2 < N_1 \), the transformer is a step down transformer. If
\( N_2 > N_1 \), then the transformer is a step up transformer. By conservation
of energy:
\begin{align*}
  P_1 &= P_2 \\
  I_1V_1 &= I_2V_2
\end{align*}
As you step up the voltage using a transformer, your current proportionally
goes down.

\subsection*{Inductance}
Consider two concentric solenoids:
\begin{align*}
  \epsilon_2 &= -N_2\ddiff{\phi_B}{t} \\
  B_1 &= n\mu_{\circ}I \\
  \phi_B &= B_1\cdot A_1 \\
  &= n\mu_{\circ}IA \\
  &= M_{12}i_1(t)
\end{align*}
where \( M_{12} \) is the mutual inductance between solenoids 1 and 2 and
\( i_1(t) \) is a current that varies over time.
\begin{align*}
  N_2\ddiff{\phi_B}{t} &= M_{21}\ddiff{i}{t} \\
  \epsilon_2 &= -M_{21}\ddiff{i}{t} \\
  M_{21} &= \frac{N_2\phi_B}{i_1}
\end{align*}

\subsubsection*{Self-Inductance}
An inductor is a circuit element that opposes change in current because of
Faraday's Law.
\begin{align*}
  L &= \frac{N\phi_B}{i} \\
  \epsilon &= -L\ddiff{i}{t}
\end{align*}
The unit of \( L \) is a Henry, and the voltage drop across an inductor is
measured as:
\begin{align*}
  \Delta V_L &= L\ddiff{i}{t} \\
  P &= IV = iL\ddiff{i}{t}
\end{align*}
Power is energy per time:
\begin{align*}
  P &= \ddiff{U}{t} \\
  \diff{U} &= P\diff{t} \\
  U_L &= \frac{1}{2}Li^2 \\
\end{align*}

\subsection*{RL Circuits}
\begin{center}
  \begin{circuitikz}
    \draw (0,0) to[battery, label=voltage source (volts)] (0,4)
      to[resistor, label=resistance in \( \Omega \) (ohms)] (4,4) -- (4,0)
      to[inductor, label=inductor \( i(t) \) (henrys)] (0,0);
  \end{circuitikz}
\end{center}
We can use Kirchoff's Loop Rule to solve this circuit.
\begin{align*}
  \epsilon-\Delta V_R-\Delta V_L &= 0 \\
  \epsilon-iR-L\ddiff{i}{t} &= 0 \\
  \ddiff{i}{t} &= \frac{\epsilon}{L}-\frac{R}{L}i \\
  i &= \frac{\epsilon}{R}(1-\e^{-\frac{R}{L}t}) \\
  &= \frac{\epsilon}{R}(1-\e^{-\frac{t}{\tau}}) \\
  \tau &= \frac{L}{R}
\end{align*}

\begin{center}
  You can find all my notes at \url{http://omgimanerd.tech/notes}. If you have
  any questions, comments, or concerns, please contact me at
  alvin@omgimanerd.tech
\end{center}

\end{document}
