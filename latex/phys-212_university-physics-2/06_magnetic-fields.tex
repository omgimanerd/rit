\documentclass{math}

\usepackage{tikz}

\title{University Physics 2}
\author{Alvin Lin}
\date{January 2018 - May 2018}

\begin{document}

\maketitle

\section*{Magnetic Fields}
The magnetic force on a particle is given as:
\[ \vec{F_B} = q\vec{V}\times\vec{B} \]
Note that the force requires the charge to be moving and is perpendicular to
both the velocity and the field since it is a cross product. Recall that
\( \vec{F_E} = q\vec{E} \), so the following relation is true.
\[ \vec{F} = q(\vec{E}+(\vec{V}\times\vec{E})) \]

\subsection*{Cross Products}
Recall the following properties of cross products:
\begin{enumerate}
  \item Magnitude is a function of the vector magnitudes and the sine of the
  angle between them.
  \[ |\vec{A}\times\vec{B}| = |\vec{A}||\vec{B}|\sin\theta \]
  \item Direction follows the right-hand rule. Put your fingers in the direction
  of A, curl your fingers towards B, and your thumb will point towards C.
  \item Using unit vectors:
  \begin{align*}
    \i\times\j &= \k \quad \i\times\k = -\j \\
    \j\times\k &= \i \quad \k\times\j = -\i \\
    \k\times\i &= \j \quad \j\times\i = -\k
  \end{align*}
\end{enumerate}

\subsection*{Lorentz Force Law}
If we're dealing with magnetic fields only:
\begin{align*}
  \vec{F_B} &= q(\vec{V}\times\vec{B}) \\
  |\vec{F_B}| &= q|\vec{V}||\vec{B}|\sin\theta \\
  If: \theta &= 90 \\
  |\vec{F_B}| &= q|\vec{V}||\vec{B}|
\end{align*}
Moving charge is current, recall that:
\[ I = nq|\vec{V}|A \]
where \( n \) is charge density. Thus the force on all charges is equal to:
\begin{align*}
  |\vec{F}| &= N(q|\vec{V}||\vec{B}|) \\
  &= nALq|\vec{V}||\vec{B}| \\
  &= (nq|\vec{V}|A)L|\vec{B}| \\
  &= IL|\vec{B}| \\
  \vec{F_B} &= I\vec{L}\times\vec{B}
\end{align*}
where \( N \) is the number of charges and \( \vec{L} \) is a vector with the
length of the wire, pointing in the direction of current. This is only true if
the magnetic field is constant over the length of the wire. Otherwise,
integration is necessary in order to calculate \( \vec{F_B} \).

\subsection*{Electric Dipoles}
TODO

\subsection*{Magnetic Dipoles}
TODO

\subsection*{Magnetic Field Creation}
Magnetic fields all arise from current loops. Atoms are current loops as well.
In physical materials, these current loops are evenly distributed and cancel
out. For magnets, these current loops are aligned, creating a stronger magnetic
field. An atom is the simplest current loop and the simply way to generate an
magnetic field. Electric monopole can exists which propagate an electric field
outwards in all direction. No analogous magnetic monopole exists. Magnetic field
lines always close on themselves.
\[ \vec{B} = \frac{\mu_{\circ}}{4\pi}\frac{q\vec{V}\times\vec{r}}{r^2} \]

\subsection*{Magnetic Field in a Wire}
\begin{center}
  \begin{tikzpicture}
    \draw (-2,0) -- (2,0);
    \draw (-2,1) -- (2,1);
    \draw (2,0.5) circle (0.5cm);
    \draw[|<->|] (0,1.5) -- (0.5,1.5) node[above,pos=0.5] {\( \diff{L} \)};
    \draw[->] (2,-1) -- (0.25,0.5) node[below,pos=0] {\( \diff{q} = \)};
  \end{tikzpicture}
\end{center}
\[ \vec{V} = \frac{\mu_{\circ}}{4\pi}
  \int I\frac{\diff{\vec{l}}\times\vec{r}}{r^2} \]

\begin{center}
  You can find all my notes at \url{http://omgimanerd.tech/notes}. If you have
  any questions, comments, or concerns, please contact me at
  alvin@omgimanerd.tech
\end{center}

\end{document}
