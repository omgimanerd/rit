\documentclass{math}

\usepackage{tikz}

\title{University Physics 2}
\author{Alvin Lin}
\date{January 2018 - May 2018}

\begin{document}

\maketitle

\section*{Magnetic Fields}
The magnetic force on a particle is given as:
\[ \vec{F_B} = q\vec{V}\times\vec{B} \]
Note that the force requires the charge to be moving and is perpendicular to
both the velocity and the field since it is a cross product. Recall that
\( \vec{F_E} = q\vec{E} \), so the following relation is true.
\[ \vec{F} = q(\vec{E}+(\vec{V}\times\vec{E})) \]

\subsection*{Cross Products}
Recall the following properties of cross products:
\begin{enumerate}
  \item Magnitude is a function of the vector magnitudes and the sine of the
  angle between them.
  \[ |\vec{A}\times\vec{B}| = |\vec{A}||\vec{B}|\sin\theta \]
  \item Direction follows the right-hand rule. Put your fingers in the direction
  of A, curl your fingers towards B, and your thumb will point towards C.
  \item Using unit vectors:
  \begin{align*}
    \i\times\j &= \k \quad \i\times\k = -\j \\
    \j\times\k &= \i \quad \k\times\j = -\i \\
    \k\times\i &= \j \quad \j\times\i = -\k
  \end{align*}
\end{enumerate}

\begin{center}
  You can find all my notes at \url{http://omgimanerd.tech/notes}. If you have
  any questions, comments, or concerns, please contact me at
  alvin@omgimanerd.tech
\end{center}

\end{document}
