\documentclass[letterpaper, 12pt]{article}

\usepackage{hyperref}

\title{ISTE 110: Ethics in Computing}
\author{Alvin Lin}
\date{February 1st, 2017}

\begin{document}

\maketitle

\subsection*{Core Values in Ethics}
``The needs of the many outweigh the needs of the few'' -Spock, Star Trek \\
It is the greatest good to the greatest number of people which is the measure of right and wrong,

\subsection*{Definition of Ethics}
\begin{itemize}
  \item A reational (justifiable) systematic analysis of conduct that can
    benefit or harm people
  \item Based in reason, people have to justify their opinions
  \item Helps us understand why people feel the way they do, a set or system of
    principles for a social system
  \item Focuses on the voluntary moral choices people make because they have
    decided on a course of action
\end{itemize}

\subsection*{Definition of Morals}
\begin{itemize}
  \item Beliefs based on how people conduct themselves
  \item How things should work, according to one's beliefs and principles
  \item Individual/internalized
  \item Can change if a person's belief changes
\end{itemize}
Plato documented Socrates' thoughts, and the study of ethics goes back at least 2400 years.
``[A person] has only one thing to consider in performing an action: whether
s/he is acting right or wrongly, like a good [person] or a bad one.''
\begin{itemize}
  \item Happy = doing ``right''
  \item Everyong acts in self-interest
  \item People are not deliberately immoral
  \item If you know what is good, you will do it
\end{itemize}

\subsection*{Why Ethical Theory?}
\begin{itemize}
  \item A theory lets us examin moral problems, reach conclusions, and defend
    those conclusions.
  \item You can make a logical and persuasive argument through the use of
    persuasive theory.
\end{itemize}
Principles:
\begin{itemize}
  \item Objectivist (Ayn Rand): The purpose of morality is to define values
    that support a person's life.
  \item Objectivist (Kant): We act out of respect for moral values.
  \item Utilitarianism: ``Greatest Happiness Principle'', principle of utility,
    action is right/wrong according to the degree that it increases or
    decreases total happiness of all involved.
  \item Social Contract: There are a set of rules that comprise morality, and
    these rules govern how people should treat each other.
\end{itemize}

\subsection*{Kohlberg's Theory of Moral Development}
\begin{enumerate}
  \item obedience (avoid punishment)
  \item self-interest (gain reward)
  \item conformity (dis/approval)
  \item law and order (duty and guilt)
  \item human rights (agreed-upon rights)
  \item universal human ethics (personal moral standards)
\end{enumerate}

\subsection*{Gilligan's Stages of the Ethic of Care}
\begin{itemize}
  \item Pre-conventional: individual survival
  \item Conventional: self-sacrifice is goodness
  \item Post-conventional: do not hurt others or yourself
\end{itemize}

\subsection*{Michael J. Quinn}
He suggested that four ethical theories are practical ones for analyzing
scenarios in the field of informational technology.
\begin{itemize}
  \item Kant
  \item Act Utilitarianism
  \item Rule Utilitarianism
  \item Social Contract
\end{itemize}

\subsection*{Reminders and Homework}
In MyCourses, navigate to Content and watch the ``Sneakers'' movie. \\
Assignment: ``Sneakers'' Movie Reaction Paper, due in Dropbox by Sunday,
February 12th, at 11:59pm.
You must submit a Word document, please do not submit a PDF.

\begin{center}
  You can find all my notes at \url{http://omgimanerd.tech/notes}. If you have
  any questions, comments, or concerns, please contact me at
  alvin@omgimanerd.tech
\end{center}

\end{document}
