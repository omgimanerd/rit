\documentclass[letterpaper, 12pt]{article}

\usepackage{hyperref}

\title{ISTE 110: Ethics in Computing}
\author{Alvin Lin}
\date{April 17th, 2017}

\begin{document}

\maketitle

\subsection*{China and the Great Firewall}
Skype, when doing business in China, had to follow the guidelines on
appropriate content. They had to censor content according to the Chinese
government. The Chinese government employs more than 2 million people to
prowl the Internet for dissent and forbidden content. More than 100 million
tweets a day are monitored in China, in contrast to the CIA, which monitors
around 5 million tweets. Any company seeking to do business in China needs to
agree to their terms of censorship. But what determines what is obscene?

\subsection*{Gateways}
All Internet traffic through China has to come through government firewalls.
It allows for inappropriate content to be censored. Despite the extensive
and sophisticated efforts for censorship, people have been getting around
the censorship efforts. China has roughly 731 million internet users out of
1,357,000,000 people.

\subsection*{Perspectives}
\textbf{Pro-censorship:} \\
A people devoid of distractions are more distractions. The government can ensure
national security by monitoring the communications the people. Censorship
allows them to maintain control by stifling dissent and encouraging national
pride. \\
\textbf{Anti-censorship:} \\
Democratizing internet usage allows for the free exchange of ideas and
innovation. People don't have to be afraid to speak their thoughts, there is
more diversity of speech and thought, which allows for growth and improvement.
The individuals of the society have the freedom to pursue their own interests.

\subsection*{Egypt}
Internet usage has rapidly expanded across Egypt and played a major role in
the 2011 revolution. In 2010, 31.4\% of the population were Internet users,
with the number currently at 49.6\%.

\subsection*{Kill Switch}
In 2010, a bill was proposed to give the president a kill switch, shutting off
the Internet in the event of a national security emergency. The president
would have emergency power over the Internet. Assuming this were even possible,
it certainly not be a good idea due to the amount of infrastructure dependent on
the Internet. Does this give too much power to the president and is it an
infringement on the freedom of speech?

\subsection*{What are our options?}
What are the pros and cons to a kill switch? Should we allow control of the
internet to control news content? What are the ethical obligations of a
government?

\subsection*{Reminders and Homework}
The Article Review is due on Sunday, April 23rd, 2017.
Work on your final project and paper.

\begin{center}
  You can find all my notes at \url{http://omgimanerd.tech/notes}. If you have
  any questions, comments, or concerns, please contact me at
  alvin@omgimanerd.tech
\end{center}

\end{document}
