\documentclass[letterpaper, 12pt]{article}

\usepackage{hyperref}

\title{ISTE 110: Ethics in Computing}
\author{Alvin Lin}
\date{February 6th, 2017}

\begin{document}

\maketitle

\subsection*{The Digital Divide}
Does the meeting of cultures cause us to re-examine our own culture in light of
another? \\
Is there a loss of traditional values and practices as many yearn to become
American? \\
Have we endangered humanity with the internet? Or did that happen with TV? Or
video games? The economy? \\ \\
\href{http://www.theglobeandmail.com/report-on-business/industry-news/marketing/molsons-newest-red-beer-fridge-touts-canadas-multicultural-side/article25116524/}{View video}

\subsection*{What is the Digital Divide?}
``The perceived gap between those who have access to the latest information
technologies and those who do not.'' \\
\textit{The Digital Divide: Facing a Crisis or Creating a Myth} \\
\textit{Benjamin Compaine, 2000}

\subsection*{Is the Divide Singular}
In 2000, roughly 400,000,000 people in the entire world had online access. In
2016, roughly 3,000,000,000 people had online access. Technology is slowly
expanding across the world and more and more peple are gaining access. \\
Al Gore claimed in 1990 that he invented the Internet. Since then, he has
promoted a global information infrastructure that pushed for universal access to
cybertechnology. \\

\subsection*{2010 UN Special Report}
``Given that the Internet has become an indispensable tool for realizing a
range of human rights, combating inequality, and accelerating development and
human progress, ensuring universal access to the Internet should be a priority
for all States. Each State should thus develop a concrete and effective policy,
in consultation with individuals from all sections of society including the
private sector and relevant government ministries to make the Internet widely
available, accessible, and affordable to all segments of the population.''

\subsection*{Problems}
Literacy, logistics, and language are barriers to Internet access, especially
for developing nations. In countries like Africa, even basic features of the
Internet are difficult to access because of the prohibitive fees and slow
speed.

\subsection*{The US Digital Divide}
\begin{itemize}
  \item 85\% of Americans use the internet
  \item 94\% White Americans
  \item 78\% African Americans
  \item 81\% Hispanic Americans
  \item 97\% Asian-American
  \item 19\% of students who have not graduated high school use the internet
  \item 66\% of high school graduates use the internet
  \item 95\% of college graduates use the internet
\end{itemize}

\subsection*{Key Issue}
Do we need a universal service policy?

\subsection*{The Argument for Universal Service}
Why is a policy that ensures universal service, as opposed to universal access,
necessary?

\subsection*{The Counterargument}
Why is a policy that ensures universal service, as opposed to universal acess,
unncessary?

\subsection*{Is this an Ethical Issue?}
There are those of us who have, and those of us who have not. There are
divisions between pockets of society. But this is a question of ``I want''
rather than a question of ``I need''.

\subsection*{Disadvantages Without Internet}
Access to knowledge is significantly limited.
Do we have a moral obligation to provide internet access to those who are
disadvantaged?

\subsection*{Disabilities}
``The power of the Web is in its universality. Access by everyone regardless of
disability is an essential aspect'' \\
\textit{Tim Berners Lee} \\
\textit{Director, World Wide Wee Consortium (W3C)}

\subsection*{Demographics}
African Americans and Whites differ in their usage pattern and demographic
characteristics. \\
Has the internet eliminated or helped to reduce racism? Is it blind to color? \\
Does information technology make the reemergence of prejudicial messages and
attitudes swifter and more likely? \\
Does the internet's wide range of distribution make for more followers?

\subsection*{Reminders and Homework}
Watch the ``Sneakers'' movie on MyCourses.
Assignemnt: ``Sneakers'' Movie Reaction Paper, due in Dropbox by Sunday,
February 12th, at 11:59pm. You must submit a Word document, please do not
submit a PDF.
Bring a printout of your topics of discussion.

\begin{center}
  You can find all my notes at \url{http://omgimanerd.tech/notes}. If you have
  any questions, comments, or concerns, please contact me at
  alvin@omgimanerd.tech
\end{center}

\end{document}
