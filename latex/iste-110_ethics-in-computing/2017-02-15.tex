\documentclass[letterpaper, 12pt]{article}

\usepackage{hyperref}
\usepackage{ulem}

\title{ISTE 110: Ethics in Computing}
\author{Alvin Lin}
\date{February 15th, 2017}

\begin{document}

\maketitle

\section*{Global Warming Presentations}
Continue global warming practice presentations. Tips:
\begin{itemize}
  \item Watch your time
  \item Practice your presentation so that there is no one dominant speaker
  \item Hone in on the relevancy of the words that you use
  \item Focus on the summary so that the audience has a takeaway.
\end{itemize}

\section*{The Brandon Mayfield Case}
In 2004, a attorney named Brandon Mayfield was arrested by the FBI in connection
with the 2004 Madrid train bombing due to a fingerprint match. He lived in
Portland, Oregon and his passport showed that he had never left the States.
Multiple times, the FBI accessed his house, copied his hard drive. He was
arrested as a material witness to a crime. During those two weeks, he had no
communication with his family. Afterwards, the Spanish government detained a
suspect with also with a fingerprint match and Mayfield was released. During
his detainment, Mayfield was subject to humiliation, poor/unsanitary living
conditions, etc. He sued and won \$2,000,000 in a settlement.

\section*{The Fourth Amendment}
The right of the people to be secure in their persons, houses, papers, and
effects, against unreasonable searches and seizues, shall not be violated, and
no Warrants shall issue, but upon probable cause, supported by Oath or
affirmation, and particularly describing the place to be searched, and the
persons or things to be seized.

\section*{Katz versus the United States}
\begin{itemize}
  \item Arrested for using a phone booth to run an illegal gambling scheme
  \item Supreme Court ruled on the legal definition of ``search''.
  \item Significantly expanded the power of the Fourth Amendment
\end{itemize}

\section*{The Red Scare}
\begin{itemize}
  \item Espionage Act of 1917: Illegal to interfere with efforts of the US
    Armed Forces.
  \item Sedition Act of 1918: Illegal to voice criticism of US government or
    military during times of war.
  \item Imigration Act of 1918: Palmer Raids, widespread deportations.
  \item Joseph McCarthy was a US Senator from Wisconsin who tried to discredit
    others by accusing them of being Communists and claimed to have a list of
    Communists inside the US government.
  \item J. Edgar Hoover was the director of the FBI from 1935 to 1972. He was
    given warrantless wiretapping permission. This was the beginning of the
    justification of spying on citizens for the reason of national security.
\end{itemize}

\section*{Project Shamrock}
The Armed Forces Security Agency initiated Project Shamrock in the 1940s to
intercept all telegraph communications entering or exiting the United States.
It was disbanded in May 1975 as its existence became public. The
``Church Committee'' led by Senator Frank Church in 1975 found that
the correspondence of most US citizens was being monitored.

\section*{Computer Security Act of 1987}
The Computer Security Act of 1987 was in response to Reagan's National Security
Decision Directive 145 in 1984 to improve the security and privacy of sensitive
information in federal computer systems. The National Institute of Standards
and Technology was created to come up with best practices, and they could draw
on the NSA for their help regarding advice on the protection of information.
The clipper chip was promoted by the NSA, who pushed for a universal key to
decrypt communications. However, it was never peer reviewed and its standards
would only apply to US technologies.

\section*{Carnivore}
Carnivore was a software designed to access a user's email and online
communications, as well as internet history.

\section*{The USA Patriot Act}
``Uniting and Strengthening America by Providing Appropriate Tools Required to
Intercept and Obstruct Terrorism''
\begin{itemize}
  \item Allows for seizure of communication records without notifying the
    accused of why they are beign seized.
  \item Can take voicemail with just a normal search warrant, instead of a
    wiretap warrant.
  \item Section 206: Roving Surveillance Authority gave carte blanche to the
    government.
  \item Section 213: Delayed search warrant notification.
  \item Section 215: Allowed access to records and other items under the
    Foreign Intelligence Surveillance Act.
\end{itemize}

\section*{Patriot Act II}
The Domestic Security Enhancement Act draft was leaked to the public in 2003
and nicknamed the ``Patriot Act II''.

\begin{center}
  You can find all my notes at \url{http://omgimanerd.tech/notes}. If you have
  any questions, comments, or concerns, feel free to contact me at
  alvin@omgimanerd.tech
\end{center}

\end{document}
