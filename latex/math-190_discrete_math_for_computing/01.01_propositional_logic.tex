\documentclass[letterpaper, 12pt]{math}

\usepackage{amsmath}
\usepackage{amssymb}

\title{Propositional Logic}
\author{Alvin Lin}
\date{Discrete Math for Computing: January 2017 - May 2017}

\begin{document}

\maketitle

\section*{Propositional Logic}
Logic gives precise meaning to mathematical statements. A \textbf{proposition}
is a statement that declares a fact. Ex:
\begin{itemize}
  \item \( 1+1 = 2 \)
  \item \( 1+1 = 3 \)
  \item The sky is blue
  \item Today is Monday
\end{itemize}
Note that the first and third are true, while the second and fourth proposition
are false. Propositions can be either true or false. \par
We use letters to represent propositions, generally \(p, q, r, s, t\). Truth
values of a proposition are denoted as True/False, T/F, or 1/0. \par
The following are not propositions:
\begin{itemize}
  \item What time is it?
  \item \( x+1 = 9 \)
  \item \( x^{2}+y^{2} = z^{2} \)
\end{itemize}

\subsection*{Negation}
Let \( p \) be a proposition. The \textbf{negation} of \( p \), denoted
\( \neg{p} \), is the statement ``it is not the case p'', read as ``not p''.
\begin{center}
  \begin{tabular}{|c|c|}
    \hline
    \( p \) & \( \neg{p} \) \\ \hline
    T       & F             \\ \hline
    F       & T             \\ \hline
  \end{tabular}
\end{center}

\subsection*{Conjunction}
Let \( p \) and \( q \) be propositions. The \textbf{conjunction} of \( p \)
and \( q \), denoted by \( p \wedge q \), is the proposition \( p\ and\ q \).
This conjunction is only true when both \( p \) and \( q \) are true.
\begin{center}
  \begin{tabular}{|c|c|c|}
    \hline
    \( p \) & \( q \) & \( p \wedge q \) \\ \hline
    T       & T       & T \\ \hline
    T       & F       & F \\ \hline
    F       & T       & F \\ \hline
    F       & F       & F \\ \hline
  \end{tabular}
\end{center}

\subsection*{Disjunction}
Let \( p \) and \( q \) be propositions. The \textbf{disjunction} of \( p \)
and \( q \), denoted by \( p \vee q \), is the proposition \( p\ or\ q \).
This conjunction is true when either \( p \) or \( q \) are true, or both.
\begin{center}
  \begin{tabular}{|c|c|c|}
    \hline
    \( p \) & \( q \) & \( p \vee q \) \\ \hline
    T       & T       & T \\ \hline
    T       & F       & T \\ \hline
    F       & T       & T \\ \hline
    F       & F       & F \\ \hline
  \end{tabular}
\end{center}

\subsection*{Exclusive Or}
Let \( p \) and \( q \) be propositions. The \textbf{exclusive or}, denoted
\( p \oplus q \), is the proposition with the following truth table:
\begin{center}
  \begin{tabular}{|c|c|c|}
    \hline
    \( p \) & \( q \) & \( p \oplus q \) \\ \hline
    T       & T       & F \\ \hline
    T       & F       & T \\ \hline
    F       & T       & T \\ \hline
    F       & F       & F \\ \hline
  \end{tabular}
\end{center}

\subsection*{Conditional Statements}
Let \( p \) and \( q \) be propositions. The conditional statement
\( p \to q \) is the proposition ``if p then q''. This is false when
\( p \) is true and \( q \) is false, and true otherwise.
\begin{center}
  \begin{tabular}{|c|c|c|}
    \hline
    \( p \) & \( q \) & \( p \to q \) \\ \hline
    T       & T       & T \\ \hline
    T       & F       & F \\ \hline
    F       & T       & T \\ \hline
    F       & F       & T \\ \hline
  \end{tabular}
\end{center}
In English, the following mean that \( p \to q \):
\begin{itemize}
  \item if p, then q
  \item p implies q
  \item p is sufficient for q
  \item p whenever q
  \item q is necessary for p
\end{itemize}

\subsection*{Converse, Contrapositive, and Inverse}
Given \( p \to q \), we could consider the following:
\begin{itemize}
  \item The \textbf{converse} of \( p \to q \) is \( q \to p \).
  \item The \textbf{contrapositive} of \( p \to q \) is
    \( \neg{q} \to \neg{p} \).
  \item The \textbf{inverse} of \( p \to q \) is
    \( \neg{p} \to \neg{q} \).
\end{itemize}
\begin{center}
  \begin{tabular}{|c|c|c|c|c|c|}
    \hline
    \( p \) & \( q \) & \( p \to q \) & \( q \to p \) &
      \( \neg{q} \to \neg{p} \) & \( \neg{p} \to \neg{q} \) \\ \hline
    T & T & T & T & T & T \\ \hline
    T & F & F & T & F & T \\ \hline
    F & T & T & F & T & F \\ \hline
    F & F & T & T & T & T \\ \hline
  \end{tabular}
\end{center}
Propositions that have the same truth table are called \textbf{equivalent}.
\( p \to q \) and \( \neg{q} \to \neg{p} \) are always equivalent.
This is useful in mathematical proofs since it may be easier to prove the
contrapositive of a proposition.

\subsection*{Biconditionals}
Let \( p \) and \( q \) be propositions. The \textbf{biconditional} statement
\( p \leftrightarrow q \) is the proposition ``p if and only if q''.
\( p \leftrightarrow q \) is equivalent to \( (p \to q) \wedge (q \to p) \).
\begin{center}
  \begin{tabular}{|c|c|c|}
    \hline
    \( p \) & \( q \) & \( p \leftrightarrow q \) \\ \hline
    T & T & T \\ \hline
    T & F & F \\ \hline
    F & T & F \\ \hline
    F & F & T \\ \hline
  \end{tabular}
\end{center}

\subsection*{Precedence of Operators}
\begin{center}
  \begin{tabular}{|c|c|}
    \hline
    Operator              & Precedence \\ \hline
    \( \neg \)            & 1 \\ \hline
    \( \wedge \)          & 2 \\ \hline
    \( \vee \oplus \)     & 3 \\ \hline
    \( \to \)             & 4 \\ \hline
    \( \leftrightarrow \) & 4 \\ \hline
  \end{tabular}
\end{center}
\[ \neg{p} \wedge q := (\neg{p}) \wedge q \]

\subsection*{Example}
Find the truth table for \( (p \vee \neg{q}) \to (p \wedge q) \).

\end{document}
