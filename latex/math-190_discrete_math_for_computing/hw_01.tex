\documentclass[letterpaper, 12pt]{math}

\usepackage{amsmath}
\usepackage{amssymb}

\title{Homework \#1}
\author{Alvin Lin}
\date{Discrete Math for Computing: January 2017 - May 2017}

\begin{document}

\maketitle

\subsection*{1a}
\( p \) = ``I will dance a jig'', \( q \) = ``I will have fun doing it''
\[ p \wedge q \therefore p \]
\begin{center}
  \begin{tabular}{|c|c|}
    \hline
    \( p \wedge q \) & \( p \) \\ \hline
    T                & T       \\ \hline
  \end{tabular}
\end{center}

\subsection*{1b}
\( p \) = ``Morgan will join a band next year'', \( q \) = ``she will be
playing alone''.
\[ p \therefore p \vee q \]
\begin{center}
  \begin{tabular}{|c|c|}
    \hline
    \( p \) & \( p \vee q \) \\ \hline
    T       & T              \\ \hline
  \end{tabular}
\end{center}

\subsection*{1c}
\( p \) = ``I work all week on this homework'', \( q \) = ``I will answer all
the problems'', \( r \) = ``I will understand all the material''.
\[ p \to q, q \to r \therefore p \to r \]
\begin{center}
  \begin{tabular}{|c|c|c|}
    \hline
    \( p \to q \) & \( q \to r \) & \( p \to r \) \\ \hline
    T             & T              & T            \\ \hline
  \end{tabular}
\end{center}

\subsection*{2a}
\[ p \wedge \neg{p} \]
\begin{center}
  \begin{tabular}{|c|c|c|}
    \hline
    \( p \) & \( \neg{p} \) & \( p \wedge \neg{p} \) \\ \hline
    T       & F             & F                      \\ \hline
    F       & T             & F                      \\ \hline
  \end{tabular}
\end{center}

\subsection*{2b}
\[ (p \vee \neg{q}) \to q \]
\begin{center}
  \begin{tabular}{|c|c|c|c|}
    \hline
    \( p \) & \( q \) & \( p \vee \neg{q} \) & \( (p \vee \neg{q}) \to q \) \\
      \hline
    T       & T       & T                    & T \\ \hline
    T       & F       & T                    & F \\ \hline
    F       & T       & F                    & T \\ \hline
    F       & F       & T                    & T \\ \hline
  \end{tabular}
\end{center}

\subsection*{2c}
\[ p \to \neg{p} \]
\begin{center}
  \begin{tabular}{|c|c|c|}
    \hline
    \( p \) & \( \neg{p} \) & \( p \to \neg{p} \) \\ \hline
    T       & F             & F                   \\ \hline
    F       & T             & T                   \\ \hline
  \end{tabular}
\end{center}

\subsection*{2d}
\[ p \oplus p \]
\begin{center}
  \begin{tabular}{|c|c|}
    \hline
    \( p \) & \( p \oplus p \) \\ \hline
    T       & F                \\ \hline
    F       & F                \\ \hline
  \end{tabular}
\end{center}

\subsection*{2e}
\[ (p \oplus q) \to (p \wedge q) \]
\begin{center}
  \begin{tabular}{|c|c|c|c|c|}
    \hline
    \( p \) & \( q \) & \( p \oplus q \) & \( p \wedge q \) &
        \( (p \oplus q) \to (p \wedge q) \) \\ \hline
    T & T & F & T & T \\ \hline
    T & F & T & F & F \\ \hline
    F & T & T & F & F \\ \hline
    F & F & F & F & T \\ \hline
  \end{tabular}
\end{center}

\subsection*{2f}
\[ (p \to q) \leftrightarrow (\neg{q} \to \neg{p}) \]
\begin{center}
  \begin{tabular}{|c|c|c|c|c|}
    \hline
    \( p \) & \( q \) & \( p \to q \) & \( \neg{q} \to \neg{p} \) &
        \( (p \to q) \leftrightarrow (\neg{q} \to \neg{p}) \) \\ \hline
    T & T & T & T & T \\ \hline
    T & F & F & F & T \\ \hline
    F & T & T & T & T \\ \hline
    F & F & T & T & T \\ \hline
  \end{tabular}
\end{center}
Tautology!

\subsection*{2g}
\[ (p \wedge q) \vee r \]
\begin{center}
  \begin{tabular}{|c|c|c|c|c|}
    \hline
    \( p \) & \( q \) & \( r \) & \( p \wedge q \) & \( (p \wedge q) \vee r \)
        \\ \hline
    T & T & T & T & T \\ \hline
    T & T & F & T & T \\ \hline
    T & F & T & F & T \\ \hline
    T & F & F & F & F \\ \hline
    F & T & T & F & T \\ \hline
    F & T & F & F & F \\ \hline
    F & F & T & F & T \\ \hline
    F & F & F & F & F \\ \hline
  \end{tabular}
\end{center}

\subsection*{2h}
\[ (p \vee q) \wedge r \]
\begin{center}
  \begin{tabular}{|c|c|c|c|c|}
    \hline
    \( p \) & \( q \) & \( r \) & \( p \vee q \) & \( (p \vee q) \wedge r \)
        \\ \hline
    T & T & T & T & T \\ \hline
    T & T & F & T & F \\ \hline
    T & F & T & T & T \\ \hline
    T & F & F & T & F \\ \hline
    F & T & T & T & T \\ \hline
    F & T & F & T & F \\ \hline
    F & F & T & F & F \\ \hline
    F & F & F & F & F \\ \hline
  \end{tabular}
\end{center}

\subsection*{2i}
\[ (p \leftrightarrow q) \leftrightarrow (q \leftrightarrow r) \]
\begin{center}
  \begin{tabular}{|c|c|c|c|c|c|}
    \hline
    \( p \) & \( q \) & \( r \) & \( (p \leftrightarrow q) \) &
    \( (q \leftrightarrow r) \) &
    \( (p \leftrightarrow q) \leftrightarrow (q \leftrightarrow r) \) \\ \hline
    T & T & T & T & T & T \\ \hline
    T & T & F & T & F & F \\ \hline
    T & F & T & F & F & T \\ \hline
    T & F & F & F & T & F \\ \hline
    F & T & T & F & T & F \\ \hline
    F & T & F & F & F & T \\ \hline
    F & F & T & T & F & F \\ \hline
    F & F & F & T & T & T \\ \hline
  \end{tabular}
\end{center}

\subsection*{4a}
If it snows tonight, then I will stay at home. \\
Converse: If I will stay at home, then it will snow tonight. \\
Contrapositive: If I will not stay at home, then it will not snow. \\
Inverse: If it will not snow tonight, then I will not stay at home.

\subsection*{4b}
I go to the beach whenever it is a sunny day. \\
Converse: It is a sunny day whenever I go to the beach. \\
Contrapositive: It is not a sunny day whenever I don't go to the beach. \\
Inverse: I don't go to the beach when it is not a sunny day.

\subsection*{4c}
A positive integer is a prime only if it has no divisors other than 1 and
itself. \\
Converse: If a positive integer has no divisors other than 1 and itself, it is
a prime. \\
Contrapositive: If a positive integer has divisors other than 1 and itself,
then it is not prime. \\
Inverse: A positive integer is not prime if it has divisors other than 1 and
itself.

\end{document}
