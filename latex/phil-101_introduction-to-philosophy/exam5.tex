\documentclass{article}

\title{Exam 5}
\author{Alvin Lin}
\date{May 17th, 2017}

\begin{document}

\maketitle

\subsubsection*{Why is social epistemology a fundamentally ethical issue
according to Fricker?}
Social epistemology relates to the silencing of certain people through
prejudices and judgments made about their knowledge. This results in the
degrading and devaluing of a person's thoughts or words based on generally
superficial qualities. Silencing and devaluing a person's thoughts and words
is akin to degrading the person themself, which makes social epistemology
a fundamentally ethical issue. The study of social epistemology seeks to right
this by helping ``hearers'' identify prejudices they may have so that their
perspective may be properly adjusted.

\subsubsection*{Davis referred to the prison industrial complex as a failure
in abolition democracy. What does she mean by this?}
Angela Davis refers to the industrial complex as a failure in abolition
democracy because it represents the system of slavery by trapping minorities
in a cycle of despair through circumstances such as poverty. The prison
industrial complex is not as overt as slavery, but still facilitated through
de facto discrimination and laws that prevent former inmates from finding jobs.
The mass incarceration of minorities becomes an anchor on their community as a
whole since those incarcerated have a difficult time finding jobs and may
resort to crime again to survive, creating a vicious cycle of poverty and crime.

\subsubsection*{If Superman lacked his dual identity with Clark Kent, and was
just Superman, how would this affect the bond with the readers?}
Superman's dual identity allows the reader some sense of relatability since
it gives a primarily supernatural and superpowered being a human side.
Superman's identity as Clark Kent is a tangible surface that readers can
relate to since we do not have superpowers. This relatability is what allows
us to build a connection to the superhero as we follow along in the comics.
Without the dual identity, the comics would be a lot less personal and would
be a lot more removed from our lives. The dual identity brings superpowers
from an abstract level to a human level.

\subsubsection*{What is meant by social and civil death? How does this relate
to the prison system, especially solitary confinement?}
Civil death is the phenomenon of having a person be dead legally even though
they are alive in body. This usually involves depriving them of their rights.
Social death involves a person's figurative death by isolating them from society
for so long that they no longer have a place in it. Prison facilitates both of
these since they confine inmates for so long that they can no longer function
as productive members of society, demonstrating the notion of punishment over
rehabilitation. Solitary confinement worsens civil death because it damages an
inmate's sense of identity and self and makes it even more difficult for them
to reintegrate into society.

\end{document}
