\documentclass{article}

\title{Exam 4}
\author{Alvin Lin}
\date{April 28th, 2017}

\begin{document}

\maketitle

\subsubsection*{What does Kierkegaard and/or Johannes de Silentio mean by ``faith''?}
Kierkegaard recognizes faith as something ``absurd'' and irrational. It cannot be rationally explained or justified and thus cannot be subjected to the laws of logic and reason. Since it cannot be observed and understood within the bounds of human reason, Kierkegaard reasons that faith is something near ``divine''. His understanding of faith is better understood as an implicit trust in the absoluteness of God and the power of God, rather than the belief in the existence of God. This requires a ``leap of faith'', so to speak, because it cannot be understood and must be undertaken without reason or thought.

\subsubsection*{What does Kierkegaard and/or Johannes de Silentio mean by a teleological suspension of the ethical? How does Caravaggio understand the teleological suspension of the ethical in his 2 paintings of the binding of Isaac?}
Kierkegaard asks the question of whether or not faith can transcend the ethics defined by society as we understand it. According to his philosophy, since faith is beyond our understanding, it has a higher precendence than logic or ethical frameworks. A teleological suspension of the ethical refers to the idea that one must suspend judgments stemming from an ethical viewpoint in favor of taking actions based on ``absurd'' faith. Caravaggio's 1598 painting demonstrates this idea since both Abraham and Isaac are depicted with a surreal sense of acceptance for what is about to happen. Abraham suspends his ethical obligation as a father in order to demonstrate his faith while Isaac accepts his fate knowing he is about to be sacrificed simply because faith dictated it. Caravaggio's 1603 rendition of the \textit{Sacrifice of Isaac} shows a much more realistic side of how Isaac would feel if he were about to be sacrificed.

\subsubsection*{Why does Kierkegaard employ pseudonyms? Does he \textit{need} to or is it merely a literary artifice?}
Kierkegaard writes under various pseudonyms in order to highlight a contrast in opinion and create dialogue. As a educated theologian, Kierkegaard uses them to take various personas in order to present his arguments. In \textit{Fear and Trembling}, Kierkegaard uses the persona Johannes de Silentio (Johannes the Silent) in order to adapt an alternative viewpoint on faith using the Biblical Abraham as a demonstrative example. While Kierkegaard's opinions clashed with the church, they did not put him in danger, so his pseudonyms were more of a literary element rather than a method of preserving his anonymity.

\subsubsection*{Explain why Kierkegaard and/or Johannes de Silentio is so fixated on the Abraham/Isaac story.}
Johannes the Silentio fixates on the Biblical story of Abraham's sacrifice of Isaac because he believes that Abraham is the ultimate example of a person who lives by faith. Abraham embraces the ``absurdity'' of faith because he absolves himself of his role as a father in order to sacrifice his son Isaac. He demonstrates no doubt in God's demands despite being asked to sacrifice the very son he was promised by God. Since he tells no one of his intent, Abraham's actions are incomprehensible and reflect the same transcendence of logic and reason that faith has.

\subsubsection*{Discuss whether, and in what respects, \textit{Frailty} functions as a critique of Kierkegaard's \textit{Fear and Trembling}. Is its critique convincing (if there is one)? Why or why not?}
Bill Paxton's \textit{Frailty} functions as a critique of \textit{Fear and Trembling} not by satirizing or highlighting contradictions in it, but by exaagerating Johannes de Silentio's idea of faith and exploring the absurd. The characters in \textit{Frailty} demonstrate that their actions are motivated by faith and that destroying ``demons'' is an act that transcends the ethical standards of society. The film includes an element of the supernatural since Adam Meiks is ``protected'' in his actions by God. \par
Though \textit{Frailty} is an extreme example, it demonstrates the idea that faith is an absolute and can be used for any justification. The character of Adam Meik in the film demonstrates Kierkegaard's idea of a ``knight of faith''. It also reflects Kierkegaard's idea of absolutes by giving Adam the role of judge, jury, and executioner. Even though the ``rightness'' of his actions is show in the fact that he is sent to strike down killers, this presents the troubling idea that one could appoint oneself judge, jury, and executioner in real life with just the motivation of faith. \textit{Frailty} serves as an effective critique by highlighting the insanity of such an idea in the context of the real world.

\end{document}
