\documentclass{article}

\title{Exam 3}
\author{Alvin Lin}
\date{March 29th, 2017}

\begin{document}

\maketitle

\subsubsection*{Describe old detectivism.}
Old detectivism is a concept stating that there is an unchallengable and
``infallible'' nature to our own state of mind. The ability to
authoritatively determine one's state of mind is granted only to the
individual.

\subsubsection*{Describe new detectivism.}
New detectivism challenges the old detectivism concepts by noting that our
sense of self-awareness is fallible. We have better access to our own states
of mind, but that access is by no means perfect. New detectivists deny that
one's relation to their state of mind is anything special or divine.

\subsubsection*{Why, according to Finkelstein, do both old and new detectivism
prove unsatisfactory?}
Old and new detectivism assert the existence of some mechanism that allows us
to perceive our own mind. Both grant that there is a uniqueness and
``aspectual richness'' to personally sensing something, and Finkelstein
argues that this leaves much to be desired because it fails to explain how such
a mechanism of authority can exist.

\subsubsection*{Describe constitutivism. What are Finkelstein’s main objections
to it?}
Constitutivism is the idea that stating one's state of mind makes it into what
we stated. We control our state of mind via our intention to feel or do
something. Finkelstein argues that this misrepresents our role in our own mind
when we ``feel'' something since it does not provide a satisfactory explanation
of the unconscious.

\subsubsection*{Describe the Middle Path Account, making sure to reference the
Myth of the Given.}
The Middle Path Account is a combination of ideas from both detectivism and
constitutivism. It states that there is an outer sense, outside of which
objects can exist, and it is independent of one's inner sense. Objects such
as pain or emotion only exist in one's inner sense because of one's conscious
awareness of it. Like the Myth of the Given, this posits that there is a
distinction between sensory stimulation and what we actually experience in
our mind. This distinction makes the two categorically different.

\subsubsection*{Why is it that case that, normally, the best person to ask when
it comes to your thoughts, feelings, attitudes, sensations, and experiences is
you? That is, why should first-person authority be routinely granted?}
Though detectivism has its flaws, it is correct in stating that you have the
most privileged access to your own mental state. It makes the most sense to
ask the person actually experiencing the thoughts, feelings, and sensations
because the person who is having the experiences has the most accurate
form of description for the experience. \par
In contrast, it is more logical to be skeptical of a person's assertion
about someone else's state of mind because they do not have privileged access
to that person's emotions and experiences. \par
In general first-person authority is routinely granted because one has no
reason to doubt the authority of a person asserting a claim on their state
of mind. Someone claiming that they feel upset has nothing to gain by lying
(unless they have a motive, which is a separate discussion) about it and thus
it does not make sense to demand evidence for the reason behind the person's
state of mind. In our day-to-day conversation, we grant first-person
authority to a person about their state of mind because we have no reason
otherwise.

\end{document}
