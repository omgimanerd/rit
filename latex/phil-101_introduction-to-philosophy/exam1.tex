\documentclass[letterpaper, 12pt]{article}

\title{Exam 1}
\author{Alvin Lin}
\date{February 15th, 2017}

\begin{document}

\maketitle

\subsubsection*{Describe and discuss the significance of an instance of
Socrates' irony.}
Throughout the entirety of \textit{The Apology}, Socrates professes that he is not wise and that his mission has been to find one wiser than he. Ironically enough, Socrates shows the opposite during his defense, proving that he is indeed wise and what he says actually means the exact opposite. His wisdom is first shown when he dissects Meletus's charge that he is corrupting the young, and continues to show as he systematically dismantles all the arguments against him. Although he resorts to various fallacies of logic, he essentially ridicules Meletus by showing that his arguments are insubstantial. This is significant in the fact that he uses his claim to ignorance to show that the accusers know nothing either.

\subsubsection*{Compare and contrast Socrates/Plato's and Averroes' attitudes
toward religion.}
Socrates uses religion to substantiate his argument. Albeit fallacious, he appeals to the oracle at Delphi to state that he is the wisest of men and seeks to prove that false. He accepts that religion and truth can coexist since he uses that to substantiate his defense. Averroes treats religion similarly and seeks to resolve any contradiction between religion and philosophy and ``truth''. However, Socrates uses an appeal to religion as a tool for his argument, and uses its literal meaning as a basis for his defense. Averroes is more primarily concerned with the role of religion as a metaphor and the differences between its application to the common believers versus intellectuals and philosphers.

\subsubsection*{Explain and assess Averroes' claim that religious law, properly
understood, makes the study of philosophy mandatory. Make sure to explain and
assess Averroes' grounds for this claim.}
Averroes claims that the study of philosophy is defined as ``the study of existing beings and reflection on them as indications of the Artisan''. In essence, since philosophy dictates the study and analysis of the world and religion requires a study of the world, the end goals of religious law and philosophy coincide. He goes on to say that in order to understand religious law and the analyses of the world, one needs to be armed with the skills of inference and logic. Since the study of philosophy involves intellectual reasoning, it is obvious that the study of religious law requires philosophy.

\subsubsection*{If you were in ancient Athens, and a juror in the trial of
Socrates, how, on the basis of arguments available from The Apology, would you
vote, and why would you vote as you do? If you find Socrates guilty, what
punishment would you recommend, and why?}
If I were a juror of the trial of Socrates, I would vote Socrates guilty of being a public disturbance. Firstly, Socrates is hypocritical in his argument when he warns the jury of being wary of sophistry in the charges against him, but does not shirk from using wordplay and a manipulation of the charges against him to defend himself. His defense of himself is based of fallacies, appeals to authority, and wordplay. His defense is further weakened by the fact that he freely admits to going around questioning public officials and ridiculing them under the claim that he is seeking someone wiser than himself. Although arrogance and public ridiculing are not crimes, it certainly doesn't help in his argument. \par
Socrates's argument that he is doing good for society and his argument of intent would certainly commute his sentence however. Since he freely admits to being a public nuisance, he would be pronounced guilty of that crime. However, Socrates makes a compelling argument that his intent is not malicious, thus the death penalty does not seem suitable. A suitable punishment, and one in which I would recommend simply out of spite, would be community service forcing Socrates to serve as a public official.

\end{document}
