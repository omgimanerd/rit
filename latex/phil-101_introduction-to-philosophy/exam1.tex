\documentclass[letterpaper, 12pt]{article}

\title{Exam 1}
\author{Alvin Lin}
\date{February 15th, 2017}

\begin{document}

\maketitle

\subsection*{Describe and discuss the significance of an instance of Socrates'
irony.}
Throughout the entirety of \textit{The Apology}, Socrates professes that he is not wise and that his mission has been to find one wiser than he. Ironically enough, Socrates shows the opposite during his defense, proving that he is indeed wise and what he says actually means the exact opposite. His wisdom is first shown when he dissects Meletus's charge that he is corrupting the young, and continues to show as he systematically dismantles all the arguments against him. Although he resorts to various fallacies of logic, he essentially ridicules Meletus by showing that his arguments are insubstantial. This is significant in the fact that he uses his claim to ignorance to show that the accusers know nothing either.

\subsection*{Compare and contrast Socrates/Plato's and Averroes' attitudes
toward religion.}
Socrates uses religion to substantiate his argument. Albeit fallacious, he appeals to the oracle at Delphi to state that he is the wisest of men and seeks to prove that false. He accepts that religion and truth can coexist since he uses that to substantiate his defense. Averroes treats religion similarly and seeks to resolve any contradiction between religion and philosophy and ``truth''. However, Socrates uses an appeal to religion as a tool for his argument, and uses its literal meaning as a basis for his defense. Averroes is more primarily concerned with the role of religion as a metaphor and the differences between its application to the common believers versus intellectuals and philosphers.

\end{document}
