\documentclass[letterpaper, 12pt]{article}

\title{Exam 2}
\author{Alvin Lin}
\date{March 10th, 2017}

\begin{document}

\maketitle

\subsubsection*{Why is Machiavelli so insistently ironic/esoteric? Why not
just come out and say what he means?}
The exoteric meaning of his writing counsels princes on how to effectively rule through the suppression of any threats to their power, which can include ruthness oppression. Machiavelli's writing itself is a threat to the sovereignty of a prince since it subtly suggests that principalities are unstable. Hilariously enough, the people in power would likely use the exoteric meaning of Machiavelli's own words to persecute or target him.

\subsubsection*{What is a republican form of government? What are its specific
advantages? Do you find Machiavelli’s AND/OR The Man Who Shot Liberty Valance’s
account of these advantages convincing or not, and why?}
A republican form of government is one in which the people hold the power of rule instead of any single entity. Every member of a republican form of government has a say in the operation of their government. Republican forms of government have the advantage of being the most utilitarian, catering to the interests and happiness of the largest group of people. This allows for change to be enacted for the betterment of the population as shown in \textit{The Man Who Shot Liberty Valance}. The people of Shinbone vote for statehood, demonstrating their power in a republic to enact change. Overall, Machiavelli's subtle argument for a republic is convincing because the entire concept of a principality hinges on the benevolence of the prince.

\subsubsection*{What is a principality? Why does Machiavelli think it is
inevitably unstable? Do you agree or disagree, and why?}
A principality is a region ruled under the autocracy of a prince. Machiavelli argues that a principality is unstable because they eventually crumble internally due to the poor leadership of the prince or the dissatisfaction of the people. Principalities can only maintain peace by becoming empires since they can justify control with wartime threats. While this is true, the course of history has shown the empires last significantly longer than their republican counterparts. Through imperialist warmongering, empires can achieve stability as long as there are regions to conquer and colonize. Though empires eventually crumble, so do republics if the population contains too many factions with dissenting opinions.

\subsubsection*{Machiavelli counsels the prince to learn how not to be good.
What does Machiavelli mean by this? Do you agree or disagree with his counsel,
and why?}
Machiavelli counsels the prince to learn how not to be good because the politics of statecraft require underhanded and often ``immoral'' actions. Machiavelli tells princes to gauge their actions on their effect on the stability of the state and not their innate moral values. Though he notes that cruelty should not be used for the sake of cruelty, Machiavelli does not shy away from telling princes that cruelty is sometimes a necessary tool as a means to an end. Even in a republic or a democracy like ours, the government must undertake actions that are morally questionable for the greater good. Some of the more controversial actions fall under the government's counterterrorism and counterespionage efforts. While the actions of the government involve immoral decisions such as indeterminate detainment, it is arguable that such actions may be necessary to preserve the security and stability of the state.

\subsubsection*{Machiavelli claims that the modern world is a contingent world,
involving unique, rapidly changing constellations of circumstances rather than
repeated and predictable patterns of events. It is a world of chance and
contingency (``fortune''), rather than a world of ordered regularity. Thus he
claims that knowledge of the great and glorious actions past leaders, of what
once made for political success, is of no use to political actors in the
present. Do you agree or disagree with this assessment (as it pertains to
EITHER Machiavelli’s time OR to the contemporary world), and why?}
One idea that Machiavelli discusses exoterically is that good political pedagogy requires imitation of the great rulers of the past. In some regards, the events of the past are applicable today since many of the problems that governments face today are variations of problems of the past. The modern world in today's context still faces problems like diplomacy between nations. Diplomacy is something that leaders will always need to know how to handle, regardless of the time period they are in. However, the world today has many ideas and technologies that leaders of the past did not. \par
Machiavelli's claim is innately correct because inventions like the Internet have caused today's moral problems to be unique and have ``rapidly changing constellations of circumstances''. Innovation and science always bring new problems that are unpredictable and patternless. Politics now take place on a global scale and are significantly more complicated than even Machiavelli could have predicted. The political situation in the Middle East and the start of World War I are both complex geopolitical situations that no ``prince'' could have predicted. The actions of rulers in the past are therefore of no use to the leaders of today with regard to the complex situations that arise in the modern world.

\end{document}
