\documentclass{math}

\usepackage{graphicx}
\usepackage{listings}

\title{Principles of Data Mining: HW 05}
\author{Alvin Lin}
\date{August 2018 - December 2018}
\begin{document}

\maketitle

\subsubsection*{Question 3a}
What stopping criteria did you use? \par
I used recursion depth as the stopping criteria. The algorithm stopped after
branching a certain number of times.

\subsubsection*{Question 3b}
Did you use any pruning or post-pruning? \par
No.

\subsubsection*{Question 3c}
What splitting decision were you using? \par
I used weighted Gini index to determine the purity of the split.

\subsubsection*{Question 3d}
What structure did your final decision tree classifier have?
\begin{lstlisting}
for row in f:
    data = preprocess_training_data_row(row)
    if data[3] >= 25: # Sugar
        if data[1] >= 30: # FlourOrOats
            category = 'Muffin'
        else:
            category = 'Cupcake'
    else:
        if data[4] >= 18: # Butter
            category = 'Cupcake'
        else:
            category = 'Muffin'
\end{lstlisting}

\subsubsection*{Question 3e}
Run the original training data back through your classifier. What was the
accuracy of your resulting classifier on the training data? \par
On the training data, there were 42 correct classifications and 19
misclassifications, giving us an accuracy rate of 68.9\%.

\subsubsection*{Question 3f}
Did your program actually create the classifier program, or did it generate
the attribute list and thresholds for you to hand-code in? \par
My program generated a decision tree stored as nodes and then translated those
nodes into Pythonic if-else statements.

\subsubsection*{Question 3g}
What else did you learn along the way? \par
I learned a lot more about recursion in Python since I had forgotten the
mechanics of how it worked. It was a good refresher on how to use recursion in
Python and the application was something I had never done before.

\subsubsection*{Question 3h}
What can you conclude? \par
Overall, none of the attributes are by themselves a very good indicator of
whether a recipe is cupcake or muffin. It seems that sugar is the best starting
attribute since cupcakes and muffins are most divided by this attribute. Using
this in combination with FlourOrOats and Butter lead me to the best distinction.

\begin{center}
  If you have any questions, comments, or concerns, please contact me at
  alvin@omgimanerd.tech
\end{center}

\end{document}
