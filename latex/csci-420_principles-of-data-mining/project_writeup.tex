\documentclass{math}

\usepackage{graphicx}
\usepackage{listings}

\title{Principles of Data Mining: GPS Project}
\author{Alvin Lin and Stefan Aleksic}
\date{August 2018 - December 2018}
\begin{document}

\maketitle

\subsection*{Background}
What is GPS?
What was the original intent?
How many satellites are there?
What other relevant details should we know about GPS?
What else did you find out that is interesting to GPS?

\subsection*{Program Pattern}
We wrote the program with Python, making generous use of the \texttt{numpy}
library to handle the large quantities of data and the \texttt{fastkml} and
\texttt{shapely} libraries to handle formatting the results of our data
aggregation into a KML file for output. These libraries make use of Python's
\texttt{lxml} module to handle the structuring of the XML necessary to follow
create the KML file. Due to the efficiency of \texttt{numpy}, we were able to
store the preprocessed input file entirely in memory for processing. \par
We did not use any particular design patterns, but separated out generic utility
methods into a utility file (\texttt{util.py}). The utility file contained
functions for parsing the longitude and latitude from the GGA and RMC format, as
well as functions for calculating great circle distance and bearing between
two latitude-longitude GPS coordintes. \par
The first thing we did with the input data was clean it. To do this, we
calculated the distance between each consecutive point and pruned out
consecutive points that were closer than a threshold. We determined this
threshold through manual testing until we achieved an array of points that
were close enough to resolve detail but sparse enough to remove clutter. \par

\subsubsection*{Detecting Left Turns}
To detect left turns in the data, we calculated the bearing between each
consecutive pair of latitude-longitude coordinates. We used a sliding window
across this resulting array of bearings to calculate a signed delta in bearing
across the window. If the delta was greater than 60 degrees, we determined
that section of the window to be a left turn and labeled it as such. \par
We did not have to do any noise removal or signal processing since much of
the noise was removed in the preprocessing stage where we cleaned out
redundant points based on adjacent proximity. However, the sliding window
method would tend to place many points across the entirety of a turn, so
as soon as a left turn is found when the bearing delta goes over the
threshold, we shift the window past the entire turn to avoid relabeling the
turn. We also had problems with detecting turns that were prolonged over a
longer stretch (such as the drive down Andrews Memorial Drive), which we
handled by increasing the window size.

\subsubsection*{Detecting Stop Signs}

\subsection*{Results}
Running \texttt{project.py} will output the KML file to stdout instead of
outputting the coordinates of the left turns and stop signs.
\begin{lstlisting}
python project.py > output.kml
\end{lstlisting}
Importing the KML into Google Earth will show the following:

\subsection*{Summary and Conclusion}

\begin{center}
  If you have any questions, comments, or concerns, please contact me at
  alvin@omgimanerd.tech
\end{center}

\end{document}
