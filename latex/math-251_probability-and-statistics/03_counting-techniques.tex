\documentclass[letterpaper, 12pt]{math}

\usepackage{forest}

\title{Probability and Statistics}
\author{Alvin Lin}
\date{Probability and Statistics: January 2017 - May 2017}

\begin{document}

\maketitle

\section*{Counting Techniques}
\textbf{Product Rule:} If the first element or object of an ordered pair can be
selected in \( n_{1} \) ways, and for each of these \( n_{1} \) ways the second
element of the pair can be selected in \( n_{2} \) ways, then the number of
pairs is \( n_{1}n_{2} \).

\subsection*{Permutation and Combination Formulas}
Number of ways of selecting r items from n distinct items:
\[ \nPr{n}{r} = P(n,r) = P_{r,n} = \nPrf{n}{r} \]
\[ \nCr{n}{r} = C(n,r) = C_{r,n} = \frac{\nPr{n}{r}}{r} = \nCrf{n}{r} \]
\[ where \]
\[ n! = (n)(n-1)(n-2)\dots(3)(2)(1) \]
\[ 5! = (5)(4)(3)(2)(1) \]
\[ 3! = (3)(2)(1) \]
\[ 1! = 1 \]
\[ 0! = 1 \]
Repetition is not allowed. For \( \nPr{n}{r} \), order matters, while for
\( \nCr{n}{r} \), order does not matter. Suppose we are choosing two colors
from R, G, B:
\[ \nPr{3}{2} = \nPrf{3}{2} = \frac{(3)(2)(1)}{1} = 6 \]
This is analogous to choosing those two colors when order does not matter,
thus there are 6 permutations: RG, RB, GR, GB, BR, BG.
\[ \nCr{3}{2} = \frac{\nPr{3}{2}}{2!} = \nCrf{3}{2} = 3 \]
This is analogous to choosing those two colors when order does matter,
thus there are 3 combinations: RG (same as GR), RB (same as BR), and BG (same
as GB).

\subsection*{Example}
\[ \nPr{4}{4} = \nPrf{4}{4} = \frac{4!}{0!} = \frac{4!}{1} = 24 \]
This is analogous to choosing a permutation of 4 items out of 4 distinct items.
Suppose we are choosing 4 colors out of R, G, B, W:
\begin{center}
  \begin{forest}
    [R
      [G [B [W] ] [W [B] ] ]
      [B [G [W] ] [W [G] ] ]
      [W [B [G] ] [G [B] ] ]
    ]
  \end{forest}
  \begin{forest}
    [G
      [R [B [W] ] [W [B] ] ]
      [B [R [W] ] [W [R] ] ]
      [W [B [R] ] [R [B] ] ]
    ]
  \end{forest}
\end{center}
\begin{center}
  \begin{forest}
    [B
      [G [R [W] ] [W [R] ] ]
      [R [G [W] ] [W [G] ] ]
      [W [R [G] ] [G [R] ] ]
    ]
  \end{forest}
  \begin{forest}
    [W
      [G [R [B] ] [W [R] ] ]
      [R [G [B] ] [B [G] ] ]
      [B [R [G] ] [G [R] ] ]
    ]
  \end{forest}
\end{center}

\subsection*{Example}
\[ \nPr{4}{0} = \nPrf{4}{0} = \frac{4!}{4!} = 1\]
This is analogous to choosing 0 items out of 4 distinct items. We can
interpret this as there was no selection and thus no result, resulting in 0
permutations, or we can interpret not selection as an action, thus 1
permutation.

\subsection*{Example}
In how many ways can six different cell phones be arranged on top of one
another?
\[ \nPr{6}{6} = \nPrf{6}{6} = \frac{6!}{0!} = 6! = 720 \]

\subsection*{Example}
Consider 5 letters a, b, c, d, e. In how many ways can three letters be
selected and arranged if repetition is not allowed?
\[ \nPr{5}{3} = \nPrf{5}{3} = \frac{5!}{2!} = 60 \]

\subsection*{Example}
While visiting NYC, the Friedmans are interested in visiting 8 museums but have
time to visit only 3. In how many ways can the Friedmans select 3 of the 8
museums to visit?
\[ \nCr{8}{3} = \nCrf{8}{3} = \frac{8!}{5!3!} = 56 \]

\subsection*{Example}
A basket contains 4 balls marked R, G, B, Y. Select a ball randomly 2 times
with replacement.
\begin{align*}
  E_{1}&: getting\ a\ G\ and\ a\ B\ without\ regard\ to\ order \\
  E_{2}&: getting\ two\ Y's
\end{align*}
Find \( P(E_{1}) \) and \( P(E_{2}) \).
\[ P(E_{1}) \neq \frac{1}{\nCr{4}{2}} = \frac{1}{6} \]
\[ P(E_{2}) \neq \frac{1}{\nCr{4}{2}} = \frac{1}{6} \]
In reality:
\[ P(E_{1}) = \frac{2}{4\times4} = \frac{1}{8} \]
\[ P(E_{2}) = \frac{1}{4\times4} = \frac{1}{16} \]

\subsection*{Example}
Computer keyboard failures can be attributed to electrical defects or
mechanical defects. A repair facility currently has 25 failed keyboards, 6
of which have electrical defects and 19 of which have mechanical defects.
How many ways are there to randomly select 5 of these keyboards for a thorough
inspection (without regard to order)?
\[ \nCr{25}{5} = \nCrf{25}{5} = 53130 \]
In how many ways can a sample of 5 keyboards be selected so that exactly two
have an electrical defect? The set of \( A \) is the set of keyboards with
electrical defects, and the set of \( B \) is the set of keyboards with
mechanical defects.
\[ n(A \cup B) = n(A) + n(B) - n(A \cap B) \]
No keyboard has both electrical and mechanical defects. \\
Those with electrical defects: \( \nCr{6}{2} \). \\
Those with mechanical defects: \( \nCr{19}{3} \). \\
Answer: \( \nCr{6}{2} \times \nCr{19}{3} = 984 \). \\
If a sample of 5 keyboards is randomly selected, what is the probability that
at least 4 of these will have a mechanical defect? There are \( \nPr{25}{5} \)
ways to choose 5 keyboards. \\
\begin{align*}
  P(at\ least\ 4\ mechanical\ defects) &=
    P(4\ mechanical\ defects \cup 5\ mechanical\ defects) \\
  &= P(4\ mechanical\ defects)+P(5\ mechanical\ defects)- \\
  & P(4\ mechanical\ defects \cap 5\ mechanical\ defects) \\
  &= P(4\ mechanical\ defects)+P(5\ mechanical\ defects) \\
  &= \frac{\nCr{19}{4}\times\nCr{6}{1}}{\nCr{25}{5}}+
    \frac{\nCr{19}{5}}{\nCr{25}{5}}
\end{align*}
Every combination has \( 5! \) corresponding permutations.

\begin{center}
  You can find all my notes at \url{http://omgimanerd.tech/notes}. If you have
  any questions, comments, or concerns, please contact me at
  alvin@omgimanerd.tech
\end{center}

\end{document}
