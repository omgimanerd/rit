\documentclass[letterpaper, 12pt]{math}

\usepackage{amsmath}
\usepackage{amssymb}
\usepackage[linguistics]{forest}

\title{Probability and Statistics}
\author{Alvin Lin}
\date{Probability and Statistics: January 2017 - May 2017}

\begin{document}

\maketitle

\section*{Continuous Random Variables}
Let \( X \) be a continuous random variable. The probability distribution or
probability density function (pdf) of \( X \) is a function \( f(x) \) such that
for any two numbers \( a \) and \( b \) with \( a \leq b \).
\[ P(a\leq X\leq b) = \int_{a}^{b}f(x)\diff{x} \]
For a function \( f(x) \) to be a valid pdf, the following conditions must begin
satisfied:
\begin{enumerate}
  \item \( f(x) \geq 0 \) for any \( x \in (-\infty,\infty) \).
  \item \( \int_{-\infty}^{\infty}f(x)\diff{x} = 1 \)
\end{enumerate}
A continuous random variable \( X \) is said to have a uniform distribution on
the interval \( [A,B] \) if the pdf of \( X \) is:
\[ f(x;A,B) =
  \begin{cases}
    \frac{1}{B-A} &, A\leq x\leq B \\
    0 &, \mathrm{otherwise}
  \end{cases}
\]

\subsection*{Example}
The direction of an imperfection with respect to a reference line on a
circular object such as tires, brake rotors, etc. is in general subject to
uncertainty. Consider the reference line connected the valve stem on a tire rim
to the center point. Let \( X \) be the angle measured clockwise to the location
of the imperfection. One possible pdf for \( X \) is:
\[ f(x) =
  \begin{cases}
    \frac{1}{360} &, 0 \leq x < 360 \\
    0 &, otherwise
  \end{cases}
\]
Find the probability that there is an imperfection between the \( 45^{\circ} \)
and the \( 90^{\circ} \) region on the rim.
\begin{align*}
  P(45\leq X\leq 90) &= \int_{45}^{90}\frac{1}{360}\diff{y} \\
  &= \frac{1}{360}[y]_{y=45}^{y=90} \\
  &= \frac{1}{360}(90-45) \\
  &= \frac{45}{360} \\
  &= \frac{1}{8}
\end{align*}

\subsection*{Cumulative Distribution Function}
The cumulative distribution function \( F(x) \) for a continuous random variable
\( x \) is defined for every number by:
\[ F(x) = P(X\leq x) = \int_{-\infty}^{\infty}f(s)\diff{s} \]

\subsection*{Proposition}
Let \( X \) be a continuous random variable with pdf \( f(x) \) and cdf
\( F(x) \). Then for any number \( a \):
\begin{align*}
  P(X>a) &= 1-P(X\leq a) \\
  &= 1-\int_{-\infty}^{a}f(s)\diff{s} \\
  &= 1-F(a)
\end{align*}
For any numbers \( a \) and \( b \) with \( a \leq b \):
\begin{align*}
  P(a\leq X\leq b) &= \int_{a}^{b}f(s)\diff{s} \\
  &= \int_{-\infty}^{a}f(s)\diff{s}+\int_{a}^{b}f(s)\diff{s}-
    \int_{-\infty}^{a}f(s)\diff{s} \\
  &= \int_{-\infty}^{b}f(s)\diff{s}-\int_{-\infty}^{a}f(s)\diff{s} \\
  &= F(b)-F(a)
\end{align*}

\begin{center}
  If you have any questions, comments, or concerns, please contact me at
  alvin@omgimanerd.tech
\end{center}

\end{document}
