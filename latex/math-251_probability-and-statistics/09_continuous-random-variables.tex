\documentclass[letterpaper, 12pt]{math}

\title{Probability and Statistics}
\author{Alvin Lin}
\date{Probability and Statistics: January 2017 - May 2017}

\begin{document}

\maketitle

\section*{Continuous Random Variables}
Let \( X \) be a continuous random variable. The probability distribution or
probability density function (pdf) of \( X \) is a function \( f(x) \) such that
for any two numbers \( a \) and \( b \) with \( a \leq b \).
\[ P(a\leq X\leq b) = \int_{a}^{b}f(x)\diff{x} \]
For a function \( f(x) \) to be a valid pdf, the following conditions must begin
satisfied:
\begin{enumerate}
  \item \( f(x) \geq 0 \) for any \( x \in (-\infty,\infty) \).
  \item \( \int_{-\infty}^{\infty}f(x)\diff{x} = 1 \)
\end{enumerate}
A continuous random variable \( X \) is said to have a uniform distribution on
the interval \( [A,B] \) if the pdf of \( X \) is:
\[ f(x;A,B) =
  \begin{cases}
    \frac{1}{B-A} &, A\leq x\leq B \\
    0 &, \mathrm{otherwise}
  \end{cases}
\]

\subsection*{Example}
The direction of an imperfection with respect to a reference line on a
circular object such as tires, brake rotors, etc. is in general subject to
uncertainty. Consider the reference line connected the valve stem on a tire rim
to the center point. Let \( X \) be the angle measured clockwise to the location
of the imperfection. One possible pdf for \( X \) is:
\[ f(x) =
  \begin{cases}
    \frac{1}{360} &, 0 \leq x < 360 \\
    0 &, otherwise
  \end{cases}
\]
Find the probability that there is an imperfection between the \( 45^{\circ} \)
and the \( 90^{\circ} \) region on the rim.
\begin{align*}
  P(45\leq X\leq 90) &= \int_{45}^{90}\frac{1}{360}\diff{y} \\
  &= \frac{1}{360}[y]_{y=45}^{y=90} \\
  &= \frac{1}{360}(90-45) \\
  &= \frac{45}{360} \\
  &= \frac{1}{8}
\end{align*}

\subsection*{Cumulative Distribution Function}
The cumulative distribution function \( F(x) \) for a continuous random variable
\( x \) is defined for every number by:
\[ F(x) = P(X\leq x) = \int_{-\infty}^{\infty}f(s)\diff{s} \]

\subsection*{Proposition}
Let \( X \) be a continuous random variable with pdf \( f(x) \) and cdf
\( F(x) \). Then for any number \( a \):
\begin{align*}
  P(X>a) &= 1-P(X\leq a) \\
  &= 1-\int_{-\infty}^{a}f(s)\diff{s} \\
  &= 1-F(a)
\end{align*}
For any numbers \( a \) and \( b \) with \( a \leq b \):
\begin{align*}
  P(a\leq X\leq b) &= \int_{a}^{b}f(s)\diff{s} \\
  &= \int_{-\infty}^{a}f(s)\diff{s}+\int_{a}^{b}f(s)\diff{s}-
    \int_{-\infty}^{a}f(s)\diff{s} \\
  &= \int_{-\infty}^{b}f(s)\diff{s}-\int_{-\infty}^{a}f(s)\diff{s} \\
  &= F(b)-F(a)
\end{align*}

\subsection*{Example}
Let \( p \) be a number between 0 and 1. The \( (100p) \)th percentile of the
distribution of a continuous random variable \( X \), denoted by \( \eta(p) \),
is defined by:
\[ p = P(x\leq\eta(p)) = F(\eta(p)) = \int_{-\infty}^{\eta(p)}f(x)\diff{x} \]
Using the previous tire rim example, find \( \eta(0.9) \), i.e. the 90th
percentile.
\begin{align*}
  p &= P(x\leq\eta(p)) \\
  0.9 &= P(X\leq\eta(0.9)) \\
  &= F(\eta(0.9)) \\
  &= \int_{-\infty}^{\eta(0.9)}f(x)\diff{x} \\
  &= \int_{0}^{\eta(0.9)}\frac{1}{360}\diff{x} \\
  &= \frac{1}{360}[x]_{x=0}^{x=\eta(0.9)} \\
  &= \frac{1}{360}[\eta(0.9)-0] \\
  &= \frac{\eta(0.0)}{360} \\
  \eta(0.9) &= (0.9)(360)
\end{align*}
The median of a continuous distribution, denoted by \( \tilde{\mu} \), is
the 50th percentile and also has special importance.
\[ \tilde{\mu} = \eta(0.5) \]
\[ 0.5 = F(\eta(0.5)) = F(\tilde{\mu}) =
   \int_{-\infty}^{\tilde{\mu}}f(x)\diff{x} \]

\subsection*{Expected Values and Variance}
The expected values and variance of a continouous random variable are similar
to those of a discrete random variable. Instead of \( \sum \), we use
\( \int \). \\
Expected value of \( X \):
\[ E(X) = \int_{-\infty}^{\infty}xf(x)\diff{x} \]
\begin{center}
where \( f(x) \) is the pdf of \( X \).
\end{center}
Expected value of \( Y = h(X) \):
\[ E(Y) = E(h(X)) = \int_{-\infty}^{\infty}h(x)f(x)\diff{x} \]
Variance of \( X \):
\[ V(X) = \int_{-\infty}^{\infty}(x-E(X))^{2}f(x)\diff{x} \]
Standard Deviation of \( X \):
\[ \sigma_{x} = \sqrt{V(X)} \]

\subsection*{Example}
Using the previous tire rim example:
\begin{align*}
  E(X) &= \int_{-\infty}^{\infty}f(x)\diff{x} \\
  &= \int_{0}^{360}x\frac{1}{360}\diff{x} \\
  &= \frac{1}{360}\frac{1}{2}[x^{2}]_{x=0}^{x=360} \\
  &= \frac{1}{2}\frac{1}{360}[360^{2}-0^{2}] \\
  &= \frac{1}{2}\frac{1}{360}360^{2} \\
  &= \frac{1}{2}360 = 180 \\
  V(X) &= \int_{-\infty}^{\infty}(x-E(X))^{2}f(x)\diff{x} \\
  &= \int_{-\infty}^{\infty}(x-180)^{2}f(x)\diff{x} \\
  &= \int_{0}^{360}(x-180)^{2}f(x)\diff{x} \\
  &= \dots
\end{align*}
We can also use the proposition that \( V(X) = E(X^{2})-[E(X)]^{2} \):
\begin{align*}
  E(X^{2}) &= \int_{-\infty}^{\infty}x^{2}f(x)\diff{x} \\
  &= \int_{0}^{360}x^{2}f(x)\diff{x} \\
  &= \int_{0}^{360}x^{2}\frac{1}{360}\diff{x} \\
  &= \frac{1}{360}\int_{0}^{360}x^{2}\diff{x} \\
  &= \frac{1}{360}\frac{1}{3}[x^{3}]_{x=0}^{x-360} \\
  &= \frac{1}{3}\frac{1}{360}[360^{3}-0^{3}] \\
  &= \frac{1}{3}360^{2} \\
  V(X) &= E(X^{2})-[E(X)]^{2} \\
  &= \frac{360^{2}}{3}-180^{2} \\
  &= \frac{1}{3}2^{2}180^{2}-180^{2} \\
  &= 180^{2}[\frac{4}{3}-1] \\
  &= \frac{180^{2}}{3} \\
  \sigma_{x} &= \sqrt{V(X)} \\
  &= \frac{180}{\sqrt{3}} \\
  &\approx 104
\end{align*}

\subsection*{Discrete vs Continuous Median}
Given discrete quiz scores 6, 7, and 9, the median score is 7. For continuous
values, the generation property of median \( \tilde{\mu} \) is:
\[ \tilde{\mu} = \eta(0.5) \]
A continuous random variable \( X \) is said to have a normal distribution
with parameters \( \mu \) and \( \sigma \) (or \( \mu \) and \( \sigma_{2} \)),
where:
\[ -\infty < \mu < \infty \]
\[ \mu\in(-\infty,\infty) \]
\[ \mu\in\R \]
\[ \mu\ \mathrm{is\ a\ real\ number} \]
and \( \sigma > 0 \), if the pdf of \( X \) is:
\[ f(x;\mu,\sigma) =
   \frac{1}{\sqrt{2\pi\sigma}}\e^{-(x-\mu)^{2}/(2\sigma)^{2}} \]
\begin{center}
  for any \( x\in\R \).
\end{center}

\begin{center}
  You can find all my notes at \url{http://omgimanerd.tech/notes}. If you have
  any questions, comments, or concerns, please contact me at
  alvin@omgimanerd.tech
\end{center}

\end{document}
