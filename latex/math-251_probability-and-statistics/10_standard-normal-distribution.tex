\documentclass{math}

\title{Probability and Statistics}
\author{Alvin Lin}
\date{Probability and Statistics: January 2017 - May 2017}

\begin{document}

\maketitle

\section*{Standard Normal Distribution}
\textbf{Standard Normal Distribution} is the normal distribution within
\( \mu = 0 \) and \( \sigma = 1 \). The pdf of a standard normal random
variable \( Z \) is:
\[ f(\delta;0,\sigma) = \frac{1}{\sqrt{2\pi}}\e^{-\frac{\delta^{2}}{2}} \]
The cdf of \( Z \) is:
\begin{align*}
  \Phi(\delta) &= P(Z\leq\delta) \\
  &= \int_{-\infty}^{\delta}f(y;0,1)\diff{y} \\
  &= \int_{-\infty}^{\delta}\frac{1}{\sqrt{2\pi}}\e^{-\frac{y^{2}}{2}}\diff{y}
\end{align*}

\subsection*{Example}
Find \( P(Z\leq 1.25) \), where \( Z \) is the standard normal random variable.
\begin{align*}
  P(Z\leq 1.25) &= \int_{\infty}^{1.25}\frac{1}{\sqrt{2\pi}}
    \e^{-\frac{y^{2}}{2}}\diff{y} \\
  &= \Phi(1.25) \\
  &\approx 0.8944
\end{align*}

\subsection*{Example}
Find \( \eta(0.99) \), 99th percentile, for the standard normal variable
\( Z \).
\begin{align*}
  0.99 &= \int_{-\infty}^{\eta(0.99)}f(\delta;0,1)\diff{\delta} \\
  &= \Phi(\eta(0.99)) \\
  \eta(0.99) &= 2.33 \\
\end{align*}

\subsection*{The \( Z_{\alpha} \)}
\( Z_{\alpha} \) denotes the value on the z axis for which the \( \alpha \) of
the area under the z curve lies to the right of \( Z_{\alpha} \). Example:
\begin{align*}
  Z_{0.1} &= \eta(0.9) \\
  0.9 &= \Phi(Z_{0.1}) \\
  &= \int_{-\infty}^{\eta(0.9)}\frac{1}{\sqrt{2\pi}}
    \e^{-\frac{y^{2}}{2}}\diff{y} \\
  Z_{0.1} &= 1.28
\end{align*}

\subsection*{Relationship between normal distribution and standard normal
             distribution}
Let \( X \) be the normal random variable pdf \( f(x;\mu,\sigma) \), and
\( Z \) be the standard normal random variable (with pdf \( f(\delta;0,1) \)).
Derivation:
\begin{align*}
  F(x;\mu,\sigma) &= \int_{-\infty}^{x}\frac{1}{\sqrt{2\pi\sigma}}
    \e^{-\frac{(y-\mu)^{2}}{2\sigma^{2}}}\diff{y} \\
  \Phi(\delta) &= \int_{-\infty}^{\delta}\frac{1}{\sqrt{2\pi}}
    \e^{-\frac{y^{2}}{2}}\diff{y} \\
  Z &= \frac{X-\mu}{\sigma} \\
  P(a\leq X\leq b) &= P(\frac{a-\mu}{\sigma}\leq Z\leq\frac{b-\mu}{\sigma}) \\
  &= P(Z\leq\frac{b-\mu}{\sigma})-P(Z\leq\frac{a-\mu}{\sigma}) \\
  &= \Phi(\frac{b-\mu}{\sigma})-\Phi(\frac{a-\mu}{\sigma}) \\
  P(X\leq a) &= P(Z\leq\frac{a-\mu}{\sigma}) \\
  P(X\geq b) &= 1-P(X\leq b) \\
  &= 1-P(Z\leq\frac{b-\mu}{\sigma}) \\
  &= 1-\Phi(\frac{b-\mu}{\sigma})
\end{align*}

\subsection*{Example}
The breakdown voltage of a randomly chosen diode of a particular type is known
to be normally distributed. Find the probability that a diode's breakdown
voltage is within 1 standard deviation of its mean value. Experiment: select
a diode and measure its breakdown voltage. X: the measured breakdown voltage.
\[ \mu = E(X) \]
\[ \sigma = \sqrt{V(X)} \]
\begin{align*}
  P(\mu-\sigma\leq X\leq\mu+\sigma) &= \int_{\mu-\sigma}^{\mu+\sigma}
    f(x;\mu,\sigma)\diff{x} \\
  &= \int_{\mu-\sigma}^{\mu+\sigma}\frac{1}{\sqrt{2\pi\sigma}}
    \e^{-\frac{(x-\mu)^{2}}{2\sigma^{2}}}\diff{x} \\
  &= P(\frac{(\mu-\sigma)-\mu}{\sigma}\leq
    Z\leq\frac{(\mu+\sigma)-\mu}{\sigma}) \\
  &= P(-1\leq Z\leq1) \\
  &= \Phi(1)-\Phi(-1) \\
  &= 0.8413-0.1587
\end{align*}

\subsection*{Relationship between two percentiles}
(100p)th percentile for normal \( (\mu,\sigma) \):
\[ = \mu +\big[\mathrm{(100p)th\ percentile\ for\ standard\ normal}\big]
   \times\sigma \]

\subsection*{Example}
The temperature reading from a thermocouple placed in a constant-temperature
medium is normally distributed with mean \( \mu \), the actual temperature
of the medium, and standard deviation \( \sigma \). What would the value of
\( \sigma \) have to be to ensure that 95\% of all readings are within
\( .1^{\circ} \) of \( \mu \)? \\
Random variable \( X \) that has a normal distribution with \( E(X) = \mu \)
and standard deviation \( \sqrt{V(X)} = \sigma \).
\begin{align*}
  0.95 &= P(\mu-0.1\leq X\leq\mu+0.1) \\
  &= P(\frac{(\mu-0.1)-\mu}{\sigma}\leq Z\leq\frac{(\mu+0.1)-\mu}{\sigma}) \\
  &= P(-\frac{0.1}{\sigma}\leq Z\leq\frac{0.1}{\sigma}) \\
  \Phi(\frac{0.1}{\sigma}) &= 0.975 \\
  \frac{0.1}{\sigma} &= 1.96 \\
  \sigma &= \frac{0.1}{1.96}
\end{align*}

\begin{center}
  You can find all my notes at \url{http://omgimanerd.tech/notes}. If you have
  any questions, comments, or concerns, please contact me at
  alvin@omgimanerd.tech
\end{center}

\end{document}
