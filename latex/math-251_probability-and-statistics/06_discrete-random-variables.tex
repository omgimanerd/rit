\documentclass[letterpaper, 12pt]{math}

\usepackage{amsmath}
\usepackage{amssymb}
\usepackage[linguistics]{forest}

\title{Probability and Statistics}
\author{Alvin Lin}
\date{Probability and Statistics: January 2017 - May 2017}

\begin{document}

\maketitle

\section*{Discrete Random Variables}
A discrete random variable is a random variable whose possible values either
constitute a finite set or else can be listed in an infinite sequence in which
there is a first element, a second element, and so on (a countably infinite
set). A countable set is finite or countably infinite.
\begin{itemize}
  \item Finite sets:
    \[ \bigg\{ 1 \bigg\} \]
    \[ \bigg\{ A,B,E \bigg\} \]
  \item Countably infinite sets:
    \[ \bigg\{ 2,4,6,8,\dots \bigg\} \]
    \[ \bigg\{ 1,\frac{1}{2},\frac{1}{3},\frac{1}{4},\dots \bigg\} \]
    \[ \bigg\{ \dots,-3,-2,-1,0,1,2,3,\dots \bigg\} \]
    \[ \bigg\{ (m,n)\in\Z\times\Z\ |\ m\geq 0,n\geq 0 \bigg\} \]
\end{itemize}
A random variable is continuous if both of the following apply:
\begin{enumerate}
  \item Its set of possible values consists either of all members in a single
    interval on the number line, possibly infinite in extent (eg:
    \( (-\infty,\infty),(0,\infty),\dots \)) or all numbers in a disjoint
    union of such intervals.
\end{enumerate}

\subsection*{Probability Mass Function}
There are two balls marked 1 and 3, respectively. Select a ball twice with
replacement.
\begin{center}
  \begin{forest}
    [
      [1 [1 (sum: 2)] [3 (sum: 4)]]
      [3 [1 (sum: 4)] [3 (sum: 6)]]
    ]
  \end{forest}
\end{center}
What is the probability that the sum is 4?
\[ \frac{2}{4} = \frac{1}{2} \]
Let \( X \) be the random variable indicating the value of the sum of the two
numbers.
\begin{align*}
  p(4) &= P(X=4) \\
  &= P(\{\omega\in S\ |\ X(\omega)=4\}) \\
  &= P(\{(1,3),(3,1)\}) \\
  &= \frac{2}{4}
\end{align*}
\( p(X=4) \) is the probability mass function (pmf), a function that gives the
probability that a random variable is equal to some exact number.

\subsection*{Example}
Using the same problem as before, the selections are independent and the
probability that we get a 1 in a selection is \( \alpha \).
\begin{center}
  \begin{forest}
    [
      [1 \( (\alpha) \) [1 \( (\alpha) \)] [3 \( (1-\alpha) \)]]
      [3 \( (1-\alpha) \) [1 \( (\alpha) \)] [3 \( (1-\alpha) \)]]
    ]
  \end{forest}
\end{center}
\begin{align*}
  p(2;\alpha) &= P(X=2) \\
  &= \alpha^{2} \\
  p(4;\alpha) &= P(X=4) \\
  &= \alpha(1-\alpha)+(1\alpha)\alpha \\
  &= 2\alpha(1-\alpha) \\
  p(6;\alpha) &= (1-\alpha)^{2}
\end{align*}
\( \alpha \) can take various values. The collection of \( p(x;\alpha) \)
\[ (p(x;\frac{1}{2}), p(x;\frac{1}{6}), \dots) \]
is a family of probability distributions.

\subsection*{Cumulative Distribution Function}
The cumulative distribution function (cdf) \( F(x) \) of a discrete random
variable \( X \) with pmf \( p(x) \) is defined for every number \( x \) by:
\[ F(x) = P(X\leq x) = \sum_{y;y\leq x}p(y) \]

\subsection*{Example}
Using the same problem as before:
\begin{align*}
  F(2) &= P(X\leq 2) \\
  &= \frac{1}{4} \\
  &= \sum_{y\leq 2}p(y) \\
  &= p(2) \\
  &= \frac{1}{4} \\
  F(4) &= P(X\leq 4) \\
  &= \sum_{y\leq 4}p(y) \\
  &= p(2)+p(4) \\
  &= \frac{1}{4}+\frac{2}{4} \\
  &= \frac{3}{4} \\
  F(6) &= P(X\leq 6) \\
  &= \sum_{y\leq 6}p(y) \\
  &= p(2)+p(4)+p(6) \\
  &= \frac{1}{4}+\frac{2}{4}+\frac{1}{4} \\
  &= 1
\end{align*}
\begin{align*}
  P(4\leq X\leq 6) &= \frac{3}{4} \\
  &= F(6)-F(2) \\
  &= (\frac{1}{4}+\frac{2}{4}+\frac{1}{4})-(\frac{1}{4}) \\
  &= \frac{3}{4} \\
  P(2\leq X\leq 4) &= \frac{3}{4} \\
  &= F(4) = P(X\leq 4) \\
  &= \frac{3}{4}
\end{align*}
For any two numbers \( a \) and \( b \) with \( a \leq b \):
\[ P(a\leq X\leq b) = F(b)-F(a-) \]
where \( a- \) represents the largest possible \( X \) value that is strictly
less than \( a \). Notation:
\begin{align*}
  probability \quad & P(event\in S) \\
  pmf\ of\ a\ random\ variable \quad & p(a\ number) \\
  cdf\ of\ a\ random\ variable \quad & F(a\ number)
\end{align*}

\subsection*{Expected Values}
Let \( X \) be a discrete random variable with set of possible values \( D \)
and pmf \( p(x) \). The expected value or mean value of \( X \) is:
\[ E(X) = \mu_{x} = \mu = \sum_{x\in D}xp(x) \]
If the random variable \( X \) has a set of possible values \( D \) and
pmf \( p(x) \), then the expected value of any function \( h(X) \) is:
\[ E[h(x)] = \sum_{x\in D}h(x)p(x) \]
If \( h(X) \) is of very special type \( h(X) = aX+b \), where \( a \) and
\( b \) are constants:
\[ E[h(x)] = E(aX+b) = \mu_{ax+b} = aE(x)+b \]

\subsection*{Example}
Extending from the same problem:
\begin{align*}
  E(X) &= \mu_{x} \\
  &= \sum_{x\in D}xp(x) \\
  &= 2p(2)+4p(4)+6p(6) \\
  &= 2\frac{1}{4}+4\frac{2}{4}+6\frac{1}{4} \\
  &= \frac{2+8+6}{4} \\
  &= 4 \\
  E(\e^{X}) &= \sum_{x\in D}\e^{x}p(x) \\
  &= \e^{2}p(2)+\e^{4}p(4)+\e^{6}p(6) \\
  &= \e^{2}\frac{1}{4}+\e^{4}\frac{2}{4}+\e^{6}\frac{1}{4} \\
  &= \frac{1}{4}[e^{2}+2e^{4}+\e^{6}] \\
  E(10X-20) &= (10\times2-20)p(2)+(10\times4-20)p(4)+(10\times6-20)p(6) \\
  &= 10E(X)-20 \\
  &= 10\times4-20 \\
  &= 20
\end{align*}

\subsection*{Variance and Standard Deviation}
Let \( X \) have pmf \( p(x) \) and expected value \( \mu \). The variance of
\( X \) is:
\[ V(X) = \sigma_{x}^{2} = \sigma^{2} = \sum_{x\in D}(x-\mu)^{2}p(x) =
   E((x-\mu)^{2}) \]
The standard deviation of \( X \) is:
\[ \sigma_{x} = \sigma = \sqrt{V(X)} = \sqrt{\sigma_{x}^{2}} \]

\subsection*{Example}
Extending from the same problem:
\begin{align*}
  V(X) &= (2-4)^{2}p(2)+(4-4)^{2}p(4)+(6-4)^{2}p(6) \\
  &= 2^{2}\frac{1}{4}+0^{2}\frac{2}{4}+2^{2}\frac{1}{4} \\
  &= 2 \\
  V(10X-20) &= 10^{2}V(X) \\
  &= 100\times2 \\
  &= 200
\end{align*}

\begin{center}
  If you have any questions, comments, or concerns, please contact me at
  alvin@omgimanerd.tech
\end{center}

\end{document}
