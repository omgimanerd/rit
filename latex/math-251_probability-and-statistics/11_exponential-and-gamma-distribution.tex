\documentclass[letterpaper, 12pt]{math}

\title{Probability and Statistics}
\author{Alvin Lin}
\date{Probability and Statistics: January 2017 - May 2017}

\begin{document}

\maketitle

\section*{Exponential and Gamma Distributions}
A random variable \( X \) is said to have an exponential distribution with
(scale) parameter \( \lambda (\lambda>0) \) if the pdf of \( X \) is:
\[ f(x;\lambda) =
  \begin{cases}
    \lambda\e^{-\lambda x} &, x\geq 0 \\
    0 &, otherwise
  \end{cases}
\]
The cdf of \( X \) is:
\[ F(x;\lambda) =
  \begin{cases}
    1-\e^{-\lambda x} &, x\geq 0 \\
    0 &, otherwise
  \end{cases}
\]

The derivative with respect to \( x \) of the cdf is equal to the pdf while the
parameter \( \lambda \) is fixed.
\begin{align*}
  F(x;\lambda) &= P(X\leq x) \\
  &= \int_{-\infty}^{x}f(t;\lambda)\diff{t} \\
  F'(x;\lambda) &= f(x;\lambda)
\end{align*}

\subsection*{Expected Values and Variance}
Expected value of \( X \):
\begin{align*}
  E(X) &= \int_{-\infty}^{\infty}xf(x;\lambda)\diff{x} \\
  &= \int_{0}^{\infty}x\lambda\e^{-\lambda x}\diff{x} \\
  &= \frac{1}{\lambda}
\end{align*}
Variance of \( X \):
\begin{align*}
  V(X) &= \int_{-\infty}^{\infty}(x-E(X))^{2}f(x;\lambda)\diff{x} \\
  &= \int_{0}^{\infty}(x-\frac{1}{\lambda})^{2}\lambda\e^{-\lambda x}\diff{x} \\
  &= \int_{0}^{\infty}x^{2}
    \lambda\e^{-\lambda x}\diff{x}-\frac{1}{\lambda^{2}} \\
  &= \frac{1}{\lambda^{2}} \\
  V(X) &= E(X^{2})-[E(X)]^{2} \\
  &= \int_{-\infty}^{\infty}x^{2}f(x;\lambda)\diff{x}-(\frac{1}{\lambda})^{2} \\
  &= \frac{1}{\lambda^{2}}
\end{align*}

\subsection*{Relationship between Poisson process and exponential distribution}
Suppose that the number of events occurring in any time interval of length
\( t \) has a Poisson distribution with parameter \( \alpha t \) (where
\( \alpha \), the rate of the event process, is the expected number of events
ocurring in 1 unit of time) and that numbers of occurrences in nonoverlapping
intervals are independent of one another. Then the distribution of elapsed time
between the occurrence of two successive events is exponential with parameter
\( \lambda = \alpha \).

\begin{center}
  You can find all my notes at \url{http://omgimanerd.tech/notes}. If you have
  any questions, comments, or concerns, please contact me at
  alvin@omgimanerd.tech
\end{center}

\end{document}
