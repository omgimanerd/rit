\documentclass[letterpaper, 12pt]{math}

\usepackage{amsmath}
\usepackage{amssymb}
\usepackage[linguistics]{forest}

\title{Probability and Statistics}
\author{Alvin Lin}
\date{Probability and Statistics: January 2017 - May 2017}

\begin{document}

\maketitle

\section*{Binomial Random Variables}
There are two balls marked S and F in a basket. Select a ball 3 times with
replacement. At each trial, S is likely to be chosen with probability \( p \),
and F is likely to be chosen with probability \( 1-p \). Let \( X \) be the
random variable of the experiment indicating the number of times S is chosen.
\begin{center}
  \begin{forest}
    [
      [S \( (p) \)
        [S \( (p) \) [S \( (p) \)] [F \( (1-p) \)]]
        [F \( (1-p) \) [S \( (p) \)] [F \( (1-p) \)]]
      ]
      [F \( (1-p) \)
        [S \( (p) \) [S \( (p) \)] [F \( (1-p) \)]]
        [F \( (1-p) \) [S \( (p) \)] [F \( (1-p) \)]]
      ]
    ]
  \end{forest}
  \begin{tabular}{|c|c|c|}
    \hline
        & \( X = 1 \) & \( X = 2 \) \\ \hline
    SSS &             & \\ \hline
    SSF &             & \checkmark \\ \hline
    SFS &             & \checkmark \\ \hline
    SFF & \checkmark  & \\ \hline
    FSS &             & \checkmark \\ \hline
    FSF & \checkmark  & \\ \hline
    FFS & \checkmark  & \\ \hline
    FFF &             & \\ \hline
  \end{tabular}
\end{center}
Find the probability mass function of \( X \), \( b(x;3,p) \):
\[ b(0;3,p) = P(X=0) = (1-p)^{3} \]
The underlying assumption is on the independence of the events:
\begin{itemize}
  \item \( A_{1} \): getting an S in the first trial
  \item \( A_{2} \): getting an S in the second trial
  \item \( A_{3} \): getting an S in the third trial
\end{itemize}
\( A_{1} \), \( A_{2} \), and \( A_{3} \) are mutually independent.
\begin{align*}
  b(0;3,p) &= P(X=0) \\
  &= (1-p)(1-p)(1-p) \\
  &= \binom{3}{0}p^{0}(1-p)^{3-0} \\
  b(1;3,p) &= p(1-p)(1-p)+(1-p)p(1-p)+(1-p)(1-p)p \\
  &= \binom{3}{1}p^{1}(1-p)^{3-1}
\end{align*}
From 3 distinct items (trial 1, trial 2, trial 3), select 1 item. There are
\( \nCr{3}{1} \) possible combinations.
\begin{align*}
  b(2;3,p) &= P(X=2) \\
  &= (p)(p)(1-p)+p(1-p)p+(1-p)(p)(p) \\
  &= \binom{3}{2}p^{2}(1-p)^{3-2} \\
  b(x;3,p) &=
  \begin{cases}
    \binom{3}{x}p^{x}(1-p)^{3-x} &, x = 0,1,2,\dots,n \\
    0 &, otherwise
  \end{cases}
\end{align*}
The above example is an example of a binomial experiment with a binomial random
variable.
\begin{enumerate}
  \item This experiment consists of a sequence of \( n \) smaller experiments
    called trials, where \( n \) is fixed in advance of the experiment.
  \item Each trial can result in one of the two possible outcomes (dichotomous
    trials), which we generically denote by \( success(S) \) or
    \( failure(S) \). The assignment of the \( S \) and \( F \) labels to the
    two sides of the dichotomy is arbitrary.
  \item The trials are independent, so that the outcome on any particular trial
    does not influence the outcome of any other trial.
  \item The probability of success \( P(S) \) is constant from trial to trial.
    We denote this probability by \( p \).
\end{enumerate}
An experiment for which the above conditions (a fixed number of dichotomous,
independent, homogeneous trials) ar satisfied is called a binomial experiment.

\subsection*{PMF of a binomial random variable X}
The probability mass function of a binomial random variable \( X \) is:
\[ b(x;n,p) =
  \begin{cases}
    \binom{n}{x}p^{x}(1-p)^{n-x} &, x = 1,2,3,\dots,n \\
    0 &, otherwise
  \end{cases}
\]
The value of \( X \) indicates the number of S'es. \par

\subsection*{CDF of a binomial random variable X}
The cumulative distribution function of a binomial random variable \( X \) is:
\begin{align*}
  B(x;n,p) &= P(X\leq x) \\
  &= \sum_{y=0}^{x}b(y;n,p) \\
  & if\ x = 0,1,2,\dots,n
\end{align*}
\( X\sim Bin(n,p) \) denotes that \( X \) is a binomial random variable with
probability mass function b(x;n,p).

\subsection*{Expected value of a binomial random variable X}
\[ E(X) = \sum_{x\in\{0,1,2,\dots,n\}}xb(x;n,p) = np \]

\subsection*{Variance of a binomial random variable X}
\[ V(X) = \sum_{x\in\{0,1,2,\dots,n\}}(x-E(X))^{2}b(x;n,p) = np(1-p) \]
The standard deviation of \( X \) is \( \sigma = \sqrt{V(X)} = \sqrt{np(1-p)} \)

\subsection*{Example}
An aircraft seam requires 25 rivets. The seam will have to be reworked if any
of these rivets are defective. Suppose rivets are defective independently of
one another, each with the same probability. If 15\% of all seams need
reworking, what is the probability that a rivet is defective?
\begin{align*}
  1-0.15 &= P(a\ seam\ does\ not\ need\ reworking) \\
  &= P(a\ seam\ has\ zero\ defective\ rivets) \\
  &= (1-p)^{25} \\
  1-p &= 0.85^{\frac{1}{25}} \\
  p &= 1-0.85^{\frac{1}{25}}
\end{align*}
Find the probability that a randomly selected seam has exactly 3 defective
rivets.
\[ \binom{25}{3}p^{3}(1-p)^{22} \]

\subsection*{Example}
A very large batch of components has arrived at a distributor. The batch
can be characterized as acceptable only if the proportion of defective
components is at most 0.10. The distributor decides to randomly select 10 or
15 components and to accept the batch only if the defective components in the
sample is at most 1 or 2, respectively. \par
Consider a simpler experiment. Select 2 components and inspect them. The
probability that the two are both defective is:
\[ (\frac{k}{N})(\frac{k-1}{N-1}) \]
where \( k \) is the number of defective components in the batch. If \( N \)
and \( k \) are large:
\[ \frac{k-1}{N-1}\sim\frac{k}{N} \]
Consider sampling without replacement from a dichotomous population of size
\( N \). If the sample size (number of trials) \( n \) is at most 5\% of the
population size. The experiment can be analyzed as though it were a binomial
experiment. \par
Trial: select a component from the batch and inspect the component for defects.
\[ p = \frac{k}{N} \]
where \( k \) is the number of defective components in the batch and \( X \) is
the number of defective components in the \( n \) trials. In the real world,
we do not replace a component after inspecting it, but we can approximate it
for the purpose of this experiment.
\begin{align*}
  X &\sim Bin(10,p) \\
  P(X\leq 2) &= \sum_{x=0}^{2}b(x;n,p) \\
  &= \sum_{x=0}{2}\binom{n}{x}p^{x}(1-p)^{n-x}
\end{align*}
\[ Method\ 1:\ n = 10 \]
\begin{center}
  \begin{tabular}{|c|c|c|c|c|c|}
    \hline
    \( p = \frac{k}{N} \) & 0.01 & 0.02 & 0.1 & 0.2 & 0.25 \\
    \hline
    \( P(X\leq 2) \) & 0.999 & 0.9985 & 0.9298 & 0.6778 & 0.5256 \\
    \hline
  \end{tabular}
\end{center}
\[ Method\ 2:\ n = 10 \]
\begin{center}
  \begin{tabular}{|c|c|c|c|c|c|}
    \hline
    \( p = \frac{k}{N} \) & 0.01 & 0.02 & 0.1 & 0.2 & 0.25 \\
    \hline
    \( P(X\leq 1) \) & 0.9957 & 0.9139 & 0.7361 & 0.3758 & 0.2440 \\
    \hline
  \end{tabular}
\end{center}
\[ Method\ 3:\ n = 15 \]
\begin{center}
  \begin{tabular}{|c|c|c|c|c|c|}
    \hline
    \( p = \frac{k}{N} \) & 0.01 & 0.02 & 0.1 & 0.2 & 0.25 \\
    \hline
    \( P(X\leq 2) \) & 0.9996 & 0.9638 & 0.8159 & 0.3980 & 0.2361 \\
    \hline
  \end{tabular}
\end{center}
Which method is the best? Our goal is that the batch is accepted if
\( p \leq 0.10 \) and rejected if \( p > 0.10 \).
\begin{center}
  \begin{tabular}{|c|c|c|c|}
    \hline
    \( p = \frac{k}{N} \) & & 0.10 & \\
    \hline
    \( P(the\ event\ such\ that\ we\ accept\ the\ batch) \) & high & & low \\
    \hline
  \end{tabular}
\end{center}
The third method is the best.

\begin{center}
  If you have any questions, comments, or concerns, please contact me at
  alvin@omgimanerd.tech
\end{center}

\end{document}
