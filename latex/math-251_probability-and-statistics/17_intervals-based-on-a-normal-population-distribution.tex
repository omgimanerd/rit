\documentclass[letterpaper, 12pt]{math}

\title{Probability and Statistics}
\author{Alvin Lin}
\date{Probability and Statistics: January 2017 - May 2017}

\begin{document}

\maketitle

\section*{Intervals Based on a Normal Population Distribution}
Assumption in 7.3:
\begin{itemize}
  \item population: normal
  \item population standard deviation \( \sigma \): unknown
  \item \( n \): no need to be large
\end{itemize}
Review 7.1:
\begin{itemize}
  \item population: normal
  \item population standard deviation \( \sigma \): known
  \item \( n \): no need to be large
  \item confidence interval for \( \mu = E(X_{1}) = E(X_{2})\dots \)
\end{itemize}
Review 7.2:
\begin{itemize}
  \item population: any
  \item population standard deviation \( \sigma \): unknown
  \item \( n \): large
  \item confidence interval for \( \mu = E(X_{1}) \)
\end{itemize}
If the number of repetitions goes to infinity, \( 100(1-\alpha)\% \) of
the intervals contain \( \mu \). Each interval is obtained from observed
values of \( X_{1},X_{2},\dots,X_{n}:x_{1},x_{2},\dots,x_{n} \). Each
interval is calculated from the observed sample proportion \( \frac{x}{n} \)
where \( x \) is the observed number of successes in the sample of size \( n \).
Review 7.3:
\begin{itemize}
  \item population: normal
  \item population standard deviation \( \sigma \): unknown
  \item \( n \): no need to be large
\end{itemize}

\subsection*{Theorem}
\[ T = \frac{\overline{X}-\mu}{S/\sqrt{n}} \]
where \( \overline{X} \) is the random variable for the sample mean, \( \mu \)
is the population mean, \( S \) is the random variable for the sample standard
deviation, and \( n \) is the sample size. This random variable \( T \) has
\( t_{n-1} \) distribution, or a \( t \) distribution with \( n-1 \) degrees of
freedom. \par
\textbf{Proposition}: \( 100(1-\alpha)\% \) confidence interval for \( \mu \)
is:
\[ \overline{x}\pm t_{\alpha/2,n-1}\cdot\frac{s}{\sqrt{n}} \]
where \( \overline{x} \) is the observed sample mean and \( s \) is the observed
sample standard deviation. Upper and lower confidence bounds for \( \mu \) are
\[ \overline{x}+t_{\alpha,n-1}\cdot\frac{s}{\sqrt{n}} \]
\[ \overline{x}-t_{\alpha,n-1}\cdot\frac{s}{\sqrt{n}} \]

\subsection*{Prediction Interval}
\begin{itemize}
  \item population: normal
  \item population standard deviation \( \sigma \): unknown
  \item \( n \): no need to be large
\end{itemize}
A prediction interval (PI) for a single observation to be selected from a normal
population distribution is:
\[ \overline{x}\pm t_{\alpha/2,n-1}\cdot s\sqrt{1+\frac{1}{n}} \]
The confidence level is \( 100(1-\alpha)\% \).

\begin{center}
  You can find all my notes at \url{http://omgimanerd.tech/notes}. If you have
  any questions, comments, or concerns, please contact me at
  alvin@omgimanerd.tech
\end{center}

\end{document}
