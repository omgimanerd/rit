\documentclass{math}

\usepackage{graphicx}

\title{Probability and Statistics}
\author{Alvin Lin}
\date{Probability and Statistics: January 2017 - May 2017}

\begin{document}

\maketitle

\section*{Large Sample Confidence Intervals}
For a population mean and proportion:
\begin{center}
  \begin{tabular}{|c|c|c|c|}
    \hline
    & population distribution type & sample size &
      population standard deviation \\
    \hline
    Section 7.1 & normal & any & known \\
    \hline
    Section 7.2 & any & large & unknown \\
    \hline
  \end{tabular}
\end{center}

\subsection*{Section 7.1}
\[ \overline{x}+z_{\alpha/2}\frac{\sigma}{\sqrt{n}} \]
is a large sample confidence interval for \( \mu \) with confidence level
\[ 100(1-\alpha)\% \]
\( \overline{x} \) represents the observed sample mean, \( \sigma \) represents
the population standard deviation, and \( n \) represents the sample size.

\subsection*{Section 7.2}
Proposition: If \( n \) is sufficiently large:
\[ Z = \frac{\overline{X}-\mu}{S/\sqrt{n}} \]
has approximately a standard normal distribution, with \( \overline{X} \) being
the random variable for the sample mean, \( S \) being the random variable for
the sample standard deviation.

\subsection*{A Confidence Interval for a Population Proportion}
\begin{align*}
  p&: \mathrm{proportion\ of\ success\ in\ a\ population} \\
  n&: \mathrm{sample\ size} \\
  X&: \mathrm{\#\ of\ successes\ in\ a\ sample,\ a\ random\ variable} \\
  x&: \mathrm{observed\ \#\ of\ successes\ in\ a\ sample} \\
  \hat{p}&: \frac{x}{n} \\
  \hat{q}&: 1-\hat{p}
\end{align*}
Assumptions:
\begin{itemize}
  \item Population size is larger than \( n \)
  \item \( np\geq 10 \)
  \item \( n(1-p)\geq 10 \)
\end{itemize}
Proposition:
\[ \scalebox{2}{$
  \tilde{p} = \frac{\hat{p}+\frac{(z_{\alpha/2})^{2}}{2n}}
  {1+\frac{(z_{\alpha/2})^{2}}{n}}
$} \]
The confidence interval for a population proportion \( p \) with confidence
level \( 100(1-\alpha)\% \) is:
\[ \scalebox{2}{$
  \tilde{p}\pm z_{\alpha/2}\frac{\sqrt{
  \frac{\hat{p}\hat{q}}{n}+\frac{(z_{\alpha/2})^{2}}{4n^{2}}
  }}{1+\frac{(z_{\alpha/2})^{2}}{n}}
$} \]
Under certain conditions, the interval is:
\[ \scalebox{2}{$
  \tilde{p}+z_{\alpha/2}\sqrt{\frac{\hat{p}\hat{q}}{n}}
$} \]

\begin{center}
  You can find all my notes at \url{http://omgimanerd.tech/notes}. If you have
  any questions, comments, or concerns, please contact me at
  alvin@omgimanerd.tech
\end{center}

\end{document}
