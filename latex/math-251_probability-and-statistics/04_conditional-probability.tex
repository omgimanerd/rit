\documentclass[letterpaper, 12pt]{math}

\usepackage{amsmath}
\usepackage{amssymb}

\title{Probability and Statistics}
\author{Alvin Lin}
\date{Probability and Statistics: January 2017 - May 2017}

\begin{document}

\maketitle

\section*{Conditional Probability}
Let \( A \) and \( B \) be events of an experiment. The conditional probability
of \( A \) given that \( B \) has occurred is:
\[ P(A|B) = \frac{P(A \cap B)}{P(B)} \]

\subsection*{Idea Behind This Formula}
Roll a die:
\begin{align*}
  A&: getting\ a\ 2 \\
  B&: getting\ an\ even\ number \\
\end{align*}
\[ P(A) = \frac{\#\ of\ outcomes\ favorable\ to\ A}{total\ \#\ of\ outcomes} \]
(Under the assumption that the outcomes are equally likely) \\

Let's consider the probability of \( A \) given that \( B \) has occurred.
Suppose the dice have detectors on the faces 2, 4, and 6. After we roll the
dice, we can detect whether the dice has landed on a 2, 4, or 6, but we will
not know which of the even numbers it has landed on.
\[ P(A|B) = \frac{\#\ of\ outcomes\ favorable\ to\ A\ and\ in\ B}
   {total\ \#\ of\ outcomes\ in\ B} \]
\[ P(A|B) = \frac{1}{3} \]
Considering the situation in the previous example, find \( P(B|A) \).
\[ P(B|A) = \frac{1}{1} = 1 \]
If \( P(B) > 0 \), the conditional probability of \( A \) given that \( B \)
has occurred is \( P(A|B) = \frac{P(A \cap B)}{P(B)} \). Thus it follows
that:
\[ P(A \cap B) \times P(B) = P(A|B) \]
This holds true when \( P(B) = 0 \) as well.

\end{document}
