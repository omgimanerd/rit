\documentclass[letterpaper, 12pt]{math}

\usepackage{amsmath}
\usepackage{amssymb}
\usepackage{pgfplots}

\title{Probability and Statistics}
\author{Alvin Lin}
\date{Probability and Statistics: January 2017 - May 2017}

\begin{document}

\maketitle

\section{Probability}
The sample space of an experiment, denoted by \( S \), is the set of all
possible outcomes of the experiment. An event of the experiment is a subset
of \( S \). An event is simple if it consists of exactly one outcome, and
compound if it consists of more than one outcome. \par
Let \( A \) and \( B \) be events of an experiment. If \( A = \emptyset \),
then it is said to be the null event. \par
If \( A \cap B = \emptyset \), then \( A \) and \( B \) are said to be mutually
exclusive, or disjoint.

\subsection{Roll a Die}
\[ S = \big\{ 1, 2, 3, 4, 5, 6 \big\} \]
Sample events:
\begin{itemize}
  \item \( E_{1}: \) getting a 1
  \item \( E_{2}: \) getting an even number
  \item \( E_{3}: \) getting a number greater than 4
  \item \( E_{4}: \) getting a number greater than 9
  \item \( E_{5}: \) getting an integer greater than 0 and less than 7
\end{itemize}
\begin{align*}
  E_{1} &= \big\{ 1 \big\} \subset S \\
  E_{2} &= \big\{ 2, 4, 6, \big\} \subset S \\
  E_{3} &= \big\{ 5, 6 \big\} \subset S \\
  E_{4} &= \big\{ \big\} = \emptyset \subset S \\
  E_{5} &= \big\{ 1, 2, 3, 4, 5, 6 \big\} \subset S
\end{align*}
\( E_{1} \) is a simple event. \( E_{2}, E_{3}, E_{5} \) are compound events.
\( E_{4} \) is the null event. \par
The events \( E_{1} \) and \( E_{2} \) are mutually exclusive or disjoint.
\[ E_{1} \cap E_{2} = \emptyset \]
The events \( E_{2} \) and \( E_{3} \) are not mutually exclusive or disjoint.
\[ E_{2} \cap E_{3} = \big\{ 6 \big\} \neq \emptyset \]

\end{document}
