\documentclass[letterpaper, 12pt]{math}

\usepackage{amsmath}
\usepackage{amssymb}
\usepackage[linguistics]{forest}

\title{Probability and Statistics}
\author{Alvin Lin}
\date{Probability and Statistics: January 2017 - May 2017}

\begin{document}

\maketitle

\section*{The Poisson Probability Distribution}
Recall binomial random variables and its probability mass function.
\[ b(x;n,p) =
  \begin{cases}
    \binom{n}{x}p^{x}(1-p)^{n-x} &, x = 0,1,2,\dots,n \\
    0 &, otherwise
  \end{cases}
\]
As \( n\to\infty \) and \( p\to 0 \), \( np\to\mu > 0 \).
\[ b(x;n,p)\to p(x;\mu) \]

\subsection*{Poisson Distribution}
\[ p(x;\mu) =
  \begin{cases}
    \frac{\e^{-\mu}\mu^{x}}{x!} &, x = 0,1,2,\dots \\
    0 &, otherwise
  \end{cases}
\]
The Poisson model is a reasonably good approximation of the binomial model
when \( n \geq 20 \) with \( p \leq 0.05 \) or \( n \geq 100 \) with
\( p \leq 0.10 \).
\begin{center}
  \begin{tabular}{|c|c|}
    \hline
    binomial & Poisson \\
    \hline
    \( E(X) = np \) & \( E(X) = \mu \) \\
    \hline
    \multicolumn{2}{|c|}{\( np \sim \mu \)} \\
    \hline
    \( V(X) = np(1-p) \) & \( V(X) = \mu \) \\
    \hline
    \multicolumn{2}{|c|}{\( np(1-p) \sim np \sim \mu \)} \\
    \hline
  \end{tabular}
\end{center}
\begin{align*}
  \sum_{x=0}^{\infty}p(x;\mu) &=
    \sum_{x=0}^{\infty}\frac{\e^{-\mu}\mu^{x}}{x!} \\
  &= \e^{-\mu}\sum_{x=0}^{\infty}\frac{\mu^{x}}{x!} \\
  &= \e^{-\mu}\e^{\mu} \\
  &= \e^{-\mu+\mu} \\
  &= \e^{0} \\
  &= 1
\end{align*}

\end{document}
