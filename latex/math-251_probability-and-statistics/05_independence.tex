\documentclass[letterpaper, 12pt]{math}

\usepackage{amsmath}
\usepackage{amssymb}
\usepackage[linguistics]{forest}

\title{Probability and Statistics}
\author{Alvin Lin}
\date{Probability and Statistics: January 2017 - May 2017}

\begin{document}

\maketitle

\section*{Independence}
Two events \( A \) and \( B \) are independent if \( P(A|B) = P(A) \), and are
dependent otherwise. For example, with the case of rolling a die twice, the
probability of rolling a 2 the second time given that a 2 occurred in the first
roll is equal to the probability of rolling a 2 the second time. The events of
rolling a 2 in the first roll and rolling a 2 in the second roll are
independent of each other.
\[ P(a\ 2\ in\ the\ second\ roll|a\ 2\ in\ the\ first\ roll) = \frac{1}{6} \]
\[ P(a\ 2\ in\ the\ second) = \frac{6}{36} = \frac{1}{6} \]

\subsection*{Example}
A basket has 3 balls marked R, B, and G. Select a ball 3 times without
replacement.
\begin{align*}
  E_{1}&: a\ G\ in\ the\ first\ and\ a\ G\ in\ the\ second \\
  E_{2}&: a\ G\ in\ the\ third \\
  E_{3}&: a\ G\ at\ least\ once \\
  E_{4}&: a\ G\ in\ the\ first \\
  E_{5}&: a\ B\ in\ the\ first
\end{align*}
\begin{center}
  \begin{forest}
    [
      [R [G [B] ] [B [G] ] ]
      [G [R [B] ] [B [R] ] ]
      [B [R [G] ] [G [R] ] ]
    ]
  \end{forest}
\end{center}
\begin{align*}
  P(A \cap B) &= P(A|B)P(B)
  &= P(A)P(B) (since\ A\ and\ B\ are\ independent)
\end{align*}
\( A \) and \( B \) are independent if and only if \( P(A \cap B) = P(A)P(B) \).
\begin{align*}
  P(E_{1}) &= \frac{0}{6} \\
  &= 0 \\
  P(E_{1} \cap E_{2}) &= 0 \\
  P(E_{1} \cap E_{2}) &= P(E_{1})P(E_{2})
\end{align*}
\( E_{1} \) and \( E_{2} \) are independent.
\begin{align*}
  P(E_{4}|E_{5}) &= \frac{0}{2} \\
  &= 0 \\
  P(E_{4}) &= \frac{2}{6} \\
  P(E_{4}|E_{5}) &\neq P(E_{4})
\end{align*}
\( E_{4} \) and \( E_{5} \) are dependent.
\begin{align*}
  P(E_{4} \cap E_{3}) &= \frac{2}{6} \\
  P(E_{4}) &= \frac{2}{6} \\
  P(E_{3}) &= \frac{6}{6} \\
  P(E_{4} \cap E_{3}) = P(E_{4})P(E_{3})
\end{align*}
\( E_{4} \) and \( E_{3} \) are independent.

\subsection*{Mutual Independence of Multiple Events}
Events \( A_{1}, A_{2}, \dots, A_{n} \) are mutually independent if for every
\( k(k = 2,3,\dots,k) \) and for every subset of indices \( i_{1}, i_{2}, i_{3},
\dots, i_{k} \) the following is true:
\[ P(A_{i_{1}} \cap A_{i_{2}} \cap \dots \cap A_{i_{k}}) =
   P(A_{i_{1}})\times P(A_{i_{2}})\times\dots\times P(A_{i_{k}}) \]

\subsection*{Example}
Let \( A_{1}, A_{2}, A_{3}, A_{4} \) be the events of an experiment such that
\( A_{1}, A_{2}, A_{3}, A_{4} \) are mutually independent.
\begin{align*}
  \bigg[P(A_{1} \cap A_{2}) &= P(A_{1})P(A_{2}) \\
  \wedge P(A_{1} \cap A_{3}) &= P(A_{1})P(A_{3}) \\
  \wedge P(A_{1} \cap A_{4}) &= P(A_{1})P(A_{3}) \\
  & \dots \\
  \wedge P(A_{1} \cap A_{2} \cap A_{3}) &= P(A_{1})P(A_{2})P(A_{3}) \\
  & \dots \\
  \wedge P(A_{1} \cap A_{2} \cap A_{3} \cap A_{4}) &=
    P(A_{1})P(A_{2})P(A_{3})P(A_{4}) \bigg]
\end{align*}

\subsection*{Proposition}
If \( A_{1},A_{2},A_{3},\dots,A_{n} \) are mutually independent, then
\( A_{1}',A_{2}',A_{3}',\dots,A_{n}' \) are mutually indepedent where
\( A_{1}' = not\ A_{1} \). \par
Consequently, if \( A_{1},A_{2},A_{3},A_{4} \) are mutually independent, then
\[ P(not\ A_{1} \cap not\ A_{2} \cap not\ A_{3}) =
   P(not\ A_{1})P(not\ A_{2})P(not\ A_{3}) \]

\subsection*{Example}
\begin{enumerate}
  \item A certain company sends 40\% of its overnight mail parcels via express
    mail service \( E_{1} \). Of these parcels, 2\% arrive after the guaranteed
    delivery time (denote the event ``late delivery'' by L). If a record of an
    overnight mailing is randomly selected from the company's file, what is the
    probability that the parcel went via \( E_{1} \) and was late?
  \item Suppose that 50\% of the overnight parcels are sent via express mail
    service \( E_{2} \). Of those sent via \( E_{2} \), only 1\% arrive late,
    whereas 5\% of the parcels handled by \( E_{3} \) arrive late. What is the
    probability that a randomly selected parcel arrives late?
  \item If a randomly selected parcel arrived on time, what is the probability
    that it was not sent via \( E_{1} \)?
\end{enumerate}
\begin{center}
  \begin{forest}
    [
      [\( E_{1}(0.4) \) [L(0.02)] [O(0.98)]]
      [\( E_{2}(0.5) \) [L(0.01)] [O(0.99)]]
      [\( E_{3}(0.1) \) [L(0.05)] [O(0.95)]]
    ]
  \end{forest}
\end{center}
\begin{enumerate}
  \item
    \[ (0.4)(0.02) = 0.008 \]
  \item
    \[ (0.4)(0.02)+(0.5)(0.01)+(0.1)(0.05) = 0.018 \]
  \item
    \[ \frac{(0.5)(0.99)+(0.1)(0.95)}{(0.4)(0.98)+(0.5)(0.99)+(0.1)(0.95)} \]
\end{enumerate}

\subsection*{Random Variable}
For a given sample space \( S \) of some experiment, a random variable \( (rv)
\) is any rule that associates a number with each outcome in \( S \).
A \( rv \) is a function whose domain is \( S \) and whose range is the set
of real numbers.

\subsection*{Example}
There are 3 balls marked 1, 3, 9 in a basket. Select a ball twice without
replacement. Let \( X \) be the sum of the two numbers.
\begin{center}
  \begin{forest}
    [
      [1 [3] [9]]
      [3 [1] [9]]
      [9 [1] [3]]
    ]
  \end{forest}
\end{center}
\[ X(1,3) = 4 \]
\[ X(3,9) = 12 \]
Any random variable whose only possible values are 0 and 1 is called a Bernoulli
random variable.

\subsection*{Example}
Determine the number of pumps in each of the two six-pump gas stations.
\begin{itemize}
  \item X = the total number of pumps in use at the two stations
  \item Y = the difference between the number of pumps in use at station 1 and
    the number of pumps in use at station 2
  \item U = the maximum number of pumps in use at the two stations
\end{itemize}
\begin{align*}
  W(observed) &= (3,4) \\
  X(W) &= 3+4 = 7 \\
  Y(W) &= 3-4 = -1 \\
  U(W) &= max(3,4) = 4
\end{align*}

\begin{center}
  If you have any questions, comments, or concerns, please contact me at
  alvin@omgimanerd.tech
\end{center}

\end{document}
