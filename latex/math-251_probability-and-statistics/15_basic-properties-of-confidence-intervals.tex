\documentclass[letterpaper, 12pt]{math}

\usepackage{amsmath}
\usepackage{amssymb}
\usepackage[linguistics]{forest}

\title{Probability and Statistics}
\author{Alvin Lin}
\date{Probability and Statistics: January 2017 - May 2017}

\begin{document}

\maketitle

\section*{Basic Properties of Confidence Intervals}
\[ B-r\leq A\leq B+r \]
Add \( -B \) to each side:
\[ -r\leq -B+A\leq r \]
Add \( -A \) to each side:
\[ -A-r\leq -B\leq -A+r \]
Multiply both sides by -1:
\[ A+r\geq B\geq A-r \]
The original statement is logically equivalent to the following:
\[ A-r\leq B\leq A+r \]
That is:
\[ B-r\leq A\leq B+r \quad\leftrightarrow\quad A-r\leq B\leq A+r \]

\subsection*{Theorem}
Consider a random sample \( X_{1},X_{2},\dots,X_{n} \) from a normal
distribution with mean value \( \mu \) (population mean) and standard deviation
\( \sigma \) (population standard deviation). Then the sample mean
\[ \overline{X} = \frac{X{1}+X_{2}+\dots+X_{n}}{n} \]
is normally distributed with with mean \( \mu \) and standard deviation
\( \frac{\sigma}{\sqrt{n}} \). The random variable
\( Z = \frac{\overline{X}-\mu}{\sigma/\sqrt{n}} \) is a standard normal random
variable. \par

\subsection*{Example}
Consider the following probability:
\begin{align*}
  P(-1.96\leq Z\leq 1.96) &= \Phi(1.96)-\Phi(-1.96) \\
  &= 0.9750-0.0250 \\
  &= 0.95 \\
  &= P(-1.96\leq\frac{\overline{X}-\mu}{\sigma/\sqrt{n}}) \\
  &= P(-1.96\frac{\sigma}{\sqrt{n}}\leq
    \overline{X}-\mu\leq1.96\frac{\sigma}{\sqrt{n}}) \\
  &= P(\mu-1.96\frac{\sigma}{\sqrt{n}}\leq\overline{X}
    \leq\mu+1.96\frac{\sigma}{\sqrt{n}}) \\
  &= P(\overline{X}-1.96\frac{\sigma}{\sqrt{n}}\leq\mu
    \leq\overline{X}+1.96\frac{\sigma}{\sqrt{n}})
\end{align*}
This represents the probability that the interval \( (\overline{X}
-1.96\frac{\sigma}{\sqrt{n}},\overline{X}+1.96\frac{\sigma}{\sqrt{n}}) \)
contains the population mean \( \mu \). After observing
\( X_{1}=x_{1},X_{2}=x_{2},\dots,X_{n}=x_{n} \), we compute the
observed sample mean \( \overline{x} \).
\[ \bigg(\overline{x}-1.96\frac{\sigma}{\sqrt{n}},
   \overline{x}+1.96\frac{\sigma}{\sqrt{n}}\bigg) \]
This interval is a 95\% confidence interval for \( \mu \).

\subsection*{Example}
Suppose we repeat taking a sample of size \( n \), finding \( \overline{x} \),
and calculating the interval 100 times. How many intervals do not contain the
population mean \( \mu \)? It could be any number. If \( N \) (the \# of
repetitions of sampling) \( \to\infty \), then 5\% of the 95\% confidence
intervals do not contain \( \mu \).

\subsection*{Definition}
\( 100(1-\alpha)\% \) confidence interval for the population mean \( \mu \) of
a normal population when the population standard deviation \( \sigma \) is known
is:
\[ \overline{x}\pm Z_{\alpha/2}\frac{\sigma}{\sqrt{n}} \]
\begin{align*}
  1-\alpha &= P(Z_{\alpha/z}\leq Z\leq Z_{\alpha/2}) \\
  &= P(\overline{X}-Z_{\alpha/2}\frac{\sigma}{\sqrt{n}}\leq \mu\leq
    \overline{X}+Z_{\alpha/2}\frac{\sigma}{\sqrt{n}}) \\
  &= P(the\ interval\ \overline{X}+Z_{\alpha/2}\frac{\sigma}{\sqrt{n}}
    \ contains\ \mu)
\end{align*}
The sample size necessary for the confidence interval to have a width \( w \)
is
\[ n = (2Z_{\alpha/2}\frac{\sigma}{w})^{2} \]

\subsection*{Example}
What is the confidence level for the interval \( \overline{x}\pm2.81
\frac{\sigma}{\sqrt{n}} \)?
\begin{align*}
  Z_{\alpha/2} &= 2.81 \\
  \alpha &= 0.0050 \\
  100(1-\alpha)\% &= 99.5\%
\end{align*}

\begin{center}
  If you have any questions, comments, or concerns, please contact me at
  alvin@omgimanerd.tech
\end{center}

\end{document}
