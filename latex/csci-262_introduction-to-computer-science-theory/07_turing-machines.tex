\documentclass{math}

\title{Introduction to Computer Science Theory}
\author{Alvin Lin}
\date{August 2017 - December 2017}

\begin{document}

\maketitle

\section*{Turing Machines}
A Turing machine is a finite-state machine in which each transition prints a
symbol on the tape. The tape head can move in either direction. The tape is
infinite to the right. A Turing machine is a septuple \( T =
(Q,\Sigma,\Gamma,\delta,q_0,q_{accept},q_{reject}) \) where:
\begin{itemize}
  \item \( Q \) is a finite set of states.
  \item \( \Sigma \) is the input alphabet such that \( \sqcup\notin\Sigma \).
  \item \( \Gamma \) is a finite set called the tape alphabet.
  \( \Sigma\cup\{\sqcup\}\subseteq\Gamma \).
  \item \( \delta \) is the transition function from
    \( (Q-\{q_{reject},q_{accept})\times\Gamma \) to
    \( Q\times\Gamma\times\{R,L\} \).
  \item \( q_0\in Q \) is the initial state.
  \item \( q_{accept}\in Q \) is the accept state.
  \item \( q_{reject}\in Q \) is the reject state.
\end{itemize}

\begin{center}
  You can find all my notes at \url{http://omgimanerd.tech/notes}. If you have
  any questions, comments, or concerns, please contact me at
  alvin@omgimanerd.tech
\end{center}

\end{document}
