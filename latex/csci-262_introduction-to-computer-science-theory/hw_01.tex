\documentclass[letterpaper, 12pt]{math}

\usepackage{amsmath}
\usepackage{amssymb}
\usepackage{geometry}
\usepackage{tikz}

\geometry{letterpaper, margin=0.5in}

\title{Intro to Computer Science Theory: Homework 1}
\author{Alvin Lin}
\date{August 2017 - December 2017}

\begin{document}

\maketitle

\subsection*{Problem 1}
Analyze the logical forms of the following statements. Use the same level of
detail as presented in page 18 of the course notes. Make sure that you add
quantifiers to ensure that all the variables are bound. See also exercise 2 of
section 2.1 of Velleman (2 points each).
\begin{enumerate}
  \item I've given you all I've got.
    \[ \forall{x}(g(x)\to G(x)) \]
    where \( g(x) \) stands for ``I've got x'' and \( G(x) \) stands for ``I've
    given you x''.
  \item Someday my prince will come.
    \[ \exists{d}P(d) \]
    where \( P(d) \) stands for ``My prince will come on day d''.
  \item Nobody loves me like my baby.
    \[ \neg\exists{x}L(x) \]
    where \( L(x) \) stands for ``x loves me like my baby''.
  \item If Fred can't do it, no one can.
    \[ \neg D(f)\to(\neg\exists{x}D(x)) \]
    where \( D(x) \) stands for ``x can do it'' and \( f \) stands for Fred.
  \item If Fred can't do it, everyone else can.
    \[ \neg D(f)\to(\forall{x}D(x)) \]
    where \( D(x) \) stands for ``x can do it`` and \( f \) stands for Fred.
\end{enumerate}

\subsection*{Problem 2}
Let \( A = \{z\in\R \mid 0 < z < 1\} \) and \( B = \{z\in\R \mid 0 \le z \le
1\} \). Say whether each of the following statements is true or false (1 point
each).
\begin{enumerate}
  \item \( \forall{x}\in A(\exists{y}\in A(x > y)) \) True
  \item \( \forall{x}\in A(\exists{y}\in A(x \ge y)) \) True
  \item \( \exists{y}\in A(\forall{x}\in A(x > y)) \) False
  \item \( \exists{y}\in A(\forall{x}\in A(x \ge y)) \) False
  \item \( \forall{x}\in B(\exists{y}\in B(x > y)) \) False
  \item \( \forall{x}\in B(\exists{y}\in B(x \ge y)) \) True
  \item \( \exists{y}\in B(\forall{x}\in B(x > y)) \) False
  \item \( \exists{y}\in B(\forall{x}\in B(x \ge y)) \) True
\end{enumerate}

\subsection*{Problem 3}
Prove that, for any language \( L \), if \( L\circ L \subseteq L \) then
\( L \circ L \circ L \subseteq L \), in the following steps.
\begin{enumerate}
  \item Rewrite in quantifiable logic form, with no free variables, using
    \( \subseteq \) in the writeup. (4 points)
    \[ \forall{L}(L\circ L\subseteq L \to L\circ L\circ L\subseteq L) \]
  \item Substitute for each occurrance \( \subseteq \) the definition of
    \( \subseteq \). (4 points)
    \[ \forall{L}(\forall{x}(x\in(L\circ L)\to x\in L)\to
      \forall{y}(y\in(L\circ L\circ L)\to y\in L)) \]
  \item Continue the proof. For each quantifier, you must use one of the proof
    strategies from Velleman. Use square brackets and logical forms to cite
    each use of a strategy, as in page 27 of the course notes. (4 points)
    \begin{itemize}
      \item Choose an arbitrary \( L \) [Velleman p.108]:
      \[ \forall{x}(x\in(L\circ L)\to x\in L)\to
        \forall{y}(y\in(L\circ L\circ L)\to y\in L) \]
      \item This is of the form \( R\to S \) [Velleman p.87], assume
      \[ \forall{x}(x\in(L\circ L)\to x\in L) \]
      \[ \forall{x}(x\in\{yz\mid y\in L,z\in L\})\to x\in L \]
      and prove:
      \[ \forall{y}(y\in(L\circ L\circ L)\to y\in L) \]
      \item Choose an arbitrary \( x \) [Velleman p.108]:
      \[ \]
    \end{itemize}
\end{enumerate}

\begin{center}
  If you have any questions, comments, or concerns, please contact me at
  alvin@omgimanerd.tech
\end{center}

\end{document}
