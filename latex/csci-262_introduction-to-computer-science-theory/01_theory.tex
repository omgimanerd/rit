\documentclass[letterpaper, 12pt]{math}

\usepackage{amsmath}
\usepackage{amssymb}
\usepackage{hyperref}

\title{Introduction to Computer Science Theory}
\author{Alvin Lin}
\date{August - December 2017}

\begin{document}

\maketitle

\section*{What is theory?}
\begin{itemize}
  \item Greek \textit{theoria}: ``a looking at, viewing, beholding''
  \item National Academy of Sciences: ``... a well-substantiated
    explanation of some aspect of the natural world, based on a body of
    facts that have been repeatedly confirmed through observation and
    experiment.
  \item A simple explanatory framework
\end{itemize}

\subsection*{Criteria for a Good Theory}
\begin{itemize}
  \item Makes falsifiable predictions.
  \item Well-supported by many independent strands of evidence.
  \item Consistent with pre-existing theories.
  \item Can be adapted and modified to account for new evidence.
  \item Parsimonious (Occam's razor)
\end{itemize}

\subsection*{Automata Theory}
\begin{itemize}
  \item The foundational theory of computer science.
  \item Based on simple, abstract, and mathematically well-defined
    machines.
  \item Uses proofs, theorems, et cetera to extend the theory.
\end{itemize}

\subsection*{Why should you care about the theory of computing?}
\begin{itemize}
  \item Any scientific inquiry should be guided by and phrased in
    terms of theory.
  \item To gain an understanding of what computers can and cannot do, akin
    to the role physics plays in engineering.
  \item To learn how to think and express yourself formally and abstractly,
    using English, not a programming language. This is use when communicating
    with clients and other nontechnical people, writing patents or scientific
    papers, or porting algorithms to other systems.
  \item Programming is itself a formal problem. Understanding more broadly how
    formal methods work results in better programming.
\end{itemize}

\section*{Sets}
A collection of objects (a.k.a. hashes, associated arrays). Examples:
\begin{itemize}
  \item The set of natural numbers
  \begin{align*}
    \mathrm{Define}\ \N&: \\
    \mathrm{Basis}&: 1\in\N \\
    \mathrm{Recursion}&: \forall n\in\N(n+1\in\N)
  \end{align*}
  \item \( \{1,-1,2,-2,\dots,345,-345\} \)
  \item \( \{x\in\N\ |\ x>517\} \)
  \item \( 2^{S} \) where \( S \) is a set.
    \[ 2^{\{0,1,2\}} = \{\emptyset,\{0\},\{1\},\{2\},\{0,1\},
      \{0,2\},\{1,2\},\{0,1,2\}\} \]
  \item \( S = \{1,\{1,\{1,\{\dots\},\dots\}\}\} \) is a set that
    contains itself.
  \item Is \( \{S\ |\ S\ is\ a\ set\ and\ S\in S\} \) a set? Yes.
  \item Is \( \{S\ |\ S\ is\ a\ set\ and\ S\notin S\} \) a set? No.
\end{itemize}
The complement of a set is denoted \( \bar{A} \) or as \( A' \).
\[ \bar{A} = \U-A \]

\section*{Languages}
A \textbf{language} is a set of strings involving symbols from some
\textbf{alphabet}. An \textbf{alphabet} is a finite set of
indivisible objects, usually denoted by \( \Sigma \). Examples:
\begin{itemize}
  \item \( \{0,1\} \)
  \item \( \{a,b,c,d,e,\dots,x,y,z\} \)
\end{itemize}

A string over \( \Sigma \) is finite (possibly empty) sequence of elements of
\( \Sigma \). Strings of alpha \( \{0,1\} \) are:
\begin{itemize}
  \item \( \epsilon \)
  \item \( 0 \)
  \item \( 1 \)
  \item \( 10010011101 \)
  \item \( 00000 \)
\end{itemize}

\( \epsilon \) denotes the \textbf{null string}: the empty sequence of elements
of \( \Sigma \). Sometimes this is denoted \( \wedge \) or \( \lambda \). The
following strings are not strings over \( \{0,1\} \):
\begin{itemize}
  \item \( 1021 \)
  \item \( 1111111111\dots \)
\end{itemize}

If \( x \) is a string of \( \Sigma \), \( |x| \) denotes the length of \( x \),
the number of alphabet symbols in the string.
\begin{itemize}
  \item \( |1| = 1 \)
  \item \( |10001010101| = 11 \)
  \item \( |00000| = 5 \)
  \item \( |\epsilon| = 0 \)
\end{itemize}

The set of all strings over \( \Sigma \) is denoted by \( \Sigma^* \).
For example, \( \{0,1\}^* = \{\epsilon,0,1,00,01,10,11,000,001,\dots\} \).
A language of \( \Sigma \) is a subset of \( \Sigma^* \).
\[ \{a,b,c\}^* = \{\epsilon,a,b,c,aa,ab,ac,ba,bb,bc,\dots\} \]
A language over \( \Sigma \) is a subset of \( \Sigma^* \). Example:
\begin{itemize}
  \item \( \{1,010,111000,110\} \)
  \item \( \emptyset \)
  \item \( \epsilon \)
  \item \( \{0,1\}^* \)
  \item \( \{x\in\{0,1\}^*\ |\ |x|\ is\ even\} \)
\end{itemize}

Because languages are sets of strings, we can construct new languages using
set operations (like union, intersection, and complementation).
\[ L_{1}, L_{2} \]
\[ L_{1}\cup L_{2} \quad L_{1}\cap L_{2} \]
If \( L \) is a language of \( \Sigma \), then \( \bar{L} \) (the complement of
\( L \)) is \( \Sigma^*-L \).
\[ L = \{0,00,000,0000,\dots\} \]
If \( \Sigma = \{0\} \):
\[ \bar{L} = \{\epsilon\} \]
If \( \Sigma = \{1,0\} \):
\[ \bar{L} = \{\epsilon,1,01,10,11,\dots\} \]

\subsection*{Operations on Strings}
Let \( x,y\in\Sigma^* \):
\begin{itemize}
  \item \( xy \) is the \textbf{concatenation} of \( x \) and \( y \). For
    example, if \( x = \text{ab} \) and \( y = \text{aabb} \), then \( xy =
    \text{abaabb} \). For all strings \( x \), \( x\epsilon = \epsilon x =
    x \).
  \item For \( i \) an integer, \( x^i \) is the concatenation of \( i\ x's \).
    For example, if \( x = abb \), then \( x^{3} = \text{abbabbabb} \). For all
    strings \( x \), \( x^0 = \epsilon \).
  \item x is a \textbf{substring} of y if and only if there exist \( w,z\in
    \Sigma^* \) such that \( wxz = y \).
  \item \( x \) is a \textbf{prefix} of \( y \) if and only if there exists a
    \( z\in\Sigma^* \) such that \( xz = y \).
  \item \( x \) is a \textbf{suffix} of \( y \) if and only if there exists a
    \( z\in\Sigma^* \) such that \( zx = y \).
\end{itemize}

\subsection*{Operations on Languages}
Operations on strings can be extended to operations on languages. Let
\( L,L_1,L_2\subseteq\Sigma^* \):
\begin{itemize}
  \item \( L_1\circ L_2 = \{xy\ |\ x\in L_1,y\in L_2\} \) is the concatenation
    of languages \( L_1 \) and \( L_2 \). What is \( \{ab,a\}\circ\{a,ba,aaaa\}
    \)
    \[ \{aba,abba,abaaaa,aa,aba,aaaaa\} \]
  \item For \( i \) an integer, \( L^i \) is the concatenation of \( i L's \).
  \item \( L^* = \cup_{i=0}^{\infty}L^i \) and
    \( L^+ = \cup_{i=1}^{\infty}L^i \).
\end{itemize}

\begin{center}
  You can find all my notes at \url{http://omgimanerd.tech/notes}. If you have
  any questions, comments, or concerns, feel free to contact me at
  alvin@omgimanerd.tech
\end{center}

\end{document}
