\documentclass[letterpaper, 12pt]{math}

\usepackage{enumerate}

\title{Introduction to Computer Science Theory}
\author{Alvin Lin}
\date{August 2017 - December 2017}

\begin{document}

\maketitle

\section*{Regular Languages and Regular Expressions}
A \textbf{regular expression} over \( \Sigma \) is a description of a language
that can be built from \( \emptyset,\{\epsilon\},\{a\} \) for \( a\in\Sigma \),
using operators, union, concatenation, and Kleene star. For example, the
following language over \( \{a,b\} \) is described by a regular expression:
\[ (\{a\}\cup\{b\})^*\circ\{a\}\circ\{a\}\circ(\{a\}\cup\{b\})^*\cup
  (\{a\}\cup\{b\})^*\circ\{b\}\circ\{b\}\circ(\{a\}\cup\{b\})^* \]
This can be written a bit more concisely as:
\[ (\{a\}\cup\{b\})^*aa(\{a\}\cup\{b\})^*\cup
  (\{a\}\cup\{b\})^*bb(\{a\}\cup\{b\})^* \]

\subsubsection*{Example}
Given regular expressions to describe:
\begin{itemize}
  \item The language of strings over \( \{a,b\} \) of even length.
  \[ (aa\cup ab\cup ba\cup bb)^* \]
  \[ ((a\cup b)(a\cup b))^* \]
  \item The language of strings over \( \{a,b,c\} \) in which all \( s \)'s
  precede all \( b \)'s and \( c \)'s, and all \( b \)'s precede all \( c \)'s.
  \[ a^*b^*c^* \]
  \item The language of strings over \( \{0,1\} \) with length greater than
  three.
  \[ (0\cup1)(0\cup1)(0\cup1)(0\cup1)(0\cup1)^* \]
  \[ (0\cup1)^4(0\cup1)^* \]
  \item The language of strings of odd length of \( \{a,b\} \) that contain
  the substring \( \{ bb \} \).
  \[ ((a\cup b)^2)^*(abb\cup bba \cup bbb)((a\cup b)^2)^* \]
  \item The language of strings of \( \{0,1\} \) that do not contain the
  substring \( 000 \).
\end{itemize}
Is \( \{a^ib^i \mid i\in\N \} \)?

\subsection*{Equivalence of Finite Automata and Regular Expressions}
It is possible to construct a nondeterministic finite automata from a regular
expression, and since deterministic finite automata can be constructed from
nondeterministic finite automata, the three models can be represented
equivalently using all three. This is something we can prove. \\
\textbf{Theorem:} A language is regular if and only if some regular expressions
describes it. \\
\textbf{Lemma:} If a language is described by a regular expression, then it is
regular. \\
\textbf{Lemma:} If a language is regular, then it is described by a regular
language. \\
Recall the recursive definition of regular languages. \( R \) is a regular
expression if it is:
\begin{enumerate}
  \item \( a \), for some \( a\in\Sigma \),
  \item \( \emptyset \),
  \item \( \epsilon \),
  \item \( R_1\cup R_2, R_1R_2, R_1^*\text{ and }(R_1) \), where \( R_1 \) and
  \( R_2 \) are regular expressions.
\end{enumerate}

\subsection*{Proof}
Base Cases (from the recursive definition):
\begin{enumerate}[{Case} 1:]
  \item \( r = a\in\Sigma \)
  \item \( r = \emptyset \)
  \item \( r = \epsilon \)
\end{enumerate}
\textbf{Induction Hypothesis:} Assume that for arbitrary regular expressions
\( r_1,r_2 \) that nondeterministic finite automata \( N_1 =
(Q_1,\Sigma_1,\delta_1,q_1,F_1), N_2 = (Q_2,\Sigma_2,\delta_2,q_2,F_2) \) exist
where \( L(r_1) = L(N_1) \) and \( L(r_2) = L(N_2) \). Assume without loss of
generality \( Q_1\cap Q_2 = \emptyset \).
\begin{enumerate}[{Case} 1:]
  \item The following NFA \( N = (Q,\Sigma,\delta,q_0,F) \) recognizes
  \( L(r_1\cup r_2) \):
  \begin{align*}
    Q &= Q_1\cup Q_2\cup \{q_0\} \text{ where } q_0\notin Q_1\cup Q_2 \\
    F &= F_1\cup F_2 \\
    \delta:Q\times\Sigma_{\epsilon} &\to 2^Q\text{ on }(q,x)\in
      Q\times\Sigma_{\epsilon} \\
    \delta(q,x) &= \begin{cases}
      \{q_1,q_2\} & \text{if } (q,z) = (q_0,\epsilon) \\
      \delta_i(q,x) & \text{if } q\in Q_i\text{ for }i\in\{1,2\} \\
      \emptyset & otherwise
    \end{cases} \\
  \end{align*}
\end{enumerate}

\begin{center}
  You can find all my notes at \url{http://omgimanerd.tech/notes}. If you have
  any questions, comments, or concerns, please contact me at
  alvin@omgimanerd.tech
\end{center}

\end{document}
