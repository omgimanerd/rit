\documentclass{math}

\usepackage[utf8]{inputenc}

\title{Introduction to Intelligent Security Systems}
\author{Alvin Lin}
\date{August 2018 - December 2018}

\begin{document}

\maketitle

\section*{Hackers: Activity and Prevention}
\begin{itemize}
  \item Hacking: showing computer expertise
  \item Cracking: breaching security on software or systems
  \item Phreaking: cracking telecom networks
  \item Spoofing: faking the originating IP address in a datagram
  \item Denial of Service (DoS): flooding a host with sufficient network traffic
    to overload it
  \item Port Scanning: searching for vulnerabilities
\end{itemize}

\subsection*{Hacker Attack Examples}
\begin{itemize}
  \item In April 2007, Estonian government networks were harassed by a denial of
    service attack by unknown foreign intruders, following the country's spat
    with Russia over the removal of a war memorial. Some government online
    services were temporarily disrupted and online banking was halted.
  \item In June 2007, the US Secretary of Defense's unclassified email account
    was hacked by unknown foreign intruders as part of a larger series of
    attacks to access and exploit the Pentagon's networks.
  \item In October 2007, China's Ministry of State Security said that foreign
    hackers, which it claimed 42\% came from Taiwan and 25\% from the US, had
    been stealing information from Chinese key areas. In 2006, when the China
    Aerospace Science and Industry Corporation intranet network was surveyed,
    spywares were found in the computers of classified departments and
    corporate leaders.
  \item In the summer of 2008, the databases of both Republican and Democratic
    presidential campaigns were hacked and downloaded by unknown foreign
    intruders.
  \item In July 2011, the US Deputy Secretary of Defense mentioned that a
    defense contractor was hacked and 24,000 files from the Department of
    Defense were stolen.
  \item Sony Hackers orchestrated multiple breaches of Sony's PlayStation
    network, knocking it offline for 24 days and costing the company an
    estimated \$171 million.
  \item In February 2016, hackers angry about the US relations with Israel
    tried to call attention to their cause by breaching the US Department of
    Justice's database. CNN reported that the hackers released data on
    10,000 Department of Homeland Security employees one day, and data on
    20,000 FBI employees the next day. The information stolen included names,
    titles, phone numbers, and email addresses, but no sensitive information
    like SSNs were obtained.
\end{itemize}
Hacker threats include denial of service, defacing, graffiti, slander,
reputation damage, loss of data, corporate espionage, or loss of financial
assets.

\subsection*{Types of Hackers}
\begin{itemize}
  \item Script kiddies: mostly kids and students that use tools created by
    black hats.
  \item Underemployed adult hackers: former script kiddies usually.
  \item Ideological hackers: hack as a mechanism to promote some political
    or ideological purpose.
  \item Criminal hackers: real criminals who are in it for personal gain.
  \item Corporate spies: relatively rare.
  \item Disgruntled employees: usually the most dangerous to a company since
    they have insider access.
  \item Professional hackers: black hat hackers are malicious hackers to cause
    harm. White hat hackers are usually professional security experts and
    consultants who offer hacking/penetration testing as part of their services.
    Grey hats do not engage in malicious activity but may use hacking methods
    that are illegal or unethical.
\end{itemize}

\subsection*{Attack Purposes}
\begin{itemize}
  \item Reconnaisance attacks are an attempt to gather sensitive information
    about network services and systems (packet sniffers, ping sweep, port scans,
    queries regarding networking information).
  \item Denial of service attacks are network attacks devised to slow down or
    crash a system by flooding it with useless traffic (ping of death, teardrop
    attack).
  \item Access attacks are when attacks try to uncover exploits and
    vulnerabilities in order to gain access to a system's network (password
    attack, trust exploitation attack, man in the middle).
\end{itemize}

\subsubsection*{Gaining Access}
\begin{itemize}
  \item Front Door: password guessing, password stealing
  \item Back Door: often left by developers as debugging or diagnostic tools
  \item Trojan Horses: hidden inside of software downloaded from the internet,
    which usually will install a backdoor
  \item Software Vulnerability Explotation: allow for privilege escalation or
    arbitrary code execution
\end{itemize}
Once insider, hackers can modify logs to cover their tracks, steal files, modify
files, install back doors, and attack other systems.

\subsection*{Spoofing}
Attackers may use spoofing to alter their identity so that they appear as
someone else.
\begin{itemize}
  \item IP Spoofing: attackers use the IP address of another computer to
    acquire information or gain access. In a flying-blind attack, they
    send messages while impersonating another machine, but cannot receive
    messages addressed to that machine. In a source routing attack, an
    attack spoofs the address of a machine and inserts itself between the
    attacked machine and the spoofed machine to intercept replies.
  \item Email Spoofing: attackers can create accounts with similar addresses,
    modify mail clients, or telnet to port 25 of a mail server to send
    messages masquerading as someone else.
  \item Web Spoofing: attacks can register a web address matching another
    entity. With man in the middle attacks, the attacker acts as a proxy
    between the web server and the client. With URL rewriting attacks, the
    attacker redirects web traffic to another site that they control. With
    tracking state attacks, attackers can steal the authentication token of a
    legitimate user after authentication.
  \item Session Hijacking: after a legitimate user authenticates, the attacker
    takes the user offline by a denial of service and then impersonates the
    user.
\end{itemize}

\begin{center}
  You can find all my notes at \url{http://omgimanerd.tech/notes}. If you have
  any questions, comments, or concerns, please contact me at
  alvin@omgimanerd.tech
\end{center}

\end{document}
