\documentclass{math}

\title{Introduction to Intelligent Security Systems}
\author{Alvin Lin}
\date{August 2018 - December 2018}

\begin{document}

\maketitle

\section*{What is AI?}
The term AI was coined by 1956 by J. McCarthy at MIT. While there is no
generally accepted definition, various techniques from many engineering
disciplines are considered as belonging to AI. It involves two major branches,
one dealing with the creation of intelligent machines and one dealing with the
empirical science concerned with computational modeling of human intelligence.
The goal of AI is to develop methods which allow us to produce thinking machines
that solve problems. A great variety of techniques have been developed and
applied over the years, such as:
\begin{itemize}
  \item search algorithms
  \item probabilistic reasoning
  \item natural language processing
  \item belief networks
\end{itemize}

\subsection*{Systemic Definitions of AI}
\begin{enumerate}
  \item Think like humans
  \item Act like humans
  \item Think rationally
  \item Act rationally
\end{enumerate}
One of the most significant papers on machine intelligence, ``Computing
Machinery and Intelligence'', was written by the British mathematician Alan
Turing. He proposed a test known as the Turing test and this approach remains
universal.

\subsection*{Problem Solving Techniques}
Hard computing involves precise models such as symbolic logical reasoning and
traditional numerical modeling and search. Soft computing involves approximate
models such as approximate reasoning and functional approximation or randomized
search. The human mental process is often too complex to be represented as an
algorithm, but most experts can represent their knowledge as a series of rules
for problem solving.

\begin{center}
  You can find all my notes at \url{http://omgimanerd.tech/notes}. If you have
  any questions, comments, or concerns, please contact me at
  alvin@omgimanerd.tech
\end{center}

\end{document}
