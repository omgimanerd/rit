\documentclass{math}

\usepackage[utf8]{inputenc}

\title{Introduction to Intelligent Security Systems}
\author{Alvin Lin}
\date{August 2018 - December 2018}

\begin{document}

\maketitle

\section*{Android Colluded Applications Attack}
Because of the interconnectedness of online businesses, all of us are generating
data and consuming data. This makes data security the biggest factor in data
quality, which involves accessibility, security, and privacy. There are many
parts that play into this, including data trustworthiness, device security,
communication security, and user privacy.

\subsection*{Application Collusion}
Colluded applications are applications that may share data through means outside
of system processes. RAM usage and CPU patterns may be reflected through this.
This is generally a violation of Android security because applications are
sandboxed. In order to use the device's resources, an application has to ask
for permission first. \par
Overt communication channels are used for explicit data communications between
applications, generally done with Intents in Android which launch an external
application to perform a service and transmit data back to the original
application.

\subsection*{Attack Scenario}
Suppose you have a contact optimizer which has permissions to access a user's
contact list and a weather app with permission to access the Internet. The
contact optimizer itself cannot leak your private contact data, but it can
collude with the weather app through Intents to send your data through the
network. Collections of maliciously designed apps can collude to perform
malicious actions.

\subsection*{Covert Communication Channels}
Covert inter-process communication creates a means for applications to
communicate where they should not be able to. Two applications reading
harmless data can communicate through the request intervals. This method is
often slow however, and not feasible for larger amounts of data.

\subsection*{Colluded Application Attack Detection}
\begin{itemize}
  \item Time Level: reactive (real time prevention), proactive (try to
    prevent it in advance through analysis)
  \item Integration Level: standalone applications or library integrated into
    Android OS to trace your information flow
  \item Involvement Level: tool usage depends on the firmware, application
    developer, and device usage.
  \item Component Level: able to analyze all application components
\end{itemize}

\begin{center}
  You can find all my notes at \url{http://omgimanerd.tech/notes}. If you have
  any questions, comments, or concerns, please contact me at
  alvin@omgimanerd.tech
\end{center}

\end{document}
