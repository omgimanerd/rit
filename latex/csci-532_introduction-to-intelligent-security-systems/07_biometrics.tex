\documentclass{math}

\usepackage[utf8]{inputenc}

\title{Introduction to Intelligent Security Systems}
\author{Alvin Lin}
\date{August 2018 - December 2018}

\begin{document}

\maketitle

\section*{Biometrics}
Tradtionally means of automatic identification:
\begin{itemize}
  \item possession-based (credit card, smart card)
  \item knowledge-based (password, PIN)
  \item biometric-based (fingerprint, iris scan)
\end{itemize}
Possession and knowledge based authentication mechanisms are replaceable, but
can be lost, stolen, forgotten, or guessed by imposters. Annually, it is
estimated there is over \$1 billion in fraudulent credit card transcations and
\$3 billion in fraudulent ATM withdrawals. These traditional approaches are
unable to differentiate between an authorized person and an imposter since they
check against something known or in possession by the person being
authenticated.

\subsection*{What is biometrics?}
\textbf{Biometrics} is the science which deals with the automated recognition of
individuals based on their biological and behavioral characteristics. It
uses \textbf{biometry}, the mathematical and statistical aspects of biology,
to recognize a person by determining the authenticity of a specific biological
and/or behavioral characteristic possessed by that person. \par
The scientific literature on quantitative measurement of humans for the purpose
of identification dates back to the 1870s and the measurement system of Alphonse
Bertillon. Bertillon used a system of body measurements such as skull diameter,
arm length, foot length, etc to identify prisoners in the United States until
the 1920s. In the 1880s, Henry Faulds, William Herchel, and Sir Francis Galton
proposed quantitative identification through fingerprint and facial
measurements. Biometrics was introduced into forensic identification by
Edmond Locard in the 1920s.

\subsection*{Authentication vs Identification}
\textbf{Verification} is the process of recognizing a person by comparing the
captured biometric characteristic with a biometric template stored in the
system. It performs a one-to-one match and checks if the person is who they
claim they are. \\
\textbf{Identification} is the process of recognizing a person by searching a
template database for a match. It performs one-to-many matches to assign an
identity to the person.

\subsection*{Uses of Biometrics}
Biometric systems are used for physical access control in places like airports
and other travel infrastructure. It can also be used for logical access in
places like banks to regulate access to money. It is also commonly used to
ensure uniqueness of individuals, such as for enrollment in a benefits program.
\par
The development of digital signal processing techniques in the 1960s led to
work in automatic human identification. Speaker and fingerprint recognition
systems were among the first to be explored. The potential application of
this technology for high-security access control was recognized, leading to
increased government usage. Retinal and signature verification systems came
in the 1980s, followed by facial recognition and iris recognition systems in
the 1990s.

\subsection*{Misconceptions}
Hollywood typically portrays facial recognition as instantaneous, tied to a
database of all criminals, and working 100\% of the time. In reality, facial
recognition algorithms are vastly affected by lighting, angle, face size, and
image quality. They require a high end computer for processing and are still
being evaluated as a tool for law enforcement. Match confidence varies depending
on the application and data sharing between law enforcement organizations is
difficult.

\subsection*{Types of Biometrics}
Biometric systems measure various physiological or behavioral characteristics
such as fingerprints, voice pattern, iris/retinal pattern, hand shape, face
shape, handwriting, keystroke usage, and finger shape. This is only a partial
list as new features such as gait, ear shape, head resonance, optical skin
reflectance, and vein structure are being developed all the time. Because of the
broad range of characteristics, imaging requirements for biometric technologies
vary greatly. Examples:
\begin{itemize}
  \item voice - one dimensional signal
  \item hand writing - several simulataneous one dimensional signals
  \item fingerprint - two dimensional image
  \item hand geometry - multiple two dimensional measures
  \item face and iris scan - time series of two dimensional images or a three
    dimensional image
\end{itemize}
Ideally, biometric characteristics for identification have five qualities.
\begin{itemize}
  \item \textbf{Robust} - unchanging on an individual over time.
  \item \textbf{Distinctive} - showing great variation over the population.
  \item \textbf{Available} - the entire population should ideally have this
    measure.
  \item \textbf{Accessible} - easy to image using electronic sensors.
  \item \textbf{Acceptable} - people do not object to having this measurement
    taken.
\end{itemize}
Quantitative measures of these five qualities have been developed.
\begin{itemize}
  \item Robusness is measured by the false negative rate (Type I error), the
    probability that a submitted sample will not match the enrollment image.
  \item Distinctiveness is measured by the false positive rate (Type II error),
    the probability that a submitted sample matches the enrollment image of
    another user.
  \item Availablility is measured by the rate of enrollment failure, the
    probability that a user will not be able to supply a readable measure to the
    system upon enrollment.
  \item Accessibility is quantified by the throughput rate of the system, the
    number of systems that can be processed in some unit time.
  \item Acceptability is measured by polling the device users.
\end{itemize}

\subsection*{Biometric Systems}
Biometric systems generally perform one of two tasks:
\begin{itemize}
  \item Positive identification - the submitted samples are from an individual
    known to the system.
  \item Negative identification - the submitted samples are from an individual
    not known to the system.
\end{itemize}
They can also have other types of classifications:
\begin{itemize}
  \item \textbf{Overt vs Covert} - if the user is aware that a biometric
    identifier is being measured, then the use is overt. Most access control
    systems are overt. Generally only some forensic applications are covert.
  \item \textbf{Habituated vs Non-habituated} - users presenting the biometric
    trait on a daily basis can be considered habituated after a short period of
    time. Users who have not presented the trait recented are considered
    non-habituated.
  \item \textbf{Attended vs Non-attended} - whether or not the use of the
    biometric device will be observed and guided by system management during
    operation.
  \item \textbf{Open vs Closed} - an open system uses public data collection,
    compression, and format standards while a closed system operates on
    proprietary formats.
\end{itemize}
For open systems, compression and transmission protocols have to be
standardized so that every user of the data can reconstruct the original signal.
Current standards for data transmission include wavelet scalar quantization
for fingerprint data, JPEG images for facial images, and code excited linear
prediction for voice data.

\subsection*{Feature Extraction}
In general, feature extraction is a form of non-reversible compression,
meaning that the original biometric image cannot be reconstructed from the
extracted features.

\begin{center}
  You can find all my notes at \url{http://omgimanerd.tech/notes}. If you have
  any questions, comments, or concerns, please contact me at
  alvin@omgimanerd.tech
\end{center}

\end{document}
