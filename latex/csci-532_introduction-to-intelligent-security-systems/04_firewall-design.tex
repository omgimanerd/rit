\documentclass{math}

\usepackage[utf8]{inputenc}

\title{Introduction to Intelligent Security Systems}
\author{Alvin Lin}
\date{August 2018 - December 2018}

\begin{document}

\maketitle

\section*{Firewall Design}
A firewall is a device or program that controls the flow of network traffic
between networks or hosts that employ differing security postures. They isolate
an organization's internal network from the larger Internet, only allowing some
traffic to pass while blocking others. It is a junction point between two
networks and sets a border line for network administration responsibility. The
term originates from firewalls and fire doors in buildings.

\subsection*{History}
\begin{itemize}
  \item 1764 - The term ``firewall'' was used to describe walls which
    separated the parts of a building most likely to have a fire from the rest
    of the structure.
  \item Late 1980s - The predecessors to firewalls for network security were the
    routers used to separate networks from one another.
  \item 1988 - The Morris Worm infected about 6000 systems.
  \item 1994 - Alice Muffett wrote a paper which provided an excellent review
    of the firewall policies and architectures of the time.
  \item 2004 - IDC coins the term UTM (unified threat management) and several
    security vendors following suit, beginning to market their firewalls that
    run multiple security functions.
  \item 2009 - Gartner defines the next generation firewall as a ``a wire-speed
    integrated network platform that performs deep inspection of traffic and
    blocking of attacks''. Results of the first widespread analysis of firewall
    management practices are published, including statistics like ``93\% felt
    their firewalls contained at least one category of error and 70\% felt
    that it was likely their rulebases contained undetected errors''.
  \item 2012 - Gartner releases a forecast that says more than 95\% of firewall
    breachers will be caused by misconfiguration and not firewall flaws.
\end{itemize}
Software firewalls operate locally on the machine they are intended to protect
while hardware firewalls are usually part of a TCP/IP router.

\subsection*{Functions}
Firewalls can filter traffic based on their source and destination addresses,
port numbers, protocols used, and packet state. They cannot prevent individual
users with modems from dialing in and out of the network. They cannot protect
against social engineering and dumpster diving. \par
Generally anyone who is responsible for a private network connected to a public
network should use a firewall. Any computer connected to a public network should
have a firewall.

\begin{center}
  You can find all my notes at \url{http://omgimanerd.tech/notes}. If you have
  any questions, comments, or concerns, please contact me at
  alvin@omgimanerd.tech
\end{center}

\end{document}
