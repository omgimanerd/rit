\documentclass{math}

\usepackage[utf8]{inputenc}

\title{Introduction to Intelligent Security Systems}
\author{Alvin Lin}
\date{August 2018 - December 2018}

\begin{document}

\maketitle

\section*{Firewall Design}
A firewall is a device or program that controls the flow of network traffic
between networks or hosts that employ differing security postures. They isolate
an organization's internal network from the larger Internet, only allowing some
traffic to pass while blocking others. It is a junction point between two
networks and sets a border line for network administration responsibility. The
term originates from firewalls and fire doors in buildings.

\subsection*{History}
\begin{itemize}
  \item 1764 - The term ``firewall'' was used to describe walls which
    separated the parts of a building most likely to have a fire from the rest
    of the structure.
  \item Late 1980s - The predecessors to firewalls for network security were the
    routers used to separate networks from one another.
  \item 1988 - The Morris Worm infected about 6000 systems.
  \item 1994 - Alice Muffett wrote a paper which provided an excellent review
    of the firewall policies and architectures of the time.
  \item 2004 - IDC coins the term UTM (unified threat management) and several
    security vendors following suit, beginning to market their firewalls that
    run multiple security functions.
  \item 2009 - Gartner defines the next generation firewall as a ``a wire-speed
    integrated network platform that performs deep inspection of traffic and
    blocking of attacks''. Results of the first widespread analysis of firewall
    management practices are published, including statistics like ``93\% felt
    their firewalls contained at least one category of error and 70\% felt
    that it was likely their rulebases contained undetected errors''.
  \item 2012 - Gartner releases a forecast that says more than 95\% of firewall
    breachers will be caused by misconfiguration and not firewall flaws.
\end{itemize}
Software firewalls operate locally on the machine they are intended to protect
while hardware firewalls are usually part of a TCP/IP router.

\subsection*{Functions}
Firewalls can filter traffic based on their source and destination addresses,
port numbers, protocols used, and packet state. They cannot prevent individual
users with modems from dialing in and out of the network. They cannot protect
against social engineering and dumpster diving. They can generally only work on
inspectable traffic and thus are not suitable for encrypted traffic. \par
Generally anyone who is responsible for a private network connected to a public
network should use a firewall. Any computer connected to a public network should
have a firewall.

\subsection*{TCP/TP Layers}
\begin{itemize}
  \item Application Layer: This layer sends and receives data for particular
    applications, such as DNS, HTTP, and SMTP. The application layer itself has
    layers of protocols within in. For example, SMTP encapsulates the request
    RFC 2822 message syntax, which encapsulates MIME, which can encapsulate
    other standards such as HTML.
  \item Transport Layer: This layer provides connection-oriented or
    connectionless services for transporting application layer services between
    networks, and can optionally ensure communication reliability. TCP and UDP
    are commonly used transport layer protocols.
  \item IP Layer: This layer routes packets across networks. IPv4 is the
    fundamental network layer protocol for TCP/IP. Other common used protocols
    at the network layer are IPv6, ICMP, and IGMP.
  \item Hardware Layer: Also known as the data link layer, this layer handles
    communications on the physical network components, such as Ethernet.
\end{itemize}

\subsection*{Configurations}
\begin{itemize}
  \item \textbf{Packet Filtering Firewall}: This is the most basic feature of a
    firewall. Also known as a stateless inspection firewall, this is typically
    done using access control lists on a network router. Filtering inbound
    traffic is known as ingress filtering, while filtering outbound traffic is
    known as egress filtering. Stateless packet filters are generally
    vulnerable to attacks and exploits that take advantage of problems in the
    TCP/IP stack and protocol stack. They are unable to detect if a packet's
    network layer addressing information has been spoofed or altered. Some
    firewalls can reassemble fragmented packets before passing them to the
    internal network.
  \item \textbf{Stateful Inspection}: Stateful inspection improves on the
    functions of packet filters by tracking the state of connections and
    blocking packets that deviate from the expected state. This is accomplished
    by examining certain values in the TCP headers to monitor the state of each
    connection. Each new packet is compared by the firewall to the firewall's
    state table to determine if the packet's state contradicts its expected
    state. This is generally used to monitor a sequence of packets.
  \item \textbf{Application Firewalls}: This improves upon standard stateful
    inspection by adding data analytics abilities through an inspection engine
    that analyzes protocols at the applicaton layer. Application firewalls can
    allow or deny access based on how an application is running over the
    network. They can also enable the identification of unexpected sequences of
    commands.
  \item \textbf{Application-Proxy Gateways}: An application-proxy gateway is a
    feature of advanced firewalls that combines lower-layer access control with
    upper-layer functionality. It uses a proxy agent that acts as align
    intermediary between two hosts that wish to communicate with each other,
    without allowing a direct connection between them. The proxy is meant to be
    transparent to the two hosts and interfaces directly with the firewall
    ruleset to determine whether a given instance of network traffic should be
    allowed through the firewall. Typically, this offers a higher level of
    security that application firewalls.
  \item \textbf{Host Based Firewalls and Personal Firewalls}: Host based and
    personal firewalls provide an additional layer of security against
    network-based attacks. They monitor and control the incoming and outgoing
    network traffic for a single host. Host based firewalls are typically
    available as part of server operating systems and can also advance to
    intrusion prevent systems.
\end{itemize}

\subsection*{Architecture with Multiple Layers of Firewaalls}
The goal of multiple layers of firewalls is to provide defense in depth. It can
also help resolve issues where users internally have varying levels of trust. A
firewall between the access points and the rest of the internal network can
prevent visitors from accessing the local network with the same privileges as
an employee. Placing a firewall within a network that already has one at the
edge requires good planning and policy coordinate to prevent inadvertent
security lapses. This increase the difficulty in detecting firewall problems.

\subsection*{NIST Recommendations}
\begin{itemize}
  \item Firewalls should block all inbound and outbound traffic that has not
    been expressly permitted by the firewall policy.
  \item Traffic with invalid source or destination addresses should always be
    blocked at the network perimeter.
  \item Traffic from outside the network containing broadcast addresses
    directed inside the network should be blocked.
  \item Firewall policy should be based on comprehensive risk analysis.
  \item Many types of IPv4 addresses should be blocked by default.
  \item Organizations should have policies for incoming and outgoing IPv6
    traffic.
\end{itemize}

\subsection*{Firewall Planning and Implementation}
\begin{enumerate}
  \item \textbf{Plan}: Identify all requirements that an organization should
    consider.
  \item \textbf{Configure}: Install necessary hardware and software and set up
    rules.
  \item \textbf{Test}: Test a prototype of the installed solution in a test or
    lab environment to evaluate functionality, performance, scalability,
    security, and identify potential risks.
  \item \textbf{Deploy}: Deploy the firewall into the enterprise environment.
  \item \textbf{Manage}: Manage the firewall throughout its lifecycle, which
    maintenance and support for operational issues.
\end{enumerate}

\begin{center}
  You can find all my notes at \url{http://omgimanerd.tech/notes}. If you have
  any questions, comments, or concerns, please contact me at
  alvin@omgimanerd.tech
\end{center}

\end{document}
