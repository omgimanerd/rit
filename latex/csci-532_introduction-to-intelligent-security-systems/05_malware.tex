\documentclass{math}

\usepackage[utf8]{inputenc}

\title{Introduction to Intelligent Security Systems}
\author{Alvin Lin}
\date{August 2018 - December 2018}

\begin{document}

\maketitle

\section*{Malware}
Malware, also known as malicious code or malicous software, refers to a program
that is inserted into a system (usually covertly) with the intent of
compromising the confidentiality, integrity, or availability of the victim's
data, applications, or operating system. Malware has become the most
significant external threat to most systems, causing widespread damage and
disruption and necessitating extensive recovery efforts within most
organizations.

\subsection*{Viruses}
Viruses are programs embedded in files that spread and do damage. They are
typically comprised of a replicator which reproduces the virus, and a payload
which performs the malicious task.
\begin{itemize}
  \item Boot Sector Virus: infects the boot or master boot record of diskettes
    and hard drives through the sharing of infected disks and pirated software
    applications.
  \item Program Virus: becomes active when the program file carrying the virus
    is opened. It then makes copies of itself and infects other programs on the
    computer.
  \item Multipartite Virus: hybrid of a boot sector and program virus that
    infects program files and affects the boot record when activated.
  \item Stealth Virus: disguises itself to prevent detection by antivirus
    software. It alters its file size or conceals itself in memory.
  \item Polymorphic Virus: acts like a chameleon by changing its virus signature
    every time it multiplies and infects a new file.
  \item Macro Virus: programmed as a macro embedded in a document, usually found
    in Microsoft Word or Excel. Once a computer is infected, every document
    produced from the computer will become infected.
\end{itemize}

\subsection*{Malware Categories}
\begin{center}
  \begin{tabular}{|c|c|c|c|c|c|c|}
    \hline
    & Virus & Worm & Trojan Horse \\
    \hline
    Self-contained? & no & yes & yes \\
    Self-replicating? & yes & yes & no  \\
    Propagation method & User interaction & Self propagation & n/a \\
    \hline
  \end{tabular}
\end{center}
All forms introduce some undesired functionality into the infected host.
Viruses are generally hidden in code while trojans may be hidden in code but
are usually not. Generally worms and viruses self-propagate, but worms run
independently to consume the resources of its host while viruses cannot run
independently and require a host program to activate it. \par
Boot sector viruses infect the boot sector of hard drives, thus guaranteeing
they get executed and loaded into memory whenever the computer is turned on.
They have declined today due to the fact that operating systems now protect
the boot sector.

\subsection*{Basic Timeline}
\begin{itemize}
  \item 1949: Theories for self-replicating programs are first developed.
  \item 1981: Apple Viruses 1, 2, and 3 are some of the first viruses found in
    the public domain. They were found on the Apple II operating systems and
    propagated through Texas A\&M via pirated computer games.
  \item 1983: Fred Cohen, while working on his dissertation, formally defines a
    computer virus as ``a computer program that can affect other computer
    programs by modifying them in such a way as ot include a possible evolved
    copy of itself''.
  \item 1986: Two programmers named Basit and Amjad replace the executable code
    in the boot sector of a floppy disk with their own code designed to infect
    each 360kb floppy disk accessed on any drive.
  \item 1987: The Lehigh virus, one of the fist file viruses, infects
    command.com files.
  \item 1988: One of the most common viruses, Jerusalem, is unleashed. Activated
    every Friday the 13th, the virus affects both .exe and .com files and
    deletes any program run on that day.
  \item 1998: The first version of the CIH viruses developed by Chen Ing Hau
    from Taiwan are released.
  \item 2000: The virus ILOVEYOU is released, capable of deleting files in
    JPEGs, MP2, or MP3 formats.
  \item 2001: The Anna Kournikova is spread by emails through compromised
    address books of Microsoft Outlook. The emails were purported to contain
    pictures of the very attractive female tennis player, but in fact hid a
    malicious virus.
  \item 2006: OSX/Leap-A is the first ever known malware discovered against
    MAC OSX.
  \item 2013: Cryptolocker, a trojan horse that encrypts files and demands a
    ransom, is released.
  \item 2014: Backoff, a malware designed to compromise point-of-sale systems
    to steal credit card data, is released.
  \item 2017: The WannaCry randomware infects more than 230,000 computers in
    over 150 countries.
\end{itemize}

\subsubsection*{ILOVEYOU virus}
In 2000, the ILOVEYOU virus infected millions of computers virtually overnight
through email. It spread through email attachments and deleted all jpeg files
in all disks. The virus also sent passwords and usernames back to the author.
Authorities traced the virus back to a young Filipino computer student who went
free because the Phillippines at that itme had no laws against hacking. This
spurred the creation of the European Union's global Cybercrime Treaty.

\subsubsection*{Stuxnet}
Stuxnet is a highly sophisticated computer worm discovered in June 2010. It
initially spread via Microsoft Windows and targeted Siemens industrial software
and equipment. While it was not the first time hackers had target industrial
systems, it was the first discovered malware that spied on and subverted
industrial systems, and the first to include a programmable logic controller
rootkit. \par
Stuxnet has been spotted in Iran, Israel, the Palestinian territories, Syria,
and Lebanon. It is one of the most sophisticated viruses developed, able to
activate computer microphones, log keystrokes, and steal data. Given its size,
some have posited that the virus could only have been developed by the United
States or Israel.

\subsubsection*{Petya.A Ransomware}
Petya.A used a handful of different tools to move through a network and infect
the computers. It used a modified EternalBlue exploit and scrambled a user's
data using AES-128. The hackers demanded \$300 for the decryption key.

\subsection*{Vulnerabilities}
Vulnerabilities are not viruses, but could allow attackers to compromise the
integrity, availability, or confidentiality of a product. One example is the
Heartbleed OpenSSL vulnerability, which allowed attackers to leak data from
the memory heap of any computer accepting an OpenSSL request. The heap may
contain anything from random data to unencrypted data processed by OpenSSL,
which may include unencrypted web certificates or plain text usernames and
passwords.

\subsection*{Malware Detection}
Modern malware can morph to avoid detection by signature based antivirus
solutions. That means that today's antivirus solutions remain necessary for
catching known virus threats, but they're no longer sufficient because there is
no know pattern for detecting advanced attack. Strategies:
\begin{itemize}
  \item There is a clear distinction between data and executable. Since a virus
    must write to the program, only allow it to write to data so it cannot
    execute.
  \item Sandbox the virus so it runs in a protected area. Libraries and system
    calls are replaced with limited privilege versions.
  \item Malicious code usurps the authority of the user, so limiting the
    information flow limits the spread of the virus.
  \item Programs run with the minimal needed privilege.
  \item Use multi-level security mechanisms by putting programs at the lowest
    level and disallowing users from operating at that level.
  \item Look for a pattern in malicious code, which is always a cat and mouse
    game with the attacker.
  \item Maintain a checksum of the good file and check to see if it changed.
  \item Validate the action of an executable against some specification,
    including intermediate results and actions.
  \item Have the code carry some proof of correctness and verify the proof
    against the code during execution.
  \item Use statistical methods such as the number of files written, volume of
    data transferred, and the usage of CPU time.
  \item Scan the hard disk, memory, and boot for known virus signatures.
\end{itemize}

\begin{center}
  You can find all my notes at \url{http://omgimanerd.tech/notes}. If you have
  any questions, comments, or concerns, please contact me at
  alvin@omgimanerd.tech
\end{center}

\end{document}
