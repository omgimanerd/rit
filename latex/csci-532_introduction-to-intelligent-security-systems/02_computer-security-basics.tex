\documentclass{math}

\title{Introduction to Intelligent Security Systems}
\author{Alvin Lin}
\date{August 2018 - December 2018}

\begin{document}

\maketitle

\section*{Computer Security Basics}
Computer Security revolves around the CIA triad of \textbf{confidentiality},
\textbf{integrity}, and \textbf{availability}.
\begin{itemize}
  \item Confidentiality: the protection of information from unauthorized
    disclosure. Enforced through access controls, cryptography, and resource
    hiding.
  \item Integrity: the protection from unauthorized modification of information.
    This involves data integrity and origin integrity. Prevention mechanisms
    prevent unauthorized access and detection mechanisms report when information
    is no longer trustworthy.
  \item Availability: resources and services are usable and operational during
    a given time period despite attacks or failures.
\end{itemize}
Data can be made 100\% confidential simply by destroying it, but then it is
no longer available. Security involves availability because data must be
available to authorized individuals in order to be secure.

\subsection*{Why is security important?}
In the beginning, there was no external threat. Computer security was not an
issue. Now, nearly all devices are networked and attackers no longer have to
physically be at your computer in order to attack it.

\subsection*{Severity of Cyber Attacks}
According to the US CERT:
\begin{itemize}
  \item 4882 vulnerabilities were reported in 2005, with 79\% launched remotely,
    and 62\% leading to a disruption of service.
  \item 6604 vulnerabilities were reported in 2006, with 85\% launched remotely,
    and 65\% leading to a disruption of service.
\end{itemize}
In December 2006, NASA was forced to block emails with attachments before
shuttle launches out of fear that they would be hacked. Many other large
companies and organizations have also been breached, compromising large amounts
of data.

\subsection*{Security and Privacy}
In 2013, Edward Snowden copied and leaked classified information from the NSA.
His disclosures revealed numerous global surveillance programs run by the NSA in
cooperation with foreign governments and telecommunications agencies. \par
By now, we're all aware of the Equifax breach that exposed the social security
numbers and sensitive information of 143 million Americans. The breach was
discovered on July 29th, but hackers had breached their system from mid-May,
giving them a month to work with all the leaked data.

\subsection*{The Dark Web}
The Dark Web is a term that refers to a collection of websites that exist
on an encrypted network and can not be accessed using traditional web browsers.
Typically, these sites hide their identity using Tor. The rise of
cryptocurrency has lead to the growth of the dark web by allowing anonymized
money exchange.

\subsection*{Purposes of Cyber Attacks}
\begin{itemize}
  \item \textbf{Reconnaissance Attack}: an attempt to gather sensitive
    information about network services and systems. Ex: packet sniffers, ping
    sweep, port scan, queries regarding internet information.
  \item \textbf{Denial of Service Attack}: a network attack designed to slow
    down or crash a system by flooding it with useless traffic. Ex: ping of
    death, teardrop attack.
  \item \textbf{Access Attacks}: attacks try to uncover exploits and
    vulnerabilities in FTP, web services, and network authentication in order
    to get access to a system's network.
\end{itemize}
\textbf{Active attacks} allow the attacker to block the communication channel
between participants on a network or permit him to send data to all parties at
once. \textbf{Passive attacks} are when an intruder with unauthorized access
actively eavesdrops on a communication.

\subsubsection*{Attack Scope}
Malicious large scale attacks are offensive attempts to create chaos and disrupt
services. Non-malicious small-scale attacks are often unintentional attacks or
accidental damage due to human operational error that may cause system crashes
or data loss. \par
Any crime that encompasses a network or computer can be deemed as cyber crime.
The practice of gathering secrets with the consent of the information holder is
termed as cyber espionage. Terrorism in the cyberspace domain is classified
under cyber terrorism.

\subsection*{Threats Against Assets}
\begin{center}
  \begin{tabular}{|p{3cm}|p{4cm}|p{4cm}|p{4cm}|}
    \hline
    & Availability & Confidentiality & Integrity \\
    \hline
    Hardware & equipment is stolen or disabled & & parts are replaced without
      authorization \\
    \hline
    Software & code or programs are removed & an unauthorized copy of software
      is made and could be executed & code is modified, causing it to fail
      during execution or cause unintended side effects \\
    \hline
    Data & files are deleted or hidden, denying access to users & an
      unauthorized read of data is performed & existing files are modified or
      replaced with new files \\
    \hline
    Communication channels & messages are destroyed or deleted or the
      communication channel is brought down & messages are copied or read
      without authorization & messages are modified, delayed, reordered,
      duplicated, or fabricated \\
    \hline
  \end{tabular}
\end{center}

\subsection*{Network Security Issues}
Information must be protected when travelling across a network. Only
authorized access should be allowed to a node. Nodes handle security
appropriately within the node itself since firewalls don't provide complete
protection.

\begin{center}
  You can find all my notes at \url{http://omgimanerd.tech/notes}. If you have
  any questions, comments, or concerns, please contact me at
  alvin@omgimanerd.tech
\end{center}

\end{document}
