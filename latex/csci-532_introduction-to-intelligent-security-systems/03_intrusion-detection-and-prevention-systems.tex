\documentclass{math}

\title{Introduction to Intelligent Security Systems}
\author{Alvin Lin}
\date{August 2018 - December 2018}

\begin{document}

\maketitle

\section*{Intrusion Detection and Prevention Systems}
Phases of Intrusion:
\begin{itemize}
  \item Intelligence Gathering: attacker observes the system to determine
    vulnerabilities
  \item Planning: attacker decides what resource to attack (usually the
    least defended component)
  \item Attack: attacker carries out the plan
  \item Hiding: attacker covers tracks of attack
  \item Future Attacks: attack installs backdoors for future entry points
\end{itemize}
Intrusion detection and prevent has three major parts:
\begin{itemize}
  \item Intrusion Prevention: protect system resources
  \item Intrusion Detection: (second line of defense) discriminates between
    intrusion attempts and regular system usage
  \item Intrusion Recovery: effective recovery models
\end{itemize}
Many layers of defense are required because there are often many security flaws
in systems. Secure systems are often expensive and not user friendly, and there
is still the an insider threat. They need to be continually improved because
the skills and tools of hackers are also continually improving.

\subsection*{Terminology}
\begin{itemize}
  \item \textbf{Audit}: activity of looking at user/system behavior, its
    effects, or the collected data
  \item \textbf{Profiling}: looking at users or systems to determine what they usually do
  \item \textbf{Anomaly}: abnormal behavior
  \item \textbf{Misuse}: activity that violates the security policy
  \item \textbf{Outsider}: someone without access rights to the system
  \item \textbf{Insider}: someone with access rights to the system
  \item \textbf{Intrusion}: misuse by outsiders and insiders
\end{itemize}

\subsection*{IDS Operation}
\begin{itemize}
  \item \textbf{Host Based}: deployed on a specific host to monitor and gather
    information from that host
  \item \textbf{Network Based}: focuses on network attacks and attempts to
    identify unauthorized, illicit, and/or anomalous behavior based on network
    traffic patterns
  \item \textbf{Network Behavior Analysis}: an extension of network based
    intrusion detection systems that inspects the packets and network
    information gathered from routers and other physical devices on the network
  \item \textbf{Wireless}: examines the wireless network for suspcious activity
    and analyzes the wireless networking protocols
\end{itemize}
\begin{center}
  \begin{tabular}{|c|p{4.5cm}|p{4.5cm}|}
    \hline
    IDS & Main Components & Security Tasks \\
    \hline
    Host Based & DMZ Switch, Host IDPS Appliances, Agents & examines data flow
      and system files related to host \\
    \hline
    Network Based & Sensors, Load Balancer, Switch, Router & logging data,
      protocol analysis, inspects data in segments. Signature and anomaly-based
      detection \\
    \hline
    Network Behavior Analysis & NBA Management switch, console, and management
      server & examines packets from network segments \\
    \hline
    Wireless & sensors, management switches, and management server & sampling of
      traffic, logging information, wireless protocol analysis \\
    \hline
  \end{tabular} \\[2cm]
  \begin{tabular}{|c|p{4.5cm}|p{4.5cm}|}
    \hline
    IDS & Advantages & Limitations \\
    \hline
    Host Based & detects local attacks before they hit the network, bandwidth
      independent, low false positive rate & delay in detection, agents use the
      resources of hosts, installation of agents may conflict with existing
      security software \\
    \hline
    Network Based & no overload, signature and anomaly-based detection &
      inability to detect encrypted information, fails to analyze during high
      load \\
    \hline
    Network Behavior Analysis & excellent detection capabilities, ability to log
      information, can often determine attack origin & collects data from
      network devices, detects attacks after damage occurs \\
    \hline
    Wireless & able to monitor different types of attacks, can detect physical
      threat location, can perform prevention techniques & unable to detect
      passive monitoring or DDoS attacks, cannot do evasion techniques using
      channel scanning \\
    \hline
  \end{tabular}
\end{center}

\begin{center}
  You can find all my notes at \url{http://omgimanerd.tech/notes}. If you have
  any questions, comments, or concerns, please contact me at
  alvin@omgimanerd.tech
\end{center}

\end{document}
