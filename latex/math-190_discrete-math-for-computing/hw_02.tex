\documentclass[letterpaper, 12pt]{math}

\usepackage{amsmath}
\usepackage{amssymb}

\title{Homework \#2}
\author{Alvin Lin}
\date{Discrete Math for Computing: January 2017 - May 2017}

\begin{document}

\maketitle

\subsection*{2a}
If it snows tonight, then I will stay at home. \\
Converse: If I will stay at home, then it will snow tonight. \\
Contrapositive: If I will not stay at home, then it will not snow
tonight. \\
Inverse: If it will not snow tonight, then I will not stay at home.

\subsection*{2b}
I go to the beach whenever it is a sunny day. \\
Converse: It is a sunny day whenever I go to the beach. \\
Contrapositive: It is not a sunny day whenever I don't go to the beach. \\
Inverse: I don't go to the beach when it is not a sunny day.

\subsection*{2c}
A positive integer is a prime only if it has no divisors other than 1 and
itself. \\
Converse: If a positive integer has no divisors other than 1 and itself, it is
a prime. \\
Contrapositive: If a positive integer has divisors other than 1 and itself,
then it is not prime. \\
Inverse: A positive integer is not prime if it has divisors other than 1 and
itself.

\subsection*{3}
Is the following expression a tautology?
\[ \neg{p} \wedge (p \to q) \to \neg{q} \]
\begin{center}
  \begin{tabular}{|c|c|c|c|c|}
    \hline
    \( p \) & \( q \) & \( p \to q \) & \( \neg{p} \wedge (p \to q) \) &
    \( \neg{p} \wedge (p \to q) \to \neg{q} \) \\ \hline
    T & T & T & F & T \\ \hline
    T & F & F & F & T \\ \hline
    F & T & T & T & F \\ \hline
    F & F & T & F & T \\ \hline
  \end{tabular}
\end{center}
Nope.

\subsection*{4}
Show that \( (p \to r) \wedge (q \to r) \) and \( (p \vee q) \to r \) are
logically equivalent.
\begin{center}
  \begin{tabular}{|c|c|c|c|c|c|}
    \hline
    \( p \) & \( q \) & \( r \) & \( p \to r \) & \( q \to r \) &
    \( (p \to r) \wedge (q \to r) \) \\ \hline
    T & T & T & T & T & T \\ \hline
    T & T & F & F & F & F \\ \hline
    T & F & T & T & T & T \\ \hline
    T & F & F & F & T & F \\ \hline
    F & T & T & T & T & T \\ \hline
    F & T & F & T & F & F \\ \hline
    F & F & T & T & T & T \\ \hline
    F & F & F & T & T & T \\ \hline
  \end{tabular}
\end{center}
\begin{center}
  \begin{tabular}{|c|c|c|c|c|}
    \hline
    \( p \) & \( q \) & \( r \) & \( p \vee q \) & \( (p \vee q) \to r \)
        \\ \hline
    T & T & T & T & T \\ \hline
    T & T & F & T & F \\ \hline
    T & F & T & T & T \\ \hline
    T & F & F & T & F \\ \hline
    F & T & T & T & T \\ \hline
    F & T & F & T & F \\ \hline
    F & F & T & F & T \\ \hline
    F & F & F & F & T \\ \hline
  \end{tabular}
\end{center}


\end{document}
