\documentclass[letterpaper, 12pt]{math}

\usepackage{amsmath}
\usepackage{amssymb}

\title{Vectors and Matrices}
\author{Alvin Lin}
\date{Discrete Math for Computing: January 2017 - May 2017}

\begin{document}

\maketitle

\section*{Vectors and Matrices}
An \textbf{\(m\times n\) matrix} is a two-dimensional array of numbers
consisting of \( m \) rows and \( n \) columns. For example:
\[ A_{2\times3} =
  \begin{bmatrix}
    1 & 2 & 3 \\
    4 & 5 & 6
  \end{bmatrix}
\]
This is a \( 2\times3 \) matrix with \( a_{1,1} = 1 \) and \( a_{2,1} = 4 \).
In general, \( a_{i,j} \) is the row \( i \), column \( j \) entry of matrix
\( A \). An \( m\times n \) matrix has dimension \( m\times n \). If
\( n = m \), then we say that A is a \textbf{square matrix}.

\subsection*{Matrix Operations}
Suppose \( A \) and \( B \) are both \( m\times n \) matrices and \( c \) is
a real number.
\begin{itemize}
  \item Matrix Addition: \( A+B = C \) where \( c_{i,j} = a_{i,j}+b_{i,j} \)
  \item Matrix Subtraction: \( A-B = D \) where \( d_{i,j} = a_{i,j}-b_{i,j} \)
  \item Scalar Multiplication: \( c\cdot A = F \) where \( f_{i,j} = c\cdot
    a_{i,j} \)
\end{itemize}

\subsection*{Example}
\begin{align*}
  \begin{bmatrix}
    1 & 5 \\
    2 & -4
  \end{bmatrix}+
  \begin{bmatrix}
    3 & 2 \\
    -7 & 9
  \end{bmatrix}&=
  \begin{bmatrix}
    4 & 7 \\
    -5 & 5
  \end{bmatrix} \\
  \begin{bmatrix}
    1 & 0 \\
    15 & -3
  \end{bmatrix}-
  \begin{bmatrix}
    3 & 10 \\
    4 & 4
  \end{bmatrix}&=
  \begin{bmatrix}
    -2 & -10 \\
    11 & -7
  \end{bmatrix} \\
  4\cdot
  \begin{bmatrix}
    5 & 1 \\
    3 & 2 \\
    -1 & 10
  \end{bmatrix}&=
  \begin{bmatrix}
    20 & 4 \\
    12 & 8 \\
    -4 & 40
  \end{bmatrix}
\end{align*}

\subsection*{Vectors}
A \textbf{vector} is a one-dimensional array of numbers which may be thought
of as a \( 1\times n \) or \( n\times1 \) matrix. For example:
\[ \vec{v} = \begin{bmatrix} 1 \\ 2 \\ 3 \end{bmatrix} \]
This is a vector with \( v_{1} = 1 \), \( v_{2} = 2 \), and \( v_{3} = 3 \).
A vector is a 1-dimensional array of numbers which encodes magnitude and
direction. A vector with length 3 with a heading of \( 45^{\circ} \) in 2D
is equivalent to:
\[ \vec{v} =
  \begin{bmatrix}
    \frac{3\sqrt{2}}{2} \\
    \frac{3\sqrt{2}}{2}
  \end{bmatrix}
\]

\subsection*{Vector Operations}
\begin{itemize}
  \item Vector Length: \( |\vec{v}| =
    \sqrt{v_{1}^{2}+v_{2}^{2}+\dots+v_{n}^{2}} \)
  \item Dot Product: \( \vec{v}\bullet\vec{w} = v_{1}w_{1}+v_{2}w_{2}+\dots+
    v_{n}w_{n} \)
  \item Angle between Vectors: \( \vec{v}\bullet\vec{w} =
    |\vec{v}|\cdot|\vec{w}|\cdot\cos(\theta) \)
\end{itemize}

\begin{center}
  If you have any questions, comments, or concerns, please contact me at
  alvin@omgimanerd.tech
\end{center}

\end{document}
