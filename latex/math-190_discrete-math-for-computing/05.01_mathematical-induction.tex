\documentclass[letterpaper, 12pt]{math}

\usepackage{amsmath}
\usepackage{amssymb}

\title{Sets}
\author{Alvin Lin}
\date{Discrete Math for Computing: January 2017 - May 2017}

\begin{document}

\maketitle

\section*{Mathematical Induction}
Mathematical inductions prove statements that depend on a positive integer
\( n \). In general, we use it to prove \( \forall{n}P(n) \) where we
normally have \( \U = \Z\geq a \) for \( a = \{0,1,2,3,4,5,\dots\} \).
Unless otherwise specified, we assume \( a = 1 \).

\subsection*{Examples}
For all \( n > 0 \):
\[ \sum_{i=1}^{n}i = \frac{n(n+1)}{2} \]
For all \( n > 0 \):
\[ \sum_{i=1}^{n}i^{2} = \frac{n(n+1)(2n+1)}{6} \]
For all \( n > 0 \), 2 divides \( n^{2} + n \):
\[ n^{2}+n \equiv 0(\bmod\ 2) \]
For all \( n > 0 \):
\[ 4^{n+1}+5^{2n-1} \equiv 0(\bmod\ 21) \]

\subsection*{Process}
\begin{enumerate}
  \item Identify the exact \( P(n) \) statement.
  \item Write out \( P(k) \) in the induction hypothesis.
  \item Write out \( P(k+1) \) in the induction.
  \item Think about how \( P(k+1) \) relates to \( P(k) \).
  \item Once you apply the induction hypothesis in the induction, the rest is
    algebra.
\end{enumerate}
With the math induction argument, we have:
\begin{align*}
  & P(1) \\
  & \underline{\forall{k}P(k)\to P(k+1)} \\
  \therefore{} & \forall{n}P(n)
\end{align*}
Using Modus Ponens:
\begin{align*}
  & P(1) \\
  & \underline{P(1)\to P(2)} \\
  \therefore{} & P(2)
\end{align*}
By hypothetical syllogism:
\begin{align*}
  & P(1)\to P(2) \\
  & \underline{P(2)\to P(3)} \\
  \therefore{} & P(1)\to P(3) \\
\end{align*}

\subsection*{Example}
Prove \( \forall{n}\sum_{i=1}^{n}(2i-1) = n^{2} \).
\begin{enumerate}
  \item Basis: \( P(n) = \sum_{i=1}^{n}(2i-1) = n^{2} \)
  \item Show \( P(1) \) is true:
    \begin{align*}
      \sum_{i=1}^{1}(2i-1) &= 1^{1} \\
      (2\cdot1-1) &= 1^{2} \\
      2-1 &= 1
    \end{align*}
  \item Induction hypothesis, assume \( P(k) \) is true for some \( k\geq 1 \).
    Assume \( \sum_{i=1}^{k}(2i-1) = k^{2} \).
  \item Induction (we need to prove this):
    \begin{align*}
      P(k+1) = \sum_{i=1}^{k+1}(2i-1) &= (k+1)^{2} \\
      \sum_{i=1}^{k+1}(2i-1) &= k^{2}+2k+1
    \end{align*}
  \item Proof:
    \begin{align*}
      \sum_{i=1}^{k+1}(2i-1) &= \sum_{i=1}^{k}(2i-1)+\sum_{k}^{k+1}(2i-1) \\
      &= \sum_{i=1}^{k}(2i-1)+(2(k+1)-1) \\
      &= k^{2}+(2k+2-1) \\
      &= k^{2}+2k-1
    \end{align*}
\end{enumerate}

\subsection*{Example}
Prove \( \forall{n}, n^{3}-n \equiv 0 (\bmod\ 3) \). Let \( P(n) \) be the idea
\( n^{3}-n \equiv 0 (\bmod\ 3) \).
\begin{enumerate}
  \item Basis: \( P(0) \) is the statement \( 0^{3}-0 \equiv 0 (\bmod\ 3) \).
    We get \( 0 \equiv 0(\bmod\ 3) \) from simplifying this statement, which is
    true.
  \item Induction hypothesis: assume that \( P(k) \) is true for some
    \( k\geq0 \). This means that \( k^{3}-k \equiv 0(\bmod\ 3) \) is true.
  \item Induction (we need to prove this):
    \[ (k+1)^{3}=(k+1) \equiv 0\ (\bmod\ 3) \]
  \item Proof:
    \begin{align*}
      (k+1)^{3}=(k+1) &\equiv k^{3}+3k^{2}+3k+1-k-1\ (\bmod\ 3) \\
      &\equiv (k^{3}-k^{2})+3k^{2}+3k \\
      &\equiv 0+0k^{2}+0k \equiv 0
    \end{align*}
\end{enumerate}

\begin{center}
  If you have any questions, comments, or concerns, please contact me at
  alvin@omgimanerd.tech
\end{center}

\end{document}
