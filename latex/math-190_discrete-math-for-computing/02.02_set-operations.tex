\documentclass[letterpaper, 12pt]{math}

\usepackage{amsmath}
\usepackage{amssymb}
\usepackage{tikz}

\title{Sets}
\author{Alvin Lin}
\date{Discrete Math for Computing: January 2017 - May 2017}

\begin{document}

\maketitle

\section*{Set Operations}

\subsection*{Union}
\[ A \cup B = \{x\ |\ x \in A \vee x \in B\} \]
Let \( A = \{1,2\} \quad B = \{3,4\} \):
\[ A \cup B = \{1,2,3,4\} \]

\subsection*{Intersection}
\[ A \cap B = \{x\ |\ x \in A \wedge x \in B\} \]
Let \( A = \{1,2,3\} \quad B = \{3,4,5\} \):
\[ A \cap B = \{3\} \]

\subsection*{Set Difference}
\[ A-B = \{x\ |\ x \in A \wedge x \notin B\} \]
Let \( A = \{a,b,c\} \quad B = \{b,c,d\} \):
\[ A-B = \{a\} \]

\subsection*{Set Complements}
Let \( u \) be the universal set. The \textbf{complement} of the set \( A \),
denoted \( \overline{A} \), is the set \( u-A \). An element belongs in \( \overline{A} \)
if \( x \notin A \).

\subsection*{Set Identities}
\begin{enumerate}
  \item Identity Laws
  \begin{itemize}
    \item \( A \cap u = A \)
    \item \( A \cup \varnothing = A \)
  \end{itemize}
  \item Domination
  \begin{itemize}
    \item \( A \cup u = u \)
    \item \( A \cap \varnothing = \varnothing \)
  \end{itemize}
  \item Idempotent
  \begin{itemize}
    \item \( A \cup A = A \)
    \item \( A \cap A = A \)
  \end{itemize}
  \item Complementation
  \begin{itemize}
    \item \( \overline{(\overline{A})} = A \)
  \end{itemize}
  \item Commutative
  \begin{itemize}
    \item \( A \cup B = B \cup A \)
    \item \( A \cap B = B \cap A \)
  \end{itemize}
  \item Associative
  \begin{itemize}
    \item \( A \cup (B \cup C) = (A \cup B) \cup C \)
    \item \( A \cap (B \cap C) = (A \cap B) \cap C \)
  \end{itemize}
  \item Distribution
  \begin{itemize}
    \item \( A \cup (B \cap C) = (A \cup B) \cap (A \cup C) \)
    \item \( A \cap (B \cup C) = (A \cap B) \cup (A \cap C) \)
  \end{itemize}
  \item De Morgan's
  \begin{itemize}
    \item \( \overline{A \cap B} = \overline{A} \cup \overline{B} \)
    \item \( \overline{A \cup B} = \overline{A} \cap \overline{B} \)
  \end{itemize}
  \item Absorption
  \begin{itemize}
    \item \( A \cup (A \cap B) = A \)
    \item \( A \cap (A \cup B) = A \)
  \end{itemize}
  \item Complement
  \begin{itemize}
    \item \( A \cup \overline{A} = u \)
    \item \( A \cap \overline{A} = \varnothing \)
  \end{itemize}
\end{enumerate}

\subsection*{Example}
Show using set identities that:
\[ \overline{A \cup (B \cap C)} = (\overline{C} \cup \overline{B}) \cap
   \overline{A} \]
\begin{align*}
  \overline{A \cup (B \cap C)} &=
    \overline{A} \cap \overline{B \cap C} \quad (De\ Morgan's) \\
  &= \overline{A} \cap (\overline{B} \cup \overline{C}) \quad (De\ Morgan's) \\
  &= (\overline{B} \cup \overline{C}) \cap \overline{A} \quad (Commutative) \\
  &= (\overline{C} \cup \overline{B}) \cap \overline{A} \quad (Commutative)
\end{align*}

\begin{center}
  If you have any questions, comments, or concerns, please contact me at
  alvin@omgimanerd.tech
\end{center}

\end{document}
