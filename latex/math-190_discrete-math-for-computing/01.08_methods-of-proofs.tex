\documentclass[letterpaper, 12pt]{math}

\usepackage{amsmath}
\usepackage{amssymb}

\title{Methods of Proofs}
\author{Alvin Lin}
\date{Discrete Math for Computing: January 2017 - May 2017}

\begin{document}

\maketitle

\section*{Methods of Proofs}

\subsection*{Exhaustive Proofs and Proof by Cases}
Sometimes, to prove \( p \to q \) is true, it is convenient to use
\( P_{1} \vee P_{2} \vee P_{3} \vee \dots \vee P_{n} \) instead of \( p \)
as a hypothesis. These types of proofs are called \textbf{proofs by
exhaustion}. \par
A \textbf{proof by cases} is one which covers all possible cases that arise in
a theorem.

\subsection*{An Exhaustive Proof}
Prove that \( (n+1)^{3} \geq 3^{n} \) if \( n \) is a positive integer
\( n \leq 4 \). Proof:
\begin{itemize}
  \item \( n = 1: 2^{3} = 8 \geq 3^{1} \)
  \item \( n = 2: 3^{3} = 27 \geq 3^{2} \)
  \item \( n = 3: 4^{3} = 64 \geq 3^{3} \)
  \item \( n = 4: 5^{3} = 125 \geq 3^{4} \)
\end{itemize}
So, by exhaustion of all cases, this proposition is true for \( a = 1,2,3,4 \).

\subsection*{A Proof By Cases}
If \( n \) is an integer, \( n^{2} \geq n \). Proof:
\begin{itemize}
  \item Case 1: \( n = 0 \)
    \[ 0^{2} = 0 \geq 0 \]
  \item Case 2: \( n > 0 \)
    \begin{align*}
      n &\geq 1 \\
      (n)(n) &\geq (1)(n) \\
      n^{2} &\geq n
    \end{align*}
  \item Case 3: \( n \leq -1 \)
    \[ n^{2} \geq 0 \]
    \[ \therefore n^{2} \geq n \]
\end{itemize}
Since all integers, fall into 1 of these cases, we conclude our proof.

\subsection*{Example}
Recall:
\[ |a| =
  \begin{cases}
    & a\ if\ a \geq 0 \\
    &-a\ if\ a < 0
  \end{cases}
\]
Prove by cases that: \( |xy| = |x||y| \). There are four cases to consider.
\begin{itemize}
  \item Case 1: \( x \geq 0, y \geq 0 \)
    \begin{align*}
      xy &\geq 0 \\
      |xy| &= xy \\
      &= |x||y|
    \end{align*}
  \item Case 2: \( x > 0, y < 0 \)
    \begin{align*}
      xy &< 0 \\
      |xy| &= -(xy) \\
      &= x(-y) \\
      &= |x||y|
    \end{align*}
  \item Case 3: \( x < 0, y > 0 \) \\
    Without loss of generality, this is the same as case 2.
  \item Case 4: \( x < 0, y < 0 \)
    \begin{align*}
      xy &\geq 0 \\
      |xy| &= xy \\
      &= (-x)(-y) \\
      &= |x||y|
    \end{align*}
\end{itemize}
In general, using without loss of generality states that the proof of a current
case is identical to a previous case (perhaps with variables switched).

\begin{center}
  If you have any questions, comments, or concerns, please contact me at
  alvin@omgimanerd.tech
\end{center}

\end{document}
