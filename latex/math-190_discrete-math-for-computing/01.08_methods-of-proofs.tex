\documentclass{math}

\title{Methods of Proofs}
\author{Alvin Lin}
\date{Discrete Math for Computing: January 2017 - May 2017}

\begin{document}

\maketitle

\section*{Methods of Proofs}

\subsection*{Exhaustive Proofs and Proof by Cases}
Sometimes, to prove \( p \to q \) is true, it is convenient to use
\( P_{1} \vee P_{2} \vee P_{3} \vee \dots \vee P_{n} \) instead of \( p \)
as a hypothesis. These types of proofs are called \textbf{proofs by
exhaustion}. \par
A \textbf{proof by cases} is one which covers all possible cases that arise in
a theorem.

\subsection*{An Exhaustive Proof}
Prove that \( (n+1)^{3} \geq 3^{n} \) if \( n \) is a positive integer
\( n \leq 4 \). Proof:
\begin{itemize}
  \item \( n = 1: 2^{3} = 8 \geq 3^{1} \)
  \item \( n = 2: 3^{3} = 27 \geq 3^{2} \)
  \item \( n = 3: 4^{3} = 64 \geq 3^{3} \)
  \item \( n = 4: 5^{3} = 125 \geq 3^{4} \)
\end{itemize}
So, by exhaustion of all cases, this proposition is true for \( a = 1,2,3,4 \).

\subsection*{A Proof By Cases}
If \( n \) is an integer, \( n^{2} \geq n \). Proof:
\begin{itemize}
  \item Case 1: \( n = 0 \)
    \[ 0^{2} = 0 \geq 0 \]
  \item Case 2: \( n > 0 \)
    \begin{align*}
      n &\geq 1 \\
      (n)(n) &\geq (1)(n) \\
      n^{2} &\geq n
    \end{align*}
  \item Case 3: \( n \leq -1 \)
    \[ n^{2} \geq 0 \]
    \[ \therefore n^{2} \geq n \]
\end{itemize}
Since all integers, fall into 1 of these cases, we conclude our proof.

\subsection*{Example}
Recall:
\[ |a| =
  \begin{cases}
    & a\ if\ a \geq 0 \\
    &-a\ if\ a < 0
  \end{cases}
\]
Prove by cases that: \( |xy| = |x||y| \). There are four cases to consider.
\begin{itemize}
  \item Case 1: \( x \geq 0, y \geq 0 \)
    \begin{align*}
      xy &\geq 0 \\
      |xy| &= xy \\
      &= |x||y|
    \end{align*}
  \item Case 2: \( x > 0, y < 0 \)
    \begin{align*}
      xy &< 0 \\
      |xy| &= -(xy) \\
      &= x(-y) \\
      &= |x||y|
    \end{align*}
  \item Case 3: \( x < 0, y > 0 \) \\
    Without loss of generality, this is the same as case 2.
  \item Case 4: \( x < 0, y < 0 \)
    \begin{align*}
      xy &\geq 0 \\
      |xy| &= xy \\
      &= (-x)(-y) \\
      &= |x||y|
    \end{align*}
\end{itemize}
In general, using without loss of generality states that the proof of a current
case is identical to a previous case (perhaps with variables switched).

\subsection*{Example}
Show that if \( x \) and \( y \) are integers and both \( xy \) and \( x+y \)
are even, then both \( x \) and \( y \) are even. Proof:
\begin{itemize}
  \item Suppose \( x \) and \( y \) are both not even. \( x \) is odd or \( y \)
    is odd (or both).
  \item Without loss of generality, assume that \( x \) is odd. Thus
    \( x = 2k+1; k\in\Z \).
  \item To show \( xy \) is odd or \( x+y \) is odd, consider two cases:
    \begin{enumerate}
      \item \( y \) is even. Then \( y = 2n; n\in\Z \).
      \[ x+y = 2k+1+2n = 2(k+n)+1 \]
      \( x+y \) is odd.
      \item \( y \) is odd. Then \( y = 2m+1; m\in\Z \).
      \[ xy = (2k+1)(2m+1) = 2km+2k+2m+1 = 2(km+k+m)+1 \]
      \( xy \) is odd.
    \end{enumerate}
\end{itemize}

\subsection*{Uniqueness Proofs}
To show uniqueness, we show if \( x \neq y \) then \( y \) does not have some
desired property.

\subsection*{Example}
Show if \( a,b \) are real numbers, with \( a \neq 0 \), then there exists a
unique real number \( r \) such that \( ar+b = 0 \).

\subsubsection*{Proof of Existence}
Let \( r = \frac{-b}{a} \). Then \( ar+b = a(\frac{-b}{a})+b = 0 \). So such an
\( r \) exists.

\subsubsection*{Proof of Uniqueness}
Suppose there is some \( s \) such that \( as+b = 0 \). \( ar+b = as+b \) where
\( r = \frac{b}{-a} \). Therefore, \( ar = as \) and \( r = s \). If
\( r \neq s \) then \( as+b \neq 0 \). So \( r = \frac{-b}{a} \) is the only
such solution.

\subsection*{Example}
Show if \( x + y \geq 2 \) (where \( x \) and \( y \) are real numbers) then
\( x \geq 1 \) or \( y \geq 1 \). Proof by contraposition:
\begin{itemize}
  \item Suppose \( \neg{(x \geq 1\ or\ y \geq 1)} \).
  \item Then \( x < 1 \) and \( y < 1 \).
  \item We must show that \( x+y < 2 \), but \( x < 1 \) and \( y < 1 \).
  \item Then \( x+y < 1 \).
\end{itemize}

\subsection*{Example}
Show that \( n^{2}+1 \geq 2^{n} \) for any positive integer \( n \leq 4 \).
Proof by exhaustion: Verify that \( n^{2}+1 \geq 2^{n} \) for \( n = 1,2,3,4 \).
\begin{enumerate}
  \item \( n = 1: 1^{2}+1 \geq 2^{1} \)
  \item \( n = 2: 2^{2}+1 \geq 2^{2} \)
  \item \( n = 3: 3^{2}+1 \geq 2^{3} \)
  \item \( n = 4: 4^{2}+1 \geq 2^{4} \)
\end{enumerate}

\subsection*{Example}
Show that every odd integer is a difference of two squares. Proof:
\begin{itemize}
  \item Let \( n = 2k+1; k\in\Z \).
    \begin{align*}
      n &= 2k+1+k^{2}-k^{2} \\
      n &= k^{2}+2k+1-k^{2} \\
      n &= (k+1)^{2}-k^{2}
    \end{align*}
  \item Thus, we've expressed \( n \) as a difference of two squares.
\end{itemize}

\begin{center}
  You can find all my notes at \url{http://omgimanerd.tech/notes}. If you have
  any questions, comments, or concerns, please contact me at
  alvin@omgimanerd.tech
\end{center}

\end{document}
