\documentclass[letterpaper, 12pt]{math}

\title{Introduction to Proofs}
\author{Alvin Lin}
\date{Discrete Math for Computing: January 2017 - May 2017}

\begin{document}

\maketitle

\section*{Introduction to Proofs}
Rough definitions/guidelines:
\begin{itemize}
  \item A \textbf{theorem} is a statement that can be shown to be true.
  \item A \textbf{proposition} is a ``less important'' theorem.
  \item A \textbf{lemma} is used as a tool for proving other results.
\end{itemize}
We show that a theorem is true by using a \textbf{proof}, which is a valid
argument that establishes the truth of the theorem. To prove a theorem
of the form \( \exists{x}(P(x) \to Q(x)) \) we show that \( P(c) \to Q(c) \)
where \( c \) is an \textit{arbitrary} element of the domain. Since \( c \)
is arbitrary, we can conclude \( \exists{x}(P(x) \to Q(x)) \).

\subsection*{Direct Proofs}
A direct proof of \( p \to q \) is constructed when the first step is the
assumption that p is true. Subsequent steps use rules of inference. Finally,
we show that q is true.

\subsection*{Example}
Give a direct proof of the theorem: ``If an is an odd integer, then \( n^{2} \)
is odd''.
\begin{itemize}
  \item An integer \( n \) is \textbf{even} if there is an integer \( k \) such
  that \( n = 2k \).
  \item An integer \( n \) is \textbf{odd} if there is an integer \( k \) such
  that \( n = 2k+1 \).
  \item Any number is either even or odd.
\end{itemize}
\[ \exists{n}(n = 2k+1) \to (n^{2} = 2k'+1) \]
Proof:
\begin{itemize}
  \item Suppose \( n \) is an odd integer.
  \item By definition, there is a \( k \in \Z \) such that \( n = 2k+1 \).
  \item Consider \( n^{2} = (2k+1)^{2} = 4k^{2}+4k+1 = 2(2k^{2}+2k)+1 \).
  \item \( 2k^{2}+2k \) is an integer, so \( n^{2} = 2k'+1 \) where
    \( k' = 2k^{2}+2k \).
  \item Thus, \( n^{2} \) is odd.
\end{itemize}

\subsection*{Example}
Prove the theorem: ``If \( m \) and \( n \) are perfect squares then \( nm \)
is also a perfect square''. \\
\textbf{Definition:} An integer \( a \) is a \textbf{perfect square} if there
exists an integer b so that \( a = b^{2} \). \\
Proof:
\begin{itemize}
  \item Suppose \( m \) and \( n \) are arbitrary perfect squares.
  \item By definition, \( m = s^{2} \), \( n = t^{2} \) for some \( s \) and
    \( t \).
  \item Consider \( nm = s^{2}t^{2} = (st)^{2} \).
  \item \( st \) is an integer, therefore \( nm \) is a perfect square.
\end{itemize}

\subsection*{Proof by Contraposition}
If \( n \) is an integer and \( 3n+2 \) is odd, then \( n \) is odd. Proof:
\begin{itemize}
  \item Suppose that \( n \) is an integer and \( 3n+2 \) is odd.
  \item Then \( 3n+2 = 2k+1 \) for some integer \( k \).
  \item Thus \( 3n = 2k-1 \) and \( n = \frac{2k-1}{3} \), which leaves us
    stuck.
\end{itemize}
It is more advantageous for us to prove the contrapositive.
\begin{itemize}
  \item Suppose that \( n \) is even. We must show that \( 3n+2 \) is even.
  \item Since \( n \) is even, \( n = 2k \) for some integer \( k \).
  \item This implies that \( 3n+2 = 3(2k)+2 = 2(3k+1) \).
  \item Since \( 3k+1 \) is an integer, \( 3n+2 = 2(3k+1) \) is even.
  \item Thus if \( n \) is even, \( 3n+2 \) is even (contrapositive).
\end{itemize}

\subsection*{Theorem}
If \( n = ab \) where \( a,b \) are positive integers, then \( a \leq \sqrt{n}
\) or \( b \leq \sqrt{n} \). Proof:
\begin{itemize}
  \item Assume \( \neg(a \leq \sqrt{n} or\ b \leq \sqrt{n}) \).
  \item Thus, \( a > \sqrt{n} and\ b > \sqrt{n} \) (De Morgan's Law).
  \item We need to show that \( n \neq ab \).
  \item Consider that \( ab > \sqrt{n}\sqrt{n} = n \).
  \item Thus, \( ab > n \) and \( ab \neq n \).
\end{itemize}

\subsection*{Vacuous Proofs}
We can show that \( p \to q \) is true when \( p \) is false since \( p \to q \)
is always true when \( p \) is false. If we show that \( p \) is false, then
this is called \textbf{vacuous proof}.

\subsection*{Example}
Show that the proposition \( P(0) \) is true when \( P(n) \): ``If \( n > 1 \),
then \( n^{2} > n \)'' and the domain is all integers. \par
\( P(0) \): ``If \( 0 > 1 \), then \( 0^{2} > 0 \)'' is true since
\( \neg{(0 > 1)} \).

\subsection*{Example}
To prove \( \exists{x}P(x) \to G(x) \) try to see if a direct proof is
promising. If not, try a proof by contraposition. \par
\textbf{Definition:} The real number \( r \) is rational if \( r = \frac{p}{q}
\) where \( p,q \) are integers and \( q \neq 0 \). \\
\textbf{Theorem 1:} The sum of two rationals is rational.
\begin{itemize}
  \item Assume that \( r \) and \( s \) are rational. We must show that
    \( r+s \) is rational as well.
  \item \( r = \frac{p}{q}; p,q \in \Z; q \neq 0 \)
  \item \( s = \frac{t}{u}; t,u \in \Z; u \neq 0 \)
  \item Consider:
    \begin{align*}
      r+s &= \frac{p}{q}+\frac{t}{u} \\
      &= \frac{pu+tq}{qu}
    \end{align*}
  \item Since \( pu+tq \) is an integer and \( qu \) is a nonzero integer,
    \( \frac{pu+tq}{qu} \) is rational.
\end{itemize}
\textbf{Theorem 2:} If n is an integer, and \( n^{2} \) is odd, then n is odd.
\begin{itemize}
  \item Suppose that \( n \) is even.
  \item Then \( n =2k \) for some integer \( k \).
  \item Thus:
    \begin{align*}
      n^{2} &= (2k)^{2} \\
      &= 2(2k^{2})
    \end{align*}
  \item \( n^{2} \) is even (contrapositive).
\end{itemize}

\subsection*{The Classic 2=1 Proof}
Let:
\begin{align*}
  a &= b \\
  a^{2} &= ab \\
  a^{2}-b^{2} &= ab-b^{2} \\
  (a-b)(a+b) &= b(a-b) \\
  a+b &= b \\
  2b &= b \\
  2 &= 1
\end{align*}
This logic is flawed because we are dividing by zero. Since \( a = b \),
\( a-b = 0 \), thus step 4 is invalid.

\subsection*{Proof by Contradiction}
Proof by contradition proves \( p \) by assuming \( \neg{p} \) is true and
showing that ``something bad'' happens. \par
To prove a statement \( p \) is true, suppose we can find a contradiction
\( q \) such that \( \neq{p} \to q \) is true. Because \( q \) is false,
but \( \neg{p} \to q \) is true, it must be that \( \neg{p} \) is false
(i.e. \( p \) is true). \par
Statements \( r \wedge \neg{r} \) are contradictions whenever \( r \) is a
proposition. Thus we can show \( p \) is true if \( \neg{p} \to
(r \wedge \neg{r}) \).

\subsection*{Example}
\textbf{Theorem:} \( \sqrt{2} \) is irrational.
\begin{itemize}
  \item Assume that \( \sqrt{2} \) is rational.
  \item Then \( \sqrt{2} = \frac{p}{q}; p,q \in \Z; q \neq 0 \) with \( p,q \)
    having no common factors.
  \item Consider:
    \begin{align*}
      (\sqrt{2})^{2} &= \frac{p^{2}}{q^{2}} \\
      2 &= \frac{p^{2}}{q^{2}} \\
      2q^{2} = p^{2}
    \end{align*}
  \item Thus \( p^{2} \) is even.
  \item \textbf{Lemma:} If \( a^{2} \) is even, then \( a \) is even.
  \item The lemma gives that \( p \) is even. \( p = 2c; c \in \Z \).
  \item This implies that:
    \begin{align*}
      2q^{2} &= (2c)^{2} \\
      q^{2} &= 2c^{2}
    \end{align*}
  \item By the lemma, \( q \) is even.
  \item This is absurd, since \( p \) and \( q \) are both even they have
    the factor \( 2 \) in common. This violates our assumption that they
    have no common factors. Thus, contradiction. \( \sqrt{2} \) cannot be
    rational, therefore it must be rational.
\end{itemize}

\subsection*{Example}
Given a proof by contradiction that ``If \( 3n+2 \) is odd, then n is odd''.
Proof:
\begin{itemize}
  \item Let \( p: ``3n+2\ is\ odd'' \) and \( q: ``n\ is\ odd'' \).
  \item Assume \( p \) and \( \neg{q} \) are true for a proof by contradiction.
  \item \( 3n+2 \) is odd and \( n \) is even.
  \item By definition, \( n = 2k \) for some integer \( k \).
  \item Now consider \( 3n+2 = 3(2k)+2 = 2(3k+1) \).
  \item Thus, \( 3n+2 \) is even.
  \item But that is a contradiction since we assumed that \( 3n+2 \) was odd.
\end{itemize}

\subsection*{Example}
To prove biconditionals \( p \leftrightarrow q \), we must show \( p \to q \)
and \( q \to p \). Show that the statements are equivalent:
\begin{align*}
  P_{1}&: n\ is\ even \\
  P_{2}&: n-1\ is\ odd \\
  P_{3}&: n^{2}\ is\ even
\end{align*}
We must show \( p_{1} \to p_{2}, p_{2} \to p_{3}, p_{3} \to p_{1} \).
\[ p_{1} \to p_{2} \]
\begin{itemize}
  \item Assume that \( n \) is even. Then \( n = 2k \) for some integer \( k \).
  \item But \( n-1 = 2(k-1)+1 \).
  \item Therefore, \( n-1 \) is odd since \( k-1 \) is an integer.
  \item Thus, \( p_{1} \to p_{2} \).
\end{itemize}
\[ p_{2} \to p_{3} \]
\begin{itemize}
  \item Assume \( n-1 \) is odd. Then \( n-1 = 2k+1 \) for some integer \( k \).
  \item Consider:
    \begin{align*}
      n &= 2k+2 \\
      n^{2} &= (2k+2)^{2} \\
      &= 4k^{2}+8k+4 \\
      &= 2(2k^{2}+4k+2)
    \end{align*}
  \item Thus, \( n^{2} \) is even and \( p_{2} \to p_{3} \).
\end{itemize}
\[ p_{3} \to p_{1} \]
\begin{itemize}
  \item \textbf{Contrapositive:} If \( n \) is not even, then \( n^{2} \) is not
    even.
  \item Thus, \( n \) is odd, so \( n = 2k+1 \) for some integer \( k \).
  \item Consider:
    \begin{align*}
      n^{2} &= (2k+1)^{2} \\
      &= 4k^{2}+4k+1 \\
      &= 2(2k^{2}+2k)+1\ (odd) \\
    \end{align*}
  \item Therefore, \( p_{3} \to p_{1} \).
\end{itemize}
The statements \( p_{1}, p_{2}, p_{3} \) are all equivalent.

\subsection*{Counterexamples}
To show a statement \( \exists{x}P(x) \) is false, it suffices to find one such
\( x \) such that \( P(x) \) does not hold. This \( x \) is called a
\textbf{counterexample}.

\subsection*{Example}
Prove or disprove: ``Every positive integer is the sum of two squares of
integers''. Look for a counterexample:
\begin{itemize}
  \item \( 1 = 1^{2}+0^{2} \)
  \item \( 2 = 1^{1}+1^{2} \)
  \item \( 3 = \dots \)
\end{itemize}
\( x = 3 \) is a counterexample to this proposition.

\begin{center}
  You can find all my notes at \url{http://omgimanerd.tech/notes}. If you have
  any questions, comments, or concerns, please contact me at
  alvin@omgimanerd.tech
\end{center}

\end{document}
