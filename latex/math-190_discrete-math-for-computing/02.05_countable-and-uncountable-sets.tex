\documentclass[letterpaper, 12pt]{math}

\usepackage{amsmath}
\usepackage{amssymb}

\title{Sets}
\author{Alvin Lin}
\date{Discrete Math for Computing: January 2017 - May 2017}

\begin{document}

\maketitle

\section*{Countable and Uncountable Sets}
Recall for a set \( S \), \( |S| \) is the cardinality of \( S \), If \( S \)
has a finite number of objects, then \( |S|\in N \). Consider the following
six sets:
\begin{itemize}
  \item \( E \) is the set of positive even numbers less than or equal
    to 100.
  \item \( \Z \) is the set of all integers.
  \item \( U \) is the set of all real numbers between 0 and 1.
  \item \( P \) is the set of all prime numbers less than or equal to
    50.
  \item \( \Q \) is the set of all rational numbers.
  \item \( \R \) is the set of all real numbers.
\end{itemize}
Both \( E \) and \( P \) are finite sets with \( |E| = 50 \) and \( |P| = 15 \).
The remaining four sets are all infinite. If \( f:A\to B \) is 1-1, then
\( |A|\leq|B| \). Recall from functions that \( f \) is 1-1 if every output
is uniquely associated to 1 input. Also, for real-valued functions, \( f \)
is 1-1 if it passed the horizontal line test. If \( A\subseteq B \), then
\( |A|\leq|B| \) by using \( f(x) = x \) where \( f:A\to B \).

\subsection*{Schr{\"o}der-Bernstein Theorem}
If \( f:A\to B \) is 1-1 and \( g:B\to A \) is 1-1, then \( |A| = |B| \).
Using the above example, \( |U| = |\R| \) since \( f(x) =
\frac{\arctan(x)+\frac{\pi}{2}}{\pi} \) where \( f:\R\to U \) is 1-1 and
\( g(x) = x \) where \( g:U\to\R \) is 1-1.

\subsection*{Infinite Sets}
Two sets \( A \) and \( B \) have the same \textbf{cardinality}, written
\( A\sim B \), if there is a bijection between them.
Let \( A \) be a set:
\begin{itemize}
  \item \( A \) is \textbf{countable} if it is finite: \( A \sim \Z^{+} \)
  \item \( A \) is \textbf{countably infinite} if \( A \sim \Z^{+} \)
  \item \( A \) is \textbf{uncountable} if it is not countable.
\end{itemize}

\begin{center}
  If you have any questions, comments, or concerns, please contact me at
  alvin@omgimanerd.tech
\end{center}

\end{document}
