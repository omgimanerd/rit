\documentclass[letterpaper, 12pt]{math}

\title{Sets}
\author{Alvin Lin}
\date{Discrete Math for Computing: January 2017 - May 2017}

\begin{document}

\maketitle

\section*{Countable and Uncountable Sets}
Recall for a set \( S \), \( |S| \) is the cardinality of \( S \), If \( S \)
has a finite number of objects, then \( |S|\in N \). Consider the following
six sets:
\begin{itemize}
  \item \( E \) is the set of positive even numbers less than or equal
    to 100.
  \item \( \Z \) is the set of all integers.
  \item \( U \) is the set of all real numbers between 0 and 1.
  \item \( P \) is the set of all prime numbers less than or equal to
    50.
  \item \( \Q \) is the set of all rational numbers.
  \item \( \R \) is the set of all real numbers.
\end{itemize}
Both \( E \) and \( P \) are finite sets with \( |E| = 50 \) and \( |P| = 15 \).
The remaining four sets are all infinite. If \( f:A\to B \) is 1-1, then
\( |A|\leq|B| \). Recall from functions that \( f \) is 1-1 if every output
is uniquely associated to 1 input. Also, for real-valued functions, \( f \)
is 1-1 if it passed the horizontal line test. If \( A\subseteq B \), then
\( |A|\leq|B| \) by using \( f(x) = x \) where \( f:A\to B \).

\subsection*{Schr{\"o}der-Bernstein Theorem}
If \( f:A\to B \) is 1-1 and \( g:B\to A \) is 1-1, then \( |A| = |B| \).
Using the above example, \( |U| = |\R| \) since \( f(x) =
\frac{\arctan(x)+\frac{\pi}{2}}{\pi} \) where \( f:\R\to U \) is 1-1 and
\( g(x) = x \) where \( g:U\to\R \) is 1-1.

\subsection*{Countably Infinite Sets}
Set \( A \) is \textbf{countably infinite} if \( |A| = |\Z^{+}| \).
Equivalently, there are 1-1 functions \( f,g \) such that \( f:A\to\Z^{+} \) and
\( g:\Z^{+}\to A \). There is a good enumeration of \( A =
\{a_{1},a_{2},\dots\) such that every number in \( A \) shows up once and every
number appears with a finite index. A set is \textbf{countable} if it is a
finite set or countably infinite. \\
\textbf{Countably Infinite Sets}: \( \Z^{+}, \Z, \Q , \N\times\N \) \\
\textbf{Uncountable Sets}: \( \R, (0,1), P(\Z): subsets\ of\ \Z \) \\
If \( A,B \) are both countable, then \( A\cup B \) is countable.

\subsection*{Enumerating a Countable Set}
A good enumeration for a set \( A \) involves listing out the values in \( A \)
with a predictable pattern so that every value in \( A \) is listed at a finite
step. For example, \( \Z = \{0,1,2,3,\dots,-1,-2,-3,\dots\} \) is not a good
enumeration as we would have to go through all infinitely many positive
integers before seeing any negative integers. A good enumeration for
\( \Z = \{0,1,-1,2,-2,3,-3,\dots\} \) so that both positive integers and
negative integers are reached in a finite number of steps.

\subsection*{Subsets of Sets}
Suppose \( A\subseteq B \):
\begin{itemize}
  \item If \( B \) is countable, then \( A \) is countable as well.
  \item If \( B \) is uncountable, then \( A \) is uncountable as well.
\end{itemize}

\subsection*{Uncountable Sets}
Two sets \( A \) and \( B \) have the same \textbf{cardinality}, written
\( A\sim B \), if there is a bijection between them.
Let \( A \) be a set:
\begin{itemize}
  \item \( A \) is \textbf{countable} if it is finite: \( |A| \in \Z \)
  \item \( A \) is \textbf{countably infinite} if \( A \sim \Z^{+} \)
  \item \( A \) is \textbf{uncountable} if it is not countable.
\end{itemize}

\begin{center}
  You can find all my notes at \url{http://omgimanerd.tech/notes}. If you have
  any questions, comments, or concerns, please contact me at
  alvin@omgimanerd.tech
\end{center}

\end{document}
