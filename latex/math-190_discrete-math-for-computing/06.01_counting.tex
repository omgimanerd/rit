\documentclass{math}

\usepackage{forest}

\title{Counting}
\author{Alvin Lin}
\date{Discrete Math for Computing: January 2017 - May 2017}

\begin{document}

\maketitle

\section*{Counting}
Counting is a higher level math topic involving subtle counting rules,
applying logic, and set theory. It involves questions such as:
\begin{enumerate}
  \item How many license plates have the form AAA-DDDD where A is an uppercase
    alphabetical character and D is a digit?
  \item Jessica, Tim, Dan, and Mary are lining up to purchase tickets at a
    movie theater. In how many ways can they line up?
  \item Two people from a group of 10 are being selected to co-chair an
    upcoming event. How many ways are there to select 2 people from 10?
  \item In a binary string of length 5, how many strings contain exactly two
    1's somewhere in the string?
  \item 75 students are enrolled in math or english. If 40 students are
    enrolled in math and 50 students are enrolled in english then how many
    students are enrolled in both?
\end{enumerate}

\subsection*{Tree Diagrams}
A \textbf{tree diagram} is a useful visual tool for enumerating outcomes. For
example, suppose we wish to list all of the outcomes from rolling a die and
then flipping a coin. The tree diagram begins at a point and then splits into
6 directions. Then each branches off into 2 directions from flipping a coin.
\begin{center}
  \begin{forest}
    [
      [1 [H] [T]]
      [2 [H] [T]]
      [3 [H] [T]]
      [4 [H] [T]]
      [5 [H] [T]]
      [6 [H] [T]]
    ]
  \end{forest}
\end{center}

\subsection*{Product Rule}
In a string of length \( k \), if there are \( n_{1} \) ways to fill position
1, \( n_{2} \) ways to fill position 2, \( \dots \) and \( n_{k} \) ways to
fill position \( k \) then there are \( n_{1}\cdot n_{2}\cdot\dots\cdot n_{k} \)
strings altogether.

\subsection*{Addition Rule}
If a set \( S \) can be divided up into \( k \) non-overlapping subsets
\( T_{1},T_{2},\dots,T_{k} \) then \( |S| = |T_{1}|+|T_{2}|+\dots+|T_{k}| \).

\subsection*{Opposite Rule}
Suppose \( S \) consists of a set of objects chosen from a larger universe of
objects \( \U \). Then \( |S| = |\U|-|\overline{S}| \).

\subsection*{General Pigeonhole Principle}
Given \( N \) objects and \( k \) boxes or categories, there is at least one box
with at least \( M \) objects where \( M = \lceil\frac{N}{k}\rceil \).

\subsection*{Permutations and Combinations}
\begin{itemize}
  \item The \textbf{factorial} of a number \( n \), or \( n \) factorial, is
    the product of all the numbers from \( n \) down to 1. Additionally, 0
    factorial is 1 by definition and \( n \) factorial is denoted by \( n! \).
    For example, \( 3! = 3\cdot2\cdot1 = 6 \).
  \item An \textbf{r permutation of \( n \) distinct objects} is an ordered
    arrangement of \( r \) distinct objects from the original \( n \) objects.
    For example, the 2 permutations of `abc' are `ab', `ba', `ac', `ca', `bc',
    and `cb'. The total number of \( r \) permutations from \( n \) objects
    is denoted by \( P(n,r) \) or \( \nPr{n}{r} \). For example, \( P(3,2) = 6
    \) since we found 6 strings of length 2 when selected from the 3 distinct
    letters in `abc'.
  \item An \textbf{r combination of \( n \) distinct objects} is an unordered
    selection of \( r \) distinct objects from the original \( n \) objects.
    For example, the 2 combinations of `abc' are \( \{a,b\},\{a,c\},\ and\
    \{b,c\} \). The total number of \( r \) combinations from \( n \) objects
    is denoted by \( C(n,r) \) or \( \nCr{n}{r} \). For example, \( C(3,2) = 3
    \) since we found 3 sets of size 2 when selecting from the 3 distinct
    letters in `abc'.
  \item \textbf{\( k \) to 1 Correspondences}: If there are \( k \) objects in
    set A that correspond to 1 object in set B, then \( |A| = k\cdot|B| \).
  \item \textbf{Permutation with Repetition}: In a string of length \( n \)
    with \( r_{1} \) letters of type 1, \( r_{2} \) letters of type 2,
    \( \dots \), and \( r_{k} \) letters of type \( k \), then there are
    \[ \frac{n!}{(r_{1}!)(r_{2}!)\cdot\cdot\cdot(r_{k}!)} \]
    distinguishable permutations when rearranging all of the letters in the
    original string.
  \item \textbf{Combination with Repetition}: In making an unordered selection
    from \( n \) objects with \( n_{1} \) objects of type 1, \( n_{2} \) objects
    of type 2, \( \dots \), and \( n_{k} \) objects of type \( k \), then there
    are
    \[ \nCr{n_{1}}{r_{1}}\cdot\nCr{n_{2}}{r_{2}}\cdot\cdot\cdot
      \nCr{n_{k}}{r_{k}} \]
    distinguishable combinations when selecting \( r_{1} \) objects of type 1,
    \( r_{2} \) objects of type 2, \( \dots \), and \( r_{k} \) objects of type
    \( k \).
\end{itemize}

\subsection*{Facts about \( P(n,r) \) and \( C(n,r) \)}
\begin{enumerate}
  \item \( P(n,r) = n(n-1)(n-2)\dots(n-r+1) \)
  \item \( P(n,r) = \nPrf{n}{r} \)
  \item \( P(n,r) = r!\nCr{n}{r} \)
  \item \( C(n,r) = \frac{n(n-1)\dots(n-r+1)}{r!} \)
  \item \( C(n,r) = \nCrf{n}{r} \)
  \item \( P(n,r) = {\nPr{n}{r}} \)
  \item \( C(n,r) = {\nCr{n}{r}} = \binom{n}{r} \)
  \item \( C(n,0) = 1 = \frac{n!1}{n!0!} = \frac{1}{1} \)
  \item \( C(n,n) = 1 \)
  \item \( C(n,r) = C(n,n-r) \)
\end{enumerate}

\subsection*{Binomial Theorem}
\begin{align*}
  (x+y)^{0} &= 1 \\
  (x+y)^{1} &= 1x+1y \\
  (x+y)^{2} &= 1x^{2}+2xy+1y^{2} \\
  (x+y)^{3} &= 1x^{3}+3x^{2}y+3xy^{2}+1y^{3} \\
  (x+y)^{n} &= \binom{n}{0}x^{n}y^{0}+\binom{n}{1}x^{n-1}y^{1}+\dots+
      \binom{n}{n-1}x^{1}y^{n-1}+\binom{n}{n}x^{0}y^{n} \\
  (x+y)^{n} &= \sum_{i=0}^{n}\binom{n}{i}x^{n-i}y^{i}
\end{align*}

\subsection*{Distributing Objects}
There are \( k^{n} \) ways to distribute \( n \) distinguishable objects amongst
\( k \) boxes. However, there are only \( \binom{n+k-1}{k-1} \) ways to
distribute \( n \) indistinguishable objects amongst \( k \) boxes.

\begin{center}
  You can find all my notes at \url{http://omgimanerd.tech/notes}. If you have
  any questions, comments, or concerns, please contact me at
  alvin@omgimanerd.tech
\end{center}

\end{document}
