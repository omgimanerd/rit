\documentclass[letterpaper, 12pt]{math}

\usepackage{amsmath}
\usepackage{amssymb}

\title{Sets}
\author{Alvin Lin}
\date{Discrete Math for Computing: January 2017 - May 2017}

\begin{document}

\maketitle

\section*{Divisibility and Modular Arithmetic}
Floor function: \( \lfloor x\rfloor:\R\to\Z \):
\begin{center}
  \begin{tabular}{|c|c|}
    \hline
    \( x \) & \( \lfloor x\rfloor \) \\
    \hline
    4.5 & 4 \\
    \hline
    4.99 & 4 \\
    \hline
    4.1 & 4 \\
    \hline
    4 & 4 \\
    \hline
    12.8 & 12 \\
    \hline
    -3.14 & -4 \\
    \hline
  \end{tabular}
\end{center}

\subsection*{Division Algorithm}
For any integer \( a \) and a positive integer \( d \), there is a unique
quotient \( q \) and a unique remainder \( r \) so that \( a = dq+r \) and
\( 0\leq r<d \).
\begin{center}
  \begin{tabular}{|c|c|c|c|}
    \hline
    \( a \) & \( d \) & \( a(\mathrm{div}\ d) = q \) & \( a(\bmod\ r) \) \\
    \hline
    23 & 5 & 4 & 3 \\
    \hline
    2017 & 12 & 168 & 1 \\
    \hline
    18 & 7 & 2 & 4 \\
    \hline
    -10 & 6 & -2 & 2 \\
    \hline
  \end{tabular}
\end{center}

\subsection*{Applications of the Mod Function}
\begin{enumerate}
  \item Hashing Functions: A rudimentary hashing function can take the form of
    \( h(x) = x(\bmod\ d) \).
  \item Pseudorandom Sequences: Given some seed value \( a_{0} \), a
    pseudorandom sequence can be generated using the sequence
    \( a_{n+1} = (ax_{n}+b) (\bmod\ d) \).
  \item Basic Cryptography: The mod function is applied for basic cryptographic
    schemes like the Caesarshift cipher.
\end{enumerate}

\subsection*{Mod Congruency}
\begin{enumerate}
  \item \( d\equiv0\ (\bmod\ d) \) \\
  \item \( a\equiv a+kd\ (\bmod\ d) \) \\
  \item \( a\equiv a-kd\ (\bmod\ d) \) \\
  \item \( a\equiv b\ (\bmod\ d)\ if\ (a-b)\ is\ a\ multiple\ of\ d \)
\end{enumerate}

\begin{center}
  If you have any questions, comments, or concerns, please contact me at
  alvin@omgimanerd.tech
\end{center}

\end{document}
