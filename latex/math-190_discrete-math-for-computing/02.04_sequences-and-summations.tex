\documentclass{math}

\title{Sequences and Summations}
\author{Alvin Lin}
\date{Discrete Math for Computing: January 2017 - May 2017}

\begin{document}

\maketitle

\section*{Sequences and Summations}
Let \( a_{n} \) be the number of binary strings of length \( n \) with substring
``10''.
\begin{center}
  \begin{tabular}{|c|c|}
    \hline
    \( n \) & \( a_{n} \) \\
    \hline
    0 & 0 \\
    \hline
    1 & 0 \\
    \hline
    2 & 1 \\
    \hline
    3 & 4 \\
    \hline
    n & ? \\
    \hline
  \end{tabular}
\end{center}
Recursive formula for \( a_{n} \):
\[ a_{n} = 2a_{n-1}+n-1 \]

\subsection*{Sequences}
A \textbf{sequence} (called \( a_{n} \) or \( f(n) \)) is a function from
\( \N \) to \( \R \). Example:
\begin{enumerate}
  \item \( \{a_{n}\} = \{2,5,8,11,14,19,\dots\} \)
  \item \( \{b_{n}\} = \{2,4,8,16,32,64,\dots\} \)
  \item \( \{c_{n}\} = \{1,2,5,10,17,26,\dots\} \)
\end{enumerate}
A \textbf{linear sequence} or \textbf{arithmetic sequence} has a constant
difference \( d \) between adjacent terms.
\[ \{a_{0},a_{0}+d,a_{0}+2d,a_{0}+3d,\dots\} \]
\[ a_{n} = dn+a_{0} \]
Difference sequences:
\begin{align*}
  \{a_{n}'\} &= \{a_{n}-a_{n-1}\mid n \geq 1\} \\
  \{a_{n}\} &= \{2,5,8,11,\dots\} \\
  \{a_{n}'\} &= \{5-2,8-5,11-8,\dots\} = \{3,3,3,\dots\} \\
  \{c_{n}\} &= \{1,2,5,10,17,26,\dots\} \\
  \{c_{n}'\} &= \{1,3,5,7,9,\dots\} \\
  \{c_{n}''\} &= \{2,2,2,2,2,\dots\}
\end{align*}
A \textbf{geometric sequence} has a constant ratio \( r \), equal to
\( \frac{a_{n+1}}{a_{n}} \).
\[ \{a_{0},a_{0}r,a_{0}r^{2},a_{0}r^{3}\} \]
\[ a_{n} = a_{0}r^{n} \]
\begin{align*}
  \{b_{n}\} &= \{2,4,8,16,\dots\} \\
  &= 2^{n} \\
  \{b_{n}'\} &= \{2,4,8,16,\dots\} \\
  \{b_{n}''\} &= \{2,4,8,16,\dots\}
\end{align*}

\subsection*{Example}
\[ \{5,15,45,135,\dots\} = \{a_{n}\} \]
\[ r = \frac{20}{5} = \frac{80}{4} = 4 \]
\[ a_{n} = 5\times 3^{n} \]

\subsection*{Example}
Find a non-recursive formula for \( a_{n} = 2a_{n-1}+n-1 \).
\begin{align*}
  a_{n} &= A2^{n}+Bn+C \\
  a_{n-1} &= A2^{n-1}+B(n-1)+C \\
  a_{n} &= 2(a_{n-1})+n-1 \\
  A2^{n}+Bn+C &= 2(A2^{n-1}+B(n-1)+C)+n-1 \\
  B &= -1 = C \\
  A &= 1 \\
  a_{n} &= 2^{n}-n-1
\end{align*}

\subsection*{Summation Notation}
\[ \sum_{i=M}^{N}f(i) = f(M)+f(M+1)+\dots+f(N) \]
\[ \sum_{i=M}^{N}f(i) = \sum_{j=M}^{N}f(j) = \sum_{k=M}^{N}f(k) \]

\subsection*{Example}
\[ \sum_{i=1}^{5}i^{2} = 1^{2}+2^{2}+3^{2}+4^{2}+5^{2} = 55 \]
\[ \sum_{j=5}^{8}(\frac{1}{j}) = \frac{1}{5}+\frac{1}{6}+\frac{1}{7}+
   \frac{1}{8} \]
\[ \sum_{k=0}^{5}(2^{k}) = 2^{0}+2^{1}+2^{2}+2^{3}+2^{4}+2^{5} = 2^{6}-1 \]

\subsection*{Summation Properties}
\begin{enumerate}
  \item
    \[ \sum_{i=M}^{N}(kf(i)) = k(\sum_{i=M}^{N}f(i)) \]
  \item
    \[ \sum_{i=M}^{N}(f(i)+g(i)) = \sum_{i=M}^{N}f(i)+\sum_{i=M}^{N}g(i) \]
  \item
    \[ \sum_{i=M}^{N}f(i) = \sum_{i=1}^{N}f(i)-\sum_{i=1}^{M-1}f(i) \]
\end{enumerate}

\subsection*{Summation Formulas}
\begin{enumerate}
  \item
    \[ \sum_{i=1}^{n}1 = n \]
  \item Gauss's Formula
    \[ \sum_{i=1}^{n}i = \frac{n(n+1)}{2} \]
  \item
    \[ \sum_{i=1}^{n}i^{2} = \frac{n(n+1)(2n+1)}{6} \]
  \item
    \[ \sum_{i=1}^{n}i^{3} = \bigg(\frac{n(n+1)}{2}\bigg)^{2} \]
\end{enumerate}

\subsection*{Geometric Sequence}
\[ \{a,ar,ar^{2},ar^{3},\dots\} \]
\[ \sum_{i=0}^{n}(ar^{i}) = a\bigg(\frac{1-r^{n+1}}{1-r}\bigg) \]

\begin{center}
  You can find all my notes at \url{http://omgimanerd.tech/notes}. If you have
  any questions, comments, or concerns, please contact me at
  alvin@omgimanerd.tech
\end{center}

\end{document}
