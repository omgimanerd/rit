\documentclass[letterpaper, 12pt]{math}

\usepackage{amsmath}
\usepackage{amssymb}

\title{Sequences and Summations}
\author{Alvin Lin}
\date{Discrete Math for Computing: January 2017 - May 2017}

\begin{document}

\maketitle

\section*{Sequences and Summations}
Let \( a_{n} \) be the number of binary strings of length \( n \) with substring
``10''.
\begin{center}
  \begin{tabular}{|c|c|}
    \hline
    \( n \) & \( a_{n} \) \\
    \hline
    0 & 0 \\
    \hline
    1 & 0 \\
    \hline
    2 & 1 \\
    \hline
    3 & 4 \\
    \hline
    n & ? \\
    \hline
  \end{tabular}
\end{center}
Recursive formula for \( a_{n} \):
\[ a_{n} = 2a_{n-1}+n-1 \]

\subsection*{Sequences}
A \textbf{sequence} (called \( a_{n} \) or \( f(n) \)) is a function from
\( \N \) to \( \R \). Example:
\begin{enumerate}
  \item \( \{a_{n}\} = \{2,5,8,11,14,19,\dots\} \)
  \item \( \{b_{n}\} = \{2,4,8,16,32,64,\dots\} \)
  \item \( \{c_{n}\} = \{1,2,5,10,17,26,\dots\} \)
\end{enumerate}
A \textbf{linear sequence} or \textbf{arithmetic sequence} has a constant
difference \( d \) between adjacent terms.
\[ \{a_{0},a_{0}+d,a_{0}+2d,a_{0}+3d,\dots\} \]
\[ a_{n} = dn+a_{0} \]
Difference sequences:
\begin{align*}
  \{a_{n}'\} &= \{a_{n}-a_{n-1}\ |\ n \geq 1\} \\
  \{a_{n}\} &= \{2,5,8,11,\dots\} \\
  \{a_{n}'\} &= \{5-2,8-5,11-8,\dots\} = \{3,3,3,\dots\} \\
  \{c_{n}\} &= \{1,2,5,10,17,26,\dots\} \\
  \{c_{n}'\} &= \{1,3,5,7,9,\dots\} \\
  \{c_{n}''\} &= \{2,2,2,2,2,\dots\}
\end{align*}
A \textbf{geometric sequence} has a constant ratio \( r \), equal to
\( \frac{a_{n+1}}{a_{n}} \).
\[ \{a_{0},a_{0}r,a_{0}r^{2},a_{0}r^{3}\} \]
\[ a_{n} = a_{0}r^{n} \]
\begin{align*}
  \{b_{n}\} &= \{2,4,8,16,\dots\} \\
  &= 2^{n} \\
  \{b_{n}'\} &= \{2,4,8,16,\dots\} \\
  \{b_{n}''\} &= \{2,4,8,16,\dots\}
\end{align*}

\subsection*{Example}
\[ \{5,15,45,135,\dots\} = \{a_{n}\} \]
\[ r = \frac{20}{5} = \frac{80}{4} = 4 \]
\[ a_{n} = 5\times 3^{n} \]

\subsection*{Example}
Find a non-recursive formula for \( a_{n} = 2a_{n-1}+n-1 \).
\begin{align*}
  a_{n} &= A2^{n}+Bn+C \\
  a_{n-1} &= A2^{n-1}+B(n-1)+C \\
  a_{n} &= 2(a_{n-1})+n-1 \\
  A2^{n}+Bn+C &= 2(A2^{n-1}+B(n-1)+C)+n-1 \\
  B &= -1 = C \\
  A &= 1 \\
  a_{n} &= 2^{n}-n-1
\end{align*}

\begin{center}
  If you have any questions, comments, or concerns, please contact me at
  alvin@omgimanerd.tech
\end{center}

\end{document}
