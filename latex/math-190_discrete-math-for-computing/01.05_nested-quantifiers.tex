\documentclass[letterpaper, 12pt]{math}

\usepackage{amsmath}
\usepackage{amssymb}

\title{Nested Quantifiers}
\author{Alvin Lin}
\date{Discrete Math for Computing: January 2017 - May 2017}

\begin{document}

\maketitle

\section*{Nested Quantifiers}
In section 1.4, we avoided nested quantifiers where one was in the scope of the
other.

\subsection*{Example}
Assume that the domain of \( x \) and \( y \) are real numbers:
\begin{enumerate}
  \item \[ \forall x \forall y (x+y = y+x) \]
  For all real numbers \( x \) and \( y \), \( x+y = y+x \)
  \item \[ \forall x \exists y (x+y = 0) \]
  For all real numbers \( x \), there exists a \( y \) such that \( x+y = 0 \).
\end{enumerate}
Translate the following into English:
\[ \forall x \forall y ((x > 0) \wedge (y < 0)) \to (xy < 0) \]
For all real \( x \) and all real \( y \) if \( x > 0 \) and \( y > 0 \), then
\( xy > 0 \). The product of a positive real number and negative real number is
negative.

\subsection*{Order of Quantifiers}
The order of quantifiers is essential, unless the quantifiers have all the same
``type'' (Either the quantifiers are all universal or all existential).

\subsection*{Example}
\[ P(x,y): x+y = y+x \]
What are the truth values when \( domain = \R \)?
\begin{enumerate}
  \item \( \forall{x}\forall{x}\ P(x,y) \) \textbf{True}
  \item \( \forall{y}\forall{x}\ P(x,y) \) \textbf{True}
\end{enumerate}
\[ Q(x,y): x+y = 0 \]
Domain is \( \R \), what are the truth values?
\begin{enumerate}
  \item \( \exists{y}\forall{x}\ Q(x,y) \) There is a real number \( y \) such
  that for all real \( x \), \( x+y = 0 \). \textbf{False}
  \item \( \forall{x}\exists{y}\ Q(x,y) \) For all real \( x \), there is a
  real number \( y \) such that \( x+y = 0 \). \textbf{True}
\end{enumerate}

\subsection*{Example}
Translate the following into a logical expression:
\begin{enumerate}
  \item The sum of two positive integers is always positive.
  \[ \forall{x}\forall{y}((x > 0) \wedge (y > 0)) \to (xy > 0) \]
  \item Every real number except 0 has a multiplicative inverse.
  \[ \forall{x}((x \neq 0) \to \exists{y}(xy = 1)) \]
\end{enumerate}

\subsection*{Example}
Write \( \neg{(\forall{x}\exists{y}(xy = 1))} \) without any negations
in the expression.
\[ \exists{x}\neg{(\exists{y}(xy=1))} \]
\[ \exists{x}\forall{y}\neg{(xy=1)} \]
\[ \exists{x}\forall{y}(xy \neq 1) \]

\end{document}
