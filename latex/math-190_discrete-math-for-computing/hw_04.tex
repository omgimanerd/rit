\documentclass[letterpaper, 12pt]{math}

\usepackage{amsmath}
\usepackage{amssymb}
\usepackage{tikz}

\title{Homework \#4}
\author{Alvin Lin}
\date{Discrete Math for Computing: January 2017 - May 2017}

\begin{document}

\maketitle

\subsection*{1}
Determine whether each of the following expressions is true or false.
\renewcommand{\labelenumi}{(\alph{enumi})}
\begin{enumerate}
  \item \( x \in \{x\} \) True
  \item \( \{x\} \in \{\{x\}\} \) True
  \item \( \{x\} \subset \{x\} \) True
  \item \( \varnothing \subset \{x\} \) True
  \item \( \varnothing \in \{x\} \) False
\end{enumerate}

\subsection*{2}
Find the power set of each set below. Assume that \( a \) and \( b \) are
distinct elements.
\begin{enumerate}
  \item \( P(\{a\}) = \{\varnothing,\{a\}\} \)
  \item \( P(\{a,b\}) = \{\varnothing,\{a\},\{b\},\{a.b\}\} \)
  \item \( P(\{\varnothing,\{\varnothing\}\}) = \{\varnothing,\{\varnothing\},
      \{\{\varnothing\}\},\{\varnothing,\{\varnothing\}\}\} \)
\end{enumerate}

\subsection*{3}
Let \( A = \{a,b,c\} \), \( B = \{c,d\} \) and \( C = \{x,z\} \). Find
\begin{enumerate}
  \item \( A\times C = \{(a,c),(a,d),(b,c),(b,d),(c,c),(c,d)\} \)
  \item \( C\times A\times B = \{(x,a,c),(x,a,d),(x,b,c),(x,b,d), \) \\
    \( (x,c,c),(x,c,d),(z,a,c),(z,a,d),(z,b,c),(z,b,d),(z,c,c),(z,c,d)\} \)
  \item \( A\times B\times C = \{(a,c,x),(a,c,z),(a,d,x),(a,d,z), \) \\
    \( (b,c,x),(b,c,z),(b,d,x),(b,d,z),(c,c,x),(c,c,z),(c,d,x),(c,d,z) \} \)
\end{enumerate}

\subsection*{4}
Suppose that
\[ A = \{a,b,c,d,e,f\}\ \mathrm{and}\ B = \{a,b,c,d,e,f,g,h,i,j,k\} \]
Find the following
\begin{enumerate}
  \item \( A\cup B = \{a,b,c,d,e,f,g,h,i,j,k\} \)
  \item \( A-B = \varnothing \)
  \item \( A\cap B = \{a,b,c,d,e,f\} \)
  \item \( B-A = \{g,h,i,j,k\} \)
\end{enumerate}

\subsection*{5}
Suppose that \( A \), \( B \), and \( C \) are arbitrary sets. Show the
following and justify:
\begin{enumerate}
  \item \( (A\cup B) \subseteq (A\cup B\cup C) \)
  \[ (A\cup B) = \{x\mid x \in A \vee x \in B \} \]
  \[ (A\cup B\cup C) = \{x\mid x \in A \vee x \in B \vee x \in C\} \]
  \begin{center}
    Identity Law
  \end{center}
  \item \( A-B = A\cap \overline{B} \)
  \[ A-B = \{x\mid x \in A \wedge x \notin B\} \]
  \[ A\cap \overline{B} = \{x\mid x \in A \wedge x \notin B\} \]
  \item \( (A-B)-C = (A-C)-(B-C) \)
  \begin{align*}
    (A-B)-C &= \{x\mid x \in A \wedge x \notin B \wedge x \notin C\} \\
    (A-C) &= \{x\mid x \in A \wedge x \notin C\} \\
    (B-C) &= \{x\mid x \in B \wedge x \notin C\} \\
    (A-C)-(B-C) &= \{x\mid x \in A \wedge x \notin C\}-
     \{x\mid x \in B \wedge x \notin C\} \\
    (A-C)-(B-C) &= \{x\mid x \in A \wedge x \notin B \wedge x \notin C\}
  \end{align*}
  \item \( \overline{A \cup (B \cap C)} = (\overline{C} \cup \overline{B}) \cap
     \overline{A} \)
  \begin{align*}
    \overline{A \cup (B \cap C)} &=
      \overline{A} \cap \overline{B \cap C} \\
    &= \overline{A} \cap (\overline{B} \cup \overline{C}) \\
    &= (\overline{B} \cup \overline{C}) \cap \overline{A} \\
    &= (\overline{C} \cup \overline{B}) \cap \overline{A}
  \end{align*}
\end{enumerate}

\subsection*{6}
Suppose that \( A \), \( B \), and \( C \) are sets. Draw the Venn diagrams
for each of the combinations below.
\begin{enumerate}
  \item
  \[ A \cap (B \cup C) \]
  \begin{center}
    \def\firstcircle{(90:1.75cm) circle (2.5cm)}
    \def\secondcircle{(210:1.75cm) circle (2.5cm)}
    \def\thirdcircle{(330:1.75cm) circle (2.5cm)}
    \begin{tikzpicture}
      \begin{scope}
        \clip \secondcircle;
        \fill[green] \firstcircle;
      \end{scope}
      \begin{scope}
        \clip \thirdcircle;
        \fill[green] \firstcircle;
      \end{scope}
      \draw \firstcircle node[text=black,above] {$A$};
      \draw \secondcircle node [text=black,below left] {$B$};
      \draw \thirdcircle node [text=black,below right] {$C$};
    \end{tikzpicture}
  \end{center}
  \item
  \[ (A-B) \cup (A-C) \cup (B-C) \]
  \begin{center}
    \def\firstcircle{(90:1.75cm) circle (2.5cm)}
    \def\secondcircle{(210:1.75cm) circle (2.5cm)}
    \def\thirdcircle{(330:1.75cm) circle (2.5cm)}
    \begin{tikzpicture}
      \fill[green] \firstcircle;
      \fill[green] \secondcircle;
      \begin{scope}
        \clip \secondcircle;
        \fill[white] \thirdcircle;
      \end{scope}
      \draw \firstcircle node[text=black,above] {$A$};
      \draw \secondcircle node [text=black,below left] {$B$};
      \draw \thirdcircle node [text=black,below right] {$C$};
    \end{tikzpicture}
  \end{center}
\end{enumerate}

\end{document}
