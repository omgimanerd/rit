\documentclass[letterpaper, 12pt]{math}

\usepackage{amsmath}
\usepackage{amssymb}

\title{Sets}
\author{Alvin Lin}
\date{Discrete Math for Computing: January 2017 - May 2017}

\begin{document}

\maketitle

\section*{Functions}
Let \( A \) and \( B \) be sets. A \textbf{function} \( f \) is a rule that
assigns to each element \( x \in A \) exactly one element \( y \in B \),
written \( y = f(x) \). \( A \) is called the \textbf{domain} while \( B \)
is called the \textbf{codomain}. The \textbf{range} of \( f \), denoted
\( ran(f) \):
\[ ran(f) = \{f(x) \in B\ |\ x \in A \} \]
\[ ran(f) \subseteq B \]

\subsection*{Example}
Let \( f:\R\to\R \) be defined as \( f(x) = x^{2} \).
\begin{itemize}
  \item Domain: \( \R \)
  \item Codomain: \( \R \)
  \item Range: \( \big[0,\infty\big) \)
\end{itemize}

\subsection*{Example}
Let \( g:\R\to[0,\infty) \) be defined as \( g(x) = x^{2} \).
\begin{itemize}
  \item Domain: \( \R \)
  \item Codomain: \( \big[0,\infty\big) \)
  \item Range: \( \big[0,\infty\big) \)
\end{itemize}

\subsection*{Function Equality}
Two functions, \( f:A\to B \) and \( g:C\to D \) are \textbf{equal}, denoted
\( f = g \), if:
\begin{itemize}
  \item \( A = C \)
  \item \( B = D \)
  \item \( f(x) = g(x) \quad \forall{x}\in A \)
\end{itemize}

\subsection*{Absolute Value}
The \textbf{absolute value} of \( x\in\R \), written \( abs(x) \) or \( |x| \),
is a piecewise function defined as:
\[ abs(x) =
  \begin{cases}
    x & if\ x\geq 0 \\
    -x & if\ x < 0
  \end{cases}
\]
\[ abs:\R\to\R \]
\begin{itemize}
  \item Domain: \( \R \)
  \item Codomain: \( \R \)
  \item Range: \( [0,\infty) \)
\end{itemize}

\subsection*{Example}
Let \( X \) be a set, \( P(X) \) be the power set of \( X \), and
\( S \subseteq X \). Let \( f:P(X)\to P(X) \) be defined as
\( f(A) = A \cup S \).
\begin{itemize}
  \item Domain: \( P(X) \)
  \item Codomain: \( P(X) \)
  \item Range: \( \{B \subseteq X\ |\ B \supseteq S\} \)
\end{itemize}

\subsection*{One-To-One}
A function is \textbf{one-to-one} if it passes the horizontal line test, that
is, for every \( x \) value, there is a unique \( y \) value.

\subsection*{Onto}
A function is \textbf{onto} if the range and codomain are the same.

\subsection*{Function Images and Preimages}
Let \( f:A\to B \) be a function. If \( S \subseteq A \), then the
\textbf{image} of \( S \), denoted \( f(S) \), is defined as:
\[ f(S) = \{f(x):x\in S \} \]
If \( T \subseteq B \), then the \textbf{preimage} of \( S \), denoted
\( f^{-1}(S) \), is defined as:
\[ f^{-1}(T) = \{x\in A\ |\ f(x)\in T\} \]

\subsection*{Example}
Let \( f:\R\to\R \) be defined by \( f(x) = x^{2} \). Determine:
\begin{enumerate}
  \item \( f([-1,2]) = [0,4] \)
  \item \( f^{-1}(\{2\}) \)
  \item \( f^{-1}(\big[-1,2\big]) \)
\end{enumerate}

\begin{center}
  If you have any questions, comments, or concerns, please contact me at
  alvin@omgimanerd.tech
\end{center}

\end{document}
