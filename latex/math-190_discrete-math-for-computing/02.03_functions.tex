\documentclass[letterpaper, 12pt]{math}

\usepackage{amsmath}
\usepackage{amssymb}

\title{Sets}
\author{Alvin Lin}
\date{Discrete Math for Computing: January 2017 - May 2017}

\begin{document}

\maketitle

\section*{Functions}
Let \( A \) and \( B \) be sets. A \textbf{function} \( f \) is a rule that
assigns to each element \( x \in A \) exactly one element \( y \in B \),
written \( y = f(x) \). \( A \) is called the \textbf{domain} while \( B \)
is called the \textbf{codomain}. The \textbf{range} of \( f \), denoted
\( ran(f) \):
\[ ran(f) = \{f(x) \in B\ |\ x \in A \} \]
\[ ran(f) \subseteq B \]

\subsection*{Example}
Let \( f:\R\to\R \) be defined as \( f(x) = x^{2} \).
\begin{itemize}
  \item Domain: \( \R \)
  \item Codomain: \( \R \)
  \item Range: \( \big[0,\infty\big) \)
\end{itemize}

\subsection*{Example}
Let \( g:\R\to[0,\infty) \) be defined as \( g(x) = x^{2} \).
\begin{itemize}
  \item Domain: \( \R \)
  \item Codomain: \( \big[0,\infty\big) \)
  \item Range: \( \big[0,\infty\big) \)
\end{itemize}

\subsection*{Function Equality}
Two functions, \( f:A\to B \) and \( g:C\to D \) are \textbf{equal}, denoted
\( f = g \), if:
\begin{itemize}
  \item \( A = C \)
  \item \( B = D \)
  \item \( f(x) = g(x) \quad \forall{x}\in A \)
\end{itemize}

\subsection*{Absolute Value}
The \textbf{absolute value} of \( x\in\R \), written \( abs(x) \) or \( |x| \),
is a piecewise function defined as:
\[ abs(x) =
  \begin{cases}
    x & if\ x\geq 0 \\
    -x & if\ x < 0
  \end{cases}
\]
\[ abs:\R\to\R \]
\begin{itemize}
  \item Domain: \( \R \)
  \item Codomain: \( \R \)
  \item Range: \( [0,\infty) \)
\end{itemize}

\subsection*{Example}
Let \( X \) be a set, \( P(X) \) be the power set of \( X \), and
\( S \subseteq X \). Let \( f:P(X)\to P(X) \) be defined as
\( f(A) = A \cup S \).
\begin{itemize}
  \item Domain: \( P(X) \)
  \item Codomain: \( P(X) \)
  \item Range: \( \{B \subseteq X\ |\ B \supseteq S\} \)
\end{itemize}

\subsection*{Function Images and Preimages}
Let \( f:A\to B \) be a function. If \( S \subseteq A \), then the
\textbf{image} of \( S \), denoted \( f(S) \), is defined as:
\[ f(S) = \{f(x):x\in S \} \]
If \( T \subseteq B \), then the \textbf{preimage} of \( S \), denoted
\( f^{-1}(S) \), is defined as:
\[ f^{-1}(T) = \{x\in A\ |\ f(x)\in T\} \]

\subsection*{Example}
Let \( f:\R\to\R \) be defined by \( f(x) = x^{2} \). Determine:
\begin{enumerate}
  \item \( f([-1,2]) = [0,4] \)
  \item \( f^{-1}(\{2\})  = \{\sqrt{2},-\sqrt{2}\} \)
  \item \( f^{-1}(\big[-1,2\big]) \)
  \begin{align*}
    f^{-1}(\{-1\}) &= \varnothing \\
    f^{-1}([-1,0]) &= \varnothing \\
    f^{-1}(\{a,b\}) &= f^{-1}(\{a\}) \cup f^{-1}(\{b\}) \\
    f^{-1}([-1,2]) &= f^{-1}([-1,0]) \cup f^{-1}([0,2]) \\
    &= \varnothing \cup [-\sqrt{2},\sqrt{2}] \\
    &= [-\sqrt{2},\sqrt{2}]
  \end{align*}
\end{enumerate}

\subsection*{One-To-One and Onto}
Let \( f:A\to B \), \( f \) is \textbf{one-to-one} if distinct elements in
\( A \) have distinct images in \( B \). To show a function is one-to-one:
\[ \forall{x,y}\in A (x\neq y \to f(x)\neq f(y)) \]
Contrapositive:
\[ \forall{x,y}\in A (f(x) = f(y) \to x = y) \]

\subsection*{Onto}
Let \( f:A\to B\), \( f \) is \textbf{onto} if the range and codomain are the
same.
\[ B = ran(f) = f(A) \]

\subsection*{Example}
Let \( x\in\R \). The greatest integer of x, denoted \( \lfloor x \rfloor \),
is the largest integer less than or equal to \( x \). \( \lfloor x \rfloor =
floor(x) = floor:\R\to\R \).
\begin{align*}
  \lfloor 2.1 \rfloor &= 2 \\
  \lfloor \pi \rfloor &= 3 \\
  \lfloor -2.1 \rfloor &= -3
\end{align*}
Is \( \lfloor x \rfloor \) one-to-one?
\begin{align*}
  \lfloor 1.1 \rfloor &= \lfloor 1 \rfloor \\
  1.1 &\neq 1
\end{align*}
Therefore, it is not one-to-one. \\
Is \( \lfloor x \rfloor \) onto? \\
No, there is no \( x\in\R \) such that \( \lfloor x \rfloor = 2.5 \).

\subsection*{Example}
Let \( f:\Z\to\Z \) be defined by \( f(n) = n^{3} \). Is \( f \) one-to-one?
\begin{itemize}
  \item Suppose \( x,y\in\Z \) and \( f(x) = f(y) \).
  \item This implies \( x^{3} = y^{3} \).
  \item Taking the cube root yields \( x = y \).
\end{itemize}
Is \( f \) onto? \\
No, since \( 2\in\Z \) but there is no \( x\in\Z \) such that \( x^{3} = 2 \).

\subsection*{Example}
Let \( X \) be a set and \( X \subseteq X \). Define \( f:P(x)\to P(x) \) as
\( f(A) = A \cup S \) Is \( f \) one-to-one? \\
No, since \( f(\varnothing) = f(S) \) but \( \varnothing \neq S \).
\begin{align*}
  f(\varnothing) &= \varnothing \cup S = S \\
  f(S) &= S \cup S = S
\end{align*}

\subsection*{Example}
Let \( f:\Z\times\Z\to\Z \) be defined by \( f(m,n) = m+n \). \\
If \( f \) one-to-one? \par
No, \( f(2,3) = f(3,2) \) but \( (2,3) \neq (3,2) \) \\
If \( f \) onto? \par
Yes, take any \( n\in\Z \). Find an ordered pair in \( \Z\times\Z \) that
maps to it. Since \( f(n-1,1) = (n-1)+1 = n \) and \( (n-1,1)\in\Z\times\Z \),
then \( f \) is onto.

\subsection*{Identity Mapping}
Let \( A \) be a set. The function \( 1_{A}:A\to A \) is defined as:
\[ 1_{A}(x) = x \]
This is called the \textbf{identity mapping}.

\subsection*{Bijections}
A function is a \textbf{bijection} if it is one-to-one and onto. A
``bijection'' is a ``one-to-one correspondence''.

\subsection*{Composition}
Let \( f:A\to A \) and \( g:B\to C \). The \textbf{composition} of \( f \)
and \( g \), denoted \( g\circ f \), is \( g\circ f:A\to C \) and defined as:
\[ (g\circ f)(x) = g(f(x)) \]

\subsection*{Inverse}
Let \( f \) be a bijection. \( (f:A\to B) \). The \textbf{inverse} of \( f \)
is the unique function \( g:B\to A \) such that \( g\circ f = 1_{A} \) and
\( f\circ g = 1_{B} \). Traditionally, we represent \( g \) as \( f^{-1} \).
Note that \( f^{-1} \neq \frac{1}{f} \).

\subsection*{Example}
Let \( A = \{1,2,3\}, B = \{x,y,z\}, C = \{\alpha,\beta,\gamma\} \). Let
\( f:A\to B \) and \( g:B\to C \) be defined as:
\begin{align*}
  f &= \{(1,x),(2,y),(3,x)\} \\
  g &= \{(x,\gamma),(y,\alpha),(z,\beta)\}
\end{align*}
Find \( f\circ g \):
\begin{align*}
  f\circ g &= undefined \\
  g\circ f &= \{(1,\gamma),(2,\alpha),(3,\beta)\}
\end{align*}

\begin{center}
  If you have any questions, comments, or concerns, please contact me at
  alvin@omgimanerd.tech
\end{center}

\end{document}
