\documentclass[letterpaper, 12pt]{math}

\usepackage{amsmath}
\usepackage{amssymb}

\title{Predicates and Quantifiers}
\author{Alvin Lin}
\date{Discrete Math for Computing: January 2017 - May 2017}

\begin{document}

\maketitle

\section*{Predicates and Quantifiers}
Propositional logic cannot express the meaning of many mathematical or
English statements.

\subsection*{Predicate Logic}
Statements with variables are called \textbf{predicates}. For example,
\( x > 9, x+2y = 1, x^{2}+y^{2} = z^{2} \).
We can view predicates as propositional functions from some domain to some
range. A predicate in the variable \( x \) will be denoted as \( P(x) \). The
domain of a predicate is often called the \textbf{universe of discourse}.
\[ P(x): x > 9 \]
\begin{align*}
  P(2)&: 2 > 9\ is\ false. \\
  P(4)&: 4 > 9\ is\ false. \\
  P(10)*: 10 > 9\ is\ true.
\end{align*}
\[ Q(x,y): x = y+2 \]
\begin{align*}
  Q(1,2)&: 1 = 2+2\ is\ false. \\
  Q(3,0)&: 3 = 0+2\ is\ false. \\
  Q(2,0)&: 2 = 0+2\ is\ true. \\
\end{align*}

\subsection*{Quantifiers}
We will use the following three quantifiers in this class:
\begin{enumerate}
  \item \textbf{Universal Quantifier}: The universal quantifier of \( P(x) \)
    is the statement ``P(x) for all values x in the domain''. Notated as
    \( \forall{x}P(x) \).
  \item \textbf{Existential Quantifier}: The existential quantifier of a
    statement \( P(x) \) is ``There exists an element x in the domain so
    that P(x) holds''. Notated as
    \( \exists{x}P(x) \).
  \item \textbf{Uniqueness Quantifier}: The uniqueness quantifier of \( P(x) \)
    is the statement ``there exists a unique x such that P(x) holds''. Notated
    as \( \exists{!x} \).
\end{enumerate}

\subsection*{Example Problem}
What are the truth values of the following if the universe of discourse is all
positive integers not exceeding four?
\begin{enumerate}
  \item \( \exists{x}P(x)\ where\ P(x):x^{2} > 10 \)
  \[ Domain = \{1, 2, 3, 4\} \]
  \[ 1^{2} \ngtr 10 \]
  \[ 2^{2} \ngtr 10 \]
  \[ 3^{2} \ngtr 10 \]
  \[ 4^{2} > 10 \]
  Therefore, this statement is true.
  \item \( \forall{x}P(x)\ where\ P(x):x^{2} > 10 \)
  \[ Domain = \{1, 2, 3, 4\} \]
  \[ 1^{2} \ngtr 10 \]
  Therefore, this statement is false.
\end{enumerate}

\subsection*{Precedence of Quantifiers}
The quantifiers \( \forall{x} \) and \( \exists{x} \) have higher precedence
that those from propositional logic. When \( \forall{x} \) and \( \exists{x} \)
are in statements together, things get tricky.
\[ \forall{x}P(x) \vee Q(x) \equiv (\forall{x}P(x)) \vee Q(x) \]

\subsection*{Binding Variables}
When a quantifier is used on the variable \( x \), we say this instance of
\( x \) is a \textbf{bound variable}. A variable that is not bound is called a
\textbf{free variable}. The part of a logical expression in which a quantifier
is applied is called the quantifier's \textbf{scope}.
\[ \exists{x}(x+y = 1) \]
\( x \) is a bound variable while \( y \) is a free variable.

\subsection*{Logical Equivalences with Quantifiers}
Statements involving predicates and quantifiers are logically equivalent if and
only if they have the same truth value no matter what predicates are used and
what universe of discourse is used. \textbf{You must prove things in full
generality}.

\subsection*{Negation of Quantifiers}
\[ \neg{\forall{x}P(x)} \equiv \forall x \neg{P(x)} \]
\[ \neg{\exists{x}P(x)} \equiv \forall x \neg{P(x)} \]
We will use these facts often, especially for De Morgan's laws for predicate
logic and set theory.

\subsection*{De Morgan's Laws for Quantifiers}
\begin{center}
  \begin{tabular}{|c|c|c|c|}
    \hline
    Negation & Equivalent & When it is true, then... &
    When it is false, then... \\ \hline
    \( \neg{\exists{x}P(x)} \) & \( \forall{x}\neg{P(x)} \) &
    P(x) is false for all x & There is an x when P(x) is true \\ \hline
    \( \neg{\forall{x}P(x)} \) & \( \exists{x}\neg{P(x)} \) &
    There is an x when P(x) is false & P(x) is true for all x \\ \hline
  \end{tabular}
\end{center}

\subsection*{Example}
Rewrite in simple form:
\[ \forall{x}(x^{2} > x) \Rightarrow \neg{(\forall{x}(x^{2}>x))} \]
\[ \exists{x}\neg{(x^{2} > x)} \equiv \exists{x}(x^{2} \leq x) \]
Rewrite in simple form:
\[ \exists{x}(x^{2} = 2) \Rightarrow \neg{(\exists{x}(x^{2} = 2))} \]
\[ \forall{x}\neg{(x^{2} = 2)} \equiv \forall{x}(x^{2} \neq 2) \]

\begin{center}
  If you have any questions, comments, or concerns, please contact me at
  alvin@omgimanerd.tech
\end{center}

\end{document}
