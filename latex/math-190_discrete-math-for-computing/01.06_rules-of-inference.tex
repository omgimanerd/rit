\documentclass[letterpaper, 12pt]{math}

\usepackage{amsmath}
\usepackage{amssymb}

\title{Rules of Inference}
\author{Alvin Lin}
\date{Discrete Math for Computing: January 2017 - May 2017}

\begin{document}

\maketitle

\section*{Rules of Inference}
Proofs in mathematics are valid arguments that establish the truth of
mathematical statements. \\
\textbf{Argument:} Sequence of statements that ends with a conclusion. \\
A conclusion is \textbf{valid} if it follows from the truth of the preceding
statements (called premises). Rules of inference are basic tools for
establishing the truth of statements. \\ \\

Consider the following:
\begin{itemize}
  \item ``If you have a current password, then you can log on to the network''
  \item ``You have a current password''
  \item ``Therefore, you can log onto the network''
\end{itemize}
Determine whether this is a valid argument. Does the conclusion following?
Let: \\
p = you have a current password, q = you can log onto the network
\begin{align*}
  & p \to q \\
  & p \\
  \therefore & q
\end{align*}
\( (p \leftrightarrow q \wedge p) \to q \) is a tautology, so the argument is
valid. \\

Suppose we replace \( p \) and \( q \) by: \\
p: you have access to the network \\
q: you can change your grade \\
\( p: True \quad p \to q: False \), \( p \to q \) is valid still, but since
the premise \( p \to q \) is false, you can't conclude q.

\subsection*{Modus Ponens}
\( (p \wedge (p \to q)) \to q \) is a rule of inference called Modus Ponens.
It is always valid, but the conclusion might not always be true.

\subsection*{Example}
Determine whether the argument is valid and whether its conclusion must be true.
If \( \sqrt{2} > \frac{3}{2} \) then \( (\sqrt{2})^{2} > (\frac{3}{2})^{2} \).
Let:
\[ p: \sqrt{2} > \frac{3}{2} \]
\[ q: (\sqrt{2})^{2} > (\frac{3}{2})^{2} \]
The conclusion is not true since one premise, namely
\( \sqrt{2} > \frac{3}{2} \) is false.

\begin{center}
  If you have any questions, comments, or concerns, please contact me at
  alvin@omgimanerd.tech
\end{center}

\end{document}
