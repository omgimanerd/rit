\documentclass[letterpaper, 12pt]{math}

\usepackage{amsmath}
\usepackage{amssymb}

\title{Rules of Inference}
\author{Alvin Lin}
\date{Discrete Math for Computing: January 2017 - May 2017}

\begin{document}

\maketitle

\section*{Rules of Inference}
Proofs in mathematics are valid arguments that establish the truth of
mathematical statements. \\
\textbf{Argument:} Sequence of statements that ends with a conclusion. \\
A conclusion is \textbf{valid} if it follows from the truth of the preceding
statements (called premises). Rules of inference are basic tools for
establishing the truth of statements.

\subsection*{Example}
Consider the following:
\begin{itemize}
  \item ``If you have a current password, then you can log on to the network''
  \item ``You have a current password''
  \item ``Therefore, you can log onto the network''
\end{itemize}
Determine whether this is a valid argument. Does the conclusion follow? Let:
\begin{itemize}
  \item \( p \): you have a current password
  \item \( q \): you can log onto the network
\end{itemize}
\begin{align*}
  & p \to q \\
  & p \\
  \therefore & q
\end{align*}
\( (p \leftrightarrow q \wedge p) \to q \) is a tautology, so the argument is
valid.

\subsection*{Example}
Suppose we replace \( p \) and \( q \) by:
\begin{itemize}
  \item \( p \): you have access to the network
  \item \( q \): you can change your grade
\end{itemize}
\[ p: True \]
\[ p \to q: False \]
\( p \to q \) is valid still, but since the premise \( p \to q \) is false, you
can't conclude \( q \). \( (p \wedge (p \to q)) \to q \) is a rule of inference
called Modus Ponens. It is always valid, but the conclusion might not always be
true.

\subsection*{Rules of Inference}
\begin{enumerate}
  \item Modus Ponens
    \[ p \]
    \[ p \to q \]
    \[ \therefore q\]
  \item Modus Tollens
    \[ \neg{q} \]
    \[ p \to q \]
    \[ \therefore \neg{p} \]
  \item Hypothetical Syllogism
    \[ p \to q \]
    \[ q \to r \]
    \[ p \to r \]
  \item Disjunctive Syllogism
    \[ p \vee q \]
    \[ \neg{p} \]
    \[ \therefore q \]
  \item Addition
    \[ p \]
    \[ \therefore p \vee q \]
  \item Simplification
    \[ p \wedge q \]
    \[ \therefore p \]
  \item Conjunction
    \[ p \]
    \[ q \]
    \[ \therefore p \wedge q \]
  \item Resolution
    \[ p \vee q \]
    \[ \neg{p} \vee r \]
    \[ q \vee r \]
\end{enumerate}

\subsection*{Example}
Determine whether the argument is valid and whether its conclusion must be true.
If \( \sqrt{2} > \frac{3}{2} \) then \( (\sqrt{2})^{2} > (\frac{3}{2})^{2} \).
Let:
\[ p: \sqrt{2} > \frac{3}{2} \]
\[ q: (\sqrt{2})^{2} > (\frac{3}{2})^{2} \]
The conclusion is not true since one premise, namely
\( \sqrt{2} > \frac{3}{2} \) is false.

\subsection*{Example}
What rules are used?
\begin{enumerate}
  \item It is sunny now. Therefore, it is either sunny or \( 32^{\circ} \)F.
    \textbf{Addition Rule}
  \item It is sunny and above \( 32^{\circ} \)F now. Therefore, it is sunny
    now. \textbf{Simplification Rule}
  \item If it rains today, we will not have a bbq. If we do not have a bbq,
    then we will tomorrow. \textbf{Hypothetial Syllogism}
\end{enumerate}

\begin{center}
  If you have any questions, comments, or concerns, please contact me at
  alvin@omgimanerd.tech
\end{center}

\end{document}
