\documentclass[letterpaper, 12pt]{math}

\usepackage{amsmath}
\usepackage{amssymb}

\title{Propositional Equivalence}
\author{Alvin Lin}
\date{Discrete Math for Computing: January 2017 - May 2017}

\begin{document}

\maketitle

\section*{Propositional Equivalence}
\( p \to q \)  and \( \neg{q} \to \neg{p} \) have the same truth table. An
implication and its contrapositive are always logically equivalent.
A compound proposition is a:
\begin{enumerate}
  \item \textbf{tautology} if it is always true.
  \item \textbf{contradiction} if it is always false.
  \item \textbf{contingency} if for some values it is true and some it is false.
\end{enumerate}
\( p \vee \neg{p} \) is a tautology. \( p \wedge \neg{p} \) is a contradiction.

\subsection*{Logical Equivalence}
The compound propositions \( p \) and \( q \) are logically equivalent is
\( p \leftrightarrow q \) is a tautology. \\
\underline{Notation:} We write \( p \equiv q \) if \( p \) and \( q \) are
logically equivalent.
\[ p \to q \equiv \neg{q} \vee q \]
\[ p \to q \equiv \neg{q} \to \neg{p} \]
\[ p \vee q \equiv \neg{p} \to q \]
\[ p \wedge q \equiv \neg{(p \to \neg{q})} \]

\subsection*{De Morgan's Laws}
\begin{enumerate}
  \item \( \neg{(p \wedge q)} \equiv \neg{p} \vee \neg{q} \)
  \item \( \neg{(p \vee q)} \equiv \neg{p} \wedge \neg{q} \)
\end{enumerate}
\begin{center}
  \begin{tabular}{|c|c|c|c|c|c|}
    \hline
    \( p \) & \( q \) & \( \neg{(p \wedge q)} \) & \( \neg{p} \vee \neg{q} \) &
    \( \neg{(p \vee q)} \) & \( \neg{p} \wedge \neg{q} \) \\ \hline
    T & T & F & F & F & F \\ \hline
    T & F & T & T & F & F \\ \hline
    F & T & T & T & F & F \\ \hline
    F & F & T & T & T & T \\ \hline
  \end{tabular}
\end{center}

\subsection*{Example}
Show that \( p \vee (q \wedge r) \equiv (p \vee q) \wedge (p \vee r) \):
\begin{center}
  \begin{tabular}{|c|c|c|c|c|c|c|c|c|}
    \hline
    \( p \) & \( q \) & \( r \) & \( q \wedge r \) & \( p \vee q \) &
    \( p \vee r \) & \( p \vee (q \wedge r) \) &
    \( (p \vee q) \wedge (p \vee r) \) \\ \hline
    T & T & T & T & T & T & T & T \\ \hline
    T & T & F & F & T & T & T & T \\ \hline
    T & F & T & F & T & T & T & T \\ \hline
    T & F & F & F & T & T & T & T \\ \hline
    F & T & T & T & T & T & T & T \\ \hline
    F & T & F & F & T & F & F & F \\ \hline
    F & F & T & F & F & T & F & F \\ \hline
    F & F & F & F & F & F & F & F \\ \hline
  \end{tabular}
\end{center}

\subsection*{Laws of Logical Equivalence}
\begin{enumerate}
  \item Identity Law:
  \begin{itemize}
    \item \( p \wedge T \equiv p \)
    \item \( p \vee F \equiv p \)
  \end{itemize}
  \item Domination Law:
  \begin{itemize}
    \item \( p \vee T \equiv T \)
    \item \( p \wedge F \equiv F \)
  \end{itemize}
  \item Idempotent Law:
  \begin{itemize}
    \item \( p \vee p \equiv p \)
    \item \( p \wedge p \equiv p \)
  \end{itemize}
  \item Double Negation:
  \begin{itemize}
    \item \( \neg{(\neg{p})} \equiv p \)
  \end{itemize}
  \item Commutative Laws:
  \begin{itemize}
    \item \( p \vee q \equiv q \vee p \)
    \item \( p \wedge q \equiv q \wedge p \)
  \end{itemize}
  \item Associative Laws:
  \begin{itemize}
    \item \( (p \vee q) \vee r \equiv p \vee (q \vee r) \)
    \item \( (p \wedge q) \wedge r \equiv p \wedge (q \wedge r) \)
  \end{itemize}
  \item Distributive Laws:
  \begin{itemize}
    \item \( p \vee (q \wedge r) \equiv (p \vee q) \wedge (p \vee r) \)
    \item \( p \wedge (q \vee r) \equiv (p \wedge q) \vee (p \wedge r) \)
  \end{itemize}
  \item De Morgan's Laws:
  \begin{itemize}
    \item \( \neg{(p \wedge q)} \equiv \neg{p} \vee \neg{q} \)
    \item \( \neg{(p \vee q)} \equiv \neg{p} \wedge \neg{q} \)
  \end{itemize}
  \item Absorption Laws:
  \begin{itemize}
    \item \( p \vee (p \wedge q) \equiv p \)
    \item \( p \wedge (p \vee q) \equiv p \)
  \end{itemize}
  \item Negation:
  \begin{itemize}
    \item \( p \vee \neg{p} \equiv T \)
    \item \( p \wedge \neg{p} \equiv F \)
  \end{itemize}
\end{enumerate}

\subsection*{Example}
Show \( \neg{(p \to q)} \) and \( p \wedge \neg{q} \) are logically equivalent
using the laws. \\
\textbf{Hint:} \( p \to q \equiv \neg{p} \vee q \)
\begin{align*}
  p \to q &\equiv \neg{p} \vee q \\
  &\equiv \neg{(\neg{p} \vee q)} \\
  &\equiv \neg{(\neg{p})} \vee \neg{q} \\
  &\equiv p \vee \neg{q}
\end{align*}

\begin{center}
  If you have any questions, comments, or concerns, please contact me at
  alvin@omgimanerd.tech
\end{center}

\end{document}
